% !TeX encoding = UTF-8 Unicode
% !TEX root = MaksimenkoThesis.tex

%%%%%%%%%%%%%%%%%%%%%%%%%%%%%%%%%%%%%%%%%%%%%%%%%%%%%%%%%%
%
%     Заключение
%
%%%%%%%%%%%%%%%%%%%%%%%%%%%%%%%%%%%%%%%%%%%%%%%%%%%%%%%%%%

%\addtocounter{secnumdepth}{-5}
\nchapter{Заключение}
%\addtocounter{secnumdepth}{5}

%\hfill
%\begin{minipage}{0.55\textwidth}
%Трудно заслужить благодарность за открытие того, как наилучшим способом обойтись без некоторых признанных, но чрезмерно сложных концепций.
%\begin{flushright}
%Э.\,В.~Дейкстра
%\end{flushright}
%\end{minipage}

В диссертационной работе рассматривается проблема оценок сложности линейных задач комбинаторной оптимизации в различных вычислительных моделях с помощью комбинаторно"=геометрических характеристик соответствующих геометрических объектов (многогранников и полиэдров задач, конусных разбиений пространства исходных данных). В качестве таких характеристик рассматриваются: размерность многогранника, число его вершин, число гиперграней, диаметр графа многогранника, кликовое число графа, сложность задачи идентификации грани, сложность расширения многогранника и число прямоугольного покрытия матрицы инциденций вершин"=гиперграней.

Б\'{о}льшая часть результатов связана с новым понятием аффинной сводимости. 
Этот тип сводимости позволяет сравнивать указанные выше характеристики многогранников (за исключением диаметра графа), ассоциированных с линейными задачами комбинаторной оптимизации. В работе сравниваются семейства многогранников следующих NP-трудных задач: коммивояжер, длиннейший путь, 3-выполнимость, рюкзак, разрез в графе, независимое множество вершин графа, покрытия и упаковки множества, булево квадратичное программирование, раскраска графа, кубический подграф, линейное упорядочивание, дерево Штейнера в графе, различные варианты задачи о назначениях (квадратичные назначения, назначения с ограничением, 3-сочетания) и некоторые другие.
Во многом этот список совпадает со списком задач из фундаментальной работы Карпа~\cite{Karp:1972}. Из серии доказательств, представленных в главе~\ref{chap:AffExamples}, следует, что семейство булевых квадратичных многогранников $\BQP$ аффинно сводится к семействам многогранников перечисленных выше задач.
Таким образом, все эти семейства многогранников наследуют от $\BQP$ высокие (сверхполиномиальные) значения следующих комбинаторно"=геометрических харарактеристик: сложность расширения, число прямоугольного покрытия матрицы инциденций вершин"=гиперграней, кликовое число графа. Дополнительно, в разделе~\ref{sec:BQP-power} доказано, что для любого $k \in \N$ и $n \ge 2^{2\cdot \lceil k/3\rceil}$ булев квадратичный многогранник $\BQP(n)$ имеет $k$-смежностную грань со сверхполиномиальным числом $2^{{\Theta}\left( n^{1 / {\left\lceil k/3\right\rceil}}\right)}$ вершин. Благодаря доказанной аффинной сводимости, каждое из перечисленных семейств многогранников тоже содержит многогранники, имеющие $k$"~смежностные грани со сверхполиномиальным (относительно размерности многогранника) числом вершин.

Отдельно рассматриваются семейства многогранников, для которых задача распознавания несмежности вершин является NP-полной. Речь идет о многогранниках следующих задач: покрытие множества, 3-выполнимость, коммивояжер, длиннейший путь, кубический подграф, рюкзак, разбиение чисел на две равные суммы, назначения с ограничением, частичное упорядочивание. Показано, что NP-полнота проверки несмежности вершин наследуется этими многогранниками от специального семейства многогранников $\NPadj$, впервые описанного Т.~Мацуи~\cite{Matsui:1995} в 1995~г. С~другой стороны, известно, что смежность вершин проверяется за полиномиальное время для многогранников следующих NP-трудных задач: разрез в графе, квадратичные задачи о назначениях и о линейном упорядочивании, упаковки и разбиения множества, 3-сочетания, раскраска графа, независимое множество вершин графа. В~разделе~\ref{sec:DoubleCovering} показано, что ни один из многогранников семейства $\NPadj$ (за исключением отрезков) не может быть гранью ни для одного многогранника из перечисленных семейств с простым критерием смежности.

Метод аффинной сводимости может быть применим и для сравнения комбинаторно"=геометрических характеристик линейных задач комбинаторной оптимизации, имеющих дополнительные ограничения на множество исходных данных (целевых векторов).
Наиболее ярким примером здесь служит задача о кратчайшем орпути, полиномиально разрешимая при отсутствии в орграфе контуров отрицательной длины. В разделе~\ref{sec:ShortPath2Assignment} показано, что конусное разбиение множества исходных данных этой задачи аффинно сводится к конусному разбиению множества исходных данных задачи о назначениях.

В главе~\ref{chap:ExtAff} вводится понятие расширенной аффинной сводимости, отличающееся от аффинной сводимости отсутствием требования биективности соответствующего аффинного отображения. Ослабление условий существенно упрощает доказательство фактов расширенной аффинной сводимости (в разделе~\ref{sec:ExtAffExamples} представлены соответствующие примеры). Вместе с тем, такой тип сводимости, вообще говоря, не позволяет сравнивать некоторые характеристики многогранников (например, число гиперграней, кликовые числа графов многогранников, NP-полноту проверки несмежности вершин). В~разделе~\ref{sec:Cook4Polytopes} доказано, что любое семейство многогранников, задача тестирования вершины которых принадлежит классу NP, расширенно аффинно сводится к семейству булевых квадратичных многогранников. Таким образом, все перечисленные в главах~\ref{chap:AffTheory}--\ref{chap:ExtAff} семейства многогранников оказываются эквивалентны друг другу относительно расширенной аффинной сводимости.

В главе~\ref{chap:Cyclic} обсуждаются свойства циклических многогранников, играющих важную роль в комбинаторной теории выпуклых многогранников. В разделе~\ref{sec:EF4Cyclic} для $d$"~мерного циклического многогранника $\CP_{d,n}(t)$ на $n$ вершинах описана расширенная формулировка размера $O(\log n)^{\lfloor d/2 \rfloor}$. Таким образом, показано, что даже многогранники с максимальным числом граней (при фиксированных размерности и числе вершин) могут иметь компактную расширенную формулировку. В разделе~\ref{sec:RidgeGraph} решена задача, сформулированная Кли в 1964 году~\cite{Klee:1964}, "--- найдено точное значение диаметра ридж"~графа циклического многогранника $\CP(d,n)$ при $n>2d$.

В главе~\ref{chap:Direct} рассматривается теория алгоритмов прямого типа (разработанная Бондаренко~\cite{BondBook:1995}), в рамках которой утверждается, что кликовое число графа (многогранника, конусного разбиения множества исходных данных) задачи является нижней оценкой сложности соответствующей оптимизационной задачи в некотором широком классе алгоритмов. %В~разделе~\ref{sec:ShortPathClique} показано, что кликовое число графа задачи о кратчайшем орпути в орграфе на $n$ вершинах с ограничением неотрицательности длин контуров равно~$\lfloor n^2/4\rfloor$. 
В~разделе~\ref{sec:ShortPathClique} приводится доказательство критерия смежности для графов решений трех полиномиально разрешимых вариантов задачи о кратчайшем пути.
На основе этого критерия и доказательства теоремы 2 из~\cite{Bondarenko:1993SW3A} делается вывод о том, что кликовое число для всех трех вариантов равно~$\lfloor n^2 / 4\rfloor$, где $n$ "--- число вершин графа, в котором ищется кратчайший путь.
В разделе~\ref{sec:NondirectAlg} приводится доказательство того, что алгоритм Куна--Манкреса для задачи о назначениях (а также алгоритм Эдмондса для задачи о паросочетаниях) не является алгоритмом прямого типа.
Кроме того, описывается достаточно универсальный и, в то же время, естественный способ модификации алгоритмов, существенно не меняющий их трудоемкости, но гарантированно выводящий их из класса алгоритмов прямого типа.
Это говорит о том, что ограничения, накладываемые в определении алгоритма прямого типа, являются довольно жесткими и не выполняются, в том числе, для некоторых классических алгоритмов комбинаторной оптимизации.
%Кроме того, проверка этих ограничений (в том случае, если они действительно выполняются), как правило, сложнее непосредственной оценки трудоемкости алгоритма.


В главе~\ref{chap:Counterexamples} представлена коллекция примеров семейств многогранников, для которых упомянутые выше комбинаторно"=геометрические характеристики значительно отличаются от современных эмпирических представлений о вычислительной сложности соответствующих оптимизационных задач.
В~разделе~\ref{subsec:CliqueCounterex} показано, что у любого выпуклого многогранника существует расширение с кликовым числом графа, равным двум.
В~разделе~\ref{sec:ExtensionCounterex} описан пример семейства многогранников, сложность расширений которых сверхполиномиальна относительно размерности, число прямоугольного покрытия матрицы инциденций вершин"=гиперграней полиномиально, а задача линейной оптимизации NP-трудна. Таким образом, ни одно из упомянутых выше семейств многогранников NP-трудных задач не может быть расширенно аффинно сведено к данному семейству.
В~разделе~\ref{sec:CounterexamplesOther} приводятся примеры двух линейных задач комбинаторной оптимизации, многогранники которых комбинаторно эквивалентны друг другу, но одна из этих задач полиномиально разрешима, а другая NP-трудна.
%Такого эффекта удается достичь за счет экспоненциальной сложности распознавания вершины многогранника последней задачи.
Этот результат говорит о том, что ни одна чисто комбинаторная характеристика многогранника не дает возможности отделить полиномиально разрешимые задачи от NP-трудных. %задач с экспоненциальной сложностью.
С другой стороны, все рассмотренные в диссертации примеры позволяют сделать предположение о том, что реальная сложность задачи оценивается сверху сложностью расширения многогранника, а снизу "--- числом прямоугольного покрытия матрицы инциденций вершин"=гиперграней.

%Открытые вопросы

%1. До сих пор остается открытым следующий вопрос. Существуют ли примеры задач, сложность которых была бы существенно меньше числа прямоугольного покрытия матрицы инциденций вершин"=гиперграней соответствующего многогранника? Во всех известных примерах сложность задачи оказывается больше этой величины. В частности, можно предположить, что сложность задачи всегда ограничена снизу числом прямоугольного покрытия, а сверху "--- сложностью расширенной формулировки.

%2. В последнем разделе последней главы были построены два примера задач, многогранники которых комбинаторно эквивалентны, но одна из задач полиномиально разрешима, а другая имеет экспоненциальную сложность. Этот результат достигнут во многом за счет того, что распознавание вершины многогранника второй задачи имеет экспоненциальную сложность, что не соответствует определению~\ref{def:family} семейства \emph{комбинаторных} многогранников и замечанию~\ref{rem:PolyPred} к определению задачи комбинаторной оптимизации. Таким образом, остается открытым вопрос о существовании пары примеров семейств комбинаторных многогранников, задача линейной оптимизации на одном из которых была бы полиномиально разрешима, а на другом "--- NP-трудна (или имела экспоненциальную сложность).

%3. В настоящее время неизвестны нижние оценки сложности расширения $\CP_{d,n}$, которые соответствовали бы верхней оценке $O(\log n)^{\lfloor d/2 \rfloor}$ из теоремы~\ref{thm:main}. Например, из~\cite{FioriniKPT:13} следует, что $\xc(\CP_{d,n}) \ge \rc(\CP_{d,n}) = O(d^2 \log n)$.

%4. Не известен статус задачи об идентификации смежности вершин многогранника деревьев Штейнера. 

%5. Является ли многогранник гамильтоновых циклов многогранником задачи выполнимость? Или, иными словами, можно ли утверждение <<подграф является гамильтоновым циклом>> (или даже <<подграф является св\'язным>>) выразить булевой формулой полиномиальной длины в КНФ с использованием только переменных, идентифицирующих наличие ребер.

%Комментарии

%1. Пример задачи с ограничением, сложность расширенной формулировки полиэдра которой как угодно больше сложности расширенной формулировки многогранника (общей) задачи. Рассмотрим в $\R^3$ правильный $n$-угольник с вершинами на окружности $x^2+y^2=1$, $z=1$. Множество его вершин обозначим $Y_n$. Сложность расширенной формулировки конуса $\cone(Y_n)$ равна $\Theta(\log n)$. Таким образом, за счет выбора $n$ мы можем сделать эту сложность как угодно большой. Рассмотрим тривиальную задачу оптимизации на отрезке $z \in [0,1]$, $x=y=0$. Сумма этого отрезка и конуса $\cone(Y_n)$ совпадает с конусом.
