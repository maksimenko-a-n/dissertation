% !TeX encoding = UTF-8 Unicode
% !TEX root = MaksimenkoThesis.tex
% Общие разделы автореферата и~диссертации

\textbf{Объект исследования.}
Многие прикладные задачи комбинаторной оптимизации допускают следующую формулировку.
Дано конечное множество $E$, каждому элементу $e$ которого приписан некоторый вес $c_e \in \R$, и~полиномиально вычислимый \emph{предикат допустимости} $f \colon 2^E \to \{\text{ложь}, \text{истина}\}$.
Подмножество $s \subseteq E$ называется \emph{допустимым решением} этой задачи, если $f(s)$ истинно.
Множество всех допустимых решений обозначим $S$, $S = \{s \subseteq E \mid f(s)\}$.
%, определяющая множество $S = \{s \subseteq E \mid f(s) = 1\}$ всех допустимых решений задачи.
Цель задачи состоит в~отыскании \emph{оптимального} решения $s \in S$ с~максимальным (минимальным) весом $c(s) = \sum_{e \in s} c_e$.

С теоретической точки зрения, большой интерес представляет \emph{массовая задача}, когда множество $E$, предикат $f$ и веса $c_e$, $e\in E$, не зафиксированы, но $E$ и $f$ однозначно определяются набором входных параметров $I$ по некоторому правилу, характеризующему данную массовую задачу. \emph{Индивидуальная задача} представляет собой частный случай массовой задачи, когда входные данные $I$ и $c_e$, $e \in E$, зафиксированы.
Наряду с массовой и индивидуальной задачами существует промежуточный вариант, когда параметры $I$ зафиксированы, а веса $c_e$ "--- нет.
Так как множество допустимых решений $S$ в таком случае определяется однозначно, для такой задачи будем использовать обозначение~$S$.

Например, в~задаче о~кратчайшем пути входные данные $I$ однозначно определяют множество городов, среди которых выделены города $A$ и $B$, и множество (участков) дорог $E$, соединяющих пары городов, веса $c_e$ являются длинами дорог, а~предикат допустимости $f$ принимает значение <<истина>> для каждого подмножества дорог, представляющего собой маршрут из~$A$ в~$B$.
Другими классическими примерами являются задача поиска минимального остовного дерева, задача о~назначениях, задача коммивояжера, задача о~рюкзаке и~многие другие.

%Область приложений, где возникают подобного рода задачи, очень широка.
Чаще всего такие задачи встречаются в~экономике при оптимизации использования ресурсов,
планировании транспортной инфраструктуры и~производства, оптимизации доставки грузов, составлении расписаний и~т.\,п.~\cite{Paschos:2014}. 
Более того, в~настоящее время задачи комбинаторной оптимизации встречаются практически везде: 
секвенирование генома, классификация биологических видов, моделирование молекул, 
планирование коммуникационных и~электрических сетей, позиционирование спутников,
производство сверхбольших интегральных схем (VLSI), %и~печатных плат, 
криптография, машинное обучение и~т.\,д.~\cite{GrotschelCO:1995}.

Как известно~\cite{SchrijverCO:2003}, во многих случаях такие задачи полезно переводить на язык геометрии.
А именно, при фиксированном $I$, для каждого допустимого решения $s \in S$ рассматривается его \emph{характеристический вектор}
$\bm{x} \in \{0,1\}^E$, координаты $x_e$, $e \in E$, которого полагаются равными единице для $e \in s$
и равны нулю для $e \not\in s$. 
Далее множество всех характеристических векторов допустимых решений обозначаем $X$, $X = X(S) \subseteq \{0,1\}^E$.
Набор весов представляется в~виде вектора $\bm{c} = (c_e) \in \R^E$.
%, называемого \emph{целевым}.
Цель задачи при такой интерпретации заключается в~поиске вектора $\bm{x} \in X$, на котором \emph{целевая функция} $\langle\bm{c}, \bm{x}\rangle$ принимает максимальное (минимальное) значение.
Так как целевая функция линейна, 
то соответствующая задача называется \emph{линейной задачей комбинаторной оптимизации}.

Заметим, что экстремальные значения линейной функции не~меняются при замене области определения $X$ выпуклой оболочкой $\conv(X)$.
Таким образом, с~каждой задачей $S$ ассоциируется некоторый выпуклый многогранник $\conv(X)$, вершинами которого служит множество~$X = X(S)$.
%Многогранники, определяемые таким способом, принято называть комбинаторными.

Во-первых, такая геометрическая интерпретация дает возможность при решении задачи пользоваться различными геометрическими инструментами, в~частности, методами линейного программирования (см., например, \cite{Shevchenko:2004}).
Как показывает опыт многочисленных исследований в~этой области, во многих случаях это позволяет на порядки увеличивать скорость и~качество решений~\cite{SchrijverCO:2003,Applegate:2003}.

Во-вторых, комбинаторная структура многогранника отражает 
структуру множества допустимых решений соответствующей задачи.
%что открывает возможности для структурного анализа сложности задачи.
Проиллюстрируем эту мысль с~помощью понятия смежных решений.
Допустимые решения $s_1$ и~$s_2$ задачи $S$ называются \emph{смежными},
если для некоторого набора весов $\bm{c}$ оба эти решения являются оптимальными
для соответствующей индивидуальной задачи и~других оптимальных решений у неё нет.
Смежность решений $s_1$ и~$s_2$ означает, что при незначительном изменении набора весов $\bm{c}$ оптимальное решение индивидуальной задачи может меняться с~$s_1$ на $s_2$ и~обратно.
То есть в~алгоритме, решающем задачу $S$, должна быть предусмотрена проверка,
чувствительная к~таким изменениям.
Это, в~свою очередь, накладывает ряд ограничений на структуру алгоритма %, решающего задачу оптимизации на множестве $S$,
и, в~итоге, может использоваться для теоретических оценок сложности соответствующей задачи.
При геометрическом подходе смежность решений интерпретируется как наличие ребра
между соответствующими вершинами многогранника~$\conv(X)$.
Таким образом, граф многогранника задачи содержит в~себе некоторую информацию о~её структурной сложности. 
Продолжая эту мысль, вполне естественно обратиться к~изучению свойств всей решетки граней (множества всех граней, упорядоченных по включению) многогранника задачи.
%Это и~является основным объектом исследований настоящей работы.


\textbf{Актуальность темы исследований.}
Широкий интерес к~задачам комбинаторной оптимизации появился в~1950-х гг.
Его возникновение было обусловлено тремя факторами:
создание первых программируемых ЭВМ, создание концепции линейного программирования Канторовичем и~Купмансом и~разработка симплекс-метода Данцигом,
а также накопленный опыт решения алгоритмических задач на графах.
Именно в~1950-х гг. был разработан венгерский метод решения задачи о~назначениях,
алгоритм Форда-Фалкерсона для задачи о~максимальном потоке, 
несколько алгоритмов для нахождения кратчайших путей,
заново открыты старые и~предложены новые алгоритмы для нахождения минимального остовного дерева,
а также их обобщение для нахождения в~матроиде базы минимального веса~\cite{SchrijverHistory:2005}.
В~это же время, после разработки Данцигом симплекс-метода, были реализованы первые успешные попытки
его применения к~различным задачам комбинаторной оптимизации.
В~частности, с~помощью техники линейного программирования был достигнут впечатляющий по тем временам прогресс в~решении задачи коммивояжера~\cite{DantzigFJ:1954}.
Развитие этой методики в настоящее время позволяет решать задачу коммивояжера для миллиона городов~\cite{Applegate:2003}.

Успешное применение симплекс"=метода стало источником
многочисленных размышлениий о~его теоретической эффективности.
Так, например, было замечено, что нижней оценкой числа шагов симплекс-метода
может служить диаметр графа многогранника.
В~связи с~этим Хирш в~1957 году сформулировал гипотезу о~том,
что диаметр графа многогранника не может быть больше разности между числом его гиперграней и~размерностью.
С~тех пор этой гипотезе уделялось значительное внимание, но лишь в~2010 году Сантосу
удалось построить пример 43-мерного многогранника с~86 гипергранями, диаметр которого больше, чем 43~\cite{Santos:2012}.
Тем не менее, в~общем виде\footnote{Верно ли, что диаметр графа ограничен сверху полиномом от числа гиперграней и~размерности многогранника?} эта гипотеза до сих пор остается открытой и~привлекает внимание видных ученых~\cite{ZieglerHirsch:2012}.

Одним из~основных результатов настоящей работы является точное значение диаметра графа
многогранника, двойственного к~циклическому. 
В~1964~г. Кли~\cite{Klee:1964} предположил, что этот диаметр равен $\lfloor n/2\rfloor$,
где $n$~--- число гиперграней. Но тремя годами позднее им же был найден контрпример~\cite{Klee:1967}.
С~тех пор задача оставалась открытой.

Вместе с~успехами и~неудачами в~решении отдельных задач в~1950--60-х~гг.
формировалось понятие эффективного алгоритма, окончательно сформулированное в~работах 
Эдмондса~\cite{Edmonds:1965} и~Кобхэма~\cite{Cobham:1964}.
В~это же время Эдмондс~\cite{Edmonds:1965b} ввел понятие задачи, имеющей <<хорошую характеризацию>>,
что, по-сути, является определением того, что позднее было названо классом NP.
Всё это послужило предпосылками к~открытию в~начале 1970-х гг. Куком~\cite{Cook:1971}, Карпом~\cite{Karp:1972} и~Левиным~\cite{Levin:1973} NP"~полных задач. %~\cite{Garey:1982}.
Примечательно то, что каждый из~них предложил свой способ свед\'{е}ния задач.
В~частности, метод аффинной сводимости, развиваемый в настоящей диссертации, по своей сути ближе всего к~сводимости Левина.

Открытие NP"~полных задач послужило мощным толчком для дальнейших исследований,
в том числе свойств многогранников, ассоциированных с~NP"~трудными задачами.
В~1978 году Пападимитриу~\cite{Papadimitriou:1978} показал, 
что задача проверки несмежности двух произвольно взятых вершин многогранника задачи коммивояжера NP"~полна,
то есть она также сложна, как и~сама задача коммивояжера.
Позднее аналогичные результаты для многогранников некоторых других NP"~трудных задач появились в работах следующих авторов: Чунг; Гейст и~Родин; Мацуи; Бондаренко и~Юров; Альфаки и~Мурти; Фиорини (см. ссылки в~\citemy{Maksimenko:2013NP}).
С~другой стороны, в~1975~г. Хватал~\cite{Chvatal:1975} нашел полиномиальный критерий смежности вершин многогранника независимых множеств.
По-сути, этот же критерий смежности верен и для многогранников упаковок множеств, многогранников разбиений множеств и многогранников трехиндексной задачи о назначениях~\cite{Ikura:1985}.
В~1984~г. Грешнев обнаружил~\cite{Greshnev:1984}, что граф многогранника задачи об $m$"~вершинной клике полон, т.\,е. задача проверки смежности вершин для него тривиальна.
Чуть позднее аналогичные результаты были получены для многогранника задачи о~максимальном разрезе~\cite{Beloshevskii:1986,Barahona:1986} и для 
булева квадратичного многогранника~\cite{Bondarenko:1987, Padberg:1989}.
%многогранника задачи о~максимальной клике в реберно"=взвешенном полном графе~\cite{Bondarenko:1985}. Последний многогранник по своему описанию очень близок к булевому квадратичному многограннику $\BQP(n)$, тоже обладающему полным графом~\cite{Bondarenko:1987, Padberg:1989}.

В диссертации показано, что семейство многогранников, описанное Мацуи~\cite{Matsui:1995}, аффинно сводится к~семействам многогранников, описанных в работах Чунг; Гейст и~Родин; Бондаренко и~Юров; Альфаки и~Мурти; Фиорини. Одним из~следствий этого является то, что все эти семейства наследуют от многогранников Мацуи NP"~полноту проверки несмежности вершин.
С другой стороны, показано, что перечисленные выше многогранники с полиномиальным критерием смежности аффинно сводятся к многогранникам Мацуи, причем аффинная сводимость в обратную сторону невозможна. Таким образом, оказалось, что многогранники Мацуи образуют своего рода водораздел между семействами многогранников с NP"~полным критерием несмежности вершин и семействами с полиномиальным критерием.


%Как оказалось впоследствии, все эти факты имеют одну и~ту же геометрическую причину,
%обнаруживаемую с~помощью аффинной сводимости~\citemy{Maksimenko:2013NP}.
%Оказывается, все эти факты имеют одну и~ту же геометрическую причину,
%обнаруживаемую с~помощью аффинной сводимости~\citemy{Maksimenko:2013NP}.
%Автор настоящей работы показал, что все упоминаемые в~этих работах многогранники
%содержат в~качестве грани многогранник двойных покрытий~\citemy{Maksimenko:2013NP},
%задача проверки смежности вершин которого NP"~полна~\cite{Matsui:1995}.

В~1979 году Хачиян~\cite{Khachiyan:1979} описал полиномиальный алгоритм для решения задачи линейного программирования.
Этот факт стал теоретическим подтверждением эффективности геометрического подхода к~решению задач комбинаторной оптимизации, что увеличило популярность исследований свойств соответствующих многогранников.
В~частности, большое внимание исследованию свойств графов многогранников уделено в~монографии Емеличева, Ковалева и~Кравцова~\cite{Emelichev:1981}.

В~1983 г. Бондаренко~\cite{BondBook:1995} ввел понятие алгоритма прямого типа и показал, что трудоемкость такого алгоритма оценивается снизу кликовым числом\footnote{В~оригинале эта характеристика называется плотностью графа.} графа многогранника соответствующей оптимизационной задачи.
Им же была доказана сверхполиномиальность кликовых чисел графов многогранников следующих NP"~трудных задач: коммивояжер, максимальная клика, 3-сочетание и~некоторых других.
С~другой стороны, кликовые числа графов многогранников оказались полиномиальны для следующих полиномиально разрешимых задач: сортировка, минимальное остовное дерево, задача о~кратчайшем пути.
Также было показано, что некоторые классические комбинаторные алгоритмы являются алгоритмами прямого типа~\cite{BondBook:1995}.
%На основании этих фактов была разработана теория алгоритмов прямого типа, утверждающая, что кликовое число является нижней оценкой сложности в~некотором <<широком классе алгоритмов>>~\cite{BondBook:1995}. 
Недавно список оценок кликовых чисел графов многогранников задач пополнился несколькими новыми результатами~%\cite{Shovgenov:2015, Nikolaev:2016, Nikolaev:2017, Shovgenov:2017}.
(см. \cite{Nikolaev:2017} и ссылки в ней).

Настоящая диссертационная работа во многом продолжает научные исследования, начало которым было положено в работах В.А.~Бондаренко. Во"~первых, в диссертации показано, что теория алгоритмов прямого типа применима и в тех случаях, когда на множество целевых векторов накладываются линейные ограничения.
В частности, кликовое число полиэдрального графа для задачи о кратчайшем пути из экспоненциального превращается в полиномиальное при ограничении, когда допускаются лишь неотрицательные длины ребер.
%(например, неотрицательность длин ребер в классической задаче о кратчайшем пути), при этом многогранник задачи превращается в полиэдр. 
Предложенная идея успешно используется для получения новых результатов в~\cite{Nikolaev:2016} и~\cite{Shovgenov:2017}. Во"~вторых, доказано, что алгоритм Куна--Манкреса для задачи о~назначениях не принадлежит классу алгоритмов прямого типа и, кроме того, описан достаточно универсальный способ модификации алгоритмов, существенно не меняющий их трудоемкости, но гарантированно выводящий их из~этого класса. В"~третьих, установлено, что сверхполиномиальность кликовых чисел графов многогранников NP"~трудных задач имеет простое объяснение "--- ко~всем этим семействам аффинно сводятся булевы квадратичные многогранники $\BQP(n)$, кликовые числа графов которых экспоненциальны по $n$.
%сверхполиномиальны относительно размерности. 
Более того, показано, что $\{\BQP(n)\}$, а вместе с ним и остальные рассматриваемые в настоящей работе семейства многогранников NP"~трудных задач, для любого $k \in \N$ содержат $k$"~смежностные\footnote{Многогранник называется \emph{$k$"~смежностным}, если любые $k$ его вершин образуют грань этого многогранника. В частности, многогранник, граф которого полон, называется 2-смежностным.} грани со сверхполиномиальным (относительно размерности многогранника) числом вершин.


%семейство булевых квадратичных многогранников $\{\BQP(n)\}$ аффинно сводится ко всем упомянутым выше многогранникам NP-трудных задач. Учитывая, что кликовое число графа многогранника $\BQP(n)$ экспоненциально, из. Таким образом, найдено простое геометрическое объяснение для 
%понятие аффинной сводимости, введенное автором настоящей диссертационной работы, объясняет сверхполиномиальность кликовых чисел графов тем, что все исследованные многогранники NP-трудных задач содержат в~качестве грани булев квадратичный многогранник, кликовое число графа которого экспоненциально.

Открытие Хачияном полиномиального алгоритма для задачи линейного программирования породило в 1980-х гг. волну попыток поиска компактного линейного описания для многогранника задачи коммивояжера.
Все эти попытки были направлены на использование идеи расширенной формулировки многогранника 
(сам термин появился позднее). 
\emph{Расширенной формулировкой} многогранника~$P$ называется набор линейных ограничений,
описывающих многогранник $Q$ такой, что $P$ является ортогональной проекцией $Q$.
Сам многогранник $Q$ называется \emph{расширением} многогранника~$P$.
К тому времени уже были известны примеры, когда число линейных неравенств, необходимых для описания многогранника, экспоненциально, а~для его расширения~--- полиномиально относительно длины входных данных задачи.
Ни одна из~попыток поиска компактной расширенной формулировки для многогранника задачи коммивояжера не привела к~успеху, и~в 1988 году Яннакакис~\cite{Yannakakis:1988}
показал, что такие попытки в~принципе не имеют перспективы, 
если предлагаемые расширенные формулировки удовлетворяют некоторым естественным условиям симметрии.
Там же была высказана гипотеза, что утверждение остается справедливым и~без условий симметрии.
Кроме того, Яннакакис показал, что число линейных неравенств в~расширенной формулировке многогранника не может быть меньше числа прямоугольного покрытия матрицы инциденций вершин"=гиперграней многогранника.
Впоследствии минимальное число линейных неравенств, необходимых для описания расширения многогранника
было названо \emph{сложностью расширения многогранника.}

В~конце 2000-х гг. расширенные формулировки вновь привлекли внимание исследователей~\cite{Conforti:2010,Kaibel:2011}, 
что привело к~появлению целого ряда новых интересных результатов в~данном направлении~\cite{FioriniPokutta:2012, Fiorini:2012polygons, KaibelPT:12, FioriniKPT:13, Rothvoss:2013, Rothvoss:2014, KaibelW:15}.
В~частности, в~2012~г. Фиорини, Массар, Покутта, Тивари и~де Вулф~\cite{FioriniPokutta:2012} доказали справедливость гипотезы Яннакакиса, показав,
что число прямоугольного покрытия матрицы инциденций вершин"=гиперграней
для булева квадратичного многогранника~$\BQP(n)$ экспоненциально относительно $n$ и, как следствие, сложность расширения тоже экспоненциальна.
Из этого результата, а также из установленного в настоящей диссертации факта аффинной сводимости булевых квадратичных многогранников ко многим другим семействам многогранников NP"~трудных задач следует, что все эти семейства тоже обладают сверхполиномиальным числом прямоугольного покрытия и сверхполиномиальной сложностью расширения. 
С другой стороны, в диссертации впервые описан пример NP"~трудной задачи, многогранники которой обладают полиномиальным числом прямоугольного покрытия.

В качестве тестирования потенциала расширенных формулировок в диссертации рассматривается задача оптимизации линейной функции на множестве вершин циклического многогранника \(\CP_{d,n} = \conv\Set*{(t, t^2, \dots, t^d)\in \R^d \given t = 1,2,\dots,n}\).
Эта задача тесно связана с задачей локализации корней многочлена %степени~$d$
с~некоторой, заранее заданной точностью. В работе показано, что сложность расширения этого многогранника равна $O(\log n)^{\lfloor d/2 \rfloor}$ (тогда как число его гиперграней имеет порядок $\Theta(n)^{\lfloor d/2 \rfloor}$ при фиксированном $d$~\cite{Gale:1963}).

В настоящей диссертации также введено понятие расширенной аффинной сводимости, сохраняющей такие свойства многогранников, как сверхполиномиальность сложности расширения и числа прямоугольного покрытия матрицы инциденций вершин"=гиперграней. Показано, что семейство многогранников любой линейной задачи комбинаторной оптимизации с предикатом допустимости из класса NP расширенно аффинно сводится к семейству $\{\BQP(n)\}$.
%В~\cite{Beasley:2013} замечено, что из этого факта и из результата S.~Burer~\cite{Burer:2009} следует, что любой 0/1"~многогранник, чье множество вершин может быть описано полиномиальным предикатом, имеет коположительное расширение\footnote{Коположительное расширение многогранника $P$ "--- это пересечение аффинного подпространства и коположительного конуса $C_n = \{X \mid y^T X y \ge 0, \ \forall y \in \R_{+}^n\}$ такое, что $P$ является его проекцией. Двойственным к коположительному конусу является полностью положительный конус $C^*_n = \{X \mid X = \sum_{y \in Y} y y^T \text{ для некоторого конечного } Y \subset \R_{+}^n\}$, с которым связано понятие полностью положительного расширения.} полиномиального размера. 
Фиорини, Массар, Патра и Тивари~\cite{Fiorini:2014} используют этот факт для того, чтобы дать ответ на следующий вопрос из работы Бюрер~\cite{Burer:2009}: <<Кроме нескольких перечисленных задач, какие типы задач могут быть представлены как коположительные программы или полностью положительные программы?>> (\foreignlanguage{english}{``Other than the handful of problems listed above, what types of problems can be represented as COPs or as CPPs?''}).

%\textbf{development}{Степень разработанности темы исследования.}{
%Текст о~степени разработанности темы.
%}

\textbf{Целью работы}
является анализ комбинаторно"=геометрических свойств многогранников, характеризующих сложность соответствующих задач комбинаторной оптимизации.
% в различных вычислительных моделях и классах алгоритмов.
Это подразумевает: 
1)~поиск численных оценок различных комбинаторно"=геометрических характеристик для наиболее востребованных семейств многогранников, 
2)~разработку методологии, упрощающей такого рода поиски,
3)~анализ перспективности использования тех или иных характеристик в~качестве оценок сложности соответствующих оптимизационных задач.


\textbf{Методы исследования.}
При исследовании семейств комбинаторных многогранников интенсивно используется новое в~данной области понятие аффинной сводимости.
Также используются методы теории выпуклых многогранников, линейного программирования, теории графов и комбинаторного анализа, теории сложности вычислений.


\textbf{Научная новизна.}
Основные результаты диссертации являются новыми и~могут быть кратко сформулированы следующим образом:
%Полученные в~диссертации результаты являются новыми. Следующие результаты являются основными:
\begin{enumerate}
\item Введено понятие аффинной сводимости семейств комбинаторных многогранников, сохраняющей такие свойства многогранников, как NP"~полнота критерия несмежности вершин и~сверхполиномиальность следующих числовых характеристик: число вершин, число гиперграней, кликовое число графа, сложность расширения, число прямоугольного покрытия матрицы инциденций вершин"=гиперграней. С помощью этого понятия получены следующие результаты:
\begin{itemize}
	\item Показано, что специальное семейство многогранников, рассмотренное Мацуи в~1995 г., аффинно сводится к~семействам многогранников следующих задач: рюкзак, коммивояжер, кубический подграф, 3-выполнимость, назначения с~линейным ограничением. %, покрытие множества 
	Одним из~следствий этого является то, что все эти семейства наследуют от многогранников Мацуи NP"~полноту проверки несмежности вершин.
	\item Показано, что семейства многогранников независимых множеств, многогранников упаковок и~разбиений множества, многогранников задачи об $n$"~назначениях для $n \ge 3$ и~многогранников раскрасок графа эквивалентны в~смысле аффинной сводимости и~аффинно сводятся к~семейству многогранников Мацуи. Кроме того, установлено, что ни один из~многогранников Мацуи (за исключением отрезков) не является гранью (в том числе несобственной) ни для одного из~многогранников независимых множеств.
\end{itemize}
\item Обнаружены особые свойства булевых квадратичных многогранников, указывающие на их особое место среди других известных семейств многогранников NP-трудных задач:
\begin{itemize}
	\item Показано, что булевы квадратичные многогранники аффинно сводятся к~семействам многогранников, перечисленным в~предыдущем пункте (рюкзак, коммивояжер и~т.\,д.), а~также к~многогранникам линейных упорядочиваний, многогранникам квадратичных назначений и~квадратичных линейных упорядочиваний. 
	Из этого следует, в~частности, что такие свойства булевых квадратичных многогранников, как сверхполиномиальность кликового числа графа и~числа прямоугольного покрытия матрицы инциденций вершин"=гиперграней автоматически наследуются всеми этими семействами.
	\item Введены в~рассмотрение семейства булевых многогранников степени~$p$.
	Показано, что эти многогранники $\lfloor 3p/2\rfloor$"~смежностны и~аффинно сводятся к~булевым квадратичным. Из этого следует, что для любого натурального $k$ булевы квадратичные многогранники (а вместе с ними и все остальные перечисленные выше семейства многогранников NP"~трудных задач) содержат $k$"~смежностные грани со сверхполиномиальным (относительно размерности многогранника) числом вершин.
\end{itemize}
\item Введено понятие расширенной аффинной сводимости, отличающееся от аффинной сводимости отсутствием требования биективности соответствующего аффинного отображения. Показано, что семейство многогранников любой линейной задачи комбинаторной оптимизации с~предикатом допустимости из класса NP расширенно аффинно сводится к~семейству булевых квадратичных многогранников. 
Таким образом, относительно расширенной аффинной сводимости все перечисленные выше семейства многогранников образуют один класс эквивалентности.
Найден пример семейства многогранников NP-трудной задачи, к которому не может быть расширенно аффинно сведено ни одно из~упомянутых выше семейств многогранников.
\item Сделаны оценки двух характеристик %сложности 
для циклических многогранников\footnote{Циклические многогранники обладают максимальным числом граней среди всех выпуклых многогранников, имеющих ту же размерность и~такое~же число вершин.}:
\begin{itemize}
	\item Описана компактная расширенная формулировка размера $O(\log n)^{\lfloor d/2 \rfloor}$ для $d$"~мерных циклических многогранников на $n$ вершинах.
	\item Найдено точное значение диаметра графа многогранника, двойственного к~циклическому.
\end{itemize}
%\item Найдено кликовое число полиэдра задачи о~кратчайшем орпути с~ограничением неотрицательности длин контуров.
\item Выполнен критический анализ перспективности использования различных известных характеристик многогранников в качестве оценок сложности соответствующих оптимизационных задач:
\begin{itemize}
	\item На примере задачи о кратчайшем пути показано, что теория алгоритмов прямого типа (разработанная В.\,А.~Бондаренко) применима и~в~тех случаях, когда на множество целевых векторов накладываются линейные ограничения (при этом многогранник задачи превращается в полиэдр).
	\item Доказано, что алгоритм Куна--Манкреса (венгерский метод) для задачи о~назначениях не принадлежит классу алгоритмов прямого типа.
	Кроме того, описывается достаточно универсальный способ модификации алгоритмов,
	существенно не меняющий их трудоемкости, но гарантированно выводящий их из~этого класса.
	\item Показано, что у любого выпуклого многогранника есть расширение, 
	%(многогранник, линейной проекцией которого является данный)
	граф которого не содержит треугольников. 
	\item Приводится пример семейства многогранников NP"~трудной задачи, число прямоугольного покрытия матрицы инциденций вершин"=гиперграней которых полиномиально. 
	\item Приводятся примеры двух линейных задач комбинаторной оптимизации, многогранники которых комбинаторно эквивалентны друг другу, но одна из~этих задач полиномиально разрешима, а~другая NP"~трудна.
\end{itemize}
\end{enumerate}


\textbf{Теоретическая и~практическая значимость.}
Работа имеет теоретический характер.
Полученные результаты могут быть использованы для исследований
сложности задач комбинаторной оптимизации и~поиска новых эффективных алгоритмов их решения. Значительная часть результатов может быть также использована в~исследованиях комбинаторно"=геометрических свойств выпуклых многогранников.

Значение полученных результатов подтверждается их цитированием как отечественными,
так и~зарубежными специалистами (список не включает соавторов соискателя):
А.В.~Николаев, А.В.~Селиверстов, Р.Ю.~Симанчев, 
%V.~Pilaud, 
%H.~Fawzi, J.~Saunderson, P.A.~Parrilo,
%S.~Massar, M.K.~Patra, H.R.~Tiwary,
%L.B.~Beasley, H.~Klauck, T.~Lee, D.O.~Theis,
%K.~Qi, Q.~Feng, K.~Zhao,
%A.~Huq.
%A.~Makkeh, M.~Pourmoradnasseri, D.O.~Theis. The Graph of the Pedigree Polytope is Asymptotically Almost Complete (Extended Abstract)
%Huchette, J., Vielma, J. P. (2017) \url{https://arxiv.org/abs/1709.10132}
%Davis-Stober, C. P., Doignon, J. P., Fiorini, S., Glineur, F., Regenwetter, M. (2017) \url{https://arxiv.org/abs/1710.02679}
\foreignlanguage{english}{
L.B.~Beas\-ley, C.P.~Davis-Stober, J.P.~Doignon, H.~Fawzi, Q.~Feng, F.~Glineur, J.~Huchette, A.~Huq, H.~Klauck, T.~Lee, A.~Makkeh, S.~Massar, P.A.~Parrilo, M.K.~Patra, V.~Pilaud, M.~Pourmoradnasseri, K.~Qi, M.~Regenwetter, J.~Saunderson, D.O.~Theis, H.R.~Tiwary, J.P.~Vi\-el\-ma, K.~Zhao.}


%\textbf{results}{Положения, выносимые на защиту:}{%
%Текст о~положениях и~результатах.
%}

\textbf{Апробация результатов.}
Результаты диссертации докладывались и~обсуждались на следующих конференциях, семинарах и симпозиумах:
Российская конференция <<Дискретная оптимизация и~исследование операций>> (Алтай, 2010), 
XVI Международная конференция <<Проблемы теоретической кибернетики>> (Нижний Новгород, 2011),
Всероссийская конференция <<Математическое программирование и~приложения>> (Екатеринбург, 2011),
Международная конференция <<Дискретная геометрия>> посвященная 100-летию А.Д.~Александрова (Ярославль, 2012),
Международная топологическая конференция <<Александровские чтения>> (Москва, 2012),
XXI Международный симпозиум по математическому программированию (Берлин, 2012),
Международный симпозиум по комбинаторной оптимизации (Оксфорд, 2012),
Международная конференция <<Дискретная оптимизация и~исследование операций>> (Новосибирск, 2013), 
26-я конференция Европейского отделения по комбинаторной оптимизации (2013, Париж),
семинар Института математической оптимизации университета Отто фон Герике (Магдебург, 2013),
семинар по дискретной математике в~Свободном университете Берлина (2013),
5-й семинар по комбинаторной оптимизации (Каржез, Франция, 2014),
9-я Международная конференция по теории графов и~комбинаторике (Гренобль, Франция, 2014),
XIII Международная конференция <<Алгебра, теория чисел и~дискретная геометрия: 
современные проблемы и~приложения>>, посвященная 85-летию С.С.~Рышкова (Тула, 2015),
5-я Международная конференция по сетевому анализу (Нижний Новгород, 2015),
семинар лаборатории <<Дискретная и~вычислительная геометрия>> ЯрГУ им.~П.Г.~Демидова,
Нижегородский семинар по дискретной математике,
семинар по теории сложности вычислений ВЦ ФИЦ ИУ РАН,
семинар кафедры дифференциальной геометрии и~приложений МГУ им.~М.В.~Ломоносова,
семинар <<Дискретная и вычислительная геометрия>> ИППИ РАН, семинар кафедры математической теории интеллектуальных систем МГУ им.~М.В.~Ломоносова, семинар кафедры математической кибернетики МГУ им.~М.В.~Ломоносова, межкафедральный семинар МФТИ по дискретной математике.


\textbf{Структура и~объем диссертации.}
Диссертация состоит из~введения, восьми глав, заключения и~списка литературы из~194 наименований. В~конце каждой библиографической ссылки в~списке литературы приводятся страницы диссертации, где эта ссылка используется.
Общий объем диссертации "--- 250 страниц, включая 22 рисунка. % 2+4+0+10+0+4+1+1
%Список литературы включает 187 наименований на 17 страницах.


\textbf{Публикации.}
Диссертация продолжает исследования, начатые автором в~его кандидатской диссертации~\citemy{MaksimenkoDiss:2004}.

Материалы диссертации опубликованы в~16 печатных работах, из~них 
12~статей~\citemy{Maksimenko:2004,
	Maksimenko:2009,Maksimenko:2012DAN,Maksimenko:2012Cook,Maksimenko:2013k,Maksimenko:2013NP,Maksimenko:2013TSP,Maksimenko:2015DAN,BogomolovFMP:2015,Maksimenko:2016complexity,Maksimenko:2017,Maksimenko:2017LOP} "--- в~изданиях, индексируемых в Scopus,
3~статьи~\citemy{Maksimenko:2011,Maksimenko:2014MAIS,Maksimenko:2016bool} "--- в~журналах, входящих в RSCI, и~одна глава в~монографии~\citemy{BondBook:2008}.
%, одна глава в~учебном пособии~\citemy{BelovBM:2006}.
%{\color{red}2 статей в~сборниках трудов конференций и~1 тезисов докладов}.

Одна публикация подготовлена в~соавторстве с~Ю.В.~Богомоловым, К.~Пашковичем и~С.~Фиорини~\citemy{BogomolovFMP:2015}. Совместная работа над основным результатом этой публикации, включенным в текст диссертации, происходила следующим образом (каждый следующий этап работы опирался на предыдущие). Формулировка задачи в целом и доказательство существования расширения размера $2 \lfloor\log_2 (n-1)\rfloor + 2$ для циклического $n$"~угольника принадлежат диссертанту; конструктивное доказательство последнего факта найдено совместно с~С.~Фиорини; аналогичный результат для трехмерного циклического многогранника получен Ю.В.~Богомоловым; компактная расширенная формулировка для произвольной размерности найдена одновременно и независимо диссертантом и~К.~Пашковичем. 


\textbf{Финансовая поддержка.}
Исследования, включенные в~диссертацию, были поддержаны грантами РФФИ~00-01-00662-a, 03-01-00822-a; 
ФЦП «Научные и~научно-педагогические кадры инновационной России» на 2009--2013 годы (гос. контракт №~02.740.11.0207),
лабораторией <<Дискретная и~вычислительная геометрия>> ЯрГУ им.~П.\,Г.~Демидова
(грант Правительства РФ №~11.G34.31.0057),
проектами №~477 и~№~984 в~рамках базовой части гос. задания на НИР ЯрГУ (2014--2016~гг.) и~гос. заданием №~1.5768.2017/П220 на НИР ЯрГУ.


%\textbf{contrib}{Личный вклад автора.}{%
%Содержание диссертации и~основные положения, выносимые на защиту, отражают персональный вклад автора в~опубликованные работы.
%Подготовка к~публикации полученных результатов проводилась совместно с~соавторами, причем вклад диссертанта был определяющим. 
%Все представленные в~диссертации результаты получены лично автором.
%}

%%%%%%%%%%%%%%%%%%%%%%%%%%%%%%%%%%%%%%%%%%%%%%%%%%%%%%%%%%
%
%     Краткое содержание работы
%
%%%%%%%%%%%%%%%%%%%%%%%%%%%%%%%%%%%%%%%%%%%%%%%%%%%%%%%%%%

%\newpage % Для автореферата
%\section*{Краткое содержание работы}
\nsection{Краткое содержание работы}

%\textbf{Во Введении} обоснована актуальность диссертационной работы, сформулирована цель и~аргументирована научная новизна исследований, показана практическая значимость полученных результатов, представлены выносимые на защиту научные положения.

В~\textbf{первой главе} перечисляются и~уточняются базовые математические понятия и~факты, используемые далее в~основной части диссертации.
В~разделах~1.1 и~1.2 вводятся необходимые понятия 
теории графов и~теории выпуклых многогранников, соответственно.
В~разделе~1.3 приводятся используемые далее понятия теории сложности вычислений %задач и~алгоритмов 
и, в~частности, теории NP-полных задач. 
%\emph{Реальной сложностью задачи} далее будем называть современные эмпирические представления о её вычислительной сложности или, другими словами, трудоемкость асимптотически наиболее эффективного из известных в настоящее время алгоритмов для её решения.

Общепринятая формулировка задачи комбинаторной оптимизации уточняется в~разделе~1.4. Там же приводится определение \emph{линейной задачи комбинаторной оптимизации}, представляющей собой (с учетом сделанных в~тексте диссертации замечаний) тройку:
\begin{enumerate}
	\item Язык входных данных $L$, $L \in P$. Далее слово $I \in L$ называется \emph{кодом задачи}.
	\item \emph{Размерность} $d \from L \to \N$ (полиномиально вычислима).
	\item \emph{Предикат допустимости} $g \from \Z^d \times L \to \{\text{ложь}, \text{истина}\}$, $g \in \NP$, определяющий \emph{множество допустимых решений} 
	\[X = X(I) = \Set*{\bm{x} \in \Z^d \given \size(\bm{x}) = \poly(\size(I) + d(I)) \text{ и } g(\bm{x}, I)}.\]
\end{enumerate}
Входными данными \emph{индивидуальной} задачи (экземпляра задачи) являются её код~$I$ и~\emph{целевой вектор} $\bm{c} \in \Z^d$.
%Целевая функция линейна: $f(\bm{x}, \bm{c}) = \bm{c}^T \bm{x}$, где $\bm{x} \in X$.
Цель задачи "--- найти среди всех допустимых решений $X(I)$ такое, на котором целевая функция $\langle\bm{c},\bm{x}\rangle$ принимает максимальное значение.
Найденное решение $\bm{x}$ называется \emph{оптимальным.}
Сложность задачи оценивается относительно величины $\size(I) + d(I)$, называемой \emph{размером} задачи. (В частности, полиномиальная вычислимость понимается в сильном смысле.)

%%%%%%%%%%%%%%%%%%%%%%%%%%%%%%%%%%%%%%%%%%%%%%%%%%%%%%%%%%
%     Глава 2
%%%%%%%%%%%%%%%%%%%%%%%%%%%%%%%%%%%%%%%%%%%%%%%%%%%%%%%%%%

Во \textbf{второй главе} вводится специфичная для темы диссертационной работы терминология и~приводится обзор известных по этой теме фактов. 

В~разделе~2.1
вводится определение семейства многогранников линейной задачи комбинаторной оптимизации. С~каждым кодом $I$ соответствующей задачи связывается многогранник $\conv(X(I))$, представляющий собой выпуклую оболочку множества допустимых решений. Таким образом, за счет произвольности выбора кода $I$, образуется семейство многогранников задачи, называемое далее \emph{комбинаторным}. %Семейство многогранников называется \emph{комбинаторным}, если все три функции, определяющие соответствующую задачу, полиномиально вычислимы.
Везде далее многогранником, как правило, называется множество его вершин, то есть его V"~описание. 
Для случая, когда на целевой вектор $\bm{c}$ задачи накладываются дополнительные линейные ограничения вида $\langle\bm{c}, \bm{a_i}\rangle \le 0$, $i \in [k]$, соответствующий \emph{полиэдр задачи} определяется как сумма Минковского многогранника $\conv(X(I))$ и~конуса %<<неприемлемых>> целевых векторов 
$\cone\{\bm{a_1}, \dots, \bm{a_k}\}$.
В~этом же разделе приводятся определения некоторых, часто встречаемых в~литературе комбинаторных многогранников и~полиэдров: булева квадратичного многогранника $\BQP(n)$, $n \in \N$; 
многогранника задачи о~рюкзаке $\Knap(\bm{a},b)$, $\bm{a} \in \Z^n$, $b \in \Z$;
многогранников путей $\Path(n)$ и~орпутей $\Dipath(n)$, $n\in \N$;
многогранников гамильтоновых циклов $\TSP(n)$ и~гамильтоновых контуров $\ATSP(n)$;
перестановочного многогранника $\Perm(n)$; многогранника задачи о~назначениях $\Birk(n)$; многогранника $\Stable(G)$ независимых множеств в~графе $G=(V,E)$;
полиэдра задачи о~кратчайшем орпути с~ограничением неотрицательности длин контуров $\ShortP(n)$; некоторых других многогранников и~полиэдров.

В~разделе~2.2 обсуждается задача идентификации граней многогранников задач. Основное внимание уделено задаче идентификации смежности вершин. %, так как она часто встречается в~литературе. 
Перечислены известные по этой теме результаты.

Раздел~2.3 посвящен краткому обзору известных фактов для таких характеристик графов многогранников задач, как число вершин, диаметр и~кликовое число, а~также, связанных с~последними двумя характеристиками, гипотезы Хирша и~теории алгоритмов прямого типа.

В~разделе~2.4 вводится понятие расширения многогранника и~приводится краткий обзор известных по этой теме фактов. \emph{Расширением} многогранника $P \subseteq \R^d$ называется многогранник $Q \subseteq \R^n$ вместе с~аффинным отображением $\alpha \from \R^n \to \R^d$, удовлетворяющим условию $P = \alpha(Q)$.
Расширения, в~первую очередь, интересны тем, что задача оптимизации на многограннике $P$ сводится к~задаче оптимизации на его расширении $Q$.
Число линейных неравенств, необходимых для описания расширения $Q$, называется \emph{размером расширения}. Известны примеры, когда размер расширения оказывается существенно меньше числа неравенств, необходимых для описания исходного многогранника $P$. \emph{Сложностью расширения} $\xc(P)$ многогранника $P$ называется минимальный размер среди всех его расширений. 

%Хорошо известно, что сложность расширения ограничена снизу размерностью многогранника, а~сверху "--- числом его вершин и~числом гиперграней. Таким образом, сложность расширения можно рассматривать в~качестве верхней оценки сложности соответствующей оптимизационной задачи. Кроме того, эта характеристика обладает и~другими интересными свойствами. В~частности, ее величина ограничена снизу чисто комбинаторной характеристикой многогранника "--- числом прямоугольного покрытия матрицы инциденций вершин"=гиперграней многогранника~\cite{Yannakakis:1988}.

Пусть $M \in \{0,1\}^{n\times k}$~--- матрица инциденций.
Множество $I\times J$, где $I\subseteq [n]$, $J\subseteq [k]$, называется \emph{0"~прямоугольником} в~матрице~$M$, если $M(i,j) = 0$ для всех $i\in I$ и~$j\in J$.
\emph{Прямоугольным покрытием} матрицы $M$ называется множество 0"~прямоугольников, объединение которых 
совпадает с~множеством нулей в~$M$.
\emph{Числом прямоугольного покрытия} матрицы называется наименьшее число 0"~прямоугольников, необходимое для её прямоугольного покрытия.
Число прямоугольного покрытия матрицы инциденций вершин"=гиперграней  многогранника $P$ обозначаем $\rc(P)$.
Эта величина и сложность расширения $\xc(P)$ связаны с размерностью $\dim(P)$, числом вершин $\vertices(P)$, числом гиперграней $\facet(P)$ и числом всех граней $\face(P)$ следующими соотношениями~\cite{Yannakakis:1988,FioriniKPT:13}:
\[
\dim(P) + 1 \le \log_2 \face(P) \le \rc(P) \le \xc(P) \le \min\{\vertices(P), \facet(P)\}.
\]
Многочисленные известные факты говорят о~том, что $\rc(P)$ дает весьма точную нижнюю оценку, а $\xc(P)$ "--- верхнюю оценку реальной вычислительной сложности соответствующей оптимизационной задачи.
Под \emph{реальной сложностью задачи} комбинаторной оптимизации здесь и далее понимаются современные эмпирические представления о её вычислительной сложности. Так, например, реальная сложность NP"~трудных задач на сегодняшний день сверхполиномиальна, а реальная сложность задачи о совершенном паросочетании равна $\Theta(n^3)$.

В~разделе~2.5, на основе фактов, изложенных во второй главе,
формулируются общие вопросы, поиску ответов на которые посвящены последующие главы диссертации.
\begin{comment}
А именно, из~перечисленных фактов следует, что многогранники NP"~трудных задач во многих случаях обладают схожими свойствами. Например: NP"~полнота задачи распознавания несмежности вершин, небольшой диаметр графа, сверхполиномиальное кликовое число графа, сверхполиномиальные сложность расширения и~число прямоугольного покрытия матрицы инциденций вершин"=гиперграней.
Часто эти сходства обусловлены тесными связями геометрического характера, обнаруживаемыми в~разное время разными исследователями, когда многогранники одной задачи аффинно эквивалентны или являются проекциями некоторых граней многогранников другой задачи.
В~связи с~этим естественными являются следующие вопросы общего характера. 
Можно ли систематически использовать такой способ сравнения для различных семейств многогранников? 
Какие выводы на основе сравнений такого типа можно сделать в~отношении различных комбинаторно"=геометрических характеристик многогранников?
Ответы на эти вопросы содержатся в~главах 3--5.
Для каждой из~рассмотренных во второй главе характеристик также естественно задать следующий вопрос.
Есть ли связь между данной характеристикой многогранника и~сложностью соответствующей оптимизационной задачи?
В~связи с~этим возникают и~вопросы более общего характера.
Какие известные в~настоящее время комбинаторно"=геометрические характеристики многогранника наиболее адекватно отражают сложность соответствующей задачи?
Есть ли связь между комбинаторным типом многогранника и~сложностью задачи оптимизации на нем?
Исследованию этих проблем посвящены главы 7 и~8.
\end{comment}


%%%%%%%%%%%%%%%%%%%%%%%%%%%%%%%%%%%%%%%%%%%%%%%%%%%%%%%%%%
%     Глава 3
%%%%%%%%%%%%%%%%%%%%%%%%%%%%%%%%%%%%%%%%%%%%%%%%%%%%%%%%%%

\textbf{В~третьей главе} описан метод аффинной сводимости, активно используемый в~последующих двух главах.

В~первом разделе вводится определение аффинной сводимости задач, служащее основой для различных его модификаций в~последующих разделах.
Линейная задача комбинаторной оптимизации $(L,d,g)$ \emph{аффинно сводится} к~задаче $(L',d',g')$, если существуют вычислимые за полиномиальное (относительно размера первой задачи) время:
\begin{enumerate}
	\item 
	Преобразование $\tau \from L \to L'$.
	\item 
	Алгоритм построения для каждого кода $I \in L$ аффинного отображения 
	$\alpha\from \R^d \to \R^{d'}$, где $d = d(I)$, $d' = d'(\tau(I))$.
	\item 
	%Сюръекция
	Функция $\beta\from Y \to X$, где $X = X(I)$ "--- множество допустимых решений первой задачи, а~$Y$ "--- множество всех таких допустимых решений $\bm{y} \in X'(\tau(I))$ второй задачи, для каждого из~которых найдется целевой вектор $\bm{c} \in \R^d$ такой, что $\bm{y}$ является оптимальным решением второй задачи с~входом $(\tau(I), \alpha(\bm{c}))$.
	Причем для любого $\bm{y} \in Y$ и~любого $\bm{c} \in \R^d$, $\bm{y}$ является оптимальным решением второй задачи с~входом $(\tau(I), \alpha(\bm{c}))$ тогда и~только тогда, когда $\beta(\bm{y})$ является оптимальным решением первой задачи с~входом $(I,\bm{c})$.
	% Функция $\beta$  не обязана быть сюръекцией по следующим причинам. Пусть $x$ таков, что $x \ne \beta(y)$ ни для какого $y \in Y$. Тогда возможны две ситуации. 1. Если $x$ является в первой задаче единственно оптимальным при некотором целевом $\bm{c}$, то правая часть вышестоящей формулы (с $\iff$) не выполняется никогда, а левая часть будет выполняться при некотором $y$. 2. Если же $x$ не является единственно оптимальным ни при каком $c$, то им можно пренебречь.
\end{enumerate}

Прототипом этого определения послужило определение аффинной сводимости из~кандидатской диссертации автора~\citemy{MaksimenkoDiss:2004}.
Ключевые отличия: отсутствует требование биективности аффинного отображения $\alpha$ и~функции $\beta$; учтена зависимость множества допустимых решений от исходного кода задачи.


В~разделе~3.2 приводится 
%определение нормального веера многогранника и~
определение конусного разбиения пространства исходных данных задачи.
\emph{Конусным разбиением пространства исходных данных} задачи линейной оптимизации на множестве $X \subset \R^d$ называется совокупность конусов вида:
\[
K(\bm{x}) = \Set*{\bm{c}\in \R^d \given  \langle\bm{c}, \bm{x}\rangle \ge \langle\bm{c}, \bm{y}\rangle, \ \forall \bm{y} \in X}, 
\]
где $\bm{x} \in X$, причем в~разбиение включаются только те конусы, размерность которых равна размерности пространства $\R^d$.
%Конусы $K(\bm{x})$ и~$K(\bm{y})$ называются \emph{смежными}, если $\dim (K(\bm{x}) \cap K(\bm{y})) = d - 1$.
Конусное разбиение является двойственной к~многограннику $\conv(X)$ конструкцией.
%~\cite{BondBook:1995}. 
%В~частности, каждый конус соответствует вершине многогранника $P$ и~две вершины этого многогранника смежны тогда и~только тогда, когда соответствующие конусы смежны.
По аналогии с~конусным разбиением всего пространства определяется разбиение множества исходных данных $Q \subseteq \R^d$ (предполагается, что $Q$ "--- полиэдр) задачи линейной оптимизации на $X \subset \R^d$. Оно состоит из~полиэдров $K(\bm{x},Q) = K(\bm{x}) \cap Q$.
%, размерность которых совпадает с~размерностью $Q$. 
Это определение полезно в~тех случаях, когда на целевой вектор накладываются линейные ограничения. Например, в~классической задаче о~кратчайшем пути "--- ограничение неотрицательности длин ребер графа.
В~конце раздела вводится определение аффинной сводимости разбиений исходных данных задач, отличающееся от определения аффинной сводимости задач требованиями биективности аффинного отображения $\alpha$ и~функции~$\beta$.

В~разделе~3.3 вводится естественный способ сравнения многогранников: если многогранник $P$ аффинно эквивалентен многограннику~$Q$ или же его грани, используем обозначение $P \lea Q$. 
%Если же многогранники $P$ и~$Q$ аффинно эквивалентны, пишем $P =_A Q$. 
Соотношение $P \lea Q$ позволяет сравнивать различные комбинаторно"=геометрические характеристики многогранников $P$ и~$Q$: числа вершин и~гиперграней, некоторые свойства графов (например, кликовые числа), сложности расширений, числа прямоугольных покрытий матриц инциденций вершин"=гиперграней и~некоторые другие.
Далее в~этом разделе приводится несколько простых примеров использования соотношения $\lea$. В~конце раздела даны определения \emph{многогранника упаковок}, представляющего собой выпуклую оболочку множества 
\(\Pack(A) = \Set*{\bm{x}\in\{0,1\}^n \given A \bm{x} \le \bm{1}}\), где $A\in\{0,1\}^{m\times n}$,
и \emph{многогранника разбиений}
\(\Part(A) = \Set*{\bm{x}\in\{0,1\}^n \given A \bm{x} = \bm{1}}\).
Непосредственно из~определений следует \(\Part(A) \lea \Pack(A)\).

В~последнем разделе третьей главы вводится определение аффинной сводимости семейств многогранников, используемое в~четвертой главе.
По аналогии с размером задачи $(L,d,g)$, \emph{размером} многогранника $\conv(X(I))$ называем величину $\size(I) + d(I)$.
Cемейство многогранников $\Pf$ \emph{аффинно сводится} к~семейству многогранников $\Qf$, если найдутся полиномиально вычислимые (относительно размера многогранника $P\in \Pf$):
\begin{enumerate}
	\item 
	Преобразование $\tau$ кода $I$ каждого многогранника $P(I)\in \Pf$ в~код $I'$ многогранника $Q(I') \in \Qf$.
	\item 
	Аффинное отображение \(\alpha\from \R^d \to \R^{d'}\), $d = d(I)$, $d' = d'(\tau(I))$,
	такое, что многогранник $\alpha(P(I))$ является гранью (возможно несобственной) многогранника $Q(\tau(I))$ и~аффинно эквивалентен $P(I)$.
\end{enumerate}
Факт аффинной сводимости $\Pf$ к~$\Qf$ обозначаем так: $\Pf \propto_A \Qf$.  

Непосредственно из~определения и~перечисленных ранее фактов выводится следующее утверждение.
Пусть $\Pf \propto_A \Qf$ и~в семействе $\Pf$ есть многогранники, имеющие одно или несколько из~следующих свойств:
сверхполиномиальность числа вершин или гиперграней (относительно размера многогранника); сверхполиномиальное кликовое число графа многогранника; NP-полнота критерия несмежности вершин; сверхполиномиальное число прямоугольного покрытия; сверхполиномиальная сложность расширения.
Тогда в~$\Qf$ имеются многогранники с~теми же свойствами.

Основные результаты раздела~3.4:
\begin{enumerate}
\item 
Семейства многогранников независимых множеств, многогранников упаковок и~многогранников разбиений эквивалентны относительно аффинной сводимости~\citemy{Maksimenko:2015DAN}.
\item
Для каждого $n\in \N$ существует граф $G = (V,E)$, $|V| = n(n+1)$, $|E| = n(2n-1)$, такой, что $\BQP(n) \lea \Stable(G)$~\citemy{Maksimenko:2015DAN,Maksimenko:2016bool}.
Если же граф $G=(V,E)$ неполный, то соотношение $\Stable(G) \lea \BQP(n)$ невозможно ни при каком $n$.
\end{enumerate}
В~конце раздела устанавливается связь между аффинной сводимостью семейств многогранников и~аффинной сводимостью конусных разбиений пространств исходных данных задач. 

%Определение аффинной сводимости конусных разбиений опубликовано в~кандидатской диссертации автора~\citemy{MaksimenkoDiss:2004}. Определение аффинной сводимости многогранников опубликовано в~\citemy{Maksimenko:2017}.
%Результаты третьей главы опубликованы в~\citemy{Maksimenko:2015DAN,Maksimenko:2016bool}.

%%%%%%%%%%%%%%%%%%%%%%%%%%%%%%%%%%%%%%%%%%%%%%%%%%%%%%%%%%
%     Глава 4
%%%%%%%%%%%%%%%%%%%%%%%%%%%%%%%%%%%%%%%%%%%%%%%%%%%%%%%%%%

В~\textbf{главе~4} представлен ряд результатов, связанных с~понятием аффинной сводимости. 

По аналогии с~многогранниками упаковок и~разбиений, в~разделе~4.1 определяется
\emph{многогранник покрытий}
\(\Cover(M) = \Set*{\bm{x}\in\{0,1\}^n \given M \bm{x} \ge \bm{1}}\),
$M\in\{0,1\}^{m\times n}$.
\emph{Многогранником двойных покрытий} называется выпуклая оболочка множества
\(\DCP(B) =  \Set*{\bm{x}\in\{0,1\}^n \given B \bm{x} = \bm{2}}\),
где $B \in \{0,1\}^{m\times n}$, причем каждая строка матрицы $B$ содержит ровно четыре единицы и~не имеет нулевых столбцов.
Впервые это семейство многогранников было рассмотрено Мацуи~\cite{Matsui:1995},
им же было показано, что $\DCP(B) \lea \Cover(M)$, где матрица $M \in \{0,1\}^{4m\times n}$ содержит ровно три единицы в~каждой строке.
Основое внимание в~этом разделе уделяется специальным многогранникам $\NPadj(A)$, где матрица $A \in \{0,1\}^{m\times n}$ содержит ровно три единицы в~каждой строке. Эти многогранники являются многогранниками двойных покрытий.
Известно~\cite{Matsui:1995}, что задача распознавания несмежности вершин для $\NPadj$ NP"~полна. Основными результатами раздела являются два утверждения, опубликованные в~\citemy{Maksimenko:2017}:
\begin{enumerate}
	\item Многогранники независимых множеств $\Stable$ аффинно сводятся к~семейству многогранников $\NPadj$.
	\item Если многогранник $\NPadj(A)$ не является отрезком, то $\NPadj(A) \lea \Stable(G)$ невозможно ни для какого графа $G$.
\end{enumerate}
Последнее свойство говорит о~безусловном структурном отличии многогранников двойных покрытий от многогранников независимых множеств и~аффинно сводящихся к~ним семейств.

{\sloppy
Далее, в~разделе~4.2 рассматриваются семейства многогранников с~NP-полным критерием несмежности вершин: многогранники задачи о~рюкзаке $\KnapEq(\bm{a},b)$, многогранники задачи о~разбиении чисел $\PRT(\bm{a})$, многогранники задачи о~назначениях с~ограничением $\CAP(\bm{a},b)$, многогранники задачи о~выполнимости $\SAT(U,C)$, многогранники задачи о~частичном упорядочивании $\POP(n)$, многогранники кубических подграфов $\Cubic(n)$.
Приводится ссылка на результат Фиорини~\cite{Fiorini:2003} о~том, что семейство многогранников задачи о~3-выполнимости эквивалентно семейству многогранников задачи о~частичном упорядочивании с~точки зрения аффинной сводимости (понятие аффинной сводимости в~его работе не используется). В~той же работе показано, что многогранники задачи о~$k$"~выполнимости не могут быть аффинно сведены к~семейству многогранников задачи об $m$"~выполнимости, если $k > m$.
Основным результатом этого раздела является серия доказательств того, что многогранники двойных покрытий $\DCP$ аффинно сводятся к~перечисленным семействам~\citemy{Maksimenko:2012DAN,Maksimenko:2013NP}.

}

В~разделе~4.3 рассматриваются многогранники линейных порядков и~многогранники деревьев Штейнера в~графе. Показано, что булевы квадратичные многогранники $\BQP(n)$ аффинно сводятся к~первому семейству~\citemy{Maksimenko:2017LOP}, а~многогранники независимых множеств $\Stable(G)$ "--- ко второму.

В~разделе~4.4 рассматриваются семейства многогранников, имеющих простой критерий смежности вершин. С~точки зрения аффинной сводимости они разбиваются на два класса эквивалентности.
Многогранники трехиндексной задачи о~назначениях $\TAP(n)$ и~несколько семейств многогранников раскрасок графа ($\ColorA(G,k)$, $\ColorB(G)$ и~$\ColorC(G)$) лежат в~одном классе эквивалентности с~многогранниками независимых множеств. А~семейства многогранников квадратичной задачи линейных упорядочиваний $\QLOP(n)$ и~квадратичной задачи о~назначениях $\QAP(n)$ оказываются эквивалентны семейству булевых квадратичных многогранников $\BQP(n)$.
Результаты этого раздела опубликованы в~\citemy{Maksimenko:2016bool}.

В~разделе~4.5 рассматриваются семейства многогранников задач, тесно связанных с~задачей коммивояжера.
Показано, что многогранники задачи о~выполнимости $\SAT(U,C)$ аффинно сводятся к~многогранникам гамильтоновых контуров $\ATSP(n)$~\citemy{Maksimenko:2011}.
Одним из~следствий этого утверждения является то, что любой $d$-мерный 0/1"~многогранник на $2^d - k$ вершинах ($0 \le k \le 2^d - 1$) аффинно эквивалентен некоторой грани многогранника $\ATSP(n)$ при $n = (2k+1)d$.
Ранее Биллера и~Сарангараджан~\cite{Billera:1996} доказали это утверждение для $n = (4k+1)d$ иными средствами.
Второй результат раздела, опубликованный в~\citemy{Maksimenko:2013TSP}, устанавливает следующую связь между семействами $\BQP$ и~$\ATSP$:
$\BQP(m) \lea \ATSP(n)$, где $n = 2 m^2 - m$.
В~подразделе~4.5.2 рассматриваются семейства многогранников следующих задач: гамильтонов цикл, гамильтонов (ор)путь, $s$"~$t$ (ор)путь, гамильтонов $s$"~$t$ (ор)путь.
Показано, что многогранники гамильтоновых контуров аффинно сводятся ко всем этим семействам. Из этого следует, в~частности, что графы многогранников этих семейств обладают сверхполиномиальным кликовым числом и~задача распознавания несмежности вершин для них NP"~полна.

По аналогии с~булевыми квадратичными многогранниками в~разделе~4.6 вводятся в~рассмотрение булевы многогранники $\BPP(n,p)$ степени $p$. 
Для $p=2$, $\BPP(n,p)$ совпадает с~$\BQP(n)$, а~для $p=1$, $\BPP(n,p)$ "--- $n$"~мерный 0/1-куб.
Показано, что $\BPP(n,p)$ $s$"~смежностен при
$s \le p + \left\lfloor p / 2 \right\rfloor$.
Для $m \in \N$ и~$k \ge 2m$ доказано, что $\BPP(k,2m) \lea \BQP(n)$ при $n > 2 \binom{k}{m}$.
Следовательно, для любого $k \in \N$ и~$n \ge 2^{2\cdot \lceil k/3\rceil}$, 
$\BQP(n)$ имеет $k$"~смежностную грань со сверхполиномиальным числом
$2^{{\Theta}\left( n^{1 / {\left\lceil k/3\right\rceil}}\right)}$ вершин.
Из этого и~из перечисленных ранее аффинных свед\'{е}ний следует, что во всех упоминаемых выше семействах многогранников NP-трудных задач имеются многогранники, содержащие $k$"~смежностные грани со сверхполиномиальным (относительно размерности многогранника) числом вершин. Результаты раздела опубликованы в~\citemy{Maksimenko:2013k}.

В~последнем разделе главы 4 рассматриваются задача о~назначениях и~задача о~кратчайшем орпути с~ограничением неотрицательности длин контуров. Показано, что конусное разбиение множества исходных данных последней аффинно сводится к~конусному разбиению пространства исходных данных первой~\citemy{MaksimenkoDiss:2004}.
Как следствие, граф полиэдра кратчайших орпутей $\ShortP(n+1)$ является подграфом графа многогранника Биркгофа $\Birk(n)$, $n \in \N$.


%%%%%%%%%%%%%%%%%%%%%%%%%%%%%%%%%%%%%%%%%%%%%%%%%%%%%%%%%%
%     Глава 5
%%%%%%%%%%%%%%%%%%%%%%%%%%%%%%%%%%%%%%%%%%%%%%%%%%%%%%%%%%

В~первом разделе \textbf{главы~5} вводится понятие расширенной аффинной сводимости, отличающееся от аффинной сводимости отсутствием ограничения биективности аффинного отображения. 
%Название связано с~понятием расширения многогранника. Если некоторая грань многогранника $Q$ или же весь этот многогранник является расширением многогранника $P$, будем использовать обозначение $P \lee Q$.
Семейство многогранников $\Pf$ \emph{расширенно аффинно сводится} к~семейству многогранников $\Qf$, если найдутся полиномиально вычислимые (относительно размера многогранника $P\in \Pf$):
\begin{enumerate}
\item 
Преобразование $\tau$ кода $I$ каждого многогранника $P = P(I)\in \Pf$ в~код $I'$ многогранника $Q = Q(I') \in \Qf$.
\item 
Система линейных уравнений $D\bm{y}=\bm{c}$, задающая грань
\(F = \Set{\bm{y}\in Q \given D\bm{y}=\bm{c}}\)
многогранника $Q$.
\item 
Аффинное отображение \(\beta\from \R^{d'} \to \R^{d}\), $d = d(I)$, $d' = d'(\tau(I))$, такое, что $P = \beta(F)$.
\end{enumerate}
Обозначение: $\Pf \propto_E \Qf$.  
Во многих случаях доказательство соотношений вида $\Pf \propto_E \Qf$ принципиально проще, чем соотношений $\Pf \propto_A \Qf$.
Минусом такого ослабления ограничений является потеря некоторых полезных свойств.
В~частности, такие свойства, как NP"~полнота проверки несмежности вершин, сверхполиномиальность числа гиперграней и~сверхполиномиальность кликового числа графа, вообще говоря, не наследуются при расширенном аффинном свед\'{е}нии.
Далее в~этом же разделе приводятся некоторые свойства общего характера для этого типа сводимости. Например, показано, что если многогранник $P \subseteq \R^d$ является образом многогранника $Q \subseteq \R^n$ при аффинном отображении $\pi \from \R^n \to \R^d$ и, кроме того, $\pi(\ext Q) = \ext P$, то граф многогранника $P$ является подграфом графа многогранника $Q$.

В~разделе~5.2 приводится несколько примеров расширенной аффинной сводимости. В~целом, благодаря ослаблению условий, доказательства соответствующих утверждений оказываются значительно более простыми, чем для (обычной) аффинной сводимости.

В~разделе~5.3 показано, что любое семейство многогранников, предикат допустимости $g$ которого принадлежит классу NP, расширенно аффинно сводится к~булевым квадратичным многогранникам~\citemy{Maksimenko:2012Cook}.
Тем самым, все упоминаемые выше в~настоящей работе семейства многогранников оказываются эквивалентны друг другу относительно расширенной аффинной сводимости.


%%%%%%%%%%%%%%%%%%%%%%%%%%%%%%%%%%%%%%%%%%%%%%%%%%%%%%%%%%
%     Глава 6
%%%%%%%%%%%%%%%%%%%%%%%%%%%%%%%%%%%%%%%%%%%%%%%%%%%%%%%%%%

В~\textbf{главе~6} рассматриваются циклические многогранники.
Как известно~\cite{McMullen:1970}, они обладают максимальным числом граней (любой размерности) среди всех выпуклых многогранников той же размерности и~с таким же числом вершин.
Благодаря этому обстоятельству циклические многогранники являются хорошей экспериментальной базой для проверки разного рода теоретических утверждений.
В~первом разделе главы вводится определение циклического многогранника 
\(\CP_d(T) = \Set*{(t, t^2, \dots, t^d) \in \R^d \given t \in T}\), где $T \subset \R$ конечно,
и~формулируется условие четности Гейла~\cite{Gale:1963}, идентифицирующее подмножества вершин, образующих гиперграни этого многогранника.
В~разделе~6.2 для многогранника $\CP_d([n])$ приводится описание расширенной формулировки размера $2\bigl(2\lfloor \log_2(n-1)\rfloor+2\bigr)^{\lfloor d/2 \rfloor}$ при $2 \le d < n$ (результат опубликован в~совместной работе~\citemy{BogomolovFMP:2015}).
В~разделе~6.3 вычисляется точное значение для диаметра $\dc$ ридж"=графа $d$"~мерного циклического многогранника на $n$ вершинах:
\(\dc= n-d  - 
\left\lceil 
\frac{n-2d}{ \left\lfloor \frac d 2 \right\rfloor+1}
\right\rceil\)
при $n > 2d$~\citemy{Maksimenko:2009}. 
Равенство $\dc= n-d$ при $d < n \le 2d$  было доказано Кли в~1964 году~\cite{Klee:1964}.


%%%%%%%%%%%%%%%%%%%%%%%%%%%%%%%%%%%%%%%%%%%%%%%%%%%%%%%%%%
%     Глава 7
%%%%%%%%%%%%%%%%%%%%%%%%%%%%%%%%%%%%%%%%%%%%%%%%%%%%%%%%%%

\textbf{Глава~7} посвящена теории алгоритмов прямого типа.
% для решения линейных задач комбинаторной оптимизации.
В~первом разделе приводится ее описание, заимствованное из~\cite{BondBook:1995}. %(см. также~\citemy{BondBook:2008}).
Ключевой особенностью алгоритма прямого типа является то, что его сложность ограничена снизу кликовым числом графа многогранника (конусного разбиения) решаемой задачи.
\begin{comment}
Известно~(см. обзор в~разделе~2.3.3), что для классических полиномиально разрешимых задач (сортировка, минимальное остовное дерево, минимальный разрез) эта характеристика не превосходит размерности многогранника.
%(В~разделе~7.2 доказано, что задача о~кратчайшем пути с~ограничением неотрицательности длин контуров тоже входит в~этот список.)
С~другой стороны, в~главах~3 и~4 показано, что булевы квадратичные многогранники $\BQP$ аффинно сводятся к~многогранникам таких NP-трудных задач, как коммивояжер, рюкзак, 3-выполнимость, 3-сочетание, покрытие и~упаковка множества, раскраска графа, кубический подграф и~многие другие. Учитывая, что кликовое число графа многогранника $\BQP(n)$ равно $2^n$, кликовые числа графов многогранников указанных задач также сверхполиномиальны по размерности многогранников.
Кроме того, в~\cite{BondBook:1995} установлено, что некоторые алгоритмы сортировки, жадный алгоритм для минимального остовного дерева,
алгоритм Дейкстры для кратчайшего пути, алгоритм Хелда--Карпа 
и реализация алгоритма ветвей и~границ для задачи коммивояжера
являются прямыми или <<прямыми>>.
\end{comment}

В~разделе~7.2 приводится доказательство критерия смежности для графов решений трех полиномиально разрешимых вариантов задачи о кратчайшем пути.
%На основе этого критерия и доказательства теоремы 2 из~\cite{Bondarenko:1993SW3A} делается вывод о том, что кликовое число для задачи о~кратчайшем пути в~орграфе на $n$ вершинах с~ограничением неотрицательности длин контуров (а также для задачи с~классическим ограничением неотрицательности длин дуг) равно $\lfloor n^2 / 4\rfloor$.
На основе этого критерия и доказательства теоремы 2 из~\cite{Bondarenko:1993SW3A} делается вывод о том, что кликовое число для всех трех вариантов равно~$\lfloor n^2 / 4\rfloor$, где $n$ "--- число вершин графа, в котором ищется кратчайший путь.
С~учетом результата раздела~4.7, это дает нижнюю оценку $\lfloor (n+1)^2 / 4\rfloor$ для кликового числа графа многогранника задачи о~назначениях $\Birk(n)$.
%В~этой связи отметим следующий факт.
%В~1977~г. Бруальди и~Гибсон показали~\cite[Theorem~6.1, Corollary~6.5]{Brualdi:1977II}, что любая 2"~смежностная грань многогранника $\Birk(n)$, число вершин которой не равно шести, является симплексом, а~максимальное число вершин такой грани совпадает с~упомянутой выше оценкой $\lfloor (n+1)^2 / 4\rfloor$.
Результаты этого раздела опубликованы в~\citemy{MaksimenkoDiss:2004} и~\citemy{Maksimenko:2004}.

В~разделе~7.3 перечисляется ряд фактов, демонстрирующих ограниченность применимости этого подхода к~оценке сложности задач.
Приводится доказательство того, что алгоритм Куна--Манкреса для задачи о~назначениях не является алгоритмом прямого типа.
Кроме того, описывается достаточно универсальный способ модификации алгоритмов,
существенно не меняющий их трудоемкости, но гарантированно выводящий их из~класса алгоритмов прямого типа. Результаты раздела опубликованы в~\citemy{Maksimenko:2014MAIS}.

%%%%%%%%%%%%%%%%%%%%%%%%%%%%%%%%%%%%%%%%%%%%%%%%%%%%%%%%%%
%     Глава 8
%%%%%%%%%%%%%%%%%%%%%%%%%%%%%%%%%%%%%%%%%%%%%%%%%%%%%%%%%%

В~\textbf{главе~8} изучается следующий вопрос.
Можно~ли, зная только комбинаторные свойства 
%(однозначно определяемые матрицей инциденций вер\-шин-ги\-пер\-гра\-ней) 
многогранника, отделить NP"~трудные задачи от полиномиально разрешимых?
% (в рамках современных представлений о~сложности задач)?
В~разное время в~качестве таких характеристик сложности рассматривались: число вершин многогранника, число его гиперграней, диаметр и~кликовое число графа, число прямоугольного покрытия матрицы инциденций вершин"=гиперграней.

В~разделе~8.1 приводятся примеры семейств многогранников, для которых значения упомянутых выше характеристик (за исключением числа прямоугольного покрытия) существенно отличаются от %эмпирически сложившихся к настоящему времени представлений о 
реальной вычислительной сложности соответствующих оптимизационных задач.

В~разделе~8.2 приводится описание NP-трудной задачи оптимизации, многогранники $\CBQP(n)$ которой получены в~результате небольшого шевеления (пертурбации) вершин булева квадратичного многогранника $\BQP(n)$. Доказано, что многогранник $\CBQP(n)$, $n \in \N$, симплициален. Следовательно, согласно~\cite{FioriniKPT:13}, число прямоугольного покрытия его матрицы инциденций вершин"=гиперграней полиномиально: $\rc(\CBQP(n)) = O(n^5)$. 
%Таким образом, это первый известный пример семейства многогранников NP"~трудной задачи, число прямоугольного покрытия для которых полиномиально.
Это означает, что ни одно из~семейств многогранников NP"~трудных задач, рассмотренных ранее в~главах~3--5
не может быть расширенно аффинно сведено к~семейству $\CBQP$.
С~другой стороны, в~\cite[Theorem~4]{Braun:2015} установлено, что любой многогранник, аппроксимирующий $\BQP(n)$ с~точностью $O(1/n)$, имеет сложность расширения порядка $2^{\Omega(n)}$.
Следовательно, сложность расширения для $\CBQP(n)$ экспоненциальна: $\xc(\CBQP(n)) = 2^{\Omega(n)}$.
%Для сравнения напомним, что многогранники задачи о~паросочетаниях в~полном графе обладают аналогичными свойствами: их сложность расширения экспоненциальна, а~число прямоугольного покрытия полиномиально. 

В~разделе~8.3 приводятся примеры двух линейных задач комбинаторной оптимизации, многогранники которых комбинаторно эквивалентны 
и~длины двоичной записи координат вершин этих многогранников одинаковы. 
При этом первая задача разрешима за~полиномиальное время, 
а~вторая "--- NP"~трудна.
%(Этого удается достичь за счет экспоненциальной сложности распознавания вершины многогранника последней задачи.)
Этот результат говорит о~том, что 
в рамках наиболее распространенной вычислительной модели (машина Тьюринга) 
ни одна чисто комбинаторная характеристика многогранника
(однозначно определяемая его решеткой граней)
не дает возможности отделить полиномиально разрешимые задачи от NP"~трудных.
%задач с~экспоненциальной сложностью.

Результаты последней главы опубликованы в~\citemy{Maksimenko:2016complexity}.

%В~\textbf{заключении} подводятся итоги диссертационного исследования.