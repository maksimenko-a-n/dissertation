% !TeX encoding = UTF-8 Unicode
% !TEX root = MaksimenkoThesis.tex

%%%%%%%%%%%%%%%%%%%%%%%%%%%%%%%%%%%%%%%%%%%%%%%%%%%%%%%%%%
%
%     Примеры аффинной сводимости
%
%%%%%%%%%%%%%%%%%%%%%%%%%%%%%%%%%%%%%%%%%%%%%%%%%%%%%%%%%%
\chapter{Примеры аффинной сводимости}
\label{chap:AffExamples}

%\begin{flushright}
%При изучении наук примеры полезнее правил.\\ \emph{И.~Ньютон}
%\end{flushright}
%\bigskip
%\hfill
%\begin{minipage}{0.4\textwidth}
%	При изучении наук примеры полезнее правил.
%	\begin{flushright}
%		Исаак Ньютон
%	\end{flushright}
%\end{minipage}

В~последнем разделе предыдущей главы в качестве примера было показано, что семейства многогранников независимых множеств $\Stable$, многогранников разбиений $\Part$ и многогранников упаковок $\Pack$ эквивалентны в смысле аффинной сводимости, а семейство булевых квадратичных многогранников $\BQP$ аффинно сводится к ним. Причем сводимость $\Stable \propto_A \BQP$ невозможна.
В~этой главе будет доказан ряд аналогичных утверждений для часто встречаемых в литературе семейств многогранников. Выявленные взаимосвязи (за исключением последнего раздела главы) изображены на рис.~\ref{fig:AffTree}. Каждая стрелка означает аффинную сводимость семейства многогранников, расположенного в начале стрелки, к семейству многогранников, на которое указывает стрелка. Перечеркнутая (крестом) стрелка означает, что аффинная сводимость в данном направлении невозможна. Сплошные стрелки "--- известные 
%или имеющие простое доказательство 
факты. Пунктирные стрелки "--- факты, выявленные в настоящей работе. Для семейства многогранников двойных покрытий и других семейств, расположенных ниже его на рис.~\ref{fig:AffTree}, задача проверки несмежности вершин NP-полна. Для семейств, расположенных выше, эта задача полиномиально разрешима. Исключением являются многогранники деревьев Штейнера, для которых статус этой задачи не установлен.

\tikzset{cross/.style={cross out, draw=black, minimum size=2*(#1-\pgflinewidth), inner sep=0pt, outer sep=0pt},
%default radius will be 4pt. 
cross/.default={4pt}}
\begin{figure}[tbh]
\centering
\begin{tikzpicture}[scale=1.3, >=stealth', thick] %, radius=2pt, delta angle=180]
\label{hierarchy}
\tikzstyle{every node}=[rounded corners,text centered,draw=black,minimum size=23pt]

\node (Cut) at (-4.1, 1.3) {\hyperref[def:CutPolytope]{Разрезы}};
\node (BQuadr) at (0, 1.3) {\hyperref[eq:BQP]{Булево квадр. прогр.}};
\node (QAss) at (4.1, 1.3) {\hyperref[eq:QAP]{Квадр. назначения}};
\draw[<->] (Cut) to (BQuadr);
\draw[->] (QAss) to[bend right = 5] (BQuadr);
\draw[->,dashed] (BQuadr) to[bend right = 5] (QAss);

%\tikzstyle{every node}=[rounded corners,text centered,draw=black,fill=lightyellow]

\node (3Assign) at (4.3, -1.05) {\hyperref[sec:3Ass]{3-Назначения}};
\node (Color) at (2.2, 0) {\hyperref[sec:Color]{Раскраска графа}};
\node (SS) at (0, -1.3) {\hyperlink{Stable}{Независ. множ.}};
%			\draw[->,draw=red] (Cut) to (Part);
\draw[->,dashed] (BQuadr) to[bend right = 10] (SS);
\draw[->] (SS) to[bend right = 10] node[cross] {} (BQuadr);
\node (Pack) at (-3.0, -1.3) {\hyperlink{def:Pack}{Упаковки}};
\node (SetPart) at (-1.7, 0) {\hyperref[eq:Part]{Разбиения}};
\draw[->] (Pack) to (SS);
\draw[->] (SetPart) to (Pack);
\draw[->] (SS) to (SetPart);
\draw[->] (SS) to[bend right = 8] (Color);
\draw[->,dashed] (Color) to[bend right = 8] (SS);
\draw[->] (3Assign) to[bend right = 5] (SS);
\draw[->,dashed] (SS) to[bend right = 5] (3Assign);

%\tikzstyle{every node}=[rounded corners,text centered,draw=black,fill=lightgray]

\node (Steiner) at (3.9, -2.15) {\hyperref[sec:SteinerTree]{Деревья Штейнера}};
\draw[->,dashed] (SS) to (Steiner);

\node (LOP) at (-4.6, 0) {\hyperref[sec:LOP]{Лин. упоряд.}};
\draw[->,dashed] (BQuadr) to[out=-172,in=30] (LOP);
\draw[->] (LOP) to node[cross]{} (BQuadr);

%\tikzstyle{every node}=[rounded corners,text centered,draw=black,fill=lightgreen]

\node (DCover) at (0, -3.0) {\hyperref[def:DCP]{Двойные покрытия}};
\draw[->,dashed] (SS) to[bend right = 20] (DCover);
\draw[->,dashed] (DCover) to[bend right = 20] node[cross] {} (SS);
\draw[->,dashed] (LOP) to[out=-135,in=180] (DCover);

\node (Part) at (-2.9, -4.3) {\hyperref[eq:KnapEq]{Разбиение чисел}};
\draw[->,dashed] (DCover) to (Part);
\node (3SAT) at (0.0, -5.6) {\hyperref[subsec:k-Sat&POP]{3-Выполнимость}};
\draw[->, dashed] (DCover) to (3SAT);
\node (Order) at (3.7, -5.6) {\hyperlink{def:POP}{Частич. упоряд.}};
\draw[<->] (3SAT) to (Order);
\node (Cubic) at (3.2, -4.3) {\hyperref[def:Cubic]{Кубические подграфы}};
\draw[->,dashed] (DCover) to (Cubic);

\node (Knapsack) at (-5.05, -5.6) {\hyperref[eq:KNAP]{Рюкзак}};
\draw[->] (Part) to (Knapsack);
\node (SAT) at (0.0, -6.9) {\hyperref[subsec:k-Sat&POP]{Выполнимость}};
\draw[->] (3SAT) to[bend right = 20] (SAT);
\draw[->] (SAT) to[bend right = 20] node[cross] {} (3SAT);
\node (Cover) at (3.2, -6.9) {\hyperref[def:Cover]{Покрытия}};
\draw[->] (Cover) to[bend right = 5] (SAT);
\draw[->,dashed] (SAT) to[bend right = 5] (Cover);

\node (Assign) at (-3.8, -6.9) {\hyperref[def:CAP]{Назначения с огранич.}};
\draw[->,dashed] (Part) to (Assign);
% 0, 2.4, 4.65
\node (ATSP) at (-2.4, -8.2) {\hyperref[sec:Travelling]{Гамильтонов контур}};
\draw[->,dashed] (SAT) to (ATSP);
\node (Path) at (0.9, -8.2) {\hyperlink{def:PathPolytope}{$s$-$t$ Путь}};
\draw[->] (ATSP) to (Path);
\node (TSP) at (3.9, -8.2) {\hyperref[sec:Travelling]{Гамильтонов цикл}};
\draw[->,dashed] (Path) to (TSP);
\end{tikzpicture}
\caption{Аффинная сводимость многогранников NP-трудных задач}
\label{fig:AffTree}
\end{figure}

В~первом разделе главы приводятся определения многогранников покрытий и двойных покрытий множества. Основое внимание в этом разделе уделяется специальным многогранникам $\NPadj(A)$, где матрица $A \in \{0,1\}^{m\times n}$ содержит ровно три единицы в каждой строке. Эти многогранники
являются многогранниками двойных покрытий и обладают следующими свойствами:
\begin{enumerate}
\item $\Stable \propto_A \NPadj$.
\item Задача распознавания несмежности вершин для $\NPadj$ NP"~полна.
\item Если многогранник $\NPadj(A)$ не является отрезком, то $\NPadj(A) \lea \Stable(G)$ невозможно ни для какого графа $G$.
\end{enumerate}
Последнее свойство говорит о безусловном структурном отличии многогранников двойных покрытий от многогранников независимых множеств и аффинно сводящихся к ним семейств.

Далее, в разделе~\ref{sec:PolytopesWithNPadj} рассматриваются семейства многогранников с NP-полным критерием несмежности вершин: многогранники задачи о рюкзаке, многогранники задачи о разбиении чисел, многогранники задачи о назначениях с ограничением, многогранники задачи о $k$-выполнимости, многогранники задачи о частичном упорядочивании, многогранники кубических подграфов. Показано, что многогранники двойных покрытий аффинно сводятся к этим семействам.

В~разделе~\ref{sec:LOP&Steiner} рассматриваются многогранники линейных порядков и многогранники деревьев Штейнера в графе. Показано, что булевы квадратичные многогранники аффинно сводятся к первому семейству, а многогранники независимых множеств "--- ко второму.

В~разделе~\ref{sec:PolytopesSimpleAdj} рассматриваются семейства многогранников, имеющих простой критерий смежности вершин. С точки зрения аффинной сводимости они разбиваются на два класса эквивалентности.
Многогранники трехиндексной задачи о назначениях и несколько семейств многогранников раскрасок графа лежат в одном классе эквивалентности с многогранниками независимых множеств. А~многогранники квадратичной задачи линейных упорядочиваний и квадратичной задачи о назначениях оказываются эквивалентны семейству булевых квадратичных многогранников.

Семейства многогранников для различных вариаций задачи коммивояжера рассматриваются в разделе~\ref{sec:TravellingAll}. Показано, что многогранники задачи о выполнимости аффинно сводятся к многогранникам асимметричной задачи коммивояжера, а последние, в свою очередь, аффинно сводятся к остальным, рассматриваемым в этом разделе семействам. В~целом картина обнаруженных взаимосвязей изображена на рис.~\ref{fig:TSPall}.

По аналогии с булевыми квадратичными многогранниками в разделе~\ref{sec:BQP-power} вводятся в рассмотрение
булевы многогранники $\BQP(n,p)$ степени $p$. 
Для $p=2$, $\BQP(n,p)$ совпадает с $\BQP(n)$. 
Для $p=1$, $\BQP(n,p)$ "--- $n$"~мерный 0/1-куб.
Показано, что $\BQP(n,p)$ $s$"~смежностен при
$s \le p + \left\lfloor p / 2 \right\rfloor$.
Для $m \in \N$ и $k \ge 2m$ доказано, что $\BQP(k,2m) \lea \BQP(n)$ при $n = \Theta(\binom{k}{m})$.
Следовательно, для любого $k \in \N$ и $n \ge 2^{2\cdot \lceil k/3\rceil}$, 
$\BQP(n)$ имеет $k$"~смежностную грань со сверхполиномиальным числом
$2^{{\Theta}\left( n^{1 / {\left\lceil k/3\right\rceil}}\right)}$ вершин.

В~последнем разделе главы рассматриваются задача о назначениях и задача о кратчайшем орпути с ограничением неотрицательности длин контуров. Показано, что конусное разбиение множества исходных данных последней аффинно сводится к конусному разбиению пространства исходных данных первой.
Как следствие, граф полиэдра кратчайших орпутей $\ShortP(n+1)$ является подграфом графа многогранника Биркгофа $\Birk(n)$, $n \in \N$.


\section{Многогранники покрытий и двойных покрытий}
\label{sec:DoubleCovering}

По аналогии с многогранниками упаковок и разбиений множества,
\emph{многогранником покрытий множества} называется выпуклая оболочка множества
\begin{equation*}
\label{def:Cover}
\Cover(M) = \Set*{\bm{x}\in\{0,1\}^n \given M \bm{x} \ge \bm{1}},
\quad \text{где } M\in\{0,1\}^{m\times n}.
\end{equation*}

\emph{Многогранником двойных покрытий} будем называть выпуклую оболочку множества
\begin{equation*}
\label{def:DCP}
\DCP(B) =  \Set*{\bm{x}\in\{0,1\}^n \given B \bm{x} = \bm{2}},
\end{equation*}
где $B \in \{0,1\}^{m\times n}$, причем каждая строка матрицы $B$ содержит ровно четыре единицы и не имеет нулевых столбцов.
По-видимому, впервые это семейство многогранников было рассмотрено Мацуи~\cite{Matsui:1995}.
Им же была установлена связь между многогранниками покрытий и двойных покрытий.

\begin{theorem}[Мацуи {\cite[theorem 4.3]{Matsui:1995}}]
Для каждой матрицы $B \in \{0,1\}^{m\times n}$ с четырьмя единицами в каждой строке несложно описать матрицу $M \in \{0,1\}^{4m\times n}$ с тремя единицами в каждой строке, что $\DCP(B) \lea \Cover(M)$.
\end{theorem}

Известно также, что задача проверки несмежности вершин для многогранников двойных покрытий NP-полна~\cite{Matsui:1995}.
Точнее, она NP-полна для некоторого подсемейства семейства $\DCP$, о котором пойдет речь ниже.
С целью упрощения обсуждения свойств этого подсемейства, мы приведем более удобное его описание, чем в первоисточнике~\cite{Matsui:1995}.

Прежде всего отметим, что вопрос <<Содержит ли многогранник разбиений $\Part(A)$ хотя бы одну точку?>> является NP-полной задачей, даже если каждая строка матрицы $A$ содержит ровно три единицы~\cite{Matsui:1995,Garey:1982}.
С каждым многогранником $\Part(A)$, $A \in \{0,1\}^{m\times n}$, удовлетворяющим этому условию, свяжем многогранник, множество вершин $\NPadj(A) \subset \{0,1\}^{3n+3}$ которого определим следующим образом.
Для трех координат вектора $\bm{x} \in \NPadj(A)$ введем особые обозначения $y_1, y_2, y_3$. 
Каждой координате $z_j$ вектора $\bm{z} = (z_1,\dots,z_n) \in \Part(A)$ будут соответствовать три координаты $x_j$, $\bar{x}_j$ и $x'_j$ вектора $\bm{x} \in \NPadj(A)$, и два ограничения
\begin{align}
x_j + \bar{x}_j &= 1, \label{eq:adj01}\\
y_1 + y_2 + x'_j + \bar{x}_j&= 2. \notag
\end{align}
А для каждого ограничения вида $z_i + z_j + z_k = 1$ из описания $\Part(A)$
(случай, когда $A$ не содержит ни одной строки, исключаем из рассмотрения)
добавим к описанию множества $\NPadj(A)$ уравнение
\[
y_3 + x_i + x'_j + x'_k = 2.
\]

Заменяя каждое уравнение~\eqref{eq:adj01} в описании многогранника $\NPadj(A)$ на 
\[
a + b + x_j + \bar{x}_j = 2, \quad a = 0, \quad b = 1,
\]
приходим к следующему выводу.

\begin{prop}
Для каждой матрицы $A \in \{0,1\}^{m\times n}$, имеющей ровно три единицы в каждой строке, несложно описать матрицу $B \in \{0,1\}^{(2n+m)\times(3n+5)}$,
что $\NPadj(A) \lea \DCP(B)$.
\end{prop}	

Более того, в \cite{Matsui:1995} описан многогранник двойных покрытий, аффинно
эквивалентный $\NPadj(A)$, но его описание требует большего числа переменных и
уравнений.

Обратим теперь внимание на то, что ограничения 
\[
y_1 = 0, \quad y_2 = 1, \quad y_3 = 1,
\]
определяют грань многогранника $\NPadj(A)$, аффинно эквивалентную многограннику $\Part(A)$.
Таким образом,
\begin{equation}
\label{eq:PartNPadj}
\Part(A) \lea \NPadj(A).
\end{equation}	
То же самое верно и для следующих наборов ограничений:
\begin{enumerate}
\item[1)] $y_1 = 1$, $y_2 = 0$, $y_3 = 1$;
\item[2)] $y_1 = 0$, $y_2 = 1$, $y_3 = 0$;
\item[3)] $y_1 = 1$, $y_2 = 0$, $y_3 = 0$.
\end{enumerate}
Введем для этих граней (точнее, множеств их вершин) следующие обозначения:
\begin{align*}
F_1 &= \Set*{\bm{x}\in \NPadj(A) \given y_1 = 0, \ y_2 = 1, \ y_3 = 1}, \\
F_2 &= \Set*{\bm{x}\in \NPadj(A) \given y_1 = 1, \ y_2 = 0, \ y_3 = 1}, \\
F_3 &= \Set*{\bm{x}\in \NPadj(A) \given y_1 = 0, \ y_2 = 1, \ y_3 = 0}, \\
F_4 &= \Set*{\bm{x}\in \NPadj(A) \given y_1 = 1, \ y_2 = 0, \ y_3 = 0}.
\end{align*}
%Обозначим эти грани (точнее, множества их вершин) $F_1$, $F_2$, $F_3$, $F_4$, 
%в порядке упоминания.
Заметим, что никакие две из этих четырех граней не имеют общих точек.
Кроме того,
\begin{equation}
\label{eq:F4F3}
F_4 = \Set*{\bm{1} - \bm{x} \given \bm{x} \in F_1}, \qquad
F_3 = \Set*{\bm{1} - \bm{x} \given \bm{x} \in F_2}.
\end{equation}

Ограничениям
\[
y_1 = y_2 = 0
\]
удовлетворяет ровно одна вершина многогранника $\NPadj(A)$, имеющая координаты
\[
x'_j = \bar{x}_j = 1, \quad y_3 = x_j = 0, \qquad j\in[n].
\]
Обозначим эту вершину $\bm{x^0}$.
Аналогично, если 
\[
y_1 = y_2 = 1,
\]
то 
\[
x'_j = \bar{x}_j = 0, \quad y_3 = x_j = 1, \qquad j\in[n].
\]
Обозначим эту вершину $\bm{\bar{x}^0}$.
Очевидно, $\bm{\bar{x}^0} = \bm{1} - \bm{x^0}$.

Из приведенных выше рассуждений следует, что 
\[
\NPadj(A) = F_1 \cup F_2 \cup F_3 \cup F_4 \cup \{\bm{x^0}, \bm{\bar{x}^0}\},
\]
причем никакие два из этих пяти множеств не имеют общих точек.
Кроме того, вершины $\bm{x^0}$ и $\bm{\bar{x}^0}$ смежны тогда и только тогда, когда $F_1 = \emptyset$ (в противном случае $\conv\{F_1 \cup F_4\}$ и $\conv\{\bm{x^0}, \bm{\bar{x}^0}\}$ имеют общую точку $\bm{1}/2$).
Таким образом, в силу того, что $F_1$ аффинно эквивалентна $\Part(A)$, приходим к следующему выводу.

\begin{theorem}[Мацуи {\cite[theorem 4.1]{Matsui:1995}}]
Задача проверки несмежности вершин $\bm{x^0}$ и $\bm{\bar{x}^0}$ многогранника $\NPadj(A)$ NP-полна.
\end{theorem}
\begin{corollary}
Для семейств многогранников покрытий $\Cover$ и двойных покрытий $\DCP$ задача проверки несмежности вершин NP-полна.
\end{corollary}	

Покажем теперь, что соотношение~\eqref{eq:PartNPadj} выполнено и в тех случаях, когда число единиц в строках матрицы $A$ отличается от трех.
Точнее, для каждой матрицы $B$ (с любым числом единиц в строках) можно построить матрицу $A$, имеющую три единицы в каждой строке, что
\[
\Part(B) \lea \NPadj(A).
\]

Согласно лемме~\ref{lem:StablePart}, для любого графа $G=(V,E)$ существует матрица $A\in\{0,1\}^{m\times n}$, $m = |E|$, $n = |V|+|E|$, имеющая ровно по три единицы в каждой строке и такая, что $\Stable(G) =_A \Part(A)$. Вместе с соотношением~\eqref{eq:PartNPadj} это дает аффинную сводимость 
\[
\Stable \propto_A \NPadj. 
\]
Но $\Part \propto_A \Stable$ (см. теорему~\ref{thm:Class1}), следовательно, $\Part \propto_A \NPadj$.

Как известно~\cite{Chvatal:1975}, многогранники $\Stable(G)$ имеют простой критерий проверки смежности вершин.
Соответственно, в предположении $\NP \ne \textup{P}$ аффинная сводимость $\NPadj \propto_A \Stable$ невозможна.
Покажем, что и без условия $\NP \ne \textup{P}$ ни один
нетривиальный многогранник семейства $\NPadj$ не может быть гранью многогранников семейства $\Stable$.

\begin{theorem}[\cite{Maksimenko:2017}]\label{thm:DCPStable}
Если многогранник $\NPadj(A)$ не является отрезком, то соотношение $\NPadj(A) \lea \Stable(G)$ невозможно ни для какого графа $G$.
\end{theorem}
\begin{proof}
Как было замечено выше, многогранник $\NPadj(A)$ обязательно содержит пару вершин $\bm{x^0}$ и $\bm{\bar{x}^0}$, и некоторое количество 
четверок вершин вида $\bm{x^{2i-1}} \in F_1$, $\bm{\bar{x}^{2i-1}} \in F_4$, $\bm{x^{2i}} \in F_2$, $\bm{\bar{x}^{2i}} \in F_3$, $i\in [k]$, $k \ge 1$.
Причем, согласно~ \eqref{eq:F4F3},
\begin{equation}
\label{3xpairs}
\bm{x^0} + \bm{\bar{x}^0} = \bm{x^{2i-1}} + \bm{\bar{x}^{2i-1}} = \bm{x^{2i}} + \bm{\bar{x}^{2i}}.
\end{equation}

Предположим, что $\NPadj(A)$ аффинно эквивалентен некоторой грани 
\[
H = \{\bm{y^0}, \bm{\bar{y}^0}, \ldots, \bm{y^{2k}}, \bm{\bar{y}^{2k}}\}
\] 
многогранника $\Stable(G)$ для некоторого графа $G = (V,E)$.
Очевидно, вершины этой грани должны наследовать свойство~\eqref{3xpairs}:
\begin{equation}
\label{3pairs}
\bm{y^0} + \bm{\bar{y}^0} = \bm{y^{2i-1}} + \bm{\bar{y}^{2i-1}} = \bm{y^{2i}} + \bm{\bar{y}^{2i}}.
\end{equation}
Покажем, что в многограннике $\Stable(G)$ есть еще пара вершин $\bm{y^*}$ и $\bm{\bar{y}^*}$, для которых
\begin{equation}
\label{ThGoal}
\bm{y^*} + \bm{\bar{y}^*} = \bm{y^0} + \bm{\bar{y}^0}.
\end{equation}
Это будет означать, что пересечение $\conv\{\bm{y^*}, \bm{\bar{y}^*}\}$ и $\conv(H)$ не пусто.
То есть $H$ не является гранью $\Stable(G)$.

\begin{figure}[hb]
	\[
	\begin{aligned}
	\bm{y^0}&=(\overbrace{\text{\texttt{1,0,1,1,}}}^{I}
	\overbrace{\text{\texttt{0,0,0,0,1,1,1}}}^{J})\\[-1.0ex]
	&\phantom{=(\text{\texttt{1,0,1,1, }}}^{j_0} \\[-2.0ex]
	\bm{\bar{y}^0}&=(\text{\texttt{1,0,1,1,}}
	\underbrace{\text{\texttt{1,1,1,1,}}}_{U_0\vphantom{\bar{U}_0}}\!
	\underbrace{\text{\texttt{0,0,0}}}_{\bar{U}_0})
	\end{aligned}
	\]
	\caption{Множества индексов $I$, $J$, $U_0$, $\bar{U}_0$.}
	\label{fig:IJU}
\end{figure}

Пусть $\bm{y^0} = (y^0_1, \ldots, y^0_m)$ и $\bm{\bar{y}^0} = (\bar{y}^0_1, \ldots, \bar{y}^0_m)$, где $m$ "--- число вершин графа $G$.
Рассмотрим множество
\[
I = \Set*{i\in [m] \given y^0_i = \bar{y}^0_i}.
\]
Так как каждая вершина в $\Stable(G)$ является 0/1-вектором, то из~\eqref{3pairs} и~\eqref{ThGoal} следует
\begin{equation}
\label{eq:8}
y^*_i = \bar{y}^*_i = y^0_i = \bar{y}^0_i = \cdots = y^{2k}_i = \bar{y}^{2k}_i  \quad \text{ при } i\in I.
\end{equation}
Далее будем рассматривать только те координаты, значения которых различны для каждой пары вершин (см. рис.~\ref{fig:IJU}):
\[
J = \Set*{j\in [m] \given y^0_j + \bar{y}^0_j = 1} = [m] \setminus I.
\]
Очевидно, $J \ne \emptyset$.

Зафиксируем какой нибудь индекс $j_0 \in J$ и для каждого $i \in \{0,1,\dots, 2k\}$ определим множество
\[
U_i =\begin{cases}
\Set*{j\in J \given y^i_j = 1},& \text{если } y^i_{j_0} = 1,\\
\Set*{j\in J \given \bar{y}^i_j = 1},& \text{иначе.}
\end{cases} 
\]
По построению все эти множества попарно различны и $j_0 \in U_i$ (см. рис.~\ref{fig:IJU}).
Для каждого $U_i$ рассмотрим его дополнение $\bar{U}_i = J \setminus U_i$.
Согласно данному определению, для любого $i \in \{0,1,\dots, 2k\}$ и для любых $p,r \in U_i$ (а также для любых $p,r \in \bar{U}_i$) найдется вершина $\bm{y} \in H$ такая, что $y_p = y_r = 1$. 
То есть неравенство $y_p + y_r \le 1$ отсутствует в описании многогранника $\Stable(G)$.

Далее нам понадобится определение \emph{симметрической разности} двух множеств $X$ и $Y$:
\[
X \symdiff Y = (Y \setminus X) \cup (X \setminus Y).
\]
Симметрическая разность обладает следующими свойствами:
\begin{enumerate}
	\item $X \symdiff Y = \emptyset \iff X = Y$.
	\item Результат выражения $X \symdiff Y \symdiff Z$ не зависит от перестановки множеств и порядка выполнения операций.
	% (коммутативность и ассоциативность).
	\item $X \symdiff Y = Z \iff X \symdiff Z = Y$.
\end{enumerate}

Введем в рассмотрение множество
\[
S = S(i,j,t) = U_i \symdiff U_j \symdiff U_t, \quad 0\le i < j < t \le 2k,
\]
Рассмотрим вектор $\bm{y^*} = \bm{y^*}(S)$ с координатами
\[
y^*_i = \begin{cases}
y^0_i,& \text{при } i \in I,\\
1,& \text{при } i \in S,\\
0,& \text{при } i \in J \setminus S,
\end{cases}
\]
и вектор $\bm{\bar{y}^*} = \bm{\bar{y}^*}(S)$ с координатами
\[
\bar{y}^*_i = \begin{cases}
y^0_i,& \text{при } i \in I,\\
0,& \text{при } i \in S,\\
1,& \text{при } i \in J \setminus S,
\end{cases}
\]

\begin{lemma}
	Векторы $\bm{y^*}$ и $\bm{\bar{y}^*}$ принадлежат $\Stable(G)$.
\end{lemma}
\begin{proof}
	Достаточно показать, что если вершины грани $H$ удовлетворяют некоторому неравенству вида $y_p + y_r \le 1$ из описания многогранника $\Stable(G)$, то $\bm{y^*}$ и $\bm{\bar{y}^*}$ тоже ему удовлетворяют.
	
	Возможны несколько случаев.
	
	\textbf{I.} Пусть $p,r\in I$, $p \ne r$.
	Так как при $i \in I$ $i$-е координаты вершин грани $H$ и векторов $\bm{y^*}$ и $\bm{\bar{y}^*}$ совпадают, то из того, что неравенство $y_p + y_r \le 1$  выполнено для $H$ следует, что оно также выполнено и для векторов $\bm{y^*}$ и $\bm{\bar{y^*}}$.
	
	\textbf{II.} Пусть $p \in I$, $r \in J$.
	(Случай $r \in I$, $p \in J$ разбирается аналогично.)
	Тогда $y^0_r + \bar{y}^0_r = y^*_r + \bar{y}^*_r = 1$.
	Следовательно, $\max\{y^*_r, \bar{y}^*_r\} = \max\{y^0_r, \bar{y}^0_r\} = 1$.
	И опять из выполнения неравенства $y_p + y_r \le 1$ для $H$ следует, что оно также выполнено для $\bm{y^*}$ и $\bm{\bar{y^*}}$.
	
	\textbf{III.} Пусть $p \in S$, $r \in J\setminus S$.
	(Случай $r \in S$, $p \in J\setminus S$, разбирается аналогично.)
	Тогда $y^*_p + y^*_r = \bar{y}^*_p + \bar{y}^*_r = 1$ и требуемое ограничение выполнено.
	
	\textbf{IV.} Пусть $p,r \in S$, $p \ne r$, где $S = S(i,j,t)$.
	(Случай $p,r \in J \setminus S$, $p \ne r$, разбирается аналогично.)
	Покажем, что в этом случае $p$ и $r$ принадлежат одновременно одному из шести множеств:
	$U_i$, $U_j$, $U_t$, $\bar{U}_i$, $\bar{U}_j$, $\bar{U}_t$.
	Если это действительно так, то (как было сказано выше при определении этих множеств) неравенство $y_p + y_r \le 1$ отсутствует в описании многогранника $\Stable(G)$.
	
	Итак, предположим, что $p,r \in U_i \symdiff U_j \symdiff U_t$, и покажем, что тогда $p$ и $r$ принадлежат одновременно одному из множеств:
	$U_i$, $U_j$, $U_t$, $\bar{U}_i$, $\bar{U}_j$, $\bar{U}_t$.
	Заметим, что $U_i \symdiff U_j \symdiff U_t$ представляет собой объединение четырех множеств:
	\[
	U_i \symdiff U_j \symdiff U_t = (U_i \cap U_j \cap U_t) \cup (U_i \cap \bar{U}_j \cap \bar{U}_t) \cup (\bar{U}_i \cap U_j \cap \bar{U}_t) \cup (\bar{U}_i \cap \bar{U}_j \cap U_t).
	\]
	Если $p$ и $r$ принадлежат одному из этих четырех множеств, то требуемое условие выполнено.
	Нетрудно проверяется, что условие выполнено и в случае, когда $p$ и $r$ принадлежат разным множествам.
	Например, если $p \in U_i \cap \bar{U}_j \cap \bar{U}_t$, а $r \in \bar{U}_i \cap \bar{U}_j \cap U_t$, то $p,r \in \bar{U}_j$.
\end{proof}

Для завершения доказательства теоремы остается показать, что найдется множество $S$ такое, что вектор $\bm{y^*} = \bm{y^*}(S)$ (а вместе с ним и вектор $\bm{\bar{y}^*} = \bm{\bar{y}^*}(S)$) будет отличаться от всех остальных вершин грани $H$.

\begin{lemma}
	Существует $t \in \{2, 3, \dots, 2k\}$ такое, что $S(0,1,t)$ отличается от каждого из множеств $U_p$ и $\bar{U}_p$, $0\le p \le 2k$.
\end{lemma}
\begin{proof}
	Начнем с простого случая, когда $k=1$.
	Нетрудно проверить, что
	\[
	S(0,1,2) = U_0 \symdiff U_1 \symdiff U_2 \not\in \{U_0, U_1, U_2, \bar{U}_0, \bar{U}_1, \bar{U}_2\},
	\]
	так как все множества различны и, кроме того, все $U_i$ имеют общий элемент $j_0$:
	\[
	U_i \symdiff U_j \ne \emptyset \quad \text{и} \quad U_i \symdiff U_j \ne J, \quad \text{при } i\ne j.
	\]
	
	Предположим теперь, что $k>1$.
	Рассмотрим тройки вида 
	\[
	U_0 \symdiff U_1 \symdiff U_i, \quad 2 \le i \le 2k.
	\]
	Как было замечено выше,
	\[
	U_0 \symdiff U_1 \symdiff U_i \not\in \{U_0, U_1, U_i\}.
	\]
	Кроме того, $U_0 \symdiff U_1 \symdiff U_i \ne \bar{U}_j$ при любом $j \in \{0,1,\dots,2k\}$, так как $j_0 \notin \bar{U}_j$.
	Предположим, что 
	\begin{equation}
	\label{eq:sym-pair}
	U_0 \symdiff U_1 \symdiff U_i = U_j
	\end{equation}
	при некотором $j \in \{2,3,\dots,2k\} \setminus \{i\}$.
	Но тогда, в силу свойств симметрической разности, %(свойство~\ref{prop:symdiff} на с.~\pageref{prop:symdiff}),
	\[
	U_0 \symdiff U_1 \symdiff U_j = U_i.
	\]
	Причем
	\[
	U_0 \symdiff U_1 \symdiff U_t \ne U_j, \quad \forall t \ne i,
	\]
	так как иначе $U_t = U_i$, что невозможно по условию.
	По тем же соображениям,
	\[
	U_0 \symdiff U_1 \symdiff U_t \ne U_i, \quad \forall t \ne j.
	\]
	
	Таким образом, все возможные индексы $i$ и $j$, для которых выполнено условие \eqref{eq:sym-pair}, разбиваются на непересекающиеся пары. 
	Но множество $\{2,3,\dots,2k\}$ содержит нечетное число индексов.
	Значит, обязательно найдется $i \in \{2,3,\dots,2k\}$, для которого
	$S(0,1,i) = U_0 \symdiff U_1 \symdiff U_i$ будет отличаться от каждого из множеств $U_p$ и $\bar{U}_p$, $0\le p \le 2k$.
\end{proof}
Теорема доказана.
\end{proof}


Учитывая, что $\NPadj(A)$ является гранью многогранника двойных покрытий, получаем
\begin{corollary}
$\DCP \npropto_A \Stable$ и $\Cover \npropto_A \Stable$.
\end{corollary}	

%%%%%%%%%%%%%%%%%%%%%%%%%%%%%%%%%%%%%%%%%%%%%%%%%%%%%%%
%
%  Многогранники с NP-полным критерием несмежности вершин
%
%%%%%%%%%%%%%%%%%%%%%%%%%%%%%%%%%%%%%%%%%%%%%%%%%%%%%%%

\section[Многогранники с NP-полным критерием несмежности вершин]{Многогранники с NP-полным критерием\\ несмежности вершин}
\label{sec:PolytopesWithNPadj}

В~этом разделе мы рассмотрим несколько семейств многогранников, к которым аффинно сводятся многогранники двойных покрытий.
Как следствие, для всех этих многогранников задача проверки несмежности вершин NP-полна.
Начнем с простых примеров.

\subsection{Многогранники задачи о рюкзаке}

Если в описании многогранников задачи о рюкзаке (уравнение~\eqref{eq:KNAP} на с.~\pageref{eq:KNAP}) неравенство заменить равенством, то получим \emph{многогранники задачи о рюкзаке с равенством}:
\begin{equation}
\label{eq:KnapEq}
\KnapEq(\bm{a},b) = \Set*{\bm{x} \in \{0,1\}^{n} \given \bm{a}^T \bm{x} = b}, \qquad \bm{a} \in \Z^n,\quad b \in \Z.
\end{equation}
Очевидно,
\[
\KnapEq(\bm{a},b) \lea \Knap(\bm{a},b).
\]
Подсемейство многогранников $\KnapEq(\bm{a},b)$, удовлетворяющее условию $2b = \bm{a}^T \bm{1}$, обозначим $\PRT(\bm{a})$.
Оно непосредственно связано с задачей о разбиении чисел, входящей в список из 21 задачи, NP-полнота которых была доказана в фундаментальной работе Карпа~\cite{Karp:1972}.

\begin{theorem}[\cite{Maksimenko:2013NP}]
Для любой матрицы $B \in \{0,1\}^{m\times n}$, имеющей ровно четыре единицы в каждой строке, можно за полиномиальное (от её размера) время построить вектор $\bm{a} \in \Z^n$ такой, %, $\|\bm{a}\|_{\infty} \le 4^m$,
что $\DCP(B) =_A \PRT(\bm{a})$.
\end{theorem}
\begin{proof}\sloppy
Достаточно заметить, что система линейных диофантовых уравнений $B \bm{x} = \bm{2}$ может быть агрегирована в одно уравнение (см. \cite{Padberg:1972, Kovalev:1977}).
Например, можно сложить все уравнения системы $B \bm{x} = \bm{2}$, предварительно умножив каждое из них на $4^i$, где $i\in[m]$ "--- номер уравнения.
\end{proof}
\begin{corollary}
Задача проверки несмежности вершин для семейств многогранников $\PRT(\bm{a})$, $\KnapEq(\bm{a},b)$ и $\Knap(\bm{a},b)$ NP-полна.
\end{corollary}
Ранее этот результат об NP-полноте был получен в работах~\cite{Chung:1980,Geist:1992,Matsui:1995} иными методами.


\subsection{Многогранники задачи о назначениях с ограничением}

Пусть $\bm{a} \in \Z^{n\times n}$, $b \in Z$.
\emph{Многогранник задачи о назначениях с ограничением} определяется как выпуклая оболочка множества
\begin{equation*}
\label{def:CAP}
\CAP(\bm{a},b) = \Set*{\bm{x} \in \Birk(n) \given \bm{a}^T \bm{x} = b},
\end{equation*}
где $\Birk(n)$ "--- многогранник Биркгофа.

Интерес к этой задаче и её многограннику обусловлен многочисленными приложениями
(см., например, \cite{Toktas:2006}).
NP-полнота задачи распознавания несмежности вершин этого многогранника была обнаружена Альфаки и Мурти~\cite{Alfakih:1998}.

Покажем, что многогранники задачи о рюкзаке с равенством аффинно сводятся к этому семейству.

\begin{theorem}[\cite{Maksimenko:2013NP}]
\label{thm:KnapEqCAP}
Для каждого $\bm{a} \in \Z^n$ и $b \in Z$ найдется $\bm{c} \in \Z^{2n\times 2n}$ такой, что
\[
\KnapEq(\bm{a},b) \lea \CAP(\bm{c},b).
\]
\end{theorem}

\begin{proof}
Воспользуемся тем, что многогранник $\Birk(2n)$ содержит грань, являющуюся $n$"~мерным кубом. %~\cite{Billera:1996}.
Эта грань лежит в пересечении гиперплоскостей $x_{ij} = 0$, для всех $(i,j)$ не входящих в множество 
\[
%\Set*{(i,j) \in [2n]^2 \given \text{$i$ нечетно и $j \in \{i,i+1\}$, либо $i$ четно и $j \in \{i-1,i\}$}}.
S = \big\{(i,i) \mid i\in[2n]\big\} \cup \big\{(2i-1,2i) \mid i\in[n]\big\} \cup \big\{(2i,2i-1) \mid i\in[n]\big\}.
\]

Для данного по условию вектора $\bm{a} = (a_1,\dots,a_n) \in \Z^n$ определим координаты вектора $\bm{c} \in \Z^{2n\times 2n}$ следующим образом:
\[
c_{ij} = \begin{cases}
a_p,& \text{если $i=j=2p$, $p\in[n]$},\\
0,& \text{иначе.}
\end{cases}
\]

Нетрудно увидеть, что множество вершин указанной выше грани ($n$"~мерного куба), попадающих гиперплоскость $\bm{c}^T \bm{x} = b$, аффинно эквивалентно многограннику $\KnapEq(\bm{a},b)$.
С другой стороны, оно же является гранью многогранника $\CAP(\bm{c},b)$.
\end{proof}


\subsection[Многогранники задачи о k-выполнимости и задачи о частичном упорядочивании]{Многогранники задачи о $k$-выполнимости\\ и задачи о частичном упорядочивании}
\label{subsec:k-Sat&POP}

Пусть $U = \{u_1, u_2, \ldots, u_d\}$ "--- множество булевых переменных.
Не уменьшая общности, полагаем $u_i \in \{0, 1\}$, где $1$ трактуется как <<истина>>, а $0$ "--- как <<ложь>>.
Пусть $C = \{D_1, D_2, \ldots, D_n\}$ "--- некоторый набор дизъюнкций
(конъюнкция дизъюнкций) над $U$.
%Кроме того, далее предполагаем, что каждая дизъюнкция состоит ровно из $k$ литералов.

Для каждого выполняющего конъюнкцию $C$ набора значений переменных
рассмотрим соответствующий вектор $\bm{u} = (u_1, \dots, u_d)\in\{0,1\}^d$.
Множество всех таких векторов обозначаем $\SAT(U,C)$,
а его выпуклая оболочка называется \emph{многогранником задачи о выполнимости}.
В~случае, если каждая дизъюнкция в $C$  состоит ровно из $k$ литералов, он называется \emph{многогранником задачи о $k$"~выполнимости}.
 
\begin{theorem}[\cite{Maksimenko:2013NP}]
\label{thm:DCPSAT}
Для каждой матрицы $B \in \{0,1\}^{m\times n}$, имеющей ровно четыре единицы в каждой строке, можно за линейное время построить конъюнкцию $C = \{D_1,\dots, D_{8m}\}$ над $U = \{u_1,\dots, u_n\}$ такую, что $|D_i|=3$, $i \in [8m]$, и 
\[
\DCP(B) =_A \SAT(U,C).
\]
\end{theorem}

\begin{proof}
Каждое уравнение вида $x_1 + x_2 + x_3 + x_4 = 2$ из системы $B \bm{x} = \bm{2}$ можно заменить набором из восьми дизъюнкций: 
\[
\bigvee_{j\neq i} x_j \ \mbox{ и } \ \bigvee_{j\neq i} \bar{x}_j, 
\qquad i=1,2,3,4.
\]
\end{proof}
\begin{corollary}
Задача распознавания несмежности двух произвольных вершин многогранника задачи о 3"~выполнимости NP-полна.
\end{corollary}

Ранее, Фиорини~\cite{Fiorini:2003} доказал этот факт (NP-полноты несмежности вершин) непосредственно. 
В~той же работе Фиорини показано, что многогранники задачи о 3"~выполнимости тесно связаны с многогранниками задачи о частичном упорядочивании.
Сформулируем ее на языке теории графов. 
Подграф $D' = (V, A')$ полного орграфа $D = (V, A)$ будем называть \emph{\hypertarget{def:POP}{частичным порядком}}, если он ацикличен и транзитивен:
\[
((u, v)\in A') \& ((v, w)\in A') \Rightarrow (u, w)\in A'.
\]

\emph{Многогранник частичных порядков} представляет собой выпуклую оболочку множества $\POP(n)$ характеристических векторов всех частичных порядков полного орграфа на $n$ вершинах.

\begin{theorem}[Фиорини {\cite[Lemma 3.2]{Fiorini:2003}}]
Многогранник задачи о 3"~выполнимости $\SAT(U,C)$, $U=\{u_1,\dots,u_d\}$, $C = \{D_1,\dots,D_n\}$, $|D_i|=3$, $i \in [n]$,  аффинно эквивалентен грани многогранника частичных порядков $\POP(m)$ при $m = 3d+7n$.
\end{theorem}

Из этого, в частности, следует NP-полнота задачи распознавания несмежности вершин для многогранника частичных порядков.

Более того, семейства многогранников частичных порядков и многогранников задачи о 3"~выполнимости эквивалентны с точки зрения аффинной сводимости.

\begin{theorem}[Фиорини {\cite[Theorem 1.2]{Fiorini:2003}}]
Многогранник $\POP(m)$ аффинно эквивалентен грани многогранника задачи о 3"~выполнимости $\SAT(U,C)$, где $|U| = n(n-1)+1$, $|C| = n(n-1)(n-3/2)$.
\end{theorem}

Рассмотрим теперь отдельно семейства многогранников задачи о $k$"~выполнимости для разных $k \ge 3$ (случай $k=2$ не представляет интереса).
Почти очевидно следующее утверждение.

\begin{lemma}[Фиорини {\cite[Lemma 4.1]{Fiorini:2003}}]
%Пусть $U = \{u_1,\dots,u_d\}$, $C = \{C_1,\dots,C_n\}$, $|C_i| \le k$, $i\in[n]$.
Многогранник задачи о выполнимости $\SAT(U,C)$ при $|D_i| \le k$, $D_i\in C$, является гранью многогранника задачи о $k$"~выполнимости $\SAT(U',C')$, где $|C'|=|C|$, $|U'|=|U| + k - m$, $m = \min\Set{|D_i| \given D_i \in C}$.
\end{lemma}

Оказывается, семейства многогранников задачи о $k$"~выполнимости для разных значений $k \ge 3$ относятся к разным классам эквивалентности с точки зрения аффинной сводимости.

\begin{theorem}[Фиорини {\cite[Proposition 4.1]{Fiorini:2003}}]
\label{thm:kSAT}
\sloppy
Существуют примеры многогранников задачи о $k$"~выполнимости $\SAT(U,C)$ при $|U|=k$ и $|C|\ge 1$ такие, что для любых $U'$ и $C'$, при условии $|D'| < k$, $D' \in C'$, справедливо
\[
\SAT(U,C) \not\lea \SAT(U',C').
\]
\end{theorem}

Рассмотрим теперь семейство всех многогранников задачи о выполнимости $\SAT = \{\SAT(U,C)\}$ и покажем, что оно эквивалентно многогранникам покрытий.

\begin{theorem}
$\Cover \propto_A \SAT \propto_A \Cover$.
\end{theorem}
\begin{proof}
Пусть $A \in \{0,1\}^{m\times n}$.
Рассмотрим $U=\{u_1,\dots,u_n\}$ и для каждого неравенства вида
\[
x_{i_1}+x_{i_2}+\dots+x_{i_k} \ge 1
\]
системы $A\bm{x} \ge \bm{1}$ добавим в $C$ дизъюнкцию
\[
u_{i_1} \vee u_{i_2} \vee \dots \vee u_{i_k}.
\]
Очевидно,
\[
\Cover(A) =_A \SAT(U,C).
\]

Докажем второе утверждение теоремы.
Пусть $U=\{u_1,\dots,u_n\}$ "--- некоторый набор булевых переменных и $C=\{D_1,\dots,D_m\}$ "--- конъюнкция дизъюнкций над $U$.
Для каждого $i\in[n]$ введем в рассмотрение пару переменных $x_i$ и $\bar{x}_i$,
и свяжем их неравенством 
\begin{equation}
\label{eq:proofSATCover}
x_i + \bar{x}_i \ge 1.
\end{equation}
Переменные $x_i$ будут соответствовать литералам $u_i$ в $C$, а переменные $\bar{x}_i$ "--- литералам $\bar{u}_i$.
Для каждой дизъюнкции $D_j$, $j\in[m]$, заменяя литералы соответствующими переменными, расставляя вместо $\vee$ знаки сложения,
и добавляя справа ${} \ge 1$, получим $m$ неравенств.
Вместе с неравенствами \eqref{eq:proofSATCover} они описывают некоторый многогранник $\Cover(A)$.
Заменяя неравенство в \eqref{eq:proofSATCover} равенством, получим грань этого многогранника,
аффинно эквивалентную $\SAT(U,C)$.
\end{proof}

\subsection{Многогранники кубических подграфов}

Пусть $G=(V,E)$ "--- полный граф на $k$ вершинах и пусть $T_k$ "--- множество всех его кубических подграфов. (Напомним, что степень каждой вершины кубического графа равна трем.)
\emph{Многогранником кубических подграфов} называется выпуклая оболочка множества 
\begin{equation*}
\label{def:Cubic}
\Cubic(k) = \Set*{\chi(H)\in\{0,1\}^E \given H \in T_k}.
\end{equation*}

Известно, что задача проверки несмежности вершин этого многогранника NP-полна~\cite{Bondarenko:1996}.
Покажем, что многогранники двойных покрытий являются гранями этого многогранника.

\begin{theorem}[\cite{Maksimenko:2013NP}]
$\DCP(A) \lea \Cubic(k)$, где	
матрица $A \in \{0,1\}^{m\times n}$ имеет ровно четыре единицы в каждой строке и
$k = 6m+4s+2t$, $s$ "--- число столбцов матрицы $A$, содержащих лишь одну единицу, $t$ "--- число столбцов матрицы $A$, содержащих ровно две единицы.
\end{theorem}
\begin{proof}
Через $\bm{x}=(x_i)\in\{0,1\}^n$ будем обозначать вектор, удовлетворяющий условию $A\bm{x}=\bm{2}$, определяющему вершины многогранника $\DCP(A)$.
Через $\bm{y}=(y_{lh})\in\R^{k(k-1)/2}$ "--- характеристический вектор кубического подграфа графа $G=(V, E)$ из описания многогранника $\Cubic(k)$.
Воспользуемся тем, что для многогранника $\Cubic(k)$ гиперплоскости вида $y_{lh}=0$	% и $y_{lh}=1$ 
являются опорными.
Искомую грань будем определять как пересечение нескольких таких гиперплоскостей.
Очевидно, если выполнено $y_{lh}=0$, то мы можем рассматривать только те подграфы графа $G$, которые не содержат ребро $(v_l, v_h)$.
Таким образом, чтобы определить нужную нам грань, мы опишем некоторый подграф $G'$ графа $G$, подразумевая, что для не входящих в него ребер выполнено $y_{lh}=0$.
	
Разберем в начале случай, когда в каждом \emph{столбце} матрицы $A$ присутствует не менее трех единиц.
%Подразумевая использование гиперплоскостей вида $y_{lh}=0$,	
%Определим подграф $G'$ графа $G$ следующим образом.
Множество вершин $V$ графа $G$ будет состоять из трех подмножеств:
$U = \{u_1, u_2, \ldots, u_m\}$, $W = \{w_1, w_2, \ldots, w_m\}$ и $T$, где $|T| = 4m$.
Обозначения элементов множества $T$ поставим в зависимость от содержимого матрицы~$A = (a_{ij})$:
\[
T=\Set{t_{ij} \given a_{ij}=1, \ 1\le i \le m, \ 1\le j \le n}.
\]
(Поскольку каждая строка матрицы $A$ содержит ровно 4 единицы, то $|T| = 4m$.)


Все вершины $t_{ij}$ с одинаковым вторым индексом $j=\const$ соединяем ребрами в циклы (порядок соединения не важен).
Напомним, что, по предположению, в каждом столбце матрицы $A$ содержится не менее трех ненулевых элементов.
Всего образуется $n$ таких циклов.
Вершины множества $W$ тоже соединяем циклом.
Добавляем в граф $G'$ ребра $(w_i, u_i)$ и ребра $(u_i, t_{ij})$,
$1\le i \le m$, $1\le j \le n$.
	
Как уже было сказано ранее, построенный подграф $G'$ определяет некоторую опорную гиперплоскость к многограннику кубических подграфов.
Рассмотрим еще одну опорную гиперплоскость, положив степени вершин множества $W$ равными трем.
Пересечение обеих указанных гиперплоскостей с многогранником кубических подграфов и будет искомой гранью.
Каждая вершина этой грани представляет собой характеристический вектор некоторого кубического подграфа $H$, содержащего все вершины множества $W$ (и все три выходящих из каждой такой вершины ребра). Из описания графа $G'$ следует, что для каждого $i \in [m]$ подграф $H$ содержит ребро $(w_i, u_i)$ и по два ребра вида $(u_i, t_{ij})$. Каждой вершине $t_{ij}$ в $G'$ инцидентны ровно три ребра: ребро $(u_i, t_{ij})$ и два ребра, соединяющие эту вершину с её соседями по циклу (с одинаковым индексом $j$). Следовательно, если ребро $(u_i, t_{ij})$ входит в кубический подграф $H$, то в него также входят и все вершины $t_{kj}$ для любого допустимого $k$ (то есть весь $j$-й столбец из $T$). Таким образом, ребро $(u_i, t_{ij})$ входит в подграф $H$ тогда и только тогда, когда в него входят все вершины $t_{kj}$ $\forall k$ и все инцидентные им ребра $(u_k, t_{kj})$. То есть значения соответствующих этим ребрам координат $y_{u_k, t_{kj}}$ характеристического вектора $\bm{y}(H)\in\R^{k(k-1)/2}$ равны друг другу. Следовательно, все координаты этого вектора линейно зависят от $n$ (число столбцов в $T$) чисел $x_j$, $x_j = y_{u_k, t_{kj}}$. Причем для любого $i \in [m]$ из вершины $u_i$ выходит ровно два ребра вида $(u_i, t_{ij})$. Иными словами, $\sum_j a_{ij} x_j = 2$, что совпадает с условием, определяющим вершины многогранника $\DCP(A)$, $A = (a_{ij})\in \{0,1\}^{m\times n}$.

%Нетрудно проверить, что множество ее вершин аффинно эквивалентно множеству вершин многогранника соответствующей задачи о двойном покрытии.
	
Остается рассмотреть тот случай, когда (некоторые) столбцы матрицы $A$ содержат одну или две единицы.
Сложность здесь состоит лишь в том, чтобы соединить соответствующие одну или две вершины множества $T$ в цикл.
Чтобы сделать цикл <<с одной вершиной>>, добавим в $T$ четыре вспомогательные: 
$s_{j1}$, $s_{j2}$, $s_{j3}$, $s_{j4}$.
Для цикла <<с двумя вершинами>> добавим	в множество $T$ две: $s_{j1}$, $s_{j2}$.
Соединим их так, как показано на рис.~\ref{fig:Cubic}.
	
\begin{figure}[tbh]
	\centering
	\tikzset{small circle/.style={inner sep = 1.5pt,draw,circle}}
	\begin{tikzpicture}[scale=0.7,
	>={Stealth[scale width=0.8]} % Определяем вид стрелок
	]
	\node[small circle,densely dotted] (u) at (-2,0) {};	
	\node[small circle] (t) at (0,0) {};	
	\node[small circle] (s4) at (2,1) {};	
	\node[small circle] (s1) at (4,1) {};	
	\node[small circle] (s3) at (2,-1) {};	
	\node[small circle] (s2) at (4,-1) {};	
	\draw (t) node[above] {$t_{ij}$} -- (s4) node[above] {$s_{j4}$} -- (s1) node[above] {$s_{j1}$} -- (s2) node[below] {$s_{j2}$} -- (s3) node[below] {$s_{j3}$} -- (t);
	\draw (s4) -- (s2) (s1) -- (s3);
	\draw[dashed] (u) node[above] {$u_{i}$} -- (t);
	\draw (-4,1) node[above] {а)};
	\begin{scope}[xshift=14cm]
	\node[small circle,densely dotted] (u1) at (-2,1) {};	
	\node[small circle,densely dotted] (u2) at (-2,-1) {};	
	\node[small circle] (t1) at (0,1) {};	
	\node[small circle] (t2) at (0,-1) {};	
	\node[small circle] (s1) at (2,1) {};	
	\node[small circle] (s2) at (2,-1) {};	
	\draw (t1) node[above] {$t_{i_1 j}$} -- (s1) node[above] {$s_{j1}$} -- (s2) node[below] {$s_{j2}$} -- (t2) node[below] {$t_{i_2 j}$};
	\draw (t1) -- (s2) (s1) -- (t2);
	\draw[dashed] (u1) node[above] {$u_{i_1}$} -- (t1) (u2) node[above] {$u_{i_2}$} -- (t2);
	\draw (-4,1) node[above] {б)};
	\end{scope}
	\end{tikzpicture}
	\caption{Циклы с <<одной>> и <<двумя>> вершинами.}
	\label{fig:Cubic}
\end{figure}
	
Очевидно, что это не внесет существенных изменений в общую конструкцию, а число вершин увеличится на $4s+2t$, где $s$ "--- число столбцов матрицы $A$, содержащих лишь одну единицу, $t$ "--- число столбцов с двумя единицами.
\end{proof}


%%%%%%%%%%%%%%%%%%%%%%%%%%%%%%%%%%%%%%%%%%%%%%%%%%%%%%%
%
% Многогранники линейных порядков и деревьев Штейнера в графе
%
%%%%%%%%%%%%%%%%%%%%%%%%%%%%%%%%%%%%%%%%%%%%%%%%%%%%%%%

\section[Многогранники линейных порядков и деревьев Штейнера в графе]
{Многогранники линейных порядков \\и деревьев Штейнера в графе}
\label{sec:LOP&Steiner}

\subsection{Многогранники линейных порядков}
\label{sec:LOP}

Пусть $D = (V, A)$ "--- полный орграф, $V = [m]$.
%Предполагается, что орграф $D$ полный, то есть $(i,j) \in A$ и $(j,i) \in A$ для всех $i,j \in V$, $i \neq j$.
Подграф $D' = (V', A')$ графа $D$ называется \emph{транзитивным}, если из условий $(v, u) \in A'$ и $(u, w) \in A'$ следует $(v, w) \in A'$.
Так как мы не рассматриваем графы с петлями, то из транзитивности следует ацикличность.
Таким образом, множество дуг транзитивного графа задает частичный порядок на множестве вершин графа.
Если для каждой пары вершин $u,v \in V'$ в множество $A'$ входит ровно одна из двух дуг $(v, u)$ и $(u, v)$, то соответствующий орграф называется \emph{турниром}.
Транзитивный турнир (точнее, множество его дуг) в орграфе $D$ будем называть \emph{линейным порядком}. %(см. определение в разделе \ref{def:linearOrdering} на с.~\pageref{def:linearOrdering}).
Каждый линейный порядок $L$ в $D$ соответствует некоторой перестановке $\pi \from [n] \to [n]$, удовлетворяющей условию 
\begin{equation}
\label{eq:piLinear}
 \pi(i) < \pi(j) \iff (i,j) \in L.
\end{equation}

Координаты $y_{ij}$, $1 \le i < j \le m$ характеристического вектора $\bm{y} \in \R^{m(m-1)/2}$ для линейного порядка $L$ в $D$ определим следующим образом:
\[
y_{ij} = \begin{cases}
1 &\text{если $(i,j)\in L$,}\\
0 &\text{если $(j,i)\in L$.}
\end{cases}
\]
\label{def:LOP}
Обозначим через $\LOP(m)$ множество всех характеристических векторов линейных порядков в~$D$.
Выпуклая оболочка $\LOP(m)$ называется \emph{многогранником линейных порядков}. 
$\LOP(m)$ может быть также определен как множество 0/1"~векторов $\bm{y}\in\{0,1\}^{m(m-1)/2}$, удовлетворяющих 3-контурым неравенствам (см., например, \cite{Grotschel:1985}):
\begin{equation}
\label{3cycle}
0 \le y_{ij} + y_{jk} - y_{ik} \le 1, \quad i < j < k.
\end{equation}
Тема исследования этого многогранника довольно популярна и насчитывает нес\-коль\-ко десятков публикаций (см., например, \cite{Doignon:2009,Kovalev:2012}, а также ссылки в них).

\begin{prop}
Для каждого $m \in \N$, $m \ge 3$, существует матрица $B \in \{0,1\}^{r\times t}$, $r = \binom{m}{3} + \binom{m}{2}$, $t = \binom{m}{3} + m(m-1) + 2$,
имеющая ровно четыре единицы в каждой строке и такая, что $\LOP(m) \lea \DCP(B)$.
\end{prop}	
\begin{proof}
В~дополнение к переменным $y_{ij}$, $1 \le i < j \le m$, из описания многогранника $\LOP(m)$, введем две переменные $z$ и $h$. Для каждой $y_{ij}$ введем дополнительную переменную $\bar{y}_{ij}$ и уравнение
\begin{equation}\label{eq:lopDCP1}
y_{ij} + \bar{y}_{ij} + z + h = 2.
\end{equation}
А каждое 3-контурное неравенство \eqref{3cycle} заменим уравнением
\begin{equation}\label{eq:lopDCP2}
y_{ij} + y_{jk} + \bar{y}_{ik} + t_{ijk} = 2,
\end{equation}
где $t_{ijk}$~--- еще одна (для каждого 3-контурного неравенства) дополнительная переменная.
С одной стороны, система из $\binom{m}{2} + \binom{m}{3}$ уравнений~\eqref{eq:lopDCP1} и~\eqref{eq:lopDCP2} определяет некоторый многогранник двойных покрытий.
С другой стороны, в пересечении опорных гиперплоскостей $z=0$ и $h=1$
находится грань этого многогранника, аффинно эквивалентная $\LOP(m)$.
\end{proof}

В~этой связи заметим, что многогранники двойных покрытий (во всяком случае, некоторые из них) едва ли могут быть гранями $\LOP(m)$, так как критерий смежности вершин последних полиномиален~\cite{Young:1978}.

Покажем, что $\LOP(2n)$ содержит в качестве грани булев квадратичный многогранник $\BQP(n)$.

\begin{theorem}[\cite{Maksimenko:2017LOP}]
$\BQP(n) \lea \LOP(2n)$, $n\in\N$.
\end{theorem}
\begin{proof}
Пусть $\bm{y} = (y_{ij}) \in \LOP(2n)$.
Воспользуемся тем, что неравенства $y_{ij} \ge 0$, при $1 \le i < j \le 2n$, 
и 3-контурные неравенства \eqref{3cycle} выполнены для всех $\bm{y} \in \LOP(2n)$,
а соответствующие равенства определяют (некоторые) опорные гиперплоскости для многогранника $\LOP(2n)$.

Покажем, что многогранник $\BQP(n)$ аффинно эквивалентен грани $F$ многогранника $\LOP(2n)$, задаваемой следующими ограничениями:
\begin{align}
y_{2i, 2j-1} &= 0, \label{eq:LOP1}\\
y_{2i-1, 2i}   + y_{2i, 2j}   - y_{2i-1, 2j} &= 0, \label{eq:LOP2}\\
y_{2i-1, 2j-1} + y_{2j-1, 2j} - y_{2i-1, 2j} &= 0, \label{eq:LOP3}
\end{align}
для всех $1 \le i < j \le n$.

Из \eqref{eq:LOP2} и \eqref{eq:LOP3} выводим
\begin{align}
y_{2i-1, 2j}   &= y_{2i-1, 2i} + y_{2i, 2j}, \label{eq:LOP4}\\
y_{2i-1, 2j-1} &= y_{2i-1, 2i} + y_{2i, 2j} - y_{2j-1, 2j}. \label{eq:LOP5}
\end{align}
Таким образом, все координаты вектора $\bm{y} \in F$ линейно зависят от координат $y_{2i-1, 2i}$, $i\in[n]$, и $y_{2i, 2j}$, $1 \le i < j \le n$.

Покажем, что значения координаты $y_{2i, 2j}$ однозначно определяются значениями координат $y_{2i-1, 2i}$ и $y_{2j-1, 2j}$.
Из \eqref{eq:LOP4} и $y_{2i-1, 2j} \le 1$ следует \(y_{2i, 2j} \le 1 - y_{2i-1, 2i}\), иными словами,
\[
y_{2i-1, 2i} = 1 \Rightarrow y_{2i, 2j} = 0.
\]
Из 3-контурного неравенства $0 \le y_{2i, 2j-1} + y_{2j-1, 2j} - y_{2i, 2j}$ и уравнения \eqref{eq:LOP1} следует \(y_{2i, 2j} \le y_{2j-1, 2j}\),
то есть
\[
y_{2j-1, 2j} = 0 \Rightarrow y_{2i, 2j} = 0.
\]
А из \eqref{eq:LOP5} и $y_{2i-1, 2j-1} \ge 0$ следует \(y_{2i, 2j} \ge y_{2j-1, 2j} - y_{2i-1, 2i}\), то есть 
\[
y_{2i, 2j} = 1, \quad \text{если $y_{2i-1, 2i} = 0$  и $y_{2j-1, 2j} = 1$}.
\]
Таким образом, учитывая, что вектор $\bm{y}\in F$ является 0/1-вектором,
\begin{equation}
\label{eq:LOPbqp}
y_{2i, 2j} = y_{2j-1, 2j} (1 - y_{2i-1, 2i}).
\end{equation}

Итак, все вершины грани $F$ должны быть 0/1-векторами, удовлетворяющими соотношению \eqref{eq:LOPbqp}, и все координаты этих векторов линейно зависят от $y_{2i-1, 2i}$, $i\in[n]$, и $y_{2i, 2j}$, $1 \le i < j \le n$ (см. уравнения \eqref{eq:LOP1}, \eqref{eq:LOP4} и \eqref{eq:LOP5}).
Покажем теперь, что каждому набору значений переменных $y_{2i-1, 2i}$, $i\in[n]$,
на самом деле соответствует некоторая вершина грани $F$.

Пусть 
\[
I_0 = \Set*{i \in [n] \given y_{2i-1, 2i} = 0}, \quad I_1 = \Set*{i \in [n] \given y_{2i-1, 2i} = 1}.
\]
Далее предполагаем, что элементы множеств $I_0 = \{i_1, \dots, i_k\}$ и $I_1 = \{i'_1, \dots, i'_{n-k}\}$ отсортированы \emph{по убыванию}.
Линейный порядок для соответствующей вершины $\bm{y} \in F$ представим перестановкой $\pi \from [n] \to [n]$ (см. условие \eqref{eq:piLinear}).
Положим
\begin{align*}
\pi(2i_s-1) &= n-k + 2 s, &
\pi(2i_s) &= \pi(2i_s-1)-1, & s &\in [k],\\
\pi(2i'_t-1) &= t, &
\pi(2i'_t) &= n + k + t, & t &\in [n-k].
\end{align*}
Так, например, в случае $n=3$ вершины грани $F \subset \LOP(6)$ соответствуют восьми перестановкам (записанным в виде $\pi^{-1}(1)\ldots\pi^{-1}(6)$, то есть если цифра $i$ располагается в этой последовательности левее $j$, то $y_{ij} = 1$)
\begin{alignat*}{4}
&654321, \qquad & k &=3 , \quad & I_0 &= \{3,2,1\}, \quad& I_1 &= \emptyset,\\
&165432, & k &=2 , \quad & I_0 &= \{3,2\},   & I_1 &= \{1\},\\
&365214, & k &=2 , \quad & I_0 &= \{3,1\},   & I_1 &= \{2\},\\
&543216, & k &=2 , \quad & I_0 &= \{2,1\},   & I_1 &= \{3\},\\
&316542, & k &=1 , \quad & I_0 &= \{3\},     & I_1 &= \{2,1\},\\
&514362, & k &=1 , \quad & I_0 &= \{2\},     & I_1 &= \{3,1\},\\
&532164, & k &=1 , \quad & I_0 &= \{1\},     & I_1 &= \{3,2\},\\
&531642, & k &=0 , \quad & I_0 &= \emptyset, & I_1 &= \{3,2,1\}.
%&654321, \quad &&165432, \quad &&541632, \quad &&543216, \\
%&261543, &&264315, &&432615, &&362514.
\end{alignat*}

Из описания перестановки $\pi$ следует, что 
$y_{2i_s - 1, 2i_s} = 0$,  при $s \in [k]$, а
$y_{2i'_t - 1, 2i'_t} = 1$, при $t \in [n-k]$.
Справедливость соотношений \eqref{eq:LOP1}--\eqref{eq:LOP3} проверяется перебором четырех случаев, в зависимости от принадлежности индексов $i,j$ множествам $I_0$, $I_1$.

Завершая доказательство, установим между $\bm{x} = (x_{ij}) \in \BQP(n)$ и $\bm{y} \in F \subset \LOP(2n)$ взаимно"=однозначное соответствие:
\begin{align*}
x_{ii} &= y_{2i-1, 2i}, & i &\in [n],\\
x_{ij} &= y_{2j-1, 2j} - y_{2i, 2j}, & 1 &\le i < j \le [n].
\end{align*}
\end{proof}

В частности, из этой теоремы и сверхполиномиальной сложности расширения для $\BQP(n)$~\cite{FioriniPokutta:2015} следует, что сложность расширения $\LOP(n)$ равна $2^{\Omega(n)}$. Недавно в~\cite{Davis:2018} для $\LOP(n)$ была описана расширенная формулировка размера $2^{\Theta(n)}$, откуда следует, что данная оценка является асимптотически оптимальной.



\subsection{Многогранники деревьев Штейнера}
\label{sec:SteinerTree}

Пусть $G=(V,E)$ "--- реберно"=взвешенный граф, а $T$ "--- некоторое подмножество его вершин. \emph{Деревом Штейнера на $T$} называется подграф $G' = (V',E')$, $T\subseteq V'$, графа $G$, являющийся деревом. Задача Штейнера (в графе) заключается в отыскании дерева Штейнера с минимальным суммарным весом ребер.
\emph{Многогранником деревьев Штейнера} называется выпуклая оболочка множества $\Steiner(G,T) \subseteq \{0,1\}^E$ всех характеристических векторов деревьев Штейнера на $T$~\cite{Chopra:1994}.

\begin{theorem}
	Для любого графа $G=(V,E)$ за линейное время можно построить граф $G'=(V',E')$, $|V'|=|V|+2|E|+1$, $|E'|=4|E|+|V|$, и указать множество $T\subset V'$ такие, что
	$\Stable(G) \lea \Steiner(G',T)$.
\end{theorem}
\begin{proof}
	Пусть $G=([n],E)$ "--- некоторый граф, определяющий многогранник $\Stable(G)$.
	Положим 
	\[
	T = \{v'_0\} \cup \Set*{t_{ij} \given i < j, \ \{i,j\}\in E}
	\]
	и определим множество вершин и ребер графа $G'=(V',E')$ следующим образом:
	\[
	V' = T \cup \{v'_1,\dots,v'_n\} \cup \Set{v'_{ij} \given i < j, \ \{i,j\}\in E},
	\]
	\[
	E' = \Set{\{v'_0, v'\} \given v'\in V'\setminus T} 
	\cup \Set{\{v'_{ij}, t_{ij}\}}
	\cup \Set{\{v'_{k}, t_{ij}\} \given k\in \{i,j\}}.
	\]
	Таким образом, каждая вершина $t_{ij}$ инцидентна ровно трем ребрам 
	\[
	\{v', t_{ij}\}, \quad v' \in V_{ij} = \{v'_{ij}, v'_i, v'_j\}.
	\]
	Причем как минимум одно из них должно входить в дерево Штейнера на $T$.
	Следовательно, неравенство
	\[
	x_{\{v'_{ij}, t_{ij}\}} + x_{\{v'_{i}, t_{ij}\}} + x_{\{v'_{j}, t_{ij}\}} \ge 1
	\]
	выполнено для каждого $\bm{x} \in \Steiner(G',T) \subset \{0,1\}^{E'}$.
	Полагая 
	\begin{equation}
	\label{eq:proofSteiner1}
	x_{\{v'_{ij}, t_{ij}\}} + x_{\{v'_{i}, t_{ij}\}} + x_{\{v'_{j}, t_{ij}\}} = 1,
	\end{equation}
	перейдем к рассмотрению соответствующей грани многогранника $\Steiner(G',T)$.
	
	Предположим, что некоторое ребро $\{v', t_{ij}\}$, $v' \in V_{ij}$ входит в дерево Штейнера на $T$. 
	Тогда, в силу того, что $t_{ij}$ и $v'_0$ должны быть связаны, а степени всех вершин из $T \setminus \{v'_0\}$ равны единице, приходим к выводу, что ребро $\{v', v'_0\}$ обязано присутствовать в этом дереве.
	Другими словами, $x_{\{v'_0, v'\}} \ge x_{\{v', t_{ij}\}}$, $v' \in V_{ij}$.
	Очевидно, ограничения 
	\[
	x_{\{v'_0, v'\}} = x_{\{v', t_{ij}\}}, \quad v' \in V_{ij},
	\]
	вместе с \eqref{eq:proofSteiner1} определяют некоторую грань $F$ многогранника $\Steiner(G',T)$.
	Таким образом, все координаты вектора $\bm{x}$, принадлежащего множеству вершин этой грани, выражаются линейно через координаты $x_{\{v'_0, v'\}}$, $v' \in V' \setminus T$.
	Кроме того, для них выполняются ограничения
	\begin{equation*}
	x_{\{v'_0, v'_{ij}\}} + x_{\{v'_0, v'_{i}\}} + x_{\{v'_0, v'_{j}\}} = 1.
	\end{equation*}
	То есть координаты $x_{\{v'_0, v'_{ij}\}}$ линейно зависят от $x_{\{v'_0, v'_{i}\}}$ и $x_{\{v'_0, v'_{j}\}}$. Положим $y_i = x_{\{v'_0, v'_{i}\}}$, $i \in [n]$ для $\bm{y} \in \Stable(G)$. Тогда последнее уравнение (при условии, что все переменные принимают значения из $\{0,1\}$) эквивалентно ограничению
	\begin{equation*}
	y_i + y_j \le 1, \quad \{i,j\} \in E.
	\end{equation*}
	Следовательно, множество вершин рассматриваемой грани $F$ аффинно эквивалентно некоторому подмножеству множества вершин $\Stable(G)$. С другой стороны, для любого набора переменных $y_i$, $i \in [n]$, удовлетворяющего неравенствам $y_i + y_j \le 1$, $\{i,j\} \in E$, однозначно строится соответствующее дерево Штейнера в $G'$. Таким образом, грань $F$ аффинно эквивалентна $\Stable(G)$.
\end{proof}



%%%%%%%%%%%%%%%%%%%%%%%%%%%%%%%%%%%%%%%%%%%%%%%%%%%%%%%
%
%  Многогранники c простым критерием смежности вершин
%
%%%%%%%%%%%%%%%%%%%%%%%%%%%%%%%%%%%%%%%%%%%%%%%%%%%%%%%

\section{Многогранники c простым критерием смежности вершин}
\label{sec:PolytopesSimpleAdj}

\subsection{Многогранник трехиндексной задачи о назначениях}
\label{sec:3Ass}

Пусть $S$ "--- конечное множество.
Координаты вектора $\bm{x}\in\R^{S\times S\times S}$
будем обозначать $x(s, t, u)$, где $s,t,u\in S$.
\emph{Трехиндексная аксиальная задача о назначениях} (или \emph{3-сочетаниях}) может быть сформулирована как следующая задача 0/1"~программирования:
\[
\sum_{s\in S} \sum_{t\in S} \sum_{u\in S} c(s,t,u) \cdot x(s,t,u) 
\rightarrow \max,
\] 
\begin{align}
&\sum_{s\in S} \sum_{t\in S} x(s,t,u) = 1 \quad \forall u\in S,\label{eq:ThreeBeg}\\
&\sum_{s\in S} \sum_{u\in S} x(s,t,u) = 1 \quad \forall t\in S,\label{eq:Three2}\\
&\sum_{t\in S} \sum_{u\in S} x(s,t,u) = 1 \quad \forall s\in S,\label{eq:Three3}\\
&x(s,t,u) \in \{0, 1\} \quad \forall s, t, u\in S,\label{eq:ThreeEnd}
\end{align}
где $c(s,t,u)\in\Z$ "--- координаты целевого вектора.
Через $\TAP(m)$, $m = |S|$, обозначим множество всех векторов $\bm{x}\in\R^{S\times S\times S}$, удовлетворяющих ограничениям~\eqref{eq:ThreeBeg}--\eqref{eq:ThreeEnd}.
Выпуклая оболочка множества $\TAP(m)$ называется \emph{многогранником трехиндексной аксиальной задачи о назначениях}.
Далее, в целях экономии места, мы будем опускать слово <<аксиальной>>.

По-видимому, первыми работами, посвященными изучению свойств этого многогранника, являются~\cite{Euler:1987} и~\cite{Balas:1989}.
Более свежий обзор имеется в~\cite{Qi:2000}.
В~работах Кравцова (см.~\cite{Kravtsov:2006} и ссылки в ней) изучаются свойства нецелочисленных вершин релаксаций этого многогранника.

Очевидно, $\TAP(m)$ является частным случаем $\Part(A)$:
\begin{equation}
\label{Ine3AP-PART}
\TAP(m) =_A \Part(A),
\end{equation}
где $A$ "--- $(3m\times m^3)$-матрица, образованная коэффициентами левых частей уравнений \eqref{eq:ThreeBeg}--\eqref{eq:Three3}.
Таким образом, семейство многогранников трехиндексной задачи о назначениях аффинно сводится к многогранникам разбиений: 
\[
\TAP \propto_A \Part.
\]

Пользуясь стандартным алгоритмом свед\'{е}ния задачи 3"~выполнимость к задаче 3-сочетаний~\cite{Garey:1982}, Авис и Тивари~\cite{AvisTiwary:2015} показали, что многогранник задачи 3"~выполнимость $\SAT(U,C)$, $|c|=3$, $c\in C$, является проекцией грани многогранника $\TAP(m)$, где $m = O(kn)$, $k=|U|$, $n=|C|$.
Однако, из \eqref{Ine3AP-PART} и теоремы~\ref{thm:Class1} об эквивалентности $\Stable$ и $\Part$ следует, что $\TAP \propto_A \Stable$.
С другой стороны, так как семейство $\DCP$ аффинно сводится к многогранникам задачи 3"~выполнимость (теорема~\ref{thm:DCPSAT}), а $\Stable$ не может быть сведено к $\DCP$ (теорема~\ref{thm:DCPStable}), то аффинное свед\'{е}ние многогранников задачи 3"~выполнимость к $\TAP$ невозможно.

Покажем, что $\Stable \propto_A \TAP$.
Таким образом, $\TAP$ окажется в одном классе эквивалентности 
%(в смысле $\propto_A$) 
вместе с $\Stable$, $\Part$ и $\Pack$.

Для графа $G(V, E)$ в определении многогранника $\Stable(G)$ через
\[
V_{\text{isol}} = \Set{v\in V \given v\notin e \  \forall e\in E},
\]
обозначим множество \emph{изолированных} вершин.


\begin{theorem}[\cite{Maksimenko:2016bool}]
	\label{TheSSP-3AP}
	$\Stable(G) \le \TAP(m)$ при $m = 3|E| + 2|V_{\text{isol}}|$.
\end{theorem}

\begin{proof}
	Множество $S$ в определении~$\TAP(m)$ составим из трех типов элементов:
	\begin{enumerate}
		\item $v$ и $\bar{v}$ для каждой изолированной вершины $v\in V_{\text{isol}}$.
		
		\item $e$ для каждого ребра $e\in E$.
		
		\item $(e, v)$ для каждого $e\in E$ и $v \in e$.
	\end{enumerate}
	Построим множество троек $Q \subset S\times S\times S$ так, чтобы грань
	\[
	F = \big\{x \in \TAP(m) \mid x(q) = 0 \ \forall q \notin Q\big\}
	\] 
	многогранника $\TAP(m)$ была аффинно эквивалентна многограннику $\Stable(G)$.
	
	Для каждого $v\in V_{\text{isol}}$ множество $Q$ будет содержать четыре тройки:
	\[
	(v, v, v), \quad (\bar{v}, \bar{v}, \bar{v}), \quad
	(v, v, \bar{v}), \quad (\bar{v}, \bar{v}, v).
	\] 
	Других троек, содержащих $v$ или $\bar{v}$ в $Q$ нет.
	Следовательно, если $\bm{x} \in F$, то для каждого $v\in V_{\text{isol}}$ возможны только две альтернативы: 
	\[
	x(v, v, v) = x(\bar{v}, \bar{v}, \bar{v}) = 1
	\quad \text{или} \quad 
	x(v, v, \bar{v}) = x(\bar{v}, \bar{v}, v) = 1.
	\] 
	
	Рассмотрим теперь элементы $e$ и $(e, v)$ множества $S$,
	где $e\in E$ и $v \in e$.
	Для каждой неизолированной вершины $v \in V \setminus V_{\text{isol}}$ рассмотрим множество инцидентных ей ребер $E(v) = \{e_{i_1}, \ldots, e_{i_p}\}$, где $p=d_G(v)$ "--- степень вершины $v$.
	Множество троек $Q$ содержит
	\begin{enumerate} 
		\item $\big( e, e, e \big)$ для каждого $e\in E$.
		
		\item $\big( (e,v), (e,v), (e,v) \big)$ для всех $e\in E$ и $v \in e$.
		
		\item $\big( e, e, (e,v) \big)$ для всех $e\in E$ и $v \in e$.
		
		\item $\big( (e_{i_q},v), (e_{i_{q+1}},v), e_{i_q} \big)$
		для каждой неизолированной вершины $v$ и каждого $e_{i_q} \in E(v)$,
		где сложение $q+1$ выполняется по модулю $p$.
	\end{enumerate} 
	
	Полагая $x(u, v, w) = 0$ для всех остальных троек $(u, v, w)\notin Q$,
	перейдем к рассмотрению соответствующей грани многогранника $\TAP(m)$.
	Перечислим некоторые свойства её вершин.
	
	Заметим, что для каждого $(e,v) \in S$ множество $Q$ содержит в точности две тройки с элементом $(e,v)$ в третьей компоненте: $\big( (e,v), (e,v), (e,v) \big)$ и $\big( e, e, (e,v) \big)$.
	Следовательно, уравнение~\eqref{eq:ThreeBeg} для $u = (e,v)$ принимает вид
	\begin{equation}
	\label{eq:Proof3AP1}
	x\big( (e,v), (e,v), (e,v) \big) + x\big( e, e, (e,v) \big) = 1.
	\end{equation}
	То есть, $x\big( (e,v), (e,v), (e,v) \big)$ выражается линейно через $x\big( e, e, (e,v) \big)$.
	
	Заметим также, что для каждого $e \in S$ множество $Q$ содержит ровно три тройки с элементом $e$ в первой компоненте: $\big( e, e, e \big)$, $\big( e, e, (e,v_1) \big)$ и $\big( e, e, (e,v_2) \big)$, где $e = \{v_1, v_2\}$.
	Следовательно, уравнение~\eqref{eq:Three3} при $s = e$ эквивалентно
	$$
	x\big( e, e, e \big) + x\big( e, e, (e,v_1) \big) + x\big( e, e, (e,v_2) \big) = 1.
	$$
	То есть
	$x\big( e, e, e \big) = 1 - x\big( e, e, (e,v_1) \big) - x\big( e, e, (e,v_2) \big)$
	и 
	\begin{equation}
	\label{IneProof3AP}
	x\big( e, e, (e,v_1) \big) + x\big( e, e, (e,v_2) \big) \le 1.
	\end{equation}
	
	Рассуждая по аналогии, получаем следующее уравнение
	$$
	x\big( (e_{i_q},v), (e_{i_q},v), (e_{i_q},v) \big) 
	+ x\big( (e_{i_q},v), (e_{i_{q+1}},v), e_{i_q} \big) = 1
	$$ 
	для каждой неизолированной $v$ и каждого $e_{i_q} \in E(v)$,
	где сложение $q+1$ выполняется по модулю $p=d_G(v)$.
	Вместе с уравнением \eqref{eq:Proof3AP1} получаем
	\begin{equation}
	\label{eq:Proof3AP2}
	x\big( (e_{i_q},v), (e_{i_{q+1}},v), e_{i_q} \big) = 
	1 - x\big( (e_{i_q},v), (e_{i_q},v), (e_{i_q},v) \big) =
	x\big( e_{i_q}, e_{i_q}, (e_{i_q},v) \big).
	\end{equation}
	Кроме того, так как
	$$
	x\big( (e_{i_{q+1}},v), (e_{i_{q+1}},v), (e_{i_{q+1}},v) \big) 
	+ x\big( (e_{i_q},v), (e_{i_{q+1}},v), e_{i_q} \big) = 1,
	$$ 
	то
	\begin{equation}
	\label{eq:Proof3AP3}
	x\big( (e_{i_q},v), (e_{i_{q+1}},v), e_{i_q} \big) = 
	1 - x\big( (e_{i_{q+1}},v), (e_{i_{q+1}},v), (e_{i_{q+1}},v) \big) =
	x\big( e_{i_{q+1}}, e_{i_{q+1}}, (e_{i_{q+1}},v) \big).
	\end{equation}
	Так как левые части формул \eqref{eq:Proof3AP2} и \eqref{eq:Proof3AP3}
	совпадают, то
	$$
	x\big( e_{i_{q+1}}, e_{i_{q+1}}, (e_{i_{q+1}},v) \big) 
	= x\big( e_{i_q}, e_{i_q}, (e_{i_q},v) \big).
	$$ 
	Значит $x\big( e, e, (e,v) \big) = x\big( e', e', (e',v) \big)$
	для любых двух ребер $e$ и $e'$, $v \in e$, $v \in e'$.
	
	Таким образом, для вершин описанной грани $F$ все переменные $x(s,t,u)$
	выражаются линейно через $x\big( e, e, (e,v) \big)$,
	а неравенство~\eqref{IneProof3AP} является неравенством вида $y_i + y_j \le 1$ из описания многогранника $\Stable(G)$.
\end{proof}


\begin{remark}
	Полученные результаты могут быть легко обобщены на случай $p$-индексной задачи о назначениях ($p > 3$).
	По аналогии, вершины $\PAP{p}(m)$ многогранника $p$-индексной аксиальной задачи о назначениях являются 0/1-векторами в $\R^{m^p}$.
	Координаты $x_{i_1 i_2 \ldots i_p}$, 
	$i_1, i_2, \ldots, i_p \in \{1,2,\ldots,p\}$, каждого такого вектора удовлетворяют следующим уравнениям:
	$$
	%\begin{equation}
	\begin{aligned}
	\sum_{i_2, i_3, \ldots, i_p} x_{i_1 i_2 \ldots i_p}      &= 1 \quad \forall i_1\in \{1,\ldots,p\},\\
	\sum_{i_1, i_3, i_4, \ldots, i_p} x_{i_1 i_2 \ldots i_p} &= 1 \quad \forall i_2\in \{1,\ldots,p\},\\
	\ldots\ldots & \ldots\\
	\sum_{i_1, i_2, \ldots, i_{p-1}} x_{i_1 i_2 \ldots i_p}  &= 1 \quad \forall i_p\in \{1,\ldots,p\}.\\
	\end{aligned}
	$$
	%\end{equation}
	Очевидно,
	$$
	\PAP{p}(m) \lea \Part(A), 
	$$
	для некоторой (просто определяемой) матрицы $A\in \{0,1\}^{pm\times m^p}$.
	С другой стороны, уравнения
	$$
	x_{i_1 i_2 \ldots i_p} = 0 \quad \forall i_p \ne i_{p-1}
	$$
	определяют грань многогранника $\PAP{p}(m)$, аффинно эквивалентную $\PAP{(p-1)}(m)$.
	Таким образом, по теореме~\ref{TheSSP-3AP},
	\[
	\Stable(G) \lea \PAP{p}(m) \quad \text{при } m = 3|E| + 2|V_{\text{isol}}| \le 3|V|(|V|-1)/2.
	\]
\end{remark}


\subsection{Многогранники квадратичной задачи линейных упорядочиваний
	 и~квадратичной задачи о назначениях}
\label{sec:QLOP}

%Более точно, задача о квадратичном упорядочивании 
% называется <<квадратичной задачей линейного упорядочивания>> \cite{Buchheim:2010}.
%То есть, по-существу, является усложнением обычной задачи линейного упорядочивания.

Выше, в разделе \ref{sec:LOP}, был определен многогранник линейных порядков $\LOP(m)$. Многогранник квадратичной задачи линейного упорядочивания является в~некотором смысле его усложнением.

Пусть
\[
I = \big\{(i,j,k,l) \ | \ i<j, \ k<l, \text{ и } (i,j) \prec (k,l) \big\},
\]
где $(i,j) \prec (k,l)$ означает либо $i<k$, либо $i=k$ и $j < l$.
Для каждой пары различных переменных $y_{ij}$ и $y_{kl}$ вводится новая переменная
\begin{equation}
\label{eq:QLOP}
z_{ijkl} = y_{ij} y_{kl}, \quad (i,j,k,l) \in I.
\end{equation}
Обозначим через $\QLOP(m)$ множество векторов $\bm{z}\in\{0,1\}^d$, 
$d = \binom{m}{2} \left( \binom{m}{2} + 1 \right) / 2$,
с координатами $y_{ij}$ и $z_{ijkl}$, удовлетворяющих ограничениям \eqref{3cycle} и~\eqref{eq:QLOP}.
Выпуклая оболочка множества $\QLOP(m)$ называется \emph{многогранником квадратичной задачи линейного упорядочивания}~\cite{Buchheim:2010}.

\begin{theorem}[Бучхейм, Вигеле и Женг~\cite{Buchheim:2010}] 
	$\QLOP(m) \lea \BQP(n)$ при $n = \binom{m}{2}$.
\end{theorem} 

Бучхейм, Вигеле и Женг~\cite{Buchheim:2010} используют этот факт в алгоритме ветвей и отсечений для решения квадратичной задачи линейного упорядочивания.

Покажем, что аффинное свед\'{е}ние в обратную сторону также возможно.

\begin{theorem}[\cite{Maksimenko:2016bool}]
	\label{ThBQP-QLOP}
	$\BQP(n) \lea \QLOP(m)$ при $m = 2n$.
\end{theorem} 

\begin{proof}
	Идея доказательства проста.
	Покажем, что среди граней $\LOP(m)$ есть $n$"~мерный куб.
	При переходе от $\LOP(m)$ к $\QLOP(m)$ этот куб преобразуется в булев квадратичный многогранник.
	
	%Пусть $m = 2n$. 
	Воспользуемся тем, что равенства $y_{ij} = 0$ и $y_{ij} = 1$ определяют опорные гиперплоскости для $\LOP(m)$ и $\QLOP(m)$.
	Пусть
	\[
	J = \{(2i-1,2i) \mid i\in[n] \}.
	\] 
	Положим 
	\begin{equation}
	\label{FaceQube}
	y_{ij} = 0 \quad \text{для всех } (i,j)\notin J, \ 1 \le i < j \le m.
	\end{equation}
	Незафиксированными остаются только переменные $y_{ij}$, где $i$ нечетно и $j = i+1$.
	Обратим внимание на 3-контурные неравенства~\eqref{3cycle}.
	Предполагая $i < j < k$, имеем два случая:
	\begin{enumerate} 
		\item Если $(i,j)\notin J$, то $y_{ij} = y_{ik} = 0$.
		Тогда неравенство~\eqref{3cycle} преобразуется в $0 \le y_{jk} \le 1$. 
		\item Если $(i,j)\in J$, то $i$ нечетно, $j = i+1$ четно и $k > i+1$.
		Следовательно, неравенство~\eqref{3cycle} эквивалентно $0 \le y_{ij} \le 1$.
	\end{enumerate} 
	Таким образом, $n$ переменных $y_{i\, i+1}$, где $i$ четно, 
	могут принимать значения 0 или 1 независимо друг от друга.
	Следовательно, гиперплоскости~\eqref{FaceQube} определяют грань $\LOP(m)$, являющуюся $n$"~мерным кубом. 
	
	Обратим внимание на переменные $z_{ijkl}$, $(i,j,k,l) \in I$.
	Если $(i,j) \notin J$ или $(k,l) \notin J$, то $z_{ijkl}=0$.
	В~случае $(i,j) \in J$ и $(k,l) \in J$ имеем $z_{ijkl}=y_{ij} y_{kl}$, 
	и, кроме того, $y_{ij}$ и $y_{kl}$ не связаны никакими другими зависимостями.
	
	Таким образом, описанная грань $\QLOP(m)$ и многогранник $\BQP(n)$ связаны следующим аффинным невырожденным отображением:
	$$
	\begin{aligned}
	x_{ii} &= y_{2i-1,\, 2i}, \quad 1\le i\le n,\\
	x_{ij} &= z_{2i-1,\, 2i,\, 2j-1,\, 2j} = y_{2i-1,\, 2i} \cdot y_{2j-1,\, 2j}, \quad 1\le i < j\le n,
	\end{aligned}
	$$
	где $\bm{y} \in \QLOP(m)$, $\bm{x} \in \BQP(n)$.
	%Let $x_{ii}$ in the~equation~\eqref{BQP} be equal to $y_{2i-1, 2i}$, $1\le i\le n$.
	%Let $x_{ij} = z_{2i-1, 2i, 2j-1, 2j}=y_{2i-1, 2i} \cdot y_{2j-1, 2j}$, $1\le i < j\le n$.
	%Thus the~face of $\QLOP(m)$ is affinely equivalent to $\BQP_n$. 
\end{proof}

Для многогранника квадратичной задачи о назначениях история повторяется почти дословно.

Множество вершин $\Birk(m)$ многогранника задачи о назначениях (многогранника Биркгофа)
состоит из векторов $\bm{y}\in\{0,1\}^{m\times m}$, удовлетворяющих условиям
\begin{align}
\sum_j y_{ij} = 1, \ \forall i \in [m], \label{eq:Ass1}\\ 
\sum_i y_{ij} = 1, \ \forall j \in [m]. \label{eq:Ass2}
\end{align}
Чтобы определить многогранник квадратичной задачи о назначениях,
введем новые переменные $z_{ijkl}$ так же, как это было сделано в~\eqref{eq:QLOP}:
%For every pair $y_{ij}$ and $y_{kl}$ there is introduced a~new variable
\begin{equation}
\label{eq:QAP}
z_{ijkl} = y_{ij} y_{kl}, \text{ где } (i,j) \prec (k,l).
\end{equation}
Обозначим через $\QAP(m)$ множество всех векторов $\bm{z}\in\{0,1\}^d$, $d = m^2 + \binom{m^2}{2}$,
с координатами $y_{ij}$ и $z_{ijkl}$, удовлетворяющих условиям \eqref{eq:Ass1}, \eqref{eq:Ass2} и~\eqref{eq:QAP}.
Выпуклая оболочка множества $\QAP(m)$ называется \emph{многогранником квадратичной задачи о назначениях}.
В~литературе также встречается описание \emph{многогранника квадратичных полуназначений}~\cite{Saito:2009}.
При определении его множества вершин $\QSAP(m)$ условие~\eqref{eq:Ass2} опускается.

\begin{theorem}[\cite{Rijal:1995, Kaibel:1997, Saito:2009}] 
	\label{thm:QAP2BQP}
	$\QAP(m) \lea \QSAP(m) \lea \BQP(n)$ при $n = m^2$.
\end{theorem} 

Эта связь используется в~\cite{Kaibel:1997} для вывода неравенств, допустимых для $\QAP(m)$.
В~частности, так как $\BQP(n)$ 2"~смежностен, то $\QAP(m)$ тоже 2"~смежностен.
В~\cite{Kaibel:1997} также показано, что многогранник линейных порядков $\LOP(m)$ и многогранник коммивояжера $\TSP(m)$ являются проекциями некоторых граней многогранника $\QAP(m)$.
%\[\LOP(m) \leE \QAP(m),  \quad  \TSP(m) \leE \QAP(m).\] 
Заметим, что аффинные сводимости $\LOP \propto_A \QAP$ и $\TSP \propto_A \QAP$ невозможны, так как $\LOP(m)$ не 2"~смежностен при $m \ge 3$~\cite{Young:1978},
а $\TSP(m)$ не 2"~смежностен при $m \ge 6$~\cite{PadbergRao:1974}.

\begin{theorem}[\cite{Maksimenko:2016bool}]
	$\BQP(n) \lea \QAP(m)$ для $m = 2n$.
\end{theorem} 

\begin{proof}
По аналогии с доказательством теоремы~\ref{ThBQP-QLOP}, достаточно заметить, что многогранник Биркгофа $\Birk(2n)$ содержит грань, являющуюся $n$"~мерным кубом.
(Здесь мы используем тот же факт, что и при доказательстве теоремы~\ref{thm:KnapEqCAP}.)
Положим
\[
S = \big\{(i,i) \mid i\in[2n]\big\} \cup \big\{(2i-1,2i) \mid i\in[n]\big\} \cup \big\{(2i,2i-1) \mid i\in[n]\big\}.
\] 
Тогда уравнения
\[
y_{ij} = 0 \quad \forall (i,j) \notin S
\]
определяют требуемую грань.
\end{proof}

Таким образом, семейства многогранников $\QLOP$, $\QAP$ и $\QSAP$ находятся в одном классе эквивалентности (в смысле $\propto_A$) вместе с $\BQP$ и $\Cut$.


%С задачей об изоморфизме подграфу также связан многогранник Юнга $\phi_n$, определяемый следующим образом~\cite{Onn:2009}. Для каждой перестановки вершин полного графа $K_n$ рассмотрим соответствующую $\binom{n}{2}\times\binom{n}{2}$-матрицу перестановок ребер этого графа. Через $\phi_n$ обозначим множество всех таких матриц при фиксированном $n\in\N$, $n\ge 3$. Выпуклая оболочка множества $\phi_n$ является частным примером многогранника Юнга и в~\cite{Onn:1993} обозначается $P((n-2,2))$, там же доказано, что этот многогранник 2-смежностен.
%Кроме того, легко показать, что $\phi_n$ является проекцией $\QAP(n)$ (см. утверждение~\ref{prop:1} ниже).

%В~\cite{Onn:2009} был сформулирован вопрос о том, являются ли многогранники $\phi_n$ и $\QAP(n)$, имеющие одинаковое число вершин, изоморфными.

%Ниже мы покажем, что многогранник $\phi_n$ не является 3-смежностным. Для $n=3$ это утверждение проверяется непосредственно, а для $n > 3$ будет доказана следующая

%\begin{lemma}	\label{lem:1}
%Многогранник $\phi_3$ аффинно эквивалентен грани многогранника $\phi_n$.
%\end{lemma}

%С другой стороны, из теоремы~\ref{thm:QAP2BQP} и 3-смежности многогранника $\BQP(n)$ следует, что $\QAP(n)$ 3-смежностен. Таким образом, $\phi_n$ и $\QAP(n)$ не могут быть изоморфны.

%Интерес к многогранникам $\phi_n$ и $\QAP(n)$, в частности, обусловлен тем, что многогранник коммивояжера $\TSP(n)$ является проекцией для каждого из них. Учитывая, что булев квадратичный многогранник $\BQP(k)$ является гранью многогранника $\TSP(n)$ при $n = 2k(2k-1)$~\cite{Maksimenko:2013TSP}, приходим к выводу, что $\BQP(k)$ является проекцией некоторой грани многогранника $\phi_n$ (многогранника Юнга $P((n-2,2))$). Более того, справедлива

%\begin{theorem}	\label{thm:2}
%Многогранник $\BQP(k)$ аффинно эквивалентен грани многогранника $\phi_{2k}$.
%\end{theorem}

%В~частности, $\phi_n$ содержит $3$-смежностную грань на $2^{\lfloor n/2\rfloor}$ вершинах.

%%%%%%%%%%%%%%%%%%%%%%%%%%%%%%%%%%%%%%%%%%%%%%%%%%%%%%%
%
% Многогранники раскрасок вершин графа
%
%%%%%%%%%%%%%%%%%%%%%%%%%%%%%%%%%%%%%%%%%%%%%%%%%%%%%%%

\subsection{Многогранники раскрасок вершин графа}
\label{sec:Color}

Одной из наиболее известных NP-полных задач является задача о раскраске вершин графа~\cite{Karp:1972}.
%, сформулированная выше, на с.~\hyperlink{pColor}{\pageref*{problem:Color}}. 

\problem{Задача о раскраске вершин графа.}
Задан граф $G = (V, E)$.
Требуется каждой его вершине $v \in V$ назначить некоторую метку $k_v \in \N$
(символизирующую номер цвета) так, чтобы любые две смежные вершины имели разные метки, а значение наибольшей метки было бы минимальным.

Эта задача имеет многочисленные приложения: от составления расписаний и распределения частот в сетях сотовой связи, до распределения ресурсов в компиляторах и автоматизации дифференцирования~\cite{Burke:2010,Palubeckis:2008}.
В~литературе можно найти несколько принципиально отличающихся друг от друга определений многогранников раскрасок графа.
Рассмотрим несколько наиболее естественных определений, отбросив, в частности, те, где размерность многогранника экспоненциальна относительно числа вершин графа. 

Далее предполагаем, что раскрашиваемый граф $G=(V,E)$ не содержит изолированных вершин (раскраска которых тривиальна).
Также исключим из рассмотрения случаи противоположного характера, когда в графе $G$ есть вершина, смежная со всеми остальными (тогда ее цвет должен отличаться от цветов всех остальных вершин).

%, а также случаи, когда граф $G$ содержит пару вершин, не смежных между собой, но смежных со всеми остальными вершинами (тогда их обе следует покрасить цветом, отличающимся от цветов остальных вершин).



\subsubsection{Стандартная формулировка}

Для каждой раскраски графа $G=(V,E)$ рассмотрим её характеристический вектор $\bm{x} \in \{0,1\}^{V\times k}$, $1 \le k \le |V|$, координаты которого определим следующим образом:
\[
x_{v,c} = \begin{cases}
1, & \text{если вершина $v$ раскрашена цветом $c$,}\\
0, & \text{иначе.}
\end{cases}
\]
Множество всех таких векторов обозначим $\ColorA(G,k)$.
Нетрудно заметить, что это множество может быть описано следующим набором ограничений~\cite{Mehrotra:1996,Delle:2016}:
\begin{align}
x_{v,c} + x_{u,c} \le 1&, \quad \{v,u\} \in E, \quad c\in[k], \label{eq:colora1}\\
\sum_{c=1}^k x_{v,c} = 1&, \quad v\in V, \label{eq:colora2}\\
x_{v,c} \in \{0,1\}&, \quad v\in V, \quad c\in[k]. \notag
\end{align}
Здесь неравенство~\eqref{eq:colora1} запрещает окрашивать смежные вершины одним цветом,
а уравнение~\eqref{eq:colora2} означает, что каждая вершина будет окрашена ровно одним цветом.
Очевидно, множество $\ColorA(G,k)$ не пусто тогда и только тогда, когда вершины графа $G$ могут быть раскрашены в $k$ цветов.
А минимизируя на этом множестве линейную функцию 
\[
f(\bm{x}) = \sum_{v\in V, \ c\in[k]} x_{v,c} |V|^c
\]
можно найти раскраску с наименьшим числом цветов.

\begin{prop}\label{prop:ColorA}
	$\Stable(G) \lea \ColorA(G,|V|)$ для любого графа $G=(V,E)$.
\end{prop}
\begin{proof}
	Для удобства положим $V = [n]$.
	Рассмотрим грань многогранника $\ColorA(G,|V|)$, лежащую в пересечении гиперплоскостей
	\[
	x_{v,c} = 0, \quad \text{где $v \ne c$ и $c \ne 1$.} 
	\]
	Тогда уравнения~\eqref{eq:colora2} примут вид
	\[
	x_{v,1} + x_{v,v} = 1, \qquad v\in [n],
	\]
	А из неравенств~\eqref{eq:colora1} нетривиальными останутся лишь
	\[
	x_{v,1} + x_{u,1} \le 1, \qquad \{v,u\} \in E.
	\]
	Очевидно, эта грань аффинно эквивалентна многограннику $\Stable(G)$.
\end{proof}

Значительное внимание в литературе (см., например, \cite{Mendez:2008,Burke:2010}) уделяется следующей модификации этого многогранника.
Положим $V = [n]$ и $k = n$, а неравенства~\eqref{eq:colora1} заменим на 
\begin{align*}
x_{v,c} + x_{u,c} &\le w_c, \qquad \{v,u\} \in E, \quad c\in[n],\\
w_c &\in \{0,1\}, \qquad c\in[n].
\end{align*}
(Если переменная $w_c$ равна нулю, то цвет $c$ не используется.)
Обозначим множество векторов $\bm{x} \in \{0,1\}^{n(n+1)}$, удовлетворяющих этим ограничениям, через $\ColorB(G)$.
При таком подходе целевая функция для задачи поиска хроматического числа графа $G$ приобретает особенно простой вид:
\[
\sum_{c\in[n]} w_c \to \min.
\]

Очевидно, 
\[
\ColorA(G,k) \lea \ColorB(G).
\]
Достаточно положить $w_c = 1$ при $c\in[k]$ и $w_c = 0$ при $c > k$.

Покажем теперь, что семейства многогранников $\ColorA$ и $\ColorB$ лежат в одном классе эквивалентности с многогранниками независимых множеств.
Так как $\Stable \propto_A \ColorA \propto_A \ColorB$, то остается доказать $\ColorB \propto_A \Stable$.

\begin{prop}\label{prop:ColorB}
	Для любого графа $G=([n],E)$ за линейное время можно построить $G'=(V',E')$, $|V'| = n(n+2)$, $|E'|=n^2(n+1)/2+(|E|+1)n$, что 
	\[
	\ColorB(G) \lea \Stable(G').
	\]
\end{prop}
\begin{proof}
	Достаточно систему ограничений
	\begin{align}
	\sum_{j=1}^n x_{ij} &= 1, \qquad i\in[n], \label{eq:proofColorB1}\\
	x_{ij} + x_{kj} &\le w_j, \qquad \{i,k\} \in E, \quad j\in[n], \label{eq:proofColorB2}
	\end{align}
	описывающую многогранник $\ColorB(G)$,
	преобразовать в систему ограничений вида $y_i + y_j \le 1$.
	
	Заметим, прежде всего, что неравенства~\eqref{eq:proofColorB2} эквивалентны неравенствам 
	\begin{align}
	x_{ij} &\le w_j, \qquad i,j\in[n], \label{eq:proofColorB2a}\\
	x_{ij} + x_{kj} &\le 1, \qquad \{i,k\} \in E, \quad j\in[n], \notag
	\end{align}
	при условии целочисленности переменных.
	
	Введем дополнительные переменные $\bar{w}_j \in \{0,1\}$, $j\in[n]$, и рассмотрим систему ограничений
	\begin{align}
	x_{ij} + x_{ik} &\le 1, \qquad i,j,k\in[n], \quad j < k, \notag \\
	w_j + \bar{w}_j &\le 1, \qquad j\in[n], \notag\\
	x_{ij} + \bar{w}_j &\le 1, \qquad i,j\in[n], \label{eq:proofColorB5}\\
	x_{ij} + x_{kj} &\le 1, \qquad \{i,k\} \in E, \quad j\in[n]. \notag %\label{eq:proofColorB6}
	\end{align}
	Очевидно, она определяет многогранник $\Stable(G')$ для некоторого графа $G'=(V',E')$, где $|V'| = n(n+2)$, $|E'|=n^2(n+1)/2+(|E|+1)n$.
	Полагая $w_j + \bar{w}_j = 1$, перейдем к рассмотрению некоторой грани этого многогранника.
	Ограничение~\eqref{eq:proofColorB5} при этом превратится в~\eqref{eq:proofColorB2a}.
	%\[x_{ij} \le w_j, \qquad i,j\in[n].\]
	%В~совокупности с неравенством~\eqref{eq:proofColorB6} оно эквивалентно неравенству~\eqref{eq:proofColorB2}.
	Остается заметить, что равенство~\eqref{eq:proofColorB1} определяет грань этой грани, аффинно эквивалентную многограннику $\ColorB(G)$.
\end{proof}

\begin{remark}
	На основе той же идеи в~\cite{Delle:2016} показано, что $\ColorA \propto_A \Stable$. Там же этот факт используется для вывода нового семейства допустимых неравенств для многогранников семейства $\ColorA$.
\end{remark}

\subsubsection{Формулировка с представителями}

Пусть $G=(V,E)$ "--- раскрашиваемый граф и $V=[n]$. 
Введем обозначение $\bar{E} = \Set{\{i,j\} \subset V \given \{i,j\}\notin E}$.
Для $i\in V$ определим $\bar{N}(i) = \Set{j\in V \given \{i,j\}\in \bar{E}}$.
Для произвольного $U \subseteq V$ через $E(U)$ обозначаем множество ребер графа $G$, оба конца которых лежат в $U$, а через $G[U] = (U, E[U])$ "--- подграф графа $G$, индуцированный множеством $U$.

Каждой вершине $i\in[n]$ графа $G$ поставим в соответствие координату $x_{ii}$, а каждой паре вершин $\{i,j\} \in \bar{E}$, $1 \le i < j \le n$, "--- координату $x_{ij}$ 0/1"~вектора $\bm{x}$.
Значения этих переменных имеют следующий смысл.
Для произвольной раскраски графа $G$ переменная $x_{ii}$ равна единице, если вершина $i$ имеет наименьший номер среди всех вершин, окрашенных тем же цветом,
то есть эта вершина является их \emph{представителем}.
А $x_{ij}=1$, если $i$ и $j$ окрашены в один цвет и $i$ является представителем для $j$.

В~\cite{Campelo:2008} в качестве многогранника раскрасок вершин графа $G$ предлагается рассмотреть выпуклую оболочку множества 0/1"~векторов $\bm{x}$, удовлетворяющих ограничениям
\begin{align}
x_{ii} + \sum_{j<i, \ j\in\bar{N}(i)}^n x_{ji} &= 1, \qquad i\in[n], \label{eq:ColorC1}\\
x_{ij} + x_{ik} &\le x_{ii}, \qquad \{j,k\} \in E(\bar{N}(i)), \quad i < j < k, \label{eq:ColorC2}\\
x_{ij} &\le x_{ii}, \qquad \text{если $j$ изолирована в $G[\bar{N}(i)]$, $i<j$.}\label{eq:ColorC3}
\end{align}
Первое ограничение гарантирует, что каждая вершина либо сама является представителем, либо представлена вершиной с меньшим номером.
Второе ограничение запрещает смежным вершинам иметь общего представителя.
Обозначим множество всех таких векторов $\bm{x}$ через $\ColorC(G)$.
Тогда минимум суммы
\begin{equation}
\label{eq:ColorC4}
\sum_{i\in[n]} x_{ii}
\end{equation}
по всем $\bm{x} \in \ColorC(G)$ совпадает с хроматическим числом графа $G$.

Учитывая целочисленность переменных, ограничения \eqref{eq:ColorC2} и~\eqref{eq:ColorC3} можно заменить следующими:
\begin{align}
x_{ij} + x_{ik} &\le 1, \qquad \{j,k\} \in E(\bar{N}(i)), \quad i < j < k, \label{eq:ColorC5}\\ 
x_{ij} &\le x_{ii}, \qquad j \in \bar{N}(i), \quad i<j. \label{eq:ColorC6}
\end{align}

Заметим, что описание многогранника $\ColorC(G)$ очень похоже на описание многогранника $\ColorB(G)$.
Поэтому следующий факт доказывается точно также, как и утверждение~\ref{prop:ColorB}.

\begin{prop}\label{prop:ColorC}
	Для любого графа $G=([n],E)$ за линейное время можно построить $G'=(V',E')$, $|V'| = 2n+|\bar{E}|$, $|E'|=O(n^3)$, что 
	\[
	\ColorC(G) \lea \Stable(G').
	\]
\end{prop}

Покажем теперь, что $\Stable \propto_A \ColorC$.

\begin{prop}
	$\Stable(G) \lea \ColorC(G')$ для любого графа $G=([n],E)$ и $G'=([n]\cup\{0\}, E)$.
\end{prop}
\begin{proof}
	Рассмотрим многогранник $\ColorC(G')$, где $G' = (V', E)$ и $V' = \{0,1,\dots,n\}$.
	По аналогии с доказательством утверждения~\ref{prop:ColorA},
	положим $x_{ij} = 0$ для всех $0 < i < j$, $\{i,j\} \in \bar{E}$
	и перейдем к рассмотрению соответствующей грани многогранника $\ColorC(G')$.
	Для остальных координат будут выполнены ограничения
	\begin{align*}
	x_{00} &= 1,\\
	x_{ii} + x_{0i} &= 1, \qquad i\in[n],\\
	x_{0i} + x_{0j} &\le 1, \qquad \{i,j\} \in E, \quad i < j.
	\end{align*}
	Таким образом, описанная грань аффинно эквивалентна $\Stable(G)$.
\end{proof}

В~\cite{Palubeckis:2008,Cornaz:2008} фактически предложена модификация многогранника $\ColorC(G)$, получаемая за счет удаления лишних переменных
(см. уравнения~\eqref{eq:ColorC1})
\[
x_{ii} = 1 - \sum_{j<i, \ j\in\bar{N}(i)}^n x_{ji}, \qquad i\in[n]. 
\]
При этом уравнения~\eqref{eq:ColorC1} заменяются неравенствами
\[
x_{ji} + x_{ki} \le 1, \qquad j,k\in\bar{N}(i), \quad j < k < i,
\]
неравенства~\eqref{eq:ColorC5} сохраняются в неизменном виде, неравенства~\eqref{eq:ColorC6} преобразуются в
\[
x_{ij} + x_{ki} \le 1, \qquad j,k \in \bar{N}(i), \quad k < i < j, 
\]
а целевая функция~\eqref{eq:ColorC4} принимает вид
\[
n - \sum_{\{i,j\}\in \bar{E}, \ i<j} x_{ij}.
\]
(Особого рассмотрения требует легко идентифицируемая ситуация, когда граф $G$ содержит пару вершин $i$ и $j$, не смежных между собой, но смежных со всеми остальными вершинами. В~этом случае следует положить $x_{ij} = 1$.)
Очевидно, эта модификация является многогранником независимых множеств $\Stable(H)$, где граф $H$ имеет $|\bar{E}|$ вершин, а его ребра определяются описанными выше неравенствами.

Таким образом, все рассмотренные выше семейства многогранников раскрасок графа эквивалентны (в смысле аффинной сводимости) семейству многогранников независимых множеств.



%%%%%%%%%%%%%%%%%%%%%%%%%%%%%%%%%%%%%%%%%%%%%%%%%%%%%%%
%
% Многогранники задачи коммивояжера
%
%%%%%%%%%%%%%%%%%%%%%%%%%%%%%%%%%%%%%%%%%%%%%%%%%%%%%%%

\section{Многогранники задачи коммивояжера}
\label{sec:TravellingAll}

\subsection{Многогранники классической задачи}
\label{sec:Travelling}

Пусть $G = ([m],E)$ "--- полный граф. Для каждого гамильтонова цикла $H \subseteq E$ в этом графе рассмотрим его характеристический вектор $\bm{y} \in \{0,1\}^E$ с координатами
\[
y_{ij} = \begin{cases}
1,& \text{если ребро $\{i,j\}$ входит в $H$,}\\
0,& \text{иначе.}
\end{cases}
\]
Выше, на с.~\pageref{def:TSP} многогранник гамильтоновых циклов (многогранник симметричной задачи коммивояжера) был определен как выпуклая оболочка множества $\TSP(m)$ всех таких векторов.
Таким же образом определен и многогранник гамильтоновых контуров (многогранник асимметричной задачи коммивояжера), представляющий собой выпуклую оболочку множества всех характеристических векторов $\ATSP(n) \subset \{0,1\}^A$ гамильтоновых контуров в полном орграфе $D = ([n],A)$.

О~сходстве многогранников $\TSP$ и $\ATSP$ было известно и раньше
(см., например, \cite{Junger:1995TSP, BondBook:1995}), здесь этот факт лишь приобретает новый вид. 

\begin{lemma}
\label{lem:ATSP2TSP}
$\ATSP(n) \lea \TSP(m)$ для $m=2n$.
\end{lemma}

\begin{proof}
Пусть $m=2n$.
Воспользуемся тем, что гиперплоскости вида $y_{ij}=0$ и $y_{ij}=1$ являются опорными для многогранника $\TSP(m)$.
Кроме того, гиперплоскости $y_{ij}=0$ и $y_{ij}=1$ разбивают множество вершин многогранника $\TSP(m)$ на два подмножества, соответствующие гамильтоновым циклам, содержащим или же несодержащим ребро $\{i,j\}$.
Этой особенностью воспользуемся, чтобы определить грань, аффинно эквивалентную многограннику $\ATSP(n)$.
	
Разобьем множество вершин $[m]$ на два равномощных подмножества $T = \{1, 2, \ldots, n\}$ и $U = \{n+1, n+2, \ldots, 2n\}$.
Рассмотрим набор $S$ гамильтоновых циклов, каждый из которых обладает следующими свойствами:
\begin{enumerate}
	\item Не проходит по ребрам вида $\{i,j\}$ и $\{n+i,n+j\}$, $1\le i < j \le n$.
	\item Обязательно содержит ребра $\{i,n+i\}$, $i \in [n]$.
\end{enumerate}
В~силу сделанного выше замечания эти циклы соответствуют множеству вершин некоторой грани многогранника $\TSP(m)$.
Заметим, что любой гамильтонов цикл из набора $S$ кроме $n$ <<обязательных>> ребер должен содержать еще $n$ ребер вида $\{i,n+j\}$, $i, j \in [n]$, $i\neq j$. 
Поставим в соответствие каждому ребру $\{i,n+j\}$ графа $G=([2n],E)$ симметричной задачи дугу $(i,j)$ орграфа $D=([n],A)$ асимметричной задачи.
Нетрудно убедиться, что таким образом будет установлено	взаимно"=однозначное соответствие между гамильтоновыми циклами набора $S$ и гамильтоновыми контурами орграфа $D$.
Отсюда следует, что указанная грань многогранника $\TSP(m)$ аффинно эквивалентна многограннику $\ATSP(n)$.
\end{proof}

С другой стороны, многогранник $\ATSP(n)$ может быть преобразован в многогранник $\TSP(m)$,
где $m=n$, с помощью очевидного линейного отображения 
$y_{ij} = x_{ij} + x_{ji}$, $1\le i < j \le n$.
При этом каждая пара противоположно направленных гамильтоновых
контуров будет отображаться в один гамильтонов цикл.

Ниже все рассмотрения будут касаться лишь многогранника асимметричной задачи.
Теперь они могут быть легко перенесены и на симметричный случай.
Мы также будем пользоваться обозначением $\ATSP(D')$ для множества всех характеристических векторов гамильтоновых контуров подграфа $D'$ полного орграфа $D$. Очевидно, $\ATSP(D') \lea \ATSP(n)$, если число вершин орграфа $D'$ не превосходит $n$.

\begin{theorem}[\cite{Maksimenko:2011}]
\label{thm:SAT2TSP}
Пусть $U=\{u_1,\dots,u_d\}$ "--- множество булевых переменных, $C=\{C_1,\dots,C_m\}$ "--- набор дизъюнкций над $U$, $\len(C)$ "--- суммарная длина всех дизъюнкций из набора $C$, измеряемая в литералах.
Тогда
\[
\SAT(U,C) \lea \ATSP(n), \quad \text{где } n = |U| + 2 \len(C).
\]
\end{theorem}

\begin{proof}
Достаточно для каждого многогранника $\SAT(U, C)$ привести пример подграфа $D'$ полного орграфа $D=(V,A)$ такого, что $\SAT(U, C)$ окажется аффинно эквивалентен многограннику $\ATSP(D')$.
С этой целью, при помощи стандартных средств классической теории сводимости~\cite{Garey:1982} мы сконструируем алгоритм сведения задачи выполнимость к задаче гамильтонов контур.
При этом между множествами допустимых решений этих задач установится взаимно"=однозначное соответствие.
То же соответствие установится и между вершинами многогранников этих задач.
Далее останется лишь показать, что оно представляет собой аффинное невырожденное отображение.
	
Искомый орграф $D'$ будем конструировать из вершин $v_i$,	$1\le i \le d$, и компонент $D_j$, $1\le j \le m$.
Вершины $v_i$ будут соответствовать переменным $u_i \in U$,
а компоненты $D_j$ "--- дизъюнкциям $C_j$.
Из каждой вершины $v_i$ в орграфе $D'$ будут выходить	ровно две дуги, соответствующие значениям <<истина>> и <<ложь>>	переменной $u_i$. 
Относительно вершины $v_i$ из всего множества компонент $D_j$, $1\le j \le m$, выделим два подмножества:
$H_i$ будет содержать все те компоненты, соответствующие дизъюнкции которых содержат литерал $u_i$;
$\bar{H}_i$ будет содержать все те компоненты, соответствующие дизъюнкции которых содержат литерал $\bar{u}_i$.
Из вершины $v_i$ дугу <<истина>> направим в <<первую>> (порядок не важен) компоненту из множества $H_i$.
Затем <<первую>> компоненту соединим дугой со <<второй>> компонентой из $H_i$ и т.\,д.
Таким образом, все компоненты из $H_i$ будут соединены ориентированной цепью. 
Из последней компоненты соответствующую дугу направим в вершину $v_{i+1}$ (сложение по модулю $d$).
Если множество $H_i$ пустое, то дугу <<истина>> из вершины $v_i$ направим непосредственно в $v_{i+1}$.
Аналогичные построения проделаем с множеством $\bar{H}_i$, начав цепочку с дуги <<ложь>>, выходящей из $v_i$ (см., для примера, рис.~\ref{fig:SATTSP}).
Итак, в каждую компоненту $D_j$ теперь входит и выходит ровно столько дуг, сколько литералов содержится в соответствующем дизъюнкте $C_j$.
		
\begin{figure}[tbh]
\centering
\tikzset{small circle/.style={inner sep = 1.5pt,draw,circle}}
\begin{tikzpicture}[scale=1.9,
>={Stealth[scale width=0.8]} % Определяем вид стрелок
]
%\node[small circle,densely dotted] (u) at (-2,0) {};	
\node[small circle] (v3) at (0,1) {};	
\node[small circle] (v2) at (0,0) {};	
\node[small circle] (v1) at (0,-1) {};	
\node[small circle] (v12) at (2.2,0) {};	
\node[small circle] (w12) at (3.2,0) {};	
\node[small circle] (v11) at (2.2,-1) {};	
\node[small circle] (w11) at (3.2,-1) {};	
\node[small circle] (v23) at (5,1) {};	
\node[small circle] (w23) at (6.5,1) {};	
\node[small circle] (v22) at (4.5,0) {};	
\node[small circle] (w22) at (6.0,0) {};	
\node[small circle] (v21) at (5,-1) {};	
\node[small circle] (w21) at (6.5,-1) {};	
\node[small circle,densely dotted] (w1) at (8,1) {};	
\node[small circle,densely dotted] (w3) at (8,0) {};	
\node[small circle,densely dotted] (w2) at (8,-1) {};	

\draw[->] (v3) node[above] {$v_3$} -- node[pos=0.03,above right] {$u_3 = \text{истина}$} (v23) node[above] {$v_{23}$};
\draw[->] (v23) -- (w23) node[above] {$w_{23}$};
\draw[->] (w23) -- (w1) node[right] {$v_1$};

\draw[->] (v2) node[below] {$v_2$} -- node[pos=0.075,above right] {$u_2 = \text{ложь}$} (v12) node[above] {$v_{12}$};
\draw[->] (v12) -- (w12) node[above] {$w_{12}$};
\draw[->] (w12) -- (v22) node[above left] {$v_{22}$};
\draw[->] (v22) -- (w22) node[above left] {$w_{22}$};
\draw[->] (w22) -- (w3) node[right] {$v_3$};

\draw[->] (v1) node[below left] {$v_1$} -- node[pos=0.075,above right] {$u_1 = \text{истина}$} (v11) node[below] {$v_{11}$};
\draw[->] (v11) -- (w11) node[below] {$w_{11}$};
\draw[->] (v21) node[below] {$v_{21}$} -- (w21) node[below] {$w_{21}$};
\draw[->] (w21) -- (w2) node[right] {$v_2$};

\draw[->] (v3) to[bend right] node[below,sloped] {$u_3 = \text{ложь}$} (v1);
\draw[->] (v2) to[bend right] (v3);
\draw[->] (v1) to[bend right] node[pos=0.25,above,sloped] {$u_1 = \text{ложь}$} (v21);
\draw[->] (w11) to[bend right] (w2);

\draw[->] (v11) to[bend left] (v12);
\draw[->] (v12) to[bend left] (v11);
\draw[->] (w11) to[bend right] (w12);
\draw[->] (w12) to[bend right] (w11);

\draw[->] (v21) to (v22);
\draw[->] (v22) to (v23);
\draw[->] (v23) to[bend left] (v21);

\draw[->] (w21) to[bend right] (w23);
\draw[->] (w23) to (w22);
\draw[->] (w22) to (w21);

\draw[dotted] ($(v12)!0.5!(w11)$) node {$D_1$} circle (1.05); 
\draw[dotted] ($(v23)!0.5!(w21)!0.06!(v22)$) circle (1.65); 
\draw ($(v21)!0.5!(w22)$) node[right] {$D_2$}; 
\end{tikzpicture}
\caption{Орграф $D'$ для формулы $(u_1\vee\bar{u}_2)\wedge(\bar{u}_1\vee\bar{u}_2\vee u_3)$.}
\label{fig:SATTSP}
\end{figure}
	
Определим внутреннее устройство компоненты $D_j$.
Она состоит из вершин $v_{jl}$ и $w_{jl}$, $1\le l \le p_j$, где $p_j$ "--- число литералов в дизъюнкте $C_j$.
Таким образом, между п\'{а}рами $v_{jl}$ и $w_{jl}$ ($1\le l \le p_j$) и литералами дизъюнкта $C_j$ можно установить взаимно-однозначное соответствие.
И, следовательно, каждая такая пара оказывается <<привязанной>>	к некоторой вершине $v_i$, точнее, к одной из двух исходящих из этой вершины дуг, соответствующих значению <<истина>> для литералов $u_i$ и $\bar{u}_i$, соответственно. 
Вершины компоненты соединены между собой <<внутренними>> дугами	трех типов:
$(v_{j l}, w_{j l})$, $(v_{j l}, v_{j \, l+1})$ и $(w_{j \, l+1}, w_{j l})$, $1\le l \le p_j$, (сложение $l+1$ выполняется по модулю $p_j$).
Других <<внутренних>> дуг у компоненты $D_j$ нет.
О~том, как <<внешние>> дуги соединяют $D_j$ с другими компонентами и вершинами $v_i$ конструируемого орграфа, в общих чертах было сказано выше.
Остается лишь уточнить детали.
Допустим, что пара вершин $v_{j1}$, $w_{j1}$ соответствует литералу $\bar{u}_i$,
присутствующему в дизъюнкте $C_j$.
Тогда, как было сказано, компонента $D_j$ является одним из звеньев цепи, соединяющей	вершину $v_i$ и другие компоненты из множества $\bar{H}_i$.
Причем начало цепи проходит по дуге <<ложь>>.
Из двух дуг, соединяющих компоненту $D_j$ с соседями по этой цепи,
входящая дуга будет оканчиваться в $v_{j1}$, а исходящая "--- начинаться в $w_{j1}$.

Покажем, что прохождение гамильтонова контура по компоненте $D_j$ однозначно определяется набором дуг этого контура, входящих извне в эту компоненту.
А именно, покажем, что наличием (или отсутствием) дуги, входящей в вершину $v_{jl}$, однозначно определяется наличие <<внешней>> дуги, исходящей из $w_{jl}$, и дуги $(v_{j\,l-1}, w_{j\,l-1})$, а также отсутствие дуг $(v_{j \, l-1}, v_{j l})$ и $(w_{j l}, w_{j \, l-1})$ (вычитание $l-1$ выполняется по модулю $p_j$).

{\sloppy
Действительно, если в некотором гамильтоновом контуре присутствует <<внешняя>> для $D_j$ дуга, входящая в $v_{jl}$, то дуги $(v_{j \, l-1}, v_{j l})$ в нем быть не может. Но~тогда покинуть вершину $v_{j \, l-1}$ данный гамильтонов контур может только по дуге $(v_{j\,l-1}, w_{j\,l-1})$. Значит, дуга $(w_{j l}, w_{j \, l-1})$ в нем отсутствует. Последнее означает, что покинуть вершину $w_{j l}$ этот гамильтонов контур может только по исходящей из нее <<внешней>> дуге.
Если же предположить, что в некотором гамильтоновом контуре отсутствует <<внешняя>> для $D_j$ дуга, входящая в $v_{jl}$, то попасть в эту вершину данный контур может только по дуге $(v_{j \, l-1}, v_{j l})$. Следовательно, дуга $(v_{j\,l-1}, w_{j\,l-1})$ в нем отсутствует. Тогда в вершину $w_{j \, l-1}$ этот контур может попасть только по дуге $(w_{j l}, w_{j \, l-1})$. Последнее означает, что исходящую из вершины $w_{j l}$ <<внешнюю>> дугу этот контур не содержит.
%С помощью аналогичных рассуждений легко проверяется случай, когда в гамильтоновом контуре отсутствует <<внешняя>> для $D_j$ дуга, входящая в $v_{jl}$.

}

Заметим теперь, что конфигурация любого гамильтонова контура в орграфе $D'$ однозначно определяется набором дуг, берущих начало в вершинах $v_i$, или, что то же самое, набором значений булевых переменных $u_i$, $1\le i \le d$.
Кроме того, гамильтонов контур в орграфе $D'$ для каждой компоненты $D_j$ должен содержать хотя бы одну входящую в нее дугу (соответственно, дизъюнкция $C_j$ должна содержать хотя бы один литерал, принимающий значение <<истина>>).
Заметим также, что все дуги, соединяющие множество компонент $H_i$ (или $\bar{H}_i$), присутствуют (или отсутствуют) в некотором гамильтоновом контуре вместе с дугой <<истина>>, выходящей из $v_i$.
Таким образом, наличие <<внешних>> дуг в гамильтоновом контуре линейно зависит от значений переменных из $U$.
А <<внутренние>> дуги компонент $D_j$, как было показано выше, линейно зависят от <<внешних>>.
Таким образом, многогранник $\SAT(U,C)$ оказывается аффинно эквивалентен $\ATSP(D')$.
\end{proof}

%\begin{remark}
%Из только что доказанной теоремы следует, что семейство многогранников задачи о $k$"~выполнимости при любом фиксированном $k \in \N$ аффинно сводится к семейству многогранников (а)симметричной задачи коммивояжера, но \emph{не следует} аффинная сводимость всего семейства $\SAT$ к $\ATSP$, так как размерность соответствующего многогранника $\ATSP(n)$ зависит не только от числа переменных $d=|U|$, но и от длины формулы $C$, которая может оказаться экспоненциальной, если число литералов в каждой дизъюнкции не ограничено сверху. В~качестве возможного развития этой темы заметим, что определение аффинной сводимости можно было бы скорректировать таким образом, чтобы полиномиальным ограничением были связаны не размерности пространств, а длины описаний многогранников. (Почти во всех рассмотренных до сих пор примерах длины описаний многогранников ограничены сверху полиномами от размерности. Исключениями являются многогранники задачи о рюкзаке (и другие, связанные с ними семейства) и многогранники задачи о покрытии множества.) Но при таком подходе аффинная сводимость была бы поставлена в зависимость от \emph{способа описания} многогранника. 
%\end{remark}


С другой стороны, из утверждения~\ref{thm:kSAT} следует, что для любого фиксированного $k \in \N$ семейство многогранников (а)симметричной задачи коммивояжера не может быть аффинно сведено к семейству многогранников задачи о $k$"~выполнимости~\cite{Fiorini:2003}.

Теорема~\ref{thm:SAT2TSP} имеет ряд интересных следствий.

Первое из них "--- NP-полнота проверки несмежности вершин для многогранников $\ATSP$ и $\TSP$, наследующих это свойство от многогранников задачи о 3"~выполнимости. В~\cite{Papadimitriou:1978} этот факт был доказан иными методами.

Второе следствие связано с тем, что семейство многогранников задачи о выполнимости совпадает с семейством 0/1"~многогранников.
Например, пусть $U = \{u_1, \dots, u_d\}$ и набор $C$ состоит из одной дизъюнкции
$\{u_1, u_2, \ldots, u_{d-1}, \bar{u}_d\}$.
Тогда множество $\SAT(U,C)$ содержит все $d$"~мерные 0/1"~вектора за исключением вектора $(0, 0, \ldots, 0, 1)$.
Теперь, чтобы убрать из этого множества еще один 0/1"~вектор, например
$(0, 1, \ldots, 1, 1)$, достаточно добавить в $C$ еще одну дизъюнкцию:
$$
C = \bigl\{\{u_1, u_2, \ldots, u_{d-1}, \bar{u}_d\},
\{u_1, \bar{u}_2, \bar{u}_3, \ldots, \bar{u}_n\}\bigr\}.
$$ 
Действуя далее в том же духе, приходим к следующему выводу.

\begin{lemma} %[о совпадении $P(X)$ и $S(U,C)$]
\label{lem:SAT01}
Для каждого $X\subseteq \{0,1\}^d$ можно построить набор дизъюнкций $C$ над $U=\{u_1,\dots,u_d\}$ так, что $X = \SAT(U,C)$ и $\len(C) = k d$, где $k = 2^d - |X|$.
\end{lemma}

А с учетом теоремы~\ref{thm:SAT2TSP}, получаем

\begin{corollary}
\label{cor:01toTSP}
Любой $d$-мерный 0/1"~многогранник на $2^d - k$ вершинах ($0 \le k \le 2^d - 1$) аффинно эквивалентен некоторой грани многогранника $\ATSP(n)$ при $n = (2k+1)d$.
\end{corollary}

Ранее Биллера и Сарангараджан~\cite{Billera:1996} доказали это утверждение для $n = (4k+1)d$ иными средствами.

В~этой связи упомянем следующий факт.
Число комбинаторно неэквивалентных $d$"~мерных 0/1-многогранников ограничено снизу величиной $2^{2^{d-2}}$ при $d\ge 6$~\cite[Proposition 8]{Ziegler:2000} (см. также~\cite[с.~102, упр.~2.6]{ZieglerBook}).
(Все 0/1"~многогранники для $d \le 5$ перечисляются в~\cite{Aichholzer:2000}.)
С другой стороны, если $f(n)$ "--- число всех граней 0/1"~многогранника $P_{\textbf{0/1}}(n) \subset \R^n$, то
\[
f(n) \le n \binom{2^n}{n} \le 2^{n^2}.
\]
Таким образом, если каждый $d$"~мерный 0/1"~многогранник является гранью некоторого многогранника $P_{\textbf{0/1}}(n)$, то $n$ экспоненциально по $d$.
В~частности, то же самое верно и в отношении многогранника $\ATSP(n)$, но с небольшой корректировкой: $n$ сверхполиномиально по $d$.

Обратим теперь внимание на то, что в некоторых частных случаях длину формулы $C$ в лемме~\ref{lem:SAT01} можно существенно уменьшить.
В~качестве иллюстрации приведем альтернативное описание многогранника $\BQP(m)$.

Заметим, что набор ограничений (из определения $\BQP(m)$)
\[
x_{ij} = x_{ii} x_{jj}, 
\]
где $x_{ii}, x_{jj} \in \{0,1\}$, $1\le i < j \le m$,
эквивалентен выполнению дизъюнкций
\[
x_{ii} \vee \bar{x}_{ij}, \quad
x_{jj} \vee \bar{x}_{ij}, \quad 
\bar{x}_{ii} \vee \bar{x}_{jj} \vee x_{ij}, \quad
1 \le i < j \le m.
\]
Таким образом, булев квадратичный многогранник $\BQP(m)$ является
многогранником задачи о выполнимости $\SAT(U, C)$, где $|U|=m(m+1)/2$, $\len(C)=7m(m-1)/2$.

Используя теорему~\ref{thm:SAT2TSP}, получаем

\begin{corollary}[\cite{Maksimenko:2011}]
	$\BQP(m) \lea \ATSP(n)$, где $n = 7{,}5 m^2 - 6{,}5 m$.
\end{corollary}

Покажем теперь, что между семействами $\BQP$ и $\ATSP$ имеется более тесная связь.
 
\begin{theorem}[\cite{Maksimenko:2013TSP}]
	$\BQP(m) \lea \ATSP(n)$, где $n = 2 m^2 - m$.
	\label{th1} 
\end{theorem}

\begin{proof}
	Также, как это было сделано в доказательстве теоремы~\ref{thm:SAT2TSP}, 
	приведем пример орграфа $D'$ на $2 m^2 - m$ вершинах такого, что $\BQP(m)$ аффинно эквивалентен многограннику $\ATSP(D')$.

	% Рисунок 1
\tikzset{small circle/.style={inner sep = 1.5pt,draw,circle}}

\newcommand{\Eij}{
	\node[small circle] (vm) at (0,2) {};	
	\node[small circle] (vp) at (0,-2) {};	
	\node[small circle] (w) at (0,0) {};	
	\node[small circle] (u) at (3,0) {};	
	\coordinate (1) at (-1.2,2);	
	\coordinate (2) at (-1.2,-2);	
	\coordinate (3) at (3,2.8);	
	\coordinate (4) at (4,2);	
	\coordinate (5) at (4,-2);	
	\coordinate (6) at (3,-2.8);	
	\node[above] at (vm) {$v^-_{ij}$};
	\node[below] at (vp) {$v^+_{ij}$};
	\node[left] at (w) {$w_{ij}$};
	\node[right] at (u) {$u_{ij}$};
	\node[left] at (1) {$1$};
	\node[left] at (2) {$2$};
	\node[right] at (4) {$4$};
	\node[right] at (5) {$5$};
	\node[left] at (3) {$3$};
	\node[left] at (6) {$6$};
}
\begin{figure}[bh]
	\centering
	\begin{tikzpicture}[scale=0.7,
	>={Stealth[scale width=0.8]} % Определяем вид стрелок
	]
	\Eij
	\draw[->] (1) to (vm);
	\draw[->] (vm) to (4);
	\draw[->] (2) to (vp);
	\draw[->] (vp) to (5);
	\draw[->] (3) to (u);
	\draw[->] (u) to (6);
	\draw[->] (vm) to (w);
	\draw[->] (w) to[bend left] (vm);
	\draw[->] (u) to (vm);
	\draw[->] (vp) to (w);
	\draw[->] (w) to[bend right] (vp);
	\draw[->] (u) to (vp);
	\draw[->] (w) to (u);
	\end{tikzpicture}
	\caption{Компонента $E_{ij}$ орграфа $D'$.}
	\label{fig:BQP2ATSP-1}
\end{figure}
	
	Построения начнем с описания свойств типичной компоненты $E_{ij}$ орграфа $D'$, $1\le i < j \le m$.
	Эта компонента состоит из четырех вершин $v^-_{ij}$, $v^+_{ij}$, $w_{ij}$ и $u_{ij}$.
	Все дуги орграфа $D'$, инцидентные этим вершинам, указаны на рисунке~\ref{fig:BQP2ATSP-1}.
	Как видно, вершина $w_{ij}$ не имеет непосредственных <<соединений>>
	с внешними для компоненты $E_{ij}$ вершинами.
	
	% Рисунок 2

\begin{figure}
	%	\centering
	\noindent
	\hspace*{\fill}
	\hspace*{0mm}
	\begin{tikzpicture}[scale=0.7,
	>={Stealth[scale width=0.8]} % Определяем вид стрелок
	]
	\node[right] at (-2.5,3.2) {$\{1\}$};
	\Eij
	\draw[->,thick] (1) to (vm);
	\draw[->,thick] (vm) to (w);
	\draw[->,thick] (w) to (u);
	\draw[->,thick] (u) to (vp);
	\draw[->,thick] (vp) to (5);
	
	\draw[->,gray] (vm) to (4);
	\draw[->,gray] (2) to (vp);
	\draw[->,gray] (3) to (u);
	\draw[->,gray] (u) to (6);
	\draw[->,gray] (w) to[bend left] (vm);
	\draw[->,gray] (u) to (vm);
	\draw[->,gray] (vp) to (w);
	\draw[->,gray] (w) to[bend right] (vp);
	\end{tikzpicture}
	\hfill
	\hspace*{0mm}
	\begin{tikzpicture}[scale=0.7,
	>={Stealth[scale width=0.8]} % Определяем вид стрелок
	]
	\node[right] at (-2.5,3.2) {$\{2\}$};
	\Eij
	\draw[->,thick] (2) to (vp);
	\draw[->,thick] (vp) to (w);
	\draw[->,thick] (w) to (u);
	\draw[->,thick] (u) to (vm);
	\draw[->,thick] (vm) to (4);
	
	\draw[->,gray] (vp) to (5);
	\draw[->,gray] (1) to (vm);
	\draw[->,gray] (3) to (u);
	\draw[->,gray] (u) to (6);
	\draw[->,gray] (w) to[bend left] (vm);
	\draw[->,gray] (u) to (vp);
	\draw[->,gray] (vm) to (w);
	\draw[->,gray] (w) to[bend right] (vp);
	\end{tikzpicture}
	\hspace*{\fill}
	\bigskip
	
	\noindent
	\hspace*{\fill}
	\begin{tikzpicture}[scale=0.7,
	>={Stealth[scale width=0.8]} % Определяем вид стрелок
	]
	\node[right] at (-2.5,3.2) {$\{1,3\}$};
	\Eij
	\draw[->,thick] (1) to (vm);
	\draw[->,thick] (vm) to (w);
	\draw[->,thick] (w) to[bend right] (vp);
	\draw[->,thick] (vp) to (5);
	\draw[->,thick] (3) to (u);
	\draw[->,thick] (u) to (6);
	
	\draw[->,gray] (vm) to (4);
	\draw[->,gray] (2) to (vp);
	\draw[->,gray] (w) to[bend left] (vm);
	\draw[->,gray] (u) to (vm);
	\draw[->,gray] (u) to (vp);
	\draw[->,gray] (vp) to (w);
	\draw[->,gray] (w) to (u);
	\end{tikzpicture}
	\hfill
	\begin{tikzpicture}[scale=0.7,
	>={Stealth[scale width=0.8]} % Определяем вид стрелок
	]
	\node[right] at (-2.5,3.2) {$\{2,3\}$};
	\Eij
	\draw[->,thick] (2) to (vp);
	\draw[->,thick] (vp) to (w);
	\draw[->,thick] (w) to[bend left] (vm);
	\draw[->,thick] (vm) to (4);
	\draw[->,thick] (3) to (u);
	\draw[->,thick] (u) to (6);
	
	\draw[->,gray] (vp) to (5);
	\draw[->,gray] (1) to (vm);
	\draw[->,gray] (u) to (vm);
	\draw[->,gray] (u) to (vp);
	\draw[->,gray] (vm) to (w);
	\draw[->,gray] (w) to[bend right] (vp);
	\draw[->,gray] (w) to (u);
	\end{tikzpicture}
	\hspace*{\fill}
	
	\caption{Четыре варианта прохождения гамильтонова контура в зависимости от набора дуг, входящих извне в компоненту $E_{ij}$.}
	\label{fig:BQP2ATSP-2}
\end{figure}
	
	Будем предполагать, кроме того, что проходящий через эту компоненту гамильтонов контур обязательно содержит ровно одну из дуг 1 и 2, оканчивающихся в вершинах $v^-_{ij}$ и $v^+_{ij}$, соответственно.
	Нетрудно проверить, что такой гамильтонов контур может иметь лишь одну из четырех конфигураций, представленных на рис.~\ref{fig:BQP2ATSP-2}.
	При этом набор дуг каждой конфигурации однозначно определяется набором дуг 1, 2 и 3, направленных извне в данную компоненту.
	
	Чтобы убедиться в этом, введем следующие обозначения.
	Координаты векторов из $\ATSP(D')$, соответствующие дугам 1--6 компоненты $E_{ij}$, обозначим $x^1_{ij}$, \ldots, $x^6_{ij}$ соответственно.
	Так как гамильтонов контур должен содержать ровно одну из дуг 1 и 2, получаем первое уравнение
	\begin{equation}
	\label{lin1}
	x^2_{ij} = 1 - x^1_{ij}.
	\end{equation}
	На рис.~\ref{fig:BQP2ATSP-2} можно углядеть еще ряд соотношений:
	\begin{equation}
	\label{lin2}
	x^5_{ij} = x^1_{ij}, \quad x^4_{ij} = x^2_{ij}, \quad x^6_{ij} = x^3_{ij}.
	\end{equation}
	
	Теперь разберемся с внутренними дугами компоненты.
	Для произвольной дуги, ведущей из вершины $a$ в вершину $b$,
	соответствующую координату характеристического вектора обозначаем $x(a,b)$.
	Из рис.~\ref{fig:BQP2ATSP-2} видны следующие взаимосвязи:
	\newcommand{\oo}[1]{\left(#1\right)}
	\begin{equation}
	\label{lin3}
	x\oo{v^+_{ij}, w_{ij}} = x^2_{ij}, \quad 
	x\oo{v^-_{ij}, w_{ij}} = x^1_{ij}, \quad 
	x\oo{  w_{ij}, u_{ij}} = 1 - x^3_{ij}.
	\end{equation}
	
	Для характеризации оставшихся четырех дуг введем дополнительную переменную
	\begin{equation}
	\label{Edge}
	e_{ij} = x^2_{ij} x^3_{ij}.
	\end{equation}
	С ее использованием оставшиеся уравнения приобретают вид
	\begin{equation}
	\label{lin4}
	\begin{aligned}
	x\oo{ w_{ij}, v^-_{ij}} &= e_{ij}, &
	x\oo{ w_{ij}, v^+_{ij}} &= x^3_{ij} - e_{ij}, \\
	%\label{lin5}
	x\oo{ u_{ij}, v^-_{ij}} &= x^2_{ij} - e_{ij}, &
	x\oo{ u_{ij}, v^+_{ij}}  &= e_{ij} + x^1_{ij} - x^3_{ij}.
	\end{aligned}
	\end{equation}
	
	Итак, компонента построена таким образом, что присутствие любой ее дуги
	в гамильтоновом контуре линейно зависит лишь от присутствия дуг 1 и 3,
	и дополнительной переменной $e_{ij}$.

\newcommand{\smalltri}[3] % Маленький треугольник
{
	\begin{scope}[xshift=#2,yshift=#3]
		\node[small circle] (vm#1) at (0, 0.6) {};
		\node[small circle] (w) at (0, 0) {};
		\node[small circle] (vp#1) at (0, -0.6) {};
		\node[small circle] (u#1) at (1.2, 0) {};
		\draw (vm#1) -- (w) -- (vp#1) -- (u#1) -- (vm#1);
		\node at (0.44,0) {$E_{#1}$};
	\end{scope}
}

\begin{figure}[tbh]
	\centering
	\begin{tikzpicture}[scale=1.1,
	>={Stealth[scale width=0.8]} % Определяем вид стрелок
	]
	\smalltri{12}{0}{0}
	\smalltri{13}{3.0cm}{0}
	\smalltri{14}{6.0cm}{0}
	\smalltri{23}{3.0cm}{-2.4cm}
	\smalltri{24}{6.0cm}{-2.4cm}
	\smalltri{34}{6.0cm}{-4.8cm}
	\node[below] at (vp12) {$v^+_{12}$};
	\node[below] at (vp23) {$v^+_{23}$};
	\node[below] at (vp34) {$v^+_{34}$};
	\node[below] at (vp14) {$v^+_{14}$};
	\node[below] at (vp13) {$v^-_{13}$};
	\node[below] at (vp24) {$v^-_{24}$};
	\node[above] at (vm12) {$v^-_{12}$};
	\node[above] at (vm13) {$v^+_{13}$};
	\node[above] at (vm14) {$v^-_{14}$};
	\draw[->] (vm12) to (vm13);
	\draw[->] (vm13) to (vm14);
	\draw[->] (vm23) to (vm24);
	\draw[->] (vp12) to (vp13);
	\draw[->] (vp13) to (vp14);
	\draw[->] (vp23) to (vp24);
	\draw[->] (u13) to (u23);
	\draw[->] (u14) to (u24);
	\draw[->] (u24) to (u34);
	\draw[->] (u12) to[out=-90,in=180] (vp23);
	\draw[->] (u23) to[out=-90,in=180] (vp34);
	
	\foreach \x in {1,2,3,4} \node[small circle] (v\x) at ({\x*3-4.3}, 1) {};
	\foreach \x in {1,2,3,4} \node[above] at (v\x) {$v_{\x}$};
	\node[small circle,dotted] (vv1) at ({12-4.3},-5.8) {};
	\foreach \x in {2,3,4} \node[small circle,dotted] (vv\x) at (9,{-2.4*\x+4.8}) {};
	\foreach \x in {1,2,3,4} \node[right] at (vv\x) {$v_{\x}$};
	\draw[->] (v1) to[out=-90,in=180] (vp12);
	\draw[->] (v1) to[out=-45,in=180] (vm12);
	\draw[->] (v2) to[out=-135,in=90] (u12);
	\draw[->] (v2) to[out=-90,in=180] (vm23);
	\draw[->] (v3) to[out=-135,in=90] (u13);
	\draw[->] (v3) to[out=-90,in=180] (vm34);
	\draw[->] (v4) to[out=-135,in=90] (u14);
	\draw[->] (v4) to (vv1);
	\draw[->] (u34) to[out=-90,in=150] (vv1);
	\draw[->] (vm14) to[out=0,in=150] (vv2);
	\draw[->] (vp14) to[out=0,in=-150] (vv2);
	\draw[->] (vm24) to[out=0,in=150] (vv3);
	\draw[->] (vp24) to[out=0,in=-150] (vv3);
	\draw[->] (vm34) to[out=0,in=150] (vv4);
	\draw[->] (vp34) to[out=0,in=-150] (vv4);
	
	
	\end{tikzpicture}
	\caption{Подграф $D'$ для $m=4$.}
	\label{fig:BQP2ATSP-3}
\end{figure}
	
	Теперь приступим к описанию всего орграфа $D'$.
	Он будет состоять из компонент $E_{ij}$, $1\le i < j \le m$, и дополнительных вершин $v_i$, $1\le i\le m$.
	Каждая компонента соединяется своими внешними дугами с <<соседними>> по индексам компонентами, а <<крайние>> компоненты соединяются с дополнительными вершинами так, как это показано на рис.~\ref{fig:BQP2ATSP-3}.
	Из каждой дополнительной вершины~$v_i$ выходит ровно две дуги.
	Таким образом, для каждого гамильтонова контура в~$D'$ справедливы следующие равенства:
	\begin{equation}
	\label{lin5}
	\begin{aligned}
	x(v_1, v^-_{12}) &= 1 - x(v_1, v^+_{12}), \qquad	x(v_m, v_1) = 1 - x(v_m, u_{1m}), \\
	x(v_i, v^-_{i, i+1}) &= 1 - x(v_i, u_{1i}), \quad i = 2,\dots,m-1.
	\end{aligned}
	\end{equation}
	
	Орграф сконструирован таким образом, что конфигурация любого его гамильтонова контура однозначно определяется набором дуг, выходящих из вершин $v_i$, $i\in[m]$.
	Таким образом, число всех гамильтоновых контуров равно $2^{m}$ и совпадает
	с числом вершин многогранника $\BQP(m)$.
	Остается определить линейные соотношения, связывающие $\BQP(m)$ и $\ATSP(D')$.
	
	Согласно определению булева квадратичного многогранника, $\BQP(m)$ состоит из векторов $\bm{y} = (y_{ij}) \in \{0,1\}^{m(m+1)/2}$, удовлетворяющих условиям $y_{ij} = y_{ii} y_{jj}$, $1 \le i < j \le m$.
	Положим
	$$
	x(v_1, v^+_{12}) = y_{11} \quad \text{и} \quad
	x(v_i, u_{1i})   = y_{ii}, \quad i=2,\dots,m.
	$$
	Из уравнений \eqref{lin2}, с учетом конструкции всего графа $D'$,
	получаем $x^2_{ij} = y_{ii}$ и $x^3_{ij} = y_{jj}$ для каждой компоненты $E_{ij}$.
	Дополнительную переменную $e_{ij}$, ранее определявшуюся уравнением \eqref{Edge}, положим равной $y_{ij}$. 
	Ясно, что это будет эквивалентная замена.
	Все остальные координаты вектора $\bm{x} \in \ATSP(D')$ определяются линейными уравнениями \eqref{lin1}--\eqref{lin3}, \eqref{lin4} и \eqref{lin5}.
	
	Таким образом, указанный орграф $D'$ на $n = 4 \binom{m}{2} + m$ вершинах определяет многогранник $\ATSP(D')$, аффинно эквивалентный многограннику $\BQP(m)$.
\end{proof}

Для сравнения заметим, что в~\cite{FioriniPokutta:2012} доказано немного более слабое утверждение о том, что $\BQP(m)$ является \emph{проекцией} некоторой грани многогранника $\ATSP(n)$ при $n = 63 m^2 - 57 m$.



%%%%%%%%%%%%%%%%%%%%%%%%%%%%%%%%%%%%%%%%%%%%%%%%%%%%%%%
%
% Многогранники различных вариаций задачи коммивояжера
%
%%%%%%%%%%%%%%%%%%%%%%%%%%%%%%%%%%%%%%%%%%%%%%%%%%%%%%%


\subsection{Многогранники близких задач}
\label{sec:TSPvarious}

В~этом разделе будут рассмотрены еще несколько многогранников, тесно связанных с многогранниками задачи коммивояжера.

\emph{Многогранником гамильтоновых орпутей} называется выпуклая оболочка множества $\HDP(n)$ всех характеристических векторов гамильтоновых орпутей полного орграфа $D=([n],A)$ на $n$ вершинах. 
Его гранью является \emph{многогранник гамильтоновых $s$-$t$ орпутей} $\HDPst(n)$, где $s,t \in [n]$. 
Так как свойства последнего не зависят от выбора концов $s$ и $t$, далее полагаем $s=1$, $t=n$.

Нетрудно заметить, что многогранник гамильтоновых контуров $\ATSP(n-1)$ аффинно эквивалентен многограннику гамильтоновых $s$-$t$ орпутей $\HDPst(n)$.
Для этого достаточно положить 
\[
x_{ij} = \begin{cases}
y_{ij}, & \text{при $i,j \in [n-1]$, $j > 1$, $i\ne j$,}\\
y_{in}, & \text{при $j = 1 < i$,}\\
\end{cases}
\]
где $\bm{x}=(x_{ij}) \in \ATSP(n-1)$, $\bm{y}=(y_{ij}) \in \HDPst(n)$.
Таким образом,
\[
\ATSP \propto_A \HDPst \propto_A \HDP.
\]
Также нетрудно показать, что все три семейства эквивалентны относительно аффинной сводимости.
А именно, многогранник $\HDP(n)$ аффинно эквивалентен $\ATSP(n+1)$.
Чтобы доказать это, заметим, что в любом гамильтоновом контуре в полном орграфе $D = ([n+1], A)$ дуги, инцидентные $(n+1)$-й вершине, однозначно определяются всеми остальными дугами этого контура. 
Таким образом, удаляя дуги, инцидентные $(n+1)$-й вершине, мы устанавливаем взаимно"=однозначное соответствие между гамильтоновыми контурами в орграфе на $n+1$ вершинах и гамильтоновыми орпутями в орграфе на $n$ вершинах.
Остается лишь заметить, что переменные, соответствующие удаляемым дугам, линейно зависят от переменных, связанных с другими дугами:
\begin{align*}
x_{i,n+1} &= 1 - \sum_{j \in [n],\ j \neq i} x_{ij}, \quad i\in[n],\\
x_{n+1,i} &= 1 - \sum_{j \in [n],\ j \neq i} x_{ji}, \quad i\in[n].
\end{align*}
Аналогичная взаимосвязь между многогранниками гамильтоновых циклов и \emph{многогранниками гамильтоновых путей} $\HP(n)$ в неориентированном графе была установлена в~\cite{Queyranne:1993}: $\TSP(n+1) =_A \HP(n)$. Там же эта взаимосвязь была использована для получения новых семейств неравенств, определяющих гиперграни многогранника симметричной задачи коммивояжера.

Кроме того, многогранник гамильтоновых $s$-$t$ орпутей $\HDPst(n)$, очевидно, является гранью многогранника $s$-$t$ орпутей $\Dipath(n)$,
определяемой опорной гиперплоскостью $H(\bm{1}, n-1)$.
Оказывается, семейства $\Dipath$ и $\HDPst$ тоже эквивалентны относительно аффинной сводимости.

\begin{lemma}
$\Dipath(n) \lea \HDPst(k)$ при $k=2n-1$.
\end{lemma}
\begin{proof}
Рассмотрим полный орграф $D = (V,A)$ на $n$ вершинах $V = \{s, v_1, v_2, \dots, v_{n-2}, t\}$.
Пусть $\Dipath(n)$ "--- множество характеристических векторов $s$-$t$ орпутей в $D$.

Построим граф $D'$, содержащий $D$ в качестве индуцированного подграфа и, кроме того, имеющий дополнительные вершины $U = \{u_1, \dots, u_{n-1}\}$ и дуги
\[
\{u_{i}, u_{i+1}\},\ \{u_{i}, v_{i}\},\ \{v_{i}, u_{i+1}\}, 
\quad i \in [n-2],
\]
а также дугу $\{u_{n-1}, s\}$ (см. рис.~\ref{fig:Dipath}).
\begin{figure}[tbh]
	\centering
	\tikzset{small circle/.style={inner sep = 1.5pt,draw,circle}}
	\begin{tikzpicture}[scale=1.0,
	>={Stealth[scale width=0.8]} % Определяем вид стрелок
	]
	\node[small circle] (u1) at (0,2) {};	
	\node[small circle] (u2) at (3,2) {};	
	\node[small circle] (u3) at (6,2) {};
	\node[small circle] (u4) at (9,2) {};
	\node[small circle] (u5) at (12,2) {};
	\node[small circle] (s) at (14,2) {};
	\node[small circle] (t) at (14,0) {};
	\node[small circle] (v1) at (1.5,0) {};	
	\node[small circle] (v2) at (4.5,0) {};	
	\node[small circle] (v4) at (10.5,0) {};	
	\draw[->] (u1) node[above] {$u_{1}$} -- (u2) node[above] {$u_{2}$};
	\draw[->] (u2) -- (u3) node[above] {$u_{3}$};
	\path (u3) -- node {$\dotsc$} (u4);
	\draw[->] (u4) node[above] {$u_{n-2}$} -- (u5) node[above] {$u_{n-1}$};
	\draw[->] (u5) -- (s);
	\draw[->] (s) node[above] {$s\vphantom{u_{n-1}}$} -- (t) node[below] {$t\strut$};
	
	\draw[->] (u1) -- (v1);
	\draw[->] (v1) -- (u2);
	\draw[->] (u2) -- (v2);
	\draw[->] (v2) -- (u3);
	\draw[->] (u4) -- (v4);
	\draw[->] (v4) -- (u5);

	\draw (v1) node[below] {$v_{1}\strut$};
	\draw (v2) node[below] {$v_{2}\strut$};
	\draw (v4) node[below] {$v_{n-2}\strut$};

\newcommand{\inarc}[1]{	
	\draw[<-] (#1) -- +(-30:-0.8);
	\draw[<-] (#1) -- +(0:-0.8);
	\draw[<-] (#1) -- +(30:-0.8);
}
\newcommand{\outarc}[1]{	
	\draw[->] (#1) -- +(30:0.8);
	\draw[->] (#1) -- +(0:0.8);
	\draw[->] (#1) -- +(-30:0.8);
}
	\inarc{v1}
	\inarc{v2}
	\inarc{v4}
	\outarc{v1}
	\outarc{v2}
	\outarc{v4}
	\inarc{t}
%	\outarc{s}
	\draw[->] (s) -- (t);
	\draw[->] (s) -- +(-105:0.8);
	\draw[->] (s) -- +(-135:0.8);
	\draw[->] (s) -- +(-165:0.8);
\end{tikzpicture}
\caption{Орграф $D'$, определяющий грань многогранника $\HDPst(2n-1)$, аффинно эквивалентную многограннику $\Dipath(n)$.}
\label{fig:Dipath}
\end{figure}
Очевидно, множество характеристических векторов гамильтоновых орпутей из $u_1$ в $t$ в графе $D'$ образует грань многогранника $\HDPst(k)$ при $k=2n-1$.
Покажем, что эта грань аффинно эквивалентна $\Dipath(n)$.

Координату характеристического вектора гамильтонова орпути в $D'$, соответствующую дуге $(v, v')$, $v, v' \in V\cup U$, обозначаем $x(v, v')$.
Из описания орграфа $D'$ следует, что для любого гамильтонова орпути в $D'$ должны выполняться равенства
\[
x(u_i, v_i) + x(u_i, u_{i+1}) = 1 = x(v_i, u_{i+1}) + x(u_i, u_{i+1}),\quad i \in [n-2],
\]
и, кроме того, $x(u_{n-1}, s) = 1$.
Из гамильтоновости орпути также следует справедливость соотношений
\[
x(u_i, v_i) = 1 - \sum_{v \in V, \ v \ne v_i} x(v, v_i)  \quad \text{и} \quad
x(v_i, u_{i+1}) = 1 - \sum_{v \in V, \ v \ne v_i} x(v_i, v),
\]
при $i \in [n-2]$.
Таким образом, все координаты характеристического вектора гамильтонова орпути в $D'$ линейно выражаются через координаты $x(v, v')$, $v,v' \in V$.
С другой стороны, легко проверить следующий факт.
Если $H$ "--- гамильтонов орпуть из $u_1$ в $t$ в орграфе $D'$, 
то $\Set*{(v,v')\in H \given v,v'\in V}$ "--- $s$-$t$ орпуть в $D$.
Причем любой $s$-$t$ орпуть в $D$ однозначно определяет соответствующий гамильтонов орпуть в $D'$.
\end{proof}

{\sloppy
Аналогичное утверждение справедливо для многогранников $s$-$t$ путей $\Path(n)$
и гамильтоновых $s$-$t$ путей $\HPst(n)$ в неориентированном графе.

}

\begin{lemma}
	$\HPst(n) \lea \Path(n) \lea \HPst(k)$ при $k=4n-6$.
\end{lemma}
\begin{proof}
Также, как и в случае с орграфом, $\HPst(n)$ является гранью многогранника $\Path(n)$, определяемой опорной гиперплоскостью $H(\bm{1}, n-1)$.

Рассмотрим полный граф $G = (V,E)$ на $n$ вершинах $V = \{s, v_1, v_2, \dots, v_{n-2}, t\}$.
Пусть $\Path(n)$ "--- множество характеристических векторов $s$-$t$ путей в $G$.
По аналогии с доказательством для орграфа, достаточно указать граф $G'$, содержащий $G$ в качестве индуцированного подграфа и определяющий соответствующую грань многогранника $\HPst(4n-6)$.
Этот граф изображен на рис.~\ref{fig:Path}.
\begin{figure}[tbh]
	\centering
	\tikzset{small circle/.style={inner sep = 1.5pt,draw,circle}}
	\begin{tikzpicture}[scale=1.3,
	>={Stealth[scale width=0.8]} % Определяем вид стрелок
	]
	\node[small circle] (u1) at (0.8,2) {};	
	\node[small circle] (u2) at (2.2,2) {};	
	\node[small circle] (um2) at (3,2) {};	
	\node[small circle] (ue2) at (3.8,2) {};	
	\node[small circle] (u3) at (5.2,2) {};
	\node[small circle] (um3) at (6,2) {};
%	\node[small circle] (ue3) at (7,2) {};
%	\node[small circle] (u4) at (8,2) {};
	\node[small circle] (um4) at (8,2) {};
	\node[small circle] (ue4) at (8.8,2) {};
	\node[small circle] (u5) at (10.2,2) {};
	\node[small circle] (um5) at (11.3,2) {};
	\node[small circle] (s) at (12.4,2) {};
	\node[small circle] (t) at (12.4,0.5) {};
	\node[small circle] (v1) at (1.5,0.5) {};	
	\node[small circle] (v2) at (4.5,0.5) {};	
	\node[small circle] (v4) at (9.5,0.5) {};	
	\draw (u1) node[above] {$u_{1}\strut$} -- (u2) node[above] {$u'_{1}\strut$};
	\draw (u2) -- (um2) node[above] {$w_{1}\strut$};
	\draw (um2) -- (ue2) node[above] {$u_{2}\strut$};
	\draw (ue2) -- (u3) node[above] {$u'_{2}\strut$};
	\draw (u3) -- (um3) node[above] {$w_{2}\strut$};
%	\draw (um3) -- (ue3) node[above] {$u'_{3}\strut$};
	\path (um3) -- node {$\dotsc$} (um4);
%	\draw (u4) node[above] {$u_{n-2}\strut$} -- (um4) node[above] {$w_{n-2}\strut$};
	\draw (um4) node[above] {$w_{n-3}\strut$} -- (ue4) node[above] {$u_{n-2}\strut$};
	\draw (ue4) -- (u5) node[above] {$u'_{n-2}\strut$};
	\draw (u5) -- (um5) node[above] {$w_{n-2}\strut$};
	\draw (um5) -- (s);
	\draw (s) node[above] {$s\strut$} -- (t) node[below] {$t\strut$};
	
	\draw (u1) -- (v1);
	\draw (v1) -- (u2);
	\draw (ue2) -- (v2);
	\draw (v2) -- (u3);
	\draw (ue4) -- (v4);
	\draw (v4) -- (u5);
	
	\draw (v1) node[below] {$v_{1}\strut$};
	\draw (v2) node[below] {$v_{2}\strut$};
	\draw (v4) node[below] {$v_{n-2}\strut$};
	
	\newcommand{\inarc}[1]{	
		\draw (#1) -- +(-30:-0.8);
		\draw (#1) -- +(0:-0.8);
		\draw (#1) -- +(30:-0.8);
	}
	\newcommand{\outarc}[1]{	
		\draw (#1) -- +(30:0.8);
		\draw (#1) -- +(0:0.8);
		\draw (#1) -- +(-30:0.8);
	}
	\inarc{v1}
	\inarc{v2}
	\inarc{v4}
	\outarc{v1}
	\outarc{v2}
	\outarc{v4}
	\inarc{t}
	%	\outarc{s}
	\draw (s) -- (t);
	\draw (s) -- +(-105:0.8);
	\draw (s) -- +(-135:0.8);
	\draw (s) -- +(-165:0.8);
	\end{tikzpicture}
	\caption{Граф $G'$, определяющий грань многогранника $\HPst(4n-6)$, аффинно эквивалентную многограннику $\Path(n)$.}
	\label{fig:Path}
\end{figure}
\end{proof}

В~дополнение к уже перечисленным результатам добавим, что $\HPst(n)$ является гранью $\TSP(n)$, лежащей в гиперплоскости $x(s,t) = 1$,
а $\HDPst(n)$ является гранью $\HPst(2n-2)$ (доказательство такое же, как и в лемме~\ref{lem:ATSP2TSP}).

% TSP  не удается свести к HPst, так как каждому циклу при этом соответствуют два пути.
% HPst  не удается свести к HDPst, так как каждому ребру при этом соответствуют две дуги.

Для доказательства эквивалентности (в смысле аффинной сводимости) всех восьми семейств $\ATSP$, $\HDPst$, $\HDP$, $\Dipath$, $\TSP$, $\HPst$, $\HP$, $\Path$ не хватает лишь одного соотношения $\TSP \propto_A \ATSP$, которое установить не удалось. 
С~другой стороны, для доказательства $\TSP \npropto_A \ATSP$ используемые ранее методы не подходят, так как оба семейства содержат в качестве граней все 0/1-многогранники.
В~целом же складывается картина, изображенная на рис.~\ref{fig:TSPall}.

\begin{figure}[tbh]
	\centering
	\tikzset{small circle/.style={inner sep = 1.5pt,draw,circle}}
	\begin{tikzpicture}[scale=1.2,
	>={Stealth[scale width=0.8]} % Определяем вид стрелок
	]
	\tikzstyle{every node}=[rounded corners,text centered,draw=black]
	\node (ATSP) at (0,3) {Гамильтонов контур};
	\node (HDP) at (0,2) {Гамильтонов орпуть};
	\node (HDPst) at (0,1) {Гамильтонов $s$-$t$ орпуть};
	\node (Dipath) at (0,0) {$s$-$t$ Орпуть};
	\draw[<->] (ATSP) -- (HDP);
	\draw[<->] (HDPst) -- (HDP);
	\draw[<->] (HDPst) -- (Dipath);

	\node (SAT) at (5,3) {Выполнимость};
	\draw[->] (SAT) -- (ATSP);

	\node (HPst) at (5,1) {Гамильтонов $s$-$t$ путь};
	\node (Path) at (5,0) {$s$-$t$ Путь};
	\draw[<->] (HPst) -- (Path);
	\draw[->] (HDPst) -- (HPst);
	
	\node (TSP) at (9.5,1) {Гамильтонов цикл};
	\node (HP) at (9.5,0) {Гамильтонов путь};
	\draw[<->] (TSP) -- (HP);
	\draw[->] (HPst) -- (TSP);
	\end{tikzpicture}
\caption{Аффинная сводимость семейств многогранников, связанных с задачей коммивояжера.}
\label{fig:TSPall}
\end{figure}

В~заключение раздела рассмотрим еще одно семейство многогранников.
\emph{Многогранником гамильтоновых графов\label{def:HamPolytope}} будем называть выпуклую оболочку множества $\Ham(n)$ характеристических векторов всех подграфов полного графа на $n$ вершинах, содержащих гамильтонов цикл.
Очевидно, $\TSP(n)$ является гранью многогранника $\Ham(n)$, образованной опорной гиперплоскостью $H(\bm{1}, n)$.
Из следствия~\ref{cor:01toTSP} (см. также~\cite{Billera:1996}) и леммы~\ref{lem:ATSP2TSP} следует, что оба эти семейства содержат в качестве граней все 0/1"~многогранники.
В частности, для каждого $n$ найдется $k$ такое, что $\Ham(n)$ аффинно эквивалентен некоторой грани $\TSP(k)$.
С другой стороны, свед\'{е}ние $\Ham \propto_A \TSP$ возможно, только если $P = \NP$, так как задача распознавания гамильтоновости графа (и, соответственно, вершины многогранника $\Ham(n)$) \NP"~полна, тогда как задача идентификации гамильтонова цикла решается просто. 
%Таким образом, если $P \neq \NP$, то $\Ham(n)$ аффинно эквивалентен некоторой грани $\TSP(k)$, только если $k$ сверхполиномиально относительно $n$.


%%%%%%%%%%%%%%%%%%%%%%%%%%%%%%%%%%%%%%%%%%%%%%%%%%%%%%%
%
% Булевы многогранники степени $p$
%
%%%%%%%%%%%%%%%%%%%%%%%%%%%%%%%%%%%%%%%%%%%%%%%%%%%%%%%

\section{\texorpdfstring{Булевы многогранники степени $p$}{Булевы многогранники степени p}}
\label{sec:BQP-power}

Как известно, булевы квадратичные многогранники 3"~смежностны, но не 4"~смежностны~\cite{Deza:1992}.
Более того, многие 0/1-многогранники, изучаемые в комбинаторной оптимизации, являются как минимум 2"~смежностными~\cite[p. 366]{Henk:2004}. %Page 366
Например, Ш. Онн установил следующий факт.

\begin{theorem}[Онн~\cite{Onn:1993}]
	Пусть $n = \lambda_1 + \ldots + \lambda_k$, где $n, \lambda_1, \ldots, \lambda_k \in \N$, 
	$k \ge 2$, и $\lambda_1 \ge \ldots \ge \lambda_k \ge k^2$.
	Тогда многогранник Юнга $P(\lambda_1, \ldots, \lambda_k)$ 
	является $\left\lfloor \frac{k^2}2 \right\rfloor$"~смежностным.
\end{theorem}

{\sloppy
Известно~\cite{Barvinok:1988, Onn:1993}, что многогранники некоторых задач комбинаторной оптимизации 
(например, коммивояжер, независимое множество, паросочетание максимального веса, изоморфизм графов)
являются образами многогранников Юнга при линейном отображении.

}

Вообще, для достаточно больших значений размерности, 
$k$"~смежностные многогранники встречаются чаще, чем какие либо другие.
Это подтверждается следующими фактами.

Пусть случайные вектора $x_1, x_2, \ldots, x_n \in \{0,1\}^d$ распределены независимо и равномерно.
Если повторения допустимы, то многогранник $P_{d,n}$ определяется так:
$P_{d,n} = \conv\{x_1, x_2, \ldots, x_n\}$ \cite{Gillmann:2006}.
Для случая, когда повторения невозможны, будем пользоваться обозначением
$Q_{d,n} = \conv\{x_1, x_2, \ldots, x_n\}$. 

\begin{theorem}[Бондаренко--Бродский~\cite{Bondarenko:2008}]
	Если $n = O(2^{d/6})$, то вероятность $\Pr(Q_{d,n}\mbox{ 2"~смежностен})$ 
	стремится к 1 при $d \rightarrow \infty$.
\end{theorem}

\begin{theorem}[Гиллман~\cite{Gillmann:2006}]
	Для каждого $k \ge 2$ существует константа $c > 0$ и $\varepsilon > 0$ %, $\varepsilon < 0.001$,
	такие, что
	\[
	\Pr(P_{d,n}\mbox{ $k$"~смежностен}) \ge 1 - 2^{-c d}, \quad \text{при }n \le 2^{\varepsilon d}.
	\] 
\end{theorem}

Ниже для каждого $k\in\N$ будет описано специальное семейство $k$"~смежностных 0/1-многогранников.
Число $2^{\Theta\left(d^{(2\left\lceil k/3\right\rceil)^{-1}}\right)}$ их вершин сверхполиномиально относительно размерности $d$ многогранника. 
Также будет показано, что эти семейства аффинно сводятся к семейству булевых квадратичных многогранников.


Рассмотрим множество 
\[
\Tensor(n) = \Set*{\bm{y}\otimes \bm{y} \given \bm{y} \in \{0,1\}^n},
\]
где $\bm{y}\otimes \bm{y} \in \{0,1\}^{n\times n}$ "--- тензорное произведение.
Для координат вектора $\bm{x} \in \Tensor(n)$ будем использовать обозначения $x(i) = y_i y_i$, $i \in [n]$, и $x(j,k) = y_j y_k$, $j,k\in [n]$, $j \neq k$, то есть
\begin{equation}
\label{eq:tensor1}
x(j,k) = x(k,j) = x(j) x(k).
\end{equation}
Очевидно, множество $\Tensor(n)$ линейно изоморфно булеву квадратичному многограннику $\BQP(n)$, вершины которого не содержат координаты $x(j,k)$ с индексами $j > k$.
По этой причине в некоторых публикациях булевым квадратичным (корреляционным) многогранником называется выпуклая оболочка множества $\Tensor(n)$ (см., например, \cite{FioriniPokutta:2015}).

Увеличивая число множителей в тензорном произведении, введем в рассмотрение множество
\[
\Tensor(n,p) = \Set*{\bm{y}\otimes \dots \otimes\bm{y} \in \{0,1\}^{n^p} \given \bm{y} \in \{0,1\}^n},
\qquad p \in \N, \quad n \ge p.
\]
Для координат вектора $\bm{x} \in \Tensor(n,p)$ будем использовать обозначения $x(i) = y_i \dots y_i$, $i \in [n]$, $x(j,k) = y_j \dots y_j y_k \dots y_k$, $j,k\in [n]$, $j \neq k$, и~т.\,д.
Ясно, что среди них будет много совпадающих по значению (см., например, уравнение~\eqref{eq:tensor1}).
Избавившись от совпадающих координат, введем в рассмотрение линейно изоморфный данному \emph{булев многогранник степени $p$}. Множество его вершин обозначаем $\BPP(n,p)$.
Координаты вектора $\bm{x} \in \BPP(n,p)$ будем индексировать непустыми подмножествами множества $[n]$, состоящими из не более, чем $p$ элементов.

Так, например, $\BPP(n,3)$ состоит из векторов $\bm{x} \in \{0,1\}^{d}$, $d = \binom{n}{1} + \binom{n}{2} + \binom{n}{3}$, координаты которых удовлетворяют уравнениям
\begin{align*}
x(i,j) &= x(i) x(j), && 1\le i < j \le n,\\
x(i,j,k) &= x(i) x(j) x(k), && 1\le i < j < k \le n.
\end{align*}

Перечислим некоторые свойства булева многогранника степени $p$.
Прежде всего, $\BPP(n,1)$ является $n$"~мерным 0/1-кубом, $\BPP(n,2)$ "--- булев квадратичный многогранник $\BQP(n)$,
$\BPP(p,p)$ "--- симплекс на $2^p$ вершинах.
Размерность многогранника $\BPP(n,p)$, при $n \ge p$, совпадает с числом координат.
Это следует из того, что его вершины, удовлетворяющие условию $\sum_{i\in[n]} x(i) \le p$, аффинно независимы.


\begin{lemma}[\cite{Maksimenko:2013k}]
\label{MaksLemma1} 
Многогранник $\BPP(n,p)$ является $s$"~смежностным при
\begin{equation*}
s \le p + \left\lfloor p / 2 \right\rfloor.
%s \le 3 \left\lfloor p / 2 \right\rfloor + (p\mod 2).
%\label{MaksPower} 
\end{equation*}
\end{lemma}

\begin{proof}
Заметим, что $\BPP(n,1)$ 1"~смежностен, а $\BPP(n,2)$ 3"~смежностен \cite{Deza:1992}.
Следовательно, для каждых трех различных вершин 
$x^1, x^2, x^3$ многогранника $\BPP(n,2)$ существует билинейная функция 
\begin{equation*}
f_{x^1 x^2 x^3}(x) = b + \sum_{1\le i \le j \le n} a_{ij} x(i) x(j), \quad b, a_{ij} \in \R,
%\label{MaksPower} 
\end{equation*}
такая, что $f_{x^1 x^2 x^3}(x^1) = f_{x^1 x^2 x^3}(x^2) = f_{x^1 x^2 x^3}(x^3) = 0$ 
и $f_{x^1 x^2 x^3}(x) > 0$ для любой другой вершины $x$ этого многогранника.
	
Проделаем то же самое для $\BPP(n,4)$.
Для любой шестерки различных вершин $x^1$, $x^2$, $x^3$, $x^4$, $x^5$, $x^6$ можно построить две функции
\begin{equation*}
\begin{aligned}
f_{x^1 x^2 x^3}(x) &= b + \sum_{1\le i \le j \le n} a_{ij} x(i)x(j),\\
f_{x^4 x^5 x^6}(x) &= d + \sum_{1\le i \le j \le n} c_{ij} x(i)x(j).
\end{aligned}
%\label{MaksPower} 
\end{equation*}
Ясно, что функция
\[
F(x) = f_{x^1 x^2 x^3}(x) \cdot f_{x^4 x^5 x^6}(x)
\]
равна 0 при $x^m$, $m = 1, \ldots, 6$, и $F(x) > 0$ для любой другой вершины $x$ многогранника $\BPP(n,4)$.
С другой стороны, $F(x)$ "--- полином 4-й степени от переменных $x(i)$, $i \in [n]$,
%\[
%F(x) =  g + \sum_{1\le i \le j \le k \le l \le n} e_{ijkl} x(i)x(j)x(k)x(l),
%\]
но при этом линеен относительно координат вектора $\bm{x} \in \BPP(n,4)$, совпадающих со значениями соответствующих произведений вида $x(i)x(j)$ или $x(i)x(j)x(k)x(l)$.
Таким образом, $F(x)$ определяет опорную гиперплоскость для $\BPP(n,4)$, и $\BPP(n,4)$ является 6"~смежностным.
	
Действуя тем же способом, можно доказать, что $\BPP(n,p)$ является $(3p/2)$"=смежностным для четных $p$.
	
Для нечетных $p$ достаточно заметить, что для каждой вершины $x^0$ куба $\BPP(n,1)$	несложно описать линейную функцию 
\begin{equation*}
f_{x^0}(x) = b + \sum_{1\le i \le j \le n} a_i x(i), \quad b, a_i \in \R,
%\label{MaksPower} 
\end{equation*}
такую, что $f_{x^0}(x^0) = 0$ и $f_{x^0}(x) > 0$ для любой другой вершины $x$ куба $\BPP(n,1)$.
\end{proof}


% Причем в $\BQP(n,p)$ есть $2^p$ вершин, не являющихся гранью.
% Для этого достаточно рассмотреть $\BQP(p+1,p)$ 
% и множество его вершин поделить на два равномощных подмножества: 
% с четным и нечетным числом единиц на главной диагонали.
% В~каждом из них будет ровно по $2^p$ вершин 
% и суммы этих двух частей будут совпадать.

\begin{remark}
\label{MaksRemark}
$\BPP(n,p)$ является гранью многогранника $\BPP(n+1,p)$, определяемой опорной гиперплоскостью $x(n+1) = 0$. 
Следовательно, $\BPP(n,p)$ --- грань многогранника $\BPP(k,p)$ для всех $k > n$.
\end{remark}

\begin{lemma}[\cite{Maksimenko:2013k}]
\label{MaksLemma2}
$\BPP(n,p)$ не является $2^p$-смежностным при $n \ge p+1$.
\end{lemma}
\begin{proof}
Из замечания \ref{MaksRemark} следует, что достаточно доказать утверждение леммы для $n=p+1$.
	
%For $p=1,2$ the statement is true.
Покажем, что $\BPP(p+1,p)$ не $2^p$-смежностен.
	
Пусть
$$
S(x) = \sum^{p+1}_{i = 1} x(i)
$$ 
есть сумма <<основных>> координат вектора $x\in\BPP(p+1,p)$.
Множество $X$ всех вершин многогранника $\BPP(p+1,p)$ разобъем на $p+2$ подмножеств:
$$
X(k) = \Set*{x\in X \given S(x) = k}, \quad k = 0,1,\ldots, p+1.
$$
Пусть
$$
Y = \bigcup_{\text{$k$ четно}} X(k), \quad \text{и} \quad
Z = X \setminus Y.
$$
Очевидно, $|X(k)| = \binom{p+1}{k}$, $|Y| = |Z| = |X|/2 = 2^p$.

Покажем, что
\begin{equation}
\label{MaksParity}
\sum_{x\in Y} x = \sum_{x\in Z} x.
\end{equation}
Если это верно, то $Y$ и $Z$ не образуют грани многогранника $\BPP(p+1,p)$.
	
Проверим равенство \eqref{MaksParity} покоординатно.
Пусть $\bar{r}$ "--- наибольшее четное, не превосходящее $p+1$, $\bar{s} = 2p+1 - \bar{r}$ "--- наибольшее нечетное, не превосходящее $p+1$.
Для $i \in [p+1]$ имеем
	\begin{equation}
	\label{MaksY1}
	\begin{aligned}
	\sum_{x\in Y} x(i) &= \sum_{x\in X(0)} x(i) + \sum_{x\in X(2)} x(i) + \sum_{x\in X(4)} x(i) + \ldots + \sum_{x\in X(\bar{r})} x(i) \\
	%                  &= 0 + \binom{p+1}{2} \frac{2}{p+1} + \binom{p+1}{4} \frac{4}{p+1} + \ldots + \binom{p+1}{\bar{r}} \frac{\bar{r}}{p+1}\\
	&= 0 + \binom{p}{1} + \binom{p}{3} + \ldots + \binom{p}{\bar{r}-1}
	\end{aligned}
	\end{equation}
	и
	\begin{equation}
	\label{MaksZ1}
	\begin{aligned}
	\sum_{x\in Z} x(i) &= \sum_{x\in X(1)} x(i) + \sum_{x\in X(3)} x(i) + \ldots + \sum_{x\in X(\bar{s})} x(i) \phantom{{} + \sum_{x\in X(0)} x(i)} \\
	%                  &= \binom{p+1}{1} \frac{1}{p+1} + \binom{p+1}{3} \frac{3}{p+1} + \ldots + \binom{p+1}{\bar{s}} \frac{\bar{s}}{p+1}\\
	&= \binom{p}{0} + \binom{p}{2} + \ldots + \binom{p}{\bar{s}-1}.
	\end{aligned}
	\end{equation}
	Из бинома Ньютона
	$$
	\sum^{p}_{k=0} (-1)^k \binom{p}{k} = 0
	$$
	следует, что \eqref{MaksY1} и \eqref{MaksZ1} равны.
	
	Продолжая тем же способом, рассмотрим координату $x(i_1, i_2, \ldots, i_m)$
	с $m$ различными индексами $1\le i_1 < i_2 < \ldots < i_m \le p+1$, $1\le m \le p$.
	Как и выше,
	\begin{equation*}
	\sum_{x\in Y} x(i_1, i_2, \ldots, i_m) - \sum_{x\in Z} x(i_1, i_2, \ldots, i_m) = (-1)^m \sum^{p+1-m}_{k=0} (-1)^k \binom{p+1-m}{k} = 0 \enspace .
	\end{equation*}
	%\qed
\end{proof}


Объединяя леммы \ref{MaksLemma1} и \ref{MaksLemma2}, получаем следующее утверждение.


\begin{theorem}[\cite{Maksimenko:2013k}]
\label{MaksT1} 
Пусть $s$ наибольшее целое, для которого $\BPP(n,p)$, $n > p$, является $s$"~смежностным.
Тогда
\[
%\label{MaksPower} 
p + \left\lfloor p / 2 \right\rfloor \le s < 2^p.
\]
\end{theorem}


Перечислим несколько очевидных замечаний:

\textbf{1.} Куб $\BPP(k,1)$, $k \ge 2$, не является 2-смежностным.
Следовательно, для $k \ge 2$, $\BPP(k,1)$ не может быть гранью 
многогранника $\BPP(n,p)$ при $n \ge p \ge 2$.

\textbf{2.} $\BPP(k,1)$ не имеет 2-смежностных граней за исключением 1-граней (ребер).
Следовательно, для $n \ge p \ge 2$, $\BPP(n,p)$ не может быть гранью куба $\BPP(k,1)$.

\textbf{3.} Булев квадратичный многогранник $\BPP(k,2)$ не является 4-смежностным.
Это означает, что $\BPP(k,2)$ не может быть гранью многогранника $\BPP(n,p)$ для $n \ge p \ge 3$.

Покажем теперь, что для четных $p \ge 4$, $\BPP(k,p)$ является гранью многогранника $\BQP(n)$ при некотором $n = \Theta(k^{\lceil p/2\rceil})$.

Для чисел $k,m \in \N$, $k \ge 2m$, определим $H(k,m)$ следующим образом:
\begin{equation}
H(k,m) = \binom{k}{m} + \binom{k}{\left\lceil m / 2 \right\rceil} 
+ \binom{k}{\left\lceil \left\lceil m / 2 \right\rceil / 2 \right\rceil} 
+ \ldots + \binom{k}{1}.
\notag%\label{} 
\end{equation}
Заметим, что $H(k,m) \ge \binom{k}{m}$, и, при постоянном $m$, 
$H(k,m) \sim \binom{k}{m}$ при $k \rightarrow \infty$.
Однако, для малых значений $k$, $H(k,m)$ может существенно отличаться от $\binom{k}{m}$. 
Вычисления показывают, что максимум для соотношения $H(k,m) / \binom{k}{m}$ равен $41 / 20$ и достигается при $k = 2m = 6$.


\begin{theorem}[\cite{Maksimenko:2013k}]
\label{MaksT2} 
$\BPP(k, 2m) \lea \BQP(n)$ для любых $m \in \N$, $k \ge 2m$ и $n \ge H(k,m)$.
\end{theorem}

\begin{proof}
Покажем, что $\BPP(k,4)$ линейно изоморфен грани булева квадратичного многогранника $\BQP(n)$	при $n = \binom{k}{2} + \binom{k}{1}$.
Вместо множества $[n]$ для индексирования координат вектора $x \in \BQP(n)$ будем использовать множество 
\[
S = [k] \cup \Set{ij \given 1 \le i < j \le k},
\]
где $ij$ следует воспринимать как множество $\{i,j\}$.
%Предположим, кроме того, что все элементы этого множества как-нибудь упорядочены, чтобы можно было выполнять сравнения вида $s < t$ для $s,t\in S$.
Исходя из этого, для координат вектора $x \in \BQP(n)$ будем использовать обозначения $x(s)$, $s\in S$, и
\[
x(s,t) \stackrel{\mathrm{def}}{\equiv} x(t,s), \qquad s,t\in S, \quad s \neq t.
\]
Согласно определению булева квадратичного многогранника,
\[
x(s,t) = x(s) x(t).
\]

Рассмотрим множество $F$ векторов $x \in \BQP(n)$, удовлетворяющих ограничениям
\begin{equation}
\label{eq:MaksCond}
x(i,j) = x(ij) = x(i,ij) = x(j,ij), \qquad 1\le i < j \le k.
\end{equation}
В~частности, для $x\in F$ выполнено 
\[
x(ij,lm) = x(ij) x(lm) = x(i) x(j) x(l) x(m), \qquad  i,j,l,m \in [k], \quad i < j, \quad l < m.
\]
Покажем, что $F$ является гранью $\BQP(n)$.
	
Воспользуемся тем, что все вершины $\BQP(n)$ удовлетворяют неравенствам
\[
x(s,t) \le x(s) \quad \text{и} \quad x(s,t) \le x(t), \qquad s,t\in S.
\]
Каждое такое неравенство определяет грань многогранника $\BQP(n)$.
Например,
\[
F'_{ij} = \Set*{ x \in \BQP(n) \given x(i,ij) = x(ij), \ x(j,ij) = x(ij) }
\]
является гранью многогранника $\BQP(n)$, при $1\le i < j \le k$.
Так как $x(i,ij) \le x(i)$ и $x(j,ij) \le x(j)$, то
\[
x(ij) \le x(i) \quad \text{и} \quad x(ij) \le x(j), \quad \forall x \in F'_{ij}.
\]
Учитывая, что $x(i,j) = x(i) x(j)$, получаем
\[
x(ij) \le x(i,j), \quad \forall x \in F'_{ij}.
\]
Следовательно, 
\[
F_{ij} = \Set*{x \in F'_{ij} \given x(ij) = x(i,j)}
\]
является гранью $F'_{ij}$, а также гранью $\BQP(n)$.
Более того, \eqref{eq:MaksCond} выполнено для всех $x \in F_{ij}$.
	
Таким образом,
\[
F = \bigcap_{1\le i < j \le k} F_{ij}
\]
является гранью $\BQP(n)$, и $F$ линейно изоморфна многограннику $\BPP(k,4)$.
В~частности, для $y \in \BPP(k,4)$ можно положить
\[
\begin{aligned}
	y(i) &= x(i),    &\quad &1 \le i \le k,\\
	y(i,j) &= x(ij),   &\quad &1 \le i < j \le k,\\
	y(i,j,l) &= x(i,jl), &\quad &1 \le i < j < l \le k,\\
	y(i,j,l,m) &= x(ij,lm), &\quad &1 \le i < j < l < m \le k.
\end{aligned}
\]
	
Действуя тем же способом не трудно проверить, что $\BPP(k, 2m)$ линейно изоморфен некоторой грани $\BQP(n)$ для $n = H(k,m)$, $k \ge 2m$, $m \in \N$.
\end{proof}

Объединяя это утверждение с теоремой \ref{MaksT1} получаем

\begin{corollary} 
Для любого $k \in \N$ и $n \ge 2^{2\cdot \lceil k/3\rceil}$ булев квадратичный многогранник $\BQP(n)$ имеет $k$-смежностную грань со сверхполиномиальным числом $2^{{\Theta}\left( n^{1 / {\left\lceil k/3\right\rceil}}\right)}$ вершин.
\end{corollary} 

% The bound $k \le 3 \left\lfloor \frac{\log_2 (1.5n)}{2}\right\rfloor$ can be increased
% to $k \le 3 \left\lfloor \log_2 n\right\rfloor$


%%%%%%%%%%%%%%%%%%%%%%%%%%%%%%%%%%%%%%%%%%%%%%%%%%%%%%%
%
%  Задача о кратчайшем пути и задача о назначениях
%
%%%%%%%%%%%%%%%%%%%%%%%%%%%%%%%%%%%%%%%%%%%%%%%%%%%%%%%


\section{\texorpdfstring{Конусные разбиения для задачи о кратчайшем орпути\\ и для задачи о назначениях}{Конусные разбиения для задачи о кратчайшем орпути и для задачи о назначениях}}
\label{sec:ShortPath2Assignment}

В~этом разделе мы рассмотрим задачу поиска кратчайшего орпути из вершины $s$ в вершину $t$ в полном реберно"=взвешенном ориентированном графе $D = (V,A)$, $s,t \in V = [n]$.
В~общем случае эта задача является NP-трудной~\cite{Garey:1982}, 
а гранью её многогранника $\Dipath(n)$ являтся многогранник гамильтоновых контуров $\ATSP(n-1)$ (см. раздел~\ref{sec:TSPvarious}).
Если же в~$D$ отсутствуют контуры отрицательной длины, то задача становится полиномиально разрешимой~\cite[sec.~8.3]{SchrijverCO:2003}, а соответствующий полиэдр \(\ShortP(n)\) имеет компактное описание (см. раздел~\ref{subsec:polyhedra}).
Чтобы пояснить естественность такого ограничения, рассмотрим следующий пример.
Предположим, что на некоторых участках дорог транспортное средство движется под гору. Тогда оно может не только не тратить энергию (топливо), но и накапливать возникающие излишки кинетической энергии. Тогда, с точки зрения энергетических потерь, эти участки дорог имеют отрицательный вес. Но затраты энергии, необходимые для прохождения замкнутого пути (контура), как гласит закон сохранения энергии, всегда будут положительны.

С целью унификации рассуждений, перейдем к рассмотрению задачи на максимум.
Ниже будем изучать свойства задачи о длиннейшем пути в реберно"=взвешенном орграфе, все контуры которого имеют отрицательный вес.

\begin{theorem}[\cite{MaksimenkoDiss:2004}]
Конусное разбиение задачи о длиннейшем пути в орграфе с контурами отрицательного веса аффинно сводится к конусному разбиению задачи о назначениях.
\end{theorem}
\begin{proof}
\newcommand{\us}{u\lefteqn{'}}
Для каждой задачи о длиннейшем пути в орграфе $D = (V, A)$, $V = \{v_1, \dots, v_{n+1}\}$, $n\in \N$, рассмотрим задачу о назначениях в двудольном графе $G = (W,E)$ с долями $U = \{u_1, \dots, u_n\}$ и $U' = \{\us_1, \dots, \us_n\}$.
Таким образом мы фиксируем преобразование кода первой задачи в код второй задачи (см. определение~\ref{def:AffReductionRestriction}).

Не уменьшая общности, будем предполагать, что искомый орпуть должен начинаться в вершине $s = v_1$ и заканчиваться в $t = v_{n+1}$. 
Соответственно, дуг, входящих в $s$ или исходящих из $t$, в орграфе $D$ нет.
Кроме того, введем следующие обозначения. Координата $c_{i,j}$ целевого вектора $\bm{c} \in \Q^{A}$ для задачи о длиннейшем пути будет равна весу дуги, ведущей из $v_i$ в $v_j$, $i \ne n+1$, $j \ne 1$.
Координата $b_{i,j}$ целевого вектора $\bm{b} \in \Q^{E}$ для задачи о назначениях будет равна весу ребра $\{u_i, \us_j\}$.

Аффинное отображение $\alpha\from \Q^{A} \to \Q^{E}$ определим следующим образом:
\[
b_{i,j} = \begin{cases}
0, & \text{если } i = j \ne 1,\\
c_{i,n+1}, & \text{если } j = 1,\\
c_{i,j}, & \text{в остальных случаях}.
\end{cases}
\]
Точно таким же образом установим взаимно"=однозначное соответствие между множеством дуг $A$ и множеством ребер $E \setminus \{\{u_2, \us_2\},\dots,\{u_n, \us_n\}\}$.
Тогда каждому простому орпути в орграфе $D$ будет поставлено в соответствие некоторое совершенное паросочетание в графе $G$, имеющее точно такой же вес.
(Если вершина $v_i$ не принадлежит орпути, то соответствующее паросочетание содержит ребро $\{u_i, \us_i\}$ нулевого веса.)

Временно нарушая общепринятое определение, \emph{контуром} в двудольном графе $G$ назовем множество ребер вида
\[
\bigl\{\{u_{i_1}, \us_{i_2}\},\{u_{i_2}, \us_{i_3}\},\dots,\{u_{i_{n-1}}, \us_{i_n}\},\{u_{i_n}, \us_{i_1}\}\bigr\},
\]
где $k \ge 2$, а $\{i_1, \dots, i_k\}$ "--- некоторое упорядоченное подмножество множества $[n]$.
Очевидно, любое паросочетание в $G$ представляет собой набор таких контуров и, возможно, ребер вида $\{u_i, \us_i\}$.
Причем каждый контур, проходящий через вершины $u_1$ и $\us_1$, соответствует некоторому орпути в $D$.
Контуры в $G$, не проходящие через $u_1$ и $\us_1$, соответствуют контурам в $D$.

Так как граф $D$ содержит только отрицательные контуры, то оптимальное паросочетание в графе $G$ с весами $\bm{b}$ может содержать только один контур, проходящий через $u_1$ и $\us_1$, и соответствующий орпути в $D$ с тем же суммарным весом дуг.
\end{proof}

\begin{corollary}
\label{cor:Short2Assign}
Граф полиэдра $\ShortP(n+1)$ является подграфом графа многогранника $\Birk(n)$, $n \in \N$.
\end{corollary}

%%%%%%%%%%%%%%%%%%%%%%%%%%%%%%%%%%%%%%%%%%%%%%%%%%%%%%%
%
% End of section
%
%%%%%%%%%%%%%%%%%%%%%%%%%%%%%%%%%%%%%%%%%%%%%%%%%%%%%%%
