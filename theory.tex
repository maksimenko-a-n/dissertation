% !TeX encoding = UTF-8 Unicode
% !TEX root = MaksimenkoThesis.tex

%%%%%%%%%%%%%%%%%%%%%%%%%%%%%%%%%%%%%%%%%%%%%%%%%%%%%%%%%%
%
%     Аффинная сводимость
%
%%%%%%%%%%%%%%%%%%%%%%%%%%%%%%%%%%%%%%%%%%%%%%%%%%%%%%%%%%
\chapter{Аффинная сводимость}
\label{chap:AffTheory}
%\begin{flushright}
%Всё познается в сравнении.
%\end{flushright}
%\medskip

В~этой главе описан метод аффинной сводимости, активно используемый в следующих двух главах.
В~разделе~\ref{sec:AffProblems} вводится определение аффинной сводимости задач, служащее основой для различных его модификаций в последующих разделах.
В~разделе~\ref{sec:Cones} приводятся определения конусного разбиения пространства исходных данных задачи и аффинной сводимости таких разбиений.
Далее, в разделе~\ref{sec:AffCompare} вводится естественный способ сравнения многогранников, позволяющий сравнивать их различные комбинаторно"=геометрические характеристики, и приводятся примеры его применения для нескольких комбинаторных семейств многогранников.
В~последнем разделе главы вводится определение аффинной сводимости семейств многогранников (наиболее часто используемое в следующей главе), приводятся относительно простые примеры его использования и устанавливается связь между аффинной сводимостью семейств многогранников и аффинной сводимостью конусных разбиений пространств исходных данных задач. 

\section{Аффинная сводимость задач}
\label{sec:AffProblems}

В соответствии с замечанием~\ref{def:Psize}, сложность линейной задачи комбинаторной оптимизации $(L,d,g)$ будем измерять относительно размера $\size(I) + d(I)$, $I \in L$.
%Согласно определению \ref{def:LCOP}, входные данные линейной задачи комбинаторной оптимизации состоят из кода $I \in L$ (определяющего множество допустимых решений $X(I)$), и целевого вектора~$\bm{c} \in \Z^d$, $d = d(I)$.

\begin{definition}[аффинная сводимость задач]
\label{def:AffReduction}
Рассмотрим две массовые линейные задачи комбинаторной оптимизации $(L,d,g)$ и~$(L',d',g')$.
Множество допустимых решений первой задачи обозначаем $X = X(I) \subset \Z^d$,
второй задачи "--- $X' = X'(I') \subset \Z^{d'}$.
Будем говорить, что задача $(L,d,g)$ \emph{аффинно сводится} к задаче $(L',d',g')$, если существуют вычислимые за полиномиальное время (относительно размера первой задачи):
\begin{enumerate}
	\item 
	Преобразование $\tau \from L \to L'$.
	\item 
	Алгоритм построения для каждого кода $I \in L$ аффинного отображения 
	\[
	\alpha\from \R^d \to \R^{d'}, \qquad \text{где } d = d(I), \  d' = d'(\tau(I)).
	\]
	\item 
	%Сюръекция
	Функция $\beta\from Y \to X$, где $X = X(I)$, а $Y$ "--- это множество всех таких $\bm{y} \in X' = X'(\tau(I))$, для каждого из которых найдется целевой вектор $\bm{c} \in \R^d$ такой, что $(\alpha(\bm{c}))^T \bm{y} \ge (\alpha(\bm{c}))^T \bm{x'}$ для всех $\bm{x'} \in X'$.
	%\[Y = \Set*{\bm{y} \in X'(I') \given \exists \bm{c} \in \R^d \ \bm{c'} = \alpha(\bm{c}), \ \forall \bm{x'} \in X'(I')\ \bm{c'}^T \bm{y} \ge \bm{c'}^T \bm{x'}}.\]
		
	Причем для любого $\bm{y} \in Y$ и любого $\bm{c} \in \R^d$ выполнено
	\[
	\Bigl(\forall \bm{x'} \in X' \quad (\alpha(\bm{c}))^T \bm{y} \ge (\alpha(\bm{c}))^T \bm{x'}\Bigr) \iff \Bigl(\forall \bm{x} \in X \quad \bm{c}^T \beta(\bm{y})  \ge \bm{c}^T \bm{x}\Bigr).
	\]
	% Функция $\beta$  не обязана быть сюръекцией по следующим причинам. Пусть $x$ таков, что $x \ne \beta(y)$ ни для какого $y \in Y$. Тогда возможны две ситуации. 1. Если $x$ является в первой задаче единственно оптимальным при некотором целевом $\bm{c}$, то правая часть вышестоящей формулы (с $\iff$) не выполняется никогда, а левая часть будет выполняться при некотором $y$. 2. Если же $x$ не является единственно оптимальным ни при каком $c$, то им можно пренебречь.
\end{enumerate}
\end{definition}
%Ликбез: 1) сюръекция -- отображение на, или же функция, принимающая все возможные значения; 2) инъекция -- каждый образ соответствует (не более чем) единственному прообразу, или, если образы совпадают, то и прообразы совпадают.

Непосредственно из определения следует, что аффинная сводимость задач влечет их полиномиальную сводимость, так как преобразование $\tau$ и аффинное отображение $\alpha$ преобразуют входные данные первой задачи во входные данные второй задачи за полиномиальное время, а функция $\beta$ преобразует оптимальное решение второй задачи в оптимальное решение первой задачи.

Корректировка определения аффинной сводимости задач для случаев с ограничением на множество целевых векторов (например, для задачи о кратчайшем пути) очевидна.
А именно, рассмотрим пару задач из определения~\ref{def:AffReduction} со следующими ограничениями. 
Предположим, что целевой вектор первой задачи должен удовлетворять предикату $h\from \R^d \to \{\text{ложь},\text{истина}\}$, где $h$ зависит от кода $I$, а целевой вектор второй задачи "--- предикату $h'$. 

\begin{definition}[аффинная сводимость задач с ограничением]
\label{def:AffReductionRestriction}
{\sloppy
Для линейной задачи комбинаторной оптимизации $(L,d,g)$ с ограничением $h$ введем обозначение для множества исходных данных:
\[
Q = Q(I) \coloneqq \Set*{\bm{c} \in \R^d \given h(\bm{c})}.
\]
А множество исходных данных задачи $(L',d',g')$ с ограничением $h'$ обозначаем $Q' = Q'(I')$.
}

Будем говорить, что задача $(L,d,g)$ с ограничением $h$ \emph{аффинно сводится} к задаче $(L',d',g')$ с ограничением $h'$, если существуют вычислимые за полиномиальное %(относительно размера первой задачи) 
время:
	\begin{enumerate}
		\item 
		Преобразование $\tau \from L \to L'$.
		\item 
		Алгоритм построения для каждого кода $I \in L$ аффинного отображения 
		\[
			\alpha\from Q \to Q', \qquad \text{где } Q = Q(I), \  Q' = Q'(\tau(I)).
		\]
		\item 
		%Сюръекция
		Функция $\beta\from Y \to X$, где $X = X(I)$, а $Y$ "--- это множество всех таких $\bm{y} \in X' = X'(\tau(I))$, для каждого из которых найдется целевой вектор $\bm{c} \in Q$ такой, что $(\alpha(\bm{c}))^T \bm{y} \ge (\alpha(\bm{c}))^T \bm{x'}$ для всех $\bm{x'} \in X'$.
		%\[Y = \Set*{\bm{y} \in X'(I') \given \exists \bm{c} \in \R^d \ \bm{c'} = \alpha(\bm{c}), \ \forall \bm{x'} \in X'(I')\ \bm{c'}^T \bm{y} \ge \bm{c'}^T \bm{x'}}.\]
		
		Причем для любого $\bm{y} \in Y$ и любого $\bm{c} \in Q$ выполнено
		\[
		\Bigl(\forall \bm{x'} \in X' \quad (\alpha(\bm{c}))^T \bm{y} \ge (\alpha(\bm{c}))^T \bm{x'}\Bigr) \iff \Bigl(\forall \bm{x} \in X \quad \bm{c}^T \beta(\bm{y})  \ge \bm{c}^T \bm{x}\Bigr).
		\] 
	\end{enumerate}
\end{definition}

Далее нас в первую очередь будут интересовать свойства многогранников задач и связанных с ними геометрических конструкций. Поэтому данное определение будет модифицировано в соответствии с целями изучения соответствующих геометрических объектов.

В~некотором смысле посредником между аффинной сводимостью задач и аффинной сводимостью многогранников является аффинная сводимость конусных разбиений пространств исходных данных задач.

%%%%%%%%%%%%%%%%%%%%%%%%%%%%%%%%%%%%%%%%%%%%%%%%%%%%%%%%%%
%
%     Конусные разбиения
%
%%%%%%%%%%%%%%%%%%%%%%%%%%%%%%%%%%%%%%%%%%%%%%%%%%%%%%%%%%
\section{Конусное разбиение пространства исходных данных}
\label{sec:Cones}

Пусть $P$ "--- полиэдр в $\R^d$. 
\emph{Конус грани} $F$ полиэдра $P$ определим следующим образом:
\[
K(F) \coloneqq \Set*{\bm{c}\in\R^d \given  \bm{c}^T \bm{x} \ge \bm{c}^T \bm{y}, \ \forall \bm{x} \in F, \ \forall \bm{y} \in P}.
\]
В~частности, $\bm{0} \in K(F)$ для любой грани $F$.
Заметим также, что если $P$ является многогранником, то при замене $P$ и его грани $F$ соответствующими множествами вершин определение конуса $K(F)$ не изменится.

Множество всех конусов $K(F)$, где $F$ пробегает множество непустых граней полиэдра $P$, называется \emph{нормальным веером}~\cite[с.~257]{ZieglerBook}.

Нормальный веер является двойственной к полиэдру геометрической структурой.
А именно, если $F$ является собственной гранью грани $G$ полиэдра $P$, то $K(G)$ является гранью конуса $K(F)$.
Обратно, любая грань конуса $K(F)$ есть конус $K(G)$, соответствующий некоторой грани $G$ полиэдра $P$ такой, что $F \subset G$. В случае, когда минимальными (непустыми) гранями $P$ являются 0-грани, справедливо равенство $\dim K(F) = d - \dim(F)$.

Так как любой выпуклый многогранник $P \subset \R^d$ однозначно определяется множеством своих вершин $X = \ext(P)$, то и его нормальный веер однозначно определяется набором конусов его вершин
\[
\K(X) \coloneqq \Set*{K(\bm{x}) \given \bm{x} \in X},
\]
который далее будем называть \emph{конусным разбиением пространства исходных данных} задачи линейной оптимизации на множестве $X$ или, короче, \emph{конусным разбиением} $\R^d$ по множеству $X$~\cite{BondBook:1995}.
Это название объясняется тем, что задачу оптимизации вдоль целевого вектора $\bm{c}\in\R^d$ на множестве $X \in \R^d$ можно переформулировать как задачу поиска такого $\bm{x} \in X$, что $\bm{c} \in K(\bm{x})$.
Аналогично определяется конусное разбиение $\K(X)$ для полиэдра. Множество $X$ в этом случае является множеством минимальных непустых граней (что соответствует определению вершины полиэдра на с.~\pageref{def:PolyVertex}). При этом набор конусов $\K(X)$ может не покрывать все пространство $\R^d$.

Пусть $\bm{x}$ и $\bm{y}$ "--- вершины полиэдра $P \subset \R^d$.
Будем называть конусы $K(\bm{x})$ и $K(\bm{y})$ \emph{смежными}, если они имеют общую гипергрань:\label{AdjCones}
\[
\dim (K(\bm{x}) \cap K(\bm{y})) = d-1.
\]
Очевидно, конусы $K(\bm{x})$ и $K(\bm{y})$ смежны тогда и только тогда, когда смежны вершины $\bm{x}$ и $\bm{y}$.
Таким образом, граф полиэдра $P$ совпадает с графом соответствующего конусного разбиения $\K(X)$.

Относительно многогранника задачи $\conv(X)$ преимущество конусного разбиения $\K(X)$ пространства исходных данных заключается в том, что имеется возможность простой интерпретации различных дополнительных линейных ограничений на целевой вектор $\bm{c}$.
Так, например, в классической задаче о кратчайшем пути накладываются ограничения $\bm{c} \le \bm{0}$ (для задачи максимизации).
То есть вместо конусного разбиения всего пространства в этой задаче рассматривается конусное разбиение отрицательного ортанта.
На языке многогранников это означает переход от многогранника путей $\Path(n)$ к полиэдру $\Path^{\uparrow}(n)$ (см. раздел~\ref{subsec:polyhedra}).

По аналогии с определением конуса $K(\bm{x})$ для вершины $\bm{x}$ многогранника $P\subset \R^d$ введем обозначение
\[
K(\bm{x}, Q) \coloneqq \Set*{\bm{c}\in Q \given  \bm{c}^T \bm{x} \ge \bm{c}^T \bm{y}, \ \forall \bm{y} \in P} = K(\bm{x}) \cap Q, 
\]
где $Q$ "--- полиэдр в $\R^d$.
Множество всех таких конусов при фиксированном $Q$ обозначим $\K(X,Q)$ и будем называть \emph{разбиением $Q$ по множеству $X$}~\cite{Maksimenko:2004}.
Граф такого разбиения определяется по аналогии с графом конусного разбиения всего пространства $\R^d$:
\begin{enumerate}
	\item $\bm{x} \in X$ является вершиной \emph{графа разбиения} $Q$ по $X$, если \[\dim (K(\bm{x}, Q)) = \dim Q.\]
	\item Вершины $\bm{x}$ и $\bm{y}$ этого графа \emph{смежны}, если 
	\[\dim (K(\bm{x}, Q) \cap K(\bm{y}, Q)) = \dim Q - 1.\]
\end{enumerate}
Таким образом, граф разбиения $\K(X,Q)$ является подграфом графа конусного разбиения $\K(X)$, если размерность полиэдра $Q$ совпадает с размерностью всего пространства $\R^d$.

Очевидно, что при переходе от всего пространства исходных данных к полиэдру $Q$ конусы нормального веера могут менять свои геометрические очертания (превращаясь в полиэдры) и уменьшаться по размерности, вплоть до полного исчезновения. С точки зрения комбинаторной структуры, нормальный веер при этом может терять часть своих элементов, новые элементы появиться не могут.

\begin{remark}
Так как решение задачи оптимизации инвариантно относительно умножения целевого вектора на положительный скаляр, то, не уменьшая общности, вместо конусного разбиения $\K(X)$, $X \subset \R^d$ можно ограничиться рассмотрением разбиения $\K(X,\Cube(-1,1))$, где 
\[
\Cube(-1,1) \coloneqq \Set{\bm{x} \in \R^d \given \bm{-1} \le \bm{x} \le \bm{1}}.
\]
%В~частности, графы этих разбиений совпадают.
\end{remark}


%\section{Аффинная сводимость}

\begin{definition}[аффинная сводимость разбиений исходных данных]
\label{def:ConesReduction}
\sloppy
Будем говорить, что разбиение множества исходных данных задачи $(L,d,g)$ с ограничением $h$ \emph{аффинно сводится} к разбиению исходных данных задачи $(L',d',g')$ с ограничением $h'$, если первая задача аффинно сводится ко второй и, кроме того, аффинное отображение $\alpha$ и функция $\beta$ в определении~\ref{def:AffReductionRestriction} биективны.
\end{definition}

Нетрудно заметить, что если условия этого определения выполнены, то аффинное отображение $\alpha$ устанавливает взаимно"=однозначное соответствие между разбиением множества исходных данных первой задачи и некоторой частью разбиения исходных данных второй задачи. Откуда следует, в частности, что граф разбиения первой задачи является подграфом графа разбиения второй.

В~качестве простого примера такого типа сводимости рассмотрим задачу коммивояжера с ограничением $\bm{-1} \le \bm{c} \le \bm{1}$ и эту же задачу, но с ограничением $\bm{c} \ge \bm{0}$. Так как значение целевой функции для задачи коммивояжера представляет собой сумму ровно $n$ координат целевого вектора ($n$ "--- число городов), то добавление одного и того же скаляра ко всем координатам не изменит решения задачи.
Следовательно, чтобы аффинно свести разбиение задачи коммивояжера с ограничением $\bm{-1} \le \bm{c} \le \bm{1}$ к задаче с ограничением $\bm{c} \ge \bm{0}$, достаточно положить $\alpha \from \bm{c} \mapsto \bm{c} + \bm{1}$.


%%%%%%%%%%%%%%%%%%%%%%%%%%%%%%%%%%%%%%%%%%%%%%%%%%%%%%%%%%
%
%     Сравнение многогранников
%
%%%%%%%%%%%%%%%%%%%%%%%%%%%%%%%%%%%%%%%%%%%%%%%%%%%%%%%%%%

\section{Сравнение многогранников}
\label{sec:AffCompare}

Как известно, сравнение является одним из наиболее распространенных и универсальных методов исследования.
Чтобы воспользоваться им, введем следующую операцию сравнения,
задающую частичный порядок на множестве всех выпуклых многогранников.

\begin{definition}\label{def:ineA}
В~случае, когда многогранник $P$ аффинно эквивалентен многограннику~$Q$ или же его грани, будем использовать обозначение $P \lea Q$.
Факт аффинной эквивалентности многогранников $P$ и $Q$ обозначаем $P =_A Q$.
\end{definition}

В~главе~\ref{chap:ExtAff} будет рассмотрен более мягкий вариант этого определения,
в котором слова <<аффинно эквивалентен>> заменены на <<является аффинным образом>>, а для соответствующего соотношения используется обозначение $\lee$. Там же будет проведен анализ различий в использовании этих двух соотношений при исследовании свойств многогранников.

Нетрудно заметить, что соотношение $\lea$ оказывается особенно полезным при сравнении комбинаторных свойств 
%(т.\,е. свойств решеток граней) 
многогранников.

\begin{property}
Если $P \lea Q$, то решетка граней многогранника $P$ 
изоморфна либо всей решетке граней многогранника $Q$ (если $P$ и $Q$ эквивалентны), либо некоторой подрешетке (индуцированной гранью многогранника $Q$), а матрица инциденций вершин"=гиперграней многогранника $P$ является подматрицей (полученной удалением строк и столбцов) матрицы инциденций многогранника $Q$. В~частности:
\begin{enumerate}
	\item Число вершин многогранника $P$ не превосходит числа вершин $Q$.
	\item Число $i$-граней многогранника $P$ не превосходит числа $i$-граней многогранника $Q$ при $i \le \dim(P)$.
	\item Граф многогранника $P$ изоморфен некоторому подграфу графа многогранника $Q$.
	\item Число гиперграней $P$ не превосходит числа гиперграней $Q$.
	\item Для чисел прямоугольных покрытий матриц инциденций вершин"=гиперграней выполнено $\rc(P) \le \rc(Q)$.
\end{enumerate}
\end{property}

В~качестве примеров перечислим некоторые очевидные соотношения
для симплексов и циклических многогранников.
Прежде всего, симплекс $\Delta_n$ является гранью симплекса $\Delta_{n+1}$, 
а в силу транзитивности соотношения $\lea$ 
получаем 
\begin{equation}
\label{eq:compareDelta}
\Delta_n \lea \Delta_{n+k} \quad \forall n,k \in \N.
\end{equation}
Кроме того, 
\begin{equation}
\label{eq:compareDeltaCP}
\Delta_m \lea \CP_n(S) \quad \text{при } m < n \le |S|, 
\end{equation}
так как циклический многогранник $\CP_n(S)$ симплициален.
По той же причине 
\begin{equation}
\label{eq:compareCP}
\CP_n(S) \nelea \CP_{n+k}(S')
\end{equation}
для всех $n, k \in \N$
и любых множеств $S$ и $S'$, при условии $|S| > n+1 > 2$.

Заметим, что множество симплексов $\Delta = \{\Delta_n\}$ и множество циклических многогранников $\CP = \Set*{\CP_n(S) \given n\in\N, \ S \subset \Q, \ |S| < \infty}$ представляют собой примеры семейств многогранников задач комбинаторной оптимизации.
В~связи с этим интерес представляет следующий вопрос.
Какая из двух ситуаций \eqref{eq:compareDelta} или~\eqref{eq:compareCP} является наиболее типичной для известных семейств многогранников задач?
Опыт показывает, что в большинстве случаев легко проверяется справедливость соотношений вида~\eqref{eq:compareDelta}.
При доказательстве такого рода соотношений удобно пользоваться следующим очевидным утверждением.

\begin{lemma}
	\label{lem:01face}
	Пусть $P \subset \R^d$ "--- 0/1"~многогранник. Тогда $F_i = \Set*{\bm{x}\in P \given x_i = 0}$ и $G_i = \Set*{\bm{x}\in P \given x_i = 1}$, $i\in[d]$,
	являются гранями (быть может несобственными) многогранника $P$.
\end{lemma}

Рассмотрим три наиболее часто встречающихся в литературе семейства многогранников: булевы квадратичные многогранники $\BQP(n)$, многогранники асимметричной задачи коммивояжера $\ATSP(n)$ и многогранники задачи о рюкзаке $\Knap(n, \bm{a}, b)$.


\begin{prop}
\begin{align*}
\BQP(n) &\lea \BQP(n+1),\\
\ATSP(n) &\lea \ATSP(n+1),\\
\Knap(\bm{a}, b) &\lea \Knap((\bm{a},0), b), \quad \bm{a} \in \R^{n}, \ b\in \R.
\end{align*}
\end{prop}
\begin{proof}
Воспользуемся леммой~\ref{lem:01face}.
Рассмотрим грань $F$ многогранника $\BQP(n+1)$, образованную гиперплоскостью $x_{n+1, n+1} = 0$. 
Тогда для всех $\bm{x} \in F$ выполняется $x_{i, n+1} = 0$ при $i\in[n+1]$.
Нетрудно увидеть, что множество вершин грани $F$ преобразуется в множество вершин многогранника $\BQP(n)$ (и, наоборот, $\BQP(n)$ в $F$) аффинным отображением (ортогональной проекцией) $x_{ij} \mapsto y_{ij}$, $1 \le i \le j \le n$. 

Для доказательства соотношения $\ATSP(n) \lea \ATSP(n+1)$ 
достаточно установить взаимно однозначное соответствие между множеством гамильтоновых контуров полного орграфа $D=(V,A)$ на $n$ вершинах и подмножеством 
гамильтоновых контуров орграфа $D'=(V',A')$ на $n+1$ вершинах, в котором удалены все дуги, входящие в вершину $v'_1$, и дуги, выходящие из $v'_{n+1}$, за исключением дуги $(v'_{n+1}, v'_1)$.
Заметим, что характеристические вектора указанного подмножества контуров оргафа $D'=(V',A')$ являются вершинами грани 
\[F = \Set*{\bm{x} \in \ATSP(n+1) \given x_{(v'_{n+1}, v'_1)} = 1}
\]
многогранника $\ATSP(n+1)$.
Очевидно, грань $F$ и многогранник $\ATSP(n)$ связаны биективным аффинным отображением
\[
	y_{(v_i, v_j)} = 
	\begin{cases}
	x_{(v'_i, v'_j)}, & \text{при }1 \le i \le n, \ 2 \le j \le n, \ i \ne j,\\
	x_{(v'_i, v'_{n+1})}, & \text{при }2 \le i \le n, \ j=1,\\
	\end{cases}
\]
где $\bm{x} \in F$, $\bm{y} \in \ATSP(n)$.

Для доказательства соотношения $\Knap(\bm{a}, b) \lea \Knap((\bm{a},0), b)$ достаточно заметить, что 
\[
\Knap((\bm{a},0), b) = \Set*{(\bm{x},x_{n+1}) \in \{0,1\}^{n+1} \given \bm{x} \in \Knap(\bm{a}, b)}.
\]
Следовательно, многогранник $\Knap(\bm{a}, b)$ аффинно эквивалентен грани многогранника $\Knap((\bm{a},0), b)$, образованной гиперплоскостью $x_{n+1} = 0$.
% (или $x_{n+1} = 1$).
\end{proof}

Соотношение \eqref{eq:compareDeltaCP} является простым примером сравнения многогранников из разных семейств.
Еще одним таким фактом, открытым независимо несколькими авторами~\cite[с.~84]{Deza:2001}, является \emph{ковариантное отображение} $\xi\from \BQP(n) \to \Cut(n+1)$, задаваемое уравнениями
\[
y_{ij} = 
\begin{cases}
x_{ii}, & \text{при } 1 \le i \le n, \ j=n+1,\\
x_{ii} + x_{jj} - 2 x_{ij}, & \text{при } 1 \le i < j \le n.\\
\end{cases}
\]
%где $\bm{x} \in \BQP(n)$, $\bm{y} \in \Cut(n+1)$.
Из невырожденности этого отображения следует
\[
\BQP(n) =_A \Cut(n+1).
\]

Рассмотрим еще два семейства многогранников, ассоциированных с задачами об упаковке и разбиении множества.

Пусть $G = \{g_1, \ldots, g_m\}$ "--- конечное множество
и $S = \{S_1, \ldots, S_n\} \subseteq 2^G$ "--- некоторый набор подмножеств множества $G$.
Рассмотрим произвольное подмножество $T \subseteq S$.
Если каждый элемент $g_i \in G$ принадлежит не более (не менее) чем одному из элементов $T$, то $T$ называется \emph{упаковкой (покрытием) множества $G$}.
Покрытие, являющееся одновременно упаковкой, называется \emph{разбиением множества $G$}.
\emph{Матрицей инциденций} элементов множества $E$ и элементов множества $S$ называется матрица $A\in\{0,1\}^{m\times n}$ с элементами
\[
A_{ij} = 
\begin{cases}
1, &\text{если $g_i\in S_j$,}\\
0, &\text{иначе.}
\end{cases}
\]


%Пусть $A\in\{0,1\}^{m\times n}$ "--- матрица инциденций элементов множества $G$, $G = \{g_1, \ldots, g_m\}$, и элементов некоторого множества $S$, $S = \{S_1, \ldots, S_n\} \subseteq 2^G$.
Выпуклая \hypertarget{def:Pack}{оболочка} множества
\begin{equation*}
\Pack(A) = \Set*{\bm{x}\in\{0,1\}^n \given A \bm{x} \le \bm{1}}
%\quad \text{где } A\in\{0,1\}^{m\times n},
\end{equation*}
называется \emph{многогранником упаковок множества}~\cite{Balas:1976}.
(Каждая вершина $\bm{x} \in \Pack(A)$ этого многогранника является характеристическим вектором некоторой упаковки $T \subseteq S$.)

Множество вершин \emph{многогранника разбиений множества} определяется по аналогии:
\begin{equation}
\label{eq:Part}
\Part(A) = \Set*{\bm{x}\in\{0,1\}^n \given A \bm{x} = \bm{1}}.
\end{equation}
Непосредственно из определения следует
\begin{equation}
\label{eq:PartPack}
\Part(A) \lea \Pack(A).
\end{equation}

Заметим также, что многогранник независимых множеств $\Stable(G)$ (см. определение на с.~\hyperlink{Stable}{\pageref*{def:Stable}}) является частным случаем многогранника упаковок множеств:
\begin{equation}
\label{eq:StablePack}
\Stable(G) =_A \Pack(A),
\end{equation}
где $A$ является матрицей инциденций ребер"=вершин графа $G$.


%%%%%%%%%%%%%%%%%%%%%%%%%%%%%%%%%%%%%%%%%%%%%%%%%%%%%%%%%%
%
%     Аффинная сводимость многогранников
%
%%%%%%%%%%%%%%%%%%%%%%%%%%%%%%%%%%%%%%%%%%%%%%%%%%%%%%%%%%

\section{Аффинная сводимость многогранников}
\label{sec:AffReductPolytopes}

%\subsection{Сравнение семейств многогранников}

Прежде, чем перейти к сравнению семейств многогранников, установим еще несколько простых соотношений между $\Stable(G)$, $\Part(A)$, $\Pack(A)$ и $\BQP(n)$.

\begin{lemma}
	\label{lem:PackStable}
	Для любой матрицы $A\in\{0,1\}^{m\times n}$ существует граф $G$ на $n$ вершинах такой, что $\Pack(A) =_A \Stable(G)$.
\end{lemma}

\begin{proof}
	Достаточно заметить, что каждое неравенство вида
	$$
	x_1 + x_2 + \ldots + x_k \le 1
	$$
	из системы $A\bm{x} \le \bm{1}$ при условии $\bm{x}\in \{0,1\}^n$
	эквивалентно набору неравенств
	$$
	x_i + x_j \le 1, \quad 1\le i < j \le k,
	$$
	определяющих некоторый многогранник независимых множеств.
\end{proof}

Учитывая соотношение~\eqref{eq:StablePack}, можно сделать вывод о том, что семейства $\{\Pack(A)\}$ и $\{\Stable(G)\}$ идентичны (состоят из одних и тех же многогранников).

\begin{lemma}[\cite{Maksimenko:2016bool}]
	\label{lem:StablePart}
	Для любого графа $G=(V,E)$ существует матрица $A\in\{0,1\}^{m\times n}$, $m = |E|$, $n = |V|+|E|$, имеющая ровно по три единицы в каждой строке, что $\Stable(G) =_A \Part(A)$.
\end{lemma}
	
\begin{proof}
	Для каждого неравенства 
	\begin{equation}
		\label{eq:Stable2}
		x_v + x_u \le 1,  \quad \{v,u\} \in E,
	\end{equation}
	из описания многогранника $\Stable(G)$ введем вспомогательную переменную $y_{vu} = 1 - x_v - x_u$.
	Остается заметить, что множество 0/1-векторов, удовлетворяющих неравенствам~\eqref{eq:Stable2}, аффинно эквивалентно множеству 0/1-векторов, удовлетворяющих уравнениям
	\begin{equation*}
		x_v + x_u + y_{vu} = 1,  \quad \{v,u\} \in E.
	\end{equation*}
\end{proof}

Таким образом, согласно соотношению~\eqref{eq:PartPack}, семейство $\{\Part(A)\}$ содержит не только все многогранники семейств $\{\Pack(A)\}$ и $\{\Stable(G)\}$, но и некоторые их грани.

Введем теперь понятие аффинной сводимости
% аналог полиномиальной сводимости Кука--Карпа--Левина~\cite{Garey:1982} для
комбинаторных семейств многогранников.
Пусть семейство $\Pf = \Set{P(I)\given I\in L}$ определяется тройкой $(L, d, g)$.
Согласно замечанию~\ref{def:Psize}, размером многогранника $P(I)$ (не путать с размером расширения), называем величину $d(I) + \size(I)$.
%Очевидно, размер многогранника есть не что иное, как длина входа соответствующей задачи комбинаторной оптимизации.
%Таким образом, размер многогранника пропорционален длине входа соответствующей задачи комбинаторной оптимизации, если размер координат целевого вектора задачи ограничен сверху некоторой константой.

%\begin{definition}
%	\label{def:Aff}
%	Семейство многогранников $P$ \emph{аффинно сводится} к семейству многогранников $Q$, если для каждого многогранника $p\in P$ найдется $q\in Q$ такой, что $p \lea q$, причем размерность пространства, в котором определен многогранник $q$, ограничена сверху полиномом от размерности пространства, в котором задан $p$.
%	Обозначение: $P \propto_A Q$.  
%\end{definition}

%Заметим, что длина входа\footnote{Предполагается использование разумной схемы кодирования входных данных~\cite{Garey:1982}.} для задачи проверки несмежности вершин пропорциональна сумме длины кода $s$ и размерности $d(s)$.


\begin{definition}[\cite{Maksimenko:2017}]
	\label{def:Aff}
	Будем говорить, что семейство многогранников $\Pf$ \emph{аффинно сводится} к семейству многогранников $\Qf$, если найдутся полиномиально вычислимые (относительно размера многогранника $P\in \Pf$):
	\begin{enumerate}
		\item 
		Преобразование $\tau$ кода $I$ каждого многогранника $P = P(I)\in \Pf$ в код $I'$ многогранника $Q = Q(I') \in \Qf$.
		\item 
		Алгоритм построения для каждого кода $I$ аффинного отображения
		\[
		\alpha\from \R^d \to \R^{d'}, \qquad d = d(I), \quad d' = d'(\tau(I)),
		\]
		такого, что многогранник $\alpha(P(I))$ является гранью (возможно несобственной) многогранника $Q(\tau(I))$ и аффинно эквивалентен $P(I)$.
	\end{enumerate}
	Факт аффинной сводимости $\Pf$ к $\Qf$ обозначаем так: $\Pf \propto_A \Qf$.  
\end{definition}

%Отметим, что это определение отличается от определения аффинной сводимости,
% введенного ранее в монографии \cite{Bondarenko:2008}, 
% в сторону усиления условий.

\begin{comment}
\begin{remark}
	В~определении \ref{def:Aff} важным условием является полиномиальная зависимость относительно размера многогранника, а~не~его размерности. 
	Дело в том, что система ограничений, определяющих многогранник, может оказаться далеко не самым экономным способом его описания. Тем не менее, нашей целью является именно сравнение входных данных задач, ассоциированных с многогранниками.
	
	В~качестве примера многогранника с неэкономным описанием можно рассмотреть $\Part(A)$, задаваемый уравнениями
	\[
	x_1 + x_i + x_j = 1, \quad 2 \le i < j \le n.
	\]
	Очевидно, он состоит из одной единственной точки $(1,0,\dots,0)$.
	Вместе с тем, согласно соотношению \eqref{eq:PartPack}, он является гранью многогранника $\Pack(A)$, определяемого неравенствами
	\[
	x_1 + x_i + x_j \le 1, \quad 2 \le i < j \le n,
	\]
	и имеющего размерность $n$.
	Такой <<неэкономный>> способ сравнения многогранников вполне допускается определением~\ref{def:Aff}.
\end{remark}
\end{comment}

%\begin{remark}
%Определение \ref{def:Aff} отличается от определения аффинной сводимости в~\cite{Maksimenko:2013NP,Maksimenko:2016bool} наличием условий полиномиальной вычислимости преобразования $A$ и коэффициентов аффинного отображения.
%Тем не менее, для всех фактов аффинной сводимости, упоминаемых в работах~\cite{Maksimenko:2013NP,Maksimenko:2016bool}, справедливость этих условий легко проверяется, так как соответствующие аффинные отображения описаны явным образом.
%\end{remark}

\begin{remark}
Как правило, доказательство аффинной сводимости семейства $\Pf$ к семейству $\Qf$ выполняется по следующей схеме.
Для каждого кода $I$, задающего многогранник $P(I)\in \Pf$ приводится описание кода $I'$ многогранника $Q(I')\in \Qf$, его грани $F$ и биективного аффинного отображения $\alpha\colon P(I) \to F$.
Полиномиальная вычислимость указанных процедур, как правило, очевидна.
Поэтому в дальнейшем мы не уделяем внимание проверке этих условий, а сам факт аффинной сводимости формулируем как утверждение вида
<<Для каждого кода $I$, задающего многогранник $P(I)$ из семейства $\Pf$, существует код $I'$, определяющий многогранник $Q(I')\in \Qf$, такой, что $P(I) \lea Q(I')$>>.
(Примерами %такого рода утверждений 
могут служить леммы \ref{lem:PackStable} и \ref{lem:StablePart}.)
Наш выбор объясняется тем, что данное утверждение, в отличие от $\Pf \propto_A \Qf$, содержит информацию о качестве свед\'{е}ния. 
\end{remark}

На основе выведенных выше соотношений приведем несколько примеров использования обозначения из определения~\ref{def:Aff}.
Так, соотношение \eqref{eq:compareDeltaCP} можно переписать в виде $\Delta \propto_A \CP$. 
А из лемм \ref{lem:StablePart}, \ref{lem:PackStable} и соотношения 	\eqref{eq:PartPack} следует
 
\begin{theorem}[\cite{Maksimenko:2015DAN}]
	\label{thm:Class1}
	$\Stable \propto_A \Part \propto_A \Pack \propto_A \Stable$,
	где $\Stable = \{\Stable(G)\}$, $\Part = \{\Part(A)\}$, $\Pack = \{\Pack(A)\}$.
\end{theorem}

Перечислим некоторые очевидные свойства этого типа сводимости. 

\begin{prop}
	%\label{thm:Prop}
	Пусть $\Pf \propto_A \Qf$. 
	Предположим, что в семействе $\Pf$ есть многогранники, имеющие одно или несколько из следующих свойств:
	\begin{enumerate}
	\item Cверхполиномиальность числа вершин или гиперграней (относительно размера многогранника).
	\item Cверхполиномиальное кликовое число графа многогранника.
	\item NP-полнота критерия несмежности вершин.
	\item Cверхполиномиальное число прямоугольного покрытия.
	\item Cверхполиномиальная сложность расширения.
	\end{enumerate}
	\noindent
	Тогда в $\Qf$ имеются многогранники с теми же свойствами.
\end{prop}

Сравним теперь семейства $\BQP = \{\BQP(n)\}$ и $\Stable$.


\begin{theorem}[\cite{Maksimenko:2015DAN,Maksimenko:2016bool}]
	\label{thm:BQPStable}
	Для каждого $n\in \N$ существует граф $G = (V,E)$, $|V| = n(n+1)$, $|E| = n(2n-1)$, такой, что
	$\BQP(n) \lea \Stable(G)$.
\end{theorem}
(Похожий результат получен в~\cite{FioriniPokutta:2015}, но с более слабым соотношением $\lee$ (см. определение~\ref{def:ineE} на с.~\pageref{def:ineE}) и при $|V| = 2 n^2$.)
% Кроме того, представленное ниже доказательство значительно проще изложенного в~\cite{FioriniPokutta:2015}.)

\begin{proof}
	Каждое равенство $x_{ij} = x_{ii} x_{jj}$ из уравнения~\eqref{eq:BQP}, определяющего булев квадратичный многогранник, эквивалентно неравенствам
	\begin{equation}
	\label{eq:Clique}
	\begin{aligned}
	x_{ii} - x_{ij} &\ge 0, \\
	x_{jj} - x_{ij} &\ge 0, \\
	x_{ii} + x_{jj} - x_{ij} &\le 1,
	\end{aligned}
	\end{equation}
	при условии $x_{ij}\in\{0, 1\}$, $1\le i \le j \le n$.
	Остается преобразовать их в систему неравенств вида $y_l + y_m \le 1$. %из~\eqref{SSP}.
	Для этого введем $n(n+1)$ новых 0/1-переменных:
	\begin{equation}
	\label{eq:BQP2SSP}
	\begin{aligned}
	s_{ij} &= x_{ij},           & 1 &\le i  <  j \le n,\\
	t_{ij} &= x_{ii} - x_{ij},  & 1 &\le i  <  j \le n,\\
	u_i    &= x_{ii},           & 1 &\le i \le n,\\
	\bar{u}_i    &= 1 - x_{ii}, & 1 &\le i \le n.\\
	\end{aligned}
	\end{equation}
	Тогда ограничения~\eqref{eq:Clique} эквивалентны
	%\begin{equation}\label{eq:SSP2}
	\[
	\begin{aligned}
	s_{ij} + \bar{u}_j & \le 1, \\
	t_{ij} + u_j       & \le 1, \\
	u_i    + \bar{u}_i & =   1, \\
	s_{ij} + t_{ij} + \bar{u}_i & = 1,
	\end{aligned}
	\]
	%\end{equation}
	при условии целочисленности всех переменных.
	Очевидно, последние два равенства (точнее, $n(n+1)/2$ подобных равенств)
	определяют некоторую грань многогранника $\Stable(G)$, где число вершин графа $G$ равно $n(n+1)$, а $n (2n - 1)$ его ребер определяют систему неравенств
	%\begin{equation}\label{eq:SSP2}
	\[
	\begin{aligned}
	s_{ij} + \bar{u}_j & \le 1, \\
	t_{ij} + u_j       & \le 1, \\
	u_i    + \bar{u}_i & \le 1, \\
	%   s_{ij} + t_{ij}    & \le 1, \\
	s_{ij} + \bar{u}_i & \le 1, \\
	t_{ij} + \bar{u}_i & \le 1.
	\end{aligned}
	\]
	%\end{equation}
	Более того, соотношения~\eqref{eq:BQP2SSP} связывают эту грань с многогранником $\BQP(n)$ невырожденным аффинным отображением.
\end{proof}

\begin{prop}
	\label{prop:StableBQP}
	Если граф $G=(V,E)$ неполный, то соотношение $\Stable(G) \lea \BQP(n)$ невозможно ни при каком $n$.
\end{prop}

\begin{proof}
Прежде всего заметим, что в множество вершин $\Stable(G)$ всегда входят вектора $\bm{0}$, $\bm{e_1}$, \dots, $\bm{e_d}$, где $d=|V|$.
Если граф $G=(V,E)$ неполный, то $\Stable(G)$ кроме <<обязательных>> векторов содержит еще как минимум один 0/1"~вектор. Пусть $\bm{x}$ "--- один из таких (необязательных) векторов. Легко заметить, что вершины $\bm{0}$ и $\bm{x}$ многогранника $\Stable(G)$ несмежны, так как соединяющий их отрезок пересекается с выпуклой оболочкой вершин $\bm{e_1}$, \dots, $\bm{e_d}$.
Следовательно, многогранник $\Stable(G)$ не является 2"~смежностным.
Остается заметить, что многогранник $\BQP(n)$ (а вместе с ним и его грани) является 2"~смежностным~\cite{Bondarenko:1987,Padberg:1989}.
\end{proof}


Покажем, что аффинная сводимость многогранников тесно связана с аффинной сводимостью конусных разбиений пространств исходных данных соответствующих задач. 

\begin{theorem}
\label{thm:Aff2Cones}
Пусть линейная задача комбинаторной оптимизации $(L,d,g)$ задает семейство многогранников $\Pf = \Set{P(I)\given I\in L}$, а задача $(L',d',g')$ "--- семейство многогранников $\Qf = \Set{Q(I')\given I'\in L'}$.
Предположим, что семейство $\Pf$ аффинно сводится к $\Qf$ и, кроме того, для каждого $P \in \Pf$ определена опорная к соответствующему многограннику $Q$ гиперплоскость
\(H(\bm{a}, h)\), $h \in \Z$, $\bm{a} \in \Z^{d'}$, $d' = d'(I')$, задающая грань $\alpha(P) = Q \cap H(\bm{a}, h)$, аффинно эквивалентную $P$.
%причем $\bm{a}^T\bm{y} + h \ge 1$ для всех $\bm{y} \in \ext(q) \setminus \ext(F)$.

Тогда разбиение множества исходных данных задачи $(d,S,g)$ с ограничением $\bm{-1} \le \bm{c} \le \bm{1}$ аффинно сводится к разбиению множества исходных данных задачи $(d',S',g')$.
\end{theorem}
\begin{proof}
Согласно определению~\ref{def:ConesReduction} (аффинной сводимости разбиений исходных данных), достаточно описать:
\begin{enumerate}
 \item[1)] преобразование $\tau\from L \to L'$, 
 \item[2)] обратимое аффинное отображение $\alpha'\from \R^d \to \R^{d'}$,
 \item[3)] биективную функцию $\beta\from \ext(F) \to \ext(P)$, где $F = \alpha(P)$ "--- грань многогранника $Q$, аффинно эквивалентная $P$, 
\end{enumerate}
удовлетворяющие условиям определения~\ref{def:AffReductionRestriction} (аффинной сводимости задач с ограничением).
Часть этих условий уже выполнена:
преобразование $\tau$ уже известно, а $\beta(\bm{y}) = \alpha^{-1}(\bm{y})$, $\bm{y} \in F$.
Заметим также, что обратное к заданному аффинному отображению может быть найдено за полиномиальное время (см., например, \cite{Winkler:1996}).
Остается описать обратимое аффинное отображение $\alpha'$ , удовлетворяющее условиям пункта 3 определения~\ref{def:AffReductionRestriction}.
	
%Согласно предположению, многогранник $p$ и грань $F = \alpha(p)$ аффинно эквивалентны.
Для определенности предположим, что аффинное отображение $\alpha$ выражается формулой 
\begin{equation}
\label{eq:aff-transf}
\bm{y} = A\bm{x} + \bm{b}, \qquad \text{где } \bm{x} \in P, \quad \bm{y} \in F.
\end{equation}
Так как $P$ и $F$ аффинно эквивалентны, то
для любого $\bm{y^*} \in F$ и любого вектора $\bm{c}\in\R^d$, $d = d(I)$,
\[
\Bigl( 
\forall \bm{y}\in F \quad \bm{c}^T A^{-1} \bm{y^*} \ge \bm{c}^T A^{-1} \bm{y} 
\Bigr)
\iff
\Bigl( 
\forall \bm{x}\in P \quad \bm{c}^T \beta(\bm{y^*}) = \bm{c}^T A^{-1}(\bm{y^*} - \bm{b}) \ge \bm{c}^T \bm{x}
\Bigr).
\]
%Заметим, что обратная матрица $A^{-1}$ может быть найдена за полиномиальное время (см., например, \cite{Winkler:1996}).

Теперь скорректируем целевой вектор $\bm{v} = \bm{c}^T A^{-1}$ таким образом, чтобы неравенство $\bm{v}^T \bm{y} > \bm{v}^T \bm{y'}$ выполнялось для 
любых $\bm{y} \in \ext{F}$ и $\bm{y'} \in \ext(Q) \setminus \ext(F)$.

Гиперплоскость \(H(\bm{a}, h)\) является опорной к $Q$ и $F = Q \cap H(\bm{a}, h)$. Иными словами, $\bm{a}^T\bm{y} = h$ для всех $\bm{y} \in F$ и, не уменьшая общности, $\bm{a}^T\bm{y'} < h$ для всех $\bm{y'} \in \ext(Q) \setminus \ext(F)$.
Так как $h \in \Z$ и $\bm{a},\bm{y'} \in \Z^{d'}$, то
\[
h - \bm{a}^T\bm{y'} \ge 1 \qquad \forall \bm{y'} \in \ext(Q) \setminus \ext(F).
\]
Выберем число $N \in \N$ так, что $N > \max_{\bm{y}\in Q} \|A^{-1} \bm{y}\|_1$, где $\|\bm{x}\|_1 = \sum_{i} |x_{i}|$. (Очевидно, это можно сделать за полиномиальное время.)
Тогда для любых $\bm{y} \in \ext{F}$, $\bm{y'} \in \ext(Q) \setminus \ext(F)$ и $\bm{-1} \le \bm{c} \le \bm{1}$ выполнено
\begin{multline*}
\bigl(\bm{c}^T A^{-1} + 2N \bm{a}^T\bigr) \bm{y} =\\
= \bm{c}^T A^{-1} \bm{y} + 2Nh > -N + 2Nh > \bm{c}^T A^{-1} \bm{y'} + 2Nh - 2N \ge\\
\ge \bigl(\bm{c}^T A^{-1} + 2N \bm{a}^T\bigr)\bm{y'}.
\end{multline*}

Таким образом, аффинная сводимость разбиения множества исходных данных задачи $(d,S,g)$ с ограничением $\bm{-1} \le \bm{c} \le \bm{1}$ к разбиению множества исходных данных задачи $(d',S',g')$ определяется преобразованием $\tau$ из определения аффинной сводимости семейств многогранников, аффинным отображением \(\alpha'\from \bm{c} \mapsto \bm{c}^T A^{-1} + 2N \bm{a}^T\) и биективной функцией $\beta: \bm{y} \mapsto A^{-1}(\bm{y} - \bm{b})$.
\end{proof}

\begin{remark}
Для всех примеров аффинной сводимости многогранников, представленных в настоящей работе, грань $F = \alpha(P)$ соответствующего многогранника $Q$ описывается как пересечение нескольких опорных (к $Q$) гиперплоскостей, задаваемых системой уравнений с целочисленными коэффициентами вида
\begin{equation*}
%\label{eq:affsupport}
A \bm{y} + \bm{b} = \bm{0},
\end{equation*}
причем для каждого $\bm{y'} \in \ext(Q) \setminus \ext(F)$ в этой системе имеется хотя бы одно уравнение, в котором левая часть отрицательна (положительна).
Очевидно, эта система может быть агрегирована в одно уравнение
\[
\bm{1}^T A \bm{y} + \bm{1}^T \bm{b} = 0,
\]
определяющее гиперплоскость $H(\bm{1}^T A, -\bm{1}^T)$, опорную к $Q$ и удовлетворяющую условиям теоремы~\ref{thm:Aff2Cones}.

Например, в доказательстве теоремы~\ref{thm:BQPStable} соответствующая система состоит из уравнений
\[
\begin{aligned}
u_i   + \bar{u}_i  - 1 & = 0,          & i &\in [n],\\
s_{ij} + t_{ij} + \bar{u}_i - 1 & = 0, & 1 &\le i < j \le n.
\end{aligned}
\]
\end{remark}


%%%%%%%%%%%%%%%%%%%%%%%%%%%%%%%%%%%%%%%%%%%%%%%%%%%%%%%
%
% End of section
%
%%%%%%%%%%%%%%%%%%%%%%%%%%%%%%%%%%%%%%%%%%%%%%%%%%%%%%%
