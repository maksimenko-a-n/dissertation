% !TeX encoding = UTF-8 Unicode
% !TEX root = MaksimenkoThesis.tex

%%%%%%%%%%%%%%%%%%%%%%%%%%%%%%%%%%%%%%%%%%%%%%%%%%%%%%%%%%
%
%     Расширенная аффинная сводимость
%
%%%%%%%%%%%%%%%%%%%%%%%%%%%%%%%%%%%%%%%%%%%%%%%%%%%%%%%%%%

\chapter{Расширенная аффинная сводимость}
\label{chap:ExtAff}
%\begin{flushright}
%	А если немного ослабить условия?
%\end{flushright}
%\medskip

В этой главе рассматриваются следствия смягчения способа сравнения многогранников, используемого в предыдущей главе.
А именно, слова <<аффинно эквивалентен>> в определении~\ref{def:ineA} будут заменены на <<является аффинным образом>>. 
На основе такого способа сравнения многогранников в первом разделе главы вводится понятие расширенной аффинной сводимости семейств многогранников, выводятся некоторые его свойства и устанавливаются связи между этим понятием и другими определениями аффинной сводимости из главы~\ref{chap:AffTheory}.
Далее, в разделе~\ref{sec:ExtAffExamples} приводятся примеры расширенной аффинной сводимости. В~целом, благодаря ослаблению условий, доказательства соответствующих утверждений оказываются значительно более простыми, чем для (обычной) аффинной сводимости.
В последнем разделе главы показано, что любое семейство многогранников, предикат допустимости которого принадлежит классу NP, расширенно аффинно сводится к булевым квадратичным многогранникам.
Тем самым, все упоминаемые выше в настоящей работе семейства многогранников оказываются эквивалентны друг другу с точки зрения расширенной аффинной сводимости.

\section{Теория}
\label{sec:ExtensionTheory}

\begin{definition}\label{def:ineE}
	Если некоторая грань многогранника $Q$ или же весь этот многогранник является расширением многогранника $P$, 
	%В случае, когда многогранник $P$ является аффинным образом многогранника~$Q$ или же его грани, 
	будем использовать обозначение $P \lee Q$.
\end{definition}

%Заметим, прежде всего, что из $P \lea Q$ следует $P \lee Q$.
Из $P \lee Q$, в частности, следует $\xc(P) \le \xc(Q)$ и~$\rc(P) \le \rc(Q)$
%Некоторые следствия, вытекающие из соотношения $P \lee Q$, были описаны выше, в разделах \ref{sec:Extension} и \ref{sec:RectCover} 
(свойство~\ref{prop:xc-compare} и его аналог для числа прямоугольного покрытия матрицы инциденций вершин"=гиперграней). 

Замена $\lea$ на $\lee$ значительно расширяет возможности сравнения многогранников. Кроме того, во многих случаях доказательство соотношений вида $P \lee Q$ принципиально проще, чем соотношений $P \lea Q$.
Минусом такого ослабления ограничений является потеря некоторых полезных свойств.
В частности, пользуясь соотношением $P \lee Q$, вообще говоря, нельзя сравнивать числа гиперграней соответствующих многогранников и кликовые числа их графов.

Рассмотрим несколько простых примеров.

Ранее (см. с.~\pageref{ProjOfSimplex}) уже упоминался тот факт, что 
любой многогранник $P$, имеющий $n$ вершин, является аффинным образом симплекса $\Delta_{n-1}$. В частности, если $P$~--- $d$"~мерный 0/1"~многогранник, то
\(P \lee \Delta_{n}\), при $n \ge 2^d-1$.

%По аналогии с $d$-мерным 0/1-многогранником введем понятие $d$-мерного
%\emph{$[k]$-многогранника}, множество вершин которого принадлежит $[k]^d$.

Пусть $a,b \in Z$, $a < b$, и $[a,b]$~--- множество целых чисел соответствующего отрезка. Пусть $k = \lceil\log_2 (b-a+1)\rceil$.
Аффинное отображение $\zeta \from \{0,1\}^k \to [a, a + 2^k - 1]$ определим следующим образом:
\begin{equation}
\label{eq:01toZ}
\zeta \from \bm{y} \mapsto a + y_1 + 2 y_2 + \dots + 2^{k-1} y_k.
\end{equation}
Очевидно, $\zeta$ полиномиально вычислимо и устанавливает взаимно"=однозначное соответствие между множеством $\{0,1\}^k$ и множеством целых чисел отрезка $[a, a + 2^k - 1]$.
Как следствие, получаем следующее

\begin{prop}\label{prop:01}
Пусть $X \subseteq S^n$, где $S$~--- множество целых чисел некоторого отрезка $[a,b]$.
Тогда существует $Y \subseteq \{0,1\}^d$, $d = n \cdot \lceil\log_2 (b-a+1)\rceil$, что
\[
X \lee Y.
\]
\end{prop}

Иными словами, при некоторых, достаточно мягких ограничениях, многогранник с целочисленными вершинами является проекцией 0/1-многогранника.

По аналогии с аффинной сводимостью (определение \ref{def:Aff}) введем понятие расширенной аффинной сводимости на основе отношения $\lee$.
Предварительно напомним, что под размером многогранника подразумевается сумма длины его кода и размерности пространства.


\begin{definition}
	\label{def:Aff2}
	Будем говорить, что семейство многогранников $\Pf$ \emph{расширенно аффинно сводится} к семейству многогранников $\Qf$, если найдутся полиномиально вычислимые (относительно размера многогранника $P\in \Pf$):
	\begin{enumerate}
		\item 
		Преобразование $\tau$ кода $I$ каждого многогранника $P = P(I)\in \Pf$ в код $I'$ многогранника $Q = Q(I') \in \Qf$.
		\item 
		Система линейных уравнений $D\bm{y}=\bm{b}$, задающая грань
		\(F = \Set{\bm{y}\in Q \given D\bm{y}=\bm{b}}\)
		многогранника $Q = Q(\tau(I))$.
		\item 
		Алгоритм построения для каждого кода $I$ аффинного отображения
		\[
		\beta\from \R^{d'} \to \R^{d}, \qquad d = d(I), \quad d' = d'(\tau(I)),
		\]
		такого, что $P(I) = \beta(F(I))$.
	\end{enumerate}
	Обозначение: $\Pf \propto_E \Qf$.  
\end{definition}

\begin{lemma}
	Для любых двух семейств многогранников $\Pf$ и $\Qf$ из $\Pf \propto_A \Qf$ следует $\Pf \propto_E \Qf$.
\end{lemma}
	
\begin{proof}
	В определении~\ref{def:Aff} требуется наличие биективного аффинного отображения $\alpha \from P \to F$,
	а в определении~\ref{def:Aff2}~--- наличие аффинного отображения $\beta \from F \to P$ и системы линейных уравнений, определяющих грань $F$.
	Если отображение биективно, то переход от одного способа описания к другому можно осуществить за полиномиальное время (см., например, \cite{Winkler:1996}).
\end{proof}

По аналогии с теоремой~\ref{thm:Aff2Cones} доказывается следующая связь этого типа сводимости с аффинной сводимостью линейных задач комбинаторной оптимизации.

\begin{theorem}
\label{thm:ExtAff2AffProblems}
Пусть линейная задача комбинаторной оптимизации $(L,d,g)$ определяет семейство многогранников $\Pf = \Set{P(I)\given I\in L}$, а задача $(L',d',g')$ "--- семейство многогранников $\Qf = \Set{Q(I')\given I'\in L'}$.
Предположим, что семейство $\Pf$ расширенно аффинно сводится к $\Qf$ и, кроме того,
система уравнений $D\bm{y}=\bm{c}$, задающая грань соответствующего многогранника $Q \in \Qf$, состоит из уравнений гиперплоскостей, опорных к $Q$.
Тогда задача $(L,d,g)$ с ограничением $\bm{-1} \le \bm{c} \le \bm{1}$ аффинно сводится к задаче $(L',d',g')$.
\end{theorem}

Перечислим некоторые очевидные свойства этого типа сводимости,
вытекающие из свойств расширений, описанных в разделах \ref{sec:Extension} и \ref{sec:RectCover}. 

%\begin{prop}
%	Пусть $P \propto_E Q$. Тогда задача линейной оптимизации на многогранниках семейства $P$ полиномиально сводится к задаче линейной оптимизации на многогранниках семейства $Q$.
%\end{prop}
	
\begin{prop}
	\label{thm:PropE}
	Пусть $\Pf \propto_E \Qf$.
	Предположим, что в семействе $\Pf$ есть многогранники, имеющие одно или несколько из следующих свойств:
	\begin{enumerate}
		\item Cверхполиномиальность числа вершин (относительно размера многогранника).
		\item Cверхполиномиальное число прямоугольного покрытия.
		\item Cверхполиномиальная сложность расширения.
	\end{enumerate}
	\noindent
	Тогда в $\Qf$ имеются многогранники с теми же свойствами.
\end{prop}

Обратим теперь внимание на ограничение $\size(\|\bm{x}\|_{\infty}) = \poly(\size(I) + d(I))$, $\bm{x} \in X(I)$, в определении~\ref{def:LCOP} линейной задачи комбинаторной оптимизации.
Таким образом, для каждой задачи $(L,d,g)$ существует функция $b \from L \to \N$ такая, что $b(I) = \poly(\size(I) + d(I))$ и $X(I) \in S^d$, где $S$ "--- множество целых точек отрезка $[-2^{b(I)}, 2^{b(I)}]$. 
%$\max_{\bm{x} \in X(I)} \size(\|\bm{x}\|_{\infty}) \le m(I)$ и , $I \in L$.


%  ДОПИСАТЬ ЗДЕСЬ!!!!!!!!!!!!!!!!!!!!!!!!!!!!!!!!!!!!!!!!!!!!!!!!!!!!!!



По аналогии с утверждением~\ref{prop:01} легко доказать следующую теорему.

\begin{theorem}[об универсальности 0/1-многогранников]
%\sloppy
	Пусть комбинаторное семейство многогранников $\Pf$ определяется тройкой $(L,d,g)$ и описанная выше функция $b \from L \to \N$ полиномиально вычислима.
	%(см. определение~\ref{def:family}).
	Тогда $\Pf$ расширенно аффинно сводится к комбинаторному семейству 0/1"=многогранников,
	определяемых тройкой $(L,d',g')$, где $d'(I) = 2b(I)d(I)$,
	$g'(\bm{y}, I) = g(\beta(\bm{y}), I)$, $\bm{y} \in \{0,1\}^{d'}$,
	а~аффинное отображение $\beta \from \{0,1\}^{d'} \to [-2^{b(I)}, 2^{b(I)}]^d$ определяется по аналогии с формулой~\eqref{eq:01toZ}.
\end{theorem}
\begin{proof}
	Проверим выполнение условий определения~\ref{def:Aff2}.
	Преобразование $\tau \from L \to L$ тривиально.
	Система линейных уравнений отсутствует (может быть представлена в виде $\bm{0}^T \bm{y} = 0$).
	Аффинное отображение $\beta$ определено в условии теоремы.
	Совпадение множества $X(I) = \Set*{\bm{x} \in [-2^{b(I)}, 2^{b(I)}]^d \given g(\bm{x}, I)}$ и образа множества $Y(I) = \Set*{\bm{y} \in \{0,1\}^{d'} \given g'(\bm{y}, I)}$ при отображении $\beta$ следует из того, что $g'(\bm{y}, I) = g(\beta(\bm{y}), I)$.
	
	Остается убедиться в полиномиальной вычислимости $\beta$ относительно размера $d(I) + \size(I)$.
	Как было сказано выше, при обсуждении формулы~\eqref{eq:01toZ}, отображение $\beta$ полиномиально вычислимо относительно $d'$.
	В свою очередь, функция $d'$ тоже полиномиально вычислима и, более того, полиномиальна относительно $d(I) + \size(I)$.
\end{proof}

Этот факт хорошо известен, и многие исследователи осознанно ограничиваются изучением семейств 0/1-многогранников~\cite{Junger:1995,Papadimitriou:1984}.

%Определение. Пусть функция $d = d(n)$ полиномиальна, а последовательность конечных множеств $X_n \in \Z^d$ такова, что для любых $n \in \N$ и $x \in \Z^d$ задача проверки принадлежности $x \in X_n$ принадлежит классу NP. Тогда семейство (последовательность) многогранников $P(n) = \conv(X_n)$ называется \emph{комбинаторным}.

%Примечание. В~\cite{Naddef:1981, MatsuiTamura:1995} комбинаторными называются многогранники, у которых для каждой пары несмежных вершин середина соединяющего их отрезка является также серединой отрезка, соединяющего (некоторую) другую пару вершин этого многогранника.

Оказывается, относительно расширенной аффинной сводимости определение линейной задачи комбинаторной оптимизации существенно не изменится, если требование $g \in P$ для предиката допустимости $g$ заменить требованием $g \in \NP$.

\begin{theorem}
	\label{thm:NP2comb}
	Пусть тройка $(L,d,g)$ определяет некоторое семейство многогранников $\Pf$ и $g \in \NP$.
	Тогда $\Pf$ расширенно аффинно сводится к некоторому комбинаторному семейству многогранников $\Qf$ с полиномиально вычислимым предикатом допустимости.
\end{theorem}
\begin{proof}
Пусть $g = g(\bm{x},I)$ принадлежит классу NP.
Тогда, согласно определению~\ref{def:NP}, существует полином $p\from\N\to\N$ и полиномиально вычислимый предикат $f\from \{0,1\}^* \to \{\text{ложь},\text{истина}\}$ такие, что 
\[
g(\bm{x},I) \iff \text{найдется } \bm{u}\in \{0,1\}^{p(d+\size(I))} 
\text{ такой, что } f(\bm{x}, \bm{u}, I).
\]

С помощью предиката $f$ определим многогранники
\[
Q(I) = \Set*{(\bm{x}, \bm{u}) \in \Z^d \times \{0,1\}^{p(d+\size(I))} \given f(\bm{x}, \bm{u}, I)}
\]
некоторого комбинаторного семейства многогранников $\Qf = \{Q(I)\}$. 
Остается заметить, что многогранник 
\[
P(I) = \Set*{\bm{x}\in\Z^d \given g(\bm{x},I)}
\]
является ортогональной проекцией многогранника $Q(I)$.
\end{proof}

\begin{remark}
\label{rem:P2NP}
Таким образом показано, что замена условия $g \in P$ на $g \in \NP$ в определении линейной задачи комбинаторной оптимизации не меняет сложностного статуса всего семейства задач и сохраняет их некоторые комбинаторно"=геометрические характеристики.
Поэтому далее предполагаем $g \in \NP$.
Примером семейства многогранников с NP-полным предикатом допустимости могут служить многогранники гамильтоновых графов $\Ham(n)$ (см. определение на с.~\pageref{def:HamPolytope}).
\end{remark}

В заключение этого раздела покажем, что соотношение $P \lee Q$ в некоторых случаях можно использовать для сравнения кликовых чисел графов многогранников $P$ и $Q$.

\begin{theorem}
Пусть многогранник $P \subseteq \R^d$ является образом многогранника $Q \subseteq \R^n$ при ортогональном проецировании $\pi \from \R^n \to \R^d$, $n > d$.
Пусть, кроме того, $\pi(\ext Q) = \ext P$.
Тогда граф многогранника $P$ является подграфом графа многогранника $Q$.
\end{theorem}

\begin{proof}
Для каждой вершины $\bm{v} \in \ext P$ определим множество
\[
W(\bm{v}) = \Set*{\bm{x} \in \ext Q \given \pi(\bm{x}) = \bm{v}}.
\]
Очевидно, $\conv (W(\bm{v}))$ является гранью многогранника $Q$.
Более того, если множество $V \subseteq \ext P$ является множеством вершин некоторой грани многогранника $P$, то $\conv \Set{W(\bm{v})\given v\in V}$~--- грань многогранника $Q$.

Для удобства полагаем, что $\R^d$ вложено в $\R^n$,
а проецирование $\pi$ преобразует $(x_1, \dots, x_d, \dots, x_n)$ в $(x_1, \dots, x_d, 0, \dots, 0)$.
Вспомогательный вектор $\bm{c} \in \R^n$ определим следующим образом.
Если $n = d+1$, то $\bm{c} = \bm{e_n}$,
иначе $\bm{c} = \lambda_{d+1} \bm{e_{d+1}} + \dots + \lambda_{n} \bm{e_n}$
и коэффициенты $\lambda_{d+1}$, \dots, $\lambda_{n}$ подобраны так, 
что для любого $\bm{v} \in \ext P$ и любых двух вершин $\bm{w}, \bm{w'} \in W(\bm{v})$ из неравенства $\bm{w} \ne \bm{w'}$ следует $\bm{c}^T \bm{w} \ne \bm{c}^T \bm{w'}$.
Ясно, что в силу конечности множества вершин многогранника $Q$ такой вектор $\bm{c}$ существует. 
С помощью этого вектора в каждом множестве $W(\bm{v})$ выделим уникальную вершину
\[
\bm{w_v} = \argmax_{\bm{w} \in W(\bm{v})} \Set{\bm{c}^T \bm{w}}.
\]

Остается показать, что если вершины $\bm{v_1}$ и $\bm{v_2}$ многогранника $P$ смежны, то соответствующие вершины 
$\bm{w_1} = \bm{w_{v_1}}$ и $\bm{w_2} = \bm{w_{v_2}}$ тоже смежны.

Предположим, что вершины $\bm{v_1}$ и $\bm{v_2}$ многогранника $P$ смежны.
Тогда многогранники $F_1 = \conv (W(\bm{v_1}))$ и $F_2 = \conv (W(\bm{v_2}))$, а также их выпуклая оболочка $F = \conv(F_1 \cup F_2)$ являются гранями многогранника $Q$.

Положим $\bm{b} = \bm{v_2} - \bm{v_1}$.
Тогда $\bm{b}^T \bm{w} = \bm{b}^T \bm{w'}$ для любых $\bm{w}, \bm{w'} \in F_1$.
Аналогично, $\bm{b}^T \bm{w} = \bm{b}^T \bm{w'}$ для любых $\bm{w}, \bm{w'} \in F_2$.
Кроме того, $\bm{b}^T \bm{w_1} \neq \bm{b}^T \bm{w_2}$.
%Причем $\aff(F_1)$ и $\aff(F_2)$ ортогональны вектору 
%(Соответственно, линейная функция $\bm{b}^T \bm{x}$ принимает разные значения для $\bm{w_1}$ и $\bm{w_2}$.)

Подберем числа $\beta, \gamma \in \R$, $\gamma > 0$, так,
чтобы линейная функция $f(\bm{x}) = \beta \bm{b}^T \bm{x} + \gamma \bm{c}^T \bm{x}$ принимала одинаковые значения для $\bm{w_1}$ и $\bm{w_2}$.
Тогда из определения вектора $\bm{c}$ и вершин $\bm{w_1}$ и $\bm{w_2}$ следует
\[
f(\bm{w_1}) = f(\bm{w_2}) > f(\bm{w}) \quad
%\text{для всех } 
\forall
\bm{w} \in W(\bm{v_1}) \cup W(\bm{v_2}) \setminus \{\bm{w_1}, \bm{w_2}\}.
\]
Следовательно, $\bm{w_1}$ и $\bm{w_2}$~--- смежные вершины многогранника $F = \conv(W(\bm{v_1}) \cup W(\bm{v_2}))$, являющегося гранью многогранника $Q$.
\end{proof}

\begin{corollary}
\label{cor:CliqueOfExtension}
Пусть многогранник $P \subseteq \R^d$ является образом многогранника $Q \subseteq \R^n$ при аффинном отображении $\pi \from \R^n \to \R^d$.
Пусть, кроме того, $\pi(\ext Q) = \ext P$.
Тогда кликовые числа графов этих многогранников связаны соотношением $\omega(P) \le \omega(Q)$.
\end{corollary}

Особо отметим, что все известные автору настоящей работы естественные примеры расширенной аффинной сводимости (в том числе и перечисленные в этой главе) удовлетворяют условиям этого следствия.
%В первую очередь этот феномен объясняется тем, что в основном речь идет о семействах 0/1-многогранников.

%%%%%%%%%%%%%%%%%%%%%%%%%%%%%%%%%%%%%%%%%%%%%%%%%%%%%%%%%%
%
%     Примеры
%
%%%%%%%%%%%%%%%%%%%%%%%%%%%%%%%%%%%%%%%%%%%%%%%%%%%%%%%%%%

\section{Примеры}
\label{sec:ExtAffExamples}

По причине существенного ослабления условий, факты расширенной аффинной сводимости встречаются в литературе гораздо чаще, чем факты аффинной сводимости.
Перечислим несколько известных примеров:
\begin{enumerate}
	\item Перестановочный многогранник $\Perm(n)$ является линейной проекцией многогранника Биркгофа: $\Perm(n) \lee \Birk(n)$ (см. с.~\pageref{Perm2Birk}).
	%	\item Многогранники совершенных паросочетаний являются многогранниками упаковок множества: $\Match \propto_A \Pack$.
	\item Связь между многогранниками гамильтоновых циклов и многогранниками гамильтоновых контуров упоминалась в разделе~\ref{sec:Travelling}:
	$\TSP(n) \lee \ATSP(n)$.
	\item Многогранник совершенных паросочетаний $\Match(2n)$ является проекцией многогранника гамильтоновых циклов~\cite{Yannakakis:1991}: $\Match(2n) \lee \TSP(6n)$.
	\item Для любого графа $G=(V,E)$, $\Stable(G) \lee \ATSP(k)$, где $k = 4 |E| + |V|$~\cite{Yannakakis:1991}.
	\item Для каждого $n\in \N$ существует граф $G = (V,E)$, $|V| = 2n^2$, такой, что $\BQP(n) \lee \Stable(G)$~\cite{FioriniPokutta:2015} (см. также теорему \ref{thm:BQPStable}).
	\item Многогранник задачи о 3-выполнимости $\SAT(U,C)$ является проекцией грани многогранника трехиндексной аксиальной задачи о назначениях~\cite{AvisTiwary:2015}: $\SAT(U,C) \lee \TAP(m)$, где $m = O(kn)$, $k=|U|$, $n=|C|$. Причем аффинная сводимость $\SAT \propto_A \TAP$ невозможна (см. раздел~\ref{sec:3Ass}).
	\item Для любого графа $G=(V,E)$ существует планарный кубический граф $G'=(V',E')$, $|V'| = O(|E|^2)$, $|E'| = O(|E|^2)$, что $\Stable(G) \lee \Stable(G')$~\cite{AvisTiwary:2015}.
	\item Нетрудно заметить, что $\Stable(G) \lee \BQP(n)$ для любого графа $G = ([n],E)$. (Соответствующая грань многогранника $\BQP(n)$ лежит в пересечении гиперплоскостей $x_{ij} = 0$, $\{i,j\} \in E$.) В то же время,	если граф $G$ неполный, то соотношение $\Stable(G) \lea \BQP(n)$ невозможно ни при каком $n$ (см. утверждение~\ref{prop:StableBQP}).
\end{enumerate}

Ряд следующих утверждений призван продемонстрировать типичные способы доказательства фактов расширенной аффинной сводимости. 

%Тот факт, что семейство многогранников задачи о рюкзаке с равенством аффинно сводится к семейству многогранников задачи о рюкзаке, очевиден (см. определения и комментарии на с.~\pageref{eq:KnapEq})

Начнем с того, что семейство многогранников задачи о рюкзаке расширенно аффинно сводится к семейству многогранников задачи о рюкзаке с равенством
(см. определение на с.~\pageref{eq:KnapEq}).

\begin{lemma}
	Пусть $n \in \N$, $\bm{a}=(a_1,\dots,a_n) \in \Z^n$ и $b \in \Z$.
	Положим 
	\[
	S_{\min} = \begin{cases}
	0,& \text{если }  \bm{a} \ge \bm{0},\\
	\sum\limits_{i:\ a_i < 0} a_i,& \text{иначе,} 
	\end{cases}
	\]
	$M = \max\{1, b - S_{\min}\}$ и 
	\[
	\bm{c} = (a_1, \dots, a_n, 1, 2, \dots, 2^k), \quad \text{где }
	k = \lceil\log_2 (M)\rceil.
	\]
	Тогда
	$\Knap(\bm{a},b) \lee \KnapEq(\bm{c},b)$.
\end{lemma}

\begin{lemma}
	Для любых $n \in \N$, $\bm{a} \in \Z^n$ и $b \in \Z$ выполнено
	$\KnapEq(\bm{a},b) \lee \BQP(n)$.
\end{lemma}
\begin{proof}
Уравнение $\bm{a}^T \bm{x} = b$ из определения многогранника задачи о рюкзаке с равенством (см. формулу~\eqref{eq:KnapEq}) заменим на
\(
(\bm{a}^T \bm{x} - b)^2 = 0,
\)
или, что то же самое,
\[
\sum_i 2 b a_i x_i - \sum_{i,j} a_i a_j x_i x_j = b^2.
\]
Очевидно, уравнение 
\[
\sum_i 2 b a_i y_{ii} - \sum_{i,j} a_i a_j y_{ij} = b^2, \quad \text{где }
\bm{y} =(y_{ij}) \in \R^{n(n+1)/2},
\]
определяет опорную гиперплоскость к многограннику $\BQP(n)$.
С другой стороны, отображение $y_{ii} \mapsto x_i$, $i\in[n]$, проецирует соответствующую грань многогранника $\BQP(n)$ на многогранник $\KnapEq(\bm{a},b)$.
\end{proof}

\begin{lemma}[\cite{Maksimenko:2013TSP}]
$\ATSP(n) \lee \BQP\left((n-1)^2\right)$. 
\end{lemma}
\begin{proof}
Координаты вектора $\bm{y} \in \BQP\left((n-1)^2\right)$ обозначим $y(ij,kl)$, где $i,j,k,l \in [n-1]$, $i \le k$ и если $i = k$, то $j \le l$.
Согласно определению булева квадратичного многогранника (см. формулу \eqref{eq:BQP}), $y(ij,kl) = y(ij,ij) y(kl,kl)$.

Очевидно, уравнения 
\[
y(ij,il) = 0, \qquad i,j,l \in [n-1], \quad j < l,
\]
задают некоторую грань $F_1$ этого многогранника.
Из этого следует
\[
\sum_{j\in[n-1]} y(ij,ij) \le 1, \qquad i \in [n-1], \text{ для всех }\bm{y}\in F_1.
\]
Положив 
\begin{equation}
\label{eq:ATSP2BQP1}
\sum_{j\in[n-1]} y(ij,ij) = 1, \qquad i \in [n-1],
\end{equation}
перейдем к рассмотрению грани $F_2 \subseteq F_1$.

Аналогично, уравнения
\[
y(ij,kj) = 0, \qquad i,j,k \in [n-1], \quad i < k,
\]
и
\begin{equation}
\label{eq:ATSP2BQP2}
\sum_{i\in[n-1]} y(ij,ij) = 1, \qquad j \in [n-1],
\end{equation}
задают некоторую грань $F_3 \subseteq F_2$ многогранника $\BQP\left((n-1)^2\right)$.

Таким образом, для каждого $\bm{y} \in F_3$,
согласно уравнениям \eqref{eq:ATSP2BQP1} и \eqref{eq:ATSP2BQP2},  
координаты $y(ij,ij)$, $i,j\in[n-1]$, образуют квадратную 0/1"~матрицу,
у которой в каждой строке и каждом столбце имеется ровно одна единица.
В частности, проекция грани $F_3$ на переменные $y(ij,ij)$, $i,j\in[n-1]$, совпадает с многогранником Биркгофа $\Birk(n-1)$.

Для $\bm{x} \in \ATSP(n)$ положим
\begin{align*}
x_{n,j} &= y(1\,j, 1\,j),     & j&\in[n-1],\\
x_{j,n} &= y(n-1\,j, n-1\,j), & j&\in[n-1],\\
x_{j,l} &= \sum_{i\in[n-2]} y(i\,j, i+1\,l), & j,l&\in[n-1], \ j\neq l.
\end{align*}
Нетрудно проверить, что это линейное отображение проецирует $F_3$ на $\ATSP(n)$.
\end{proof}


\begin{lemma}[\cite{Maksimenko:2017LOP}]
	$\Stable(G) \lee \LOP(2n)$ для любого графа $G = (V,E)$, $|V|=n$.
\end{lemma}
\begin{proof}
	Рассмотрим грань $F$ многогранника $\LOP(2n)$, лежащую в пересечении опорных гиперплоскостей $x_{i, n+j} = x_{j, n+i} = 0$, $\{i,j\} \in E$, $i < j$.
	Тогда из 3-контурных неравенств (см.~\eqref{3cycle})
	\[
	0 \le x_{i, j} + x_{j, n+i} - x_{i, n+i} \quad \text{и} \quad
	x_{i, j} + x_{j, n+j} - x_{i, n+j} \le 1
	\]
	следует $x_{i, n+i} + x_{j, n+j} \le 1$, $\{i,j\} \in E$.
	То есть отображение $\alpha\from x_{i, n+i} \mapsto y_i$ проецирует грань $F$ в многогранник $\Stable(G)$.
	
	Остается показать, что для каждого $\bm{y} \in \Stable(G)$ найдется $\bm{x} \in F$ такой, что $\bm{y} = \alpha(\bm{x})$.
	
	Выберем произвольно $\bm{y} \in \Stable(G)$ и положим
	\[
	I_0 = \Set*{i \in [n] \given y_i = 0}, \quad I_1 = \Set*{i \in [n] \given y_i = 1}.
	\]
	Далее предполагаем, что элементы множеств $I_0 = \{i_1, \dots, i_k\}$ и $I_1 = \{i'_1, \dots, i'_{n-k}\}$ отсортированы (не важно как).
	Линейный порядок для соответствующей вершины $\bm{x} \in F$ представим перестановкой $\pi \from [n] \to [n]$ (см. условие \eqref{eq:piLinear}).
	Положим
	\begin{align*}
	\pi(n+i_s) &= s, &
	\pi(i_s) &= 2n-k + s, & s &\in [k],\\
	\pi(n+i'_t) &= n + t, &
	\pi(i'_t) &= k + t, & t &\in [n-k].
	\end{align*}
	Из описания перестановки $\pi$ следует, что 
	$x_{i_s, n+i_s} = 0$,  при $s \in [k]$, а
	$x_{i'_t, n+i'_t} = 1$, при $t \in [n-k]$.
	Причем, если $x_{i, n+j} = 1$ для некоторых $i,j \in [n]$, то $x_{i, n+i} = x_{j, n+j} = 1$.
	То есть из условия $x_{i, n+i} + x_{j, n+j} \le 1$ следует $x_{i, n+j} = x_{j, n+i} = 0$.
\end{proof}

Справедливость следующего утверждения очевидна.

\begin{lemma}
Пусть в семействе многогранников $\Pf$ множество вершин каждого многогранника вычисляется (генерируется) за полиномиальное относительно размера многогранника время.
%, а число вершин ограничено сверху полиномом от размерности (числа координат).
Тогда $\Pf$ расширенно аффинно сводится к семейству симплексов $\Delta$. %(Кодом симплекса является множество его вершин.)
\end{lemma}
%\begin{proof}
%\end{proof}

%%%%%%%%%%%%%%%%%%%%%%%%%%%%%%%%%%%%%%%%%%%%%%%%%%%%%%%%%%
%
%     Теорема Кука
%
%%%%%%%%%%%%%%%%%%%%%%%%%%%%%%%%%%%%%%%%%%%%%%%%%%%%%%%%%%

\section{Аналог теоремы Кука для многогранников}
\label{sec:Cook4Polytopes}

Пусть $C$~--- булева формула в конъюнктивной нормальной форме (см. определение на с.~\pageref{def:CNF}).
Обозначим через $\var(C)$ число переменных, участвующих в этой формуле.
Через $C(\bm{x})$ обозначим значение, возвращаемое этой формулой для набора значений переменных $\bm{x} \in \{0,1\}^{\var(C)}$.
Для множества всех таких булевых формул введем обозначение CNF.

Согласно теореме Кука~\cite{Cook:1971}, для любого языка $L \subseteq \{0,1\}^*$ из класса NP существует полиномиально вычислимое относительно $n\in\N$ преобразование $T = T(n)$, $T(n) \in \text{CNF}$, такое, что для всех $n\in\N$ выполнено $\var(T(n)) \ge n$, а для каждого $\bm{x} \in \{0,1\}^n$ справедливо %$\var(T_n) \ge n$ и 
\[
\bm{x}\in L  \iff \text{найдется }\bm{y}\in\{0,1\}^{\var(T)-n} \text{ такой, что } T(\bm{x},\bm{y}),
\]
где $T(\bm{x},\bm{y})$~--- значение, возвращаемое булевой формулой $T$ при подстановке набора значений переменных $(\bm{x},\bm{y})$.

Доказательство следующей теоремы полностью аналогично доказательству теоремы~\ref{thm:NP2comb}.

\begin{theorem}
Пусть тройка $(L,d,g)$ определяет некоторое семейство многогранников $\Pf$,
а предикат допустимости $g$ принадлежит классу NP.
Тогда это семейство расширенно аффинно сводится к семейству многогранников задачи о выполнимости $\SAT$.
\end{theorem}

В частности, все упоминаемые выше в настоящей работе семейства многогранников удовлетворяют условиям этой теоремы.

В предыдущей главе было показано, что булевы квадратичные многогранники аффинно сводятся к многим известным семействам многогранников, ассоциированным с NP-трудными задачами.
Покажем теперь, что все эти семейства расширенно аффинно сводятся к $\BQP$.

\begin{theorem}[\cite{Maksimenko:2012Cook}]
\label{thm:SAT2BQP}
Пусть $U$ "--- набор булевых переменных, $\cC$ "--- набор дизъюнкций над $U$, $\len(\cC)$ "--- суммарная длина всех дизъюнкций из набора $\cC$, измеряемая в литералах.
Тогда \(\SAT(U,\cC) \lee \BQP(n)\), где $n = |U| + \len(\cC)$.
\end{theorem}

%\emph{Доказательство.}
\begin{proof}
Воспользуемся идеей сведения задачи о выполнимости к задаче о клике из фундаментальной работы Карпа~\cite{Karp:1972}.
%Основой доказательства будет служить идея сведения задачи о выполнимости к задаче о клике из фундаментальной работы Карпа~\cite{Karp:1972}.

Вместо множества $[n]$ для индексации координат $x_{ij}$ вектора $\bm{x} \in \BQP(n)$ будем пользоваться множеством
\[
R = U \cup \Set*{(a, C_i) \given C_i \in \cC, \ \text{ $a$ "--- литерал, входящий в $C_i$}}.
\]
Соответственно, координаты вектора $\bm{x} \in \BQP(n)$ обозначаем $x(r, r')$, $r, r' \in R$.
В~частности, $x(r, r') = x(r, r) x(r', r')$ для всех $\bm{x} \in \BQP(n)$.

Рассмотрим подмножество вершин $F$ многогранника $\BQP(n)$, удовлетворяющих следующим ограничениям:
\begin{align}
x\bigl((u,C_i), (u,C_i)\bigr) &\le x\bigl(u,u\bigr), && C_i \in \cC, \quad u \in C_i, \label{eq:SAT2BQP1}\\
x\bigl(u,u\bigr) \cdot x\bigl((\bar{u},C_i), (\bar{u},C_i)\bigr) &= 0, && C_i \in \cC, \quad \bar{u} \in C_i, \label{eq:SAT2BQP2}\\
x\bigl((u,C_i), (u,C_i)\bigr) \cdot x\bigl((\bar{u},C_j), (\bar{u},C_j)\bigr) &= 0, && C_i, C_j \in \cC, \quad u \in C_i, \quad \bar{u} \in C_j, \label{eq:SAT2BQP3}\\
\sum_{a\in C_i} x\bigl((a,C_i), (a,C_i)\bigr) &= 1, && C_i \in \cC. \label{eq:SAT2BQP4}
\end{align}
Заметим, что $F$ является гранью многогранника $\BQP(n)$ (точнее, $\conv(F)$ является гранью $\conv(\BQP(n))$), так как каждое из этих ограничений эквивалентно принадлежности к некоторой опорной гиперплоскости. А именно, неравенство~\eqref{eq:SAT2BQP1} соответствует опорной гиперплоскости $x\bigl((u,C_i), (u,C_i)\bigr) = x\bigl(u,(u,C_i)\bigr)$, ограничение~\eqref{eq:SAT2BQP2} соответствует опорной гиперплоскости $x\bigl(u,(\bar{u},C_i)\bigr) = 0$, а~\eqref{eq:SAT2BQP3} соответствует $x\bigl((u,C_i), (\bar{u},C_j)\bigr) = 0$.
В свою очередь, уравнение~\eqref{eq:SAT2BQP4} определяется гиперплоскостью
\[
\sum_{a\in C_i} x\bigl((a,C_i), (a,C_i)\bigr) - 2 \sum_{\substack{a,b\in C_i,\\ a \ne b}} x\bigl((a,C_i), (b,C_i)\bigr) = 1.
\]

Обозначая координаты вектора $\bm{y} \in \SAT(U,\cC)$ через $y(u)$, $u\in U$,
рассмотрим аффинное отображение $\beta\from x(u,u) \mapsto y(u)$, проецирующее грань $F$ на $\SAT(U,\cC)$.
Включение $\beta(F) \subseteq \SAT(U,\cC)$ следует из того, что любой $\bm{x} \in \BQP(n)$, удовлетворяющий ограничениям \eqref{eq:SAT2BQP1}--\eqref{eq:SAT2BQP4}, соответствует некоторому набору значений переменных $U$, выполняющему каждую дизъюнкцию из $\cC$.
Обратное включение следует из того, что для любого набора значений переменных $U$, выполняющего $\cC$, несложно построить пример вектора $\bm{x} \in F$, координаты $x(u,u)$ которого принимают значения соответствующих переменных.
\end{proof}

\begin{corollary}
\label{cor:SAT2BQP}
Пусть четверка $(L,d,k,g)$ определяет некоторое семейство многогранников $\Pf$, а предикат допустимости $g$ принадлежит классу NP. Тогда это семейство расширенно аффинно сводится к семейству булевых квадратичных многогранников.
\end{corollary}

%Аналог теоремы Кука для многогранников~\cite{Maksimenko:2012} цитируется в работах~\cite{Seliverstov,Beasley:2013,Fiorini:2014,Huq:2016}. В диссертации~\cite{Huq:2016} этот результат назван теоремой Максименко. 

В~\cite{Beasley:2013} замечено, что из теоремы~\ref{thm:SAT2BQP} и результата S.~Burer~\cite{Burer:2009} следует, что любой 0/1-многогранник, чье множество вершин может быть описано полиномиальным предикатом, имеет коположительное расширение\footnote{Коположительное расширение многогранника $P$ "--- это пересечение аффинного подпространства и коположительного конуса $C_n = \{X \mid y^T X y \ge 0, \ \forall y \in \R_{+}^n\}$ такое, что $P$ является его проекцией. Двойственным к коположительному конусу является полностью положительный конус $C^*_n = \{X \mid X = \sum_{y \in Y} y y^T \text{ для некоторого конечного } Y \subset \R_{+}^n\}$, с которым связано понятие полностью положительного расширения.} полиномиального размера. 
А в~\cite{Fiorini:2014} теорема~\ref{thm:SAT2BQP} используется для того, чтобы дать ответ на следующий вопрос из работы S.~Burer~\cite{Burer:2009}: <<Кроме нескольких перечисленных задач, какие типы задач могут быть представлены как коположительные программы или полностью положительные программы?>> (\foreignlanguage{english}{``Other than the handful of problems listed above, what types of problems can be represented as COPs or as CPPs?''}).
