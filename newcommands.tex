% !TeX encoding = UTF-8 Unicode
% LaTeX команды

% New theorems
\theoremseparator{.} % Точка после заголовка теоремы
\theoremstyle{plain}
%\newtheorem{theorem}[equation]{Теорема} % Для единой нумерации формул и теорем
%\newtheorem{lemma}[equation]{Лемма}
\newtheorem{theorem}{Теорема}[chapter]
\newtheorem{lemma}[theorem]{Лемма}
\newtheorem{prop}[theorem]{Утверждение}
\newtheorem{corollary}[theorem]{Следствие}
\newtheorem{conjecture}[theorem]{Гипотеза}
\newtheorem{property}[theorem]{Свойство}
\newtheorem{question}[theorem]{Вопрос}
\newcommand{\thm}[2]{% Propositions for intro
	\begin{trivlist}
		\item[]\textbf{#1} \emph{#2}
	\end{trivlist}%
	}

%\theoremstyle{definition}
\theorembodyfont{\upshape}
\newtheorem{definition}{Определение}[chapter]
\newtheorem{remark}{Замечание}[chapter]
\newtheorem{example}{Пример}[chapter]

\theoremstyle{nonumberplain}
%\theoremseparator{}
\theoremsymbol{\rule{1ex}{1ex}}
\newtheorem{proof}{Доказательство}

% New commands
\newcommand{\Sum}{\sum\limits}
\newcommand{\eps}{\varepsilon}  %epsilon
\newcommand{\R}{\mathbb{R}}  %Set of real numbers
\newcommand{\N}{\mathbb{N}}  %Set of natural numbers
\newcommand{\Z}{\mathbb{Z}}  %Set of integer
\newcommand{\Q}{\mathbb{Q}}  %Set of rational
%\DeclareMathOperator{\P}{\textsf{P}}  %Probability
\renewcommand{\emptyset}{\varnothing} %Russian empty set
\newcommand{\NP}{\textup{NP}} 
\newcommand{\coNP}{\textup{co-NP}} 
\newcommand\op[1]{\mathop{\rm #1}\nolimits}
\renewcommand\vec[1]{\ensuremath{\mathbf{#1}}}

%\renewcommand\dim{\op{dim}}
\DeclareMathOperator*{\argmax}{argmax}
\DeclareMathOperator{\const}{const}
\DeclareMathOperator{\conv}{conv}
\DeclareMathOperator{\cone}{cone}
\DeclareMathOperator{\aff}{aff}
\DeclareMathOperator{\len}{len}
\DeclareMathOperator{\var}{var}
\DeclareMathOperator{\size}{size}
\DeclareMathOperator{\poly}{poly}
\DeclareMathOperator{\ext}{ext}
%\DeclareMathOperator{\pyr}{pyr}
\DeclareMathOperator{\Size}{Size}  
\DeclareMathOperator{\Lat}{\mathcal L} % face lattice
\DeclareMathOperator{\Pert}{Perturb} % perturbation
\DeclareMathOperator{\xc}{xc} %Extension complexity
\DeclareMathOperator{\rc}{rc} %Rectangle covering number
\DeclareMathOperator{\rank}{rank} %Matrix rank
\DeclareMathOperator{\sgn}{sgn} %Sign
\DeclareMathOperator{\supp}{supp} %Support
\DeclareMathOperator{\diam}{diam} %Graph diameter
\newcommand{\dc}{{\Delta_{c}(d,n)}} % Diameter of the ridge graph of a cyclic polytope
\DeclareMathOperator{\vertices}{vert} %The number of vertices
\DeclareMathOperator{\facet}{facet} %The number of facets
\DeclareMathOperator{\face}{face} %The number of faces
\renewcommand{\le}{\leqslant}         
\renewcommand{\ge}{\geqslant}         
\renewcommand{\leq}{\le}
\renewcommand{\geq}{\ge}
\newcommand{\lea}{\le_A} 
\newcommand{\nelea}{\not\le_A} 
\newcommand{\lee}{\le_E} 
\newcommand{\npropto}{\lefteqn{\;\not}\propto}
\newcommand{\scalar}[1]{\langle #1\rangle}
\newcommand{\from}{\colon}
\newcommand{\symdiff}{\bigtriangleup}
\newcommand{\compare}{\stackrel{?}{<}}

\newcommand{\tx}{\tilde x} % For the adjacency of shortest paths
\newcommand{\ty}{\tilde y}
\newcommand{\tz}{\tilde z}


\newcommand{\cC}{{\mathcal C}}  %КНФ
\newcommand{\K}{{\mathcal K}}  % Конусное разбиение
\newcommand{\Pf}{{\mathcal P}}  % Семейство многогранников
\newcommand{\Qf}{{\mathcal Q}}  % Семейство многогранников 2

\newcommand{\Cube}{\textup{Cube}} 
\newcommand{\Cross}{\textup{Cross}} 
\newcommand{\EP}{\textup{Э}} % Difficult problem
\newcommand{\CP}{{\mathcal C}} % Cyclic polytope
\newcommand{\CPO}{\textup{Ц}_{\textup{общ}}} % Cyclic polytope
\newcommand{\BQP}{P_{\textup{BQP}}} % Boolean quadric polytope (Correlation polytope)
\newcommand{\Cpert}{\textup{CP}} % Cyclic perturbation
\DeclareMathOperator{\M}{\eps} % perturbation 
\newcommand{\CBQP}{P_{\textup{CBQP}}} % Cyclic perturbation of Boolean quadric polytope
\newcommand{\BPP}{P_{\textup{BPP}}} % Boolean p-power polytope
\newcommand{\Tensor}{P_{\textup{tensor}}} % Tensor product polytope
\newcommand{\RBQP}{P_{\textup{RBQP}}} % Relaxation of Boolean quadric polytope
%\newcommand{\SSP}{\textup{STAB}} % Stable set polytope
\newcommand{\Stable}{P_{\textup{stab}}} % Stable set polytope
\newcommand{\TSP}{P_{\textup{TSP}}} % Travelling salesman polytope
\newcommand{\ATSP}{P_{\textup{ATSP}}} % Travelling salesman polytope
\newcommand{\HDP}{P_{\textup{HDP}}} % Hamiltonian dipath polytope
\newcommand{\Ham}{P_{\textup{Hgraph}}} % Hamiltonian graph polytope
\newcommand{\HDPst}{P_{\textup{s-t-HDP}}} % Hamiltonian dipath polytope
\newcommand{\HP}{P_{\textup{HP}}} % Hamiltonian path polytope
\newcommand{\HPst}{P_{\textup{s-t-HP}}} % Hamiltonian stpath polytope
\newcommand{\Path}{P_{\textup{path}}} % Path polytope
\newcommand{\Dipath}{P_{\textup{dipath}}} % DiPath polytope
\newcommand{\ShortP}{P_{\textup{shortpath}}} % Short Path polyhedron
\newcommand{\MinCut}{P_{\textup{mincut}}} % Min Cut polyhedron
\newcommand{\Cycle}{\textup{Cycle}} % Cycles in complete digraph
\newcommand{\Knap}{P_{\textup{knap}}} % Knapsack polytope
\newcommand{\KnapEq}{P_{\textup{eq}}} % Equality knapsack polytope
\newcommand{\PRT}{P_{\textup{numpart}}} % Numbers partitioning polytope
\newcommand{\SAT}{P_{\textup{sat}}} % SAT polytope
\newcommand{\KSAT}[1]{P_{\textup{#1-sat}}} % SAT polytope
\newcommand{\Tree}{P_{\textup{tree}}} % Spanning tree polytope
\newcommand{\Match}{P_{\textup{match}}} % Matching polytope
\newcommand{\Cut}{P_{\textup{cut}}} % Cut polytope
\newcommand{\Cubic}{P_{\textup{3-factor}}} % Cubic subgraph polytope
\newcommand{\QAP}{P_{\textup{QA}}} % Quadratic assignment polytope
\newcommand{\QSAP}{P_{\textup{QSA}}} % Quadratic semi-assignment polytope
\newcommand{\PAP}[1]{P_{\textup{$#1$-A}}} % 3-assignment polytope
\newcommand{\TAP}{P_{\textup{3-A}}} % 3-assignment polytope
\newcommand{\CAP}{P_{\textup{CA}}} % Constrained assignment polytope
\newcommand{\Pack}{P_{\textup{pack}}} % Set packing polytope
\newcommand{\Cover}{P_{\textup{cover}}} % Set covering polytope
\newcommand{\Part}{P_{\textup{part}}} % Set partition polytope
\newcommand{\NPadj}{P_{\textup{Matsui}}} % Matsui polytope
\newcommand{\DCP}{P_{\textup{2cover}}} % Double covering polytope
\newcommand{\POP}{P_{\textup{PO}}} % Partial ordering polytope
\newcommand{\LOP}{P_{\textup{LO}}} % Linear ordering polytope
\newcommand{\QLOP}{P_{\textup{QLO}}} % Quadratic linear ordering polytope
\newcommand{\ColorA}{P_{\textup{color1}}} % Graph coloring polytope
\newcommand{\ColorB}{P_{\textup{color2}}} % Graph coloring polytope
\newcommand{\ColorC}{P_{\textup{color3}}} % Graph coloring polytope
\newcommand{\ColorD}{P_{\textup{color4}}} % Graph coloring polytope
\newcommand{\Clique}{P_{\textup{clique}}} % Clique polytope
\newcommand{\Perm}{P_{\textup{perm}}} % Permutahedron
\newcommand{\Birk}{P_{\textup{Birk}}} % Birkhoff polytope
\newcommand{\Steiner}{P_{\textup{Steiner}}} % Steiner tree polytope

%\newcommand{\bigO}{\mathcal{O}}  % Big O 


%%%%%%%%%%%%%%%%%%%%%%%%%%%%%%%%%%%%%%%%
%% Заплатка для правильной работы неразрывного дефиса "~ и короткого тире "--
%% russianb.ldf        begin
%%%%%%%%%%%%%%%%%%%%%%%%%%%%%%%%%%%%%%%%
\makeatletter
\newcommand*{\glue}{\nobreak\hskip\z@skip}%  NEW!!!
%\declare@shorthand{russian}{"~}{\textormath{\leavevmode\hbox{-}}{-}}%  OLD!!!
\declare@shorthand{russian}{"~}{\glue\hbox{-}\glue}%  NEW!!!
\def\cdash#1#2#3{\def\tempx@{#3}%
	\def\tempa@{-}\def\tempb@{~}\def\tempc@{*}%
	\ifx\tempx@\tempa@\@Acdash\else
	\ifx\tempx@\tempb@\@Bcdash\else
	\ifx\tempx@\tempc@\@Ccdash\else
	%\errmessage{Wrong usage of cdash}%  OLD!!!
	\@Dcdash#3\fi\fi\fi}%  NEW!!!
%\def\@Acdash{\ifdim\lastskip>\z@\unskip\nobreak\hskip.2em\fi
%  \cyrdash\hskip.2em\ignorespaces}%
%\def\@Bcdash{\leavevmode\ifdim\lastskip>\z@\unskip\fi%  OLD!!!
% \nobreak\cyrdash\penalty\exhyphenpenalty\hskip\z@skip\ignorespaces}%  OLD!!!
%\def\@Ccdash{\leavevmode
% \nobreak\cyrdash\nobreak\hskip.35em\ignorespaces}%
\def\@Bcdash{\,\textendash\,\hskip\z@skip\ignorespaces}%  NEW!!!
\def\@Dcdash#1{\,\textendash\,\hskip\z@skip\ignorespaces#1}%  NEW!!!
\makeatother
%%%%%%%%%%%%%%%%%%%%%%%%%%%%%%%%%%%%%%%%
%% russianb.ldf        end
%%%%%%%%%%%%%%%%%%%%%%%%%%%%%%%%%%%%%%%%


\usepackage{mathtools} % Разного рода тюнинг для мат формул
%\mathtoolsset{showonlyrefs} % Нумеровать только цитируемые уравнения
% ,showmanualtags} % для нумерации в ручном режиме
%\RequirePackage{mathtools} % Дополнительные возможности для набора формул
\providecommand\given{} % Для вертикальной черты, определяющей начало условия
% Создаем команду для оформления множеств
\newcommand\SetSymbol[1][]{%
	\nonscript\:#1\vert
	\allowbreak \nonscript\:	\mathopen{}}
\DeclarePairedDelimiterX\Set[1]\{\}{%
	\renewcommand\given{\SetSymbol[\delimsize]}	#1} 
