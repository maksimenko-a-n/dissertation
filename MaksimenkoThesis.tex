% !TeX encoding = UTF-8 Unicode
%
% Внимание, при компиляции возникает ошибка, 
% связанная с выбором языка в .toc
% Проблема решается удалением строки \select@language {russian} из thesis.toc
% и включением декларации \nofiles.
% Или же закомментариванием \tableofcontents, как временная мера
% ПРАВИЛЬНЫЙ способ решения проблемы: \renewcommand\tocsectionfont{}
%
% Еще одна проблема -- опция times каким-то чудесным образом отключает пакет cmap
% Проблема находится в строках 
% \renewcommand\rmdefault{ftm} и \renewcommand\ttdefault{cmtt}
% после строки \usepackage{pscyr} в файле disser.cls
% Решение:
% \input glyphtounicode.tex 
% \pdfgentounicode=1
%
\documentclass[%
%draft,
doctor,         % тип документа
natbib,         % использовать пакет natbib для "сжатия" цитирований
%subf,           % использовать пакет subcaption для вложенной нумерации рисунков
pagebackref,    % обратные ссылки в списке литературы -- номера страниц 
href,           % использовать пакет hyperref для создания гиперссылок
%colorlinks=true, % цветные гиперссылки
%,fixint=false  % отключить прямые знаки интегралов
%,classified    % гриф секретности
%,libcat        % номер УДК
%,facsimile     % отображать факсимиле диссертанта
]{disser}

% Замена cmap
%\input glyphtounicode.tex 
%\pdfgentounicode=1
%\usepackage{cmap} %!!! cmap не работает ни здесь ни в стилевом файле

%\usepackage{mmap} % Рекомендация ВАК
\usepackage[T2A]{fontenc}
\usepackage[utf8]{inputenc}
\usepackage[english,russian]{babel}
%\usepackage[a-1b,usecharset]{pdfx} % Рекомендация ВАК
\usepackage[a-1b]{pdfx} % Рекомендация ВАК

%\usepackage{xcolor} В disser.cls нужно заменить пакет color на xcolor
\usepackage{tikz}
\usetikzlibrary{calc, graphs,babel,arrows.spaced, arrows.meta, bending}
\usetikzlibrary{shapes.misc} % for cross
%\usetikzlibrary{}

\renewcommand\tocsectionfont{}

\usepackage[
  a4paper, mag=1000,
  left=2.5cm, right=1cm, top=2cm, bottom=2cm, headsep=0.7cm, footskip=1cm
]{geometry}

\usepackage{verbatim} % Для окружения comment
\usepackage[intlimits]{amsmath}
\usepackage{amssymb,amsfonts}
%\usepackage{amsthm} % Конфликт с пакетом disser. Уже определена команда \openbox
\usepackage[amsmath, thref, hyperref, thmmarks]{ntheorem} % Для окружений типа Теорема
\usepackage{bm} % Для выделения жирным шрифтом математических символов
\usepackage{array}
\usepackage[linesnumbered,lined,ruled]{algorithm2e} % Для оформления псевдокода
\usepackage{placeins} % Для \FloatBarrier

\usepackage{paralist} % Для компактных по высоте списков (перечислений)

%\hypersetup{pagebackref = true, backref = page} % Обратные ссылки из списка литературы, нужно включать эту опцию сразу с подключением hyperref. Иначе (через \hypersetup) она не работает.

% Что-то новое ?????????????
%\ifpdf\usepackage{epstopdf}\fi
%\usepackage[autostyle]{csquotes}

% Список сокращений и условных обозначений
\usepackage[intoc,nocfg,russian]{nomencl}
\newcommand{\nomencl}[2]{#1 --- #2\nomenclature{#1}{#2}}
\setlength{\nomlabelwidth}{3em}
\setlength{\nomitemsep}{-\parsep}
\renewcommand{\nomlabel}[1]{#1 ---}
\makenomenclature

% Шрифт Times в тексте как основной
\usepackage{tempora}
% Команда \textsc{} НЕ РАБОТАЕТ
\newcommand{\problem}[1]{\emph{#1}}
\renewcommand{\textsc}[1]{\textbf{#1}}
% альтернативный пакет из дистрибутива TeX Live
%\usepackage{cyrtimes}

% Шрифт Times в формулах как основной
\usepackage[varg,cmbraces,cmintegrals]{newtxmath}
% альтернативный пакет
%\usepackage[subscriptcorrection,nofontinfo]{mtpro2}

% Номера страниц снизу и по центру
%\pagestyle{footcenter}
%\chapterpagestyle{footcenter}

% Точка с запятой в качестве разделителя между номерами цитирований
%\setcitestyle{semicolon}

% Ссылки на работы соискателя включаются в общий список литературы
\let\citemy=\cite
%\let\citeown=\cite

% Использовать полужирное начертание для векторов
\let\vec=\mathbf

% Путь к файлам с иллюстрациями
%\graphicspath{{fig/}}

%\setcounter{tocdepth}{2} % Глубина содержания. Значение 2 вызывает ошибку

% LaTeX �������

% New theorems
\theoremseparator{.} % ����� ����� ��������� �������
\theoremstyle{plain}
\newtheorem{theorem}{�������}[chapter]
\newtheorem{lemma}[theorem]{�����}
\newtheorem{prop}[theorem]{�����������}
\newtheorem{corollary}[theorem]{���������}
\newtheorem{conjecture}[theorem]{��������}
\newtheorem{property}[theorem]{��������}
\newtheorem{question}[theorem]{������}

%\theoremstyle{definition}
\theorembodyfont{\upshape}
\newtheorem{definition}{�����������}[chapter]
\newtheorem{remark}{���������}[chapter]
\newtheorem{example}{������}[chapter]

\theoremstyle{nonumberplain}
%\theoremseparator{}
\theoremsymbol{\rule{1ex}{1ex}}
\newtheorem{proof}{��������������}

% New commands
\newcommand{\Sum}{\sum\limits}
\newcommand{\eps}{\varepsilon}  %epsilon
\newcommand{\R}{\mathbb{R}}  %Set of real numbers
\newcommand{\N}{\mathbb{N}}  %Set of natural numbers
\newcommand{\Z}{\mathbb{Z}}  %Set of integer
\newcommand{\Q}{\mathbb{Q}}  %Set of rational
\renewcommand{\emptyset}{\varnothing} %Russian empty set
\newcommand{\NP}{\textup{NP}} 
\newcommand{\coNP}{\textup{co-NP}} 
\newcommand\op[1]{\mathop{\rm #1}\nolimits}
\renewcommand\vec[1]{\ensuremath{\mathbf{#1}}}

%\renewcommand\dim{\op{dim}}
\DeclareMathOperator*{\argmax}{argmax}
\DeclareMathOperator{\const}{const}
\DeclareMathOperator{\conv}{conv}
\DeclareMathOperator{\cone}{cone}
\DeclareMathOperator{\aff}{aff}
\DeclareMathOperator{\len}{len}
\DeclareMathOperator{\var}{var}
\DeclareMathOperator{\size}{size}
\DeclareMathOperator{\poly}{poly}
\DeclareMathOperator{\ext}{ext}
%\DeclareMathOperator{\pyr}{pyr}
\DeclareMathOperator{\Size}{Size}  
\DeclareMathOperator{\Lat}{\mathcal L} % face lattice
\DeclareMathOperator{\Pert}{Perturb} % perturbation
\DeclareMathOperator{\xc}{xc} %Extension complexity
\DeclareMathOperator{\rc}{rc} %Rectangle covering number
\DeclareMathOperator{\rank}{rank} %Matrix rank
\DeclareMathOperator{\sgn}{sgn} %Sign
\DeclareMathOperator{\supp}{supp} %Support
\DeclareMathOperator{\diam}{diam} %Graph diameter
\newcommand{\dc}{{\Delta_{c}(d,n)}} % Diameter of the ridge graph of a cyclic polytope
\DeclareMathOperator{\vertices}{vert} %The number of vertices
\DeclareMathOperator{\facet}{facet} %The number of facets
\DeclareMathOperator{\face}{face} %The number of faces
\renewcommand{\le}{\leqslant}         
\renewcommand{\ge}{\geqslant}         
\renewcommand{\leq}{\le}
\renewcommand{\geq}{\ge}
\newcommand{\lea}{\le_A} 
\newcommand{\nelea}{\not\le_A} 
\newcommand{\lee}{\le_E} 
\newcommand{\npropto}{\lefteqn{\;\not}\propto}
\newcommand{\scalar}[1]{\langle #1\rangle}
\newcommand{\from}{\colon}
\newcommand{\symdiff}{\bigtriangleup}
\newcommand{\compare}{\stackrel{?}{<}}

\newcommand{\tx}{\tilde x} % For the adjacency of shortest paths
\newcommand{\ty}{\tilde y}
\newcommand{\tz}{\tilde z}


\newcommand{\cC}{{\mathcal C}}  %���
\newcommand{\K}{{\mathcal K}}  % �������� ���������

\newcommand{\Cube}{\textup{Cube}} 
\newcommand{\Cross}{\textup{Cross}} 
\newcommand{\EP}{\textup{�}} % Difficult problem
\newcommand{\CP}{{\mathcal C}} % Cyclic polytope
\newcommand{\CPO}{\textup{�}_{\textup{���}}} % Cyclic polytope
\newcommand{\BQP}{P_{\textup{BQP}}} % Boolean quadric polytope (Correlation polytope)
\newcommand{\Cpert}{\textup{CP}} % Cyclic perturbation
\DeclareMathOperator{\M}{\eps} % perturbation 
\newcommand{\CBQP}{P_{\textup{CBQP}}} % Cyclic perturbation of Boolean quadric polytope
\newcommand{\BPP}{P_{\textup{BPP}}} % Boolean p-power polytope
\newcommand{\Tensor}{P_{\textup{tensor}}} % Tensor product polytope
\newcommand{\RBQP}{P_{\textup{RBQP}}} % Relaxation of Boolean quadric polytope
%\newcommand{\SSP}{\textup{STAB}} % Stable set polytope
\newcommand{\Stable}{P_{\textup{stab}}} % Stable set polytope
\newcommand{\TSP}{P_{\textup{TSP}}} % Travelling salesman polytope
\newcommand{\ATSP}{P_{\textup{ATSP}}} % Travelling salesman polytope
\newcommand{\HDP}{P_{\textup{HDP}}} % Hamiltonian dipath polytope
\newcommand{\Ham}{P_{\textup{Hgraph}}} % Hamiltonian graph polytope
\newcommand{\HDPst}{P_{\textup{s-t-HDP}}} % Hamiltonian dipath polytope
\newcommand{\HP}{P_{\textup{HP}}} % Hamiltonian path polytope
\newcommand{\HPst}{P_{\textup{s-t-HP}}} % Hamiltonian stpath polytope
\newcommand{\Path}{P_{\textup{path}}} % Path polytope
\newcommand{\Dipath}{P_{\textup{dipath}}} % DiPath polytope
\newcommand{\ShortP}{P_{\textup{shortpath}}} % Short Path polyhedron
\newcommand{\MinCut}{P_{\textup{mincut}}} % Min Cut polyhedron
\newcommand{\Cycle}{\textup{Cycle}} % Cycles in complete digraph
\newcommand{\Knap}{P_{\textup{knap}}} % Knapsack polytope
\newcommand{\KnapEq}{P_{\textup{eq}}} % Equality knapsack polytope
\newcommand{\PRT}{P_{\textup{numpart}}} % Numbers partitioning polytope
\newcommand{\SAT}{P_{\textup{sat}}} % SAT polytope
\newcommand{\KSAT}[1]{P_{\textup{#1-sat}}} % SAT polytope
\newcommand{\Tree}{P_{\textup{tree}}} % Spanning tree polytope
\newcommand{\Match}{P_{\textup{match}}} % Matching polytope
\newcommand{\Cut}{P_{\textup{cut}}} % Cut polytope
\newcommand{\Cubic}{P_{\textup{3-factor}}} % Cubic subgraph polytope
\newcommand{\QAP}{P_{\textup{QA}}} % Quadratic assignment polytope
\newcommand{\QSAP}{P_{\textup{QSA}}} % Quadratic semi-assignment polytope
\newcommand{\PAP}[1]{P_{\textup{$#1$-A}}} % 3-assignment polytope
\newcommand{\TAP}{P_{\textup{3-A}}} % 3-assignment polytope
\newcommand{\CAP}{P_{\textup{CA}}} % Constrained assignment polytope
\newcommand{\Pack}{P_{\textup{pack}}} % Set packing polytope
\newcommand{\Cover}{P_{\textup{cover}}} % Set covering polytope
\newcommand{\Part}{P_{\textup{part}}} % Set partition polytope
\newcommand{\NPadj}{P_{\textup{matsui}}} % Matsui polytope
\newcommand{\DCP}{P_{\textup{2cover}}} % Double covering polytope
\newcommand{\POP}{P_{\textup{PO}}} % Partial ordering polytope
\newcommand{\LOP}{P_{\textup{LO}}} % Linear ordering polytope
\newcommand{\QLOP}{P_{\textup{QLO}}} % Quadratic linear ordering polytope
\newcommand{\ColorA}{P_{\textup{color1}}} % Graph coloring polytope
\newcommand{\ColorB}{P_{\textup{color2}}} % Graph coloring polytope
\newcommand{\ColorC}{P_{\textup{color3}}} % Graph coloring polytope
\newcommand{\ColorD}{P_{\textup{color4}}} % Graph coloring polytope
\newcommand{\Clique}{P_{\textup{clique}}} % Clique polytope
\newcommand{\Perm}{P_{\textup{perm}}} % Permutahedron
\newcommand{\Birk}{P_{\textup{birk}}} % Birkhoff polytope
\newcommand{\Steiner}{P_{\textup{steiner}}} % Steiner tree polytope

%\newcommand{\bigO}{\mathcal{O}}  % Big O 


%%%%%%%%%%%%%%%%%%%%%%%%%%%%%%%%%%%%%%%%
%% �������� ��� ���������� ������ ������������ ������ "~ � ��������� ���� "--
%% russianb.ldf        begin
%%%%%%%%%%%%%%%%%%%%%%%%%%%%%%%%%%%%%%%%
\makeatletter
\newcommand*{\glue}{\nobreak\hskip\z@skip}%  NEW!!!
%\declare@shorthand{russian}{"~}{\textormath{\leavevmode\hbox{-}}{-}}%  OLD!!!
\declare@shorthand{russian}{"~}{\glue\hbox{-}\glue}%  NEW!!!
\def\cdash#1#2#3{\def\tempx@{#3}%
	\def\tempa@{-}\def\tempb@{~}\def\tempc@{*}%
	\ifx\tempx@\tempa@\@Acdash\else
	\ifx\tempx@\tempb@\@Bcdash\else
	\ifx\tempx@\tempc@\@Ccdash\else
	%\errmessage{Wrong usage of cdash}%  OLD!!!
	\@Dcdash#3\fi\fi\fi}%  NEW!!!
%\def\@Acdash{\ifdim\lastskip>\z@\unskip\nobreak\hskip.2em\fi
%  \cyrdash\hskip.2em\ignorespaces}%
%\def\@Bcdash{\leavevmode\ifdim\lastskip>\z@\unskip\fi%  OLD!!!
% \nobreak\cyrdash\penalty\exhyphenpenalty\hskip\z@skip\ignorespaces}%  OLD!!!
%\def\@Ccdash{\leavevmode
% \nobreak\cyrdash\nobreak\hskip.35em\ignorespaces}%
\def\@Bcdash{\,\textendash\,\hskip\z@skip\ignorespaces}%  NEW!!!
\def\@Dcdash#1{\,\textendash\,\hskip\z@skip\ignorespaces#1}%  NEW!!!
\makeatother
%%%%%%%%%%%%%%%%%%%%%%%%%%%%%%%%%%%%%%%%
%% russianb.ldf        end
%%%%%%%%%%%%%%%%%%%%%%%%%%%%%%%%%%%%%%%%


\usepackage{mathtools} % ������� ���� ������ ��� ��� ������
%\mathtoolsset{showonlyrefs} % ���������� ������ ���������� ���������
% ,showmanualtags} % ��� ��������� � ������ ������
%\RequirePackage{mathtools} % �������������� ����������� ��� ������ ������
\providecommand\given{} % ��� ������������ �����, ������������ ������ �������
% ������� ������� ��� ���������� ��������
\newcommand\SetSymbol[1][]{%
	\nonscript\:#1\vert
	\allowbreak \nonscript\:	\mathopen{}}
\DeclarePairedDelimiterX\Set[1]\{\}{%
	\renewcommand\given{\SetSymbol[\delimsize]}	#1} 


\begin{document}

% Переопределение стандартных заголовков
%\def\contentsname{Содержание}
%\def\conclusionname{Выводы}
%\def\bibname{Литература}

% Включение файла с общим текстом диссертации и автореферата
% (текст титульного листа и характеристика работы).
% ����� ���� ���������� ����� ����������� � ������������
\institution{%����� ��� <<
����������� ��������������� ����������� ��.~�.\,�.~��������}%>>}

\topic{������������� �������� �������������� �������������� �����}

\author{���������� ��������� ����������}

\specnum{01.01.09}
\spec{���������� ���������� � �������������� �����������}

%\scon{���������� �.\,�.}
%\sconstatus{�.~�.-�.~�., ����.}
%\sconsnd{��� ������� ������������}
%\sconsndstatus{�.~�.-�.~�., ����.}

\city{���������}
\date{\number\year}

% ����� ������� ������������ � �����������
\mkcommonsect{object}{������ ������������.}{%
������������ \emph{������ ������������� ����������� � �������� ������� ��������} ��������� �������.
���� �������� ��������� $E$, ������� �������� $e$ �������� �������� ��������� ��� $c_e \in \R$, 
� ������������� ���������� ������� (��������) $f \colon 2^E \to \{\text{����}, \text{������}\}$.
������������ $s \subseteq E$ ���������� \emph{���������� ��������} ���� ������,
���� $f(s)$ �������.
��������� ���� ���������� ������� ��������� $S$, $S = \{s \subseteq E \mid f(s)\}$.
%, ������������ ��������� $S = \{s \subseteq E \mid f(s) = 1\}$ ���� ���������� ������� ������.
���� ������ ������� � ��������� \emph{������������} ������� $s \in S$ � ������������ (�����������) ��������� ����� ���������.

��� �������, ���������� ������� ������������ \emph{����� ������}, ����� $S$ �����������, 
� ���� $c_e$ ���������� �����������.
����� �������, ����� ������ ���������� ������������ ���������� ���� ����� ���������� ������� $S$,
� ���� $c_e$ ������������ ����� ����� ������� ������ \emph{������� ������}.
����� ��� ����������� ����� ������ ����� ������������ ������������ ��������� � �������~$S$.

��������, � ������ � ���������� ���� ��������� $E$ ���� ��������� �����, ����������� ������,
���� $c_e$ �������� ������� �����, � �������� $f$ ��������� �������� <<������>> ��� ������� ��������� �����,
��������������� ����� ������� �� ������ $A$ � ����� $B$.
������� ������������� ��������� �������� ������ ������ ������������ ��������� ������,
������ � �����������, ������ ������������, ������ � ������� � ������ ������.

%������� ����������, ��� ��������� ��������� ���� ������, ����� ������.
���� ����� ����� ������ ����������� � ��������� ��� ����������� ������������� ��������,
������������ ������������ �������������� � ������������, ����������� �������� ������, ����������� ���������� �~�.\,�.~\cite{Paschos:2014}. 
����� ����, � ��������� ����� ������ ������������� ����������� ����������� ����������� �����: 
�������������� ������, ������������� ������������� �����, ������������� �������, 
������������ ���������������� � ������������� �����, ���������������� ���������,
������ � ������������ ������������ ������������ ���� (VLSI) � �������� ����, 
������������, �������� �������� �~�.\,�.~\cite{GrotschelCO:1995}.

��� ��������~\cite{SchrijverCO:2003}, �� ������ ������� ����� ������ ������� ���������� �� ���� ���������.
� ������, ��� ������� ����������� ������� $s \in S$ ��������������� ��� \emph{������������������ ������}
$\bm{x} \in \{0,1\}^E$, ���������� $x_e$, $e \in E$, �������� ���������� ������� ������� ��� $e \in s$
� ����� ���� ��� $e \not\in s$. 
����� ��������� ���� ������������������ �������� ���������� $X$.
����� ����� �������������� � ���� ������� $\bm{c} = (c_e) \in \R^E$.
���� ������ ��� ����� ������������� ����������� � ������ ������������������� ������� $\bm{x} \in X$,
�� ������� ������� ������� $\bm{c}^T \bm{x}$ ��������� ������������� (������������) ��������.
����� ��, ��� ������������� �������� ������� ������� �� �������� ��� ������ ������� ����������� $X$
� �������� ��������� $\conv(X)$.
����� �������, � ������ ������� ������������� ����������� ������������� ��������� �������� ������������ $\conv(X)$, ��������� �������� ������ ��������� $X$.
�������������, ������������ ����� ��������, ������� �������� ��������������.

� ����� �������, �������������� ������������� ���� ����������� ��� ������� ������ ������������ ���������� ��������������� �������������, � ���������, �������� ��������� ����������������.
��� ���������� ���� �������������� ������������ � ���� �������, �� ������ ������� �������������� ������ ��������� �� ������� ����������� �������� � �������� �������~\cite{SchrijverCO:2003}.

� ������ �������, ������������� ��������� ������������� �������� 
��������� ��������� ���������� ������� ��������������� ������.
%��� ��������� ����������� ��� ������������ ������� ��������� ������.
��������������� ��� ����� � ������� ������� ������� �������.
���������� ������� $s_1$ � $s_2$ ������ $S$ ���������� \emph{��������},
���� ��� ���������� ������ ����� $c_e$ ��� ��� ������� �������� ������������
��� ��������������� ������� ������ � ������ ����������� ������� � �� ���.
��������� ������� $s_1$ � $s_2$ ��������, ��� ��� �������������� ���������
������� ������ ������� ������ � ����������� ������� ����� �������� � $s_1$ �� $s_2$ � �������.
�� ���� � ���������, �������� ������, ������ ���� ������������� ��������,
�������������� � ����� ����������.
���, � ���� �������, ����������� ��� ����������� �� ��������� ���������, ��������� ����� ������ $S$,
�, � �����, ����� �������������� ��� ������������� ������ ��������� ������.
��� �������������� ������� ��������� ������� ���������������� ��� ������� �����
����� ���������������� ��������� ������������� $\conv(X)$.
����� �������, ���� ������������� ������ �������� � ���� ��������� ���������� � � ����������� ���������. 
��������� ��� �����, ������ ����������� ���������� � �������� ������� ���� ������� ������
(��������� ���� ������, ������������� �� ���������) ������������� ������.
��� � �������� �������� �������� ������������ ��������� ������.
}

% ����� ������� ������������ � �����������
\mkcommonsect{actuality}{������������ ���� ������������.}{%
������� ������� � ������� ������������� ����������� �������� � 1950-� ��.
��� ������������� ���� ����������� ����� ���������:
�������� ������ ��������������� ���, �������� ��������� ��������� ���������������� ������������ � ��������� � ���������� ��������-������ ��������,
� ����� ����������� ���� ������� ��������������� ����� �� ������.
������ � 1950-� ��. ��� ���������� ���������� ����� ������� ������ � �����������,
�������� �����-���������� ��� ������ � ������������ ������, 
��������� ���������� ��� ���������� ���������� �����,
������ ������� ������ � ���������� ����� ��������� ��� ���������� ������������ ��������� ������,
� ����� �� ��������� ��� ���������� � �������� ���� ������������ ����~\cite{SchrijverHistory:2005}.
� ��� �� �����, ����� ���������� �������� ��������-������, ���� ����������� ������ �������� �������
��� ���������� � ��������� ������� ������������� �����������.
� ���������, � ������� ������� ��������� ���������������� ��� ��������� ������������ �� ��� �������� �������� � ������� ������ ������������~\cite{DantzigFJ:1954}.

�������� ���������� ��������-������ ����� ����������
�������������� ������������ � ��� ������������� �������������.
���, ��������, ���� ��������, ��� ������ ������� ����� ����� ��������-������
����� ������� ������� ����� �������������.
� ����� � ���� � 1957 ���� ���� �������������� ���������� �������� ����� � ���,
��� ������� ����� ������������� �� ����� ���� ������ �������� ����� ������ ��� ����������� � ������������.
� ��� ��� ���� �������� ��������� ������������ ��������, �� ���� � 2010 ���� �������
������� ��������� ������ 43-������� ������������� � 86 ������������, ������� �������� ������, ��� 43~\cite{Santos:2012}.
��� �� �����, � ����� ����\footnote{����� ��, ��� ������� ����� ��������� ������ ��������� �� ����� ����������� � ����������� �������������?} ��� �������� �� ��� ��� �������� �������� � ���������� �������� ������ ������~\cite{ZieglerHirsch:2012}.

����� �� ����������� ������ ��������� ������ �������� ������ �������� �������� �����
�������������, ������������� � ������������. 
� 1964~�. ���~\cite{Klee:1964} �����������, ��� ���� ������� ����� $\lfloor n/2\rfloor$,
��� $n$~--- ����� �����������. �� ����� ������ ������� �� �� ��� ������ �����������~\cite{Klee:1967}.
� ��� ��� ������ ���������� ��������.

������ � �������� � ��������� � ������� ��������� ����� � 1950--60-�~��.
������������� ������� ������������ ���������, ������������ ���������������� � ������� 
��������~\cite{Edmonds:1965} � �������~\cite{Cobham:1964}.
� ��� �� ����� �������~\cite{Edmonds:1965b} ���� ������� ������, ������� <<������� ��������������>>,
���, ��-����, �������� ������������ ����, ��� ������� ���� ������� ������� NP.
�� ��� ��������� ������������� � �������� � ������ 1970-� ��. �����~\cite{Cook:1971}, ������~\cite{Karp:1972} � �������~\cite{Levin:1973} NP-������ �����~\cite{Garey:1982}.
������������� ��, ��� ������ �� ��� ��������� ���� ������ ����\'���� �����.
� ���������, ������� �������� ����������, ���������� ����� �� �������� ����������� ��������� ������, �� ����� ���� ����� ����� � ���������� ������.

�������� NP-������ ����� ��������� ������ ������� ��� ���������� ������������,
� ��� ����� ������� ��������������, ��������������� � NP-�������� ��������.
� 1978 ���� ������������~\cite{Papadimitriou:1978} �������, 
��� ������ �������� ���������\footnote{������, �����������} 
���� ����������� ������ ������ ������������� ������ ������������ NP-�����,
�.\,�. ��� ����� ������, ��� � ���� ������ ������������.
������� ����������� ���������� ��� �������������� ��������� ������ NP-������� �����
���� �������� ������, ����� � �����, �����, ���������� � ������, ������� � �����, �������~\cite{Maksimenko:2013NP}.
� ������ �������, �~1975~�. ������ �������~\cite{Chvatal:1975}, ��� ��� ������������� ����������� �������� ��� ������ ������������� ���������.
����� ����, �~1984~�. ������� �������~\cite{Greshnev:1984}, ��� ���� ������������� ������ �� $m$-��������� ����� �����, �.\,�. ������ �������� ��������� ������ ��� ���� ����������.
�~1986~�. ����������� ��������� ��� ������������� ������ � ������������ �������
��� ���������� ������� �����������~\cite{Beloshevskii:1986} � �������� � ������~\cite{Barahona:1986}.

%��� ��������� ������������, ��� ��� ����� ����� ���� � �� �� �������������� �������,
%�������������� � ������� �������� ����������~\cite{Maksimenko:2013NP}.
%�����������, ��� ��� ����� ����� ���� � �� �� �������������� �������,
%�������������� � ������� �������� ����������~\cite{Maksimenko:2013NP}.
%����� ��������� ������ �������, ��� ��� ����������� � ���� ������� �������������
%�������� � �������� ����� ������������ ������� ��������~\cite{Maksimenko:2013NP},
%������ �������� ��������� ������ �������� NP-�����~\cite{Matsui:1995}.

� 1979 ���� ������~\cite{Khachiyan:1979} �������� �������������� �������� ��� ������� ������ ��������� ����������������.
���� ���� ���� ������������� �������������� ������������� ��������������� ������� � ������� ����� ������������� �����������, ��� ��������� ������������ ������������ ������� ������������� ��������������.
� ���������, ������� �������� ������������ ������� ������ �������������� ������� � ���������� ���������, �������� � ��������~\cite{Emelichev:1981}.

� 1980-� ��. ����������~\cite{BondBook:1995} ��� ������������ ������� ������ �������������� ����� ���������, ��� �������� �����\footnote{� ��������� ��� �������������� ���������� ���������� �����} ����� ������������� ������ ������������� ��������������� ��������� ��������������� ��������������� ������. 
� ������, �� ���� �������� ��������������������� �������� ����� ������ �������������� ��������� NP-������� �����: �����������, ������������ ����� � ��������� ��������, 3-���������.
� ������ �������, �������� ����� ������ �������������� ��������� ������������� ��� ��������� ������������� ���������� �����: ����������, ����������� �������� ������, ������ � ���������� ����.
�� ��������� ���� ������ ���� ����������� ������ ���������� ������� ����,
������������, ��� �������� ����� �������� ������ ������� ��������� � ��������� <<������� ������ ����������>>~\cite{BondBook:1995}. 
������� ���������� � ��� ������� ��������� ������ ������ �������� ����� 
����������� ������ ������������~\cite{Antonov:2012, Nikolaev:2013, Shovgenov:2015}.
�� ������ ������� �������� ����������, ���������� ������� ��������� ������,
������������ ���������� ����� ����������� ���, ��� ��� ������������� �������������
�������� � �������� ����� ����� ������������ ������������,
�������� ����� ����� �������� ���������������.

� ������ �������, �������� �������� ��������������� ��������� ��� ������ ��������� ����������������
�������� ����� ������� ������ ����������� ��������� �������� ��� ������������� ������ ������������.
��� ��� ������� ���� ���������� �� ������������� ���� ������������ ������������� ������������� 
(��� ������ �������� �������). 
\emph{����������� ��������������} ������������� $P$ ���������� ����� �������� �����������,
����������� ������������ $Q$ �����, ��� $P$ �������� ������������� ��������� $Q$.
��� ������������ $Q$ ���������� \emph{�����������} ������������� $P$.
� ���� ������� ��� ���� �������� �������, ����� ����� �������� ���������� ����������� ��� �������� ������������� ���������������, � ��� ��� ����������~--- ������������� ������������ ����� ������� ������ ������.
�� ���� �� ����� ������� �� ������� � ������ � � 1988 ���� ���������~\cite{Yannakakis:1988}
�������, ��� ����� ������� � �������� �� ����� �����������, 
���� ������������ ����������� ������������� ������������� ��������� ������������ �������� ���������.
��� �� ���� ��������� ��������, ��� ����������� �������� ������������ � ��� ������� ���������.
����� ����, ��������� �������, 
��� ����� �������� ���������� � ����������� ������������� ������������� �� ����� ���� ������, 
��� ����� �������������� �������� ������� ���������� ������-����������� �������������.
������������ ����������� ����� �������� ����������, ����������� ��� �������� ���������� �������������
���� ������� \emph{���������� ���������� �������������.}

� ����� 2000-� ��. �������� ���������� ����� ��������� �������� ��������������
� � 2012~�. �������, ������, �������, ������ � �� ����� ��������,
��� ����� �������������� �������� ������� ���������� ������-�����������
��� ������ ������������� ������������� ���������������~\cite{FioriniPokutta:2012}.
���� ������ ������� �������� �������� ����� �������������� � � ��������� ����� � ���� ������� ������� �������� ������~\cite{Fiorini:2012polygons, KaibelPT:12, FioriniKPT:13, Rothvoss:2013, Rothvoss:2014, KaibelW:15}.
}

%\mkcommonsect{development}{������� ��������������� ���� ������������.}{
%����� � ������� ��������������� ����.
%}

\mkcommonsect{objective}{���� ������.}{%
����� ������ �������� ������ ����������� �������������
��� ������� ������ ��������������, ��������������� � �������� ������������� �����������.
��� �������������: 
1) ����� ��������� ������ ��������� ������������� ������������� ��� �������� �������������� �������� ��������������, 
2) ���������� �����������, ���������� ������ ���� ������,
3) ������ ��������������� ������������� ��� ��� ���� ������������� � �������� ������ ��������� ��������������� ��������������� �����.
}

\mkcommonsect{novelty}{������� �������.}{%
��� ���������� � ����������� ���������� �������� ������. 
��������� ���������� �������� ���������:
\begin{enumerate}
\item ������� ������ �������� �������� ����� ������������� ������������� � ������������.
\item ������� ������� �������� ���������� �������� ������������� ��������������.
��������, ��� ������������� ������� �������� ������� �������� � �������������� ��������� �����: ������, �����������, �������� ���������, ���������� �������, 3-������������, ���������� � �������� ������������. ������, � ���������, ������� NP-������� �������� ����������� ������ ��� �������������� ���� ��������.
\item ��������, ��� ������ ������������ �������������, ������������� ������������� ��������� �������������� � ������������� ������������ ���������� ������������ � ������ �������� ����������. � ���� ����� ������� ������ ��������������� ����� ������������� ����������� ��������, ������������� �������� � ��������� ���������, ������������� ������ �� $n$-����������� ��� $n \ge 3$. � ��� ����� ������� ����� �������� ������������� ������� ��������. � ���������, ������ ������������ ������������� ������� �������� �� ���� ������������� � ���� � ���������� ������� ���������� ��������������.
����� ������� ��������, ��� ��������� ������������� �������������� ��������� ������� ������������ �������������� ������������� ����������� ������� ������ ���������� ����������� ��������������, ��������������� � NP-�������� ��������.
\item ��������, ��� ��� ������ ������������ $k$ ������ ������������ ������������� �������� $k$-������� ����� �� ������������������� ������������ ����������� ������������� ������ ������.
\item ��������, ��� ��������� �������������� ����� ������ ������������� ����������� � �������� ������� �������� ������� �������� (� ������ ������) � ��������� ������� ������������ ��������������. 
����� �������, ������������ ������ �������� ���������� ��� ������������� ���� ��������� �������������� �������� ���� ����� ���������������.
\item ������� ���������� ����������� ������������� ��� ����������� �������������� ������������ ����\footnote{����������� ������������� �������� ������������ ������ ������ ����� ���� ��������������, ������� �� �� ����������� � ����� �� ����� ������.}.
\item ��� ���������� ����������� � ���������� ������������� ������������� ��������� �������������� ��������� ������� �����, ��������� ������� ������������� ���������� �� �������� ��������������� �������������.
\end{enumerate}
}

\mkcommonsect{value}{������������� � ������������ ����������.}{%
������ ����� ������������� ��������.
���������� ���������� ����� ���� ������������ ��� ������������
��������� ����� ������������� ����������� � ������ ����� ����������� ���������� �� �������. ������������ ����� ����������� ����� ���� ����� ������������ � ������������� ������������� ������� �������� ��������������.

�������� ���������� ����������� �������������� �� ������������ ��� ��������������,
��� � ����������� ������������� (������ �� �������� ��������� ����������):
�.�.~��������, �.�.~�����������, �.�.~��������, 
%V.~Pilaud, 
%H.~Fawzi, J.~Saunderson, P.A.~Parrilo,
%S.~Massar, M.K.~Patra, H.R.~Tiwary,
%L.B.~Beasley, H.~Klauck, T.~Lee, D.O.~Theis,
%K.~Qi, Q.~Feng, K.~Zhao.
L.B.~Beasley, H.~Fawzi, Q.~Feng, H.~Klauck, T.~Lee, S.~Massar, 
P.A.~Parrilo, M.K.~Patra, V.~Pilaud, 
K.~Qi, J.~Saunderson, D.O.~Theis, H.R.~Tiwary, K.~Zhao.
}

\mkcommonsect{methods}{������ ������������.}{%
��� ������������ �������� ������������� �������������� ���������� ������������ ����� � ������ ������� ������� �������� ����������.
����� ������������ ������ ������ �������� ��������������, ������ ������, ��������� ����������������, �������������� �������.
}

%\mkcommonsect{results}{���������, ��������� �� ������:}{%
%����� � ���������� � �����������.
%}

\mkcommonsect{approbation}{��������� �����������.}{%
���������� ����������� ������������� � �����������
�� ���������� ����������� <<���������� ����������� � ������������ �������>> (�����, 2010), 
�� XVI ������������� ����������� <<�������� ������������� �����������>> (������ ��������, 2011),
�� ������������� ����������� <<�������������� ���������������� � ����������>> (������������, 2007, 2011),
�� ������������� ����������� <<���������� ���������>> ����������� 100-����� �.�.~������������ (���������, 2012),
�� ������������� �������������� ����������� <<��������������� ������>> (������, 2012),
�� XXI ������������� ���������� �� ��������������� ���������������� (������, 2012),
�� ������������� ���������� �� ������������� ����������� (�������, 2012),
�� ������������� ����������� <<���������� ����������� � ������������ ��������>> (�����������, 2013), 
�� 26-� ����������� ������������ ��������� �� ������������� ����������� (2013, �����),
�� �������� ��������� �������������� ����������� ������������ ���� ��� ������ (���������, 2013),
�� �������� �� ���������� ���������� � ��������� ������������ ������� (2013),
�� 5-�� �������� �� ������������� ����������� (������, �������, 2014),
�� 9-� ������������� ����������� �� ������ ������ � ������������� (��������, �������, 2014),
�� XIII ������������� ����������� <<�������, ������ ����� � ���������� ���������: 
����������� �������� � ����������>>, ����������� 85-����� �.�.~������� (����, 2015),
�� 5-� ������������� ����������� �� �������� ������� (������ ��������, 2015),
�� �������� ����������� <<���������� � �������������� ���������>> ���� ��.~�.�.~��������,
�� ������������� �������� �� ���������� ����������,
�� �������� ������� ���������������� ��������� � ���������� ��� ��.~�.�.~����������.
}

\mkcommonsect{struct}{��������� � ����� �����������.}{%
����������� ������� �� ��������, ������ ����������, $n$ ����, ���������� � ������������.
����� ����� ����������� $P$ �������, �� ��� $p_1$ �������� ������, ������� $f$ ��������.
������������ �������� $B$ ������������ �� $p_2$ ���������.
}

\mkcommonsect{pub}{����������.}{%
��������� ����������� ������������ � $N$ �������� �������, �� ��� 
���� ����� � ����������~\cite{BondBook:2008},
���� ����� � ������� �������~\cite{BelovBM:2006},
13 ������~--- � ��������, ��������������� ���~\citemy{Maksimenko:2004,
Maksimenko:2009,Maksimenko:2011,Maksimenko:2012DAN,Maksimenko:2012Cook,Maksimenko:2013k,Maksimenko:2013NP,Maksimenko:2013TSP,Maksimenko:2014MAIS,Maksimenko:2015DAN,BogomolovFMP:2015, Maksimenko:2016bool, Maksimenko:2016complexity}, 
{\color{red}2 ������ � ��������� ������ ����������� 
� 1 ������� ��������}.
���������� ����� ���������� ����������� ��������� � ���������� �.�.~�����������, 
�.~���������� � �.~�������, ������ ����� ����������� ��� ������������. 
����������� ���������� ������������, ������� ������� � ��� ������������ �����������~\cite{MaksimenkoDiss:2004}.
}

\mkcommonsect{finance}{���������� ���������.}{%
������������, ���������� � �����������, ���� ���������� �������� ����~00-01-00662-a, 03-01-00822-a; 
��� �������� � ������-�������������� ����� ������������� ������ �� 2009--2013 ���� 
(��������������� �������� �~02.740.11.0207),
������������ <<���������� � �������������� ���������>> ���� ��.~�.\,�.~��������
(����� ������������� �� �~11.G34.31.0057),
� ����� ��������� �~477 � �~984 � ������ ������� ����� ���. ������� �� ��� ���� ��.~�.\,�.~�������� (2014--2016~��.).
}

%\mkcommonsect{contrib}{������ ����� ������.}{%
%���������� ����������� � �������� ���������, ��������� �� ������, �������� ������������ ����� ������ � �������������� ������.
%���������� � ���������� ���������� ����������� ����������� ��������� � ����������, ������ ����� ����������� ��� ������������. 
%��� �������������� � ����������� ���������� �������� ����� �������.
%}


% номер копии для грифа секретности
%\copynum{1}
% класс доступа
%\classlabel{Для служебного пользования}

% номер УДК
%\libcatnum{12345}

\title{ДИССЕРТАЦИЯ\\
на соискание ученой степени\\
доктора физико-математических наук}

\def\specskip{0pt} % Верт. отступ между специальностями
\maketitle

%%
%% Titlepage in English
%%
%
%\institution{Name of Organization}
%
%\title{Doctoral Dissertation}
%
%% Topic
%\topic{Dummy Title}
%
%% Author
%\author{Author's Name}
%
%\specnum{01.04.05}
%\spec{Optics}
%
%%\specsndnum{01.04.07}
%%\specsnd{Condensed matter physics}
%
%% Scientific consultants
%\scon{B.\,B.~Baranov}
%\sconstatus{Professor}
%%\sconsnd{P.\,P.~Petrov}
%%\sconsndstatus{Professor}
%
%% City & Year
%\city{Saint Petersburg}
%\date{\number\year}
%
%\maketitle[en]

% Содержание
\tableofcontents

% Введение
\intro
% !TEX root = MaksimenkoThesis.tex
% ����� ������� ������������ �~�����������

\textbf{������ ������������.}
������ ���������� ������ ������������� ����������� ��������� ��������� ������������.
���� �������� ��������� $E$, ������� �������� $e$ �������� �������� ��������� ��� $c_e \in \R$, �~������������� ���������� \emph{�������� ������������} $f \colon 2^E \to \{\text{����}, \text{������}\}$.
������������ $s \subseteq E$ ���������� \emph{���������� ��������} ���� ������, ���� $f(s)$ �������.
��������� ���� ���������� ������� ��������� $S$, $S = \{s \subseteq E \mid f(s)\}$.
%, ������������ ��������� $S = \{s \subseteq E \mid f(s) = 1\}$ ���� ���������� ������� ������.
���� ������ ������� �~��������� \emph{������������} ������� $s \in S$ �~������������ (�����������) ����� $c(s) = \sum_{e \in s} c_e$.

� ������������� ����� ������, ������� ������� ������������ \emph{�������� ������}, ����� ��������� $E$, �������� $f$ � ���� $c_e$, $e\in E$, �� �������������, �� $E$ � $f$ ���������� ������������ ������� ������� ���������� $I$ �� ���������� �������, ���������������� ������ �������� ������. \emph{�������������� ������} ������������ ����� ������� ������ �������� ������, ����� ������� ������ $I$ � $c_e$, $e \in E$, �������������.
������ � �������� � �������������� �������� ���������� ������������� �������, ����� ��������� $I$ �������������, � ���� $c_e$ "--- ���.
��� ��� ��������� ���������� ������� $S$ � ����� ������ ������������ ����������, ��� ����� ������ ����� ������������ �����������~$S$.

��������, �~������ �~���������� ���� ������� ������ $I$ ���������� ���������� ��������� �������, ����� ������� �������� ������ $A$ � $B$, � ��������� (��������) ����� $E$, ����������� ���� �������, ���� $c_e$ �������� ������� �����, �~�������� ������������ $f$ ��������� �������� <<������>> ��� ������� ������������ �����, ��������������� ����� ������� ��~$A$ �~$B$.
������� ������������� ��������� �������� ������ ������ ������������ ��������� ������, ������ �~�����������, ������ ������������, ������ �~������� �~������ ������.

%������� ����������, ��� ��������� ��������� ���� ������, ����� ������.
���� ����� ����� ������ ����������� �~��������� ��� ����������� ������������� ��������,
������������ ������������ �������������� �~������������, ����������� �������� ������, ����������� ���������� �~�.\,�.~\cite{Paschos:2014}. 
����� ����, �~��������� ����� ������ ������������� ����������� ����������� ����������� �����: 
�������������� ������, ������������� ������������� �����, ������������� �������, 
������������ ���������������� �~������������� �����, ���������������� ���������,
������������ ������������ ������������ ���� (VLSI), %�~�������� ����, 
������������, �������� �������� �~�.\,�.~\cite{GrotschelCO:1995}.

��� ��������~\cite{SchrijverCO:2003}, �� ������ ������� ����� ������ ������� ���������� �� ���� ���������.
� ������, ��� ������������� $I$, ��� ������� ����������� ������� $s \in S$ ��������������� ��� \emph{������������������ ������}
$\bm{x} \in \{0,1\}^E$, ���������� $x_e$, $e \in E$, �������� ���������� ������� ������� ��� $e \in s$
� ����� ���� ��� $e \not\in s$. 
����� ��������� ���� ������������������ �������� ���������� ������� ���������� $X$, $X = X(S) \subseteq \{0,1\}^E$.
����� ����� �������������� �~���� ������� $\bm{c} = (c_e) \in \R^E$.
%, ����������� \emph{�������}.
���� ������ ��� ����� ������������� ����������� �~������ ������� $\bm{x} \in X$, �� ������� \emph{������� �������} $\langle\bm{c}, \bm{x}\rangle$ ��������� ������������ (�����������) ��������.
��� ��� ������� ������� �������, 
�� ��������������� ������ ���������� \emph{�������� ������� ������������� �����������}.

�������, ��� ������������� �������� �������� ������� ��~�������� ��� ������ ������� ����������� $X$ �������� ��������� $\conv(X)$.
����� �������, �~������ ������� $S$ ������������� ��������� �������� ������������ $\conv(X)$, ��������� �������� ������ ���������~$X = X(S)$.
%�������������, ������������ ����� ��������, ������� �������� ��������������.

��-������, ����� �������������� ������������� ���� ����������� ��� ������� ������ ������������ ���������� ��������������� �������������, �~���������, �������� ��������� ���������������� (��., ��������, \cite{Shevchenko:2004}).
��� ���������� ���� �������������� ������������ �~���� �������, �� ������ ������� ��� ��������� �� ������� ����������� �������� �~�������� �������~\cite{SchrijverCO:2003}.

��-������, ������������� ��������� ������������� �������� 
��������� ��������� ���������� ������� ��������������� ������.
%��� ��������� ����������� ��� ������������ ������� ��������� ������.
��������������� ��� ����� �~������� ������� ������� �������.
���������� ������� $s_1$ �~$s_2$ ������ $S$ ���������� \emph{��������},
���� ��� ���������� ������ ����� $\bm{c}$ ��� ��� ������� �������� ������������
��� ��������������� �������������� ������ �~������ ����������� ������� � �� ���.
��������� ������� $s_1$ �~$s_2$ ��������, ��� ��� �������������� ��������� ������ ����� $\bm{c}$ ����������� ������� �������������� ������ ����� �������� �~$s_1$ �� $s_2$ �~�������.
�� ���� �~���������, �������� ������ $S$, ������ ���� ������������� ��������,
�������������� �~����� ����������.
���, �~���� �������, ����������� ��� ����������� �� ��������� ��������� %, ��������� ������ ����������� �� ��������� $S$,
�, �~�����, ����� �������������� ��� ������������� ������ ��������� ��������������� ������.
��� �������������� ������� ��������� ������� ���������������� ��� ������� �����
����� ���������������� ��������� �������������~$\conv(X)$.
����� �������, ���� ������������� ������ �������� �~���� ��������� ���������� �~� ����������� ���������. 
��������� ��� �����, ������ ����������� ���������� �~�������� ������� ���� ������� ������ (��������� ���� ������, ������������� �� ���������) ������������� ������.
%��� �~�������� �������� �������� ������������ ��������� ������.


\textbf{������������ ���� ������������.}
������� ������� �~������� ������������� ����������� �������� �~1950-� ��.
��� ������������� ���� ����������� ����� ���������:
�������� ������ ��������������� ���, �������� ��������� ��������� ���������������� ������������ �~��������� �~���������� ��������-������ ��������,
� ����� ����������� ���� ������� ��������������� ����� �� ������.
������ �~1950-� ��. ��� ���������� ���������� ����� ������� ������ �~�����������,
�������� �����-���������� ��� ������ �~������������ ������, 
��������� ���������� ��� ���������� ���������� �����,
������ ������� ������ �~���������� ����� ��������� ��� ���������� ������������ ��������� ������,
� ����� �� ��������� ��� ���������� �~�������� ���� ������������ ����~\cite{SchrijverHistory:2005}.
�~��� �� �����, ����� ���������� �������� ��������-������, ���� ����������� ������ �������� �������
��� ���������� �~��������� ������� ������������� �����������.
�~���������, �~������� ������� ��������� ���������������� ��� ��������� ������������ �� ��� �������� �������� �~������� ������ ������������~\cite{DantzigFJ:1954}.

�������� ���������� ��������-������ ����� ����������
�������������� ������������ �~��� ������������� �������������.
���, ��������, ���� ��������, ��� ������ ������� ����� ����� ��������-������
����� ������� ������� ����� �������������.
�~����� �~���� ���� �~1957 ���� ������������� �������� �~���,
��� ������� ����� ������������� �� ����� ���� ������ �������� ����� ������ ��� ����������� �~������������.
�~��� ��� ���� �������� ��������� ������������ ��������, �� ���� �~2010 ���� �������
������� ��������� ������ 43-������� ������������� �~86 ������������, ������� �������� ������, ��� 43~\cite{Santos:2012}.
��� �� �����, �~����� ����\footnote{����� ��, ��� ������� ����� ��������� ������ ��������� �� ����� ����������� �~����������� �������������?} ��� �������� �� ��� ��� �������� �������� �~���������� �������� ������ ������~\cite{ZieglerHirsch:2012}.

����� ��~�������� ����������� ��������� ������ �������� ������ �������� �������� �����
�������������, ������������� �~������������. 
�~1964~�. ���~\cite{Klee:1964} �����������, ��� ���� ������� ����� $\lfloor n/2\rfloor$,
��� $n$~--- ����� �����������. �� ����� ������ ������� �� �� ��� ������ �����������~\cite{Klee:1967}.
�~��� ��� ������ ���������� ��������.

������ �~�������� �~��������� �~������� ��������� ����� �~1950--60-�~��.
������������� ������� ������������ ���������, ������������ ���������������� �~������� 
��������~\cite{Edmonds:1965} �~�������~\cite{Cobham:1964}.
�~��� �� ����� �������~\cite{Edmonds:1965b} ���� ������� ������, ������� <<������� ��������������>>,
���, ��-����, �������� ������������ ����, ��� ������� ���� ������� ������� NP.
�� ��� ��������� ������������� �~�������� �~������ 1970-� ��. �����~\cite{Cook:1971}, ������~\cite{Karp:1972} �~�������~\cite{Levin:1973} NP"~������ �����. %~\cite{Garey:1982}.
������������� ��, ��� ������ ��~��� ��������� ���� ������ ����\'���� �����.
�~���������, ����� �������� ����������, ����������� � ��������� �����������, �� ����� ���� ����� ����� �~���������� ������.

�������� NP"~������ ����� ��������� ������ ������� ��� ���������� ������������,
� ��� ����� ������� ��������������, ��������������� �~NP"~�������� ��������.
�~1978 ���� ������������~\cite{Papadimitriou:1978} �������, 
��� ������ �������� ����������� ���� ����������� ������ ������ ������������� ������ ������������ NP"~�����,
�� ���� ��� ����� ������, ��� �~���� ������ ������������.
������� ����������� ���������� ��� �������������� ��������� ������ NP"~������� ����� ��������� � ������� ��������� �������: ����; ����� �~�����; �����; ���������� �~����; ������� �~�����; ������� (��. ������ �~\citemy{Maksimenko:2013NP}).
�~������ �������, �~1975~�. ������~\cite{Chvatal:1975} ����� �������������� �������� ��������� ������ ������������� ����������� ��������.
��-����, ���� �� �������� ��������� ����� � ��� �������������� �������� ��������, �������������� ��������� �������� � �������������� ������������� ������ � �����������~\cite{Ikura:1985}.
�~1984~�. ������� ���������~\cite{Greshnev:1984}, ��� ���� ������������� ������ �� $m$"~��������� ����� �����, �.\,�. ������ �������� ��������� ������ ��� ���� ����������.
���� ������� ����������� ���������� ���� �������� ��� ������������� ������ �~������������ �������~\cite{Beloshevskii:1986,Barahona:1986} � ��� 
������ ������������� �������������~\cite{Bondarenko:1987, Padberg:1989}.
%������������� ������ �~������������ ����� � �������"=���������� ������ �����~\cite{Bondarenko:1985}. ��������� ������������ �� ������ �������� ����� ������ � �������� ������������� ������������� $\BQP(n)$, ���� ����������� ������ ������~\cite{Bondarenko:1987, Padberg:1989}.

� ����������� ��������, ��� ��������� ��������������, ��������� �����~\cite{Matsui:1995}, ������� �������� �~���������� ��������������, ��������� � ������� ����; ����� �~�����; ���������� �~����; ������� �~�����; �������. ����� ��~��������� ����� �������� ��, ��� ��� ��� ��������� ��������� �� �������������� ����� NP"~������� �������� ����������� ������.
� ������ �������, ��������, ��� ������������� ���� ������������� � �������������� ��������� ��������� ������� �������� � �������������� �����, ������ �������� ���������� � �������� ������� ����������. ����� �������, ���������, ��� ������������� ����� �������� ������ ���� ���������� ����� ����������� �������������� � NP"~������ ��������� ����������� ������ � ����������� � �������������� ���������.


%��� ��������� ������������, ��� ��� ����� ����� ���� �~�� �� �������������� �������,
%�������������� �~������� �������� ����������~\citemy{Maksimenko:2013NP}.
%�����������, ��� ��� ����� ����� ���� �~�� �� �������������� �������,
%�������������� �~������� �������� ����������~\citemy{Maksimenko:2013NP}.
%����� ��������� ������ �������, ��� ��� ����������� �~���� ������� �������������
%�������� �~�������� ����� ������������ ������� ��������~\citemy{Maksimenko:2013NP},
%������ �������� ��������� ������ �������� NP"~�����~\cite{Matsui:1995}.

�~1979 ���� ������~\cite{Khachiyan:1979} ������ �������������� �������� ��� ������� ������ ��������� ����������������.
���� ���� ���� ������������� �������������� ������������� ��������������� ������� �~������� ����� ������������� �����������, ��� ��������� ������������ ������������ ������� ��������������� ��������������.
�~���������, ������� �������� ������������ ������� ������ �������������� ������� �~���������� ���������, �������� �~��������~\cite{Emelichev:1981}.

�~1980-� ��. ����������~\cite{BondBook:1995} ���������, ��� �������� �����\footnote{�~��������� ��� �������������� ���������� ���������� �����.} ����� ������������� ������ ������������� ��������������� ��������� ��������������� ��������������� ������. 
� ������, �� ���� �������� ��������������������� �������� ����� ������ �������������� ��������� NP"~������� �����: �����������, ������������ �����, 3-��������� �~��������� ������.
�~������ �������, �������� ����� ������ �������������� ��������� ������������� ��� ��������� ������������� ���������� �����: ����������, ����������� �������� ������, ������ �~���������� ����.
�� ��������� ���� ������ ���� ����������� ������ ���������� ������� ����, ������������, ��� �������� ����� �������� ������ ������� ��������� �~��������� <<������� ������ ����������>>~\cite{BondBook:1995}. 
������� ������ ������ �������� ����� ������ �������������� ����� ���������� ����������� ������ ������������~%\cite{Shovgenov:2015, Nikolaev:2016, Nikolaev:2017, Shovgenov:2017}.
\cite{Nikolaev:2017}.

��������� ��������������� ������ �� ������ ���������� ������� ������������, ������ ������� ���� �������� � ������� �.�.~����������. ��"~������, � ����������� ��������, ��� ������ ���������� ������� ���� ��������� � � ��� �������, ����� �� ��������� ������� �������� ������������� �������� �����������.
� ���������, �������� ����� �������������� ����� ��� ������ � ���������� ���� �� ����������������� ������������ � �������������� ��� �����������, ����� ����������� ���� ��������������� ����� �����.
%(��������, ����������������� ���� ����� � ������������ ������ � ���������� ����), ��� ���� ������������ ������ ������������ � �������. 
������������ ���� ������� ������������ ��� ��������� ����� ����������� �~\cite{Nikolaev:2016} �~\cite{Shovgenov:2017}. ��"~������, ��������, ��� �������� ����--�������� (���������� �����) ��� ������ �~����������� �� ����������� ������ ���������� ������� ���� �, ����� ����, ������ ���������� ������������� ������ ����������� ����������, ����������� �� �������� �� ������������, �� �������������� ��������� �� ��~����� ������. �"~�������, �����������, ��� ��������������������� �������� ����� ������ �������������� NP"~������� ����� ����� ������� ���������� "--- ��~���� ���� ���������� ������� �������� ������ ������������ ������������� $\{\BQP(n)\}$, �������� ����� ������ ������� ��������������� �� $n$.
%������������������ ������������ �����������. 
����� ����, ��������, ��� $\{\BQP(n)\}$, � ������ � ���� � ��������� ��������������� � ��������� ������ ��������� �������������� NP"~������� �����, ��� ������ ������������ $k$ �������� $k$"~�����������\footnote{������������ ���������� \emph{$k$"~�����������}, ���� ����� $k$ ��� ������ �������� ����� ����� �������������. � ���������, ������������, ���� �������� �����, ���������� 2-�����������.} ����� �� ������������������� (������������ ����������� �������������) ������ ������.


%��������� ������� ������������ �������������� $\{\BQP(n)\}$ ������� �������� �� ���� ���������� ���� �������������� NP-������� �����. ��������, ��� �������� ����� ����� ������������� $\BQP(n)$ ���������������, ��. ����� �������, ������� ������� �������������� ���������� ��� 
%������� �������� ����������, ��������� ������� ��������� ��������������� ������, ��������� ��������������������� �������� ����� ������ ���, ��� ��� ������������� ������������� NP-������� ����� �������� �~�������� ����� ����� ������������ ������������, �������� ����� ����� �������� ���������������.

�������� �������� ��������������� ��������� ��� ������ ��������� ���������������� �������� � 1980-� ��. ����� ������� ������ ����������� ��������� �������� ��� ������������� ������ ������������.
��� ��� ������� ���� ���������� �� ������������� ���� ����������� ������������ ������������� 
(��� ������ �������� �������). 
\emph{����������� �������������} �������������~$P$ ���������� ����� �������� �����������,
����������� ������������ $Q$ �����, ��� $P$ �������� ������������� ��������� $Q$.
��� ������������ $Q$ ���������� \emph{�����������} �������������~$P$.
� ���� ������� ��� ���� �������� �������, ����� ����� �������� ����������, ����������� ��� �������� �������������, ���������������, �~��� ��� ����������~--- ������������� ������������ ����� ������� ������ ������.
�� ���� ��~������� ������ ���������� ����������� ������������ ��� ������������� ������ ������������ �� ������� �~������ �~� 1988 ���� ���������~\cite{Yannakakis:1988}
�������, ��� ����� ������� �~�������� �� ����� �����������, 
���� ������������ ����������� ������������ ������������� ��������� ������������ �������� ���������.
��� �� ���� ��������� ��������, ��� ����������� �������� ������������ �~��� ������� ���������.
����� ����, ��������� �������, ��� ����� �������� ���������� �~����������� ������������ ������������� �� ����� ���� ������ ����� �������������� �������� ������� ���������� ������"=����������� �������������.
������������ ����������� ����� �������� ����������, ����������� ��� �������� ���������� �������������
���� ������� \emph{���������� ���������� �������������.}

�~����� 2000-� ��. ����������� ������������ ����� ��������� �������� ��������������~\cite{Conforti:2010,Kaibel:2011}, 
��� ������� �~��������� ������ ���� ����� ���������� ����������� �~������ �����������~\cite{FioriniPokutta:2012, Fiorini:2012polygons, KaibelPT:12, FioriniKPT:13, Rothvoss:2013, Rothvoss:2014, KaibelW:15}.
�~���������, �~2012~�. �������, ������, �������, ������ �~�� ����~\cite{FioriniPokutta:2012} �������� �������������� �������� ����������, �������,
��� ����� �������������� �������� ������� ���������� ������"=�����������
��� ������ ������������� �������������~$\BQP(n)$ ��������������� ������������ $n$ �, ��� ���������, ��������� ���������� ���� ���������������.
�� ����� ����������, � ����� �� �������������� � ��������� ����������� ����� �������� ���������� ������� ������������ �������������� �� ������ ������ ���������� �������������� NP"~������� ����� �������, ��� ��� ��� ��������� ���� �������� ������������������� ������ �������������� �������� � ������������������� ���������� ����������. 
� ������ �������, � ����������� ������� ������ ������ NP"~������� ������, ������������� ������� �������� �������������� ������ �������������� ��������.

� �������� ������������ ���������� ����������� ������������ � ����������� ��������������� ������ ����������� �������� ������� �� ��������� ������ ������������ ������������� \(\CP_{d,n} = \conv\Set*{(t, t^2, \dots, t^d)\in \R^d \given t = 1,2,\dots,n}\).
��� ������ ����� ������� � ������� ����������� ������ ���������� %�������~$d$
�~��������� ������� �������� ���������. � ������ ��������, ��� ��������� ���������� ����� ������������� ����� $O(\log n)^{\lfloor d/2 \rfloor}$ (����� ��� ����� ��� ����������� ����� ������� $\Theta(n)^{\lfloor d/2 \rfloor}$ ��� ������������� $d$~\cite{Gale:1963}).

� ��������� ����������� ����� ������� ������� ����������� �������� ����������, ����������� ����� �������� ��������������, ��� ��������������������� ��������� ���������� � ����� �������������� �������� ������� ���������� ������"=�����������. ��������, ��� ��������� �������������� ����� �������� ������ ������������� ����������� � ���������� ������������ �� ������ NP ���������� ������� �������� � ��������� ������� ������������ ��������������.
%�~\cite{Beasley:2013} ��������, ��� �� ����� ����� � �� ���������� S.~Burer~\cite{Burer:2009} �������, ��� ����� 0/1"~������������, ��� ��������� ������ ����� ���� ������� �������������� ����������, ����� ��������������� ����������\footnote{��������������� ���������� ������������� $P$ "--- ��� ����������� ��������� ��������������� � ���������������� ������ $C_n = \{X \mid y^T X y \ge 0, \ \forall y \in \R_{+}^n\}$ �����, ��� $P$ �������� ��� ���������. ������������ � ���������������� ������ �������� ��������� ������������� ����� $C^*_n = \{X \mid X = \sum_{y \in Y} y y^T \text{ ��� ���������� ��������� } Y \subset \R_{+}^n\}$, � ������� ������� ������� ��������� �������������� ����������.} ��������������� �������. 
�������, ������, ����� � ������~\cite{Fiorini:2014} ���������� ���� ���� ��� ����, ����� ���� ����� �� ��������� ������ �� ������ �����~\cite{Burer:2009}: <<����� ���������� ������������� �����, ����� ���� ����� ����� ���� ������������ ��� ��������������� ��������� ��� ��������� ������������� ���������?>> (\foreignlanguage{english}{``Other than the handful of problems listed above, what types of problems can be represented as COPs or as CPPs?''}).

%\textbf{development}{������� ��������������� ���� ������������.}{
%����� �~������� ��������������� ����.
%}

\textbf{����� ������}
�������� ������ ������������"=�������������� ������� ��������������, ��������������� ��������� ��������������� ����� ������������� ����������� � ��������� �������������� ������� � ������� ����������.
��� �������������: 
1)~����� ��������� ������ ��������� ������������"=�������������� ������������� ��� �������� �������������� �������� ��������������, 
2)~���������� �����������, ���������� ������ ���� ������,
3)~������ ��������������� ������������� ��� ��� ���� ������������� �~�������� ������ ��������� ��������������� ��������������� �����.


\textbf{������ ������������.}
��� ������������ �������� ������������� �������������� ���������� ������������ ����� �~������ ������� ������� �������� ����������.
����� ������������ ������ ������ �������� ��������������, ��������� ����������������, ������ ������ � �������������� �������, ������ ��������� ����������.


\textbf{������� �������.}
�������� ���������� ����������� �������� ������ �~����� ���� ������ �������������� ��������� �������:
%���������� �~����������� ���������� �������� ������. ��������� ���������� �������� ���������:
\begin{enumerate}
\item ������� ������� �������� ���������� �������� ������������� ��������������, ����������� ����� �������� ��������������, ��� NP"~������� �������� ����������� ������ �~��������������������� ��������� �������� �������������: ����� ������, ����� �����������, �������� ����� �����, ��������� ����������, ����� �������������� �������� ������� ���������� ������"=�����������. � ������� ����� ������� �������� ��������� ����������:
\begin{itemize}
	\item ��������, ��� ����������� ��������� ��������������, ������������� ����� �~1995 �., ������� �������� �~���������� �������������� ��������� �����: ������, �����������, ���������� �������, 3-������������, ���������� �~�������� ������������. %, �������� ��������� 
	����� ��~��������� ����� �������� ��, ��� ��� ��� ��������� ��������� �� �������������� ����� NP"~������� �������� ����������� ������.
	\item ��������, ��� ��������� �������������� ����������� ��������, �������������� �������� �~��������� ���������, �������������� ������ �� $n$"~����������� ��� $n \ge 3$ �~�������������� ��������� ����� ������������ �~������ �������� ���������� �~������� �������� �~��������� �������������� �����. ����� ����, �����������, ��� �� ���� ��~�������������� ����� (�� ����������� ��������) �� �������� ������ (� ��� ����� �������������) �� ��� ������ ��~�������������� ����������� ��������.
\end{itemize}
\item ���������� ������ �������� ������� ������������ ��������������, ����������� �� �� ������ ����� ����� ������ ��������� �������� �������������� NP-������� �����:
\begin{itemize}
	\item ��������, ��� ������ ������������ ������������� ������� �������� �~���������� ��������������, ������������� �~���������� ������ (������, ����������� �~�.\,�.), �~����� �~�������������� �������� ��������������, �������������� ������������ ���������� �~������������ �������� ��������������. 
	�� ����� �������, �~���������, ��� ����� �������� ������� ������������ ��������������, ��� ��������������������� ��������� ����� ����� �~����� �������������� �������� ������� ���������� ������"=����������� ������������� ����������� ����� ����� �����������.
	\item ������� �~������������ ��������� ������� �������������� �������~$p$.
	��������, ��� ��� ������������� $\lfloor 3p/2\rfloor$"~���������� �~������� �������� �~������� ������������ ��������������. �� ����� �������, ��� ��� ������ ������������ $k$ ������ ������������ ������������� (� ������ � ���� � ��� ��������� ������������� ���� ��������� �������������� NP"~������� �����) �������� $k$"~����������� ����� �� ������������������� (������������ ����������� �������������) ������ ������.
\end{itemize}
\item ������� ������� ����������� �������� ����������, ������������ �� �������� ���������� ����������� ���������� ������������ ���������������� ��������� �����������. ��������, ��� ��������� �������������� ����� �������� ������ ������������� ����������� �~���������� ������������ �� ������ NP ���������� ������� �������� �~��������� ������� ������������ ��������������. 
����� �������, ������������ ����������� �������� ���������� ��� ������������� ���� ��������� �������������� �������� ���� ����� ���������������.
������ ������ ��������� �������������� NP-������� ������, � �������� �� ����� ���� ���������� ������� ������� �� ���� ��~���������� ���� �������� ��������������.
\item ������� ������ ���� ������������� %��������� 
��� ����������� ��������������\footnote{����������� ������������� �������� ������������ ������ ������ ����� ���� �������� ��������������, ������� �� �� ����������� �~�����~�� ����� ������.}:
\begin{itemize}
	\item ������� ���������� ����������� ������������ ������� $O(\log n)^{\lfloor d/2 \rfloor}$ ��� $d$"~������ ����������� �������������� �� $n$ ��������.
	\item ������� ������ �������� �������� ����� �������������, ������������� �~������������.
\end{itemize}
%\item ������� �������� ����� �������� ������ �~���������� ������ �~������������ ����������������� ���� ��������.
\item �������� ����������� ������ ��������������� ������������� ��������� ��������� ������������� �������������� � �������� ������ ��������� ��������������� ��������������� �����:
\begin{itemize}
	\item �� ������� ������ � ���������� ���� ��������, ��� ������ ���������� ������� ���� (������������� �.\,�.~����������) ��������� � � ��� �������, ����� �� ��������� ������� �������� ������������� �������� ����������� (��� ���� ������������ ������ ������������ � �������).
	\item ��������, ��� �������� ����--�������� (���������� �����) ��� ������ �~����������� �� ����������� ������ ���������� ������� ����.
	����� ����, ����������� ���������� ������������� ������ ����������� ����������,
	����������� �� �������� �� ������������, �� �������������� ��������� �� ��~����� ������.
	\item ��������, ��� � ������ ��������� ������������� ���� ���������� (������������), ���� �������� �� �������� �������������. 
	\item ���������� ������ ��������� �������������� NP"~������� ������, ����� �������������� �������� ������� ���������� ������"=����������� ������� �������������. 
	\item ���������� ������� ���� �������� ����� ������������� �����������, ������������� ������� ������������ ������������ ���� �����, �� ���� ��~���� ����� ������������� ���������, �~������ NP"~������.
\end{itemize}
\end{enumerate}


\textbf{������������� �~������������ ����������.}
������ ����� ������������� ��������.
���������� ���������� ����� ���� ������������ ��� ������������
��������� ����� ������������� ����������� �~������ ����� ����������� ���������� �� �������. ������������ ����� ����������� ����� ���� ����� ������������ �~������������� ������������"=�������������� ������� �������� ��������������.

�������� ���������� ����������� �������������� �� ������������ ��� ��������������,
��� �~����������� ������������� (������ �� �������� ��������� ����������):
�.�.~��������, �.�.~�����������, �.�.~��������, 
%V.~Pilaud, 
%H.~Fawzi, J.~Saunderson, P.A.~Parrilo,
%S.~Massar, M.K.~Patra, H.R.~Tiwary,
%L.B.~Beasley, H.~Klauck, T.~Lee, D.O.~Theis,
%K.~Qi, Q.~Feng, K.~Zhao,
%A.~Huq.
%A.~Makkeh, M.~Pourmoradnasseri, D.O.~Theis. The Graph of the Pedigree Polytope is Asymptotically Almost Complete (Extended Abstract)
%Huchette, J., Vielma, J. P. (2017) \url{https://arxiv.org/abs/1709.10132}
%Davis-Stober, C. P., Doignon, J. P., Fiorini, S., Glineur, F., Regenwetter, M. (2017) \url{https://arxiv.org/abs/1710.02679}
\foreignlanguage{english}{
L.B.~Beas\-ley, C.P.~Davis-Stober, J.P.~Doignon, H.~Fawzi, Q.~Feng, F.~Glineur, J.~Huchette, A.~Huq, H.~Klauck, T.~Lee, A.~Makkeh, S.~Massar, P.A.~Parrilo, M.K.~Patra, V.~Pilaud, M.~Pourmoradnasseri, K.~Qi, M.~Regenwetter, J.~Saunderson, D.O.~Theis, H.R.~Tiwary, J.P.~Vi\-el\-ma, K.~Zhao.}


%\textbf{results}{���������, ��������� �� ������:}{%
%����� �~���������� �~�����������.
%}

\textbf{��������� �����������.}
���������� ����������� ������������� �~����������� �� ��������� ������������, ��������� � �����������:
���������� ����������� <<���������� ����������� �~������������ ��������>> (�����, 2010), 
XVI ������������� ����������� <<�������� ������������� �����������>> (������ ��������, 2011),
������������� ����������� <<�������������� ���������������� �~����������>> (������������, 2011),
������������� ����������� <<���������� ���������>> ����������� 100-����� �.�.~������������ (���������, 2012),
������������� �������������� ����������� <<��������������� ������>> (������, 2012),
XXI ������������� ��������� �� ��������������� ���������������� (������, 2012),
������������� ��������� �� ������������� ����������� (�������, 2012),
������������� ����������� <<���������� ����������� �~������������ ��������>> (�����������, 2013), 
26-� ����������� ������������ ��������� �� ������������� ����������� (2013, �����),
������� ��������� �������������� ����������� ������������ ���� ��� ������ (���������, 2013),
������� �� ���������� ���������� �~��������� ������������ ������� (2013),
5-� ������� �� ������������� ����������� (������, �������, 2014),
9-� ������������� ����������� �� ������ ������ �~������������� (��������, �������, 2014),
XIII ������������� ����������� <<�������, ������ ����� �~���������� ���������: 
����������� �������� �~����������>>, ����������� 85-����� �.�.~������� (����, 2015),
5-� ������������� ����������� �� �������� ������� (������ ��������, 2015),
������� ����������� <<���������� �~�������������� ���������>> ���� ��.~�.�.~��������,
������������� ������� �� ���������� ����������,
������� �� ������ ��������� ���������� �� ��� �� ���,
������� ������� ���������������� ��������� �~���������� ��� ��.~�.�.~����������,
������� <<���������� � �������������� ���������>> ���� ���.


\textbf{��������� �~����� �����������.}
����������� ������� ��~��������, ������ ����, ���������� �~������ ���������� ��~195 ������������. �~����� ������ ����������������� ������ �~������ ���������� ���������� �������� �����������, ��� ��� ������ ������������.
����� ����� ����������� "--- 256 �������, ������� 22 �������. % 2+4+0+10+0+4+1+1
%������ ���������� �������� 187 ������������ �� 17 ���������.


\textbf{����������.}
����������� ���������� ������������, ������� ������� �~��� ������������ �����������~\citemy{MaksimenkoDiss:2004}.

��������� ����������� ������������ �~16 �������� �������, ��~��� 
12~������~\citemy{Maksimenko:2004,
	Maksimenko:2009,Maksimenko:2012DAN,Maksimenko:2012Cook,Maksimenko:2013k,Maksimenko:2013NP,Maksimenko:2013TSP,Maksimenko:2015DAN,BogomolovFMP:2015,Maksimenko:2016complexity,Maksimenko:2017,Maksimenko:2017LOP} "--- �~��������, ������������� � Scopus,
3~������~\citemy{Maksimenko:2011,Maksimenko:2014MAIS,Maksimenko:2016bool} "--- �~��������, �������� � RSCI, �~���� ����� �~����������~\citemy{BondBook:2008}.
%, ���� ����� �~������� �������~\citemy{BelovBM:2006}.
%{\color{red}2 ������ �~��������� ������ ����������� �~1 ������� ��������}.

���� ���������� ������������ �~����������� �~�.�.~�����������, �.~���������� �~�.~�������~\citemy{BogomolovFMP:2015}. ���������� ������ ��� �������� ����������� ���� ����������, ���������� � ����� �����������, ����������� ��������� ������� (������ ��������� ���� ������ �������� �� ����������). ������������ ������ � ����� � �������������� ������������� ���������� ������� $2 \lfloor\log_2 (n-1)\rfloor + 2$ ��� ������������ $n$"~��������� ����������� �����������; �������������� �������������� ���������� ����� ������� ��������� �~�.~�������; ����������� ��������� ��� ����������� ������������ ������������� ������� �.�.~�����������; ���������� ����������� ������������ ��� ������������ ����������� ������� ������������ � ���������� ������������ �~�.~����������. 


\textbf{���������� ���������.}
������������, ���������� �~�����������, ���� ���������� �������� ����~00-01-00662-a, 03-01-00822-a; 
��� �������� �~������-�������������� ����� ������������� ������ �� 2009--2013 ���� (���. �������� �~02.740.11.0207),
������������ <<���������� �~�������������� ���������>> ���� ��.~�.\,�.~��������
(����� ������������� �� �~11.G34.31.0057),
��������� �~477 �~�~984 �~������ ������� ����� ���. ������� �� ��� ���� (2014--2016~��.) �~���. �������� �~1.5768.2017/�220 �� ��� ����.


%\textbf{contrib}{������ ����� ������.}{%
%���������� ����������� �~�������� ���������, ��������� �� ������, �������� ������������ ����� ������ �~�������������� ������.
%���������� �~���������� ���������� ����������� ����������� ��������� �~����������, ������ ����� ����������� ��� ������������. 
%��� �������������� �~����������� ���������� �������� ����� �������.
%}

%%%%%%%%%%%%%%%%%%%%%%%%%%%%%%%%%%%%%%%%%%%%%%%%%%%%%%%%%%
%
%     ������� ���������� ������
%
%%%%%%%%%%%%%%%%%%%%%%%%%%%%%%%%%%%%%%%%%%%%%%%%%%%%%%%%%%

%\newpage % ��� ������������
%\section*{������� ���������� ������}
\nsection{������� ���������� ������}

%\textbf{�� ��������} ���������� ������������ ��������������� ������, �������������� ���� �~��������������� ������� ������� ������������, �������� ������������ ���������� ���������� �����������, ������������ ��������� �� ������ ������� ���������.

�~\textbf{������ �����} ������������� �~���������� ������� �������������� ������� �~�����, ������������ ����� �~�������� ����� �����������.
�~��������~1.1 �~1.2 �������� ����������� ������� 
������ ������ �~������ �������� ��������������, ��������������.
�~�������~1.3 ���������� ������������ ����� ������� ������ ��������� ���������� %����� �~���������� 
�, �~���������, ������ NP-������ �����. 
\emph{�������� ���������� ������} ����� ����� �������� ����������� ������������ ������������� � � �������������� ��������� ���, ������� �������, ������������ �������������� �������� ������������ �� ��������� � ��������� ����� ���������� ��� � �������.

������������ ������������ ������ ������������� ����������� ���������� �~�������~1.4. ��� �� ���������� ����������� \emph{�������� ������ ������������� �����������}, �������������� ����� (� ������ ��������� �~������ ����������� ���������) ������:
\begin{enumerate}
	\item ���� ������� ������ $L$, $L \in P$. ����� ����� $I \in L$ ���������� \emph{����� ������}.
	\item \emph{�����������} $d \from L \to \N$ (������������� ���������).
	\item \emph{�������� ������������} $g \from \Z^d \times L \to \{\text{����}, \text{������}\}$, $g \in \NP$, ������������ \emph{��������� ���������� �������} 
	\[X = X(I) = \Set*{\bm{x} \in \Z^d \given \size(\|\bm{x}\|_{\infty}) = \poly(\size(I) + d(I)) \text{ � } g(\bm{x}, I)}.\]
\end{enumerate}
�������� ������� \emph{��������������} ������ (���������� ������) �������� � ���~$I$ �~\emph{������� ������} $\bm{c} \in \Z^d$.
%������� ������� �������: $f(\bm{x}, \bm{c}) = \bm{c}^T \bm{x}$, ��� $\bm{x} \in X$.
���� ������ "--- ����� ����� ���� ���������� ������� $X(I)$ �����, �� ������� ������� ������� $\langle\bm{c},\bm{x}\rangle$ ��������� ������������ ��������.
��������� ������� $\bm{x}$ ���������� \emph{�����������.}

%%%%%%%%%%%%%%%%%%%%%%%%%%%%%%%%%%%%%%%%%%%%%%%%%%%%%%%%%%
%     ����� 2
%%%%%%%%%%%%%%%%%%%%%%%%%%%%%%%%%%%%%%%%%%%%%%%%%%%%%%%%%%

�� \textbf{������ �����} �������� ����������� ��� ���� ��������������� ������ ������������ �~���������� ����� ��������� �� ���� ���� ������. 

�~�������~2.1
�������� ����������� ��������� �������������� �������� ������ ������������� �����������. �~������ ����� $I$ ��������������� ������ ����������� ������������ $\conv(X(I))$, �������������� ����� �������� �������� ��������� ���������� �������. ����� �������, �� ���� �������������� ������ ���� $I$, ���������� ��������� �������������� ������, ���������� ����� \emph{�������������}. %��������� �������������� ���������� \emph{�������������}, ���� ��� ��� �������, ������������ ��������������� ������, ������������� ���������.
����� ����� ��������������, ��� �������, ���������� ��������� ��� ������, �� ���� ��� V"~��������. 
��� ������, ����� �� ������� ������ $\bm{c}$ ������ ������������� �������������� �������� ����������� ���� $\langle\bm{c}, \bm{a_i}\rangle \le 0$, $i \in [k]$, ��������������� \emph{������� ������} ������������ ��� ����� ����������� ������������� $\conv(X(I))$ �~������ %<<������������>> ������� �������� 
$\cone\{\bm{a_1}, \dots, \bm{a_k}\}$.
�~���� �� ������� ���������� ����������� ���������, ����� ����������� �~���������� ������������� �������������� �~���������: ������ ������������� ������������� $\BQP(n)$, $n \in \N$; 
������������� ������ �~������� $\Knap(\bm{a},b)$, $\bm{a} \in \Z^n$, $b \in \Z$;
�������������� ����� $\Path(n)$ �~������� $\Dipath(n)$, $n\in \N$;
�������������� ������������� ������ $\TSP(n)$ �~������������� �������� $\ATSP(n)$;
���������������� ������������� $\Perm(n)$; ������������� ������ �~����������� $\Birk(n)$; ������������� $\Stable(G)$ ����������� �������� �~����� $G=(V,E)$;
�������� ������ �~���������� ������ �~������������ ����������������� ���� �������� $\ShortP(n)$; ��������� ������ �������������� �~���������.

�~�������~2.2 ����������� ������ ������������� ������ �������������� �����. �������� �������� ������� ������ ������������� ��������� ������. %, ��� ��� ��� ����� ����������� �~����������. 
����������� ��������� �� ���� ���� ����������.

������~2.3 �������� �������� ������ ��������� ������ ��� ����� ������������� ������ �������������� �����, ��� ����� ������, ������� �~�������� �����, �~�����, ��������� �~���������� ����� ����������������, �������� ����� �~������ ���������� ������� ����.

�~�������~2.4 �������� ������� ���������� ������������� �~���������� ������� ����� ��������� �� ���� ���� ������. \emph{�����������} ������������� $P \subseteq \R^d$ ���������� ������������ $Q \subseteq \R^n$ ������ �~�������� ������������ $\alpha \from \R^n \to \R^d$, ��������������� ������� $P = \alpha(Q)$.
����������, �~������ �������, ��������� ���, ��� ������ ����������� �� ������������� $P$ �������� �~������ ����������� �� ��� ���������� $Q$.
����� �������� ����������, ����������� ��� �������� ���������� $Q$, ���������� \emph{�������� ����������}. �������� �������, ����� ������ ���������� ����������� ����������� ������ ����� ����������, ����������� ��� �������� ��������� ������������� $P$. \emph{���������� ����������} $\xc(P)$ ������������� $P$ ���������� ����������� ������ ����� ���� ��� ����������. 

%������ ��������, ��� ��������� ���������� ���������� ����� ������������ �������������, �~������ "--- ������ ��� ������ �~������ �����������. ����� �������, ��������� ���������� ����� ������������� �~�������� ������� ������ ��������� ��������������� ��������������� ������. ����� ����, ��� �������������� �������� �~������� ����������� ����������. �~���������, �� �������� ���������� ����� ����� ������������� ��������������� ������������� "--- ������ �������������� �������� ������� ���������� ������"=����������� �������������~\cite{Yannakakis:1988}.

����� $M \in \{0,1\}^{n\times k}$~--- ������� ����������.
��������� $I\times J$, ��� $I\subseteq [n]$, $J\subseteq [k]$, ���������� \emph{0"~���������������} �~�������~$M$, ���� $M(i,j) = 0$ ��� ���� $i\in I$ �~$j\in J$.
\emph{������������� ���������} ������� $M$ ���������� ��������� 0"~���������������, ����������� ������� 
��������� �~���������� ����� �~$M$.
\emph{������ �������������� ��������} ������� ���������� ���������� ����� 0"~���������������, ����������� ��� � �������������� ��������.
����� �������������� �������� ������� ���������� ������"=�����������  ������������� $P$ ���������� $\rc(P)$.
��� �������� � ��������� ���������� $\xc(P)$ ������� � ������������ $\dim(P)$, ������ ������ $\vertices(P)$, ������ ����������� $\facet(P)$ � ������ ���� ������ $\face(P)$ ���������� �������������~\cite{Yannakakis:1988,FioriniKPT:13}:
\[
\dim(P) + 1 \le \log_2 \face(P) \le \rc(P) \le \xc(P) \le \min\{\vertices(P), \facet(P)\}.
\]
�������������� ��������� ����� ������� �~���, ��� $\rc(P)$ ���� ������ ������ ������ ������, � $\xc(P)$ "--- ������� ������ �������� �������������� ��������� ��������������� ��������������� ������.

�~�������~2.5, �� ������ ������, ���������� �� ������ �����,
������������� ����� �������, ������ ������� �� ������� ��������� ����������� ����� �����������.
\begin{comment}
� ������, ��~������������� ������ �������, ��� ������������� NP"~������� ����� �� ������ ������� �������� ������� ����������. ��������: NP"~������� ������ ������������� ����������� ������, ��������� ������� �����, ������������������� �������� ����� �����, ������������������� ��������� ���������� �~����� �������������� �������� ������� ���������� ������"=�����������.
����� ��� �������� ����������� ������� ������� ��������������� ���������, ��������������� �~������ ����� ������� ���������������, ����� ������������� ����� ������ ������� ������������ ��� �������� ���������� ��������� ������ �������������� ������ ������.
�~����� �~���� ������������� �������� ��������� ������� ������ ���������. 
����� �� �������������� ������������ ����� ������ ��������� ��� ��������� �������� ��������������? 
����� ������ �� ������ ��������� ������ ���� ����� ������� �~��������� ��������� ������������"=�������������� ������������� ��������������?
������ �� ��� ������� ���������� �~������ 3--5.
��� ������ ��~������������� �� ������ ����� ������������� ����� ����������� ������ ��������� ������.
���� �� ����� ����� ������ ��������������� ������������� �~���������� ��������������� ��������������� ������?
�~����� �~���� ��������� �~������� ����� ������ ���������.
����� ��������� �~��������� ����� ������������"=�������������� �������������� ������������� �������� ��������� �������� ��������� ��������������� ������?
���� �� ����� ����� ������������� ����� ������������� �~���������� ������ ����������� �� ���?
������������ ���� ������� ��������� ����� 7 �~8.
\end{comment}


%%%%%%%%%%%%%%%%%%%%%%%%%%%%%%%%%%%%%%%%%%%%%%%%%%%%%%%%%%
%     ����� 3
%%%%%%%%%%%%%%%%%%%%%%%%%%%%%%%%%%%%%%%%%%%%%%%%%%%%%%%%%%

\textbf{�~������� �����} ������ ����� �������� ����������, ������� ������������ �~����������� ���� ������.

�~������ ������� �������� ����������� �������� ���������� �����, �������� ������� ��� ��������� ��� ����������� �~����������� ��������.
�������� ������ ������������� ����������� $(L,d,g)$ \emph{������� ��������} �~������ $(L',d',g')$, ���� ���������� ���������� �� �������������� (������������ ����� ������� ������ ������ ������) �����:
\begin{enumerate}
	\item 
	�������������� $\tau \from L \to L'$.
	\item 
	�������� ���������� ��� ������� ���� $I \in L$ ��������� ����������� 
	$\alpha\from \R^d \to \R^{d'}$, ��� $d = d(I)$, $d' = d'(\tau(I))$.
	\item 
	%���������
	������� $\beta\from Y \to X$, ��� $X = X(I)$ "--- ��������� ���������� ������� ������ ������, �~$Y$ "--- ��� ��������� ���� ����� ���������� ������� $\bm{y} \in X' = X'(\tau(I))$ ������ ������, ��� ������� ��~������� �������� ������� ������ $\bm{c} \in \R^d$ �����, ��� $\bm{y}$ �������� ����������� �������� ������ ������ �~������ $(\tau(I), \alpha(\bm{c}))$.
	������ ��� ������ $\bm{y} \in Y$ �~������ $\bm{c} \in \R^d$, $\bm{y}$ �������� ����������� �������� ������ ������ �~������ $(\tau(I), \alpha(\bm{c}))$ ����� �~������ �����, ����� $\beta(\bm{y})$ �������� ����������� �������� ������ ������ �~������ $(I,\bm{c})$.
	% ������� $\beta$  �� ������� ���� ���������� �� ��������� ��������. ����� $x$ �����, ��� $x \ne \beta(y)$ �� ��� ������ $y \in Y$. ����� �������� ��� ��������. 1. ���� $x$ �������� � ������ ������ ����������� ����������� ��� ��������� ������� $\bm{c}$, �� ������ ����� ����������� ������� (� $\iff$) �� ����������� �������, � ����� ����� ����� ����������� ��� ��������� $y$. 2. ���� �� $x$ �� �������� ����������� ����������� �� ��� ����� $c$, �� �� ����� ����������.
\end{enumerate}

��� ����������� �������� ������������������� ������� ����������� �������� ���������� ��~������������ ����������� ������~\citemy{MaksimenkoDiss:2004}.
�������� �������: ����������� ���������� ������������ ��������� ����������� $\alpha$ �~������� $\beta$; ������ ����������� ��������� ���������� ������� �� ������ ��������� ���� ������.


�~�������~3.2 ���������� 
%����������� ����������� ����� ������������� �~
����������� ��������� ��������� ������������ �������� ������ ������.
\emph{�������� ���������� ������������ �������� ������} ������ �������� ����������� �� ��������� $X \subset \R^d$ ���������� ������������ ������� ����:
\[
K(\bm{x}) = \Set*{\bm{c}\in \R^d \given  \langle\bm{c}, \bm{x}\rangle \ge \langle\bm{c}, \bm{y}\rangle, \ \forall \bm{y} \in X}, 
\]
��� $\bm{x} \in X$, ������ �~��������� ���������� ������ �� ������, ����������� ������� ����� ����������� ������������ $\R^d$.
%������ $K(\bm{x})$ �~$K(\bm{y})$ ���������� \emph{��������}, ���� $\dim (K(\bm{x}) \cap K(\bm{y})) = d - 1$.
�������� ��������� �������� ������������ �~������������� $P=\conv(X)$ ������������.
%~\cite{BondBook:1995}. 
%�~���������, ������ ����� ������������� ������� ������������� $P$ �~��� ������� ����� ������������� ������ ����� �~������ �����, ����� ��������������� ������ ������.
�� �������� �~�������� ���������� ����� ������������ ������������ ��������� ��������� �������� ������ $Q \subseteq \R^d$ (��������������, ��� $Q$ "--- �������) ������ �������� ����������� �� $X \subset \R^d$. ��� ������� ��~��������� $K(\bm{x},Q) = K(\bm{x}) \cap Q$.
%, ����������� ������� ��������� �~������������ $Q$. 
��� ����������� ������� �~��� �������, ����� �� ������� ������ ������������� �������� �����������. ��������, �~������������ ������ �~���������� ���� "--- ����������� ����������������� ���� ����� �����.
�~����� ������� �������� ����������� �������� ���������� ��������� �������� ������ �����, ������������ �� ����������� �������� ���������� ����� ������������ ������������ ��������� ����������� $\alpha$ �~�������~$\beta$.

�~�������~3.3 �������� ������������ ������ ��������� ��������������: ���� ������������ $P$ ������� ������������ �������������~$Q$ ��� �� ��� �����, ���������� ����������� $P \lea Q$. 
%���� �� ������������� $P$ �~$Q$ ������� ������������, ����� $P =_A Q$. 
����������� $P \lea Q$ ��������� ���������� ��������� ������������"=�������������� �������������� �������������� $P$ �~$Q$: ����� ������ �~�����������, ��������� �������� ������ (��������, �������� �����), ��������� ����������, ����� ������������� �������� ������ ���������� ������"=����������� �~��������� ������.
����� �~���� ������� ���������� ��������� ������� �������� ������������� ����������� $\lea$. �~����� ������� ���� ����������� \emph{������������� ��������}, ��������������� ����� �������� �������� ��������� 
\begin{equation*}
\Pack(A) = \Set*{\bm{x}\in\{0,1\}^n \given A \bm{x} \le \bm{1}},
\quad \text{��� } A\in\{0,1\}^{m\times n},
\end{equation*}
� ������������� ���������
\(\Part(A) = \Set*{\bm{x}\in\{0,1\}^n \given A \bm{x} = \bm{1}}\).
��������������� ��~����������� ������� \(\Part(A) \lea \Pack(A)\).

�~��������� ������� ������� ����� �������� ����������� �������� ���������� �������� ��������������, �������� ����� ������������ �~��������� �����.
����� ����� ���� ������������� �~����������� ������������, �~������� �� ���������, ����� �������� \emph{��������} ������������� (�� ������ �~�������� ����������).
C�������� �������������� $\Pf$ \emph{������� ��������} �~��������� �������������� $\Qf$, ���� �������� ������������� ���������� (������������ ������� ������������� $P\in \Pf$):
\begin{enumerate}
	\item 
	�������������� $\tau$ ���� $I$ ������� ������������� $P = P(I)\in \Pf$ �~��� $I'$ ������������� $Q = Q(I') \in \Qf$.
	\item 
	�������� ����������� \(\alpha\from \R^d \to \R^{d'}\), $d = d(I)$, $d' = d'(\tau(I))$,
	�����, ��� ������������ $\alpha(P(I))$ �������� ������ (�������� �������������) ������������� $Q(\tau(I))$ �~������� ������������ $P(I)$.
\end{enumerate}
���� �������� ���������� $\Pf$ �~$\Qf$ ���������� ���: $\Pf \propto_A \Qf$.  

��������������� ��~����������� �~������������� ����� ������ ��������� ��������� �����������.
����� $\Pf \propto_A \Qf$ �~� ��������� $\Pf$ ���� �������������, ������� ���� ��� ��������� ��~��������� �������:
��������������������� ����� ������ ��� ����������� (������������ ������� �������������); ������������������� �������� ����� ����� �������������; NP-������� �������� ����������� ������; ������������������� ����� �������������� ��������; ������������������� ��������� ����������.
����� �~$\Qf$ ������� ������������� �~���� �� ����������.

�������� ���������� �������~3.4:
\begin{enumerate}
\item 
��������� �������������� ����������� ��������, �������������� �������� �~�������������� ��������� ������������ ������������ �������� ����������~\citemy{Maksimenko:2015DAN}.
\item
��� ������� $n\in \N$ ���������� ���� $G = (V,E)$, $|V| = n(n+1)$, $|E| = n(2n-1)$, �����, ��� $\BQP(n) \lea \Stable(G)$~\citemy{Maksimenko:2015DAN,Maksimenko:2016bool}.
���� �� ���� $G=(V,E)$ ��������, �� ����������� $\Stable(G) \lea \BQP(n)$ ���������� �� ��� ����� $n$.
\end{enumerate}
�~����� ������� ��������������� ����� ����� �������� ����������� �������� �������������� �~�������� ����������� �������� ��������� ����������� �������� ������ �����. 

%����������� �������� ���������� �������� ��������� ������������ �~������������ ����������� ������~\citemy{MaksimenkoDiss:2004}. ����������� �������� ���������� �������������� ������������ �~\citemy{Maksimenko:2017}.
%���������� ������� ����� ������������ �~\citemy{Maksimenko:2015DAN,Maksimenko:2016bool}.

%%%%%%%%%%%%%%%%%%%%%%%%%%%%%%%%%%%%%%%%%%%%%%%%%%%%%%%%%%
%     ����� 4
%%%%%%%%%%%%%%%%%%%%%%%%%%%%%%%%%%%%%%%%%%%%%%%%%%%%%%%%%%

�~\textbf{�����~4} ����������� ��� �����������, ��������� �~�������� �������� ����������. 

�� �������� �~��������������� �������� �~���������, �~�������~4.1 �������� ����������� �������������� �������� �~������� �������� ���������. 
\emph{�������������� ��������} ���������� �������� �������� ���������
\begin{equation*}
%\label{def:Cover}
\Cover(M) = \Set*{\bm{x}\in\{0,1\}^n \given M \bm{x} \ge \bm{1}},
\quad \text{��� } M\in\{0,1\}^{m\times n}.
\end{equation*}
\emph{�������������� ������� ��������} ���������� �������� �������� ���������
\(\DCP(B) =  \Set*{\bm{x}\in\{0,1\}^n \given B \bm{x} = \bm{2}}\),
��� $B \in \{0,1\}^{m\times n}$, ������ ������ ������ ������� $B$ �������� ����� ������ ������� �~�� ����� ������� ��������.
������� ��� ��������� �������������� ���� ����������� �����~\cite{Matsui:1995},
�� �� ���� ��������, ��� $\DCP(B) \lea \Cover(M)$, ��� ������� $M \in \{0,1\}^{4m\times n}$ �������� ����� ��� ������� �~������ ������.
������� �������� �~���� ������� ��������� ����������� �������������� $\NPadj(A)$, ��� ������� $A \in \{0,1\}^{m\times n}$ �������� ����� ��� ������� �~������ ������. ��� ������������� �������� ��������������� ������� ��������.
��������~\cite{Matsui:1995}, ��� ������ ������������� ����������� ������ ��� $\NPadj$ NP"~�����. ��������� ������������ ������� �������� ��� �����������, �������������� �~\citemy{Maksimenko:2017}:
\begin{enumerate}
	\item ������������� ����������� �������� $\Stable$ ������� �������� �~��������� �������������� $\NPadj$.
	\item ���� ������������ $\NPadj(A)$ �� �������� ��������, �� $\NPadj(A) \lea \Stable(G)$ ���������� �� ��� ������ ����� $G$.
\end{enumerate}
��������� �������� ������� �~����������� ����������� ������� �������������� ������� �������� �� �������������� ����������� �������� �~������� ���������� �~��� ��������.

{\sloppy
�����, �~�������~4.2 ��������������� ��������� �������������� �~NP-������ ��������� ����������� ������: ������������� ������ �~������� $\KnapEq(\bm{a},b)$, ������������� ������ �~��������� ����� $\PRT(\bm{a})$, ������������� ������ �~����������� �~������������ $\CAP(\bm{a},b)$, ������������� ������ �~������������ $\SAT(U,C)$, ������������� ������ �~��������� �������������� $\POP(n)$, ������������� ���������� ��������� $\Cubic(n)$.
���������� ������ �� ��������� �������~\cite{Fiorini:2003} �~���, ��� ��������� �������������� ������ �~3-������������ ������������ ��������� �������������� ������ �~��������� �������������� �~����� ������ �������� ���������� (������� �������� ���������� �~��� ������ �� ������������). �~��� �� ������ ��������, ��� ������������� ������ �~$k$"~������������ �� ����� ���� ������� ������� �~��������� �������������� ������ �� $m$"~������������, ���� $k > m$.
�������� ����������� ����� ������� �������� ����� ������������� ����, ��� ������������� ������� �������� $\DCP$ ������� �������� �~������������� ����������~\citemy{Maksimenko:2012DAN,Maksimenko:2013NP}.

}

�~�������~4.3 ��������������� ������������� �������� �������� �~������������� �������� �������� �~�����. ��������, ��� ������ ������������ ������������� $\BQP(n)$ ������� �������� �~������� ���������~\citemy{Maksimenko:2017LOP}, �~������������� ����������� �������� "--- �� �������.

�~�������~4.4 ��������������� ��������� ��������������, ������� ������� �������� ��������� ������. �~����� ������ �������� ���������� ��� ����������� �� ��� ������ ���������������.
������������� ������������� ������ �~����������� $\TAP(n)$ �~��������� �������� �������������� ��������� ����� ($\ColorA(G,k)$, $\ColorB(G)$ �~$\ColorC(G)$) ����� �~����� ������ ��������������� �~��������������� ����������� ��������. �~��������� �������������� ������������ ������ �������� �������������� $\QLOP(n)$ �~������������ ������ �~����������� $\QAP(n)$ ����������� ������������ ��������� ������� ������������ �������������� $\BQP(n)$.
���������� ����� ������� ������������ �~\citemy{Maksimenko:2016bool}.

�~�������~4.5 ��������������� ��������� �������������� �����, ����� ��������� �~������� ������������.
��������, ��� ������������� ������ �~������������ $\SAT(U,C)$ ������� �������� �~�������������� ������������� �������� $\ATSP(n)$~\citemy{Maksimenko:2011}.
����� ��~��������� ����� ����������� �������� ��, ��� ����� $d$-������ 0/1"~������������ �� $2^d - k$ �������� ($0 \le k \le 2^d - 1$) ������� ������������ ��������� ����� ������������� $\ATSP(n)$ ��� $n = (2k+1)d$.
����� ������� �~�������������~\cite{Billera:1996} �������� ��� ����������� ��� $n = (4k+1)d$ ����� ����������.
������ ��������� �������, �������������� �~\citemy{Maksimenko:2013TSP}, ������������� ��������� ����� ����� ����������� $\BQP$ �~$\ATSP$:
$\BQP(m) \lea \ATSP(n)$, ��� $n = 2 m^2 - m$.
�~����������~4.5.2 ��������������� ��������� �������������� ��������� �����: ����������� ����, ����������� (��)����, $s$"~$t$ (��)����, ����������� $s$"~$t$ (��)����.
��������, ��� ������������� ������������� �������� ������� �������� �� ���� ���� ����������. �� ����� �������, �~���������, ��� ����� �������������� ���� �������� �������� ������������������� �������� ������ �~������ ������������� ����������� ������ ��� ��� NP"~�����.

�� �������� �~�������� ������������� ��������������� �~�������~4.6 �������� �~������������ ������ ������������� $\BPP(n,p)$ ������� $p$. 
��� $p=2$, $\BPP(n,p)$ ��������� �~$\BQP(n)$, �~��� $p=1$, $\BPP(n,p)$ "--- $n$"~������ 0/1-���.
��������, ��� $\BPP(n,p)$ $s$"~���������� ���
$s \le p + \left\lfloor p / 2 \right\rfloor$.
��� $m \in \N$ �~$k \ge 2m$ ��������, ��� $\BPP(k,2m) \lea \BQP(n)$ ��� $n > 2 \binom{k}{m}$.
�������������, ��� ������ $k \in \N$ �~$n \ge 2^{2\cdot \lceil k/3\rceil}$, 
$\BQP(n)$ ����� $k$"~����������� ����� �� ������������������� ������
$2^{{\Theta}\left( n^{1 / {\left\lceil k/3\right\rceil}}\right)}$ ������.
�� ����� �~�� ������������� ����� �������� ����\'���� �������, ��� �� ���� ����������� ���� ���������� �������������� NP-������� ����� ������� �������������, ���������� $k$"~����������� ����� �� ������������������� (������������ ����������� �������������) ������ ������. ���������� ������� ������������ �~\citemy{Maksimenko:2013k}.

�~��������� ������� ����� 4 ��������������� ������ �~����������� �~������ �~���������� ������ �~������������ ����������������� ���� ��������. ��������, ��� �������� ��������� ��������� �������� ������ ��������� ������� �������� �~��������� ��������� ������������ �������� ������ ������~\citemy{MaksimenkoDiss:2004}.
��� ���������, ���� �������� ���������� ������� $\ShortP(n+1)$ �������� ��������� ����� ������������� �������� $\Birk(n)$, $n \in \N$.


%%%%%%%%%%%%%%%%%%%%%%%%%%%%%%%%%%%%%%%%%%%%%%%%%%%%%%%%%%
%     ����� 5
%%%%%%%%%%%%%%%%%%%%%%%%%%%%%%%%%%%%%%%%%%%%%%%%%%%%%%%%%%

�~������ ������� \textbf{�����~5} �������� ������� ����������� �������� ����������, ������������ �� �������� ���������� ����������� ����������� ������������ ��������� �����������. 
%�������� ������� �~�������� ���������� �������������. ���� ��������� ����� ������������� $Q$ ��� �� ���� ���� ������������ �������� ����������� ������������� $P$, ����� ������������ ����������� $P \lee Q$.
��������� �������������� $\Pf$ \emph{���������� ������� ��������} �~��������� �������������� $\Qf$, ���� �������� ������������� ���������� (������������ ������� ������������� $P\in \Pf$):
\begin{enumerate}
\item 
�������������� $\tau$ ���� $I$ ������� ������������� $P = P(I)\in \Pf$ �~��� $I'$ ������������� $Q = Q(I') \in \Qf$.
\item 
������� �������� ��������� $D\bm{y}=\bm{c}$, �������� �����
\(F = \Set{\bm{y}\in Q \given D\bm{y}=\bm{c}}\)
������������� $Q$.
\item 
�������� ����������� \(\beta\from \R^{d'} \to \R^{d}\), $d = d(I)$, $d' = d'(\tau(I))$, �����, ��� $P = \beta(F)$.
\end{enumerate}
�����������: $\Pf \propto_E \Qf$.  
�� ������ ������� �������������� ����������� ���� $\Pf \propto_E \Qf$ ������������� �����, ��� ����������� $\Pf \propto_A \Qf$.
������� ������ ���������� ����������� �������� ������ ��������� �������� �������.
�~���������, ����� ��������, ��� NP"~������� �������� ����������� ������, ��������������������� ����� ����������� �~��������������������� ��������� ����� �����, ������ ������, �� ����������� ��� ����������� �������� ����\'����.
����� �~���� �� ������� ���������� ��������� �������� ������ ��������� ��� ����� ���� ����������. ��������, ��������, ��� ���� ������������ $P \subseteq \R^d$ �������� ������� ������������� $Q \subseteq \R^n$ ��� �������� ����������� $\pi \from \R^n \to \R^d$ �, ����� ����, $\pi(\ext Q) = \ext P$, �� ���� ������������� $P$ �������� ��������� ����� ������������� $Q$.

�~�������~5.2 ���������� ��������� �������� ����������� �������� ����������. �~�����, ��������� ���������� �������, �������������� ��������������� ����������� ����������� ����������� ����� ��������, ��� ��� (�������) �������� ����������.

�~�������~5.3 ��������, ��� ����� ��������� ��������������, �������� ������������ �������� ����������� ������ NP, ���������� ������� �������� �~������� ������������ ��������������~\citemy{Maksimenko:2012Cook}.
��� �����, ��� ����������� ���� �~��������� ������ ��������� �������������� ����������� ������������ ���� ����� ������������ ����������� �������� ����������.


%%%%%%%%%%%%%%%%%%%%%%%%%%%%%%%%%%%%%%%%%%%%%%%%%%%%%%%%%%
%     ����� 6
%%%%%%%%%%%%%%%%%%%%%%%%%%%%%%%%%%%%%%%%%%%%%%%%%%%%%%%%%%

�~\textbf{�����~6} ��������������� ����������� �������������.
��� ��������~\cite{McMullen:1970}, ��� �������� ������������ ������ ������ (����� �����������) ����� ���� �������� �������������� ��� �� ����������� �~� ����� �� ������ ������.
��������� ����� �������������� ����������� ������������� �������� ������� ����������������� ����� ��� �������� ������� ���� ������������� �����������.
�~������ ������� ����� �������� ����������� ������������ ������������� 
\(\CP_d(T) = \Set*{(t, t^2, \dots, t^d) \in \R^d \given t \in T}\), $T \subset \R$,
�~������������� ������� �������� �����~\cite{Gale:1963}, ���������������� ������������ ������, ���������� ���������� ����� �������������.
�~�������~6.2 ��� ������������� $\CP_d([n])$ ���������� �������� ����������� ������������ ������� $2\bigl(2\lfloor \log_2(n-1)\rfloor+2\bigr)^{\lfloor d/2 \rfloor}$ ��� $2 \le d < n$ (��������� ����������� �~���������� ������~\citemy{BogomolovFMP:2015}).
�~�������~6.3 ����������� ������ �������� ��� �������� $\dc$ ����"=����� $d$"~������� ������������ ������������� �� $n$ ��������:
\(\dc= n-d  - 
\left\lceil 
\frac{n-2d}{ \left\lfloor \frac d 2 \right\rfloor+1}
\right\rceil\)
��� $n > 2d$~\citemy{Maksimenko:2009}. 
��������� $\dc= n-d$ ��� $d < n \le 2d$  ���� �������� ��� �~1964 ����~\cite{Klee:1964}.


%%%%%%%%%%%%%%%%%%%%%%%%%%%%%%%%%%%%%%%%%%%%%%%%%%%%%%%%%%
%     ����� 7
%%%%%%%%%%%%%%%%%%%%%%%%%%%%%%%%%%%%%%%%%%%%%%%%%%%%%%%%%%

\textbf{�����~7} ��������� ������ ���������� ������� ����.
% ��� ������� �������� ����� ������������� �����������.
�~������ ������� ���������� �������� ���� ������, �������������� ��~\cite{BondBook:1995}. %(��. �����~\citemy{BondBook:2008}).
�������� ������������ ��������� ������� ���� �������� ��, ��� ��� ��������� ���������� ����� �������� ������ ����� ������������� (��������� ��������� ��������� �������� ������) �������� ������.
\begin{comment}
��������~(��. ����� �~�������~2.3.3), ��� ��� ������������ ������������� ���������� ����� (����������, ����������� �������� ������, ����������� ������) ��� �������������� �� ����������� ����������� �������������.
%(�~�������~7.2 ��������, ��� ������ �~���������� ���� �~������������ ����������������� ���� �������� ���� ������ �~���� ������.)
�~������ �������, �~������~3 �~4 ��������, ��� ������ ������������ ������������� $\BQP$ ������� �������� �~�������������� ����� NP-������� �����, ��� �����������, ������, 3-������������, 3-���������, �������� �~�������� ���������, ��������� �����, ���������� ������� �~������ ������. ��������, ��� �������� ����� ����� ������������� $\BQP(n)$ ����� $2^n$, �������� ����� ������ �������������� ��������� ����� ����� ������������������ �� ����������� ��������������.
����� ����, �~\cite{BondBook:1995} �����������, ��� ��������� ��������� ����������, ������ �������� ��� ������������ ��������� ������,
�������� �������� ��� ����������� ����, �������� �����--����� 
� ���������� ��������� ������ �~������ ��� ������ ������������
�������� ������� ��� <<�������>>.
\end{comment}

�~�������~7.2 ���������� �������������� �������� ��������� ��� ������ ������� ���� ������������� ���������� ��������� ������ � ���������� ����.
%�� ������ ����� �������� � �������������� ������� 2 ��~\cite{Bondarenko:1993SW3A} �������� ����� � ���, ��� �������� ����� ��� ������ �~���������� ���� �~������� �� $n$ �������� �~������������ ����������������� ���� �������� (� ����� ��� ������ �~������������ ������������ ����������������� ���� ���) ����� $\lfloor n^2 / 4\rfloor$.
�� ������ ����� �������� � �������������� ������� 2 ��~\cite{Bondarenko:1993SW3A} �������� ����� � ���, ��� �������� ����� ��� ���� ���� ��������� �����~$\lfloor n^2 / 4\rfloor$, ��� $n$ "--- ����� ������ �����, � ������� ������ ���������� ����.
�~������ ���������� �������~4.7, ��� ���� ������ ������ $\lfloor (n+1)^2 / 4\rfloor$ ��� ��������� ����� ����� ������������� ������ �~����������� $\Birk(n)$.
%�~���� ����� ������� ��������� ����.
%�~1977~�. �������� �~������ ��������~\cite[Theorem~6.1, Corollary~6.5]{Brualdi:1977II}, ��� ����� 2"~����������� ����� ������������� $\Birk(n)$, ����� ������ ������� �� ����� �����, �������� ����������, �~������������ ����� ������ ����� ����� ��������� �~���������� ���� ������� $\lfloor (n+1)^2 / 4\rfloor$.
���������� ����� ������� ������������ �~\citemy{MaksimenkoDiss:2004} �~\citemy{Maksimenko:2004}.

�~�������~7.3 ������������� ��� ������, ��������������� �������������� ������������ ����� ������� �~������ ��������� �����.
���������� �������������� ����, ��� �������� ����--�������� (���������� �����) ��� ������ �~����������� �� �������� ���������� ������� ����.
����� ����, ����������� ���������� ������������� ������ ����������� ����������,
����������� �� �������� �� ������������, �� �������������� ��������� �� ��~������ ���������� ������� ����. ���������� ������� ������������ �~\citemy{Maksimenko:2014MAIS}.

%%%%%%%%%%%%%%%%%%%%%%%%%%%%%%%%%%%%%%%%%%%%%%%%%%%%%%%%%%
%     ����� 8
%%%%%%%%%%%%%%%%%%%%%%%%%%%%%%%%%%%%%%%%%%%%%%%%%%%%%%%%%%

�~\textbf{�����~8} ��������� ��������� ������.
�����~��, ���� ������ ������������� �������� 
%(���������� ������������ �������� ���������� ���\-���-��\-���\-���\-���) 
�������������, �������� NP"~������� ������ �� ������������� ����������?
% (� ������ ����������� ������������� �~��������� �����)?
�~������ ����� �~�������� ����� ������������� ��������� ���������������: ����� ������ �������������, ����� ��� �����������, ������� �~�������� ����� �����, ����� �������������� �������� ������� ���������� ������"=�����������.

�~�������~8.1 ���������� ������� �������� ��������������, ��� ������� �������� ���������� ���� ������������� (�� ����������� ����� �������������� ��������) ����������� ���������� �� %����������� ����������� � ���������� ������� ������������� � 
�������� �������������� ��������� ��������������� ��������������� �����.

�~�������~8.2 ���������� �������� NP-������� ������ �����������, ������������� $\CBQP(n)$ ������� �������� �~���������� ���������� ��������� (�����������) ������ ������ ������������� ������������� $\BQP(n)$. ��������, ��� ������������ $\CBQP(n)$, $n \in \N$, ������������. �������������, ��������~\cite{FioriniKPT:13}, ����� �������������� �������� ��� ������� ���������� ������"=����������� �������������: $\rc(\CBQP(n)) = O(n^5)$. 
%����� �������, ��� ������ ��������� ������ ��������� �������������� NP"~������� ������, ����� �������������� �������� ��� ������� �������������.
��� ��������, ��� �� ���� ��~�������� �������������� NP"~������� �����, ������������� ����� �~������~3--5
�� ����� ���� ���������� ������� ������� �~��������� $\CBQP$.
�~������ �������, �~\cite[Theorem~4]{Braun:2015} �����������, ��� ����� ������������, ���������������� $\BQP(n)$ �~��������� $O(1/n)$, ����� ��������� ���������� ������� $2^{\Omega(n)}$.
�������������, ��������� ���������� ��� $\CBQP(n)$ ���������������: $\xc(\CBQP(n)) = 2^{\Omega(n)}$.
%��� ��������� ��������, ��� ������������� ������ �~�������������� �~������ ����� �������� ������������ ����������: �� ��������� ���������� ���������������, �~����� �������������� �������� �������������. 

�~�������~8.3 ���������� ������� ���� �������� ����� ������������� �����������, ������������� ������� ������������ ������������ 
�~����� �������� ������ ��������� ������ ���� �������������� ���������. 
��� ���� ������ ������ ��������� ��~�������������� �����, 
�~������ "--- NP"~������.
%(����� ������� ������� �� ���� ���������������� ��������� ������������� ������� ������������� ��������� ������.)
���� ��������� ������� �~���, ��� 
%� ������ �������� ���������������� �������������� ������ (������ ��������) 
�� ���� ����� ������������� �������������� �������������
(���������� ������������ ��� �������� ������)
�� ���� ����������� �������� ������������� ���������� ������ �� NP"~�������.
%����� �~���������������� ����������.

���������� ��������� ����� ������������ �~\citemy{Maksimenko:2016complexity}.

%�~\textbf{����������} ���������� ����� ���������������� ������������.

% Обзор литературы
%\input{review}

% Основная часть
%% Глава 1
% !TeX encoding = windows-1251
% !TEX root = MaksimenkoThesis.tex

\chapter{�������� �������}

� ���� ����� ���������� ����������� � ������������� ��������� �����, �������� ������� ��� ��������� ����������� ����������� � ����������� ������.
�~��������~\ref{sec:graphs} �~\ref{sec:polytopes} �������� ����������� ������� 
������ ������ � ������ �������� ��������������, ��������������.
�~�������~\ref{sec:complexity} ���������� ��������� ������� ������ ��������� ���������� %����� � ���������� 
�, � ���������, ������ NP-������ �����. ������������ ������������ ������ ������������� ����������� ���������� � �������~\ref{sec:CO}.
%������� ������������� (��������) ������ �������� � �������~\ref{sec:ProblemPolytopes}, ��� �� ���������� �������������� �������. ������~\ref{sec:Survey} �������� ������ ��������� ����������� �� ������ ����.
%� ���������, � ��� ������ ������������ ���������� � ��������� ������ �������������� �����, � ����� � ����������� ������������� ��������������.
%�~�������~\ref{sec:questions} ������������� ����� �������, ������ �� ������� ����� ������������ � ����������� ������.


\section{��������� � �����}
\label{sec:graphs}

\subsection{���������}

%��������� ����������� ����� ���������� ����� $\N$.
��� ��������� $\{1,2,\dots,n\}$, ��� $n\in \N$, ����� ������������ ������������~$[n]$. 
����� ����� %���������� �����, �� ������������� 
��������������� ����� $x$ ���������� $\lfloor x \rfloor$.
���������� �����, ������� ��� ������ ����� $x$, ���������� $\lceil x\rceil$.

����� $E$ "--- ��������� �������� ���������. 
��������� ���� ����������� ��������� $E$ ������������ $2^E$.
\emph{������������������ ��������} ������������ $T \in 2^E$ ���������� 0/1"~������ $v = \chi(T) \in \{0,1\}^E$ � ������������
\[
v_{e} = \begin{cases}
1,& \text{���� $e\in T$,}\\
0,& \text{���� $e\in E\setminus T$.}
\end{cases}
\]
���� �� �������� ��������� $E$ �������������, �� ������������������ ������ ����� ���� ��������� ��� $v = \chi(T) \in \{0,1\}^{|E|}$ � ������������
\[
v_{i} = \begin{cases}
1,& \text{���� $e_i\in T$,}\\
0,& \text{���� $e_i\in E\setminus T$.}
\end{cases}
\]
%����� �������, ������� ������ 0/1-������������� � $\R^d$ ����� ���������������� ��� ������������������ ������� ��������� ����������� $d$-����������� ��������� $S$.


\subsection{�����}
\emph{������} ��� \emph{����������������� ������} ���������� ������������� ���� $G = (V, E)$,
��� $V$ "--- �������� ���������, � $E$ "--- ��������� ��������� �������������� ����������� ��������� $V$.
�������� ��������� $V$ ���������� \emph{���������} ����� $G$, � �������� ��������� $E$ "--- ��� \emph{�������}.
������� $v$ � $u$ ���������� \emph{�������} ����� $\{v,u\}$.
��� ��������������� � ����������� ����� �� �������� ������� ����� (��������� $E$ �� �������� ���������� ���������) � ������ (����� ������ ����� $e \in E$ �������� ������� ���������).
���� ���������� \emph{������,} ���� ������ ���� ��� ������ �������� ����� ����� �����.
%������ ���� �� $n$ �������� ������������ $K_n$.

������� $v$ � $u$ ����� $G = (V,E)$ ���������� \emph{��������} � $G$, ���� $\{v,u\} \in E$.
���� �� $\{v,u\} \notin E$, �� ������� $v$ � $u$ ���������� \emph{����������.}
\emph{��������} ������� $v \in G$ ���������� ����� ������� � ��� ������.
����, ������� ������ ������� �������� ����� ����, ���������� \emph{����������.}
������������ ������ $V' \subseteq V$ ���������� \emph{������} � ����� $G$, ���� ����� ��� ������� �� $V'$ ������.
������������ ������ (��������) ����� � $G$ ���������� \emph{�������� ������} ����� $G$ � ������������ $\omega(G)$.
������������ ������ $V' \subseteq V$ ���������� \emph{�����������} � ����� $G$, ���� ����� ��� ������� �� $V'$ ��������.

����� $G' = (V', E')$ "--- ��� ���� ����.
����� $G$ � $G'$ ���������� �����������, ���� ���������� �������"=����������� ����������� $f \from V \to V'$ �����,
��� $\{v, u\} \in E$ ����� � ������ �����, ����� $\{f(v), f(u)\} \in E'$.
���� $G'$ ���������� \emph{���������} ����� $G$, ���� $V' \subseteq V$ � $E' \subseteq E$.
�����, ��� ���������, ����� ���� $G'$ ���������� �������� ����� $G$ ����� �������� ��������� ����� $G$.
������� $G'$ ����� $G$ ���������� \emph{�����������} ��� \emph{��������������}, ���� $\forall v, u \in V'$ �� $\{v, u\} \in E$ ������� $\{v, u\} \in E'$.

\emph{�����} � ����� $G$ ���������� ��������� ����� ���� 
\[P = \{\{v_1, v_2\}, \{v_2, v_3\}, \dots, \{v_{k-1}, v_k\}\},
\] 
��� $v_1$, $v_2$, \dots, $v_k$ "--- ������� ��������� �������, $k \ge 2$.
� ����� ������ ����� ��������, ��� ���� $P$ \emph{���������} ������� $v_1$ � $v_k$, � �������� ��� \emph{$v_1$-$v_k$ �����}.
���� ���������� \emph{��\'�����}, ���� ����� ��� ��� ������� ��������� ��������� ���� � ���� �����.
���� � ����� $G$ ���������� \emph{�������������}, ���� ������ ������� ����� ����������� ���� �� ������ ����� ����� ����.
%\emph{������} ���� ���������� ����� ������������ ��� �����.
\emph{�����������} ����� ��������� $v$ � $u$ � ����� $G$ ���������� ���������� ����� ����� � ����������� �� ����;
���� �� ����� ���� �� ����������, �� ���������� ���������� ������~$+\infty$.
\emph{���������} $\diam(G)$ ����� $G$ ���������� ���������� ���������� ����� ��� ��������� (����� �������������� ����� ���� ��� ������).

\emph{������} � ����� $G$ ���������� ��������� ����� ���� 
\[
C = \{\{v_1, v_2\}, \{v_2, v_3\}, \dots, \{v_{k-1}, v_k\}, \{v_k, v_1\}\},
\] 
��� $v_1$, $v_2$, \dots, $v_k$ "--- ������� ��������� �������, $k \ge 3$.
���� � ����� ���������� \emph{�������������}, ���� ������ ������� ����� ����������� ����� ���� ������ ����� �����.
���� � ����� ���� ����������� ����, �� � ��� ���� ���������� \emph{�������������.}
���� ��� ������ ���������� \emph{�����}, � ������� ��� "--- \emph{�������}.

������� $v$ � ����� $e$ ���������� \emph{������������}, ���� $v\in e$.
\emph{�������� ����������} ������"=����� ����� $G = (V,E)$ ���������� ������� $M \in \{0,1\}^{n\times k}$, $n = |V|$, $k = |E|$, �������� ������� ������������ ��������� �������:
\[
M_{ij} = 
\begin{cases}
1, &\text{���� $v_i\in e_j$,}\\
0, &\text{�����.}
\end{cases}
\]

����� ��������, ��� ������������ ����� $E' \subseteq E$ \emph{���������} ������� $V$,
���� ������ ������� $v \in V$ ���������� ���� �� ������ ����� �� $E'$.
����������, ������������ ������ $V' \subseteq V$ \emph{���������} ����� $E$,
���� ������ ����� $e \in E$ ���������� ���� �� ����� ������� �� $V'$.

��� ����� � ����� ���������� \emph{��������}, ���� ��� �������� ����� �������, � ��������� ������ ��� ���������� \emph{����������.}
��������� ������� ��������� ����� ����� ���������� \emph{��������������}. 
�������������, ����������� ��� ������� �����, ���������� \emph{�����������.} 
����� �������, ����������� ������������� ����� ���� ������ � ������ � ������ ������ ������.

\emph{��������} � ����� $G = (V, E)$ ���������� ��������� ����� ���� 
\[
\delta(U) \coloneqq \Set*{\{u,v\} \in E \given u\in U,\ v\in V \setminus U}, \quad \text{��� } U \subseteq V. 
\]
�� ����������� �������, ��� $\delta(U) = \delta(V \setminus U)$.
������ $\delta(U)$ ���������� \emph{$s$-$t$ ��������}, ���� $s \in U$ � $t \in V\setminus U$\label{def:stcut}.
%\emph{���������� ���������} $X$ ���������� ����� ��� ������� ���������������� �����������, ����������� ������� ��������� � $X$.
���� $G = (V, E)$ ���������� \emph{����������}, ���� ��������� ��� ������ $V$ ����� ������� �� ��� \emph{����} $U$ � $V \setminus U$ ���, ��� $\delta(U) = E$. 
���������� ���� ���������� \emph{������ ����������}, ���� $\{u, v\} \in E$ ��� ����� $u \in U$ � $v \in V \setminus U$.


���� $G = (V,E)$ ���������� \emph{�������"=����������}, ���� �� ��������� ��� ����� $E$ ������ ������� ����� $f \from E \to \R$.
����� $f(e)$ ���������� \emph{�����} ����� $e \in E$.
\emph{����� ������������} $E' \subseteq E$ ��� \emph{��������} $G' = (V', E')$ �������"=����������� ����� $G$ ���������� ����� ����� �������� � ���� �����.
���� $G$ ���������� \emph{��������"=����������}, ���� ������ ������� $g \from V \to \R$.
� ����� ������ ����� $g(v)$ ���������� \emph{�����} ������� $v \in V$, � \emph{����� ������������} $V' \subseteq V$ ���������� ����� ����� �������� � ���� ������.

\subsection{�������}

\emph{��������������� ������} ��� \emph{��������} ���������� ������������� ���� $D = (V, A)$, ��� $V$ "--- �������� ���������, ���������� \emph{���������� ������}, $A$ "--- ��������� ��������� ������������� ��� ������, ���������� \emph{������}. 
�����, ��� � ��� ������, ����� ������������, ��� � $A$ ��� ������� ��� � ������.
����������� ������������� ���� ������� ��� ������ ����������� (������ � ���������� �����������) �� �������.

����� $(v, u) \in A$.
������� $v$ ���������� \emph{�������} ���� $(v, u)$, � ������� $u$ "--- �� \emph{������}.
������ �� ���� ������ � ���� $(v, u)$ ���������� \emph{������������} ���� �����.
������ ����� �������� \emph{������,} ���� ������ ������������� ���� ��� ������ �������� ���� ����� �����, �� ���� $|A| = |V| (|V| - 1)$.
%������ ���������� \emph{��������,} ���� ��� ����� (���������������) ���� ������ $u,v \in V$ ����� ���� �� ��� $(u, v)$ � $(v, u)$ ����������� $A$.

\emph{�������} ��� ������ \emph{�����} � ������� $D$ ����� �������� ��������� ��� ���� 
\[P = \{(v_1, v_2), (v_2, v_3), \dots, (v_{k-1}, v_k)\},
\] 
��� $v_1$, $v_2$, \dots, $v_k$ "--- ������� ��������� �������, $k \ge 2$.
������� $v_1$ � $v_k$ ����������, ��������������, \emph{�������} � \emph{������} ���� $P$, 
� ������� $v_2$, \dots, $v_k$ "--- ��� \emph{����������� ���������}.

\emph{��������} � ������� $D$ ���������� ��������� ��� ���� 
\[
C = \{(v_1, v_2), (v_2, v_3), \dots, (v_{k-1}, v_k), (v_k, v_1)\},
\] 
��� $v_1$, $v_2$, \dots, $v_k$ "--- ������� ��������� �������, $k \ge 2$.
������, �� ���������� ��������, ���������� \emph{������������.}

\emph{��������} � ������� $D = (V, A)$ ���������� ��������� ��� ���� 
\[
\delta^+(U) \coloneqq \Set*{(u,v) \in A \given u\in U,\ v\in V \setminus U}, \quad \text{��� } U \subseteq V. 
\]
������ $\delta^+(U)$ ���������� \emph{$s$-$t$ ��������}, ���� $s \in U$ � $t \in V\setminus U$\label{def:stdicut}.

\begin{comment}
������ $D = (V, A)$ ���������� \emph{������������}, ���� �� ������� $(v, u) \in A$ � $(u, w) \in A$ ������� $(v, w) \in A$.
��� ��� �� �� ������������� ����� � �������, �� �� �������������� ������� ������������.
����� �������, ��������� ��� ������������� ����� ������ ��������� ������� �� ��������� ������ ����� �, ��������, ������ ��������� ������� ����� ���� ����������� ���������� ��� ���������� ������������� �����.
���� ��� ������ ���� ������ $u,v \in V$ � ��������� $A$ ������ ����� ���� �� ���� ��� $(v, u)$ � $(u, v)$, �� ��������������� ������ ���������� \emph{��������}.\label{def:linearOrdering}
������������ ������ ������ �������� ������� �� ��������� ������.
������� �����, ������ � ��������� (��������) ������� �� ����� ����� ������������� ��������������� ������������ ������ (������).
\end{comment}

\section{�������������}
\label{sec:polytopes}

%� ���� ������� ������������� ��������� .
��� ��������� ���������������� ������� � ������ ������ �������� �������������� ����� �������������� ������������ ����������~\cite{Emelichev:1981} �~\cite{ZieglerBook}.


��� $\R^d$ ����� �������� ������������ ���� ������"=�������� ����� $d$ � ������������� ������������. 
���� ������-������� �� $\R^d$ ����� �������� ���������� �������: $\bm{x}, \bm{x_1}, \bm{y}, \bm{z} \in \R^d$.
������"=�������, ������������ �� ����� ����� ��� �� ����� ������, ����� ���������� $\bm{0}$ � $\bm{1}$ ��������������
(�� ����������� ���� �� ��������� �����, ��� ��� ����� ��������������).
��������� �������, ���������� ������������ ����� � $\R^d$, ���������� $\bm{e_1}$, \dots, $\bm{e_d}$
(����� �������, $\sum_{i\in[d]} \bm{e_i} = \bm{1}$).
������ ������������ ��������~\cite{ZieglerBook},  ������"=������� ����� ����� ���������� �������.


\emph{���������������} �~$\R^d$ ���������� ���������
\[
H(\bm{a},b) \coloneqq \Set*{\bm{x}\in\R^d \given \bm{a}^T \bm{x} = b}, 
%\qquad \bm{a}\in\R^d, \ \bm{a} \ne \bm{0}, \ b\in\R,
\]
��� $\bm{a} \in \R^d$ "--- ������ ������� ��������������, $\bm{a} \ne \bm{0}$, � ����� $b \in \R$ ���������� �������� �������� �������������� ������������ ������ ���������,
$\bm{a}^T \bm{x}$ "--- ��������� ������������ ������"=������ $\bm{a}^T$ �� ������"=������� $\bm{x}$, ���, ������� �������, ��������� ������������ $\langle\bm{a},\bm{x}\rangle$.


�������� ���������� $\sum_{i\in[n]} \lambda_i \bm{x_i}$ ����� $\bm{x_1}$, \dots, $\bm{x_n}$ �� $\R^d$,
��� $\lambda_i \in \R$, $i\in[n]$,
���������� \emph{�������� �����������},
���� $\sum_{i\in[n]} \lambda_i = 1$.
\emph{�������� ���������} $\aff(X)$ ��������� ��������� $X \subseteq \R^d$ ���������� ��������� ���� �������� ���������� ������� ������ ����� �� $X$.
��������� ����� ���������� \emph{������� �����������}, ���� �� ���� ����� ����� ��������� �� ����������� �������� �������� ��������� ��� �����.
\emph{�������� ������������} ��������� $X$ ���������� �������� ������� ������������ ������������ $S \subseteq X$ ����� ����, ��� ������� $\aff(S) = \aff(X)$. � ���������, �������� ����������� ������� ��������� ����� $-1$.
\emph{�����������} $\dim(X)$ ��������� $X$ �������� ������ ��� �������� �����������. (��� ��� ����� ��������������� ������ �������� ��������� �, � ���������, �������� �������� �������� ��������, �� ������������ � ������� ������������� ������������� ����������� �� ���������.)
��������� $X \in \R^d$ ���������� \emph{�������� ����������������}, ���� ������ � ������ ������ ����� ���������� ������� ��� �������� ��� �� �������� ����������.
�������������� �������� �������� ��������� ���������������.
����� ����, ������ �������� ��������������� ����������� $d-k$ � $\R^d$ ����� ���� ������������ ��� ����������� $k$ ���������������~\cite{Emelichev:1981}.

���������� $\sum_{i\in[n]} \lambda_i \bm{x_i}$ ����� $\bm{x_1}$, \dots, $\bm{x_n}$ �� $\R^d$,
��� $\lambda_i \ge 0$, $i\in[n]$,
���������� \emph{���������� �����������}.
\emph{���������� ���������} ��������� ��������� $X = \{\bm{x_1}, \dots, \bm{x_n}\} \subset \R^d$ ���������� ��������� ���� ���������� ���������� ��� �����:
\[
\cone(X) \coloneqq \Set*{\sum_{i=1}^n \lambda_i \bm{x_i} \given \lambda_i \ge 0}.
\]
�������� ����� ������� ������������� ������
\[
\R^d_+ \coloneqq \Set*{\bm{x} \in \R^d \given \bm{x} \ge \bm{0}} = \cone\{\bm{e_1},\dots,\bm{e_d}\}.
\]

�������� ���������� $\sum_{i\in[n]} \lambda_i \bm{x_i}$
����� $\bm{x_1}$, \dots, $\bm{x_n}$ �� $\R^d$ ���������� \emph{��������}, ���� $\lambda_i \ge 0$, $i\in[n]$.
��������� $X \subseteq \R^d$ ���������� \emph{��������}, ���� ��� ����� ���� ����� $\bm{x}, \bm{y} \in X$ 
��� �������� ��� �� �������� ����������.
%��� ��������� �������� ����������� �� ������� \([\bm{x}, \bm{y}] = \{\lambda \bm{x} + (1 - \lambda) \bm{y} \mid \lambda \in [0, 1]\}\).
������� �������� ��������� ��������� ����� ������� \emph{��������� ����������������} 
\[
H^+(\bm{a},b) \coloneqq \Set*{\bm{x}\in\R^d \given \bm{a}^T \bm{x} \ge b},
\]
������������ ��������������� $H(\bm{a},b)$.
����� ��������� ��������� ���������� \emph{�������}, ���� ��� �� �������� �������� ����������� ������� ���� ������ ����� ����� ���������.
��������� ���� ������� ����� ��������� $X$ ������������ $\ext(X)$.
\emph{�������� ���������} ��������� ��������� $X = \{\bm{x_1}, \dots, \bm{x_n}\}
\subset \R^d$ ���������� ��������� ���� �������� ���������� ��� �����:
\[
\conv(X) \coloneqq \Set*{ \sum_{i=1}^n \lambda_i \bm{x_i} \given \sum_{i=1}^n \lambda_i = 1, \ \lambda_i \ge 0, \ \lambda_i \in \R}.
\]
�������� �������� (�������������) ��������� $X \subseteq \R^d$ ������������ ����� ����������� �������� �������� ���� �������� ������� ����� �� $X$. 
%�������� ������� �����������~\cite{Caratheodory:1911}, ��� ������� $X \subseteq \R^d$,
%\[
%\conv(X) = \left\{\sum_{i=1}^n \lambda_i \bm{x_i} \;\bigg|\; 
%\{\bm{x_1}, \dots, \bm{x_n}\} \subseteq X, \ n \le d+1, \  \sum_{i=1}^n \lambda_i = 1, \ \lambda_i \ge 0\right\}.\]

� �������� �������� �������� ����� ������� ������� \emph{����� �����������} ���� �������� �������� $X\subseteq \R^d$ � $Y\subseteq \R^d$:
\[
X+Y \coloneqq \Set*{x+y \given x \in X, \ y\in Y}.
\]
� ���������, ��� ����� $X$ � $Y$
\[
\conv(X+Y) = \conv(X) + \conv(Y).
\]

\emph{�������� ��������������} ���������� �������� �������� ��������� ��������� �����. %� ��������� ������������ $\R^d$.
��� ��� ����� ���� ������ ������ � �������� ��������������, ����� �������� ����� ����������.
\emph{���������} ���������� ����������� ��������� ����� ��������� ���������������, ���, ������� �������, ��������� ������� ������� �������� ����������
\(A\bm{x} \ge \bm{b}\), ��� $A \in \R^{m\times d}$, $\bm{x}\in \R^d$, $\bm{b}\in \R^m$.

\begin{theorem}[�����--����������]
	\sloppy
	��������� $P$ �������� �������������� ����� � ������ �����, ����� $P$ "--- ������������ �������.
\end{theorem}

����� �������, ������ ������������ ����� ���� ������ ����� ��������������� ���������:
\begin{enumerate}
	\item ��� �������� �������� ��������� ��������� ����� $X$. ����� ��������� $X$ ���������� \emph{$V$-���������} �������������.
	\item ��� ����������� ��������� ����� ��������� ���������������. ����� ��������������� ������� �������� ���������� (�, ��������, ���������) ���������� ��� \emph{$H$-���������.}
\end{enumerate}
�� �� ����� ����� � � ��������� �������� ��� ��������� ���������~\cite{ZieglerBook}. \emph{$V$-��������� ��������} $P$ ���������� ������������ ��������� ��������� �����~$X$ �~��������� ��������� �������� $Y$ �����, ���
\[
P = \conv(X) + \cone(Y).
\]

$H$-�������� ����� ���������� \emph{��������}, \emph{��������} ��� \emph{�������} ���������~\cite{Schrijver:1998, Zolotykh:2012}.
� ���� �������, $V$-�������� �������� \emph{���������} ��� \emph{����������} ���������.
������ ���������� $H$-�������� ������������� (��������) �� ��� $V$-�������� ���������� \emph{������� ���������� �������� ��������}.
��� ������������ (�����������) ������ ���������� $V$-�������� �� $H$-�������� � ������� ��� ��� ������������ ����� ��������� \emph{������ ���������� ������������� �������� �������������}.
��� ������ �������� �������������� �������~\cite{Khachiyan:2008}.
������������� ����� ��������� � ��������� ����� �������� � ������� ����� ����� � �����������~\cite{BastrakovDiss:2016}.

%\emph{������������} $\dim(P)$ ������������� $P$ ���������� ����������� ������������ ����������� ��� ��������� ���������������.
������ ���������~\cite{Grunbaum:2003, Emelichev:1981, ZieglerBook}, 
�������� ������������ ����������� $d$ ����� �������� \emph{$d$"~��������������.}
���������� �������� $d$-������������� �������� \emph{$d$-��������}, �������������� ����� �������� �������� $d+1$ ������� ����������� ����� � $\R^n$, $n\ge d$.

����� ��������, ��� ����������� $\bm{a}^T \bm{x} \ge b$ \emph{���������} ��� ��������� $X \subseteq \R^d$, 
���� ��� ��������� ��� ���� $\bm{x}\in X$.
\emph{������} ������������� $P$ ���������� ����� ��������� ���� 
\[
F = \{\bm{x} \in P \mid \bm{a}^T \bm{x} = b\},
\]
��� ����������� $\bm{a}^T \bm{x} \ge b$ ��������� ��� $P$.
�� ����, ��� $\bm{a}$ ����� ���� ������~$\bm{0}$, �������, ��� ������ ��������� � ��� ������������ $P$ �������� �������~$P$, ��� ���������� \emph{��������������} ������� $P$.
��������� ����� ���������� \emph{������������}~\cite{Emelichev:1981}.

���� �������������� $H(\bm{a},b)$ ����� ���� �� ���� ����� ����� � �������������� $P \subseteq \R^d$ 
� ��� ���� $P$ ������� ����� � ���������������� $H^+(\bm{a},b)$, �� �������������� $H(\bm{a},b)$ � ���������������� $H^+(\bm{a},b)$ ���������� \emph{��������} �~$P$.
����� �������, ������ ����������� ����� ������������� ���� ����������� ������������� � ��������� ��� ������� ���������������.

\emph{������������} $\dim(F)$ ����� $F$ ���������� ����������� ������������ ����������� � ��������� ���������������.
����� ����������� $k$ ���������� \emph{$k$-�������}, 0-�����~--- \emph{���������} �������������, 1-�����~--- ��� \emph{�������}.
�������� ������������, ��� ��������� ������� ����� ������������� ��������� � ���������� ��� ������.
$(d-1)$-����� $d$-������������� ���������� \emph{������������}. 
��� $(d-2)$-������ � ������������� ����������� ��� ����������� �������, 
�� �������� ������������~\cite{ZieglerBook}, ��������~\cite{Bastrakov:2011}, �������~\cite{Deza:2001} (�� ����. ridge).
�� ����� �������������� �������� \emph{����}.
��������� ��������\label{def:PolyVertex}, ������ ������, ���������� ��� ����������� �������� �����, � ������������ "--- ������������ ����������� �����~\cite[�.~79]{ZieglerBook}.
��� �� �����, ��� ������� ���������, ��������������� � ��������� ������, �������� 0-�����, �, �������������, �� ������� � ����� ������������ �����, ��� � ��� �������������.

� ���������, ����� ������ � ����������� $d$-��������� ����� $d+1$, � ����� ��� $k$-������, $k \in [d-2]$, ����� $\binom{d+1}{k+1}$. 
����� �������, ����� ���� ������ $d$-��������� ����� $2^{d+1}$.
%(���� ������������ ���� ��� ����������� � ����������.)

\emph{������} ��� \emph{1-��������} ������������� ���������� ��������� ��� ������ � ����� (������, ��� ������, ���������� ����� �������������).
����� ��������, ������� ����� ���� ������, �� ���������� � ��� ����.
%\label{ridge-graph}
%\emph{����"=������} ������������� ����� �������� ��������� ��� ����������� � ������ (������, ��� �����������, ���������� �����).

��� ������ ������������� ����������� ��������� �����������.

\begin{prop}[\cite{ZieglerBook}]
	����� $P$ "--- ������������, � $V = \ext(P)$ "---  ��������� ��� ������. �����:
	\begin{enumerate}
		\item $P = \conv(V)$. %(������������ �������� �������� ��������� ����� ������).
		\item ���� $F$~--- ��������� ����� ������������� $P$, �� $F$~--- ���� ������������ � $\ext(F) = F \cap V$.
		\item ����� ����������� ������ ������������� $P$~--- ����� ����� $P$.
		\item ����� ����� ������������� ����� �������� ��� ������.
	\end{enumerate}	
\end{prop}

������������ ���������� \emph{��������������}, ���� ��� ��� ���������� �������� �����������.
������������ ���������� \emph{�������}, ���� ������ ��� ������� ����������� ����� $d$ �����������, ��� $d$ "--- ����������� �������������.
C�������� � �������� �������������� �������� ��������� ������� �, ������������, �������������� ��������������. 
�������� �������� ������������� �������� $d$-������ \emph{0/1-��������} (���, ������, \emph{$d$-���})~\cite{ZieglerBook}:
\[
\Cube_d \coloneqq \Set*{\bm{x} \in\R^d \given \bm{0} \le \bm{x} \le \bm{1}} 
= \conv\left\{\{0,1\}^d\right\}.
\]
�������� ��������������� ������������� ����� ������� $d$-������ \emph{������������}, � ������������� ���������� ���������� \emph{���������}~\cite{ZieglerBook}:
\[
\Cross_d \coloneqq \Bigl\{\bm{x} = (x_1,\dots,x_d)\in\R^d \Bigm| \sum_{i \in [d]} |x_i| \le 1\Bigr\} 
= \conv\left\{\bm{e_1},-\bm{e_1},\dots,\bm{e_d},-\bm{e_d}\right\}.
\]

������, ��� ���������� �������� ����������� ��������� $\{0, 1\}$, ���������� \emph{0/1"~��������}.
������������, ��� ������� �������� �������� 0/1"~���������, ���������� \emph{0/1"~��������������}.
������� �������, 0/1"~������������ ������������ �����
�������� �������� ���������� ������������ ������ ���� $\Cube_d$.
������� �����, ��� $\ext\conv(X) = X$ ��� ������ $X \subseteq \{0, 1\}^d$.

����� ��������, ��� ��������� �� $n \ge d+1$ ����� � $\R^d$ ��������� \emph{� ����� ���������}, ���� ������� $d+1$ �� ��� �� ����� � ����� ��������������~\cite{ZieglerBook}.
������� �������� ���������� 
\(A\bm{x} \ge \bm{b}\), ��� $A \in \R^{m\times d}$, $\bm{x}\in \R^d$, $\bm{b}\in \R^m$, $m \ge d+1$, ���������� \emph{�����}, ���� ��� ������ $\bm{x}\in \R^d$ ������������ ���������� � ��������� �� ����� ��� $d$ �� ���� ����������.

������ �������������� � ������� �������������� ����� � ��������� ������~\cite{ZieglerBook}.
�������� �������� ��������� �����, ����������� � ����� ���������, �������� �������������� ��������������.
����������, ������� �������� ���������� ������ ����, 
��������� ������� ������� ����������, ���������� ������� ������������.
��������������, ����� ������������ ����� ���� ������������ � �������������� �� ���� ���������� ��������� (�����������) ��� ������.
����������, ����� ������������ ������� ����� ���� ������������ � ������� ������������ �� ���� ���������� ��������� (�����������) ������������� ������� ����������� ��� �������� ����������.

������������ ���������� \emph{$k$-�����������} ($k\in \N$), ���� ����� $k$ ��� ������ �������� ���������� ������ ��������� ����������� ����� ����� �������������.
� ���������, ����� ������������ �������� 1-�����������, � $d$"~�������� �������� $k$-����������� ��� $k \in [d]$.

�������� ���������� ��������� $k$-����������� ��������������, ������������ �� ���������, �������� ����������� �������������~\cite{Grunbaum:2003,Emelichev:1981,ZieglerBook}.
��������� ������ \emph{������������ �������������} ������������ ��������� �������:
\label{page:cyclic}
\[
\CP_d(T) \coloneqq \Set*{(t, t^2, \dots, t^d) \in \R^d \given t \in T},
\]
��� ��������� $T \subset \R$ "--- �������.
(�������, ��� $\CP_d(T)$ �������� ���������� ��� $|T| \le d+1$.)
��������, ��� ��� ������������� �������������, $\lfloor d/2\rfloor$"=����������
(�� ���� ����� ������������ ������� ��������� ����� $d$"~��������������, �� ���������� �����������)
� �������� ���������� ������ ������ (������ �����������) ����� ���� $d$"~�������������� � ��� �� ������ ������ $n = |T|$.

���� �� ����� ������������ ��� $\CP_d(T)$ ������� ���� ����������� � ��� �������, ����� ��������� $T$ ����� ����������� ���.
��������, ���� $T$ "--- ��������� ����� ����� ������� $[a,b]$, 
�� $\CP_d(T)$ ����� ���������� ����� ������������� $\CP_d(a, b)$.

\begin{comment}
����� �� ����� ������� � ������ � ��� ������ �������� �������� ��� ��������������� �������� �������� ����������� ��������.
����� $P$ "--- ��������� $d$-������������, ��������� � ������������ $\R^n$, $n > d$,
�~����� ����� $\bm{x} \in \R^n$ ��~����������� �������� �������� ����� �������������.
\emph{���������} ��� $P$ ���������� �������� �������� $\conv\{P\cup \bm{x}\}$.
������������ $P$ ���������� \emph{����������} �������� $\conv\{P\cup \bm{x}\}$, � ����� $\bm{x}$ "--- �� \emph{�������} (��� \emph{�������� ��������}).
������� �������� �������� ��� ����� �� ���������, � ����� ��� �������� ��� ��� �������.
� ���������, ����� ����������� �������� ����� �� ������� ������ ����� ����������� ���������, �� �� ����� � ��� ����� ������.
�������� ��� �������������� �������������� ����� �������� �������������� ��������������.
�������� ��� $k$-����������� �������������� ���� $k$-����������.
\end{comment}


\subsection{������� ������}

\emph{�������� ������} ������������� $P$ ���������� ��������� $L(P)$ ���� ��� ������, 
�������� ������������� �� ���������.

\begin{comment}
� �������� ������� ���������� 4-������ �������� ������������ $P$, ���������� ���������, ��������� ������� "--- 
���������� ��� ��� ����� ������� (��. ���.~\ref{fig:cube7}).
$P$ ����� 8 ������, 8 ����������� (���� �� ��� ���������� �� ���.~\ref{fig:cube7}), 19 ����� � 19 ������.
\begin{figure}[hb]%
	\centering
	%\includegraphics[width=\columnwidth]{filename}%
	\begin{tikzpicture}[scale=2.0, line join = round]
	\coordinate (4) at (0,0,1);
	\coordinate (6) at (1,0,0);
	\coordinate (7) at (0,0,0);
	\coordinate (2) at (1,0,1);
	\coordinate (1) at (0,1,1);
	\coordinate (3) at (1,1,0);
	\coordinate (5) at (0,1,0);
	\draw (6) -- (3) -- (2) -- (1) -- (4) (5) -- (3) -- (1) -- (5) (6) -- (2) -- (4);
	\draw[dashed, thin] (6) -- (7) -- (5) (7) -- (4);
	\foreach \i in {1,...,7} {\draw (\i) node[circle, draw, fill = white, inner sep = 1pt] {\i};}
	\end{tikzpicture}
	\caption{���������� ��� ��� ����� �������}%
	\label{fig:cube7}%
\end{figure}

������� ������ ������ ��������������� ����������� ��������� �����, 
�������������� ����� ������������ �� ��������� ����,
������� �������� ������������� ����� �������������.
������� �� ��������� ����� ��������� �� � ������ �� ���� ������ $f$ � $g$,
��� ������� ������������ ����������� ��������� �������:
\begin{compactenum}
	\item[1)] $f$ �������� ������ $g$ (� ���� ������ $g$ ����������� �� ��������� ���� $f$);
	\item[2)] �� ���������� ����� $h$, ������������ �� $f$ � $g$, � �����, ��� $f$~--- ����� $h$ � $h$~--- ����� $g$.
\end{compactenum}
������� ������ �������� ��� ���������� ����� ��� ������� ���������� �� ���.~\ref{fig:cube7Hasse}.

\begin{figure}%
	\centering
	%\includegraphics[width=\columnwidth]{filename}%
	\begin{tikzpicture}[x=4mm,y=13mm,new set=import nodes, >=stealth']
	\begin{scope}[nodes={set=import nodes}] % make all nodes part of this set
	%\node (p) at (0,4) {$p_1$};
	\node[circle, draw, inner sep = 0pt, minimum size = 12pt] (polytope) at (0,4) {};
	\node[circle, draw, inner sep = 0pt, minimum size = 12pt] (emptyset) at (0,-1) {};
	\foreach \i in {1,...,8} {
		\node[circle, draw, inner sep = 0pt, minimum size = 14pt] (f\i) at ({(\i-4.5)*2.0},3) {$f_{\i}$};
	}	
	\foreach \i in {1,...,19} {
		\node[circle, draw, inner sep = 2pt] (r\i) at ({\i - 10},2) {};
	}	
	\foreach \i in {1,...,19} {
		\node[circle, draw, inner sep = 2pt] (e\i) at ({\i - 10},1) {};
	}	
	\foreach \i in {1,...,8} {
		\node[circle, draw, inner sep = 0pt, minimum size = 14pt] (v\i) at ({(\i-4.5)*2.0},0) {$v_{\i}$};
	}	
	\end{scope}
	\draw (15, 4) node[left] {���� ������������};
	\draw (15, 3) node[left] {����������};
	\draw (15, 2) node[left] {�����};
	\draw (15, 1) node[left] {�����};
	\draw (15, 0) node[left] {�������};
	\draw (15, -1) node[left] {������ ���������};
	\graph {
		(import nodes); % "import" the nodes
		polytope -- {f1, f2, f3, f4, f5, f6, f7, f8};
		emptyset -- {v1, v2, v3, v4, v5, v6, v7, v8};
		e1 -- {v1, v2}; e2 -- {v1, v3}; e3 -- {v2, v3}; e4 -- {v2, v4}; e5 -- {v1, v4};
		e6 -- {v1, v5}; e7 -- {v3, v5}; e8 -- {v3, v6}; e9 -- {v2, v6}; e10 -- {v4, v7};
		e11 -- {v5, v7}; e12 -- {v6, v7};
		\foreach \i/\j in {1/13,2/14,3/15,4/16,5/17,6/18,7/19} {e\j -- {v\i, v8};};
		%\foreach \i in {1,...,14} {r\i -- e\i;}
		r1 -- {e1, e2, e3}; r2 -- {e1, e4, e5}; r3 -- {e2, e6, e7}; r4 -- {e3, e8, e9};
		r5 -- {e4, e9, e10, e12}; r6 -- {e5, e6, e10, e11}; r7 -- {e7, e8, e11, e12};
		r8 -- {e1, e13, e14}; r9 -- {e2, e14, e15}; r10 -- {e3, e13, e15}; r11 -- {e4, e13, e16}; r12 -- {e5, e14, e16};
		r13 -- {e6, e14, e17}; r14 -- {e7, e15, e17}; r15 -- {e8, e15, e18}; r16 -- {e9, e13, e18}; r17 -- {e10, e16, e19};
		r18 -- {e11, e17, e19}; r19 -- {e12, e18, e19};
		f1 -- {r1, r2, r3, r4, r5, r6, r7};
		f2 -- {r1, r8, r9, r10}; f3 -- {r2, r8, r11, r12}; f4 -- {r3, r9, r13, r14}; f5 -- {r4, r10, r15, r16};
		f7 -- {r5, r11, r16, r17, r19}; f6 -- {r6, r12, r13, r17, r18}; f8 -- {r7, r14, r15, r18, r19};
	};
	\end{tikzpicture}
	\caption{��������� ����� ������� ������ �������� ��� ���������� ����� ��� �������}%
	\label{fig:cube7Hasse}%
\end{figure}
\end{comment}

��� ������������� ���������� \emph{������������ ��������������}, ���� �� ������� ������ ���������.
���� ��� ������������� ������������ ������������, �� �������, ��� ��� �������� ��������������� ������ \emph{�������������� ����}.
�������� � �������� �������������� �������������, ���������� ������������ ��� �������� ������, ���������� \emph{��������������}.
%, ����� ���� �������� �� �� ������ ������� � ������������� �������������� ��������� (��. ���������~\ref{rem:combinatorial} ����).
� ���������, ����������� �������������, ����� ��� ������ � ����� ����������� �������� �������������� ����������������, � ����������������, �������� � $k$"=������������� "--- �������������� ����������.

%\begin{remark}
%\label{rem:combinatorial}
%���������� ��������� ������ ����� � ������� ������������� �������������, � ������ ������� �������������� ���������� �� ������ ��������� ���� �������� � ��������������, �� � ��������� ������, ��� ����������� ������� ������ ������� ������ ������������ (��������, ��������� ���������� (��. ������~\ref{sec:Extension})). �����, � ������������ ������ ��������� �� ����� �������������� ���� ��������, �� � ��� �������, ����� ��� �����, ��������, ���������� ������������ �������� ������, ����� �������� ����� ��������������.
%, � ��������, ��������� �� ��������� �������������, "--- \emph{������������"=���������������}.
%\end{remark}

��� ������������� ���������� \emph{�������������} ���� � �����, ���� �� ������� ������ �������������.
� ���������, ���� $P$ � $Q$ �����������, �� ������� $P$ ������������� ����������� $Q$, ����� $P$ "--- ������ $Q$ �~�.\,�.
�������� ������������ �������������� ����� ������� $d$-��� � $d$-������ �������.
%������������, ������������ �� ���.~\ref{fig:cube7}, ����������� ������ ����.
%��� ����� �������� ������������� ������ ���� ������������� �������� $d$-��������.
������, ������������, ������������ ���������������, �������� �������, 
� ������������, ������������ ��������, "--- ��������������~\cite{Emelichev:1981}.


\begin{comment}
����� $d$-������������ $P$ ����� � ���� $P = \conv(V)$ � $\bm{0}$ �������� ���������� ������ ����� �������������. (���������� ���������� ������� ������ ����� �������� �� ���� �������� �������� $\bm{x} \mapsto \bm{x} + \bm{x_0}$.)
\emph{�������} � $P$ ���������� ������������
\[
P^* \coloneqq \Set*{\bm{x}\in \R^d \given \bm{y}^T \bm{x} \le 1, \ \forall \bm{y} \in V}.
\] 
������ $P^*$ �������� �������� ������������� � $P$ �������������~\cite{Emelichev:1981,ZieglerBook}.
\end{comment}


����� $P$ "--- ��������� ������������, $V = \{v_1, \dots, v_n\}$~--- ��������� ��� ������,
� $F = \{F_1, \dots, F_k\}$~--- ��������� ��� �����������.
����� \emph{������� ���������� ������"=�����������} $M=(m_{ij})\in \{0,1\}^{n\times k}$ 
������������� $P$ ������������ ��������� �������:
\[
m_{ij} = \begin{cases}
1, & \text{���� } v_i \in F_j,\\
0, & \text{�����.}
\end{cases}
\]
������� $M^T$ ���������� \emph{�������� ���������� �����������"=������.}

������� ������ ������������� ���������� ����������������� �� ��� ������� ���������� ������"=�����������. 
(���� �� �������� ����������� ���������� ������� ���� ������ ������ �~\cite{KaibelP:02}.)
\begin{comment}
���, ��������, ��������� ����� �� ���.~\ref{fig:cube7Hasse} ����������������� �� ������� ����������
\begin{equation}
\begin{pmatrix}
1 & 1 & 1 & 1 & 0 & 1 & 0 & 0 \\
1 & 1 & 1 & 0 & 1 & 0 & 1 & 0 \\
1 & 1 & 0 & 1 & 1 & 0 & 0 & 1 \\
1 & 0 & 1 & 0 & 0 & 1 & 1 & 0 \\
1 & 0 & 0 & 1 & 0 & 1 & 0 & 1 \\
1 & 0 & 0 & 0 & 1 & 0 & 1 & 1 \\
1 & 0 & 0 & 0 & 0 & 1 & 1 & 1 \\
0 & 1 & 1 & 1 & 1 & 1 & 1 & 1 \\
\end{pmatrix}
\label{eq:Minc}
\end{equation}
\end{comment}
����� �������, ��� 
%����� 
������������� �������� ������������� ���������� ������������ �� ��� ������� ����������, � ����� ������������ ����� �/��� �������� ���� ������� �� ������ ���� �������.
����� ��������, ��� ������� ���������� ������"=����������� ������������ �������������� ������������� ���� � ����� ��������� ���������������� �, ��������, ������������� ����� �/��� ��������.
%��������, ��������� ������������ �������� (��������������) ��������� � ������� \eqref{eq:Minc} ������� � �������������� ���������������� ������������� ������ ����.


%\subsection{�������� � ����������� �������������� ��������������}
\subsection{�������� �����������}

����������� ���� $\bm{x} \mapsto A \bm{x} + \bm{b}$, 
��� $\bm{x} \in \R^d$, $A \in \R^{m\times d}$, $\bm{b} \in \R^m$, ���������� \emph{��������}. 
������� ������� ��������� �������������� �������� \emph{������������� ��������} $(x_1, x_2, \dots, x_d) \mapsto (x_1, x_2, \dots, x_m, 0, \dots, 0)$, $d > m$.
��� ������������� $P\subseteq \R^d$ � $Q\subseteq \R^m$ ���������� \emph{������� ��������������},
���� ���������� �������"=����������� �������� ����������� $\alpha\from P \to Q$.
�� �������� ��������������� �������������� ������� �� ������������� ���������������.

����� ��� $d$-��������� ������� ������������.
������� ����� ������ ������������� $d$-��������� ����� ��������������� ��� ������������ �������
\[
\Delta_d = \Set*{\bm{x} \in \R^{d+1} \given \bm{1}^T \bm{x} = 1, \ \bm{x} \ge \bm{0}}
= \conv\{\bm{e_1}, \dots, \bm{e_{d+1}}\}.
\]\label{ProjOfSimplex}
�������� ��������, ��� ����� ������������, ������� $n$ ������, �������� �������� ������� ��������� $\Delta_{n-1}$.

\begin{comment}
\emph{����������� ���������������} ���������� ������"=�������� ����������� ����
\[
\tau(\bm{x}) = \frac{\alpha(\bm{x})}{\bm{a}^T \bm{x} + b},
\]
��� $\alpha$ "--- �������� �����������, ����������� �������� $\bm{a}$ � $\bm{x}$ ���������, $b\in \R$.

����������� �������������� �������� ���������� ����������~\cite{ZieglerBook}:
\begin{enumerate}
\item ����� $P$ � $Q$ "--- �������������. ���� ����������� �������������� $\tau \from P \to Q$ �������"=����������,	�� ������������� $P$ � $Q$ ������������ ������������.
\item ����� ������������ $Q$ �������� �������� ������� ��������� ����� ������������� $P$. ����� ���������� ����������� �������������� $\tau \from P \to Q$.
\end{enumerate}
\end{comment}
 


\section{��������� ����� � ����������}
\label{sec:complexity}

��������������, ��� �������� ������ � �������� ������ ��������� ����������~\cite{Arora:2009, Goldreich:2008} �, �~���������, ������ NP"=������ �����~\cite{Garey:1982}.
��� �� �����, ����� �������� ��������������� � ����������, ���������� ��������� �������� ������� � ����������.

��� ��������������� ��������� ���� ������� $f\from \N \to \R_+$ � $g\from \N \to \R_+$ ������������ ����������� �����������:

$f = O(g)$, ���� �������� $c > 0$ � $n_0 \in \N$ �����, ��� $f(n) \le c \cdot g(n)$ $\forall n \ge n_0$.

$f = \Omega(g)$, ���� �������� $c > 0$ � $n_0 \in \N$, ��� $f(n) \ge c \cdot g(n)$ $\forall n \ge n_0$.

$f = \Theta(g)$, ���� $f = O(g)$ � $f = \Omega(g)$.

$f = o(g)$, ���� $\forall c > 0$ �������� $n_c \in \N$, ��� $f(n) < c \cdot g(n)$ $\forall n \ge n_c$.

�������~$f\from \N \to \R_+$ ����� �������� \emph{��������������} � ���������� $f(n) = \poly(n)$,
���� �������� $k \in \N$, ��� $f(n) = O(n^k)$.
%������� $p=p(n)$ �����, ��� $f(n) \le p(n)$ ��� ���� $n \in \N$.
������� $f$ ���������� \emph{�������������������},
���� $f(n) = \Omega(n^k)$ ��� ������ $k \in \N$. 
%��������~$p$.
���, ��������, ������� $f(n) = a^{\ln n}$, ��� $a > 0$, �������� ��������������, �~������� $g(n) = a^{\ln^{1+\varepsilon} n}$ ��� $a > 1$ � $\varepsilon > 0$ "--- �������������������.
������������ ����� ����������� ������ ��������� ��������� �� ��������������� �������� ����� ��������������� � �������������������� ���������.
������, ����, ��� �������, ���� �� ������� ����� ��������������� ��������� �~��������� ���� $f(n) = \Omega\left(a^{n^{\varepsilon}}\right)$,
 ��� $a > 1$, $\varepsilon > 0$.
������� $f$ ���������� \emph{����������������},
���� $f(n) = 2^{\Theta(n)}$.
%���� $f(n) = \Omega(a^n)$ ��� ��������� $a > 1$ �, ������ � ����, $f(n) = O(2^{\poly(n)})$.
%��������� ������ ���������� \emph{�����������������}~\cite{BondBook:2008}.

������� ������ �������������� ������ ������ ������������
����� ����������� ������� ������������� ����� ����� � ����� (��������� ������ ����� �������� ������������ �������).
%�� ����������� ������������, ��� ������� ������ �������������� ������ ���������� � �������� ������������������ ��������� ������������ �������. 
�� ����������� ������������, ��� ������� ������ ���������� � �������� ������������������ $s \in \{0,1\}^* = \bigcup_{m \in \N} \{0,1\}^m$ ��������� �������� �������~\cite[�.~34--38]{Garey:1982}, \cite[p.~2]{Arora:2009}. 
%�� ���� ���� ������ ����������� ��������� .
� ���������, ������ ������������ ����� $n$ �������� $\lceil\log_2 (n+1)\rceil$ ���.
����������: $\size(n) = \lceil\log_2 (n+1)\rceil$.
%(�����������, ����� ����� $I$ ������������ ���: $|I|$. � ������, ���� ������ �������� ������� (����� ��� ������������) �����, ��� ������� ��������. ������� � ��������� ������ ������������ ����� ������������� $\size(I)$.)
\begin{comment}
��������������, $\size(k) = \size(|k|) + 1$ ��� ������ �����~$k$.
������������ ����� $p$ �������������� ����� ������� ������� ����� $k$ (���������) � $n$ (�����������), ��� $k \in \Z$, $n \in \N$.
%, �� ���� $\size(p) = \size(k) + \size(n) + 1$.  
� ������������� ������������� ����� ������� ������ ������ ������ ���������� ������ ����� ���� ��������������� �����.
�� �������� ��, ���� �����, ����� ������� ������ ��������������� ������������ ���������� ����� �� ����� ������ ����������� �� ���.
��� �� �����, ����� ������ �� ���� ��������� ������������� (� ������ ������, �� �����, ��� �����������) ������������ ����� ������.
\end{comment}

\emph{�������\'�� ���������� ���������} ���������� �������, 
������� ������� ������������ $n$ ������ � ������������ ������������ ����� (����� ��������), ������������� ���������� ��� ��������� ������� ������ ����� $n$~\cite[�.~18]{Garey:1982},~\cite[p.~32]{Goldreich:2008}.
����� ����� ��� \emph{���������� ���������} ����� �������� 
��� �������\'�� ���������.
��� (�������\'��) \emph{���������� ������} ����� �������� ��������� (��������������) �������� ������ ��������� ��� ������ ���������.
(� ��������� ������ ��������������� ������ ���������� ������.)
��������� ���������� �, ��������������, ����� ����������� ������� �� ������ ����������. ��������� �����, ����� ����, ����� �������� �� �������������� ����������� (�������������� ���� ��� ����� ����������������), ������������� �� ����� ��������������� ����������.

� ����� �������, �������� ������ �������--��������~\cite[c.~33]{Goldreich:2008}
(����� ����������� ������� ׸���--�������� � ������� �����~\cite[c.~26]{Arora:2009}),
����� ��������� ����������� �������������� ������ 
����� ���� ��������������� (�������������) ������� ��������
� (�� ����� ���) �������������� ����������� ������� ������.

� ������ �������, ������ ������ ������� (��������, ����������� ������� P � NP) ����������� ����������� ��������, ���� ������ �� � ������ �������������� ������ <<������ ��������>>.
%~\cite[c.~259]{Arora:2009}. 
������� ����� ����� �������� �������� � �� ������, ����� ������� ��� �������� ������. (����� ����, ������ <<��������� �������������>> �������������� ������ ��������� ������������ ������� � ����������� �������� ��� ��������� ���������� ���� ��������.) ���, ��������, � �������������� ������ <<������ �������>> ��������� ������ ���������� ������� ������ � ������ ���������� ������ ������ �����~\cite[c.~260]{Arora:2009}. ����������� ������� ������� ���� � ��� �������������� ������ <<�������� ����������� ���������>> (��. ������~\ref{sec:DirectTheory}). � ������ ���� ������ ��������� ����� �������� ������ ������������� ����������� ���������� ������ ��������� �� ����� �����~\cite{Moshkov:1982}. ��������������� � ��������� ������ �������������� ��������� ����� � �������� ����� ������� � ��������������� ��������, �������� �������������� �������������. ����� ���������� ���������� ������������ ���� ������� ��������� �����~\ref{chap:Polytopes}.

\begin{comment}
� ������ �������, ��� ������������ � ��������� ����� �������������� ���������� ������������ � ������ ������ �������--��������~\cite[c.~33]{Goldreich:2008},
������� ����� �������� ������� ׸���--�������� � ������� �����~\cite[c.~26]{Arora:2009}.

\textbf{����� �������--��������.}
\emph{����� ��������� ����������� �������������� ������ 
����� ���� ��������������� (�������������) ������� ��������
� (�� ����� ���) �������������� ����������� ������� ������.}

� ���������, ���� ����� ����������, ��� �������������� ������ � ������ $P$ (������������� ���������� �����) �� ������� �� ������ ��������� ����������� �������������� ������.

�������������� ���� �������, ���, ������������, � ������� ��������� ���������� ����� ������ ������, �������������� ������� � ������ $P$ � ��������� ����� �� ��������.
��� �� �����, ����������� ���������� ���������� ��������� ����������� (������, ����������� �� ���������������) ���� �������� ��� �������� (����������� ����� ����������� ������ ������� �~\cite{Aaronson:2008}).
\end{comment}

�������������� ������ ����� ������ ������������ � ���� �������, ����������� ������� �������� ������� ������, � ���������~--- ������� ������.
������� ����� �������� \emph{������������� ����������}, ���� ��������������� ������ ����� �������������� ��������� �� ������ ��������.
%������� (������), �� ���������� ������������� ����������, ���������� \emph{��������������}~\cite[�.~21]{Garey:1982}.

\emph{����������} ���������� ������� � ���������� �������� $\{\text{����},\text{������}\}$ (��� $\{0,1\}$).
����� �������� $g$, �������� �� ��������� $\{0,1\}^*$, ���������� ��������� ����
$L = \Set{x \in \{0,1\}^* \given g(x)}$, �, ��������, ����� ���� $L \subseteq \{0,1\}^*$ ��������� ������� ���������� ��������� ��������, ��������������� ������ �������� ������� $x \in L$.
����� �������, ��� ������� � �����, ���������������� ��� ������, ������������� ����� � ��� ��������������� ����������.
� ���������, ����� $L \in P$ ($g \in P$), ���� ������ �������� ������� $x \in L$ (���������� ��������� $g$) ������������� ���������.

\begin{definition}[������ NP � co-NP~{\cite{Arora:2009}}]
	\label{def:NP}
	���� $L \subseteq\{0,1\}^*$ ����������� ������ NP, ���� �������� ������� $p\from\N\to\N$ � ������������� ���������� �������� $g\from \{0,1\}^* \times \{0,1\}^* \to \{\text{����},\text{������}\}$
	�����, ��� ��� ������� $x \in \{0,1\}^*$
	\[
	x\in L \iff \text{�������� } u\in \{0,1\}^{p(\size(x))} 
	\text{ �����, ��� } g(x,u).
	\]

���� $L$ ����������� ������ co-NP, ���� ���� $\{0,1\}^* \setminus L$ ����������� NP.

�������, ��� ������ �������� ������� $x \in L$ ����������� ������ NP (co"~NP),
���� ���� $L$ ����������� ����� ������.
\end{definition}

������������ ������� �������������� ���������� ����� ������������ �� ������� � ����������� �� ���� �������� �����~\cite{Garey:1982, Goldreich:2008}.
����� �������� �������� (� ����� ������ �������� ����������� �����������) ������������, ����� �����, ���� �� ��������� ����, ��� �������������� ����������� �� ����� �������� �������������� ���������� �� �������� (������ ��� ���������� ����������� �� ����).
�������� �������� �������� ����������� ����� ���� ����������, ��������������, ��� �������� ������ � �������� ���������� ������ ��������~\cite{Garey:1982, Goldreich:2008}:

\begin{definition}[�������������� ����������]
	������ $\Pi$ \emph{������������� ��������} � ������ $\Pi'$,
	���� ���������� ���������� ������ $M$ �����, ��� ��� ����� ������� $f$, �������� ������ $\Pi'$, ������ $M$ � �������� $f$ ������ ������ $\Pi$ �� �������������� �����. (��������� � ������� $f$ ����������� �� ������� �������.)
\end{definition}

������ �� ������ NP (co-NP) ���������� \emph{NP-������} (\emph{co-NP-������}),
���� ����� ������ ������ �� ����� ������ ������������� �������� � ���.

%�� �� �������� ����� ����������� ������� NP � co-NP, � ����� ������� NP-������ (� co-NP-������) ������~\cite{Garey:1982}, ��� ��� ��� ������� �������� ����������� � ���������� ����� ������������� ��������� (� �������, ��������, �� ������� NP-������� ������).

� �������� ������� �������� ������������ �������� ��������� NP-������ ������ � ������������ ������� �������~\cite[�.~56]{Garey:1982}.

����� $U = \{u_1, u_2, \ldots, u_k\}$ "---
��������� ������� ����������. 
������ ������ ���������� ����� ��������� ���� ���� �� ���� ��������: 
$1$ ��� $0$.
%$1 \overset{\text{def}}{=} \text{<<������>>}$ ��� $0 \overset{\text{def}}{=} \text{<<����>>}$.
���� $u\in U$, �� $u$ � $\bar{u}$, $\bar{u} = 1 - u$, ���������� \emph{����������}.
��������� ���������, �������� $\{u_2, \bar{u}_4, u_5\}$,
���������� \emph{�����������} ��� $U$ � ������ ������������ $u_2 \vee \bar{u}_4 \vee u_5$.
�������, ��� ���������� \emph{���������} (��������� �������� 1) ��� ��������� ������ �������� ����������, ���� ���� �� ���� �� �������� � ��� ���������
����� 1.
����� $C = \{D_1, D_2, \ldots, D_m\}$ "--- ��������� ����� ����������, ����� ���������� \emph{�����������}.
� ����� ������� �������, ��� ������ ������� $C$ ������ � \emph{������������� ���������� �����}\label{def:CNF} (���).
���������� $C$ ���������� \emph{����������}, ���� ���������� ����� �������� ���������� �����, ��� ������������ ����������� ��� ���������� �� $C$.
\medskip

\problem{������ � ������������.}
���� ��������� ���������� $U$ � ����������~$C$ ��� $U$.
����� ��, ��� $C$ ���������?
\medskip

%� ������ ������ NP-����� ������� ���������� ��������� ���� ������� ������ (� ������ ������ ����������) ��� ������� ����� �� ������ ������ �����������.
%��� ��������� ������ ���������� ������, � ������� ������ �������� ������ �������������� ������ "--- ������~\cite{Garey:1982}.
����������� ������ � ������������ ��������� ������ NP-������ ������� (���� ��� ������ ������ �������)~\cite{Cook:1971}.
\emph{������ � $k$-������������} ������������ ����� ������� ������ ������ � ������������, ����� ������ ���������� �������� ����� $k$ ���������.
��� ��� ��� ������ ������������ $k \ge 3$ ������ � ������������ ������������� ��������~\cite{Karp:1972} � ������ � $k$-������������, �� ��������� ����� �������� NP-������.

\emph{NP-��������} �������� ����� �������� ��, � ������� ������������� �������� NP-������ ������.
�������, ��� ��������� ������������� ��� ����������� NP-������� ������ ������������ ������������ ���������� �� �����. 
� ���� ������, � ���������, ������ NP-������� � co-NP-������� ����� ���������� ���� �� ����� ��� ������� $\NP \ne \coNP$.
�� �� ����� ����� ������������ ������������� �������������� ���������� �� ��������. 
��� ����� ����������� co-NP-������ ������ �������� NP-��������.

\begin{comment}
����� ��� ����� ����������� ����������� ������ $D^p$~\cite{PapadimitriouY:1984}:
\[
D^p = \{L_1 \cap L_2 \mid L_1\in \NP, \ L_2\in \coNP\}.
\]
� ���������, NP � co-NP �������� �������������� $D^p$, ������ �������, ��� ��� $D^p$-������ ������ �������� NP-��������.
� �������� ������� �������� ��� $D^p$-������ ������~\cite{PapadimitriouY:1984}:
\begin{enumerate}
	\item \problem{������������--��������������.} ���� ��� ������ �������. ����� ��, ��� ������ ���������, � ������ "--- �����������.
	\item \problem{������ �����.} ��� ���� $G$ � ����� $k \in \N$. ����� ��, ��� $k$ �������� �������� ������ ����� �����.
\end{enumerate}
\end{comment}



\section{������ ������������� �����������}
\label{sec:CO}

� ����������� A. Schrijver ``Combinatorial optimization'' (2003) ���������� ��������� ����������� ������������� �����������: <<������������� ����������� ���� ����������� ������ � �������� ��������� ��������>>~\cite[p.~1]{SchrijverCO:2003}. ������� ���� ����� ������ �����������, ����������� ����������� ���������� ����� �, ������ � ���, ����������� ������ ������������� ������ � ��������� �������� �����, �������� � ����������� ������ NPO (NP optimization)~\cite{Ausiello:2011, Momke:2009}.

%http://www.nada.kth.se/~viggo/problemlist/
\begin{definition} %[������ ������������� �����������]
\label{def:COP}
\emph{������ �� ������ NPO} ������������ ����� ��������:
\begin{enumerate}
	\item \emph{���� ������� ������} $L$, $L \in P$. ������ ����� $I \in L$ ���������� \emph{�������� �������} ������.
	\item \emph{�������� ������������} $g = g(x, I) \in \{\text{����}, \text{������}\}$, ������������ \emph{��������� ���������� �������} 
	\[X = X(I) = \Set*{x \in \{0,1\}^* \given \size(x) = \poly(\size(I)) \text{ � } g(x, I)}.\]
	\item \emph{������� �������} $f = f(x, I) \in \Q$, ��� $x \in X$.
	\item \emph{����������� �����������:} $\min$ ��� $\max$.  �� ���������, $\max$.
\end{enumerate}
�������� $g$ � ������� ������� $f$ ������������� ���������.
���� ������� ������ $I$ �� �������������, �� ������ ���������� \emph{��������}, ����� "--- \emph{��������������} (��� \emph{�����������} ������).
���� ������ "--- ����� ���� ���������� ������� $X$ ����� �����, �� ������� ������� ������� $f$ ��������� ����������� �������� ($\min$ ��� $\max$ � ������������ � ������������ �����������).
\end{definition}

�������� �������� ������������ ��������� ������ �������� ������ ������������� ����������� � �������� ������� ��������. 
%� [\cite{Junger:1995,Onn:2004}] �������� ������� ������������� ����������� ���������� ������ ����������� �� ������������� ��������� E

\begin{definition}
\label{def:LCOP}
\emph{�������� ������ ������������� �����������} ������������ ����� ��������:
\begin{enumerate}
	\item ���� ������� ������ $L$, $L \in P$. ����� ����� $I \in L$ ���������� \emph{����� ������}.
	\item \emph{�����������} $d \from L \to \N$.
	\item �������� ������������ $g \from \Z^d \times L \to \{\text{����}, \text{������}\}$, ������������ ��������� ���������� ������� 
	\[X = X(I) = \Set*{\bm{x} \in \Z^d \given \size(\bm{x}) = \poly(\size(I) + d(I)) \text{ � } g(\bm{x}, I)}.\]
	\item ����������� �����������: $\min$ ��� $\max$. �� ���������, $\max$.
\end{enumerate}
��� � ������, ��������������, ��� ������� $d$ � $g$ ������������� ���������.
�������� ������� �������������� ������ �������� � ��� $I$ � \emph{������� ������} $\bm{c} \in \Q^d$.
������� ������� �������: $f(\bm{x}, \bm{c}) = \bm{c}^T \bm{x}$, ��� $\bm{x} \in X$.
\end{definition}

\begin{comment}
\begin{remark}
	\label{rem:CombOpt4}
	%������� $d = d(I)$, $S = S(I)$ � $g = g(u,I)$ ���������� ��������� ���������� ������� ��������������� ������. �������� ������� ������ ����� ���� ������� � ��� ������� ����� �������, ����� ��� ������������� ����� ��������� ������� ���� �� ������. 
	%�� ���� ������� 
	� ����������, ��� �������� ����� �� �� ������������� ���� ������� ������, �����������, ��� �� ���� ������ �������� ������ ���������� ������.
	��������, ��� ����������� ����������� �� ��������� "--- ��������, �������� ������ ������������� ����������� ����� ������������ �� ���������, �~����� �������: ������������ $d = d(I)$ � ���������� ������������ $g = g(x, I)$.
\end{remark}
\end{comment}

\begin{remark}
������ ����������� ������ ����� ���� ������������� � ������ ������������ (� ��������) � ������� ������ $\bm{c} \coloneqq -\bm{c}$.
�������� ��� ��������������, ����� �� � �������� ����� ������������� ������ �� ��������, � �������� ������� ������������� ����������� �������� ������ $(L,d,g)$.
\end{remark}

\begin{remark}
��������� �������� ������� $\bm{c}$ �� ������������� ������ ��~������ ������������� �������. 
����� ����, �� ���� ��������� $\bm{c}$ �� ���������� ����� ������� ������������ ��� ���������, ��� ����� ������� ������������� �� �������������� ������������ ������� $\size(\bm{c})$ �����.
�������, �� �������� ��������, ����� ������������, ��� ������ $\bm{c}$ "--- �������������.
\end{remark}


� �������� ������� ���������� ������������ ������ � �������. 
���� $n$ ��������� � ��� ������� �� ��� �������� ������ $a_i \in \Z$ � ��������� $c_i \in \Z$. ����� ����, �������� ������ ������� $b \in \Z$.
��������� ������� ������������ ��������� ���, ����� �� ��������� ������ ��� ������ ������� �������, � �� ��������� ��������� ���� �� ������������.

����� ����������� ��� ������ � ���� �������� ������ ������������� �����������, ������� ������������ $s \subseteq [n]$ �������� � ������������ ��� ������������������ ������ $\bm{x} = \chi(s) \in \{0,1\}^n$. ��� ����� �������������, � ������ ��������� ����� $\bm{x} \in \{0,1\}^n$, ������������ ������������ �������� ������� $\bm{c}^T \bm{x}$, ��� $\bm{c} = (c_i) \in \Z^n$ "--- ������ ���������� ���������, ��� ������� $\bm{a}^T \bm{x} \le b$, ��� $\bm{a} = (a_i) \in \Z^n$ "--- ������ �������� ���������.
� ������������ �����������~\ref{def:LCOP}, ��� ������ ������� �� ������ $(n, \bm{a}, b)$, ����������� $d = n$, � �������� $g$ ��������� �������������� ������� $\bm{a}^T \bm{x} \le b$ � $\bm{x} \in \{0,1\}^n$. ����� ����� ������������, ��� ����������� $n$ ������� $\bm{a}$ ���������� � ��� ��������. ��� ��������� ��������� �� ���� ������ ���������� �������� $n$.


%,Onn:2004
����� �� ����� ������������� \emph{������} �������� ������ ������������� �����������, ������� ��, ��� ���������, �������� ������������� �����������.
� ����� �������, ������ � ����� ������������ �������� ����� ����������� �� ��������~\cite{Junger:1995}.
%: ��������������� ������ �� ������ (���������� ����, ����������� � �.�.), ������ � �������, ������ �������������� ��������� ���������������� � ������ ������.
� ������ �������, ������ ������� ����� ������ ���� � ��������� ����� ������� ��������� ������, ��� ���� ������ � ���������� ������� �������� �� ������ ������� �������� ����� ����������, ���� �������� �� � ��������� ����.
� �������� ������� �������, ��� ����������� ��� �������������� ��� ���������, �������� ���������� �����.

\problem{������ ������ ������������� ����������������.}
��� ����� (������������) ������������� ���������� 
\[
p(\bm{x}) = \sum\limits_{\mathclap{1\le i \le j \le n}} c_{ij} x_i x_j, 
\quad \bm{x} = (x_1, \dots, x_n) \in \{0, 1\}^n. 
\]
��������� ����� $\bm{x}$, ��� ������� $p(\bm{x})$ ��������� ���������.

� ����� ������������ � ���� ������ ������ ������� $\bm{x} = (x_1, \dots, x_n) \in \{0, 1\}^n$ ��������������� ������ $\bm{y} \in \{0, 1\}^{n(n+1)/2}$ � ������������ $y_{ij} = x_i x_j$, $1 \le i \le j \le n$.
� ���� ������ ����������� $d = n(n+1)/2$, �������� ������������ ��������� ���������� ������� $y_{ij} = y_{ii} y_{jj}$ � $y_{ij} \in \{0,1\}$,
� ������� ������� ���������� ��������: $f(\bm{y}) = \sum_{1\le i \le j \le n} c_{ij} y_{ij}$.

\problem{������ ������ ��������� ���������� �� ��������� ����� ����� �������.}
\label{def:PolyMax}
����: ����� (������������) ������������� ���������� $p(x) = \sum_{k = 1}^d c_k x^k$
� ��� ����� $a, b \in \Z$, $a \le b$.
����� ����� $x \in [a, b]$ ��� ������� $p(x)$ ��������� ���������� ��������.

�� �������� � ���������� �������, � ������ ������ ��������� ���������� ����� ������ ���������� $x \in \Z$ ����������� ������ $\bm{y} \in \Z^d$, ���������� �������� ������������� ������� $y_k = y_1^k$, $k \in [d]$. ����� ������� ������� $f(\bm{y}) = \sum_{k=1}^{d} c_k y_k$.

% ����� ������� �������� �������� ����� ������������� ����������� � �������������� ���������������� ���� � \cite{Junger:1995} "Practical problem solving with cutting plane algorithms in combinatorial optimization"
% ����� �� ��� ������ ���������� NP-��������

�������� ����� ��������� �������� ����� ������������� ����������� ������������� �������������� �������� $\bm{x} \in \{0,1\}^d$. ��� ������� � ���, ��� ������ ���������� ������ ��������� ��������� ������������~\cite{Junger:1995,Onn:2004}.
���� �������� ��������� $E$, �������� ������������ $g\from 2^E \to \{\text{����}, \text{������}\}$ � ������� ����� $c\from E \to \Q$.
��� ������� ������������ $T\subseteq E$ ����������
�������� ������� ������� $f(T) = \sum_{e\in T} c(e)$.
��������� �����
\(T^* = \argmax_{T\subseteq E}\Set{f(T) \given g(T)}\).
��������� ���������� ������� ����� ������ ������� �� ������������������ ��������  ����������� ��������� $E$, ��� ������� $g$ ��������� �������� <<������>>.

������, ����� ������������� ����� �����������, �������� �����������, ������������� �� ����� �������� �������. � ������������ �����������~\ref{def:COP}, ���������� ���� ������� ������������� � ���� ������� ������.
�� ������ ������� ��� ����������� �������� ���������.
�������� ����� ������� ������������ ������ � ���������� ����, � ������� ����������� ����������������� ���� �������� �����, �� ���� $\bm{c} \ge \bm{0}$, ��������� ������ �� ������ NP-������� � ����� ������������� ����������~\cite[sec.~7.5b, 8.6b]{SchrijverCO:2003}.
����� ����� �������� ����� ������ ��������� �������� ������������� ����������� \emph{� ������������ �� ��������� �������� ������ (������� ��������)}.
�~����� ������, ���� ���� ������ � ����������� �� �������������� ����� ������������ ���� $\bm{c} \in P$, ��� $P=P(I)$ "--- ��������� �������.
(�~������ ������ � ���������� ����, $P = \R^d_+$.)


%% Глава 2
% !TEX root = MaksimenkoThesis.tex
%%%%%%%%%%%%%%%%%%%%%%%%%%%%%%%%%%%%%%%%%%%%%%%%%%%%%%%%%
%
%     ������������� �����
%
%%%%%%%%%%%%%%%%%%%%%%%%%%%%%%%%%%%%%%%%%%%%%%%%%%%%%%%%%

\chapter{������������� �����}
\label{chap:Polytopes}
%\begin{flushright}
%��� �������� ���� ������� �������� ������\\ \emph{�.~������}
%\end{flushright}

�������� ���� ���� ����� "--- ����� ��������� ����������� �� ���� �����������.
����� ���������� � �������� ������� ��������� �������������� (���������) ������.
�~�������~\ref{sec:Ident} ����������� ������ ������������� ������ �������������� �����. �~�������~\ref{sec:PolyhedralGraph} ����������� ��������� ��������� ����� ��� ����� ������������� ������������� �����, ��� ����� ������, ������� � �������� ����� ����� �������������.
�~�������~\ref{sec:ExtensionsAndRC} �������� ������� ���������� �������������, ��� ��������� � ����� �������������� �������� ������� ���������� ������"=�����������. �~�������~\ref{sec:questions} ������������� ����� �������, ������ �� ������� ����� ������������ � ����������� ������.

\section{������������� � �������� �����}
\label{sec:ProblemPolytopes}

������� �������� �� ��, ��� �������������� �������� ������ ������������� ����������� �� ���� �������� � ����������� �������� ������� ������� $f(\bm{x}) = \bm{c}^T \bm{x}$ �� ��������� �������� ��������� ���������� ������� $X \in \Z^d$.
%(�~���������, $X$ �������� ������������� ������ ���� $\Cube_d$ � ������ ���������������� ������, ����� $S = \{0,1\}$.)
������ ����������� �������� ������� ������� ����������� ������������ ������ ��������� $X$ ��� �������� ��������� $\conv(X)$.
����� �������, �������� ������ ������������� ����������� ������������ ����������� �������� ������� �� �������� ������������� $\conv(X)$.
%, ���������� \emph{�������������� ������}.
%�������, ��� ���� � ��� �� ������������ ������������� ��������� �������������� �����, ������������ ���� �� ����� ������ �������� ���������.
%\emph{�����} ������ ������������� ����� ����� �������� ��� $I$ ��������������� �������������� ������, ������������ ��������� $X = X(I)$.
%, ������� ���������� ��������� ���������� ������� $X$ (�� ���� �� �������� ���������� ���������� � ������� �������).
��� ����� ������������� �������� ������ ������������� � ���������� ��������������.
� ���������, � ������~\cite{Papadimitriou:1984}
������������ ������������� ������ ������������� ����������� ��� ������������������ 0/1"~�������������� $\{P_n \mid n\in\N\}$ �����,
��� ��� ������ ������� $\bm{v}$ � ������� $n$ (���� �������������)
�� ����� �� �������������� ����� ���������,
�������� �� $\bm{v}$ �������� ������������� $P_n$.


\begin{definition}\label{def:family}
	������ �������� ������ ������������� ����������� $(L, d, g)$ %(����������� ����������� ����� ���� �����) 
	������������� \emph{������������� ��������� ��������������} $\Set{P(I)\given I\in L}$, ��� ������������ $P(I)$ ������������ ����� �������� �������� ��������� $X(I)$ �� �����������~\ref{def:LCOP}.
%	\[
%	X(I) = \Set*{\bm{x}\in\Z^d \given \size(\|\bm{x}\|_{\infty}) = \poly(\size(I)) \text{ � } g(\bm{x}, I)}.
%	\]
	����� ��������, ��� ��������� $\Set{P(I)\given I\in L}$ \emph{������������ ������� $(L, d, g)$}, � $I \in L$ ����� �������� \emph{�����} ������������� $P(I)$.
\end{definition}

\begin{remark}
	����� � ����� ��� ����������� ����������� �������������� �� ����� ����� �������� �������������� ��������� ��� ������, ������������ V"~�������� �������������.
	��� �������������� ������� � ���, ��� ���, � ������ �������, ����� ������������ ��� �������� ��� ���������������: �������� ����������� � ����������� � ������� ���������������. ����������� ������ �� ���� �������� �������� �������� ������������, ������ ��������� ������ ����� ������������� �������� ����������� ���������� ���� �������� � ��������� ������ ��������� �������������.
\end{remark}

\begin{remark}
��������� ����� ���������� ��������� $g$ ����� $d \cdot \poly(\size(I) + d) + \size(I)$, �� ���� ������������� ������������ ����� $d + \size(I)$, ���������� � ������ \ref{chap:AffTheory}--\ref{chap:ExtAff} \emph{�������� �������������} $X(I)$.
\end{remark}
\begin{comment}
\begin{remark}
����� ������������ $\{0,1,\dots,k\}^d$ ������ ����� ������ $\{a,a+1,\dots,b\}^d$, $a,b\in\Z$, ����������� ���, ��� ����� ��������� $X \subseteq \{a,a+1,\dots,b\}^d$ ������������ ��������� $\bm{x} \mapsto \bm{x} - \bm{a}$ ������������� � ��������� $X' \subseteq \{0, k\}^d$, ��� $k = b-a$.
��� ���� �� �������� �� ������ ������������"=�������������� �������� ����� ���������, �� � ��������� ������ �������� ����������� �� ���.
(��������������, ��� $a$ � $b$, �����, ��� � $k$ � �����������~\ref{def:family}, ������������� ��������� ������������ ���� �������������.)
\end{remark}
\end{comment}
\begin{remark} 
�~\cite[�.~420]{ZieglerBook} ������������� �������������� ���������� ����� ������������ ������������� �������� ��������������.
�~\cite{Naddef:1981, MatsuiTamura:1995} �������������� ���������� �������������, � ������� ��� ������ ���� ��������� ������ �������� ������������ �� ������� �������� ����� ��������� �������, ������������ (���������) ������ ���� ������ ����� �������������.
�~��������� ������ ��� ����������� �� ������������.
\end{remark}

��������� � ��������.

� ������� ������ ������������� ���������������� ������������� \emph{����� ������������ ������������}
\begin{equation}
\label{eq:BQP}
\BQP(n) = \Set*{\bm{x}=(x_{ij}) \in \{0, 1\}^{n(n+1)/2} \given x_{ij} = x_{ii} x_{jj}, \ 1 \le i < j \le n},
\end{equation}
���������� ������� ����������� ����������~\cite{Deza:2001}.
� ������ ������ �������� $n \in \N$ �������� ����� �������������.

����� ������������ ������������ ����� ������ � \emph{�������������� ��������}{\label{def:CutPolytope}} $\Cut(n) \subseteq \R^{n(n-1)/2}$, ��������� �������� �������� ������������������ ������� �������� ������� ������������������ ����� �� $n$ ��������~\cite{Deza:2001}.

� ������� ��������� ��������� ���������� �� ��������� ����� ����� ������� $[a, b]$ ������ ����������� ������������
\begin{equation}
\label{eq:Cyclic}
\CP_d(a,b) = \Set*{(x, x^2, \dots, x^d) \given \ a \le x \le b, \ x\in \Z}.
\end{equation}
����� ����� ������������� ������ ������ $(d,a,b)$.

\emph{������������ ������ � �������} ������������ ����� �������� �������� ������������ 0/1-��������,
%������ ��������� $\Cube_d$, 
������������� ���������������� $H^-(\bm{a}, b)$, $\bm{a} = (a_i) \in \Z^n$, $b \in \Z$:
\begin{equation}
\label{eq:KNAP}
\Knap(\bm{a},b) = \Set*{\bm{x} \in \{0,1\}^{n} \given \bm{a}^T \bm{x} \le b}.
\end{equation}
����� ������������� �������� ���� $(\bm{a},b)$.

\hypertarget{def:PathPolytope}{�����} $G(V,E)$ "--- ������ ����������������� ���� �� $n$ ($n=|V|$) ��������, ����� ������� �������� ���: $s$ � $t$.
����� $W \subseteq 2^E$ "--- ��������� ���� $s$-$t$ ����� � ���� �����.
\emph{�������������� �����} ���������� �������� �������� ��������� ���� ������������������ �������� $\Path(n) \subseteq \{0,1\}^E$ ��� ����� �� $W$.
���������� ������������ \emph{������������ �������} $\Dipath(n) \subseteq \{0,1\}^A$ ��� ������� ���������������� ����� $D(V,A)$ �� $n$ ��������.

� �������������� ����� ����� ������ \emph{������������ ������ ������������} ���
\label{def:TSP}
\emph{������������ ������������� ������} %~\cite{Emelichev:1981} 
$\TSP(n)$, �������������� ����� �������� �������� ��������� ������������������ �������� ���� ������������� ������ ������� ������������������ ����� �� $n$ ��������.
� ���� �������, �������� �������� ��������� ������������������ �������� ������������� �������� � ������ ��������������� ����� �� $n$ �������� ���������� \emph{�������������� ������������� ������ ������������} ��� \emph{�������������� ������������� ��������} � ������������ $\ATSP(n)$.

������ � ���������� ������� (��-��������, ����� �� ����� �������������� �� ��������) ������������� ��������� \emph{��������������� ��������������}~\cite{Emelichev:1981} ��� \emph{������������}~\cite{ZieglerBook}.\label{def:perm-birk}
��������� ����������� $\Perm(n)$ �������� �������,
���������� ������������� �������������� ��������� ������� $(1, 2, \dots, n)$.
���� �� ��� ������������ $\pi \from [n] \to [n]$
������ ������� $(\pi(1), \pi(2), \dots, \pi(n))$
�� ���������� ��������������� ������� $\bm{x} \in \{0,1\}^{n\times n}$ � ������������
\[
x_{ij} = \begin{cases}
1,& \text{���� $\pi(i) = j$,}\\
0& \text{�����,}
\end{cases}
\]
�� ������� \emph{������������ ��������} $\Birk(n)$, ������� ��� ���������� \emph{�������������� ���������������� ������} � \emph{�������������� ������ � �����������}~\cite{Emelichev:1981}.
��������������� ������� $\bm{x}$ ����� ����� ���������������� ��� ������������������ ������ ������������ ������������� � ������ ���������� �����,
������ ���� �������� �������� �� $n$ ������. 
� ���� ����� ������ ������������ $\Birk(n)$ ����� ���� ����� ������ �������������� ����������� ������������� � ���������� �����.

\emph{������������ ����������� �������������} $\Match(n)$ ������������ ��� �������� �������� ���� ������������������ �������� ����������� ������������� � ������ ����� �� $n$ ��������.

������ � ������ � ������ �������"=���������� ����� $G(V,E)$ ������������ (�� ���������� ���� �������� �����) ��������� (�� ���� ������������ ��� ������� �����) ������ ����� ���� ����������������� ��� ������ ����������� �� \emph{������������� �������� ��������} $\Tree(n) \subset \{0,1\}^{E}$, $n = |V|$, ��������� �������� �������� ������������������ ������� �������� �������� � �����~$G$.

���������� �������������� �������� �������� �������� ������������� ��������� (������, ��� ���������). 
�~\cite{Feichtner:2005} ���������� ������ ������������, �� ������ � ��� ������� � ����� ����������� ���� ��������������.
0/1"~������������ � $\R^n$ ���������� \emph{�������������� ��������}, ���� �� ����� � �������������� $H(\bm{1}, r)$ ��� ��������� ����� $r \in [n]$ � ��� ������� ������������� ���������� �������� ���������\label{matroid}: ������� $\bm{x}$ � $\bm{y}$ ������ ����� � ������ �����, ����� �������� $i,j \in [n]$, $i \ne j$, �����, ��� $\bm{x} - \bm{y} = \bm{e_i} - \bm{e_j}$. ��� ���� ����� $r$ ���������� \emph{������} ��������. � ���������, ���� �������� �������� �������� � ������ ����� �� $n$ �������� ����� $n-1$.

��� ���� ��������� ��������������, ����� ����������� � ����������, "--- \emph{\hypertarget{Stable}{�������������} ����������� ��������} � ����� $G=(V,E)$~\cite{Chvatal:1975}, ����� ���������� \emph{��������������� �������� ������}~\cite{Nemhauser:1975}:
\begin{equation*}
\label{def:Stable}
\Stable(G) = \Set*{\bm{x}\in\{0,1\}^V \given x_v + x_u \le 1 \text{ ��� ������� ����� } \{v,u\} \in E}.
\end{equation*}
%��������� ������ \emph{������������� ����������� ��������} $\Stable(G) \subseteq \{0,1\}^V$ ������� �� ������������������ �������� ����������� �������� ����� $G(V,E)$, �������� $n$ ������.

%� ����� ������������� �������� ����� ������������� ����������� ����� ���������� \emph{�������������� ���������������}. 
%��� ����� �� ����������� ��������, ���� ����� �������� ���������� ������������� ������������� �������� 0/1-���������������.

����, ������ �� ������ ������ �������������� �����, ����������� � ����������, ����� �������� � �����~\ref{chap:AffExamples} ��� ����������� ���������.

%���������������� �������� ������� �������������� ����� ����������� ������� ���������.
%�������� ������� ������� � ���, ��� ����������� �������� ������� ������� �� ��������� $X$ � �� ��� �������� �������� $\conv(X)$ ���������.
��� ��� ���� ������� ����, �������� ������ ������������� ����������� ����� ���� �������������� ��� ������ ����������� �������� ������� �� �������� �������������.
��� � 1954 ���� ������, ��������� � �������~\cite{DantzigFJ:1954}, �������� �� ��� ���� � ��������� ������������� �������� ��������"=�����, �������� ������������� �� ��� �������� ��������� � ������� ������ ������������.
������������ ���� ������ ����� ���������� ����������, ������� ������� ����� ��������� ������� � �� ����������� ��� ������� ����� ������ ����~\cite{SchrijverCO:2003}.
�����������, ������������ ���� ��� ����� ������ (���������) ������� �� ��������� ������� ��� �������� ������������� ��������������� ��������������.
���, ��������, ����������� ������������� ������ ������ ������� ��������� ��������������� ������, ��� ��� ����� ����� �������� ������� ��� ������ ������������ �������.
������� ��������� ������������� ������������� �������������, � ������� ������� ����� ��������� ��������� ��������������� ������, �������� ����� ��� ������, ����� �����������, �������� �������������� ����� �������������, �������� �������������� ������� ���������� ������"=�����������.
��������� ������������"=�������������� ������������� (�� ���������� ����� ��������������) ����� ������� ��������� ������������� ����� ������������� (����������, �������, ����� �~�.~�.), ��������� ������ ����������� ��� ������� �������������, ����������� ����� ����������� ���������� ������������� � ������ ������.
%��������� �������������, �� ���������� ����� �������������� (��. ���������~\ref{rem:combinatorial}), ����� ������� 
%��������� ������������� ����� ������������� (����������, �������, ����� �~�.~�.), ��������� ������ ����������� ��� ������� �������������, ����������� ����� ����������� ���������� �������������.
���� ������ �� ���� ������������� � ������ ��������� �������� ����� ������������� ����������� ����� ����������� ���� ����,
� ��������~\ref{sec:Ident}, \ref{sec:PolyhedralGraph} �~\ref{sec:ExtensionsAndRC}.
�� ������ ��� ������� ����� ����������� �� ����� ������������� �� �������� ������ �����, ����� �� ������� ������ ������������� ��������� ������������ �����������. ��������, ����������������� ��������� �������� �������.


\subsection{�������� �����}
\label{subsec:polyhedra}

������ ��������, ��� ������ � ���������� (��)���� ������������� ��������� ��� ������� ����������������� ���� ����� (���)~\cite{Dijkstra:1959}.
������ � ���, ���� ����� ��� �����������, �� ������ ���������� NP"~�������~\cite{Garey:1982}.
�� ���� ������ ����������� �������� ������� ������� �� ������������� (��)����� NP"~������.
��� �� ������� ������������ ������ � ������������ ����������������� ��������� �������� �������?
������������� ��������� � ���� ������ �������� ������� ��������� �������������~\cite{SchrijverCO:2003}.

\emph{����������} ������������� $P \subseteq \R^d$ ���������� �������
\[
P^{\uparrow} = \Set{\bm{y} \in \R^d \given \exists \bm{x} \in P \ \bm{y} \ge \bm{x}} = P + \R^d_+,
\]
��� $\R^d_+ = \Set{\bm{x} \in \R^d \given \bm{x} \ge \bm{0}} = \cone\{\bm{e_1}, \dots, \bm{e_d}\}$.

%�������� ������, ��� $P^{\uparrow}$ �������� ������������� ��������� ������������� $Q = \Set{(\bm{y},\bm{x}) \in \R^d\times\R^d \given \bm{y} \ge \bm{x} \text{ ��� ���������� $\bm{x} \in P$}}$.

����� �������, ������ � ���������� (��)���� � ������������ ����������������� ���� ����� (���) ������������� ������ ����������� �� �������� $\Path^{\uparrow}(n)$ ($\Dipath^{\uparrow}(n)$).

����������� ������� ����� ��������� ������� ��� ������������� ���������� ������ � ����������� ������� � ������ ����������������� ������� ���������� ����� �� $n$ ��������.
� ���� ����� ���������� ��������� ������ $\Cut(n)$ ������������� �������� � ������ �� ���� ������� � �������� ������������, ��������������� ������� �������.
��������� �������� �������� ����� ��������� ��������� $\MinCut(n)$
� ����� �������� \emph{��������� ��������}~\cite{Conforti:2004}. 
��� ������� ������� �������, ��� �~\cite{Skutella:2010} ����������� �������� �������� �������������� $s$-$t$ �������� � ������������ (��������) ������.
� ������ ������� ����� ��� �������� ����������� � ��������� $\Path^{\uparrow}(n)$~\cite{SchrijverCO:2003}.

������� ������� ������������ ������� ������ � ���������� ���� � ������ ������� $G(V,A)$ ��� �������, ��� �~$G$ ����������� ������� ������������� �����:
\[
\ShortP(n) = \Dipath(n) + \cone(\Cycle(n)),
\]
��� $n = |V|$, � $\Cycle(n)$ "--- ��������� ������������������ �������� �������� � ����� $G$.
������ ���, � �� ������� $\Dipath^{\uparrow}(n)$, � ����� � ���������� �������� \emph{��������� ���������� �������}.
��������~\cite{Saigal:1969, Vohra:2011}, ��� ��� H-�������� ����������� ����������, ��� � �������� $\Dipath^{\uparrow}(n)$.
� ���� ����� $|A|$ �����������, ������������ ������������� ���� $x_a \ge 0$, $a \in A$.
� ��� ������� ����� � ����������� ���������������, ������������ ���������� ����������� (��� �������� ����������������� � �������� ������� $G$).
�������� ����� ������ ���, ��������� �� (���������) �������~$s$, � ������ �������� � �� ��� ����� �������.
�������� ����� ������ ���, �������� � (��������) �������~$t$, �~������ ��������� �� �� ����� �������.
�� �� �������� ��� ����� ������ ������� ����� ���� (������� ���������� ������).

����� �������, ��� ����� ������ ����������� $\bm{c}^T \bm{x} \to \max$ �� �������� ��������� $X\subset \Z^d$, ������� ������ $\bm{c}$ ������� ������ ������������� ������ ����������� ���� $\bm{c}^T \bm{a_i} \le 0$, $i \in [k]$, ��������������� \emph{������� ������} ������������ ����� ����� ����������� ������������� $\conv(X)$ � ������ %<<������������>> ������� �������� 
$\cone\{\bm{a_1}, \dots, \bm{a_k}\}$.

%%%%%%%%%%%%%%%%%%%%%%%%%%%%%%%%%%%%%%%%%%%%%%%%%%%%%%%%%
%
%     ������ ������������� �����
%
%%%%%%%%%%%%%%%%%%%%%%%%%%%%%%%%%%%%%%%%%%%%%%%%%%%%%%%%%

\section{������ ������������� �����}
\label{sec:Ident}

������, ��� ������� � ���������� ������������� �������������� (���������), ������� �������� �� ������ ������������� ����� (�������, �����, ���������� �~�.\,�.). 

������� ������ \emph{������ ������������� �����} ������� �� ���� ������������� (��������) � �������� ����������� �����. 
%����������� ��������������, ��� ����� �������� ������������ $g$ �� �����������~\ref{def:COP}, �������������� ��� �� ����������� �������������� ������������� ������� $\bm{x}$ ��������� ���������� ������� $X \subseteq \R^d$ (� ���� ������ �������������� ������ �������� �������� �������� $\conv(X)$). 
%���� �� ���� ���� �� � �������������, � � ��������, �������������� ����� ����� ����������� ������������� ������ � ������ <<������������>> ������� ��������, �� � ���������� � ���������, ����������������� ������� �������������, ������ ����������� ��������, ���������������� ������������� ���� ������.
%��� �������, ��� ���������� ����� ��� ��������� ��������� �������� ������������� ����������� (��. ���������~\ref{rem:PolyPred}).
������ �������� ����������� ����� ������� �� � �����������.
������������� ������� ������������ �������� ���������,
����� "--- ����� ��������� (���������), 
���������� "--- �������������� ���������������� ��������� ���������.
������� ������ ������������� ����� �������� ����� �� ��� ���.

\subsection{������������� �������}

����������� ��� ���� �������������� ��������� � ��������� ����� ���������� ����� ������������� ������� �������� �������.
� �������� ��� ������� � ���, ��� ��� � �������������, ��� � � ���������� ������������� ���������� �������� ��������� ��� �������, ����� ��������� ���������� ������� $X$ ������� ������ �� 0/1-��������.
� ���� ������ $X$ ��������� � ���������� ������ ������������� $\conv(X)$, � �������� ������������ $g$ (�� �����������~\ref{def:family}) �������������� ������� ������������� ������.
���� �� ����� ������ ������������� ���� �� ������ 0/1-�������, ��, ������ ������, ������������� ������� �� �������� � ���������� ������ ��������� ������������ � ����� ����������� ��������� ����� ��������� ���������� ������� $X$. � ����� ������, ��� ������ �������� co"~NP"~������~\cite[Theorem 18.5]{Schrijver:1998}.

������������� ������� ����� ������������ �������������,
��� ������� ��� ������ ���������� ��������� ������������ �������� NP"~������� ��� �� ����� ���������������� ���������.
��������, �~\cite{Yannakakis:1991} ��������������� ������������, ��������� �������� �������� ������������������ ������� ���� ������������� ��������� ������� ����� (�� $n$ ��������).
��� ��� ������ �������� ��������������� ����� �������� NP"~������~\cite{Karp:1972}, �� � ������ ������������� ������� (� ����� ������ ���������� ��������� ������������) ������ ������������� NP"~�����.

\subsection{������������� ����������}
\label{subsec:IdentFacet}

��������~\cite{SchrijverCO:2003}, ��� ��� ������� $\NP \ne \coNP$ ������� ������ ������������� ���������� ��� ��������� �������������� ����� NP"~������� �������� ������ ������������� ����������� �� ����� ���� ����������� �� �������������� �����.
����� ����, ������ ������������� ����������
�������� $D^p$-������ (�, �������������, NP"~�������)
��� �������������� ������ ������������~\cite{PapadimitriouW:1988}, ������ � �����~\cite{PapadimitriouY:1984} � ������ � �������� ��������������~\cite{Fiorini:2006}.

\begin{comment}
� ������� ������������� ���������� ����� ������� ������ \emph{������������� ������� ��������������}:
��� ������ $\bm{a}\in\Z^d$ � $b \in \Z$ ���������,
�������� �� �������������� $H(\bm{a}, b)$ ������� ��� ������� ������������� $P$.
��������, ��� ��� ������ $D^p$-����� ��� �������������� ������ ������������ � ������ � ����� ~\cite{PapadimitriouY:1984}.
� ��� ������������� ������ � ������� �������� co"~NP"~������� ������������� ����������� �����������~\cite{Hartvigsen:1992}.
\end{comment}

\subsection{������������� �����}

������ ������������� ����� ������������� ����� ���������� \emph{������� � ��������� ������}.
������� � ���� ������ ������ �������������� ���, ��� �������� ��������� ������ ������������� ����� ��������� ������� ��� ���������� ������������ ���������, ������������� ������� ���������� ������~\cite{MatsuiTamura:1995}.
����� ����, ����� �� ��������� � ������������ ����� ������������� ������ ��������������, ��� ������� � �������� �����, ��� ������������ ����� ��������� ��������� ��������� � ���� ������� ����������� � ��������� ������.

������ �����, ������������ �������� ��������� ������ ��� ��������� (� ��� ������� �������������) ������������� ���������� ����� ������������� �����������:
������ � ���������� �������, ������ � ����������� �������� ������, ������ � ����������� � � �������������� (� ������ �������"=���������� �����), ������ � ���������� ����, ������ � ����������� ������� � ������ ��������� ��������� ���������� �� ��������� ����� ����� �������.

%���������� ��������� ��������� ���������� � ������ ��������� ������ ��� �������������� ����� ������������� �����������. ������ � ������������� ���������� �����.

���� (������~\ref{matroid}, �.~\pageref{matroid}), ��� ����������� ������������� �������� ��� ������������� �������� ��������� ��� ������.
������� $\bm{x}$ � $\bm{y}$ ����������� $\Perm(n)$ ������ ����� � ������ �����, ����� �������� $i\in[n-1]$ �����, ��� ������ $\bm{y}$
���������� �� ������� $\bm{x}$ ������������� $i$-� � $(i+1)$-� ���������~\cite{Emelichev:1981}.

\begin{lemma}[\cite{Balinski:1974}]
\label{lem:BirkAdj}
��� ������� ������������� �������� $\Birk(n)$ ������ ����� � ������ �����, ����� �������������� �������� $p_1 \symdiff p_2 = (p_1 \setminus p_2) \cup (p_2 \setminus p_1)$ ��������������� ������������� $p_1$ � $p_2$
�������� (����) ����.
\end{lemma}

� �������� ��� �� ������������� �������� ��������� � ��� ������������� ����������� ������������� $\Match(n)$~\cite{PadbergRao:1974, Chvatal:1975}.

��� ������ ��������� $\Path^{\uparrow}(n)$, $\Dipath^{\uparrow}(n)$ � �������� ���������� ������� $\ShortP(n)$ �������� ��������� �������� ���������.
�������� ��� ������������ ��� �������.

\begin{lemma}[\cite{SchrijverCO:2003,Maksimenko:2004}]
\label{lem:path}
����� $\tx$ � $\ty$ "--- ��� ��������� $s$-$t$ ���� � ������ ������� $D$ �� $n$ ��������.
����� �� ������������������ ������� $\chi(\tx)$ � $\chi(\ty)$ �������� �������� ��������� �������� $\ShortP(n)$ (�������� $\Dipath^{\uparrow}(n)$) ����� � ������ �����, ����� �������������� �������� $\tx \symdiff \ty$ �������� ����������������� ����, ��������� �� ���� �������, ������� ����� ������ � ����� �����, � �� ������� ������ ����� ������.
\end{lemma}

\begin{remark}
%�������� ���������, ��������� � �����~\ref{lem:path}, ���������� ��� ��� �������� $\Dipath^{\uparrow}(n)$, ��� � ��� $\ShortP(n)$.
�������������� ���� ����� ��� �������� $\Dipath^{\uparrow}(n)$ ���� �~\cite[theorem~13.4, p.~202]{SchrijverCO:2003}, �� �������� ����������. ��� �������, ��� ���� ����������� ���� ������� $\tx$ � $\ty$ �������� ������ ������ $\tz$, �� ������ $\bm{w} = \chi(\tx) + \chi(\ty) - \chi(\tz)$ ���� �������� ������������������ �������� ���������� ������.
� ���������, ��� ����������� ������� ��� �������, ������������ �� ���.~\ref{fig:contrpath}. 
%����� ����, �~\cite{SchrijverCO:2003} ����������� �������������� ����, ��� ���� ����������� ������� $\tx$ � $\ty$ �� �������� ������ ������, �� $\tx \symdiff \ty$ �������� ����������������� ����.
�~������~\cite{Rispoli:1992}, ����������� �������� ����� �������� $\ShortP(n)$, ����� ���������� ���������� ��� � ����������� ������ �������� (������������, ��� $\ShortP(n)$ ��������� � �������������� ������� $\Dipath(n)$, � � �������� �������� ����������� ���� ���������), ��� � � ������������ �������� ��������� ������ (������� ������� ����� � �������).
������ �������������� ���� ����� ���������� ����, �~�������~\ref{sec:ShortPathClique}.
%��� �� �����, ��� ���������� ����� �����������, ���� ��������, ��� ������ $\bm{x}$ ������������ ����� ����� ������������������� ������� ���������� ������ �, ���� �����, ������������������ �������� ���������� ��������. 
%����� �������, $\bm{x}$ ����������� �������� $\ShortP(n)$ (�, ��� ���������, �������� $\Dipath^{\uparrow}(n)$), � �� ��������� $\chi(p_1) + \chi(p_2) = \chi(p_3) + \bm{x}$ ������� ����������� $\chi(p_1)$ � $\chi(p_2)$.
\end{remark}
\begin{figure}%
	\centering
	\begin{tikzpicture}[>=stealth']
	\begin{scope}[yshift=2ex, xshift=-5cm]
		\foreach \i in {0,...,3} {
			\node[circle, draw, inner sep = 2pt] (b\i) at (\i,0) {};
		}	
		\draw[->] (b0) node[left] {$s$} -- (b1);
		\draw[->] (b1) -- (b2);
		\draw[->] (b2) -- (b3) node[right] {$t$};
		\node[left] at (-1,0) {$\tx:$};
	\end{scope}
	\begin{scope}[yshift=-2ex, xshift=-5cm]
		\foreach \i in {0,...,3} {
			\node[circle, draw, inner sep = 2pt] (b\i) at (\i,0) {};
		}	
		\draw[->] (b0) node[left] {$s$} to[bend right] (b2);
		\draw[->] (b2) to[bend right] (b1);
		\draw[->] (b1) to[bend right] (b3) (b3) node[right] {$t$};
		\node[left] at (-1,0) {$\ty:$};
	\end{scope}
	\begin{scope}[yshift=2ex, xshift=5cm]
		\foreach \i in {0,...,3} {
			\node[circle, draw, inner sep = 2pt] (b\i) at (\i,0) {};
		}	
		\draw[->] (b0) node[left] {$s$} to[bend right] (b2);
		\draw[->] (b2) -- (b3) node[right] {$t$};
		\node[left] at (-1,0) {$\tz:$};
	\end{scope}
	\begin{scope}[yshift=-2ex, xshift=5cm]
		\foreach \i in {0,...,3} {
			\node[circle, draw, inner sep = 2pt] (b\i) at (\i,0) {};
		}	
		\draw[->] (b0) node[left] {$s$} -- (b1);
		\draw[->] (b2) to[bend right] (b1);
		\draw[->] (b1) -- (b2);
		\draw[->] (b1) to[bend right] (b3) (b3) node[right] {$t$};
		\node[left] at (-1,0) {$\tx\cup \ty \setminus \tz:$};
	\end{scope}
	\end{tikzpicture}
	\caption{������ ���� ������� $\tx$ � $\ty$, ��������������� ���� ��������� ������ ������������� $\Dipath^{\uparrow}(4)$}%
	\label{fig:contrpath}%
\end{figure}

\begin{lemma}[\cite{Nikolaev:2016}]
\label{lem:adjmincut}
������� $\bm{x}$ � $\bm{y}$ �������� �������� $\MinCut(n)$ ������ ����� � ������ �����, ����� ��� ��������������� �������� $\delta(A)$ � $\delta(B)$
� ������ ����� $G(V,E)$ ����������� ���� �� �������:
\[
A \cap B = \emptyset \quad \text{���} \quad 
A \subset B \quad \text{���} \quad 
B \subset A \quad \text{���} \quad 
A \cup B = V.
\]
\end{lemma}
\begin{remark}
� ������������ �������� ��������� �~\cite{Nikolaev:2016} �������� ��������� �������. ��� �� �����, ��� ����������� ����������� ����� ��������������.
\end{remark}

������������ �������� ��������� ������ ��� ��������� $s$-$t$ �������� � ������ (��)�����~\cite{Skutella:2010} �������� ������� ������� ����������� �����~\ref{lem:adjmincut} � ��� ������,
��� ��� $s$-$t$ �������� ������ ��������� ������� $A \cap B \ne \emptyset$ (��� ���  $s \in A \cap B$) � $A \cup B \ne V$ (��� ��� $t \notin A \cup B$).
%\ref{def:stdicut}

�������� ��������� ������ ������������ ������������� $\CP_d(a,b)$ �������� �����~\cite{Gale:1963}. ������ ��� ��� ������� ������ ��� $d \ge 4$.

��������� ������ � ����������� ������������� ��������� ������ �� �������������� NP"~������� �����.
�� ����� ��������� �� ��� ������.
������ �����, ���������� �������������, ��� ������� ����������� �������������� ������������ ������ � ��������� ������.

�������� ��������� ������ ������������� �������� $\Cut(n)$ � ������ ������������� ������������� $\BQP(n)$ ���� ���������� ����������� ����������� ��������~\cite{Bondarenko:1987,Beloshevskii:1986,Barahona:1986,Padberg:1989}.

%{\color{red}������ �������. ��������� \cite{MatsuiTamura:1995}.}

��������~\cite{Chvatal:1975}, ���� ������ ������������� ����������� �������� $\Stable(G)$ ������ ����� � ������ �����, ����� �������������� �������� ��������������� ����������� �������� ���������� ������� ������� �~$G$.
��-����, ���� �� �������� ��������� ����� � ��� �������������� �������� �������� � �������������� ��������� ��������~\cite{Ikura:1985}.
����� ����, ������� ������� �������������� ��������� �������� �������� ������������� ������������� (� ����� ��������������) ������ � �����������
(��. ������~\ref{sec:3Ass}).
%{\color{red}(��., �������, \cite{Balas:1989}), ����� ������ �� ��������������� ������ � ��������� ����� <<�������� ����������>>}.

�������������� �������� ��� �������� ��������� ������ ������������� �������� �������� ������ �~\cite{Young:1978}.

������������� ��������� ������ co"~NP"~����� ��� �������������� ��������� �����: ������ ������������~\cite{Papadimitriou:1978}, ������ � �������~\cite{Chung:1980, Geist:1992, Matsui:1995}, 
������ � �������� ���������~\cite{Matsui:1995}, 
������ � ���������� ��������~\cite{Bondarenko:1996},
������ � ����������� � ������������~\cite{Alfakih:1998}, 
������ � 3-������������
� ������ � ��������� ��������������~\cite{Fiorini:2003}.


%%%%%%%%%%%%%%%%%%%%%%%%%%%%%%%%%%%%%%%%%%%%%%%%%%%%%%%%%
%
%     ����������� � �������������� ����� �������������
%
%%%%%%%%%%%%%%%%%%%%%%%%%%%%%%%%%%%%%%%%%%%%%%%%%%%%%%%%%

\section{�������������� ����� �������������}
\label{sec:PolyhedralGraph}

\subsection{������� �������������}

����� ������ ������������� �������� ������� ������� ����� ����� ������������ ��������� ��� ������� ������ ������������� �����������, ������������ ������ ������� ���� ���������� �������.
����������, ��� ������ ����������� ������ � ��� �������, ����� ������� ����������� ��������� �������� ���� ���������� �������. 
%� ������ ������ ������ ��������� ���������� ������� ���������� ���������� ��� ������������ ������� $U = S^d$ � ������� ������� ���������� ���������� $|S|^d$.

�������, ��� �������������� ��������� ������������� ������� ������������� ��� �� ����������� ���������������� ���������� ������ ����� ������.
� ����� ����������� ����� ����������� ���������� ������ �������� ����� ������ ������������� ������ � ������� $\Knap(\bm{a}, b)$.
������ ������������� ������� ����� ������������� �������� �������. ���������� ���������, ��� ������ ������ $\bm{x}$ �������� 0/1-�������� � ������������� ����������� $\bm{a^T} \bm{x} \le b$.

\begin{prop}
	������ ���������� ����� ������ ������������� $\Knap(\bm{a}, b)$ �������� NP"~�������.
\end{prop}
\begin{proof}
���������� ������� ������ ������, ������� $2b = \bm{a}^T \bm{1}$.
�������, ��� ��� ����� ������� ����� 0/1"~�������� $\bm{x}$,
��������������� ����������� $\bm{a}^T \bm{x} \le b$, 
��������� � ������ 0/1"~�������� $\bm{y}$, ��������������� ����������� $\bm{a}^T \bm{y} \ge b$.
(����������� $\bm{y} = \bm{1} - \bm{x}$ ���������� �������"=����������� ������������ ����� ����� �����������.)

����� $N$ "--- ����� ������ ������������� $\Knap(\bm{a}, b)$,
� $K$ "--- ����� ��� ������, ��������������� ��������� $\bm{a}^T \bm{x} = b$.
� ���� ���������� ���� ���������, ��� ����� ������� ������������ $2 N = 2^n + K$, ��� $n$ "--- ����������� ������� $\bm{a}$, � $2^n$ "--- ����� ���� 0/1-�������� ���� �����������.
����� �������, ������ ���������� $N$ ������������ ���������� $K$.
�� ��� ������ �������� ����������� $K > 0$ �������� NP"~������ (������ � ����� ��������~\cite{Garey:1982}).
\end{proof}

%\subsection{���������� � ������� �����}

%�������� ������� ���� �����������, ������������ ������ �������� �������� ��������� ��������� ���������� ������� $X$.
%������� ����� ����� � ���� ������� �������������� <<(��)������ ������������>> ��������, ��� ������� ������������� ���������� ��������, ���������������� ��������� $X$.

\subsection{������� �����}
\label{sec:diameter}

%������� ����� ������������� $P$ ����� ���������� $\delta(P)$.

�������� ���������� ��� ������ ��������� ������ �������������� �������� ��� ����, ��� ��������"=�����~\cite{Dantzig:1951} ������ ������ ��������� ����������������, �������� �� ������ ����� ������������� �� �������� ������� � �����������.
����� �������, ���� ������� ������ ����� (����. pivot rule) � ��������"=������ �������� (�������� ���������� ���� �� ����������� �������), �� ������� ����� ������������� ����� ����� ����� ��������"=������ ��� ��������� ������ �������� �������.

� 1957 ���� ������ ���� �������� �������� � ���, ��� ������� ����� �������� �� ����������� �������� ����� ������ ��� ����������� � ������������~\cite{Dantzig:1963}.
������ ����� ���������� ���� �������� ��� ������ ����������� "--- 4-������ ������� � 8 ������������ � 15 ���������, ������� ����� �������� ����� 5~\cite{Klee:1967}.
������� �������� ���� ��������������� �� ������ ������������ ��������� (��������������) � ������ � 2010 ���� ��� ������ ����������� ��� ��������������~\cite{Santos:2012}.
��� �� �����, ���������� ��������������, ������� ����� ������� ��� �� ���� �� � ��� ���� ������ ����� �����������, �������� ����� ����������� �������~\cite{Santos:2013}.
� ������ �������, �� ��� ��� �� �������� �������������� ������� ������ �������� ����� ������������� �������������.
������� ������� ����� ����� ����������� ������������ �������� �������������� �������� �����:

\begin{conjecture}[�������������� �������� �����]
C��������� ����� �������������� ������� $f(n, d)$,
��� ��� ������ $d$"~������� ������������� (��������)
� $n$ ������������ ��� ������� ����� �� ��������� $f(n, d)$.
\end{conjecture}

��� ������� �� ����������� ���� (� ������ �������) �����������,
��� �������� ������� � ������������ ���������� ������ ��������������� (�� ���� �� ���������� �� ������� ������� �����) ��������� ��������� ���������������� �� ������ ��������"=������.
������ ���������� ������ ��������� �������� ������� � ������ �������������� ����� XXI ����~\cite{Smale:1998}.

%\begin{remark}
%��� ������ ��������� ������ � ������� �������� ����� �� ������������� ������� �������� �������� �� ��, ��� ��� ������ ����������, ���� H-�������� ������������� ����������� ��� �� �� ����� ������������ ��������.
%\end{remark}

� ���� ���������� ���� ������, ������ ��������� ������ ��� ��������� �������� �������������� ��������� ������� ����� �����.
����� �� ���, � ����� ������ �� ����� ������ ������ �� ���� ���� ����� ����� �~\cite{Kim:2010,Santos:2013}. 
����� �� �� ���������� � ������ ������� ��������� ��������� ������ ��������� ������ �������������� ��������������� � �������� ������������� �����������:
\begin{enumerate}
\item ������� ����������� $\Perm(n)$ ����� $n(n-1)/2$~\cite{Emelichev:1981}.
\item ��������������� �� �������� ��������� ������ ������������� �������� �������, ��� ������� ��� ����� ������ ���� ����� ����� $r$. � ���������, ������� ����� ������������� �������� �������� $\Tree(n)$ ����� $n-1$ ��� $n \ge 4$.
\item ������� ����� �������� ���������� ����� $\ShortP(n)$ (� ����� �������� $\Dipath^{\uparrow}(n)$) ����� ���� ��� $n\ge 4$. ���������� ��������, ��� �������, ��������������� ����, ������������ �� ����� ������������ ���� $(s,t)$, ������ �� ����� ���������� ��������� �������� (��. �������� ��������� � �����~\ref{lem:path}).
%~\cite{Rispoli:1992} ({\color{red}������ � ������������ ������������� � � �������� ���������}).
\item ������� ����� �������� $\MinCut(n)$ (� ��������� $s$-$t$ �������� � ������ (��)���\-��) ����� ���� ��� $n\ge 4$. ���������� ��������, ��� �������, ��������������� ������� $\delta(S)$, ��� $S$ ������� �� ����� �������, ������ �� ����� ���������� ��������� �������� (��. �������� ��������� � �����~\ref{lem:adjmincut}).
\item ������� ����� ������������� �������� $\Birk(n)$ ����� ���� ��� $n \ge 4$~\cite{Balinski:1974}. �� �� ����� � ��� ������������� ������������� $\Match(n)$~\cite{PadbergRao:1974}.
\item ������� ����� ������������ ������������� ����� �������~\cite{Gale:1963}.
\item �������� ������ ������������� �������� $\Cut(n)$ � ������ ������������� ������������� $\BQP(n)$ ����� �������~\cite{Bondarenko:1987,Beloshevskii:1986,Barahona:1986,Padberg:1989}.
\item ������� ����� ������������� ������������� �������� $\ATSP(n)$ ����� ���� ��� $n \ge 6$~\cite{PadbergRao:1974}.
� ��� �� ������ ������� ����������� ���� ��� ������������� $k$"~���������� � ��������� ������.
\item ������� ����� ������������� ������ ������������ $\TSP(n)$ �� ����������� �������~\cite{RispoliCosares:1998}.
\item ������� ������������� �������� �������� $\LOP(n)$ (��. ����������� �� �.~\pageref{def:LOP}) ����� ����~\cite{Young:1978}.
\item ������� ������������� $k$-������ �� ����������� ����~\cite{Girlich:2006}.
\end{enumerate}
���� ������ ����� ���� �������� ����� ������� (�� �������� ����� ������� ������������) �������� ��������� ������ ��� ������������ ��������������~\cite{Borgwardt:2015, DeLoera:2014, Kim:2010},
���������� �����������~\cite{Ceballos:2015,Ceballos:2016} � �������������� ��������� ���������~\cite{Borgwardt:2013}.
%���������� (��\'��� ��???) �������� ��� ������������� � ������ ����� � ����������� ����~\cite{Rispoli:1992}.

������, ��������, ��� ������� ����� 0/1"~������������� �� ��������� ��� �����������~\cite{Naddef:1989}.
���� �� $X \subseteq \{0,1,\dots,k\}^d$, �� ������� ����� ������������� $\conv(X)$ �� ��������� $k\cdot \dim(\conv(X)) \le k d$~\cite{KleinschmidtOnn:1992}.
� �������� ������� �������, ��� ��� ����������� $k=n-1$, $\dim(\Perm(n)) = n-1$, � ������� ����� $n(n-1)/2$.


\subsection{�������� ����� �����}
\label{sec:CliqueNumber}

�~1980-� ����� �.\,�.~���������� ���� ������� ������ ���������� ������� ���� ��� ����� ������������� �����������~\cite{BondBook:1995} (��. ����� ����� �����������, ������������ � ����������� ������ �~\cite{BondBook:2008}). 
�������� ������������ ���������� ����� ������ �������� ��, 
��� �� ������������ ����������� ����� �������� ������ ����� ������������� ��������������� �������� ������ ������������� ����������� (�~��������� %\cite{BondBook:1995} 
�������� ����� ���������� ����������).
����� ��������� �������� ���� ������ ���������� ����, � �����~\ref{chap:Direct}.
����� �� �� �������� ���� ������� ����� ��������� �����������.
����� ����� �������� ����� ����� ������������� $P$ ���������� $\omega(P)$.

�~\cite{BondBook:1995} ��������, ��� ��������� ����������, 
������ �������� ��� �������� (� ���������, ��� ������ � ����������� �������� ������), 
�������� �������� ��� ����������� ���� �~�����,
%�������� �������� ��� ������ � ���������� ������������� � ������������ �����,
�������� �����--�����--�������� � �������� ������ �~������ ��� ������ ������������
%, �~����� ��������� ������ ������������� ���������
�������� ����������� ������� ����.
��� ���� ����������� ��������������������� �������� ����� ������ �������������� ��������������� � ������ NP"~�������� ��������, ��� ������
� ������������ �������~\cite{Beloshevskii:1986,Barahona:1986},
������ � ������������ ����� � �������"=���������� �����~\cite{Greshnev:1984,Bondarenko:1985}, 
������ �� �������� �������� � ��������������� �������������~\cite{Shovgenov:2015}, 
������ � ���������� �����~\cite{Bondarenko:1996},
������ ������������~\cite{Bondarenko:1983}, 
%������ 3-������������, %������ �������� �~�����,
������ � 3-�����������,
������ ��������� �� �����
% ������ � �������� �~�������� ��������� 
�~��������� ������~\cite{BondBook:1995}. %, Shovgenov:2017}.
�~�� �� �����, �������� ����� ������������� ��� ��������� �������������� (��������������� � ������������� ����������� ��������): 
��� ����������� ��� ����� ����~\cite{Gaiha:1977}, ��� ������������� �������� �������� � ������ ����� �� $n$ �������� ����� $\lfloor n^2/4\rfloor$ ��� $n > 3$~\cite{Belov:1985}, ��� ���������� ��������������� �������������� ���������� ������ ������������ ������������ (������������ ������������� ���� ����)~\cite{BondBook:1995},
��� ������������� ������������� ������ ��������������� ����������� ������������~\cite{Nikolaev:2017}.

��� �������������� ����� ���� ����� ���������� ��� ��� �������,
����� �� ������� ������ ������������� ��������� �����������.
��������, ����������� ����������������� ��� ������������ ����� � ���������� ���� � ����������� ������� �~�����.
������ ��������, ��� ��� ������ ����������� ��� ������ NP"~������, � � ���~--- ������������� ���������.
� ����� ������� ������ ����� ������������� ��������������� ���� ��������� ��������� ������������ �������� ������ ������~\cite{Maksimenko:2004}.
� ���������, ��� ����� � ���������� ���� � ����������� ������� � ������������ ����������������� ����� �������� ��������� �������� ������� ��������������� ��������� $\ShortP(n)$ � $\MinCut(n)$ (� �������������, ��� ���� �� �������� ������� ����� ��������).
��������, ��� ��� ����� ���������� ����� �������� ����� ������ �������������� ���������������, � ��� ����������� �����������������~--- �������������~\cite{Maksimenko:2004, Nikolaev:2013}.

������ � ���, �������� ������� ������������� ���������� ����� � ���������������� $\omega(P)$.

� ������ �������, ��� �������������� ������������ ������ ������ ������������� ����������������, ������� �������������� H-�������� � ���������������� �������� ����� �����~\cite{Bondarenko:1987, Padberg:1989}.

������ ������ ������ � ������� ��������� ��������� ���������� �� ��������� ����� ����� �������.
��� ��������, ��� ������ ��������� ������������� ������������ ������� ���������� � ��������� �� ����� ����� ����� �������~\cite{Pan:2002, Sagraloff:2016}. (� ��������� �� ������������� ������������ ���������� $\Theta(d^2(d + \tau))$, ��� $d$ "--- ������� ����������, � $\tau$ ����� ������������ �� ���� �����: ����� ������ ������������� ���������� � �������� ����� ����� �������.)
�~������ �������, ��������� � ���� ������� ����������� ������������ $\CP_d(a,b)$ 2-���������� (������� �������, ��� ���� �����) ��� $d\ge 4$, � ����� ��� ������ ��������� � ������ ����� ����� ������� $[a,b]$.
�� ���� �������� ����� $\omega(\CP_d(a,b)) = b-a+1$ ��������� ���������������� ��������, ���� ����� ������� $[a, b]$ ���������������.
%������� �� ��������� ����������� �������������� � ������ ����, � �������~\ref{sec:cyclic}.



%%%%%%%%%%%%%%%%%%%%%%%%%%%%%%%%%%%%%%%%%%%%%%%%%%%%%%%%%
%
%     ���������� � ���������� �������������
%
%%%%%%%%%%%%%%%%%%%%%%%%%%%%%%%%%%%%%%%%%%%%%%%%%%%%%%%%%

\section{���������� � ���������� �������������}
\label{sec:ExtensionsAndRC}

\subsection{����������} %� ������ �����������}

������ ������, ������ ���������� H-�������� �������� �� ��� V"~�������� �������� �������������� �������~\cite{Khachiyan:2008}.
���� �� H"~�������� �������� ������ ��� ���� ������� �����, �� �������� ������ ������������� ����������� ������������� � ������ ��������� ����������������. ���������, ��� ��������~\cite{Khachiyan:1979, Karmarkar:1984}, ������������� ��������� ������������ ����������� (����� ����������), ����� ����������� (����������), � ������� ������������� (����� �����).

�� ��������, ��� �������, ���������� H-�������� �������� �������� ����������� ������� �������.
�� ������ ������� ������� �������� ��� ������ ������������� ���������� (��. ������~\ref{subsec:IdentFacet} ����).

\begin{comment}
������� �����, ��� ��� ������� �������� ������ ������������� ����������� �������� ��������� ���������������� �� ����������� ������� H-�������� ������������� ������.
� 1982 ���� ���� � ������������ ��������~\cite{KarpP:1982}, ��� ��� ������������ ������� ����� ������ ���������� ����� ����������� �������� ������� ������ �����������: ��� ��������� ������������� $P$ � ������� $\bm{v}$ � ������������� ������������
����������, ����������� �� ������ �������������
�, ���� �� �����������, �� ������������� �������������� (�������� �����������), ���������� $v$ �� $P$.
\end{comment}

\subsection{���������� ��������������}
\label{sec:Extension}

����� ��� �������������, ��������������� � �������� ������������� �����������, ����� ���������������� ����� �����������, ��� ������ ����������� ����������� ���������������� ������������� ������� ��������� ����������������. 
���� �� �������� � ������� ���� �������� ������� � �������� � ����������� ������������ �������������~\cite{Kaibel:2011,Conforti:2013}. 
 
\emph{�����������} ������������� (��������) $P \subseteq \R^d$ ���������� ������������ (�������) $Q \subseteq \R^n$ ������ � �������� ������������ $\alpha \from \R^n \to \R^d$, ��������������� ������� $P = \alpha(Q)$.
��� ��� ��� ��� ����� � ������ ������� �������� ������������� (��������) $Q$, �� ����� �� ����� ����� �������� ����������� ��������������� ���� ������������ (�������), ������������, ��� �������� ����������� $\alpha$ ������������� ���� �������� �����������.
%� ���������, ����� ������� �������� ����� �����������.
����� H-�������� ���������� ������ � ������������ $\alpha$ ���������� \emph{����������� ���������} ��� \emph{����������� �������������}.

�� ����������� ���������� �������, ��� ������ ����������� �������� ������� �� ������������� (��������) �������� � ������ ����������� �������� ������� �� ��� ����������.
� ������, ���� ������� $P\subseteq \R^d$ �������� ������� �������� $Q\subseteq \R^n$ ��� ����������� $\alpha(\bm{y}) = A \bm{y} + \bm{b}$, ��� $A \in \Q^{d\times n}$, $\bm{y} \in \Q^n$, $\bm{b} \in \Q^d$,
�� ������ ����������� �������� ������� $\bm{c}^T \bm{x}$ ��� ����������� $\bm{x} \in P$
������������ ����������� �������� ������� $\bm{c}^T A \bm{y} + \bm{c}^T \bm{b}$ ��� ����������� $\bm{y} \in Q$.

������������ �������� ���������� ������������� ����������� ������������ �������� ���������� $\Perm(n)$.\label{Perm2Birk}
��� H-�������� ��������� ��������~\cite{Rado:1952,ZieglerBook},
� ����� ����������� ����� $2^n-2$.
� ������ �������, �������� �������� (��. ����������� �� �.~\pageref{def:perm-birk}), ��� $\Perm(n)$ �������� �������� ��������� ������������� �������� $\Birk(n)$ ��� �������������� $x \mapsto x \cdot (1,2,\dots,n)$, ��� $x$ "--- $n \times n$-�������, � $(1,2,\dots,n)$ "--- ������"=�������.
(��~���� ������������ �������� �������� ����������� �����������.)
�������������, 
%��� ������ �������� ������� $\bm{c}\in\Q^n$
%�������� ��� $\bm{c}^T \bm{y}$ ��� $\bm{y}\in \Perm(n)$
%��������� � ���������� ��������� $\sum_{i,j} c_{i} x_{ij} j$ ��� $x \in \Birk(n)$. ��~���� 
������ �������� ����������� �� ����������� �������� �~������ �������� ����������� �� ������������� ��������.
��� ���� ��� H-�������� ���������� ��������� ����� $2n-1$ ��������� � $n^2$ ����������~\cite{Birkhoff:1946}.
����� ����, �� ��� �����~\cite{Goemans:2015} ��� ����������� $\Perm(n)$ ���� ������� ����������� ������������ � ������ ���������� $\Theta(n \log n)$ (����������� ��������� � ��������� �� ����������� ���������).

\emph{�������� ���������� (����������� ������������)}
���������� ����� ����������� (����� ���������� � H-��������) ����������.
����������� ���� ������ ����������, ��� ������ ���������� ����� ���� ����������� ������ ������� ��������� �������������. 
� ��������� ����� �������� ����� ������ �������� ����� �� ������������� ���������� ������� �� ���� �������� � ����������� ������������~\cite{Kaibel:2011, Conforti:2013}.

\emph{���������� ���������� (����������� ������������)} $\xc(P)$ ������������� (��������) $P$ ���������� ����������� ������ ����� ���� ��� ����������.
%��������~\cite{FioriniKPT:13}, ��� �������� $\xc(P)$ �� ���������, ���� � �������� ���������� ������������� ������ ������������ �������� (�������������).
��������� ���������� �������������� ����� ������ ����������� � ������ �������������� ���� ����������������, ��� ����������� �������������, ����� ��� ������ � �����������.

\begin{property}\label{prop:xc-base}
	����� $P$~--- �������� ������������, $\vertices(P)$~--- ����� ��� ������, $\facet(P)$~--- ����� ��� �����������, $\face(P)$~--- ����� ���� ������ (� ��� ����� �������������). �����
\begin{enumerate}
	\item $\xc(P) > \dim(P)$, ��� ��� ����� ����������� ������������� ������ ������ ��� �����������.
	\item $\xc(P) \le \facet(P)$.
	\item $\xc(P) \le \vertices(P)$, ��� ��� ����� ������������ �� $n$ �������� �������� �������� ������� ��������� $\Delta_{n-1}$.
	\item $\xc(P) \ge \log_2 \face(P)$~\cite{Goemans:2015}.
% ������� ������ ������������� ������������ � ������� ������ ��� ����������.
\end{enumerate}
\end{property}

����� ����, ��� �������������� �������� ��������� ������������ ���������,
����������� ������ ������ �� ������ ���������.

\begin{property}\label{prop:xc-compare}
	���� ������������ $Q$ ��� ���� �� ��� ������ �������� ����������� ������������� $P$, �� $\xc(P) \le \xc(Q)$.
\end{property}

�� ��������� ��������� ��� � ���� ����������� ���� �������� �������� ����� ����� �����������. ���������� �������� ���������� (� ��������� ��������� ������):
\begin{enumerate}
	\item ��� ���������������� ������������� $\xc(\Perm(n)) = \Theta(n \log n)$~\cite{Goemans:2015}.
	\item ��������� ���������� ������������� �������� �������� $\Tree(n)$ ���������� ������ ��������� $O(n^3)$, ��� $n$~--- ����� ������ �����~\cite{Martin:1991}.
	\item ��������� ���������� �������� �������� $\MinCut(n)$ ����� $O(n^3)$~\cite{Carr:2009,Conforti:2013}.
	\item ��������� ���������� �������� $s$-$t$ �������� ����� $\Theta(n^2)$, ��� $n$~--- ����� ������ �����~\cite{Garg:1995, Conforti:2013}.
	\item ����� ���������� � �������� �������� ���������� ������� $\ShortP(n)$ ��������� � ������ ��� � ��������������� ������� (��. ������~\ref{subsec:polyhedra}), ��~����~$\xc(\ShortP(n)) \le n(n-1)$.
	\item ����� ��� �����������, ��� ��� �������� ������������� �������� $\Birk(n)$ ���������� $n^2$ ����������. �~\cite{FioriniKPT:13} ��������, ��� ��� �������� ��� ���������� �������� ����� ���������� ����� ������������: $\xc(\Birk(n)) = n^2$.
	\item ��������� ���������� ��� ������������� ����������� ������������� � ��������� ������ �������������~\cite{Barahona:1993}.
	\item ���������� ��������� ��� ���������� ����������� ������������ �� ������ ����������� �������������� � ������������� ������������ ���������� ������������� ����������������~\cite{Kaibel:2010}.
	\item ��������� ���������� ��� ������ ������������� ������������� ���������������: $\xc(\BQP(n)) = 2^{\Theta(n)}$~\cite{FioriniPokutta:2015, KaibelW:15}.
	\item ��������� ���������� ��� ������������� ������ ������������ ���������������: $\xc(\TSP(n)) = 2^{\Omega(n)}$~\cite{Rothvoss:2014}.
\end{enumerate}
��� ������, �� �������� � ���� ������ �����������, ����� ����� � �������~\cite{Vanderbeck:2010, Kaibel:2011, Conforti:2013}.

������, ���������, ��� ��������� ���������� �������������� ������ ����������� � ������������ ������������� ��������������� � ��������� ��������������� �������� ����� ������������� �����������.
��� �� �����, ������� ��� ������ ������ ������������� ����������� �������� �������������� ��������� ������ � ��������� ���������� � �������������. 
��� ������������ ������ � �������������� � ������ ����� �� $n$ ��������.
��� � ������� � ��������� ����� �������� ��������� ��������� ���������� � ������������� $O(n^3)$, ��� $n$~--- ����� ������ �����~\cite{SchrijverCO:2003}.
�~������ �������, �~2014~���� ������� �������~\cite{Rothvoss:2014}, ��� ��������� ���������� ������������� $\Match(n)$ ���� ������ ����� $2^{\Omega(n)}$.

������� �����, ��� ��������� ����������� ������������ �� �������� ������������� ��������������� �������������. %(��. ���������~\ref{rem:combinatorial}).
���� ���� �������������� ��������� ������� ��������.
��������� ���������� ����������� $n$"~��������� ����� $O(\log n)$~\cite{BenTal:2001}.
��� ���� ���������� ������� (������������) $n$"~����������, ��������� ���������� ������� ����� $\Omega(\sqrt{n})$~\cite{Fiorini:2012polygons}.
� ������ �������, ������� ������ $n$"~��������� ���������� ������������ ������ ��� ������.

������ ��������� ������ ��� ��������� ���������� ������ ����� ��������� �� �������� \ref{prop:xc-base} � \ref{prop:xc-compare}.
���������������� ������� ������ ������������ ���������� ��������������� �������� ����������.
������ ������, ��� �������, ��������� �� ���������������� ������� ����������~\cite{Yannakakis:1991}, ��� ������������ ������� ��� ����������� ���� �����������. 

\begin{definition}\label{def:slack}
����� $X = \{\bm{x_1}, \dots, \bm{x_n}\}$~--- ��������� ������ (V-��������) ������������� $P \subseteq \R^d$, � ����������� $\bm{a_j}^T \bm{x} \le b_j$, $\bm{a_j}\in \R^d$, $b_j \in \R$, $j \in [m]$, ���������� ��� ���������� (H-�������� ��� ����� ���������).
������ $M(i,j)$ \emph{������� �������} $M \in \R^{n\times m}$ ��� �������� V-�������� � H-�������� ������������� $P$ ������������ ��������� �������:
\[
M(i,j) = b_j - \bm{a_j}^T \bm{x_i}.
\]
� ���������, $M(i,j) \ge 0$, �� ���� ������� $M$ \emph{��������������.}
\end{definition}

�� ����������� ����� �������, ��� ������� ������� $M$ ������� � �������� ���������� ������"=����������� $K \in \{0,1\}^{n\times m}$ ������������� $P$:
\[
K(i,j) = \begin{cases}
1, & \text{���� $M(i,j) = 0$},\\
0, & \text{�����}.\\
\end{cases}
\]

\begin{definition}\label{def:nonneg}
������������� ��������������� ������� $M \in \R_+^{n\times m}$ � ���� ������������ ��������������� ������ $T$ � $U$ ������� $n\times r$ � $r\times m$, ��������������, ���������� \emph{��������������� �������������}.
\emph{��������������� ������} $\rank_+(M)$ ������� $M$ ���������� ���������� ����������� $r$, ��� �������� �������� ����� ������������.
��������������� ���� ����� ���� ����� ��������� ��� ���������� $r\in\N$ �����, ��� $M$ ����������� � ���� ����� $r$ ��������������� ������ ����� ����.
\end{definition}

% ������ $\Q$ �� $\R$ ��� ������ $T$ � $U$ ������������ ����� ��������� ��������������� ����, �� �������������� ������������ ����������� ������������ ������ ����� �������� ������������� (��� �����, ��� �� �������� ��� � ����� �������) � ��� ���� ������������� �������� � ��������. �� ���� � ����� ������ �������� ������� ����� ��������������� � ������������� ������� �������������.

\begin{theorem}[���������~\cite{Yannakakis:1988, FioriniKPT:13}]
\label{thm:Yannakakis}
%\sloppy
��������� ���������� ������������� ����� ���������������� ����� ��� ������� �������.
\end{theorem}

��������~\cite{Conforti:2010}, ��� ����������� ����������� ����� ����� � ��� ���������� ��������� ��� ��������������� ������������� ������� ������� �������.

� ��������� ������� ��������������� ������������� ������ ���� ������� (�������~\ref{thm:rcxc}), ����������� ���������� ��������� ������ ������ ��������� ����������.

\subsection{����� �������������� ��������}
\label{sec:RectCover}

����� ��� ���� ��������, ��� ������� ������� ����� ������� � �������� ���������� �����������"=������.
� � ��������������� ������ ������� ������� ������� ����� �������������� �������� ������� ����������, ���������� � ��������� ������ ��� ������������� ��������.

\begin{definition}[\cite{FioriniKPT:13}]\label{def:rect}
����� $M \in \{0,1\}^{n\times k}$~--- ������� ����������.
��������� $I\times J$, ��� $I\subseteq [n]$, $J\subseteq [k]$, ���������� \emph{0"~���������������} �~�������~$M$, ���� $M(i,j) = 0$ ��� ���� $i\in I$ �~$j\in J$.
\emph{������������� ���������} ������� $M$ ���������� ��������� 0"~���������������, ����������� ������� 
��������� � ���������� ������� ����� �~$M$.
\emph{������ �������������� ��������} ������� ���������� ���������� ����� 0"~���������������, ����������� ��� � �������������� ��������.
����� �������������� �������� ������� ���������� ������"=����������� (�����������"=������) ������������� $P$ ���������� $\rc(P)$.
\end{definition}

� �������� ������� �� ���.~\ref{fig:8gon} ���������� ������� ���������� ���������������, � �� ���.~\ref{fig:8gonRect}~--- ����� ����������� � 0-���������������. ������ ���������� ����������� ���������������, �������� 6 �����������, ����������� �� ���.~\ref{fig:8gonEF}.

\newcommand{\paintentry}[4]{\node[fill opacity = #4] at ({#2 - 0.5}, {8.5 - #1}) {#3};}
% ������ ����
\newcommand{\paintzero}[2]{\paintentry{#1}{#2}{0}{1.0}}
% ������ 0-�������������
\newcommand{\paintr}[2]{%
	\foreach \i in {#1}{
		\foreach \j in {#2}
		{\paintzero{\i}{\j}}
	}	
}
% ������ �������
\newcommand{\paintone}[3]{\paintentry{#1}{#2}{1}{#3}}
% ������ ��� ������� ������� ���������� 8-���������
\newcommand{\paintones}[1][1.0]{\paintone11{#1} \paintone18{#1} \foreach \i in {2,...,8} {\paintone{\i}{\i}{#1} \paintone{\i}{\i-1}{#1}}}
% ����� ����������� ������� � ������������� �� ���������
\tikzset{eightgon/.style={scale=0.6, thin, line join = round, %baseline=-1mm,
baseline={([yshift=-\the\dimexpr\fontdimen22\textfont2\relax] current bounding box.center)}}}

\begin{figure}%[ht]
\centering
$\left(\tikz[eightgon]{
%	\node at (-0.1,4) {$\left(\rule{0pt}{2.6cm}\right.$}; 
	\paintones % ������ ��������
	% ������ ����
	\paintr{1}{2,...,7}
	\paintr{2}{3,...,8}
	\paintr{3}{1,4,5,...,8}
	\paintr{4}{1,2,5,6,7,8}
	\paintr{5}{1,2,3,6,7,8}
	\paintr{6}{1,...,4,7,8}
	\paintr{7}{1,...,5,8}
	\paintr{8}{1,...,6}
}\right)$
\caption{������� ���������� �����������"=������ ���������������.}
\label{fig:8gon}	
\end{figure}


\begin{figure}
	\newcommand{\paintrect}[2]{%
		$\left(\tikz[eightgon]{
			\paintones[0.5] % ������ ��������
			\paintr{#1}{#2} % ������ 0-�������������
		}\right)$
	}
	\centering
	\paintrect{1,...,4}{5,6,7}
	\quad
	\paintrect{5,...,8}{1,2,3}
	\quad
	\paintrect{4,5}{1,2,6,7,8}
	\\
	\paintrect{3,6}{1,4,7,8}
	\quad
	\paintrect{2,7}{3,4,5,8}
	\quad
	\paintrect{1,8}{2,...,6}
	\caption{����� 0-���������������, ����������� ������� ���������� ���������������. (������� ��������� ��� �������� ����������.)}
	\label{fig:8gonRect}	
\end{figure}

\begin{figure}
	\centering
	\begin{tikzpicture}[scale=2]
	\foreach \i in {0,1,4,5} {\coordinate (\i) at
		({cos(\i*45-45)},{sin(\i*45-45)});}
	\foreach \i in {2,3,6,7} {\coordinate (\i) at
		({cos(\i*45-45)},{sin(\i*45-45)});}
	\draw (7) \foreach \i in {0,...,7} {-- (\i)};
	\draw (0) -- (6) (1) -- (5) (2) -- (4);
	\draw[dashed] (3) -- (7);
	\end{tikzpicture}
	\caption{���������� ��������������� � $\R^3$ (��� ������), ������� 6 �����������.}
	\label{fig:8gonEF}	
\end{figure}

��������������� �� ����������� \ref{def:nonneg} � \ref{def:rect} �������, ��� ����� �������������� �������� ������� ���������� �� ����������� ���������������� ����� ������� �������.
����� �������, ����������� ���������

\begin{theorem}[���������~\cite{Yannakakis:1988}]
	\label{thm:rcxc}
	$\xc(P) \ge \rc(P)$ ��� ������ ��������� ������������� $P$.
\end{theorem}

�������� �����, ��� ������ ������ � ��������~\ref{prop:xc-base} ����� � ��� ����� �������������� ��������.

\begin{property}[\cite{FioriniKPT:13}]\label{prop:rc-base}
	����� $P$ "--- �������� ������������, $\face(P)$ "--- ����� ���� ��� ������. �����
	\[
	\dim(P) + 1 \le \log_2 \face(P) \le \rc(P).
	\]
\end{property}

����� ����, ��������~\ref{prop:xc-compare} ��� ������ $\xc(\cdot)$ �� $\rc(\cdot)$ ���� �������� ������.
�~����� ������ ������� ����� �������������� �������� ������� ���������� �����������"=������ ������������� ����� ������������ �~\cite{FioriniKPT:13}.

����������� ��� ��������� � ��������� ����� ������ ������ ��������� ���������� �������������� �������� � �������������� �������~\ref{thm:rcxc} � (���������� �������������) ������� \ref{prop:rc-base}, \ref{prop:xc-compare}.
������������ �������� ��������� ��� �������:
\begin{enumerate}
	\item ��� ������������� ������������� $\Match(n)$ ����������� ���������������� ������ ������ ��������� ����������~\cite{Rothvoss:2014}.
	��� ���� $\rc(\Match(n)) \in [n^2, n^4]$, ��������~\cite{FioriniKPT:13}.
	\item �~\cite{Fiorini:2012polygons} ���������� ���� ������������� $n$"~����������, ��������� ���������� ������� �� ������ $\sqrt{2n}$, ����� ���
	����� �������������� �������� ������� ���������� $n$"~��������� ��������� � ��������� �� $\log_2 (2n)$ �� $2\log_2 (2n)$ (��� ������� ��~\cite{BenTal:2001, Fiorini:2012polygons}).
	\item �������� ������������� �������� �������������� ��������� � ���������������� ���������� ����������~\cite{Rothvoss:2013} (� �������������� ������������ ��� ����, ��� ����� ��������� ��������� ������ ��������������� ������������ ����� ��������� ���������"=��������~\cite{Dukes:2003}). � �� �� ����� ����� �������������� �������� ��� ��� �� ��������� �������� �� ����� ��������� ���������"=��������~\cite{Kaibel:2016}.
\end{enumerate}
�������, ��� ������ � �������������� �������� ������������� ����������, � ����� ��������������� � ��������� ���� ������ ������������� ���������� ����� (��������, ����������� �� �������� ����������� $n$"~��������� � ����������� �� �������� �������� ��������).
����� �������, ��� ��������� ����� ������� � ���, ��� ����� �������������� �������� ������� ���������� ������"=����������� ������������� ���� ������ ������ ������ ������ ��������� ��������������� ��������������� ������.
%(������������� ������������� ����������� ����� �������� ���� ����������.)



\section{�������}
\label{sec:questions}

%{\color{red}������ �����. ������� ����� ������ ���������!}

%����� $S = \{P(n)\}$ "--- ��������� ��������� ��������������, ������ �� ������� ����������� ��� �������� �������� ���������� ��������� $X_n \subset \R^d$, $n\in\N$, $d = d(n)$. 
%� ���� ���������� ������� ��������������� ������ OPT(S): ��� ����� $n\in\N$ � ������� ������ $\bm{c} \in \R^d$; ��������� ����� $\max \Set*{\bm{c}^T \bm{x} \given \bm{x} \in P(n)}$.

��� ������� �� ������������� ���� ������, ������������� NP"~������� ����� �� ������ ������� �������� ������� ����������. ��������: NP"~������� ������ ������������� ����������� ������, ��������� ������� �����, ������������������� �������� ����� �����, ������������������� ��������� ���������� � ����� �������������� �������� ������� ���������� ������"=�����������.
����� ��� �������� ����������� ������� ������� ��������������� ���������, ��������������� � ������ ����� ������� ���������������. 
���������� ��������� ��������� ��������:
\begin{enumerate}
\item ����� ������������ ������������ $\BQP(n)$ ������� ������������ ������������� �������� $\Cut(n-1)$~\cite[������~5.2]{Deza:2001}.
\item ������������ ����������� ������������� $\Match(2n)$ ���� ��������
��������� ����� ������������� ������ ������������ $\TSP(6n)$~\cite{Yannakakis:1991}.
\item ��� ������� ����� $G = (V,E)$ ������������ �������� ������ $\Stable(G)$ �������� ��������� ��������� ����� ������������� ������������� ������ ������������ $\ATSP(n)$, ��� $n = |V| + 4|E|$~\cite{Yannakakis:1991}.
\item ������������ $\ATSP(n)$ �������� ������ ������������� $\TSP(2n)$~\cite{Junger:1995TSP}.
\item �������~\cite{Fiorini:2003} �������, ��� ������������� ������ � 3"~������������ � ������������� ������ � ��������� �������������� �������� ������� ���� ����� ��� ���������� ������ ���� ������������� (������ ������������ ��. � �������~\ref{subsec:k-Sat&POP}).
\item ���� � ������~\cite{AvisTiwary:2015} ��������, ��� ������������ ������ � 3"~������������ �������� ��������� ����� ������������� ������������� ������ � ����������� (��������� "--- � �������~\ref{sec:3Ass}).
\item �������, ������ � ����~\cite{Buchheim:2010} ��������, ��� ������������ ������������ ������ ��������� �������������� ������� ������������ ����� ������ ������������� ������������� (��. ������~\ref{sec:QLOP}).
����������� ���� ��� ������������� ������������ ������ � ����������� ���������� ���������� ����������� ��������~\cite{Rijal:1995, Kaibel:1997, Saito:2009}.
\end{enumerate}
� ����� � ���� ������������� �������� ��������� ������� ������ ���������. ����� �� �������������� ������������ ����� ������ ��������� ��� ��������� (������) �������� ��������������? 
����� ������ �� ������ ��������� ������ ���� ����� ������� � ��������� ��������� ������������"=�������������� ������������� ��������������?
������ �� ��� ������� ���������� � ������ \ref{chap:AffTheory}--\ref{chap:ExtAff}.

%� ��� �������� ����� ���������� ��������� ������������� �������� ��������������, ��� ������� ������ ������������� ����� NP"~������? ��� ��� ���������� �� ��������, ��� ������� ��� ������ ������������� ���������?

���������� ������������� � ���� ����� �������������� ��������������:
��������� ������������� ����� (�������, �����, ����������), ����� ������, ����� �����������, ������� �����, �������� ����� �����, ��������� ����������, ����� �������������� �������� ������� ���������� ������"=�����������.
��� ������ �� ��� ����������� ������ ��������� ������.
���� �� ����� ����� ������ ��������������� ������������� � ���������� ��������������� ��������������� ������?
����� ����, ��������� ������� ����� ������ ���������.
����� ��������� � ��������� ����� ������������"=�������������� �������������� ������������� �������� ��������� �������� ��������� ��������������� ������?
���� �� ����� ����� ������������� ����� ������������� � ���������� ������ ����������� �� ���?
������������ ���� ������� ��������� ����� \ref{chap:Direct} �~\ref{chap:Counterexamples}.

%% Глава 3
%%%%%%%%%%%%%%%%%%%%%%%%%%%%%%%%%%%%%%%%%%%%%%%%%%%%%%%%%%
%
%     �������� ����������
%
%%%%%%%%%%%%%%%%%%%%%%%%%%%%%%%%%%%%%%%%%%%%%%%%%%%%%%%%%%
\chapter{�������� ����������}
\label{chap:AffTheory}
%\begin{flushright}
%�� ��������� � ���������.
%\end{flushright}
%\medskip

�~���� ����� ������� ������ �������� ����������, ������� ������������ � ��������� ���� ������.
�~�������~\ref{sec:AffProblems} �������� ����������� �������� ���������� �����, �������� ������� ��� ��������� ��� ����������� � ����������� ��������.
�~�������~\ref{sec:Cones} ���������� ����������� ��������� ��������� ������������ �������� ������ ������ � �������� ���������� ����� ���������.
�����, � �������~\ref{sec:AffCompare} �������� ������������ ������ ��������� ��������������, ����������� ���������� �� ��������� ������������"=�������������� �������������� ���������, � ���������� ������� ��� ���������� ��� ���������� ������������� �������� ��������������.
�~��������� ������� ����� �������� ����������� �������� ���������� �������� �������������� (�������� ����� ������������ � ��������� �����), ���������� ������������ ������� ������� ��� ������������� � ��������������� ����� ����� �������� ����������� �������� �������������� � �������� ����������� �������� ��������� ����������� �������� ������ �����. 

\section{�������� ���������� �����}
\label{sec:AffProblems}

�������� ������������ \ref{def:COP}, \ref{def:LCOP} � \ref{def:family}, ������� ������ �������� ������ ������������� ����������� ������� �� ���� $I$, ����������� ����� ������������� (������������� ��������� ���������� ������� $X(I)$), � �������� �������~$\bm{c}$.

\begin{definition}[�������� ���������� �����]
\label{def:AffReduction}
���������� ��� �������� �������� ������ ������������� �����������.
��� ���� ������������� � �������� ������� ������ ������ ���������� ����������� $I$ � $\bm{c}$, ��������������. ��� ������ ������ "--- $I'$ � $\bm{c'}$.
������ �������, ������������ ��������� ���������� ������� $X = X(I) \subset \Z^d$ ������ ������, ���������� $d$, $S$ � $g$. ��� ������ ������ ���������� ����������� $X' = X'(I') \subset \Z^{d'}$, $d'$, $S'$ � $g'$, ��������������. 
%��������� ���������� ������� ������ ������ ���������� $X' = X'(I')$.

����� ��������, ��� �������� ������ ������������� ����������� $(d,S,g)$ \emph{������� ��������} � ������ $(d',S',g')$, ���� ���������� ���������� �� �������������� (������������ ����� ������� ������ $(I,\bm{c})$ ������ ������) �����:
\begin{enumerate}
	\item 
	�������������� $\tau$ ������� ���� $I$ ������ ������ � ��� $I'$ ��� ������ ������: \[\tau \from I \mapsto I'.\]
	\item 
	�������� ���������� ��� ������� ���� $I$ ��������� ����������� 
	\[
	\alpha\from \R^d \to \R^{d'}, \qquad \text{��� } d = d(I), \quad d' = d'(\tau(I)).
	\]
	\item 
	%���������
	������� $\beta\from Y \to X$, ��� $X = X(I)$, � $Y$ "--- ��� ��������� ���� ����� $\bm{y} \in X' = X'(\tau(I))$, ��� ������� �� ������� �������� ������� ������ $\bm{c} \in \R^d$ �����, ��� $(\alpha(\bm{c}))^T \bm{y} \ge (\alpha(\bm{c}))^T \bm{x'}$ ��� ���� $\bm{x'} \in X'$.
	%\[Y = \Set*{\bm{y} \in X'(I') \given \exists \bm{c} \in \R^d \ \bm{c'} = \alpha(\bm{c}), \ \forall \bm{x'} \in X'(I')\ \bm{c'}^T \bm{y} \ge \bm{c'}^T \bm{x'}}.\]
		
	������ ��� ������ $\bm{y} \in Y$ � ������ $\bm{c} \in \R^d$ ���������
	\[
	\Bigl(\forall \bm{x'} \in X' \quad (\alpha(\bm{c}))^T \bm{y} \ge (\alpha(\bm{c}))^T \bm{x'}\Bigr) \iff \Bigl(\forall \bm{x} \in X \quad \bm{c}^T \beta(\bm{y})  \ge \bm{c}^T \bm{x}\Bigr).
		\] 
\end{enumerate}
\end{definition}
%������: 1) ��������� -- ����������� ��, ��� �� �������, ����������� ��� ��������� ��������; 2) �������� -- ������ ����� ������������� (�� ����� ���) ������������� ���������, ���, ���� ������ ���������, �� � ��������� ���������.

��������������� �� ����������� �������, ��� �������� ���������� ����� ������ �� �������������� ����������, ��� ��� �������������� $\tau$ � �������� ����������� $\alpha$ ����������� ������� ������ ������ ������ �� ������� ������ ������ ������ �� �������������� �����, � ������� $\beta$ ����������� ����������� ������� ������ ������ � ����������� ������� ������ ������.

������������� ����������� �������� ���������� ����� ��� ������� � ������������ �� ��������� ������� �������� (��������, ��� ������ � ���������� ����) ��������.
� ������, ���������� ���� ����� �� �����������~\ref{def:AffReduction} �� ���������� �������������. 
�����������, ��� ������� ������ ������ ������ ������ ������������� ��������� $p\from \R^d \to \{\text{����},\text{������}\}$, ��� $p$ ������� �� ���� $I$, � ������� ������ ������ ������ "--- ��������� $p'$. 

\begin{definition}[�������� ���������� ����� � ������������]
\label{def:AffReductionRestriction}
{\sloppy
��� �������� ������ ������������� ����������� $(d,S,g)$ � ������������ $p$ ������ ����������� ��� ��������� �������� ������:
\[
Q = Q(I) \coloneqq \Set*{\bm{c} \in \R^d \given p(\bm{c})}.
\]
� ��������� �������� ������ ������ $(d',S',g')$ � ������������ $p'$ ���������� $Q' = Q'(I')$.
}

����� ��������, ��� ������ $(d,S,g)$ � ������������ $p$ \emph{������� ��������} � ������ $(d',S',g')$ � ������������ $p'$, ���� ���������� ���������� �� �������������� ������������ ������� ������� ������ ������ �����:
	\begin{enumerate}
		\item 
		�������������� $\tau$ ������� ���� $I$ ������ ������ � ��� $I'$ ��� ������ ������: \[\tau \from I \mapsto I'.\]
		\item 
		�������� ���������� ��� ������� ���� $I$ ��������� ����������� 
		\[
			\alpha\from Q \to Q', \qquad \text{��� } Q' = Q(\tau(I)).
		\]
		\item 
		%���������
		������� $\beta\from Y \to X$, ��� $X = X(I)$, � $Y$ "--- ��� ��������� ���� ����� $\bm{y} \in X' = X'(\tau(I))$, ��� ������� �� ������� �������� ������� ������ $\bm{c} \in Q$ �����, ��� $(\alpha(\bm{c}))^T \bm{y} \ge (\alpha(\bm{c}))^T \bm{x'}$ ��� ���� $\bm{x'} \in X'$.
		%\[Y = \Set*{\bm{y} \in X'(I') \given \exists \bm{c} \in \R^d \ \bm{c'} = \alpha(\bm{c}), \ \forall \bm{x'} \in X'(I')\ \bm{c'}^T \bm{y} \ge \bm{c'}^T \bm{x'}}.\]
		
		������ ��� ������ $\bm{y} \in Y$ � ������ $\bm{c} \in Q$ ���������
		\[
		\Bigl(\forall \bm{x'} \in X' \quad (\alpha(\bm{c}))^T \bm{y} \ge (\alpha(\bm{c}))^T \bm{x'}\Bigr) \iff \Bigl(\forall \bm{x} \in X \quad \bm{c}^T \beta(\bm{y})  \ge \bm{c}^T \bm{x}\Bigr).
		\] 
	\end{enumerate}
\end{definition}

����� ��� � ������ ������� ����� ������������ �������� �������������� ����� � ��������� � ���� �������������� �����������. ������� ������ ����������� ����� �������������� � ������������ � ������ �������� ��������������� �������������� ��������.

�~��������� ������ ����������� ����� �������� ����������� ����� � �������� ����������� �������������� �������� �������� ���������� �������� ��������� ����������� �������� ������ �����.

%%%%%%%%%%%%%%%%%%%%%%%%%%%%%%%%%%%%%%%%%%%%%%%%%%%%%%%%%%
%
%     �������� ���������
%
%%%%%%%%%%%%%%%%%%%%%%%%%%%%%%%%%%%%%%%%%%%%%%%%%%%%%%%%%%
\section{�������� ��������� ������������ �������� ������}
\label{sec:Cones}

����� $P$ "--- �������� ������������ � $\R^d$. 
\emph{����� �����} $F$ ������������� $P$ ��������� ��������� �������:
\[
K(F) \coloneqq \Set*{\bm{c}\in\R^d \given  \bm{c}^T \bm{x} \ge \bm{c}^T \bm{y}, \ \forall \bm{x} \in F, \ \forall \bm{y} \in P}.
\]
�~���������, $\bm{0} \in K(F)$ ��� ����� ����� $F$.
������� �����, ��� ����������� ������ ����� �� ���������, ���� ������������ $P$ � ��� ����� $F$ �������� ���������������� ����������� ������.

��������� ���� ������� $K(F)$, ��� $F$ ��������� ��������� �������� ������ ������������� $P$, ���������� \emph{���������� ������}~\cite[�.~257]{ZieglerBook}.

���������� ���� �������� ������������ � ������������� �������������� ����������.
� ������, $\dim K(F) = d - \dim(F)$ �, ���� $F$ �������� ����������� ������ ����� $G$ ������������� $P$, �� $K(G)$ �������� ������ ������ $K(F)$.
�������, ����� ����� ������ $K(F)$ ���� ����� $K(G)$, ��������������� ��������� ����� $G$ ������������� $P$ �����, ��� $F \subset G$.

��� ��� ����� �������� ������������ $P$ ���������� ������������ ���������� ����� ������ $X = \ext(P)$, �� � ��� ���������� ���� ���������� ������������ ������� ������� ��� ������
\[
\K(X) \coloneqq \Set*{K(\bm{x}) \given \bm{x} \in X},
\]
������� ����� ����� �������� \emph{�������� ���������� ������������ �������� ������} ������ �������� ����������� �� ��������� $X$ ���, ������, \emph{�������� ����������} $\R^d$ �� ��������� $X$~\cite{BondBook:1995}.
��� �������� ����������� ���, ��� ������ ����������� ����� �������� ������� $\bm{c}\in\R^d$ �� ��������� $X \in \R^d$ ����� ����������������� ��� ������ ������ ������ $\bm{x} \in X$, ��� $\bm{c} \in K(\bm{x})$.

����� $\bm{x}$ � $\bm{y}$ "--- ������� ������������� $P \subset \R^d$.
����� �������� ������ $K(\bm{x})$ � $K(\bm{y})$ \emph{��������}, ���� ��� ����� ����� ����������:\label{AdjCones}
\[
\dim (K(\bm{x}) \cap K(\bm{y})) = d-1.
\]
��������, ������ $K(\bm{x})$ � $K(\bm{y})$ ������ ����� � ������ �����, ����� ������ ������� $\bm{x}$ � $\bm{y}$.
��������������, ���� ������������� $\conv(X)$ ��������� � ������ ��������� ��������� $\K(X)$.

������������ ������������� ������ $\conv(X)$ ������������ ��������� ��������� $\K(X)$ ������������ �������� ������ ����������� � ���, ��� ������� ����������� ������� ������������� ��������� �������������� �������� ����������� �� ������� ������ $\bm{c}$.
���, ��������, � ������������ ������ � ���������� ���� ������������� ����������� $\bm{c} \le \bm{0}$ (��� ������ ������������).
�� ���� ������ ��������� ��������� ����� ������������ � ���� ������ ��������������� �������� ��������� �������������� �������.
�� ����� �������������� ��� �������� ������� �� ������������� ����� $\Path(n)$ � �������� $\Path^{\uparrow}(n)$ (��. ������~\ref{subsec:polyhedra}).

�� �������� � ������������ ������ $K(\bm{x})$ ��� ������� $\bm{x}$ ������������� $P\subset \R^d$ ������ �����������
\[
K(\bm{x}, Q) \coloneqq \Set*{\bm{c}\in Q \given  \bm{c}^T \bm{x} \ge \bm{c}^T \bm{y}, \ \forall \bm{y} \in P} = K(\bm{x}) \cap Q, 
\]
��� $Q$ "--- ������� � $\R^d$.
��������� ���� ����� ������� ��� ������������� $Q$ ��������� $\K(X,Q)$ � ����� �������� \emph{���������� $Q$ �� ��������� $X$}~\cite{Maksimenko:2004, BondBook:2008}.
���� ������ ��������� ������������ �� �������� � ������ ��������� ��������� ����� ������������ $\R^d$:
\begin{enumerate}
	\item $\bm{x} \in X$ �������� �������� \emph{����� ���������} $Q$ �� $X$, ���� \[\dim (K(\bm{x}, Q)) = \dim Q.\]
	\item ������� $\bm{x}$ � $\bm{y}$ ����� ����� \emph{������}, ���� 
	\[\dim (K(\bm{x}, Q) \cap K(\bm{y}, Q)) = \dim Q - 1.\]
\end{enumerate}
����� �������, ���� ��������� $\K(X,Q)$ �������� ��������� ����� ��������� ��������� $\K(X)$, ���� ����������� �������� $Q$ ��������� � ������������ ����� ������������ $\R^d$.

��������, ��� ��� �������� �� ����� ������������ �������� ������ � �������� $Q$ ������ ����������� ����� ����� ������ ���� �������������� ��������� (����������� � ��������) � ����������� �� �����������, ������ �� ������� ������������. � ����� ������ ������������� ���������, ���������� ���� ��� ���� ����� ������ ����� ����� ���������, ����� �������� ��������� �� �����.

\begin{remark}
��� ��� ������� ������ ����������� ����������� ������������ ��������� �������� ������� �� ������������� ������, ��, �� �������� ��������, ������ ��������� ��������� $\K(X)$, $X \subset \R^d$ ����� ������������ ������������� ��������� $\K(X,\Cube(-1,1))$, ��� 
\[
\Cube(-1,1) \coloneqq \Set{\bm{x} \in \R^d \given \bm{-1} \le \bm{x} \le \bm{1}}.
\]
%�~���������, ����� ���� ��������� ���������.
\end{remark}


%\section{�������� ����������}

\begin{definition}[�������� ���������� ��������� �������� ������]
\label{def:ConesReduction}
\sloppy
����� ��������, ��� ��������� ��������� �������� ������ ������ $(d,S,g)$ � ������������ $p$ \emph{������� ��������} � ��������� �������� ������ ������ $(d',S',g')$ � ������������ $p'$, ���� ������ ������ ������� �������� �� ������ �, ����� ����, �������� ����������� $\alpha$ � ������� $\beta$ � �����������~\ref{def:AffReductionRestriction} ���������.
\end{definition}

�������� ��������, ��� ���� ������� ����� ����������� ���������, �� �������� ����������� $\alpha$ ������������� �������"=����������� ������������ ����� ���������� ��������� �������� ������ ������ ������ � ��������� ������ ��������� �������� ������ ������ ������. ������ �������, � ���������, ��� ���� ��������� ������ ������ �������� ��������� ����� ��������� ������.

�~�������� �������� ������� ������ ���� ���������� ���������� ������ ������������ � ������������ $\bm{-1} \le \bm{c} \le \bm{1}$ � ��� �� ������, �� � ������������ $\bm{c} \ge \bm{0}$. ��� ��� �������� ������� ������� ��� ������ ������������ ������������ ����� ����� ����� $n$ ��������� �������� ������� ($n$ "--- ����� �������), �� ���������� ������ � ���� �� ������� �� ���� ����������� �� ������� ������� ������.
�������������, ����� ������� ������ ��������� ������ ������������ � ������������ $\bm{-1} \le \bm{c} \le \bm{1}$ � ������ � ������������ $\bm{c} \ge \bm{0}$, ���������� �������� $\alpha \from \bm{c} \mapsto \bm{c} + \bm{1}$.


%%%%%%%%%%%%%%%%%%%%%%%%%%%%%%%%%%%%%%%%%%%%%%%%%%%%%%%%%%
%
%     ��������� ��������������
%
%%%%%%%%%%%%%%%%%%%%%%%%%%%%%%%%%%%%%%%%%%%%%%%%%%%%%%%%%%

\section{��������� ��������������}
\label{sec:AffCompare}

��� ��������, ��������� �������� ����� �� �������� ���������������� � ������������� ������� ������������.
����� ��������������� ��, ������ ��������� �������� ���������,
�������� ��������� ������� �� ��������� ���� �������� ��������������.

\begin{definition}\label{def:ineA}
�~������, ����� ������������ $P$ ������� ������������ �������������~$Q$ ��� �� ��� �����, ����� ������������ ����������� $P \lea Q$.
���� �������� ��������������� �������������� $P$ � $Q$ ���������� $P =_A Q$.
\end{definition}

�~�����~\ref{chap:ExtAff} ����� ���������� ����� ������ ������� ����� �����������,
� ������� ����� <<������� ������������>> �������� �� <<�������� �������� �������>>, � ��� ���������������� ����������� ������������ ����������� $\lee$. ��� �� ����� �������� ������ �������� � ������������� ���� ���� ����������� ��� ������������ ������� ��������������.

�������� ��������, ��� ����������� $\lea$ ����������� �������� ��������
��� ��������� ������������� ������� (�.\,�. ������� ������� ������) ��������������.

\begin{property}
���� $P \lea Q$, �� ������� ������ ������������� $P$ 
��������� ���� ���� ������� ������ ������������� $Q$ (���� $P$ � $Q$ ������������), ���� ��������� ���������� (�������������� ������ ������������� $Q$), � ������� ���������� ������"=����������� ������������� $P$ �������� ����������� (���������� ��������� ����� � ��������) ������� ���������� ������������� $Q$. �~���������:
\begin{enumerate}
	\item ����� ������ ������������� $P$ �� ����������� ����� ������ $Q$.
	\item ����� $i$-������ ������������� $P$ �� ����������� ����� $i$-������ ������������� $Q$ ��� $i \le \dim(P)$.
	\item ���� ������������� $P$ ��������� ���������� �������� ����� ������������� $Q$.
	\item ����� ����������� $P$ �� ����������� ����� ����������� $Q$.
	\item ��� ����� ������������� �������� ������ ���������� ������"=����������� ��������� $\rc(P) \le \rc(Q)$.
\end{enumerate}
\end{property}

�~�������� �������� ���������� ��������� ��������� �����������
��� ���������� � ����������� ��������������.
������ �����, �������� $\Delta_n$ �������� ������ ��������� $\Delta_{n+1}$, 
� � ���� �������������� ����������� $\lea$ 
�������� 
\begin{equation}
\label{eq:compareDelta}
\Delta_n \lea \Delta_{n+k} \quad \forall n,k \in \N.
\end{equation}
����� ����, 
\begin{equation}
\label{eq:compareDeltaCP}
\Delta_m \lea \CP_n(S) \quad \text{��� } m < n \le |S|, 
\end{equation}
��� ��� ����������� ������������ $\CP_n(S)$ ������������.
�� ��� �� ������� 
\begin{equation}
\label{eq:compareCP}
\CP_n(S) \nelea \CP_{n+k}(S')
\end{equation}
��� ���� $n, k \in \N$
� ����� �������� $S$ � $S'$, ��� ������� $|S| > n+1 > 2$.

�������, ��� ��������� ���������� $\Delta = \{\Delta_n\}$ � ��������� ����������� �������������� $\CP = \Set*{\CP_n(S) \given n\in\N, \ S \subset \Q, \ |S| < \infty}$ ������������ ����� ������� �������� �������������� ����� ������������� �����������.
�~����� � ���� ������� ������������ ��������� ������.
����� �� ���� �������� \eqref{eq:compareDelta} ���~\eqref{eq:compareCP} �������� �������� �������� ��� ��������� �������� �������������� �����?
���� ����������, ��� � ����������� ������� ����� ����������� �������������� ����������� ����~\eqref{eq:compareDelta}.
��� �������������� ������ ���� ����������� ������ ������������ ��������� ��������� ������������.

\begin{lemma}
	\label{lem:01face}
	����� $P \subset \R^d$ "--- 0/1"~������������. ����� $F_i = \Set*{\bm{x}\in P \given x_i = 0}$ � $G_i = \Set*{\bm{x}\in P \given x_i = 1}$, $i\in[d]$,
	�������� ������� (���� ����� ��������������) ������������� $P$.
\end{lemma}

���������� ��� �������� ����� ������������� � ���������� ��������� ��������������: ������ ������������ ������������� $\BQP(n)$, ������������� ������������� ������ ������������ $\ATSP(n)$ � ������������� ������ � ������� $\Knap(n, \bm{a}, b)$.


\begin{prop}
\begin{align*}
\BQP(n) &\lea \BQP(n+1),\\
\ATSP(n) &\lea \ATSP(n+1),\\
\Knap(\bm{a}, b) &\lea \Knap((\bm{a},0), b), \quad \bm{a} \in \R^{n}, \ b\in \R.
\end{align*}
\end{prop}
\begin{proof}
������������� ������~\ref{lem:01face}.
���������� ����� $F$ ������������� $\BQP(n+1)$, ������������ ��������������� $x_{n+1, n+1} = 0$. 
����� ��� ���� $\bm{x} \in F$ ����������� $x_{i, n+1} = 0$ ��� $i\in[n+1]$.
�������� �������, ��� ��������� ������ ����� $F$ ������������� � ��������� ������ ������������� $\BQP(n)$ (�, ��������, $\BQP(n)$ � $F$) �������� ������������ (������������� ���������) $x_{ij} \mapsto y_{ij}$, $1 \le i \le j \le n$. 

��� �������������� ����������� $\ATSP(n) \lea \ATSP(n+1)$ 
���������� ���������� ������� ����������� ������������ ����� ���������� ������������� �������� ������� ������� $D=(V,A)$ �� $n$ �������� � ������������� 
������������� �������� ������� $D'=(V',A')$ �� $n+1$ ��������, � ������� ������� ��� ����, �������� � ������� $v'_1$, � ����, ��������� �� $v'_{n+1}$, �� ����������� ���� $(v'_{n+1}, v'_1)$.
�������, ��� ������������������ ������� ���������� ������������ �������� ������ $D'=(V',A')$ �������� ��������� ����� 
\[F = \Set*{\bm{x} \in \ATSP(n+1) \given x_{(v'_{n+1}, v'_1)} = 1}
\]
������������� $\ATSP(n+1)$.
��������, ����� $F$ � ������������ $\ATSP(n)$ ������� ���������� �������� ������������
\[
	y_{(v_i, v_j)} = 
	\begin{cases}
	x_{(v'_i, v'_j)}, & \text{��� }1 \le i \le n, \ 2 \le j \le n, \ i \ne j,\\
	x_{(v'_i, v'_{n+1})}, & \text{��� }2 \le i \le n, \ j=1,\\
	\end{cases}
\]
��� $\bm{x} \in F$, $\bm{y} \in \ATSP(n)$.

��� �������������� ����������� $\Knap(\bm{a}, b) \lea \Knap((\bm{a},0), b)$ ���������� ��������, ��� 
\[
\Knap((\bm{a},0), b) = \Set*{(\bm{x},x_{n+1}) \in \{0,1\}^{n+1} \given \bm{x} \in \Knap(\bm{a}, b)}.
\]
�������������, ������������ $\Knap(\bm{a}, b)$ ������� ������������ ����� ������������� $\Knap((\bm{a},0), b)$, ������������ ��������������� $x_{n+1} = 0$.
% (��� $x_{n+1} = 1$).
\end{proof}

����������� \eqref{eq:compareDeltaCP} �������� ������� �������� ��������� �������������� �� ������ ��������.
��� ����� ����� ������, �������� ���������� ����������� ��������~\cite[�.~84]{Deza:2001}, �������� \emph{������������ �����������} $\xi\from \BQP(n) \to \Cut(n+1)$, ���������� �����������
\[
y_{ij} = 
\begin{cases}
x_{ii}, & \text{��� } 1 \le i \le n, \ j=n+1,\\
x_{ii} + x_{jj} - 2 x_{ij}, & \text{��� } 1 \le i < j \le n.\\
\end{cases}
\]
%��� $\bm{x} \in \BQP(n)$, $\bm{y} \in \Cut(n+1)$.
�� ��������������� ����� ����������� �������
\[
\BQP(n) =_A \Cut(n+1).
\]

���������� ��� ��� ��������� ��������������, ��������������� � �������� �� �������� � ��������� ���������.

����� $A\in\{0,1\}^{m\times n}$ "--- ������� ���������� ��������� ��������� $G$, $G = \{g_1, \ldots, g_m\}$, � ��������� ���������� ��������� $S$, $S = \{S_1, \ldots, S_n\} \subseteq 2^G$.
�������� \hypertarget{def:Pack}{��������} ���������
\begin{equation*}
\Pack(A) = \Set*{\bm{x}\in\{0,1\}^n \given A \bm{x} \le \bm{1}}
%\quad \text{��� } A\in\{0,1\}^{m\times n},
\end{equation*}
���������� \emph{�������������� �������� ���������}~\cite{Balas:1976}.
(������ ������� $\bm{x} \in \Pack(A)$ ����� ������������� �������� ������������������ �������� ��������� �������� $T \subseteq S$.)

��������� ������ \emph{������������� ��������� ���������} ������������ �� ��������:
\begin{equation}
\label{eq:Part}
\Part(A) = \Set*{\bm{x}\in\{0,1\}^n \given A \bm{x} = \bm{1}}.
\end{equation}
��������������� �� ����������� �������
\begin{equation}
\label{eq:PartPack}
\Part(A) \lea \Pack(A).
\end{equation}

������� �����, ��� ������������ ����������� �������� $\Stable(G)$ (��. ����������� �� �.~\hyperlink{Stable}{\pageref*{def:Stable}}) �������� ������� ������� ������������� �������� ��������:
\begin{equation}
\label{eq:StablePack}
\Stable(G) =_A \Pack(A),
\end{equation}
��� $A$ �������� �������� ���������� �����"=������ ����� $G$.


%%%%%%%%%%%%%%%%%%%%%%%%%%%%%%%%%%%%%%%%%%%%%%%%%%%%%%%%%%
%
%     �������� ���������� ��������������
%
%%%%%%%%%%%%%%%%%%%%%%%%%%%%%%%%%%%%%%%%%%%%%%%%%%%%%%%%%%

\section{�������� ���������� ��������������}
\label{sec:AffReductPolytopes}

%\subsection{��������� �������� ��������������}

������, ��� ������� � ��������� �������� ��������������, ��������� ��� ��������� ������� ����������� ����� $\Stable(G)$, $\Part(A)$, $\Pack(A)$ � $\BQP(n)$.

\begin{lemma}
	\label{lem:PackStable}
	��� ����� ������� $A\in\{0,1\}^{m\times n}$ ���������� ���� $G$ �� $n$ �������� �����, ��� $\Pack(A) =_A \Stable(G)$.
\end{lemma}

\begin{proof}
	���������� ��������, ��� ������ ����������� ����
	$$
	x_1 + x_2 + \ldots + x_k \le 1
	$$
	�� ������� $A\bm{x} \le \bm{1}$ ��� ������� $\bm{x}\in \{0,1\}^n$
	������������ ������ ����������
	$$
	x_i + x_j \le 1, \quad 1\le i < j \le k,
	$$
	������������ ��������� ������������ ����������� ��������.
\end{proof}

�������� �����������~\eqref{eq:StablePack}, ����� ������� ����� � ���, ��� ��������� $\{\Pack(A)\}$ � $\{\Stable(G)\}$ ��������� (������� �� ����� � ��� �� ��������������).

\begin{lemma}[\cite{Maksimenko:2016bool}]
	\label{lem:StablePart}
	��� ������ ����� $G=(V,E)$ ���������� ������� $A\in\{0,1\}^{m\times n}$, $m = |E|$, $n = |V|+|E|$, ������� ����� �� ��� ������� � ������ ������, ��� $\Stable(G) =_A \Part(A)$.
\end{lemma}
	
\begin{proof}
	��� ������� ����������� 
	\begin{equation}
		\label{eq:Stable2}
		x_v + x_u \le 1,  \quad \{v,u\} \in E,
	\end{equation}
	�� �������� ������������� $\Stable(G)$ ������ ��������������� ���������� $y_{vu} = 1 - x_v - x_u$.
	�������� ��������, ��� ��������� 0/1-��������, ��������������� ������������~\eqref{eq:Stable2}, ������� ������������ ��������� 0/1-��������, ��������������� ����������
	\begin{equation*}
		x_v + x_u + y_{vu} = 1,  \quad \{v,u\} \in E.
	\end{equation*}
\end{proof}

����� �������, �������� �����������~\eqref{eq:PartPack}, ��������� $\{\Part(A)\}$ �������� �� ������ ��� ������������� �������� $\{\Pack(A)\}$ � $\{\Stable(G)\}$, �� � ��������� �� �����.

������ ������ ������� �������� ����������
% ������ �������������� ���������� ����--�����--������~\cite{Garey:1982} ���
 �������� ��������������.
����� ����� ���� ������������� � ����������� ������������, � ������� �� ���������, ����� �������� \emph{��������} ������������� (�� ������ � �������� ����������).
%��������, ������ ������������� ���� �� ��� ����, ��� ����� ����� ��������������� ������ ������������� �����������.
����� �������, ������ ������������� �������������� ����� ����� ��������������� ������ ������������� �����������, ���� ������ ��������� �������� ������� ������ ��������� ������ ��������� ����������.

%\begin{definition}
%	\label{def:Aff}
%	��������� �������������� $P$ \emph{������� ��������} � ��������� �������������� $Q$, ���� ��� ������� ������������� $p\in P$ �������� $q\in Q$ �����, ��� $p \lea q$, ������ ����������� ������������, � ������� ��������� ������������ $q$, ���������� ������ ��������� �� ����������� ������������, � ������� ����� $p$.
%	�����������: $P \propto_A Q$.  
%\end{definition}

%�������, ��� ����� �����\footnote{�������������� ������������� �������� ����� ����������� ������� ������~\cite{Garey:1982}.} ��� ������ �������� ����������� ������ ��������������� ����� ����� ���� $s$ � ����������� $d(s)$.


\begin{definition}[\cite{Maksimenko:2017}]
	\label{def:Aff}
	����� ��������, ��� ��������� �������������� $P$ \emph{������� ��������} � ��������� �������������� $Q$, ���� �������� ������������� ���������� (������������ ������� ������������� $p\in P$):
	\begin{enumerate}
		\item 
		�������������� $\tau$ ���� $I$ ������� ������������� $p = p(I)\in P$ � ��� $I'$ ������������� $q = q(I') \in Q$.
		\item 
		�������� ���������� ��� ������� ���� $I$ ��������� ����������� 
		\[
		\alpha\from \R^d \to \R^{d'}, \qquad d = d(I), \quad d' = d'(\tau(I)),
		\]
		������, ��� ������������ $\alpha(p)$ �������� ������ (�������� �������������) ������������� $q$ � ������� ������������ $p$.
	\end{enumerate}
	���� �������� ���������� $P$ � $Q$ ���������� ���: $P \propto_A Q$.  
\end{definition}

%�������, ��� ��� ����������� ���������� �� ����������� �������� ����������,
% ���������� ����� � ���������� \cite{Bondarenko:2008}, 
% � ������� �������� �������.

\begin{remark}
	�~����������� \ref{def:Aff} ������ �������� �������� �������������� ����������� ������������ ������� �������������, �~��~��� �����������. 
	���� � ���, ��� ������� �����������, ������������ ������������, ����� ��������� ������ �� ����� ��������� �������� ��� ��������. ��� �� �����, ����� ����� �������� ������ ��������� ������� ������ �����, ��������������� � ���������������.
	
	�~�������� ������� ������������� � ����������� ��������� ����� ����������� $\Part(A)$, ���������� �����������
	\[
	x_1 + x_i + x_j = 1, \quad 2 \le i < j \le n.
	\]
	��������, �� ������� �� ����� ������������ ����� $(1,0,\dots,0)$.
	������ � ���, �������� ����������� \eqref{eq:PartPack}, �� �������� ������ ������������� $\Pack(A)$, ������������� �������������
	\[
	x_1 + x_i + x_j \le 1, \quad 2 \le i < j \le n,
	\]
	� �������� ����������� $n$.
	����� <<�����������>> ������ ��������� �������������� ������ ����������� ������������~\ref{def:Aff}.
\end{remark}

%\begin{remark}
%����������� \ref{def:Aff} ���������� �� ����������� �������� ���������� �~\cite{Maksimenko:2013NP,Maksimenko:2016bool} �������� ������� �������������� ������������ �������������� $A$ � ������������� ��������� �����������.
%��� �� �����, ��� ���� ������ �������� ����������, ����������� � �������~\cite{Maksimenko:2013NP,Maksimenko:2016bool}, �������������� ���� ������� ����� �����������, ��� ��� ��������������� �������� ����������� ������� ����� �������.
%\end{remark}

\begin{remark}
��� �������, �������������� �������� ���������� ��������� $P$ � ��������� $Q$ ����������� �� ��������� �����.
��� ������� ���� $I$, ��������� ������������ $p(I)\in P$ ���������� �������� ���� $I'$ ������������� $q(I')\in Q$, ��� ����� $F$ � ����������� ��������� ����������� $\alpha\colon p(I) \to F$.
�������������� ������������ ��������� ��������, ��� �������, ��������.
������� � ���������� �� �� ������� �������� �������� ���� �������, � ��� ���� �������� ���������� ����������� ��� ����������� ����
<<��� ������� ���� $I$, ��������� ������������ $p(I)$ �� ��������� $P$, ���������� ��� $I'$, ������������ ������������ $q(I')\in Q$, �����, ��� $p(I) \lea q(I')$>>.
(��������� %������ ���� ����������� 
����� ������� ����� \ref{lem:PackStable} � \ref{lem:StablePart}.)
��� ����� ����������� ���, ��� ������ �����������, � ������� �� $P \propto_A Q$, �������� ���������� � �������� ����\'����. 
\end{remark}

�� ������ ���������� ���� ����������� �������� ��������� �������� ������������� ����������� �� �����������~\ref{def:Aff}.
���, ����������� \eqref{eq:compareDeltaCP} ����� ���������� � ���� $\Delta \propto_A \CP$. 
� �� ���� \ref{lem:StablePart}, \ref{lem:PackStable} � ����������� 	\eqref{eq:PartPack} �������
 
\begin{theorem}[\cite{Maksimenko:2015DAN}]
	\label{thm:Class1}
	$\Stable \propto_A \Part \propto_A \Pack \propto_A \Stable$,
	��� $\Stable = \{\Stable(G)\}$, $\Part = \{\Part(A)\}$, $\Pack = \{\Pack(A)\}$.
\end{theorem}

���������� ��������� ��������� �������� ����� ���� ����������. 

\begin{prop}
	%\label{thm:Prop}
	����� $P \propto_A Q$. 
	�����������, ��� � ��������� $P$ ���� �������������, ������� ���� ��� ��������� �� ��������� �������:
	\begin{enumerate}
	\item C�������������������� ����� ������ ��� ����������� (������������ ������� �������������).
	\item C������������������ �������� ����� ����� �������������.
	\item NP-������� �������� ����������� ������.
	\item C������������������ ����� �������������� ��������.
	\item C������������������ ��������� ����������.
	\end{enumerate}
	\noindent
	����� � $Q$ ������� ������������� � ���� �� ����������.
\end{prop}

������� ������ ��������� $\BQP = \{\BQP(n)\}$ � $\Stable$.


\begin{theorem}[\cite{Maksimenko:2015DAN,Maksimenko:2016bool}]
	\label{thm:BQPStable}
	��� ������� $n\in \N$ ���������� ���� $G = (V,E)$, $|V| = n(n+1)$, $|E| = n(2n-1)$, �����, ���
	$\BQP(n) \lea \Stable(G)$.
\end{theorem}
(������� ��������� ������� �~\cite{FioriniPokutta:2015}, �� � ����� ������ ������������ $\lee$ (��. �����������~\ref{def:ineE} �� �.~\pageref{def:ineE}) � ��� $|V| = 2 n^2$.)
% ����� ����, �������������� ���� �������������� ����������� ����� ����������� �~\cite{FioriniPokutta:2015}.)

\begin{proof}
	������ ��������� $x_{ij} = x_{ii} x_{jj}$ �� ���������~\eqref{eq:BQP}, ������������� ����� ������������ ������������, ������������ ������������
	\begin{equation}
	\label{eq:Clique}
	\begin{aligned}
	x_{ii} - x_{ij} &\ge 0, \\
	x_{jj} - x_{ij} &\ge 0, \\
	x_{ii} + x_{jj} - x_{ij} &\le 1,
	\end{aligned}
	\end{equation}
	��� ������� $x_{ij}\in\{0, 1\}$, $1\le i \le j \le n$.
	�������� ������������� �� � ������� ���������� ���� $y_l + y_m \le 1$. %��~\eqref{SSP}.
	��� ����� ������ $n(n+1)$ ����� 0/1-����������:
	\begin{equation}
	\label{eq:BQP2SSP}
	\begin{aligned}
	s_{ij} &= x_{ij},           & 1 &\le i  <  j \le n,\\
	t_{ij} &= x_{ii} - x_{ij},  & 1 &\le i  <  j \le n,\\
	u_i    &= x_{ii},           & 1 &\le i \le n,\\
	\bar{u}_i    &= 1 - x_{ii}, & 1 &\le i \le n.\\
	\end{aligned}
	\end{equation}
	����� �����������~\eqref{eq:Clique} ������������
	%\begin{equation}\label{eq:SSP2}
	\[
	\begin{aligned}
	s_{ij} + \bar{u}_j & \le 1, \\
	t_{ij} + u_j       & \le 1, \\
	u_i    + \bar{u}_i & =   1, \\
	s_{ij} + t_{ij} + \bar{u}_i & = 1,
	\end{aligned}
	\]
	%\end{equation}
	��� ������� ��������������� ���� ����������.
	��������, ��������� ��� ��������� (������, $n(n+1)/2$ �������� ��������)
	���������� ��������� ����� ������������� $\Stable(G)$, ��� ����� ������ ����� $G$ ����� $n(n+1)$, � $n (2n - 1)$ ��� ����� ���������� ������� ����������
	%\begin{equation}\label{eq:SSP2}
	\[
	\begin{aligned}
	s_{ij} + \bar{u}_j & \le 1, \\
	t_{ij} + u_j       & \le 1, \\
	u_i    + \bar{u}_i & \le 1, \\
	%   s_{ij} + t_{ij}    & \le 1, \\
	s_{ij} + \bar{u}_i & \le 1, \\
	t_{ij} + \bar{u}_i & \le 1.
	\end{aligned}
	\]
	%\end{equation}
	����� ����, �����������~\eqref{eq:BQP2SSP} ��������� ��� ����� � �������������� $\BQP(n)$ ������������� �������� ������������.
\end{proof}

\begin{prop}
	\label{prop:StableBQP}
	���� ���� $G=(V,E)$ ��������, �� ����������� $\Stable(G) \lea \BQP(n)$ ���������� �� ��� ����� $n$.
\end{prop}

\begin{proof}
������ ����� �������, ��� � ��������� ������ $\Stable(G)$ ������ ������ ������� $\bm{0}$, $\bm{e_1}$, \dots, $\bm{e_d}$, ��� $d=|V|$.
���� ���� $G=(V,E)$ ��������, �� $\Stable(G)$ ����� <<������������>> �������� �������� ��� ��� ������� ���� 0/1"~������. ����� $\bm{x}$ "--- ���� �� ����� (��������������) ��������. ����� ��������, ��� ������� $\bm{0}$ � $\bm{x}$ ������������� $\Stable(G)$ ��������, ��� ��� ����������� �� ������� ������������ � �������� ��������� ������ $\bm{e_1}$, \dots, $\bm{e_d}$.
�������������, ������������ $\Stable(G)$ �� �������� 2"~�����������.
�������� ��������, ��� ������������ $\BQP(n)$ (� ������ � ��� � ��� �����) �������� 2"~�����������~\cite{Padberg:1989}.
\end{proof}


�������, ��� �������� ���������� �������������� ����� ������� � �������� ����������� �������� ��������� ����������� �������� ������ ��������������� �����. 

\begin{theorem}
\label{thm:Aff2Cones}
����� �������� ������ ������������� ����������� $(d,S,g)$ ������ ��������� �������������� $P = \{p(I)\}$, � ������ $(d',S',g')$ "--- ��������� �������������� $Q = \{q(I')\}$.
�����������, ��� ��������� $P$ ������� �������� � $Q$ �, ����� ����, ��� ������� $p \in P$ ���������� ������� � ���������������� ������������� $q$ ��������������
\(H(\bm{a}, h)\), $h \in \Z$, $\bm{a} \in \Z^{d'}$, $d' = d'(I')$, �������� ����� $\alpha(p) = q \cap H(\bm{a}, h)$, ������� ������������� $p$.
%������ $\bm{a}^T\bm{y} + h \ge 1$ ��� ���� $\bm{y} \in \ext(q) \setminus \ext(F)$.

����� ��������� ��������� �������� ������ ������ $(d,S,g)$ � ������������ $\bm{-1} \le \bm{c} \le \bm{1}$ ������� �������� � ��������� ��������� �������� ������ ������ $(d',S',g')$.
\end{theorem}
\begin{proof}
�������� �����������~\ref{def:ConesReduction} (�������� ���������� ��������� �������� ������), ���������� �������:
\begin{enumerate}
 \item[1)] �������������� $\tau\from I \mapsto I'$, 
 \item[2)] ��������� �������� ����������� $\alpha'\from \R^d \to \R^{d'}$,
 \item[3)] ���������� ������� $\beta\from \ext(F) \to \ext(p)$, ��� $F = \alpha(p)$ "--- ����� ������������� $q$, ������� ������������� $p$, 
\end{enumerate}
��������������� �������� �����������~\ref{def:AffReductionRestriction} (�������� ���������� ����� � ������������).
����� ���� ������� ��� ���������:
�������������� $\tau$ ��� ��������, � $\beta(\bm{y}) = \alpha^{-1}(\bm{y})$, $\bm{y} \in F$.
������� �����, ��� �������� � ��������� ��������� ����������� ����� ���� ������� �� �������������� ����� (��., ��������, \cite{Winkler:1996}).
�������� ������� ��������� �������� ����������� $\alpha'$ , ��������������� �������� ������ 3 �����������~\ref{def:AffReductionRestriction}.
	
%�������� �������������, ������������ $p$ � ����� $F = \alpha(p)$ ������� ������������.
��� �������������� �����������, ��� �������� ����������� $\alpha$ ���������� �������� 
\begin{equation}
\label{eq:aff-transf}
\bm{y} = A\bm{x} + \bm{b}, \qquad \text{��� } \bm{x} \in p, \quad \bm{y} \in F.
\end{equation}
��� ��� $p$ � $F$ ������� ������������, ��
��� ������ $\bm{y^*} \in F$ � ������ ������� $\bm{c}\in\R^d$, $d = d(I)$,
\[
\Bigl( 
\forall \bm{y}\in F \quad \bm{c}^T A^{-1} \bm{y^*} \ge \bm{c}^T A^{-1} \bm{y} 
\Bigr)
\iff
\Bigl( 
\forall \bm{x}\in p \quad \bm{c}^T \beta(\bm{y^*}) = \bm{c}^T A^{-1}(\bm{y^*} - \bm{b}) \ge \bm{c}^T \bm{x}
\Bigr).
\]
%�������, ��� �������� ������� $A^{-1}$ ����� ���� ������� �� �������������� ����� (��., ��������, \cite{Winkler:1996}).

������ ������������� ������� ������ $\bm{v} = \bm{c}^T A^{-1}$ ����� �������, ����� ����������� $\bm{v}^T \bm{y} > \bm{v}^T \bm{y'}$ ����������� ��� 
����� $\bm{y} \in \ext{F}$ � $\bm{y'} \in \ext(q) \setminus \ext(F)$.

�������������� \(H(\bm{a}, h)\) �������� ������� � $q$ � $F = q \cap H(\bm{a}, h)$. ����� �������, $\bm{a}^T\bm{y} = h$ ��� ���� $\bm{y} \in F$ �, �� �������� ��������, $\bm{a}^T\bm{y'} < h$ ��� ���� $\bm{y'} \in \ext(q) \setminus \ext(F)$.
��� ��� $h \in \Z$ � $\bm{a},\bm{y'} \in \Z^{d'}$, ��
\[
h - \bm{a}^T\bm{y'} \ge 1 \qquad \forall \bm{y'} \in \ext(q) \setminus \ext(F).
\]
������� ����� $N \in \N$ ���, ��� $N > \max_{\bm{y}\in q} \|A^{-1} \bm{y}\|_1$, ��� $\|\bm{x}\|_1 = \sum_{i} |x_{i}|$. (��������, ��� ����� ������� �� �������������� �����.)
����� ��� ����� $\bm{y} \in \ext{F}$, $\bm{y'} \in \ext(q) \setminus \ext(F)$ � $\bm{-1} \le \bm{c} \le \bm{1}$ ���������
\begin{multline*}
\bigl(\bm{c}^T A^{-1} + 2N \bm{a}^T\bigr) \bm{y} =\\
= \bm{c}^T A^{-1} \bm{y} + 2Nh > -N + 2Nh > \bm{c}^T A^{-1} \bm{y'} + 2Nh - 2N \ge\\
\ge \bigl(\bm{c}^T A^{-1} + 2N \bm{a}^T\bigr)\bm{y'}.
\end{multline*}

����� �������, �������� ���������� ��������� ��������� �������� ������ ������ $(d,S,g)$ � ������������ $\bm{-1} \le \bm{c} \le \bm{1}$ � ��������� ��������� �������� ������ ������ $(d',S',g')$ ������������ ��������������� $\tau$ �� ����������� �������� ���������� �������� ��������������, �������� ������������ \(\alpha'\from \bm{c} \mapsto \bm{c}^T A^{-1} + 2N \bm{a}^T\) � ���������� �������� $\beta: \bm{y} \mapsto A^{-1}(\bm{y} - \bm{b})$.
\end{proof}

\begin{remark}
��� ���� �������� �������� ���������� ��������������, �������������� � ��������� ������, ����� $F = \alpha(p)$ ���������������� ������������� $q$ ����������� ��� ����������� ���������� ������� (� $q$) ���������������, ���������� �������� ��������� � �������������� �������������� ����
\begin{equation*}
%\label{eq:affsupport}
A \bm{y} + \bm{b} = \bm{0},
\end{equation*}
������ ��� ������� $\bm{y'} \in \ext(q) \setminus \ext(F)$ � ���� ������� ������� ���� �� ���� ���������, � ������� ����� ����� ������������ (������������).
��������, ��� ������� ����� ���� ������������ � ���� ���������
\[
\bm{1}^T A \bm{y} + \bm{1}^T \bm{b} = 0,
\]
������������ �������������� $H(\bm{1}^T A, -\bm{1}^T)$, ������� � $q$ � ��������������� �������� �������~\ref{thm:Aff2Cones}.

��������, � �������������� �������~\ref{thm:BQPStable} ��������������� ������� ������� �� ���������
\[
\begin{aligned}
u_i   + \bar{u}_i  - 1 & = 0,          & i &\in [n],\\
s_{ij} + t_{ij} + \bar{u}_i - 1 & = 0, & 1 &\le i < j \le n.
\end{aligned}
\]
\end{remark}


%%%%%%%%%%%%%%%%%%%%%%%%%%%%%%%%%%%%%%%%%%%%%%%%%%%%%%%
%
% End of section
%
%%%%%%%%%%%%%%%%%%%%%%%%%%%%%%%%%%%%%%%%%%%%%%%%%%%%%%%

%% Глава 4

%%%%%%%%%%%%%%%%%%%%%%%%%%%%%%%%%%%%%%%%%%%%%%%%%%%%%%%%%%
%
%     ������� �������� ����������
%
%%%%%%%%%%%%%%%%%%%%%%%%%%%%%%%%%%%%%%%%%%%%%%%%%%%%%%%%%%
\chapter{������� �������� ����������}
\label{chap:AffExamples}

\begin{flushright}
��� �������� ���� ������� �������� ������.\\ \emph{�.~������}
\end{flushright}
\bigskip
%\hfill
%\begin{minipage}{0.4\textwidth}
%	��� �������� ���� ������� �������� ������.
%	\begin{flushright}
%		����� ������
%	\end{flushright}
%\end{minipage}

� ��������� ������� ���������� ����� � �������� ������� ���� ��������, ��� ��������� �������������� ����������� �������� $\Stable$, �������������� ��������� $\Part$ � �������������� �������� $\Pack$ ������������ � ������ �������� ����������, � ��������� ������� ������������ �������������� $\BQP$ ������� �������� � ���. ������ ���������� $\Stable \propto_A \BQP$ ����������.
� ���� ����� ����� ������� ��� ����������� ����������� ��� ����� ����������� � ���������� �������� ��������������. ���������� ����������� (�� ����������� ���������� ������� �����) ���������� �� ���.~\ref{fig:AffTree}. ������ ������� �������� �������� ���������� ��������� ��������������, �������������� � ������ �������, � ��������� ��������������, �� ������� ��������� �������. ������������� (�������) ������� ��������, ��� �������� ���������� � ������ ����������� ����������. �������� ������� "--- ��������� 
%��� ������� ������� �������������� 
�����. ���������� ������� "--- �����, ���������� � ��������� ������. ��� ��������� �������������� ������� �������� � ������ ��������, ������������� ���� ��� �� ���.~\ref{fig:AffTree}, ������ �������� ����������� ������ NP-�����. ��� ��������, ������������� ����, ��� ������ ������������� ���������. ����������� �������� ������������� �������� ��������, ��� ������� ������ ���� ������ �� ����������.

\tikzset{cross/.style={cross out, draw=black, minimum size=2*(#1-\pgflinewidth), inner sep=0pt, outer sep=0pt},
%default radius will be 4pt. 
cross/.default={4pt}}
\begin{figure}[tbh]
\centering
\begin{tikzpicture}[scale=1.3, >=stealth', thick] %, radius=2pt, delta angle=180]
\label{hierarchy}
\tikzstyle{every node}=[rounded corners,text centered,draw=black,minimum size=23pt]

\node (Cut) at (-4.1, 1.3) {\hyperref[def:CutPolytope]{�������}};
\node (BQuadr) at (0, 1.3) {\hyperref[eq:BQP]{������ �����. �����.}};
\node (QAss) at (4.1, 1.3) {\hyperref[eq:QAP]{�����. ����������}};
\draw[<->] (Cut) to (BQuadr);
\draw[->] (QAss) to[bend right = 5] (BQuadr);
\draw[->,dashed] (BQuadr) to[bend right = 5] (QAss);

%\tikzstyle{every node}=[rounded corners,text centered,draw=black,fill=lightyellow]

\node (3Assign) at (4.3, -1.05) {\hyperref[sec:3Ass]{3-����������}};
\node (Color) at (2.2, 0) {\hyperref[sec:Color]{��������� �����}};
\node (SS) at (0, -1.3) {\hyperlink{Stable}{�������. ����.}};
%			\draw[->,draw=red] (Cut) to (Part);
\draw[->,dashed] (BQuadr) to[bend right = 10] (SS);
\draw[->] (SS) to[bend right = 10] node[cross] {} (BQuadr);
\node (Pack) at (-3.0, -1.3) {\hyperlink{def:Pack}{��������}};
\node (SetPart) at (-1.7, 0) {\hyperref[eq:Part]{���������}};
\draw[->] (Pack) to (SS);
\draw[->] (SetPart) to (Pack);
\draw[->] (SS) to (SetPart);
\draw[->] (SS) to[bend right = 8] (Color);
\draw[->,dashed] (Color) to[bend right = 8] (SS);
\draw[->] (3Assign) to[bend right = 5] (SS);
\draw[->,dashed] (SS) to[bend right = 5] (3Assign);

%\tikzstyle{every node}=[rounded corners,text centered,draw=black,fill=lightgray]

\node (Steiner) at (3.9, -2.15) {\hyperref[sec:SteinerTree]{������� ��������}};
\draw[->,dashed] (SS) to (Steiner);

\node (LOP) at (-4.6, 0) {\hyperref[sec:LOP]{���. ������.}};
\draw[->,dashed] (BQuadr) to[out=-172,in=30] (LOP);
\draw[->] (LOP) to node[cross]{} (BQuadr);

%\tikzstyle{every node}=[rounded corners,text centered,draw=black,fill=lightgreen]

\node (DCover) at (0, -3.0) {\hyperref[def:DCP]{������� ����.}};
\draw[->,dashed] (SS) to[bend right = 20] (DCover);
\draw[->,dashed] (DCover) to[bend right = 20] node[cross] {} (SS);
\draw[->,dashed] (LOP) to[out=-135,in=180] (DCover);

\node (Part) at (-2.9, -4.3) {\hyperref[eq:KnapEq]{��������� �����}};
\draw[->,dashed] (DCover) to (Part);
\node (3SAT) at (0.0, -5.6) {\hyperref[subsec:k-Sat&POP]{3-������������}};
\draw[->, dashed] (DCover) to (3SAT);
\node (Order) at (3.7, -5.6) {\hyperlink{def:POP}{������. ������.}};
\draw[<->] (3SAT) to (Order);
\node (Cubic) at (3.2, -4.3) {\hyperref[def:Cubic]{���������� ��������}};
\draw[->,dashed] (DCover) to (Cubic);

\node (Knapsack) at (-5.05, -5.6) {\hyperref[eq:KNAP]{������}};
\draw[->] (Part) to (Knapsack);
\node (SAT) at (0.0, -6.9) {\hyperref[subsec:k-Sat&POP]{������������}};
\draw[->] (3SAT) to[bend right = 20] (SAT);
\draw[->] (SAT) to[bend right = 20] node[cross] {} (3SAT);
\node (Cover) at (3.2, -6.9) {\hyperref[def:Cover]{��������}};
\draw[->] (Cover) to[bend right = 5] (SAT);
\draw[->,dashed] (SAT) to[bend right = 5] (Cover);

\node (Assign) at (-3.8, -6.9) {\hyperref[def:CAP]{���������� � �������.}};
\draw[->,dashed] (Part) to (Assign);
% 0, 2.4, 4.65
\node (ATSP) at (-2.4, -8.2) {\hyperref[sec:Travelling]{����������� ������}};
\draw[->,dashed] (SAT) to (ATSP);
\node (Path) at (0.9, -8.2) {\hyperlink{def:PathPolytope}{$s$-$t$ ����}};
\draw[->] (ATSP) to (Path);
\node (TSP) at (3.9, -8.2) {\hyperref[sec:Travelling]{����������� ����}};
\draw[->,dashed] (Path) to (TSP);
\end{tikzpicture}
\caption{�������� ���������� �������������� NP-������� �����}
\label{fig:AffTree}
\end{figure}

� ������ ������� ����� ���������� ����������� �������������� �������� � ������� �������� ���������. ������� �������� � ���� ������� ��������� ����������� �������������� $\NPadj(A)$, ��� ������� $A \in \{0,1\}^{m\times n}$ �������� ����� ��� ������� � ������ ������. ��� �������������
�������� ��������������� ������� �������� � �������� ���������� ����������:
\begin{enumerate}
\item $\Stable \propto_A \NPadj$.
\item ������ ������������� ����������� ������ ��� $\NPadj$ NP"~�����.
\item ���� ������������ $\NPadj(A)$ �� �������� ��������, �� $\NPadj(A) \lea \Stable(G)$ ���������� �� ��� ������ ����� $G$.
\end{enumerate}
��������� �������� ������� � ����������� ����������� ������� �������������� ������� �������� �� �������������� ����������� �������� � ������� ���������� � ��� ��������.

�����, � �������~\ref{sec:PolytopesWithNPadj} ��������������� ��������� �������������� � NP-������ ��������� ����������� ������: ������������� ������ � �������, ������������� ������ � ��������� �����, ������������� ������ � ����������� � ������������, ������������� ������ � $k$-������������, ������������� ������ � ��������� ��������������, ������������� ���������� ���������. ��������, ��� ������������� ������� �������� ������� �������� � ���� ����������.

� �������~\ref{sec:LOP&Steiner} ��������������� ������������� �������� �������� � ������������� �������� �������� � �����. ��������, ��� ������ ������������ ������������� ������� �������� � ������� ���������, � ������������� ����������� �������� "--- �� �������.

� �������~\ref{sec:PolytopesSimpleAdj} ��������������� ��������� ��������������, ������� ������� �������� ��������� ������. � ����� ������ �������� ���������� ��� ����������� �� ��� ������ ���������������.
������������� ������������� ������ � ����������� � ��������� �������� �������������� ��������� ����� ����� � ����� ������ ��������������� � ��������������� ����������� ��������. �~������������� ������������ ������ �������� �������������� � ������������ ������ � ����������� ����������� ������������ ��������� ������� ������������ ��������������.

��������� �������������� ��� ��������� �������� ������ ������������ ��������������� � �������~\ref{sec:TravellingAll}. ��������, ��� ������������� ������ � ������������ ������� �������� � �������������� ������������� ������ ������������, � ���������, � ���� �������, ������� �������� � ���������, ��������������� � ���� ������� ����������. � ����� ������� ������������ ������������ ���������� �� ���.~\ref{fig:TSPall}.

�� �������� � �������� ������������� ��������������� � �������~\ref{sec:BQP-power} �������� � ������������
������ ������������� $\BQP(n,p)$ ������� $p$. 
��� $p=2$, $\BQP(n,p)$ ��������� � $\BQP(n)$. 
��� $p=1$, $\BQP(n,p)$ "--- $n$"~������ 0/1-���.
��������, ��� $\BQP(n,p)$ $s$"~���������� ���
$s \le p + \left\lfloor p / 2 \right\rfloor$.
��� $m \in \N$ � $k \ge 2m$ ��������, ��� $\BQP(k,2m) \lea \BQP(n)$ ��� $n = \Theta(\binom{k}{m})$.
�������������, ��� ������ $k \in \N$ � $n \ge 2^{2\cdot \lceil k/3\rceil}$, 
$\BQP(n)$ ����� $k$"~����������� ����� �� ������������������� ������
$2^{{\Theta}\left( n^{1 / {\left\lceil k/3\right\rceil}}\right)}$ ������.

� ��������� ������� ����� ��������������� ������ � ����������� � ������ � ���������� ������ � ������������ ����������������� ���� ��������. ��������, ��� �������� ��������� ��������� �������� ������ ��������� ������� �������� � ��������� ��������� ������������ �������� ������ ������.
��� ���������, ���� �������� ���������� ������� $\ShortP(n+1)$ �������� ��������� ����� ������������� �������� $\Birk(n)$, $n \in \N$.


\section{������������� �������� � ������� ��������}
\label{sec:DoubleCovering}

�� �������� � ��������������� �������� � ��������� ���������,
\emph{�������������� �������� ���������} ���������� �������� �������� ���������
\begin{equation*}
\label{def:Cover}
\Cover(M) = \Set*{\bm{x}\in\{0,1\}^n \given M \bm{x} \ge \bm{1}},
\quad \text{��� } M\in\{0,1\}^{m\times n}.
\end{equation*}

\emph{�������������� ������� ��������} ����� �������� �������� �������� ���������
\begin{equation*}
\label{def:DCP}
\DCP(B) =  \Set*{\bm{x}\in\{0,1\}^n \given B \bm{x} = \bm{2}},
\end{equation*}
��� $B \in \{0,1\}^{m\times n}$, ������ ������ ������ ������� $B$ �������� ����� ������ ������� � �� ����� ������� ��������.
��-��������, ������� ��� ��������� �������������� ���� ����������� �����~\cite{Matsui:1995}.
�� �� ���� ����������� ����� ����� ��������������� �������� � ������� ��������.

\begin{theorem}[����� {\cite[theorem 4.3]{Matsui:1995}}]
��� ������ ������� $B \in \{0,1\}^{m\times n}$ � �������� ��������� � ������ ������ �������� ������� ������� $M \in \{0,1\}^{4m\times n}$ � ����� ��������� � ������ ������, ��� $\DCP(B) \lea \Cover(M)$.
\end{theorem}

�������� �����, ��� ������ �������� ����������� ������ ��� �������������� ������� �������� NP-�����~\cite{Matsui:1995}.
������, ��� NP-����� ��� ���������� ������������ ��������� $\DCP$, � ������� ������ ���� ����.
� ����� ��������� ���������� ������� ����� ������������, �� �������� ����� ������� ��� ��������, ��� � ��������������~\cite{Matsui:1995}.

������ ����� �������, ��� ������ <<�������� �� ������������ ��������� $\Part(A)$ ���� �� ���� �����?>> �������� NP-������ �������, ���� ���� ������ ������ ������� $A$ �������� ����� ��� �������~\cite{Matsui:1995,Garey:1982}.
� ������ �������������� $\Part(A)$, $A \in \{0,1\}^{m\times n}$, ��������������� ����� �������, ������ ������������, ��������� ������ $\NPadj(A) \subset \{0,1\}^{3n+3}$ �������� ��������� ��������� �������.
��� ���� ��������� ������� $\bm{x} \in \NPadj(A)$ ������ ������ ����������� $y_1, y_2, y_3$. 
������ ���������� $z_j$ ������� $\bm{z} = (z_1,\dots,z_n) \in \Part(A)$ ����� ��������������� ��� ���������� $x_j$, $\bar{x}_j$ � $x'_j$ ������� $\bm{x} \in \NPadj(A)$, � ��� �����������
\begin{align}
x_j + \bar{x}_j &= 1, \label{eq:adj01}\\
y_1 + y_2 + x'_j + \bar{x}_j&= 2. \notag
\end{align}
� ��� ������� ����������� ���� $z_i + z_j + z_k = 1$ �� �������� $\Part(A)$
(������, ����� $A$ �� �������� �� ����� ������, ��������� �� ������������)
������� � �������� ��������� $\NPadj(A)$ ���������
\[
y_3 + x_i + x'_j + x'_k = 2.
\]

������� ������ ���������~\eqref{eq:adj01} � �������� ������������� $\NPadj(A)$ �� 
\[
a + b + x_j + \bar{x}_j = 2, \quad a = 0, \quad b = 1,
\]
�������� � ���������� ������.

\begin{prop}
��� ������ ������� $A \in \{0,1\}^{m\times n}$, ������� ����� ��� ������� � ������ ������, �������� ������� ������� $B \in \{0,1\}^{(2n+m)\times(3n+5)}$,
��� $\NPadj(A) \lea \DCP(B)$.
\end{prop}	

����� ����, � \cite{Matsui:1995} ������ ������������ ������� ��������, �������
������������� $\NPadj(A)$, �� ��� �������� ������� �������� ����� ���������� �
���������.

������� ������ �������� �� ��, ��� ����������� 
\[
y_1 = 0, \quad y_2 = 1, \quad y_3 = 1,
\]
���������� ����� ������������� $\NPadj(A)$, ������� ������������� ������������� $\Part(A)$.
����� �������,
\begin{equation}
\label{eq:PartNPadj}
\Part(A) \lea \NPadj(A).
\end{equation}	
�� �� ����� ����� � ��� ��������� ������� �����������:
\begin{enumerate}
\item[1)] $y_1 = 1$, $y_2 = 0$, $y_3 = 1$;
\item[2)] $y_1 = 0$, $y_2 = 1$, $y_3 = 0$;
\item[3)] $y_1 = 1$, $y_2 = 0$, $y_3 = 0$.
\end{enumerate}
������ ��� ���� ������ (������, �������� �� ������) ��������� �����������:
\begin{align*}
F_1 &= \Set*{\bm{x}\in \NPadj(A) \given y_1 = 0, \ y_2 = 1, \ y_3 = 1}, \\
F_2 &= \Set*{\bm{x}\in \NPadj(A) \given y_1 = 1, \ y_2 = 0, \ y_3 = 1}, \\
F_3 &= \Set*{\bm{x}\in \NPadj(A) \given y_1 = 0, \ y_2 = 1, \ y_3 = 0}, \\
F_4 &= \Set*{\bm{x}\in \NPadj(A) \given y_1 = 1, \ y_2 = 0, \ y_3 = 0}.
\end{align*}
%��������� ��� ����� (������, ��������� �� ������) $F_1$, $F_2$, $F_3$, $F_4$, 
%� ������� ����������.
�������, ��� ������� ��� �� ���� ������� ������ �� ����� ����� �����.
����� ����,
\begin{equation}
\label{eq:F4F3}
F_4 = \Set*{\bm{1} - \bm{x} \given \bm{x} \in F_1}, \qquad
F_3 = \Set*{\bm{1} - \bm{x} \given \bm{x} \in F_2}.
\end{equation}

������������
\[
y_1 = y_2 = 0
\]
������������� ����� ���� ������� ������������� $\NPadj(A)$, ������� ����������
\[
x'_j = \bar{x}_j = 1, \quad y_3 = x_j = 0, \qquad j\in[n].
\]
��������� ��� ������� $\bm{x^0}$.
����������, ���� 
\[
y_1 = y_2 = 1,
\]
�� 
\[
x'_j = \bar{x}_j = 0, \quad y_3 = x_j = 1, \qquad j\in[n].
\]
��������� ��� ������� $\bm{\bar{x}^0}$.
��������, $\bm{\bar{x}^0} = \bm{1} - \bm{x^0}$.

�� ����������� ���� ����������� �������, ��� 
\[
\NPadj(A) = F_1 \cup F_2 \cup F_3 \cup F_4 \cup \{\bm{x^0}, \bm{\bar{x}^0}\},
\]
������ ������� ��� �� ���� ���� �������� �� ����� ����� �����.
����� ����, ������� $\bm{x^0}$ � $\bm{\bar{x}^0}$ ������ ����� � ������ �����, ����� $F_1 = \emptyset$ (� ��������� ������ $\conv\{F_1 \cup F_4\}$ � $\conv\{\bm{x^0}, \bm{\bar{x}^0}\}$ ����� ����� ����� $\bm{1}/2$).
����� �������, � ���� ����, ��� $F_1$ ������� ������������ $\Part(A)$, �������� � ���������� ������.

\begin{theorem}[����� {\cite[theorem 4.1]{Matsui:1995}}]
������ �������� ����������� ������ $\bm{x^0}$ � $\bm{\bar{x}^0}$ ������������� $\NPadj(A)$ NP-�����.
\end{theorem}
\begin{corollary}
��� �������� �������������� �������� $\Cover$ � ������� �������� $\DCP$ ������ �������� ����������� ������ NP-�����.
\end{corollary}	

������� ������, ��� �����������~\eqref{eq:PartNPadj} ��������� � � ��� �������, ����� ����� ������ � ������� ������� $A$ ���������� �� ����.
������, ��� ������ ������� $B$ (� ����� ������ ������ � �������) ����� ��������� ������� $A$, ������� ��� ������� � ������ ������, ���
\[
\Part(B) \lea \NPadj(A).
\]

�������� �����~\ref{lem:StablePart}, ��� ������ ����� $G=(V,E)$ ���������� ������� $A\in\{0,1\}^{m\times n}$, $m = |E|$, $n = |V|+|E|$, ������� ����� �� ��� ������� � ������ ������ � �����, ��� $\Stable(G) =_A \Part(A)$. ������ � ������������~\eqref{eq:PartNPadj} ��� ���� �������� ���������� 
\[
\Stable \propto_A \NPadj. 
\]
�� $\Part \propto_A \Stable$ (��. �������~\ref{thm:Class1}), �������������, $\Part \propto_A \NPadj$.

��� ��������~\cite{Chvatal:1975}, ������������� $\Stable(G)$ ����� ������� �������� �������� ��������� ������.
��������������, � ������������� $\NP \ne \textup{P}$ �������� ���������� $\NPadj \propto_A \Stable$ ����������.
�������, ��� � ��� ������� $\NP \ne \textup{P}$ �� ����
������������� ������������ ��������� $\NPadj$ �� ����� ���� ������ �������������� ��������� $\Stable$.

\begin{theorem}[\cite{Maksimenko:2017}]\label{thm:DCPStable}
���� ������������ $\NPadj(A)$ �� �������� ��������, �� ����������� $\NPadj(A) \lea \Stable(G)$ ���������� �� ��� ������ ����� $G$.
\end{theorem}
\begin{proof}
��� ���� �������� ����, ������������ $\NPadj(A)$ ����������� �������� ���� ������ $\bm{x^0}$ � $\bm{\bar{x}^0}$, � ��������� ���������� 
�������� ������ ���� $\bm{x^{2i-1}} \in F_1$, $\bm{\bar{x}^{2i-1}} \in F_4$, $\bm{x^{2i}} \in F_2$, $\bm{\bar{x}^{2i}} \in F_3$, $i\in [k]$, $k \ge 1$.
������, ��������~ \eqref{eq:F4F3},
\begin{equation}
\label{3xpairs}
\bm{x^0} + \bm{\bar{x}^0} = \bm{x^{2i-1}} + \bm{\bar{x}^{2i-1}} = \bm{x^{2i}} + \bm{\bar{x}^{2i}}.
\end{equation}

�����������, ��� $\NPadj(A)$ ������� ������������ ��������� ����� 
\[
H = \{\bm{y^0}, \bm{\bar{y}^0}, \ldots, \bm{y^{2k}}, \bm{\bar{y}^{2k}}\}
\] 
������������� $\Stable(G)$ ��� ���������� ����� $G = (V,E)$.
��������, ������� ���� ����� ������ ����������� ��������~\eqref{3xpairs}:
\begin{equation}
\label{3pairs}
\bm{y^0} + \bm{\bar{y}^0} = \bm{y^{2i-1}} + \bm{\bar{y}^{2i-1}} = \bm{y^{2i}} + \bm{\bar{y}^{2i}}.
\end{equation}
�������, ��� � ������������� $\Stable(G)$ ���� ��� ���� ������ $\bm{y^*}$ � $\bm{\bar{y}^*}$, ��� �������
\begin{equation}
\label{ThGoal}
\bm{y^*} + \bm{\bar{y}^*} = \bm{y^0} + \bm{\bar{y}^0}.
\end{equation}
��� ����� ��������, ��� ����������� $\conv\{\bm{y^*}, \bm{\bar{y}^*}\}$ � $\conv(H)$ �� �����.
�� ���� $H$ �� �������� ������ $\Stable(G)$.

\begin{figure}[hb]
	\[
	\begin{aligned}
	\bm{y^0}&=(\overbrace{\text{\texttt{1,0,1,1,}}}^{I}
	\overbrace{\text{\texttt{0,0,0,0,1,1,1}}}^{J})\\[-1.0ex]
	&\phantom{=(\text{\texttt{1,0,1,1, }}}^{j_0} \\[-2.0ex]
	\bm{\bar{y}^0}&=(\text{\texttt{1,0,1,1,}}
	\underbrace{\text{\texttt{1,1,1,1,}}}_{U_0\vphantom{\bar{U}_0}}\!
	\underbrace{\text{\texttt{0,0,0}}}_{\bar{U}_0})
	\end{aligned}
	\]
	\caption{��������� �������� $I$, $J$, $U_0$, $\bar{U}_0$.}
	\label{fig:IJU}
\end{figure}

����� $\bm{y^0} = (y^0_1, \ldots, y^0_m)$ � $\bm{\bar{y}^0} = (\bar{y}^0_1, \ldots, \bar{y}^0_m)$, ��� $m$ "--- ����� ������ ����� $G$.
���������� ���������
\[
I = \Set*{i\in [m] \given y^0_i = \bar{y}^0_i}.
\]
��� ��� ������ ������� � $\Stable(G)$ �������� 0/1-��������, �� ��~\eqref{3pairs} �~\eqref{ThGoal} �������
\begin{equation}
\label{eq:8}
y^*_i = \bar{y}^*_i = y^0_i = \bar{y}^0_i = \cdots = y^{2k}_i = \bar{y}^{2k}_i  \quad \text{ ��� } i\in I.
\end{equation}
����� ����� ������������� ������ �� ����������, �������� ������� �������� ��� ������ ���� ������ (��. ���.~\ref{fig:IJU}):
\[
J = \Set*{j\in [m] \given y^0_j + \bar{y}^0_j = 1} = [m] \setminus I.
\]
��������, $J \ne \emptyset$.

����������� ����� ������ ������ $j_0 \in J$ � ��� ������� $i \in \{0,1,\dots, 2k\}$ ��������� ���������
\[
U_i =\begin{cases}
\Set*{j\in J \given y^i_j = 1},& \text{���� } y^i_{j_0} = 1,\\
\Set*{j\in J \given \bar{y}^i_j = 1},& \text{�����.}
\end{cases} 
\]
�� ���������� ��� ��� ��������� ������� �������� � $j_0 \in U_i$ (��. ���.~\ref{fig:IJU}).
��� ������� $U_i$ ���������� ��� ���������� $\bar{U}_i = J \setminus U_i$.
�������� ������� �����������, ��� ������ $i \in \{0,1,\dots, 2k\}$ � ��� ����� $p,r \in U_i$ (� ����� ��� ����� $p,r \in \bar{U}_i$) �������� ������� $\bm{y} \in H$ �����, ��� $y_p = y_r = 1$. 
�� ���� ����������� $y_p + y_r \le 1$ ����������� � �������� ������������� $\Stable(G)$.

����� ��� ����������� ����������� \emph{�������������� ��������} ���� �������� $X$ � $Y$:
\[
X \symdiff Y = (Y \setminus X) \cup (X \setminus Y).
\]
�������������� �������� �������� ���������� ����������:
\begin{enumerate}
	\item $X \symdiff Y = \emptyset \iff X = Y$.
	\item ��������� ��������� $X \symdiff Y \symdiff Z$ �� ������� �� ������������ �������� � ������� ���������� ��������.
	% (��������������� � ���������������).
	\item $X \symdiff Y = Z \iff X \symdiff Z = Y$.
\end{enumerate}

������ � ������������ ���������
\[
S = S(i,j,t) = U_i \symdiff U_j \symdiff U_t, \quad 0\le i < j < t \le 2k,
\]
���������� ������ $\bm{y^*} = \bm{y^*}(S)$ � ������������
\[
y^*_i = \begin{cases}
y^0_i,& \text{��� } i \in I,\\
1,& \text{��� } i \in S,\\
0,& \text{��� } i \in J \setminus S,
\end{cases}
\]
� ������ $\bm{\bar{y}^*} = \bm{\bar{y}^*}(S)$ � ������������
\[
\bar{y}^*_i = \begin{cases}
y^0_i,& \text{��� } i \in I,\\
0,& \text{��� } i \in S,\\
1,& \text{��� } i \in J \setminus S,
\end{cases}
\]

\begin{lemma}
	������� $\bm{y^*}$ � $\bm{\bar{y}^*}$ ����������� $\Stable(G)$.
\end{lemma}
\begin{proof}
	���������� ��������, ��� ���� ������� ����� $H$ ������������� ���������� ����������� ���� $y_p + y_r \le 1$ �� �������� ������������� $\Stable(G)$, �� $\bm{y^*}$ � $\bm{\bar{y}^*}$ ���� ��� �������������.
	
	�������� ��������� �������.
	
	\textbf{I.} ����� $p,r\in I$, $p \ne r$.
	��� ��� ��� $i \in I$ $i$-� ���������� ������ ����� $H$ � �������� $\bm{y^*}$ � $\bm{\bar{y}^*}$ ���������, �� �� ����, ��� ����������� $y_p + y_r \le 1$  ��������� ��� $H$ �������, ��� ��� ����� ��������� � ��� �������� $\bm{y^*}$ � $\bm{\bar{y^*}}$.
	
	\textbf{II.} ����� $p \in I$, $r \in J$.
	(������ $r \in I$, $p \in J$ ����������� ����������.)
	����� $y^0_r + \bar{y}^0_r = y^*_r + \bar{y}^*_r = 1$.
	�������������, $\max\{y^*_r, \bar{y}^*_r\} = \max\{y^0_r, \bar{y}^0_r\} = 1$.
	� ����� �� ���������� ����������� $y_p + y_r \le 1$ ��� $H$ �������, ��� ��� ����� ��������� ��� $\bm{y^*}$ � $\bm{\bar{y^*}}$.
	
	\textbf{III.} ����� $p \in S$, $r \in J\setminus S$.
	(������ $r \in S$, $p \in J\setminus S$, ����������� ����������.)
	����� $y^*_p + y^*_r = \bar{y}^*_p + \bar{y}^*_r = 1$ � ��������� ����������� ���������.
	
	\textbf{IV.} ����� $p,r \in S$, $p \ne r$, ��� $S = S(i,j,t)$.
	(������ $p,r \in J \setminus S$, $p \ne r$, ����������� ����������.)
	�������, ��� � ���� ������ $p$ � $r$ ����������� ������������ ������ �� ����� ��������:
	$U_i$, $U_j$, $U_t$, $\bar{U}_i$, $\bar{U}_j$, $\bar{U}_t$.
	���� ��� ������������� ���, �� (��� ���� ������� ���� ��� ����������� ���� ��������) ����������� $y_p + y_r \le 1$ ����������� � �������� ������������� $\Stable(G)$.
	
	����, �����������, ��� $p,r \in U_i \symdiff U_j \symdiff U_t$, � �������, ��� ����� $p$ � $r$ ����������� ������������ ������ �� ��������:
	$U_i$, $U_j$, $U_t$, $\bar{U}_i$, $\bar{U}_j$, $\bar{U}_t$.
	�������, ��� $U_i \symdiff U_j \symdiff U_t$ ������������ ����� ����������� ������� ��������:
	\[
	U_i \symdiff U_j \symdiff U_t = (U_i \cap U_j \cap U_t) \cup (U_i \cap \bar{U}_j \cap \bar{U}_t) \cup (\bar{U}_i \cap U_j \cap \bar{U}_t) \cup (\bar{U}_i \cap \bar{U}_j \cap U_t).
	\]
	���� $p$ � $r$ ����������� ������ �� ���� ������� ��������, �� ��������� ������� ���������.
	�������� �����������, ��� ������� ��������� � � ������, ����� $p$ � $r$ ����������� ������ ����������.
	��������, ���� $p \in U_i \cap \bar{U}_j \cap \bar{U}_t$, � $r \in \bar{U}_i \cap \bar{U}_j \cap U_t$, �� $p,r \in \bar{U}_j$.
\end{proof}

��� ���������� �������������� ������� �������� ��������, ��� �������� ��������� $S$ �����, ��� ������ $\bm{y^*} = \bm{y^*}(S)$ (� ������ � ��� � ������ $\bm{\bar{y}^*} = \bm{\bar{y}^*}(S)$) ����� ���������� �� ���� ��������� ������ ����� $H$.

\begin{lemma}
	���������� $t \in \{2, 3, \dots, 2k\}$ �����, ��� $S(0,1,t)$ ���������� �� ������� �� �������� $U_p$ � $\bar{U}_p$, $0\le p \le 2k$.
\end{lemma}
\begin{proof}
	������ � �������� ������, ����� $k=1$.
	�������� ���������, ���
	\[
	S(0,1,2) = U_0 \symdiff U_1 \symdiff U_2 \not\in \{U_0, U_1, U_2, \bar{U}_0, \bar{U}_1, \bar{U}_2\},
	\]
	��� ��� ��� ��������� �������� �, ����� ����, ��� $U_i$ ����� ����� ������� $j_0$:
	\[
	U_i \symdiff U_j \ne \emptyset \quad \text{�} \quad U_i \symdiff U_j \ne J, \quad \text{��� } i\ne j.
	\]
	
	����������� ������, ��� $k>1$.
	���������� ������ ���� 
	\[
	U_0 \symdiff U_1 \symdiff U_i, \quad 2 \le i \le 2k.
	\]
	��� ���� �������� ����,
	\[
	U_0 \symdiff U_1 \symdiff U_i \not\in \{U_0, U_1, U_i\}.
	\]
	����� ����, $U_0 \symdiff U_1 \symdiff U_i \ne \bar{U}_j$ ��� ����� $j \in \{0,1,\dots,2k\}$, ��� ��� $j_0 \notin \bar{U}_j$.
	�����������, ��� 
	\begin{equation}
	\label{eq:sym-pair}
	U_0 \symdiff U_1 \symdiff U_i = U_j
	\end{equation}
	��� ��������� $j \in \{2,3,\dots,2k\} \setminus \{i\}$.
	�� �����, � ���� ������� �������������� ��������, %(��������~\ref{prop:symdiff} �� �.~\pageref{prop:symdiff}),
	\[
	U_0 \symdiff U_1 \symdiff U_j = U_i.
	\]
	������
	\[
	U_0 \symdiff U_1 \symdiff U_t \ne U_j, \quad \forall t \ne i,
	\]
	��� ��� ����� $U_t = U_i$, ��� ���������� �� �������.
	�� ��� �� ������������,
	\[
	U_0 \symdiff U_1 \symdiff U_t \ne U_i, \quad \forall t \ne j.
	\]
	
	����� �������, ��� ��������� ������� $i$ � $j$, ��� ������� ��������� ������� \eqref{eq:sym-pair}, ����������� �� ���������������� ����. 
	�� ��������� $\{2,3,\dots,2k\}$ �������� �������� ����� ��������.
	������, ����������� �������� $i \in \{2,3,\dots,2k\}$, ��� ��������
	$S(0,1,i) = U_0 \symdiff U_1 \symdiff U_i$ ����� ���������� �� ������� �� �������� $U_p$ � $\bar{U}_p$, $0\le p \le 2k$.
\end{proof}
������� ��������.
\end{proof}


��������, ��� $\NPadj(A)$ �������� ������ ������������� ������� ��������, ��������
\begin{corollary}
$\DCP \npropto_A \Stable$ � $\Cover \npropto_A \Stable$.
\end{corollary}	

%%%%%%%%%%%%%%%%%%%%%%%%%%%%%%%%%%%%%%%%%%%%%%%%%%%%%%%
%
%  ������������� � NP-������ ��������� ����������� ������
%
%%%%%%%%%%%%%%%%%%%%%%%%%%%%%%%%%%%%%%%%%%%%%%%%%%%%%%%

\section[������������� � NP-������ ��������� ����������� ������]{������������� � NP-������ ���������\\ ����������� ������}
\label{sec:PolytopesWithNPadj}

� ���� ������� �� ���������� ��������� �������� ��������������, � ������� ������� �������� ������������� ������� ��������.
��� ���������, ��� ���� ���� �������������� ������ �������� ����������� ������ NP-�����.
������ � ������� ��������.

\subsection{������������� ������ � �������}

���� � �������� �������������� ������ � ������� (���������~\eqref{eq:KNAP} �� �.~\pageref{eq:KNAP}) ����������� �������� ����������, �� ������� \emph{������������� ������ � ������� � ����������}:
\begin{equation}
\label{eq:KnapEq}
\KnapEq(\bm{a},b) = \Set*{\bm{x} \in \{0,1\}^{n} \given \bm{a}^T \bm{x} = b}, \qquad \bm{a} \in \Z^n,\quad b \in \Z.
\end{equation}
��������,
\[
\KnapEq(\bm{a},b) \lea \Knap(\bm{a},b).
\]
������������ �������������� $\KnapEq(\bm{a},b)$, ��������������� ������� $2b = \bm{a}^T \bm{1}$, ��������� $\PRT(\bm{a})$.
��� ��������������� ������� � ������� � ��������� �����, �������� � ������ �� 21 ������, NP-������� ������� ���� �������� � ��������������� ������ �����~\cite{Karp:1972}.

\begin{theorem}[\cite{Maksimenko:2013NP}]
��� ����� ������� $B \in \{0,1\}^{m\times n}$, ������� ����� ������ ������� � ������ ������, ����� �� �������������� (�� � �������) ����� ��������� ������ $\bm{a} \in \Z^n$ �����, %, $\|\bm{a}\|_{\infty} \le 4^m$,
��� $\DCP(B) =_A \PRT(\bm{a})$.
\end{theorem}
\begin{proof}
���������� ��������, ��� ������� �������� ����������� ��������� $B \bm{x} = \bm{2}$ ����� ���� ������������ � ���� ��������� (��. \cite{Padberg:1972, Kovalev:1977}).
��������, ����� ������� ��� ��������� ������� $B \bm{x} = \bm{2}$, �������������� ������� ������ �� ��� �� $4^i$, ��� $i\in[m]$ "--- ����� ���������.
\end{proof}
\begin{corollary}
������ �������� ����������� ������ ��� �������� �������������� $\PRT(\bm{a})$, $\KnapEq(\bm{a},b)$ � $\Knap(\bm{a},b)$ NP-�����.
\end{corollary}
����� ���� ��������� �� NP-������� ��� ������� � �������~\cite{Chung:1980,Geist:1992,Matsui:1995} ����� ��������.


\subsection{������������� ������ � ����������� � ������������}

����� $\bm{a} \in \Z^{n\times n}$, $b \in Z$.
\emph{������������ ������ � ����������� � ������������} ������������ ��� �������� �������� ���������
\begin{equation*}
\label{def:CAP}
\CAP(\bm{a},b) = \Set*{\bm{x} \in \Birk(n) \given \bm{a}^T \bm{x} = b},
\end{equation*}
��� $\Birk(n)$ "--- ������������ ��������.

������� � ���� ������ � � ������������� ���������� ��������������� ������������
(��., ��������, \cite{Toktas:2006}).
NP-������� ������ ������������� ����������� ������ ����� ������������� ���� ���������� ������� � �����~\cite{Alfakih:1998}.

�������, ��� ������������� ������ � ������� � ���������� ������� �������� � ����� ���������.

\begin{theorem}[\cite{Maksimenko:2013NP}]
\label{thm:KnapEqCAP}
��� ������� $\bm{a} \in \Z^n$ � $b \in Z$ �������� $\bm{c} \in \Z^{2n\times 2n}$ �����, ���
\[
\KnapEq(\bm{a},b) \lea \CAP(\bm{c},b).
\]
\end{theorem}

\begin{proof}
������������� ���, ��� ������������ $\Birk(2n)$ �������� �����, ���������� $n$"~������ �����. %~\cite{Billera:1996}.
��� ����� ����� � ����������� ��������������� $x_{ij} = 0$, ��� ���� $(i,j)$ �� �������� � ��������� 
\[
%\Set*{(i,j) \in [2n]^2 \given \text{$i$ ������� � $j \in \{i,i+1\}$, ���� $i$ ����� � $j \in \{i-1,i\}$}}.
S = \big\{(i,i) \mid i\in[2n]\big\} \cup \big\{(2i-1,2i) \mid i\in[n]\big\} \cup \big\{(2i,2i-1) \mid i\in[n]\big\}.
\]

��� ������� �� ������� ������� $\bm{a} = (a_1,\dots,a_n) \in \Z^n$ ��������� ���������� ������� $\bm{c} \in \Z^{2n\times 2n}$ ��������� �������:
\[
c_{ij} = \begin{cases}
a_p,& \text{���� $i=j=2p$, $p\in[n]$},\\
0,& \text{�����.}
\end{cases}
\]

�������� �������, ��� ��������� ������ ��������� ���� ����� ($n$"~������� ����), ���������� �������������� $\bm{c}^T \bm{x} = b$, ������� ������������ ������������� $\KnapEq(\bm{a},b)$.
� ������ �������, ��� �� �������� ������ ������������� $\CAP(\bm{c},b)$.
\end{proof}


\subsection[������������� ������ � k-������������ � ������ � ��������� ��������������]{������������� ������ � $k$-������������\\ � ������ � ��������� ��������������}
\label{subsec:k-Sat&POP}

����� $U = \{u_1, u_2, \ldots, u_d\}$ "--- ��������� ������� ����������.
�� �������� ��������, �������� $u_i \in \{0, 1\}$, ��� $1$ ���������� ��� <<������>>, � $0$ "--- ��� <<����>>.
����� $C = \{C_1, C_2, \ldots, C_n\}$ "--- ��������� ����� ����������
(���������� ����������) ��� $U$.
%����� ����, ����� ������������, ��� ������ ���������� ������� ����� �� $k$ ���������.

��� ������� ������������ ���������� $C$ ������ �������� ����������
���������� ��������������� ������ $\bm{u} = (u_1, \dots, u_d)\in\{0,1\}^d$.
��������� ���� ����� �������� ���������� $\SAT(U,C)$,
� ��� �������� �������� ���������� \emph{�������������� ������ � ������������}.
� ������, ���� ������ ���������� � $C$  ������� ����� �� $k$ ���������, �� ���������� \emph{�������������� ������ � $k$"~������������}.
 
\begin{theorem}[\cite{Maksimenko:2013NP}]
\label{thm:DCPSAT}
��� ������ ������� $B \in \{0,1\}^{m\times n}$, ������� ����� ������ ������� � ������ ������, ����� �� �������� ����� ��������� ���������� $C = \{C_1,\dots, C_{8m}\}$ ��� $U = \{u_1,\dots, u_n\}$ �����, ��� $|C_i|=3$, $i \in [8m]$, � 
\[
\DCP(B) =_A \SAT(U,C).
\]
\end{theorem}

\begin{proof}
������ ��������� ���� $x_1 + x_2 + x_3 + x_4 = 2$ �� ������� $B \bm{x} = \bm{2}$ ����� �������� ������� �� ������ ����������: 
\[
\bigvee_{j\neq i} x_j \ \mbox{ � } \ \bigvee_{j\neq i} \bar{x}_j, 
\qquad i=1,2,3,4.
\]
\end{proof}
\begin{corollary}
������ ������������� ����������� ���� ������������ ������ ������������� ������ � 3"~������������ NP-�����.
\end{corollary}

�����, �������~\cite{Fiorini:2003} ������� ���� ���� (NP-������� ����������� ������) ���������������. 
� ��� �� ������ ������� ��������, ��� ������������� ������ � 3"~������������ ����� ������� � ��������������� ������ � ��������� ��������������.
������������ �� �� ����� ������ ������. 
������� $D' = (V, A')$ ������� ������� $D = (V, A)$ ����� �������� \emph{\hypertarget{def:POP}{��������� ��������}}, ���� �� ��������� � �����������:
\[
((u, v)\in A') \& ((v, w)\in A') \Rightarrow (u, w)\in A'.
\]

\emph{������������ ��������� ��������} ������������ ����� �������� �������� ��������� $\POP(n)$ ������������������ �������� ���� ��������� �������� ������� ������� �� $n$ ��������.

\begin{theorem}[������� {\cite[Lemma 3.2]{Fiorini:2003}}]
������������ ������ � 3"~������������ $\SAT(U,C)$, $U=\{u_1,\dots,u_d\}$, $C = \{C_1,\dots,C_n\}$, $|C_i|=3$, $i \in [n]$,  ������� ������������ ����� ������������� ��������� �������� $\POP(m)$ ��� $m = 3d+7n$.
\end{theorem}

�� �����, � ���������, ������� NP-������� ������ ������������� ����������� ������ ��� ������������� ��������� ��������.

����� ����, ��������� �������������� ��������� �������� � �������������� ������ � 3"~������������ ������������ � ����� ������ �������� ����������.

\begin{theorem}[������� {\cite[Theorem 1.2]{Fiorini:2003}}]
������������ $\POP(m)$ ������� ������������ ����� ������������� ������ � 3"~������������ $\SAT(U,C)$, ��� $|U| = n(n-1)+1$, $|C| = n(n-1)(n-3/2)$.
\end{theorem}

���������� ������ �������� ��������� �������������� ������ � $k$"~������������ ��� ������ $k \ge 3$ (������ $k=2$ �� ������������ ��������).
����� �������� ��������� �����������.

\begin{lemma}[������� {\cite[Lemma 4.1]{Fiorini:2003}}]
%����� $U = \{u_1,\dots,u_d\}$, $C = \{C_1,\dots,C_n\}$, $|C_i| \le k$, $i\in[n]$.
������������ ������ � ������������ $\SAT(U,C)$ ��� $|c| \le k$, $c\in C$, �������� ������ ������������� ������ � $k$"~������������ $\SAT(U',C')$, ��� $|C'|=|C|$, $|U'|=|U| + k - m$, $m = \min\Set{|c| \given c \in C}$.
\end{lemma}

�����������, ��������� �������������� ������ � $k$"~������������ ��� ������ �������� $k \ge 3$ ��������� � ������ ������� ��������������� � ����� ������ �������� ����������.

\begin{theorem}[������� {\cite[Proposition 4.1]{Fiorini:2003}}]
\label{thm:kSAT}
���������� ������� �������������� ������ � $k$"~������������ $\SAT(U,C)$ ��� $|U|=k$ � $|C|\ge 1$ �����, ��� ��� ����� $U'$ � $C'$, ��� ������� $|c'| < k$, $c' \in C'$, �����������
\[
\SAT(U,C) \not\lea \SAT(U',C').
\]
\end{theorem}

���������� ������ ��������� ���� �������������� ������ � ������������ $\SAT = \{\SAT(U,C)\}$ � �������, ��� ��� ������������ �������������� ��������.

\begin{theorem}
$\Cover \propto_A \SAT \propto_A \Cover$.
\end{theorem}
\begin{proof}
����� $A \in \{0,1\}^{m\times n}$.
���������� $U=\{u_1,\dots,u_n\}$ � ��� ������� ����������� ����
\[
x_{i_1}+x_{i_2}+\dots+x_{i_k} \ge 1
\]
������� $A\bm{x} \ge \bm{1}$ ������� � $C$ ����������
\[
u_{i_1} \vee u_{i_2} \vee \dots \vee u_{i_k}.
\]
��������,
\[
\Cover(A) =_A \SAT(U,C).
\]

������� ������ ����������� �������.
����� $U=\{u_1,\dots,u_n\}$ "--- ��������� ����� ������� ���������� � $C=\{C_1,\dots,C_m\}$ "--- ���������� ���������� ��� $U$.
��� ������� $i\in[n]$ ������ � ������������ ���� ���������� $x_i$ � $\bar{x}_i$,
� ������ �� ������������ 
\begin{equation}
\label{eq:proofSATCover}
x_i + \bar{x}_i \ge 1.
\end{equation}
���������� $x_i$ ����� ��������������� ��������� $u_i$ � $C$, � ���������� $\bar{x}_i$ "--- ��������� $\bar{u}_i$.
��� ������ ���������� $C_j$, $j\in[m]$, ������� �������� ���������������� �����������, ���������� ������ $\vee$ ����� ��������,
� �������� ������ ${} \ge 1$, ������� $m$ ����������.
������ � ������������� \eqref{eq:proofSATCover} ��� ��������� ��������� ������������ $\Cover(A)$.
������� ����������� � \eqref{eq:proofSATCover} ����������, ������� ����� ����� �������������,
������� ������������� $\SAT(U,C)$.
\end{proof}

\subsection{������������� ���������� ���������}

����� $G=(V,E)$ "--- ������ ���� �� $k$ �������� � ����� $T_k$ "--- ��������� ���� ��� ���������� ���������. (��������, ��� ������� ������ ������� ����������� ����� ����� ����.)
\emph{�������������� ���������� ���������} ���������� �������� �������� ��������� 
\begin{equation*}
\label{def:Cubic}
\Cubic(k) = \Set*{\chi(H)\in\{0,1\}^E \given H \in T_k}.
\end{equation*}

��������, ��� ������ �������� ����������� ������ ����� ������������� NP-�����~\cite{Bondarenko:1996}.
�������, ��� ������������� ������� �������� �������� ������� ����� �������������.

\begin{theorem}[\cite{Maksimenko:2013NP}]
$\DCP(A) \lea \Cubic(k)$, ���	
������� $A \in \{0,1\}^{m\times n}$ ����� ����� ������ ������� � ������ ������ �
$k = 6m+4s+2t$, $s$ "--- ����� �������� ������� $A$, ���������� ���� ���� �������, $t$ "--- ����� �������� ������� $A$, ���������� ����� ��� �������.
\end{theorem}
\begin{proof}
����� $\bm{x}=(x_i)\in\{0,1\}^n$ ����� ���������� ������, ��������������� ������� $A\bm{x}=\bm{2}$, ������������� ������� ������������� $\DCP(A)$.
����� $\bm{y}=(y_{lh})\in\R^{k(k-1)/2}$ "--- ������������������ ������ ����������� �������� ����� $G=(V, E)$ �� �������� ������������� $\Cubic(k)$.
������������� ���, ��� ��� ������������� $\Cubic(k)$ �������������� ���� $y_{lh}=0$	% � $y_{lh}=1$ 
�������� ��������.
������� ����� ����� ���������� ��� ����������� ���������� ����� ���������������.
��������, ���� ��������� $y_{lh}=0$, �� �� ����� ������������� ������ �� �������� ����� $G$, ������� �� �������� ����� $(v_l, v_h)$.
����� �������, ����� ���������� ������ ��� �����, �� ������ ��������� ������� $G'$ ����� $G$, ������������, ��� ��� �� �������� � ���� ����� ��������� $y_{lh}=0$.
	
�������� � ������ ������, ����� � ������ \emph{�������} ������� $A$ ������������ �� ����� ���� ������.
%������������ ������������� ��������������� ���� $y_{lh}=0$,	
%��������� ������� $G'$ ����� $G$ ��������� �������.
��������� ������ $V$ ����� $G$ ����� �������� �� ���� �����������:
$U = \{u_1, u_2, \ldots, u_m\}$, $W = \{w_1, w_2, \ldots, w_m\}$ � $T$, ��� $|T| = 4m$.
����������� ��������� ��������� $T$ �������� � ����������� �� ����������� �������~$A = (a_{ij})$:
\[
T=\Set{t_{ij} \given a_{ij}=1, \ 1\le i \le m, \ 1\le j \le n}.
\]
(��������� ������ ������ ������� $A$ �������� ����� 4 �������, �� $|T| = 4m$.)


��� ������� $t_{ij}$ � ���������� ������ �������� $j=\const$ ��������� ������� � ����� (������� ���������� �� �����).
��������, ���, �� �������������, � ������ ������� ������� $A$ ���������� �� ����� ���� ��������� ���������.
����� ���������� $n$ ����� ������.
������� ��������� $W$ ���� ��������� ������.
��������� � ���� $G'$ ����� $(w_i, u_i)$ � ����� $(u_i, t_{ij})$,
$1\le i \le m$, $1\le j \le n$.
	
��� ��� ���� ������� �����, ����������� ������� $G'$ ���������� ��������� ������� �������������� � ������������� ���������� ���������.
���������� ��� ���� ������� ��������������, ������� ������� ������ ��������� $W$ ������� ����.
����������� ����� ��������� ��������������� � �������������� ���������� ��������� � ����� ������� ������.
�������� ���������, ��� ��������� �� ������ ������� ������������ ��������� ������ ������������� ��������������� ������ � ������� ��������.
	
\begin{figure}[tbh]
\centering
\tikzset{small circle/.style={inner sep = 1.5pt,draw,circle}}
	\begin{tikzpicture}[scale=0.7,
	>={Stealth[scale width=0.8]} % ���������� ��� �������
	]
	\node[small circle,densely dotted] (u) at (-2,0) {};	
	\node[small circle] (t) at (0,0) {};	
	\node[small circle] (s4) at (2,1) {};	
	\node[small circle] (s1) at (4,1) {};	
	\node[small circle] (s3) at (2,-1) {};	
	\node[small circle] (s2) at (4,-1) {};	
	\draw (t) node[above] {$t_{ij}$} -- (s4) node[above] {$s_{j4}$} -- (s1) node[above] {$s_{j1}$} -- (s2) node[below] {$s_{j2}$} -- (s3) node[below] {$s_{j3}$} -- (t);
	\draw (s4) -- (s2) (s1) -- (s3);
	\draw[dashed] (u) node[above] {$u_{ij}$} -- (t);
	\draw (-4,1) node[above] {�)};
	\begin{scope}[xshift=14cm]
	\node[small circle,densely dotted] (u1) at (-2,1) {};	
	\node[small circle,densely dotted] (u2) at (-2,-1) {};	
	\node[small circle] (t1) at (0,1) {};	
	\node[small circle] (t2) at (0,-1) {};	
	\node[small circle] (s1) at (2,1) {};	
	\node[small circle] (s2) at (2,-1) {};	
	\draw (t1) node[above] {$t_{i_1 j}$} -- (s1) node[above] {$s_{j1}$} -- (s2) node[below] {$s_{j2}$} -- (t2) node[below] {$t_{i_2 j}$};
	\draw (t1) -- (s2) (s1) -- (t2);
	\draw[dashed] (u1) node[above] {$u_{i_1 j}$} -- (t1) (u2) node[above] {$u_{i_2 j}$} -- (t2);
	\draw (-4,1) node[above] {�)};
	\end{scope}
	\end{tikzpicture}
	\caption{����� � <<�����>> � <<�����>> ���������.}
	\label{fig:Cubic}
\end{figure}
	
�������� ����������� ��� ������, ����� (���������) ������� ������� $A$ �������� ���� ��� ��� �������.
��������� ����� ������� ���� � ���, ����� ��������� ��������������� ���� ��� ��� ������� ��������� $T$ � ����.
����� ������� ���� <<� ����� ��������>>, ������� � $T$ ������ ���������������: 
$s_{j1}$, $s_{j2}$, $s_{j3}$, $s_{j4}$.
��� ����� <<� ����� ���������>> �������	� ��������� $T$ ���: $s_{j1}$, $s_{j2}$.
�������� �� ���, ��� �������� �� ���.~\ref{fig:Cubic}.
	
��������, ��� ��� �� ������ ������������ ��������� � ����� �����������, � ����� ������ ���������� �� $4s+2t$, ��� $s$ "--- ����� �������� ������� $A$, ���������� ���� ���� �������, $t$ "--- ����� �������� � ����� ���������.
\end{proof}


%%%%%%%%%%%%%%%%%%%%%%%%%%%%%%%%%%%%%%%%%%%%%%%%%%%%%%%
%
% ������������� �������� �������� � �������� �������� � �����
%
%%%%%%%%%%%%%%%%%%%%%%%%%%%%%%%%%%%%%%%%%%%%%%%%%%%%%%%

\section[������������� �������� �������� � �������� �������� � �����]
{������������� �������� �������� \\� �������� �������� � �����}
\label{sec:LOP&Steiner}

\subsection{������������� �������� ��������}
\label{sec:LOP}

����� $D = (V, A)$ "--- ������, $V = [m]$.
��������������, ��� ������ $D$ ������, �� ���� $(i,j) \in A$ � $(j,i) \in A$ ��� ���� $i,j \in V$, $i \neq j$.
������������ ������ (������, ��������� ��� ���) � ������� $D$ ����� �������� \emph{�������� ��������} (��. ����������� � ������� \ref{def:linearOrdering} �� �.~\pageref{def:linearOrdering}).
������ �������� ������� $L$ � $D$ ������������� ��������� ������������ $\pi \from [n] \to [n]$, ��������������� ������� 
\begin{equation}
\label{eq:piLinear}
 \pi(i) < \pi(j) \iff (i,j) \in L.
\end{equation}

���������� $y_{ij}$, $1 \le i < j \le m$ ������������������� ������� $\bm{y} \in \R^{m(m-1)/2}$ ��� ��������� ������� $L$ � $D$ ��������� ��������� �������:
\[
y_{ij} = \begin{cases}
1 &\text{���� $(i,j)\in L$,}\\
0 &\text{���� $(j,i)\in L$.}
\end{cases}
\]
\label{def:LOP}
��������� ����� $\LOP(m)$ ��������� ���� ������������������ �������� �������� �������� �~$D$.
�������� �������� $\LOP(m)$ ���������� \emph{�������������� �������� ��������}. 
$\LOP(m)$ ����� ���� ����� ��������� ��� ��������� 0/1"~�������� $\bm{y}\in\{0,1\}^{m(m-1)/2}$, ��������������� 3-�������� ������������ (��., ��������, \cite{Grotschel:1985}):
\begin{equation}
\label{3cycle}
0 \le y_{ij} + y_{jk} - y_{ik} \le 1, \quad i < j < k.
\end{equation}
���� ������������ ����� ������������� �������� ��������� � ����������� ���\-����\-�� �������� ���������� (��., ��������, \cite{Doignon:2009,Kovalev:2012}, � ����� ������ � ���).

\begin{prop}
��� ������� $m \in \N$, $m \ge 3$, ���������� ������� $B \in \{0,1\}^{r\times t}$, $r = \binom{m}{3} + \binom{m}{2}$, $t = \binom{m}{3} + m(m-1) + 2$,
������� ����� ������ ������� � ������ ������ � �����, ��� $\LOP(m) \lea \DCP(B)$.
\end{prop}	
\begin{proof}
� ���������� � ���������� $y_{ij}$, $1 \le i < j \le m$, �� �������� ������������� $\LOP(m)$, ������ ��� ���������� $z$ � $h$. ��� ������ $y_{ij}$ ������ �������������� ���������� $\bar{y}_{ij}$ � ���������
\begin{equation}\label{eq:lopDCP1}
y_{ij} + \bar{y}_{ij} + z + h = 2.
\end{equation}
� ������ 3-��������� ����������� \eqref{3cycle} ������� ����������
\begin{equation}\label{eq:lopDCP2}
y_{ij} + y_{jk} + \bar{y}_{ik} + t_{ijk} = 2,
\end{equation}
��� $t_{ijk}$~--- ��� ���� (��� ������� 3-���������� �����������) �������������� ����������.
� ����� �������, ������� �� $\binom{m}{2} + \binom{m}{3}$ ���������~\eqref{eq:lopDCP1} �~\eqref{eq:lopDCP2} ���������� ��������� ������������ ������� ��������.
� ������ �������, � ����������� ������� ��������������� $z=0$ � $h=1$
��������� ����� ����� �������������, ������� ������������� $\LOP(m)$.
\end{proof}

� ���� ����� �������, ��� ������������� ������� �������� (�� ������ ������, ��������� �� ���) ���� �� ����� ���� ������� $\LOP(m)$, ��� ��� �������� ��������� ������ ��������� ������������~\cite{Young:1978}.

�������, ��� $\LOP(2n)$ �������� � �������� ����� ����� ������������ ������������ $\BQP(n)$.

\begin{theorem}[\cite{Maksimenko:2017LOP}]
$\BQP(n) \lea \LOP(2n)$, $n\in\N$.
\end{theorem}
\begin{proof}
����� $\bm{y} = (y_{ij}) \in \LOP(2n)$.
������������� ���, ��� ����������� $y_{ij} \ge 0$, ��� $1 \le i < j \le 2n$, 
� 3-��������� ����������� \eqref{3cycle} ��������� ��� ���� $\bm{y} \in \LOP(2n)$,
� ��������������� ��������� ���������� (���������) ������� �������������� ��� ������������� $\LOP(2n)$.

�������, ��� ������������ $\BQP(n)$ ������� ������������ ����� $F \subset \LOP(2n)$, ���������� ���������� �������������:
\begin{align}
y_{2i, 2j-1} &= 0, \label{eq:LOP1}\\
y_{2i-1, 2i}   + y_{2i, 2j}   - y_{2i-1, 2j} &= 0, \label{eq:LOP2}\\
y_{2i-1, 2j-1} + y_{2j-1, 2j} - y_{2i-1, 2j} &= 0, \label{eq:LOP3}
\end{align}
��� ���� $1 \le i < j \le n$.

�� \eqref{eq:LOP2} � \eqref{eq:LOP3} �������
\begin{align}
y_{2i-1, 2j}   &= y_{2i-1, 2i} + y_{2i, 2j}, \label{eq:LOP4}\\
y_{2i-1, 2j-1} &= y_{2i-1, 2i} + y_{2i, 2j} - y_{2j-1, 2j}. \label{eq:LOP5}
\end{align}
����� �������, ��� ���������� ������� $\bm{y} \in F$ ������� ������� �� ��������� $y_{2i-1, 2i}$, $i\in[n]$, � $y_{2i, 2j}$, $1 \le i < j \le n$.

�������, ��� �������� ���������� $y_{2i, 2j}$ ���������� ������������ ���������� ��������� $y_{2i-1, 2i}$ � $y_{2j-1, 2j}$.
�� \eqref{eq:LOP4} � $y_{2i-1, 2j} \le 1$ ������� \(y_{2i, 2j} \le 1 - y_{2i-1, 2i}\), ����� �������,
\[
y_{2i-1, 2i} = 1 \Rightarrow y_{2i, 2j} = 0.
\]
�� 3-���������� ����������� $0 \le y_{2i, 2j-1} + y_{2j-1, 2j} - y_{2i, 2j}$ � ��������� \eqref{eq:LOP1} ������� \(y_{2i, 2j} \le y_{2j-1, 2j}\),
�� ����
\[
y_{2j-1, 2j} = 0 \Rightarrow y_{2i, 2j} = 0.
\]
� �� \eqref{eq:LOP5} � $y_{2i-1, 2j-1} \ge 0$ ������� \(y_{2i, 2j} \ge y_{2j-1, 2j} - y_{2i-1, 2i}\), �� ���� 
\[
y_{2i, 2j} = 1, \quad \text{���� $y_{2i-1, 2i} = 0$  � $y_{2j-1, 2j} = 1$}.
\]
����� �������, ��������, ��� ������ $\bm{y}\in F$ �������� 0/1-��������,
\begin{equation}
\label{eq:LOPbqp}
y_{2i, 2j} = y_{2j-1, 2j} (1 - y_{2i-1, 2i}).
\end{equation}

����, ��� ������� ����� $F$ ������ ���� 0/1-���������, ���������������� ����������� \eqref{eq:LOPbqp}, � ��� ���������� ���� �������� ������� ������� �� $y_{2i-1, 2i}$, $i\in[n]$, � $y_{2i, 2j}$, $1 \le i < j \le n$ (��. ��������� \eqref{eq:LOP1}, \eqref{eq:LOP4} � \eqref{eq:LOP5}).
������� ������, ��� ������� ������ �������� ���������� $y_{2i-1, 2i}$, $i\in[n]$,
�� ����� ���� ������������� ��������� ������� ����� $F$.

����� 
\[
I_0 = \Set*{i \in [n] \given y_{2i-1, 2i} = 0}, \quad I_1 = \Set*{i \in [n] \given y_{2i-1, 2i} = 1}.
\]
����� ������������, ��� �������� �������� $I_0 = \{i_1, \dots, i_k\}$ � $I_1 = \{i'_1, \dots, i'_{n-k}\}$ ������������� \emph{�� ��������}.
�������� ������� ��� ��������������� ������� $\bm{y} \in F$ ���������� ������������� $\pi \from [n] \to [n]$ (��. ������� \eqref{eq:piLinear}).
�������
\begin{align*}
\pi(2i_s-1) &= n-k + 2 s, &
\pi(2i_s) &= \pi(2i_s-1)-1, & s &\in [k],\\
\pi(2i'_t-1) &= t, &
\pi(2i'_t) &= n + k + t, & t &\in [n-k].
\end{align*}
���, ��������, � ������ $n=3$ ������� ����� $F \subset \LOP(6)$ ������������� ������ ������������� (���������� � ���� $\pi^{-1}(1)\ldots\pi^{-1}(6)$, �� ���� ���� ����� $i$ ������������� � ���� ������������������ ����� $j$, �� $y_{ij} = 1$)
\begin{alignat*}{4}
&654321, \qquad & k &=3 , \quad & I_0 &= \{3,2,1\}, \quad& I_1 &= \emptyset,\\
&165432, & k &=2 , \quad & I_0 &= \{3,2\},   & I_1 &= \{1\},\\
&365214, & k &=2 , \quad & I_0 &= \{3,1\},   & I_1 &= \{2\},\\
&543216, & k &=2 , \quad & I_0 &= \{2,1\},   & I_1 &= \{3\},\\
&316542, & k &=1 , \quad & I_0 &= \{3\},     & I_1 &= \{2,1\},\\
&514362, & k &=1 , \quad & I_0 &= \{2\},     & I_1 &= \{3,1\},\\
&532164, & k &=1 , \quad & I_0 &= \{1\},     & I_1 &= \{3,2\},\\
&531642, & k &=0 , \quad & I_0 &= \emptyset, & I_1 &= \{3,2,1\}.
%&654321, \quad &&165432, \quad &&541632, \quad &&543216, \\
%&261543, &&264315, &&432615, &&362514.
\end{alignat*}

�� �������� ������������ $\pi$ �������, ��� 
$y_{2i_s - 1, 2i_s} = 0$,  ��� $s \in [k]$, �
$y_{2i'_t - 1, 2i'_t} = 1$, ��� $t \in [n-k]$.
�������������� ����������� \eqref{eq:LOP1}--\eqref{eq:LOP3} ����������� ��������� ������� �������, � ����������� �� �������������� �������� $i,j$ ���������� $I_0$, $I_1$.

�������� ��������������, ��������� ����� $\bm{x} = (x_{ij}) \in \BQP(n)$ � $\bm{y} \in F \subset \LOP(2n)$ �������"=����������� ������������:
\begin{align*}
x_{ii} &= y_{2i-1, 2i}, & i &\in [n],\\
x_{ij} &= y_{2j-1, 2j} - y_{2i, 2j}, & 1 &\le i < j \le [n].
\end{align*}
\end{proof}





\subsection{������������� �������� ��������}
\label{sec:SteinerTree}

����� $G=(V,E)$ "--- �������"=���������� ����, � $T$ "--- ��������� ������������ ��� ������. \emph{������� �������� �� $T$} ���������� ������� $G' = (V',E')$, $T\subseteq V'$, ����� $G$, ���������� �������. ������ �������� (� �����) ����������� � ��������� ������ �������� � ����������� ��������� ����� �����.
\emph{�������������� �������� ��������} ���������� �������� �������� ��������� $\Steiner(G,T) \subseteq \{0,1\}^E$ ���� ������������������ �������� �������� �������� �� $T$~\cite{Chopra:1994}.

\begin{theorem}
	��� ������ ����� $G=(V,E)$ �� �������� ����� ����� ��������� ���� $G'=(V',E')$, $|V'|=|V|+|E|+1$, $|E'|=4|E|+|V|$, � ������� ��������� $T\subset V'$ �����, ���
	$\Stable(G) \lea \Steiner(G',T)$.
\end{theorem}
\begin{proof}
	����� $G=([n],E)$ "--- ����, ������������ ������������ $\Stable(G)$.
	������� 
	\[
	T = \{v'_0\} \cup \Set*{t_{ij} \given i < j, \ \{i,j\}\in E}
	\]
	� ��������� ��������� ������ � ����� ����� $G'=(V',E')$ ��������� �������:
	\[
	V' = T \cup \{v'_1,\dots,v'_n\} \cup \Set{v'_{ij} \given i < j, \ \{i,j\}\in E},
	\]
	\[
	E' = \Set{\{v'_0, v'\} \given v'\in V'\setminus T} 
	\cup \Set{\{v'_{ij}, t_{ij}\}}
	\cup \Set{\{v'_{k}, t_{ij}\} \given k\in \{i,j\}}.
	\]
	����� �������, ������ ������� $t_{ij}$ ���������� ����� ���� ������ 
	\[
	\{v', t_{ij}\}, \quad v' \in V_{ij} = \{v'_{ij}, v'_i, v'_j\}.
	\]
	������ ��� ������� ���� �� ��� ������ ������� � ������ �������� �� $T$.
	�������������, �����������
	\[
	x_{\{v'_{ij}, t_{ij}\}} + x_{\{v'_{i}, t_{ij}\}} + x_{\{v'_{j}, t_{ij}\}} \ge 1
	\]
	��������� ��� ������� $\bm{x} \in \Steiner(G',T) \subset \{0,1\}^{E'}$.
	������� 
	\begin{equation}
	\label{eq:proofSteiner1}
	x_{\{v'_{ij}, t_{ij}\}} + x_{\{v'_{i}, t_{ij}\}} + x_{\{v'_{j}, t_{ij}\}} = 1,
	\end{equation}
	�������� � ������������ ��������������� ����� ������������� �������� ��������.
	
	�����������, ��� ��������� ����� $\{v', t_{ij}\}$, $v' \in V_{ij}$ ������ � ������ �������� �� $T$. 
	�����, � ���� ����, ��� $t_{ij}$ � $v'_0$ ������ ���� �������, � ������� ���� ������ �� $T \setminus \{v'_0\}$ ����� �������, �������� � ������, ��� ����� $\{v', v'_0\}$ ������� �������������� � ���� ������.
	������� �������, $x_{\{v'_0, v'\}} \ge x_{\{v', t_{ij}\}}$, $v' \in V_{ij}$.
	��������, ����������� 
	\[
	x_{\{v'_0, v'\}} = x_{\{v', t_{ij}\}}, \quad v' \in V_{ij},
	\]
	������ � \eqref{eq:proofSteiner1} ���������� ��������� ����� ������������� $\Steiner(G',T)$.
	����� �������, ��� ���������� ������� $\bm{x}$, �������������� ��������� ������ ���� �����, ���������� ������� ����� ���������� $x_{\{v'_0, v'\}}$, $v' \in V' \setminus T$.
	����� ����, ��� ��� ����������� �����������
	\begin{equation*}
	x_{\{v'_0, v'_{ij}\}} + x_{\{v'_0, v'_{i}\}} + x_{\{v'_0, v'_{j}\}} = 1.
	\end{equation*}
	�� ���� ���������� $x_{\{v'_0, v'_{ij}\}}$ ������� ������� �� $x_{\{v'_0, v'_{i}\}}$ � $x_{\{v'_0, v'_{j}\}}$, � ��������� ��������� ������������ �����������
	\begin{equation*}
	x_{\{v'_0, v'_{i}\}} + x_{\{v'_0, v'_{j}\}} \le 1.
	\end{equation*}
	
	����� �������, ��������� ����� ������������� $\Steiner(G',T)$ ����������� ������� ������������ ������������� $\Stable(G)$, ���� �������� $y_i = x_{\{v'_0, v'_{i}\}}$ ��� $\bm{y} \in \Stable(G)$.
\end{proof}



%%%%%%%%%%%%%%%%%%%%%%%%%%%%%%%%%%%%%%%%%%%%%%%%%%%%%%%
%
%  ������������� c ������� ��������� ��������� ������
%
%%%%%%%%%%%%%%%%%%%%%%%%%%%%%%%%%%%%%%%%%%%%%%%%%%%%%%%

\section{������������� c ������� ��������� ��������� ������}
\label{sec:PolytopesSimpleAdj}

\subsection{������������ ������������� ������ � �����������}
\label{sec:3Ass}

����� $S$ "--- �������� ���������.
���������� ������� $\bm{x}\in\R^{S\times S\times S}$
����� ���������� $x(s, t, u)$, ��� $s,t,u\in S$.
\emph{������������� ���������� ������ � �����������} (��� \emph{3-����������}) ����� ���� �������������� ��� ��������� ������ 0/1"~����������������:
\[
\sum_{s\in S} \sum_{t\in S} \sum_{u\in S} c(s,t,u) \cdot x(s,t,u) 
\rightarrow \max,
\] 
\begin{align}
&\sum_{s\in S} \sum_{t\in S} x(s,t,u) = 1 \quad \forall u\in S,\label{eq:ThreeBeg}\\
&\sum_{s\in S} \sum_{u\in S} x(s,t,u) = 1 \quad \forall t\in S,\label{eq:Three2}\\
&\sum_{t\in S} \sum_{u\in S} x(s,t,u) = 1 \quad \forall s\in S,\label{eq:Three3}\\
&x(s,t,u) \in \{0, 1\} \quad \forall s, t, u\in S,\label{eq:ThreeEnd}
\end{align}
��� $c(s,t,u)\in\Z$ "--- ���������� �������� �������.
����� $\TAP(m)$, $m = |S|$, ��������� ��������� ���� �������� $\bm{x}\in\R^{S\times S\times S}$, ��������������� ������������~\eqref{eq:ThreeBeg}--\eqref{eq:ThreeEnd}.
�������� �������� ��������� $\TAP(m)$ ���������� \emph{�������������� ������������� ���������� ������ � �����������}.
�����, � ����� �������� �����, �� ����� �������� ����� <<����������>>.

��-��������, ������� ��������, ������������ �������� ������� ����� �������������, ��������~\cite{Euler:1987} �~\cite{Balas:1989}.
����� ������ ����� ������� �~\cite{Qi:2000}.
� ������� �������� (��.~\cite{Kravtsov:2006} � ������ � ���) ��������� �������� ��������������� ������ ���������� ����� �������������.

��������, $\TAP(m)$ �������� ������� ������� $\Part(A)$:
\begin{equation}
\label{Ine3AP-PART}
\TAP(m) =_A \Part(A),
\end{equation}
��� $A$ "--- $(3m\times m^3)$-�������, ������������ �������������� ����� ������ ��������� \eqref{eq:ThreeBeg}--\eqref{eq:Three3}.
����� �������, ��������� �������������� ������������� ������ � ����������� ������� �������� � �������������� ���������: 
\[
\TAP \propto_A \Part.
\]

��������� ����������� ���������� ����\'���� ������ 3"~������������ � ������ 3-���������~\cite{Garey:1982}, ���� � ������~\cite{AvisTiwary:2015} ��������, ��� ������������ ������ 3"~������������ $\SAT(U,C)$, $|c|=3$, $c\in C$, �������� ��������� ����� ������������� $\TAP(m)$, ��� $m = O(kn)$, $k=|U|$, $n=|C|$.
������, �� \eqref{Ine3AP-PART} � �������~\ref{thm:Class1} �� ��������������� $\Stable$ � $\Part$ �������, ��� $\TAP \propto_A \Stable$.
� ������ �������, ��� ��� ��������� $\DCP$ ������� �������� � �������������� ������ 3"~������������ (�������~\ref{thm:DCPSAT}), � $\Stable$ �� ����� ���� ������� � $\DCP$ (�������~\ref{thm:DCPStable}), �� �������� ����\'���� �������������� ������ 3"~������������ � $\TAP$ ����������.

�������, ��� $\Stable \propto_A \TAP$.
����� �������, $\TAP$ �������� � ����� ������ ��������������� 
%(� ������ $\propto_A$) 
������ � $\Stable$, $\Part$ � $\Pack$.

��� ����� $G(V, E)$ � ����������� ������������� $\Stable(G)$ �����
\[
V_{\text{isol}} = \Set{v\in V \given v\notin e \  \forall e\in E},
\]
��������� ��������� ������������� ������.

\begin{theorem}[\cite{Maksimenko:2016bool}]
	\label{TheSSP-3AP}
	$\Stable(G) \le \TAP(m)$ ��� $m = 3|E| + 2|V_{\text{isol}}|$.
\end{theorem}

\begin{proof}
	��������� $S$ � �����������~$\TAP(m)$ �������� �� ���� ����� ���������:
	\begin{enumerate}
		\item $v$ � $\bar{v}$ ��� ������ ������������� ������� $v\in V_{\text{isol}}$.
		
		\item $e$ ��� ������� ����� $e\in E$.
		
		\item $(e, v)$ ��� ������� $e\in E$ � $v \in e$.
	\end{enumerate}
	�������� ��������� ����� $Q \subset S\times S\times S$ ���, ����� �����
	\[
	F = \big\{x \in \TAP(m) \mid x(q) = 0 \ \forall q \notin Q\big\}
	\] 
	������������� $\TAP(m)$ ���� ������� ������������ ������������� $\Stable(G)$.
	
	��� ������� $v\in V_{\text{isol}}$ ��������� $Q$ ����� ��������� ������ ������:
	\[
	(v, v, v), \quad (\bar{v}, \bar{v}, \bar{v}), \quad
	(v, v, \bar{v}), \quad (\bar{v}, \bar{v}, v).
	\] 
	������ �����, ���������� $v$ ��� $\bar{v}$ � $Q$ ���.
	�������������, ���� $\bm{x} \in F$, �� ��� ������� $v\in V_{\text{isol}}$ �������� ������ ��� ������������: 
	\[
	x(v, v, v) = x(\bar{v}, \bar{v}, \bar{v}) = 1
	\quad \text{���} \quad 
	x(v, v, \bar{v}) = x(\bar{v}, \bar{v}, v) = 1.
	\] 
	
	���������� ������ �������� $e$ � $(e, v)$ ��������� $S$,
	��� $e\in E$ � $v \in e$.
	��� ������ ��������������� ������� $v \in V \setminus V_{\text{isol}}$ ���������� ��������� ����������� �� ����� $E(v) = \{e_{i_1}, \ldots, e_{i_p}\}$, ��� $p=d_G(v)$ "--- ������� ������� $v$.
	��������� ����� $Q$ ��������
	\begin{enumerate} 
		\item $\big( e, e, e \big)$ ��� ������� $e\in E$.
		
		\item $\big( (e,v), (e,v), (e,v) \big)$ ��� ���� $e\in E$ � $v \in e$.
		
		\item $\big( e, e, (e,v) \big)$ ��� ���� $e\in E$ � $v \in e$.
		
		\item $\big( (e_{i_q},v), (e_{i_{q+1}},v), e_{i_q} \big)$
		��� ������ ��������������� ������� $v$ � ������� $e_{i_q} \in E(v)$,
		��� �������� $q+1$ ����������� �� ������ $p$.
	\end{enumerate} 
	
	������� $x(u, v, w) = 0$ ��� ���� ��������� ����� $(u, v, w)\notin Q$,
	�������� � ������������ ��������������� ����� ������������� $\TAP(m)$.
	���������� ��������� �������� � ������.
	
	�������, ��� ��� ������� $(e,v) \in S$ ��������� $Q$ �������� � �������� ��� ������ � ��������� $(e,v)$ � ������� ����������: $\big( (e,v), (e,v), (e,v) \big)$ � $\big( e, e, (e,v) \big)$.
	�������������, ���������~\eqref{eq:ThreeBeg} ��� $u = (e,v)$ ��������� ���
	\begin{equation}
	\label{eq:Proof3AP1}
	x\big( (e,v), (e,v), (e,v) \big) + x\big( e, e, (e,v) \big) = 1.
	\end{equation}
	�� ����, $x\big( (e,v), (e,v), (e,v) \big)$ ���������� ������� ����� $x\big( e, e, (e,v) \big)$.
	
	������� �����, ��� ��� ������� $e \in S$ ��������� $Q$ �������� ����� ��� ������ � ��������� $e$ � ������ ����������: $\big( e, e, e \big)$, $\big( e, e, (e,v_1) \big)$ � $\big( e, e, (e,v_2) \big)$, ��� $e = \{v_1, v_2\}$.
	�������������, ���������~\eqref{eq:Three3} ��� $s = e$ ������������
	$$
	x\big( e, e, e \big) + x\big( e, e, (e,v_1) \big) + x\big( e, e, (e,v_2) \big) = 1.
	$$
	�� ����
	$x\big( e, e, e \big) = 1 - x\big( e, e, (e,v_1) \big) - x\big( e, e, (e,v_2) \big)$
	� 
	\begin{equation}
	\label{IneProof3AP}
	x\big( e, e, (e,v_1) \big) + x\big( e, e, (e,v_2) \big) \le 1.
	\end{equation}
	
	��������� �� ��������, �������� ��������� ���������
	$$
	x\big( (e_{i_q},v), (e_{i_q},v), (e_{i_q},v) \big) 
	+ x\big( (e_{i_q},v), (e_{i_{q+1}},v), e_{i_q} \big) = 1
	$$ 
	��� ������ ��������������� $v$ � ������� $e_{i_q} \in E(v)$,
	��� �������� $q+1$ ����������� �� ������ $p=d_G(v)$.
	������ � ���������� \eqref{eq:Proof3AP1} ��������
	\begin{equation}
	\label{eq:Proof3AP2}
	x\big( (e_{i_q},v), (e_{i_{q+1}},v), e_{i_q} \big) = 
	1 - x\big( (e_{i_q},v), (e_{i_q},v), (e_{i_q},v) \big) =
	x\big( e_{i_q}, e_{i_q}, (e_{i_q},v) \big).
	\end{equation}
	����� ����, ��� ���
	$$
	x\big( (e_{i_{q+1}},v), (e_{i_{q+1}},v), (e_{i_{q+1}},v) \big) 
	+ x\big( (e_{i_q},v), (e_{i_{q+1}},v), e_{i_q} \big) = 1,
	$$ 
	��
	\begin{equation}
	\label{eq:Proof3AP3}
	x\big( (e_{i_q},v), (e_{i_{q+1}},v), e_{i_q} \big) = 
	1 - x\big( (e_{i_{q+1}},v), (e_{i_{q+1}},v), (e_{i_{q+1}},v) \big) =
	x\big( e_{i_{q+1}}, e_{i_{q+1}}, (e_{i_{q+1}},v) \big).
	\end{equation}
	��� ��� ����� ����� ������ \eqref{eq:Proof3AP2} � \eqref{eq:Proof3AP3}
	���������, ��
	$$
	x\big( e_{i_{q+1}}, e_{i_{q+1}}, (e_{i_{q+1}},v) \big) 
	= x\big( e_{i_q}, e_{i_q}, (e_{i_q},v) \big).
	$$ 
	������ $x\big( e, e, (e,v) \big) = x\big( e', e', (e',v) \big)$
	��� ����� ���� ����� $e$ � $e'$, $v \in e$, $v \in e'$.
	
	����� �������, ��� ������ ��������� ����� $F$ ��� ���������� $x(s,t,u)$
	���������� ������� ����� $x\big( e, e, (e,v) \big)$,
	� �����������~\eqref{IneProof3AP} �������� ������������ ���� $y_i + y_j \le 1$ �� �������� ������������� $\Stable(G)$.
\end{proof}


\begin{remark}
	���������� ���������� ����� ���� ����� �������� �� ������ $p$-��������� ������ � ����������� ($p > 3$).
	�� ��������, ������� $\PAP{p}(m)$ ������������� $p$-��������� ���������� ������ � ����������� �������� 0/1-��������� � $\R^{m^p}$.
	���������� $x_{i_1 i_2 \ldots i_p}$, 
	$i_1, i_2, \ldots, i_p \in \{1,2,\ldots,p\}$, ������� ������ ������� ������������� ��������� ����������:
	$$
	%\begin{equation}
	\begin{aligned}
	\sum_{i_2, i_3, \ldots, i_p} x_{i_1 i_2 \ldots i_p}      &= 1 \quad \forall i_1\in \{1,\ldots,p\},\\
	\sum_{i_1, i_3, i_4, \ldots, i_p} x_{i_1 i_2 \ldots i_p} &= 1 \quad \forall i_2\in \{1,\ldots,p\},\\
	\ldots\ldots & \ldots\\
	\sum_{i_1, i_2, \ldots, i_{p-1}} x_{i_1 i_2 \ldots i_p}  &= 1 \quad \forall i_p\in \{1,\ldots,p\}.\\
	\end{aligned}
	$$
	%\end{equation}
	��������,
	$$
	\PAP{p}(m) \lea \Part(A), 
	$$
	��� ��������� (������ ������������) ������� $A\in \{0,1\}^{pm\times m^p}$.
	� ������ �������, ���������
	$$
	x_{i_1 i_2 \ldots i_p} = 0 \quad \forall i_p \ne i_{p-1}
	$$
	���������� ����� ������������� $\PAP{p}(m)$, ������� ������������� $\PAP{(p-1)}(m)$.
	����� �������, �� �������~\ref{TheSSP-3AP},
	\[
	\Stable(G) \lea \PAP{p}(m) \quad \text{��� } m = 3|E| + 2|V_{\text{isol}}| \le 3|V|(|V|-1)/2.
	\]
\end{remark}


\subsection{������������� ������������ ������ �������� ��������������
	 �~������������ ������ � �����������}
\label{sec:QLOP}

%����� �����, ������ � ������������ �������������� 
% ���������� <<������������ ������� ��������� ��������������>> \cite{Buchheim:2010}.
%�� ����, ��-��������, �������� ����������� ������� ������ ��������� ��������������.

����, � ������� \ref{sec:LOP}, ��� ��������� ������������ �������� �������� $\LOP(m)$. ������������ ������������ ������ ��������� �������������� �������� �~��������� ������ ��� �����������.

�����
\[
I = \big\{(i,j,k,l) \ | \ i<j, \ k<l, \text{ � } (i,j) \prec (k,l) \big\},
\]
��� $(i,j) \prec (k,l)$ �������� ���� $i<k$, ���� $i=k$ � $j < l$.
��� ������ ���� ��������� ���������� $y_{ij}$ � $y_{kl}$ �������� ����� ����������
\begin{equation}
\label{eq:QLOP}
z_{ijkl} = y_{ij} y_{kl}, \quad (i,j,k,l) \in I.
\end{equation}
��������� ����� $\QLOP(m)$ ��������� ���� �������� $\bm{z}\in\{0,1\}^d$, 
$d = \binom{m}{2} \left( \binom{m}{2} + 1 \right) / 2$,
� ������������ $y_{ij}$ � $z_{ijkl}$, ��������������� ������������ \eqref{3cycle} �~\eqref{eq:QLOP}.
�������� �������� ��������� $\QLOP(m)$ ���������� \emph{�������������� ������������ ������ ��������� ��������������}~\cite{Buchheim:2010}.

\begin{theorem}[�������, ������ � ����~\cite{Buchheim:2010}] 
	$\QLOP(m) \lea \BQP(n)$ ��� $n = \binom{m}{2}$.
\end{theorem} 

�������, ������ � ����~\cite{Buchheim:2010} ���������� ���� ���� � ��������� ������ � ��������� ��� ������� ������������ ������ ��������� ��������������.

�������, ��� �������� ����\'���� � �������� ������� ����� ��������.

\begin{theorem}[\cite{Maksimenko:2016bool}]
	\label{ThBQP-QLOP}
	$\BQP(n) \lea \QLOP(m)$ ��� $m = 2n$.
\end{theorem} 

\begin{proof}
	���� �������������� ������.
	�������, ��� ����� ������ $\LOP(m)$ ���� $n$"~������ ���.
	��� �������� �� $\LOP(m)$ � $\QLOP(m)$ ���� ��� ������������� � ����� ������������ ������������.
	
	%����� $m = 2n$. 
	������������� ���, ��� ��������� $y_{ij} = 0$ � $y_{ij} = 1$ ���������� ������� �������������� ��� $\LOP(m)$ � $\QLOP(m)$.
	�����
	\[
	J = \{(2i-1,2i) \mid i\in[n] \}.
	\] 
	������� 
	\begin{equation}
	\label{FaceQube}
	y_{ij} = 0 \quad \text{��� ���� } (i,j)\notin J, \ 1 \le i < j \le m.
	\end{equation}
	������������������ �������� ������ ���������� $y_{ij}$, ��� $i$ ������� � $j = i+1$.
	������� �������� �� 3-��������� �����������~\eqref{3cycle}.
	����������� $i < j < k$, ����� ��� ������:
	\begin{enumerate} 
		\item ���� $(i,j)\notin J$, �� $y_{ij} = y_{ik} = 0$.
		����� �����������~\eqref{3cycle} ������������� � $0 \le y_{jk} \le 1$. 
		\item ���� $(i,j)\in J$, �� $i$ �������, $j = i+1$ ����� � $k > i+1$.
		�������������, �����������~\eqref{3cycle} ������������ $0 \le y_{ij} \le 1$.
	\end{enumerate} 
	����� �������, $n$ ���������� $y_{i\, i+1}$, ��� $i$ �����, 
	����� ��������� �������� 0 ��� 1 ���������� ���� �� �����.
	�������������, ��������������~\eqref{FaceQube} ���������� ����� $\LOP(m)$, ���������� $n$"~������ �����. 
	
	������� �������� �� ���������� $z_{ijkl}$, $(i,j,k,l) \in I$.
	���� $(i,j) \notin J$ ��� $(k,l) \notin J$, �� $z_{ijkl}=0$.
	� ������ $(i,j) \in J$ � $(k,l) \in J$ ����� $z_{ijkl}=y_{ij} y_{kl}$, 
	�, ����� ����, $y_{ij}$ � $y_{kl}$ �� ������� �������� ������� �������������.
	
	����� �������, ��������� ����� $\QLOP(m)$ � ������������ $\BQP(n)$ ������� ��������� �������� ������������� ������������:
	$$
	\begin{aligned}
	x_{ii} &= y_{2i-1,\, 2i}, \quad 1\le i\le n,\\
	x_{ij} &= z_{2i-1,\, 2i,\, 2j-1,\, 2j} = y_{2i-1,\, 2i} \cdot y_{2j-1,\, 2j}, \quad 1\le i < j\le n,
	\end{aligned}
	$$
	��� $\bm{y} \in \QLOP(m)$, $\bm{x} \in \BQP(n)$.
	%Let $x_{ii}$ in the~equation~\eqref{BQP} be equal to $y_{2i-1, 2i}$, $1\le i\le n$.
	%Let $x_{ij} = z_{2i-1, 2i, 2j-1, 2j}=y_{2i-1, 2i} \cdot y_{2j-1, 2j}$, $1\le i < j\le n$.
	%Thus the~face of $\QLOP(m)$ is affinely equivalent to $\BQP_n$. 
\end{proof}

��� ������������� ������������ ������ � ����������� ������� ����������� ����� ��������.

��������� ������ $\Birk(m)$ ������������� ������ � ����������� (������������� ��������)
������� �� �������� $\bm{y}\in\{0,1\}^{m\times m}$, ��������������� ��������
\begin{align}
\sum_j y_{ij} = 1, \ \forall i \in [m], \label{eq:Ass1}\\ 
\sum_i y_{ij} = 1, \ \forall j \in [m]. \label{eq:Ass2}
\end{align}
����� ���������� ������������ ������������ ������ � �����������,
������ ����� ���������� $z_{ijkl}$ ��� ��, ��� ��� ���� ������� �~\eqref{eq:QLOP}:
%For every pair $y_{ij}$ and $y_{kl}$ there is introduced a~new variable
\begin{equation}
\label{eq:QAP}
z_{ijkl} = y_{ij} y_{kl}, \text{ ��� } (i,j) \prec (k,l).
\end{equation}
��������� ����� $\QAP(m)$ ��������� ���� �������� $\bm{z}\in\{0,1\}^d$, $d = m^2 + \binom{m^2}{2}$,
� ������������ $y_{ij}$ � $z_{ijkl}$, ��������������� �������� \eqref{eq:Ass1}, \eqref{eq:Ass2} �~\eqref{eq:QAP}.
�������� �������� ��������� $\QAP(m)$ ���������� \emph{�������������� ������������ ������ � �����������}.
� ���������� ����� ����������� �������� \emph{������������� ������������ ��������������}~\cite{Saito:2009}.
��� ����������� ��� ��������� ������ $\QSAP(m)$ �������~\eqref{eq:Ass2} ����������.

\begin{theorem}[\cite{Rijal:1995, Kaibel:1997, Saito:2009}] 
	\label{thm:QAP2BQP}
	$\QAP(m) \lea \QSAP(m) \lea \BQP(n)$ ��� $n = m^2$.
\end{theorem} 

��� ����� ������������ �~\cite{Kaibel:1997} ��� ������ ����������, ���������� ��� $\QAP(m)$.
� ���������, ��� ��� $\BQP(n)$ 2"~����������, �� $\QAP(m)$ ���� 2"~����������.
�~\cite{Kaibel:1997} ����� ��������, ��� ������������ �������� �������� $\LOP(m)$ � ������������ ������������ $\TSP(m)$ �������� ���������� ��������� ������ ������������� $\QAP(m)$.
%\[\LOP(m) \leE \QAP(m),  \quad  \TSP(m) \leE \QAP(m).\] 
�������, ��� �������� ���������� $\LOP \propto_A \QAP$ � $\TSP \propto_A \QAP$ ����������, ��� ��� $\LOP(m)$ �� 2"~���������� ��� $m \ge 3$~\cite{Young:1978},
� $\TSP(m)$ �� 2"~���������� ��� $m \ge 6$~\cite{PadbergRao:1974}.

\begin{theorem}[\cite{Maksimenko:2016bool}]
	$\BQP(n) \lea \QAP(m)$ ��� $m = 2n$.
\end{theorem} 

\begin{proof}
�� �������� � ��������������� �������~\ref{ThBQP-QLOP}, ���������� ��������, ��� ������������ �������� $\Birk(2n)$ �������� �����, ���������� $n$"~������ �����.
(����� �� ���������� ��� �� ����, ��� � ��� �������������� �������~\ref{thm:KnapEqCAP}.)
�������
\[
S = \big\{(i,i) \mid i\in[2n]\big\} \cup \big\{(2i-1,2i) \mid i\in[n]\big\} \cup \big\{(2i,2i-1) \mid i\in[n]\big\}.
\] 
����� ���������
\[
y_{ij} = 0 \quad \forall (i,j) \notin S
\]
���������� ��������� �����.
\end{proof}

����� �������, ��������� �������������� $\QLOP$, $\QAP$ � $\QSAP$ ��������� � ����� ������ ��������������� (� ������ $\propto_A$) ������ � $\BQP$ � $\Cut$.


%� ������� �� ����������� �������� ����� ������ ������������ ���� $\phi_n$, ������������ ��������� �������~\cite{Onn:2009}. ��� ������ ������������ ������ ������� ����� $K_n$ ���������� ��������������� $\binom{n}{2}\times\binom{n}{2}$-������� ������������ ����� ����� �����. ����� $\phi_n$ ��������� ��������� ���� ����� ������ ��� ������������� $n\in\N$, $n\ge 3$. �������� �������� ��������� $\phi_n$ �������� ������� �������� ������������� ���� � �~\cite{Onn:1993} ������������ $P((n-2,2))$, ��� �� ��������, ��� ���� ������������ 2-����������.
%����� ����, ����� ��������, ��� $\phi_n$ �������� ��������� $\QAP(n)$ (��. �����������~\ref{prop:1} ����).

%�~\cite{Onn:2009} ��� ������������� ������ � ���, �������� �� ������������� $\phi_n$ � $\QAP(n)$, ������� ���������� ����� ������, �����������.

%���� �� �������, ��� ������������ $\phi_n$ �� �������� 3-�����������. ��� $n=3$ ��� ����������� ����������� ���������������, � ��� $n > 3$ ����� �������� ���������

%\begin{lemma}	\label{lem:1}
%������������ $\phi_3$ ������� ������������ ����� ������������� $\phi_n$.
%\end{lemma}

%� ������ �������, �� �������~\ref{thm:QAP2BQP} � 3-��������� ������������� $\BQP(n)$ �������, ��� $\QAP(n)$ 3-����������. ����� �������, $\phi_n$ � $\QAP(n)$ �� ����� ���� ���������.

%������� � �������������� $\phi_n$ � $\QAP(n)$, � ���������, ���������� ���, ��� ������������ ������������ $\TSP(n)$ �������� ��������� ��� ������� �� ���. ��������, ��� ����� ������������ ������������ $\BQP(k)$ �������� ������ ������������� $\TSP(n)$ ��� $n = 2k(2k-1)$~\cite{Maksimenko:2013TSP}, �������� � ������, ��� $\BQP(k)$ �������� ��������� ��������� ����� ������������� $\phi_n$ (������������� ���� $P((n-2,2))$). ����� ����, �����������

%\begin{theorem}	\label{thm:2}
%������������ $\BQP(k)$ ������� ������������ ����� ������������� $\phi_{2k}$.
%\end{theorem}

%� ���������, $\phi_n$ �������� $3$-����������� ����� �� $2^{\lfloor n/2\rfloor}$ ��������.

%%%%%%%%%%%%%%%%%%%%%%%%%%%%%%%%%%%%%%%%%%%%%%%%%%%%%%%
%
% ������������� ��������� ������ �����
%
%%%%%%%%%%%%%%%%%%%%%%%%%%%%%%%%%%%%%%%%%%%%%%%%%%%%%%%

\subsection{������������� ��������� ������ �����}
\label{sec:Color}

����� �� �������� ��������� NP-������ ����� �������� ������ � ��������� ������ �����~\cite{Karp:1972}, ���������������� ����, �� �.~\hyperlink{pColor}{\pageref*{problem:Color}}. 
��� ����� �������������� ����������: �� ����������� ���������� � ������������� ������ � ����� ������� �����, �� ������������� �������� � ������������ � ������������� �����������������~\cite{Burke:2010,Palubeckis:2008}.
� ���������� ����� ����� ��������� ������������� ������������ ���� �� ����� ����������� �������������� ��������� �����.
���������� ��������� �������� ������������ �����������, ��������, � ���������, ��, ��� ����������� ������������� ��������������� ������������ ����� ������ �����. 

����� ������������, ��� �������������� ���� $G=(V,E)$ �� �������� ������������� ������ (��������� ������� ����������).
����� �������� �� ������������ ������ ���������������� ���������, ����� � ����� $G$ ���� �������, ������� �� ����� ���������� (����� �� ���� ������ ���������� �� ������ ���� ��������� ������).

%, � ����� ������, ����� ���� $G$ �������� ���� ������, �� ������� ����� �����, �� ������� �� ����� ���������� ��������� (����� �� ��� ������� ��������� ������, ������������ �� ������ ��������� ������).



\subsubsection{����������� ������������}

��� ������ ��������� ����� $G=(V,E)$ ���������� � ������������������ ������ $\bm{x} \in \{0,1\}^{V\times k}$, $1 \le k \le |V|$, ���������� �������� ��������� ��������� �������:
\[
x_{v,c} = \begin{cases}
1, & \text{���� ������� $v$ ���������� ������ $c$,}\\
0, & \text{�����.}
\end{cases}
\]
��������� ���� ����� �������� ��������� $\ColorA(G,k)$.
�������� ��������, ��� ��� ��������� ����� ���� ������� ��������� ������� �����������~\cite{Mehrotra:1996,Delle:2016}:
\begin{align}
x_{v,c} + x_{u,c} &\le 1, \qquad \{v,u\} \in E, \quad c\in[k], \label{eq:colora1}\\
\sum_{c=1}^k x_{v,c} &= 1, \qquad v\in V, \label{eq:colora2}\\
x_{v,c} &\in \{0,1\}, \qquad v\in V, \quad c\in[k]. \notag
\end{align}
����� �����������~\eqref{eq:colora1} ��������� ���������� ������� ������� ����� ������,
� ���������~\eqref{eq:colora2} ��������, ��� ������ ������� ����� �������� ����� ����� ������.
��������, ��������� $\ColorA(G,k)$ �� ����� ����� � ������ �����, ����� ������� ����� $G$ ����� ���� ���������� � $k$ ������.
� ����������� �� ���� ��������� �������� ������� 
\[
f(\bm{x}) = \sum_{v\in V, \ c\in[k]} x_{v,c} |V|^c
\]
����� ����� ��������� � ���������� ������ ������.

\begin{prop}\label{prop:ColorA}
	$\Stable(G) \lea \ColorA(G,|V|)$ ��� ������ ����� $G=(V,E)$.
\end{prop}
\begin{proof}
	��� �������� ������� $V = [n]$.
	���������� ����� ������������� $\ColorA(G,|V|)$, ������� � ����������� ���������������
	\[
	x_{v,c} = 0, \quad \text{��� $v \ne c$ � $c \ne 1$.} 
	\]
	����� ���������~\eqref{eq:colora2} ������ ���
	\[
	x_{v,1} + x_{v,v} = 1, \qquad v\in [n],
	\]
	� �� ����������~\eqref{eq:colora1} �������������� ��������� ����
	\[
	x_{v,1} + x_{u,1} \le 1, \qquad \{v,u\} \in E.
	\]
	��������, ��� ����� ������� ������������ ������������� $\Stable(G)$.
\end{proof}

������������ �������� � ���������� (��., ��������, \cite{Mendez:2008,Burke:2010}) ��������� ��������� ����������� ����� �������������.
������� $V = [n]$ � $k = n$, � �����������~\eqref{eq:colora1} ������� �� 
\begin{align*}
x_{v,c} + x_{u,c} &\le w_c, \qquad \{v,u\} \in E, \quad c\in[n],\\
w_c &\in \{0,1\}, \qquad c\in[n].
\end{align*}
(���� ���������� $w_c$ ����� ����, �� ���� $c$ �� ������������.)
��������� ��������� �������� $\bm{x} \in \{0,1\}^{n(n+1)}$, ��������������� ���� ������������, ����� $\ColorB(G)$.
��� ����� ������� ������� ������� ��� ������ ������ �������������� ����� ����� $G$ ����������� �������� ������� ���:
\[
\sum_{c\in[n]} w_c \to \min.
\]

��������, 
\[
\ColorA(G,k) \lea \ColorB(G).
\]
���������� �������� $w_c = 1$ ��� $c\in[k]$ � $w_c = 0$ ��� $c > k$.

������� ������, ��� ��������� �������������� $\ColorA$ � $\ColorB$ ����� � ����� ������ ��������������� � ��������������� ����������� ��������.
��� ��� $\Stable \propto_A \ColorA \propto_A \ColorB$, �� �������� �������� $\ColorB \propto_A \Stable$.

\begin{prop}\label{prop:ColorB}
	��� ������ ����� $G=([n],E)$ �� �������� ����� ����� ��������� $G'=(V',E')$, $|V'| = n(n+2)$, $|E'|=n^2(n+1)/2+(|E|+1)n$, ��� 
	\[
	\ColorB(G) \lea \Stable(G').
	\]
\end{prop}
\begin{proof}
	���������� ������� �����������
	\begin{align}
	\sum_{j=1}^n x_{ij} &= 1, \qquad i\in[n], \label{eq:proofColorB1}\\
	x_{ij} + x_{kj} &\le w_j, \qquad \{i,k\} \in E, \quad j\in[n], \label{eq:proofColorB2}
	\end{align}
	����������� ������������ $\ColorB(G)$,
	������������� � ������� ����������� ���� $y_i + y_j \le 1$.
	
	�������, ������ �����, ��� �����������~\eqref{eq:proofColorB2} ������������ ������������ 
	\begin{align}
	x_{ij} &\le w_j, \qquad i,j\in[n], \label{eq:proofColorB2a}\\
	x_{ij} + x_{kj} &\le 1, \qquad \{i,k\} \in E, \quad j\in[n], \notag
	\end{align}
	��� ������� ��������������� ����������.
	
	������ �������������� ���������� $\bar{w}_j \in \{0,1\}$, $j\in[n]$, � ���������� ������� �����������
	\begin{align}
	x_{ij} + x_{ik} &\le 1, \qquad i,j,k\in[n], \quad j < k, \notag \\
	w_j + \bar{w}_j &\le 1, \qquad j\in[n], \notag\\
	x_{ij} + \bar{w}_j &\le 1, \qquad i,j\in[n], \label{eq:proofColorB5}\\
	x_{ij} + x_{kj} &\le 1, \qquad \{i,k\} \in E, \quad j\in[n]. \notag %\label{eq:proofColorB6}
	\end{align}
	��������, ��� ���������� ������������ $\Stable(G')$ ��� ���������� ����� $G'=(V',E')$, ��� $|V'| = n(n+2)$, $|E'|=n^2(n+1)/2+(|E|+1)n$.
	������� $w_j + \bar{w}_j = 1$, �������� � ������������ ��������� ����� ����� �������������.
	�����������~\eqref{eq:proofColorB5} ��� ���� ����������� �~\eqref{eq:proofColorB2a}.
	%\[x_{ij} \le w_j, \qquad i,j\in[n].\]
	%� ������������ � ������������~\eqref{eq:proofColorB6} ��� ������������ �����������~\eqref{eq:proofColorB2}.
	�������� ��������, ��� ���������~\eqref{eq:proofColorB1} ���������� ����� ���� �����, ������� ������������� ������������� $\ColorB(G)$.
\end{proof}

\begin{remark}
	�� ������ ��� �� ���� �~\cite{Delle:2016} ��������, ��� $\ColorA \propto_A \Stable$. ��� �� ���� ���� ������������ ��� ������ ������ ��������� ���������� ���������� ��� �������������� ��������� $\ColorA$.
\end{remark}

\subsubsection{������������ � ���������������}

����� $G=(V,E)$ "--- �������������� ���� � $V=[n]$. 
������ ����������� $\bar{E} = \Set{\{i,j\} \subset V \given \{i,j\}\notin E}$.
��� $i\in V$ ��������� $\bar{N}(i) = \Set{j\in V \given \{i,j\}\in \bar{E}}$.
��� ������������� $U \subseteq V$ ����� $E(U)$ ���������� ��������� ����� ����� $G$, ��� ����� ������� ����� � $U$, � ����� $G[U] = (U, E[U])$ "--- ������� ����� $G$, �������������� ���������� $U$.

������ ������� $i\in[n]$ ����� $G$ �������� � ������������ ���������� $x_{ii}$, � ������ ���� ������ $\{i,j\} \in \bar{E}$, $1 \le i < j \le n$, "--- ���������� $x_{ij}$ 0/1"~������� $\bm{x}$.
�������� ���� ���������� ����� ��������� �����.
��� ������������ ��������� ����� $G$ ���������� $x_{ii}$ ����� �������, ���� ������� $i$ ����� ���������� ����� ����� ���� ������, ���������� ��� �� ������,
�� ���� ��� ������� �������� �� \emph{��������������}.
� $x_{ij}=1$, ���� $i$ � $j$ �������� � ���� ���� � $i$ �������� �������������� ��� $j$.

�~\cite{Campelo:2008} � �������� ������������� ��������� ������ ����� $G$ ������������ ����������� �������� �������� ��������� 0/1"~�������� $\bm{x}$, ��������������� ������������
\begin{align}
x_{ii} + \sum_{j<i, \ j\in\bar{N}(i)}^n x_{ji} &= 1, \qquad i\in[n], \label{eq:ColorC1}\\
x_{ij} + x_{ik} &\le x_{ii}, \qquad \{j,k\} \in E(\bar{N}(i)), \quad i < j < k, \label{eq:ColorC2}\\
x_{ij} &\le x_{ii}, \qquad \text{���� $j$ ����������� � $G[\bar{N}(i)]$, $i<j$.}\label{eq:ColorC3}
\end{align}
������ ����������� �����������, ��� ������ ������� ���� ���� �������� ��������������, ���� ������������ �������� � ������� �������.
������ ����������� ��������� ������� �������� ����� ������ �������������.
��������� ��������� ���� ����� �������� $\bm{x}$ ����� $\ColorC(G)$.
����� ������� �����
\begin{equation}
\label{eq:ColorC4}
\sum_{i\in[n]} x_{ii}
\end{equation}
�� ���� $\bm{x} \in \ColorC(G)$ ��������� � ������������� ������ ����� $G$.

�������� ��������������� ����������, ����������� \eqref{eq:ColorC2} �~\eqref{eq:ColorC3} ����� �������� ����������:
\begin{align}
x_{ij} + x_{ik} &\le 1, \qquad \{j,k\} \in E(\bar{N}(i)), \quad i < j < k, \label{eq:ColorC5}\\ 
x_{ij} &\le x_{ii}, \qquad j \in \bar{N}(i), \quad i<j. \label{eq:ColorC6}
\end{align}

�������, ��� �������� ������������� $\ColorC(G)$ ����� ������ �� �������� ������������� $\ColorB(G)$.
������� ��������� ���� ������������ ����� �����, ��� � �����������~\ref{prop:ColorB}.

\begin{prop}\label{prop:ColorC}
	��� ������ ����� $G=([n],E)$ �� �������� ����� ����� ��������� $G'=(V',E')$, $|V'| = 2n+|\bar{E}|$, $|E'|=O(n^3)$, ��� 
	\[
	\ColorC(G) \lea \Stable(G').
	\]
\end{prop}

������� ������, ��� $\Stable \propto_A \ColorC$.

\begin{prop}
	$\Stable(G) \lea \ColorC(G')$ ��� ������ ����� $G=([n],E)$ � $G'=([n]\cup\{0\}, E)$.
\end{prop}
\begin{proof}
	���������� ������������ $\ColorC(G')$, ��� $G' = (V', E)$ � $V' = \{0,1,\dots,n\}$.
	�� �������� � ��������������� �����������~\ref{prop:ColorA},
	������� $x_{ij} = 0$ ��� ���� $0 < i < j$, $\{i,j\} \in \bar{E}$
	� �������� � ������������ ��������������� ����� ������������� $\ColorC(G')$.
	��� ��������� ��������� ����� ��������� �����������
	\begin{align*}
	x_{00} &= 1,\\
	x_{ii} + x_{0i} &= 1, \qquad i\in[n],\\
	x_{0i} + x_{0j} &\le 1, \qquad \{i,j\} \in E, \quad i < j.
	\end{align*}
	����� �������, ��������� ����� ������� ������������ ������������� $\Stable(G)$.
\end{proof}

�~\cite{Palubeckis:2008,Cornaz:2008} ���������� ���������� ����������� ������������� $\ColorC(G)$, ���������� �� ���� �������� ������ ����������
(��. ���������~\eqref{eq:ColorC1})
\[
x_{ii} = 1 - \sum_{j<i, \ j\in\bar{N}(i)}^n x_{ji}, \qquad i\in[n]. 
\]
��� ���� ���������~\eqref{eq:ColorC1} ���������� �������������
\[
x_{ji} + x_{ki} \le 1, \qquad j,k\in\bar{N}(i), \quad j < k < i,
\]
�����������~\eqref{eq:ColorC5} ����������� � ���������� ����, �����������~\eqref{eq:ColorC6} ������������� �
\[
x_{ij} + x_{ki} \le 1, \qquad j,k \in \bar{N}(i), \quad k < i < j, 
\]
� ������� �������~\eqref{eq:ColorC4} ��������� ���
\[
n - \sum_{\{i,j\}\in \bar{E}, \ i<j} x_{ij}.
\]
(������� ������������ ������� ����� ���������������� ��������, ����� ���� $G$ �������� ���� ������ $i$ � $j$, �� ������� ����� �����, �� ������� �� ����� ���������� ���������. � ���� ������ ������� �������� $x_{ij} = 1$.)
��������, ��� ����������� �������� �������������� ����������� �������� $\Stable(H)$, ��� ���� $H$ ����� $|\bar{E}|$ ������, � ��� ����� ������������ ���������� ���� �������������.

����� �������, ��� ������������� ���� ��������� �������������� ��������� ����� ������������ (� ������ �������� ����������) ��������� �������������� ����������� ��������.



%%%%%%%%%%%%%%%%%%%%%%%%%%%%%%%%%%%%%%%%%%%%%%%%%%%%%%%
%
% ������������� ������ ������������
%
%%%%%%%%%%%%%%%%%%%%%%%%%%%%%%%%%%%%%%%%%%%%%%%%%%%%%%%

\section{������������� ������ ������������}
\label{sec:TravellingAll}

\subsection{������������� ������������ ������}
\label{sec:Travelling}

����� $G = ([m],E)$ "--- ������ ����. ��� ������� ������������ ����� $H \subset E$ � ���� ����� ���������� ��� ������������������ ������ $\bm{y} \in \{0,1\}^E$ � ������������
\[
y_{ij} = \begin{cases}
1,& \text{���� ����� $\{i,j\}$ ������ � $H$,}\\
0,& \text{�����.}
\end{cases}
\]
�����, �� �.~\pageref{def:TSP} ������������ ������������� ������ (������������ ������������ ������ ������������) ��� ��������� ��� �������� �������� ��������� $\TSP(m)$ ���� ����� ��������.
����� �� ������� ��������� � ������������ ������������� �������� (������������ ������������� ������ ������������), �������������� ����� �������� �������� ��������� ���� ������������������ �������� $\ATSP(n) \subset \{0,1\}^A$ ������������� �������� � ������ ������� $D = ([n],A)$.

� �������� �������������� $\TSP$ � $\ATSP$ ���� �������� � ������
(��., ��������, \cite{Junger:1995TSP, BondBook:1995}), ����� ���� ���� ���� ����������� ����� ���. 

\begin{lemma}
\label{lem:ATSP2TSP}
$\ATSP(n) \lea \TSP(m)$ ��� $m=2n$.
\end{lemma}

\begin{proof}
����� $m=2n$.
������������� ���, ��� �������������� ���� $y_{ij}=0$ � $y_{ij}=1$ �������� �������� ��� ������������� $\TSP(m)$.
����� ����, �������������� $y_{ij}=0$ � $y_{ij}=1$ ��������� ��������� ������ ������������� $\TSP(m)$ �� ��� ������������, ��������������� ������������� ������, ���������� ��� �� ������������ ����� $\{i,j\}$.
���� ������������ �������������, ����� ���������� �����, ������� ������������� ������������� $\ATSP(n)$.
	
�������� ��������� ������ $[m]$ �� ��� ����������� ������������ $T = \{1, 2, \ldots, n\}$ � $U = \{n+1, n+2, \ldots, 2n\}$.
���������� ����� $S$ ������������� ������, ������ �� ������� �������� ���������� ����������:
\begin{enumerate}
	\item �� �������� �� ������ ���� $\{i,j\}$ � $\{n+i,n+j\}$, $1\le i < j \le n$.
	\item ����������� �������� ����� $\{i,n+i\}$, $i \in [n]$.
\end{enumerate}
� ���� ���������� ���� ��������� ��� ����� ������������� ��������� ������ ��������� ����� ������������� $\TSP(m)$.
�������, ��� ����� ����������� ���� �� ������ $S$ ����� $n$ <<������������>> ����� ������ ��������� ��� $n$ ����� ���� $\{i,n+j\}$, $i, j \in [n]$, $i\neq j$. 
�������� � ������������ ������� ����� $\{i,n+j\}$ ����� $G=([2n],E)$ ������������ ������ ���� $(i,j)$ ������� $D=([n],A)$ ������������� ������.
�������� ���������, ��� ����� ������� ����� �����������	�������"=����������� ������������ ����� �������������� ������� ������ $S$ � �������������� ��������� ������� $D$.
������ �������, ��� ��������� ����� ������������� $\TSP(m)$ ������� ������������ ������������� $\ATSP(n)$.
\end{proof}

� ������ �������, ������������ $\ATSP(n)$ ����� ���� ������������ � $\TSP(m)$,
��� $m=n$, � ������� ���������� ��������� ����������� 
$y_{ij} = x_{ij} + x_{ji}$, $1\le i < j \le n$.
��� ���� ������ ���� �������������� ������������ �������������
�������� ����� ������������ � ���� ����������� ����.

���� ��� ������������ ����� �������� ���� ������������� ������������� ������.
������ ��� ����� ���� ����� ���������� � �� ������������ ������.
�� ����� ����� ������������ ������������ $\ATSP(D')$ ��� ��������� ���� ������������������ �������� ������������� �������� �������� $D'$ ������� ������� $D$. ��������, $\ATSP(D') \lea \ATSP(n)$, ���� ����� ������ ������� $D'$ �� ����������� $n$.

\begin{theorem}
\label{thm:SAT2TSP}
����� $U=\{u_1,\dots,u_d\}$ "--- ����� ������� ����������, $C=\{C_1,\dots,C_m\}$ "--- ����� ���������� ��� $U$, $\len(C)$ "--- ��������� ����� ���� ���������� �� ������ $C$, ���������� � ���������.
�����
\[
\SAT(U,C) \lea \ATSP(n), \quad \text{��� } n = |U| + 2 \len(C).
\]
\end{theorem}

\begin{proof}
���������� ��� ������� ������������� $\SAT(U, C)$ �������� ������ �������� $D'$ ������� ������� $D=(V,A)$ ������, ��� $\SAT(U, C)$ �������� ������� ������������ ������������� $\ATSP(D')$.
� ���� �����, ��� ������ ����������� ������� ������������ ������ ����������~\cite{Garey:1982} �� ������������� �������� �������� ������ ������������ � ������ ����������� ������.
��� ���� ����� ����������� ���������� ������� ���� ����� ����������� �������"=����������� ������������.
�� �� ������������ ����������� � ����� ��������� �������������� ���� �����.
����� ��������� ���� ��������, ��� ��� ������������ ����� �������� ������������� �����������.
	
������� ������ $D'$ ����� �������������� �� ������ $v_i$,	$1\le i \le d$, � ��������� $D_j$, $1\le j \le m$.
������� $v_i$ ����� ��������������� ���������� $u_i \in U$,
� ���������� $D_j$ "--- ����������� $C_j$.
�� ������ ������� $v_i$ � ������� $D'$ ����� ��������	����� ��� ����, ��������������� ��������� <<������>> � <<����>>	���������� $u_i$. 
������������ ������� $v_i$ �� ����� ��������� ��������� $D_j$, $1\le j \le m$, ������� ��� ������������:
$H_i$ ����� ��������� ��� �� ����������, ��������������� ���������� ������� �������� ������� $u_i$;
$\bar{H}_i$ ����� ��������� ��� �� ����������, ��������������� ���������� ������� �������� ������� $\bar{u}_i$.
�� ������� $v_i$ ���� <<������>> �������� � <<������>> (������� �� �����) ���������� �� ��������� $H_i$.
����� <<������>> ���������� �������� ����� �� <<������>> ����������� �� $H_i$ � �.\,�.
����� �������, ��� ���������� �� $H_i$ ����� ��������� ��������������� �����. 
�� ��������� ���������� ��������������� ���� �������� � ������� $v_{i+1}$ (�������� �� ������ $d$).
���� ��������� $H_i$ ������, �� ���� <<������>> �� ������� $v_i$ �������� ��������������� � $v_{i+1}$.
����������� ���������� ��������� � ���������� $\bar{H}_i$, ����� ������� � ���� <<����>>, ��������� �� $v_i$ (��., ��� �������, ���.~\ref{fig:SATTSP}).
����, � ������ ���������� $D_j$ ������ ������ � ������� ����� ������� ���, ������� ��������� ���������� � ��������������� ��������� $C_j$.
		
\begin{figure}[tbh]
\centering
\tikzset{small circle/.style={inner sep = 1.5pt,draw,circle}}
\begin{tikzpicture}[scale=1.9,
>={Stealth[scale width=0.8]} % ���������� ��� �������
]
%\node[small circle,densely dotted] (u) at (-2,0) {};	
\node[small circle] (v3) at (0,1) {};	
\node[small circle] (v2) at (0,0) {};	
\node[small circle] (v1) at (0,-1) {};	
\node[small circle] (v12) at (2.2,0) {};	
\node[small circle] (w12) at (3.2,0) {};	
\node[small circle] (v11) at (2.2,-1) {};	
\node[small circle] (w11) at (3.2,-1) {};	
\node[small circle] (v23) at (5,1) {};	
\node[small circle] (w23) at (6.5,1) {};	
\node[small circle] (v22) at (4.5,0) {};	
\node[small circle] (w22) at (6.0,0) {};	
\node[small circle] (v21) at (5,-1) {};	
\node[small circle] (w21) at (6.5,-1) {};	
\node[small circle,densely dotted] (w1) at (8,1) {};	
\node[small circle,densely dotted] (w3) at (8,0) {};	
\node[small circle,densely dotted] (w2) at (8,-1) {};	

\draw[->] (v3) node[above] {$v_3$} -- node[pos=0.03,above right] {$u_3 = \text{������}$} (v23) node[above] {$v_{23}$};
\draw[->] (v23) -- (w23) node[above] {$w_{23}$};
\draw[->] (w23) -- (w1) node[right] {$v_1$};

\draw[->] (v2) node[below] {$v_2$} -- node[pos=0.075,above right] {$u_2 = \text{����}$} (v12) node[above] {$v_{12}$};
\draw[->] (v12) -- (w12) node[above] {$w_{12}$};
\draw[->] (w12) -- (v22) node[above left] {$v_{22}$};
\draw[->] (v22) -- (w22) node[above left] {$w_{22}$};
\draw[->] (w22) -- (w3) node[right] {$v_3$};

\draw[->] (v1) node[below left] {$v_1$} -- node[pos=0.075,above right] {$u_1 = \text{������}$} (v11) node[below] {$v_{11}$};
\draw[->] (v11) -- (w11) node[below] {$w_{11}$};
\draw[->] (v21) node[below] {$v_{21}$} -- (w21) node[below] {$w_{21}$};
\draw[->] (w21) -- (w2) node[right] {$v_2$};

\draw[->] (v3) to[bend right] node[below,sloped] {$u_3 = \text{����}$} (v1);
\draw[->] (v2) to[bend right] (v3);
\draw[->] (v1) to[bend right] node[pos=0.25,above,sloped] {$u_1 = \text{����}$} (v21);
\draw[->] (w11) to[bend right] (w2);

\draw[->] (v11) to[bend left] (v12);
\draw[->] (v12) to[bend left] (v11);
\draw[->] (w11) to[bend right] (w12);
\draw[->] (w12) to[bend right] (w11);

\draw[->] (v21) to (v22);
\draw[->] (v22) to (v23);
\draw[->] (v23) to[bend left] (v21);

\draw[->] (w21) to[bend right] (w23);
\draw[->] (w23) to (w22);
\draw[->] (w22) to (w21);

\draw[dotted] ($(v12)!0.5!(w11)$) node {$D_1$} circle (1.05); 
\draw[dotted] ($(v23)!0.5!(w21)!0.06!(v22)$) circle (1.65); 
\draw ($(v21)!0.5!(w22)$) node[right] {$D_2$}; 
\end{tikzpicture}
\caption{������ $D'$ ��� ������� $(u_1\vee\bar{u}_2)\wedge(\bar{u}_1\vee\bar{u}_2\vee u_3)$.}
\label{fig:SATTSP}
\end{figure}
	
��������� ���������� ���������� ���������� $D_j$.
��� ������� �� ������ $v_{jl}$ � $w_{jl}$, $1\le l \le p_j$, ��� $p_j$ "--- ����� ��������� � ��������� $C_j$.
����� �������, ����� �\'����� $v_{jl}$ � $w_{jl}$ ($1\le l \le p_j$) � ���������� ��������� $C_j$ ����� ���������� �������-����������� ������������.
�, �������������, ������ ����� ���� ����������� <<�����������>>	� ��������� ������� $v_i$, ������, � ����� �� ���� ��������� �� ���� ������� ���, ��������������� �������� <<������>> ��� ��������� $u_i$ � $\bar{u}_i$, ��������������. 
������� ���������� ��������� ����� ����� <<�����������>> ������	���� �����:
$(v_{j l}, w_{j l})$, $(v_{j l}, v_{j \, l+1})$ � $(w_{j \, l+1}, w_{j l})$, $1\le l \le p_j$, (�������� $l+1$ ����������� �� ������ $p_j$).
������ <<����������>> ��� � ���������� $D_j$ ���.
� ���, ��� <<�������>> ���� ��������� $D_j$ � ������� ������������ � ��������� $v_i$ ��������������� �������, � ����� ������ ���� ������� ����.
�������� ���� �������� ������.
��������, ��� ���� ������ $v_{j1}$, $w_{j1}$ ������������� �������� $\bar{u}_i$,
��������������� � ��������� $C_j$.
�����, ��� ���� �������, ���������� $D_j$ �������� ����� �� ������� ����, �����������	������� $v_i$ � ������ ���������� �� ��������� $\bar{H}_i$.
������ ������ ���� �������� �� ���� <<����>>.
�� ���� ���, ����������� ���������� $D_j$ � �������� �� ���� ����,
�������� ���� ����� ������������ � $v_{j1}$, � ��������� "--- ���������� � $w_{j1}$.

�������, ��� ����������� ������������ ������� �� ���������� $D_j$ ���������� ������������ ������� ��� ����� �������, �������� ����� � ��� ����������.
� ������, �������, ��� �������� (��� �����������) ����, �������� � ������� $v_{jl}$, ���������� ������������ ������� <<�������>> ����, ��������� �� $w_{jl}$, � ���� $(v_{j\;l-1}, w_{j\;l-1})$, � ����� ���������� ��� $(v_{j \, l-1}, v_{j l})$ � $(w_{j l}, w_{j \, l-1})$ (��������� $l-1$ ����������� �� ������ $p_j$).

�������������, ���� � ��������� ������������� ������� ������������ <<�������>> ��� $D_j$ ����, �������� � $v_{jl}$, �� ���� $(v_{j \, l-1}, v_{j l})$ � ��� ���� �� �����. �� ����� �������� ������� $v_{j \, l-1}$ ������ ����������� ������ ����� ������ �� ���� $(v_{j\;l-1}, w_{j\;l-1})$. ������, ���� $(w_{j l}, w_{j \, l-1})$ � ��� �����������. ��������� ��������, ��� �������� ������� $w_{j l}$ ���� ����������� ������ ����� ������ �� ��������� �� ��� <<�������>> ����.
� ������� ����������� ����������� ����� ����������� ������, ����� � ������������� ������� ����������� <<�������>> ��� $D_j$ ����, �������� � $v_{jl}$.
	
������� ������, ��� ������������ ������ ������������ ������� � ������� $D'$ ���������� ������������ ������� ���, ������� ������ � �������� $v_i$, ���, ��� �� �� �����, ������� �������� ������� ���������� $u_i$, $1\le i \le d$.
����� ����, ����������� ������ � ������� $D'$ ��� ������ ���������� $D_j$ ������ ��������� ���� �� ���� �������� � ��� ���� (��������������, ���������� $C_j$ ������ ��������� ���� �� ���� �������, ����������� �������� <<������>>).
������� �����, ��� ��� ����, ����������� ��������� ��������� $H_i$ (��� $\bar{H}_i$), ������������ (��� �����������) � ��������� ������������� ������� ������ � ����� <<������>>, ��������� �� $v_i$.
����� �������, ������� <<�������>> ��� � ������������� ������� ������� ������� �� �������� ���������� �� $U$.
� <<����������>> ���� ��������� $D_j$, ��� ���� �������� ����, ������� ������� �� <<�������>>.
����� �������, ������������ $\SAT(U,C)$ ����������� ������� ������������ $\ATSP(D')$.
\end{proof}

{\color{red}
\begin{remark}
�� ������ ��� ���������� ������� �������, ��� ��������� �������������� ������ � $k$"~������������ ��� ����� ������������� $k \in \N$ ������� �������� � ��������� �������������� (�)������������ ������ ������������, �� \emph{�� �������} �������� ���������� ����� ��������� $\SAT$ � $\ATSP$, ��� ��� ����������� ���������������� ������������� $\ATSP(n)$ ������� �� ������ �� ����� ���������� $d=|U|$, �� � �� ����� ������� $C$, ������� ����� ��������� ����������������, ���� ����� ��������� � ������ ���������� �� ���������� ������.
� �������� ���������� �������� ���� ���� �������, ��� ����������� �������� ���������� ����� ���� �� ��������������� ����� �������, ����� �������������� ������������ ���� ������� �� ����������� �����������, � ����� �������� ��������������.
(����� �� ���� ������������� �� ��� ��� �������� ����� �������� �������������� ���������� ������ ���������� �� �����������. ������������ �������� ������������� ������ � ������� (� ������, ��������� � ���� ���������) � ������������� ������ � �������� ���������.)
�� ��� ����� ������� �������� ���������� ���� �� ���������� � ����������� �� \emph{������� ��������} �������������. 
\end{remark}
}

� ������ �������, �� �����������~\ref{thm:kSAT} �������, ��� ��������� �������������� (�)������������ ������ ������������ �� ����� ���� ������� ������� � ��������� �������������� ������ � $k$"~������������ ��� ����� ������������� $k \in \N$~\cite{Fiorini:2003}.

�������~\ref{thm:SAT2TSP} ����� ��� ���������� ���������.
������ �� ��� "--- NP-������� �������� ����������� ������ ��� �������������� $\ATSP$ � $\TSP$, ����������� ��� �������� �� �������������� ������ � 3"~������������. �~\cite{Papadimitriou:1978} ���� ���� ��� ������� ����� ��������.

������� ������ �������� �� ��, ��� ��������� �������������� ������ � ������������ ��������� � ���������� 0/1"~��������������.
��������, ����� $U = \{u_1, \dots, u_d\}$ � ����� $C$ ������� �� ����� ����������
$\{u_1, u_2, \ldots, u_{d-1}, \bar{u}_d\}$.
����� ��������� $\SAT(U,C)$ �������� ��� $d$"~������ 0/1"~������� �� ����������� ������� $(0, 0, \ldots, 0, 1)$.
������, ����� ������ �� ����� ��������� ��� ���� 0/1"~������, ��������
$(0, 1, \ldots, 1, 1)$, ���������� �������� � $C$ ��� ���� ����������:
$$
C = \bigl\{\{u_1, u_2, \ldots, u_{d-1}, \bar{u}_d\},
\{u_1, \bar{u}_2, \bar{u}_3, \ldots, \bar{u}_n\}\bigr\}.
$$ 
�������� ����� � ��� �� ����, �������� � ���������� ������.

\begin{lemma} %[� ���������� $P(X)$ � $S(U,C)$]
\label{lem:SAT01}
��� ������� $X\subseteq \{0,1\}^d$ ����� ��������� ����� ���������� $C$ ��� $U=\{u_1,\dots,u_d\}$ ���, ��� $X = \SAT(U,C)$ � $\len(C) = k d$, ��� $k = 2^d - |X|$.
\end{lemma}

�������, ��� � ��������� ������� ������� ����� ������� $C$ ����� ����������� ���������.
� �������� ����������� �������� �������������� �������� ������������� $\BQP(m)$.

�������, ��� ����� ����������� (�� ����������� $\BQP(m)$)
\[
x_{ij} = x_{ii} x_{jj}, 
\]
��� $x_{ii}, x_{jj} \in \{0,1\}$, $1\le i < j \le m$,
������������ ���������� ����������
\[
x_{ii} \vee \bar{x}_{ij}, \quad
x_{jj} \vee \bar{x}_{ij}, \quad 
\bar{x}_{ii} \vee \bar{x}_{jj} \vee x_{ij}, \quad
1 \le i < j \le m.
\]
����� �������, ����� ������������ ������������ $\BQP(m)$ ��������
�������������� ������ ������������ $\SAT(U, C)$, ��� $|U|=m(m+1)/2$, $\len(C)=7m(m-1)/2$.

��������� �������~\ref{thm:SAT2TSP}, ��������

\begin{corollary}[\cite{Maksimenko:2011}]
$\BQP(m) \lea \ATSP(n)$, ��� $n = 7{,}5 m^2 - 6{,}5 m$.
\end{corollary}

����� �������, �~\cite{FioriniPokutta:2012} ���� �������� �����������, �� ������� ����� ������ ����������� � ���, ��� $\BQP(m)$ �������� \emph{���������} ��������� ����� ������������� $\ATSP(n)$ ��� $n = 63 m^2 - 57 m$.

������� ������ ����� ����� ����� �� �����~\ref{lem:SAT01} � �������~\ref{thm:SAT2TSP}.

\begin{corollary}
����� 0/1"~������������ � $\R^d$ �� $2^d - k$ �������� ($0 \le k \le 2^d - 1$) ������� ������������ ��������� ����� ������������� $\ATSP(n)$ ��� $n = (2k+1)d$.
\end{corollary}

����� ������� � �������������~\cite{Billera:1996} �������� ��� ����������� ��� $n = (4k+1)d$ ����� ����������.

� ���� ����� �������� ��������� ����.
����� ������������ ��������������� $d$"~������ 0/1-�������������� ���������� ����� ��������� $2^{2^{d-2}}$ ��� $d\ge 6$~\cite[Proposition 8]{Ziegler:2000} (��. �����~\cite[�.~102, ���.~2.6]{ZieglerBook}).
(�~\cite{Aichholzer:2000} ������������� ��� 0/1"~������������� ��� $d \le 5$.)
� ������ �������, ���� $f(n)$ "--- ����� ���� ������ 0/1"~������������� $P_{\textbf{0/1}}(n) \subset \R^n$, ��
\[
f(n) \le n \binom{2^n}{n} \le 2^{n^2}.
\]
����� �������, ���� ������ $d$"~������ 0/1"~������������ �������� ������ ���������� ������������� $P_{\textbf{0/1}}(n)$, �� $n$ ��������������� �� $d$.
� ���������, �� �� ����� ����� � � ��������� ������������� $\ATSP(n)$, �� � ��������� ��������������: $n$ ������������������ �� $d$.


\subsection{������������� ������� �����}
\label{sec:TSPvarious}

� ���� ������� ����� ����������� ��� ��������� ��������������, ����� ��������� � ��������������� ������ ������������.

\emph{�������������� ������������� �������} ���������� �������� �������� ��������� $\HDP(n)$ ���� ������������������ �������� ������������� ������� ������� ������� $D=([n],A)$ �� $n$ ��������. 
��� ������ �������� \emph{������������ ������������� $s$-$t$ �������} $\HDPst(n)$, ��� $s,t \in [n]$. 
��� ��� �������� ���������� �� ������� �� ������ ������ $s$ � $t$, ����� �������� $s=1$, $t=n$.

�������� ��������, ��� ������������ ������������� �������� $\ATSP(n-1)$ ������� ������������ ������������� ������������� $s$-$t$ ������� $\HDPst(n)$.
��� ����� ���������� �������� 
\[
x_{ij} = \begin{cases}
y_{ij}, & \text{��� $i,j \in [n-1]$, $j > 1$, $i\ne j$,}\\
y_{in}, & \text{��� $j = 1 < i$,}\\
\end{cases}
\]
��� $\bm{x}=(x_{ij}) \in \ATSP(n-1)$, $\bm{y}=(y_{ij}) \in \HDPst(n)$.
����� �������,
\[
\ATSP \propto_A \HDPst \propto_A \HDP.
\]
����� �������� ��������, ��� ��� ��� ��������� ������������ ������������ �������� ����������.
� ������, ������������ $\HDP(n)$ ������� ������������ $\ATSP(n+1)$.
����� �������� ���, �������, ��� � ����� ������������� ������� � ������ ������� $D = ([n+1], A)$ ����, ����������� $(n+1)$-� �������, ���������� ������������ ����� ���������� ������ ����� �������. 
����� �������, ������ ����, ����������� $(n+1)$-� �������, �� ������������� �������"=����������� ������������ ����� �������������� ��������� � ������� �� $n+1$ �������� � �������������� �������� � ������� �� $n$ ��������.
�������� ���� ��������, ��� ����������, ��������������� ��������� �����, ������� ������� �� ����������, ��������� � ������� ������:
\begin{align*}
x_{i,n+1} &= 1 - \sum_{j \in [n],\ j \neq i} x_{ij}, \quad i\in[n],\\
x_{n+1,i} &= 1 - \sum_{j \in [n],\ j \neq i} x_{ji}, \quad i\in[n].
\end{align*}
����������� ����������� ����� ��������������� ������������� ������ $\TSP(n+1)$ � \emph{��������������� ������������� �����} $\HP(n)$ � ����������������� ����� ���� ����������� �~\cite{Queyranne:1993}: $\TSP(n+1) =_A \HP(n)$. ��� �� ��� ����������� ���� ������������ ��� ��������� ����� �������� ����������, ������������ ���������� ������������� ������������ ������ ������������.

����� ����, ������������ ������������� $s$-$t$ ������� $\HDPst(n)$, ��������, �������� ������ ������������� $s$-$t$ ������� $\Dipath(n)$,
������������ ������� ��������������� $H(\bm{1}, n-1)$.
�����������, ��������� $\Dipath$ � $\HDPst$ ���� ������������ ������������ �������� ����������.

\begin{lemma}
$\Dipath(n) \lea \HDPst(k)$ ��� $k=2n-1$.
\end{lemma}
\begin{proof}
���������� ������ ������ $D = (V,A)$ �� $n$ �������� $V = \{s, v_1, v_2, \dots, v_{n-2}, t\}$.
����� $\Dipath(n)$ "--- ��������� ������������������ �������� $s$-$t$ ������� � $D$.

�������� ���� $D'$, ���������� $D$ � �������� ��������������� �������� �, ����� ����, ������� �������������� ������� $U = \{u_1, \dots, u_{n-1}\}$ � ����
\[
\{u_{i}, u_{i+1}\},\ \{u_{i}, v_{i}\},\ \{v_{i}, u_{i+1}\}, 
\quad i \in [n-2],
\]
� ����� ���� $\{u_{n-1}, s\}$ (��. ���.~\ref{fig:Dipath}).
\begin{figure}[tbh]
	\centering
	\tikzset{small circle/.style={inner sep = 1.5pt,draw,circle}}
	\begin{tikzpicture}[scale=1.0,
	>={Stealth[scale width=0.8]} % ���������� ��� �������
	]
	\node[small circle] (u1) at (0,2) {};	
	\node[small circle] (u2) at (3,2) {};	
	\node[small circle] (u3) at (6,2) {};
	\node[small circle] (u4) at (9,2) {};
	\node[small circle] (u5) at (12,2) {};
	\node[small circle] (s) at (14,2) {};
	\node[small circle] (t) at (14,0) {};
	\node[small circle] (v1) at (1.5,0) {};	
	\node[small circle] (v2) at (4.5,0) {};	
	\node[small circle] (v4) at (10.5,0) {};	
	\draw[->] (u1) node[above] {$u_{1}$} -- (u2) node[above] {$u_{2}$};
	\draw[->] (u2) -- (u3) node[above] {$u_{3}$};
	\path (u3) -- node {$\dotsc$} (u4);
	\draw[->] (u4) node[above] {$u_{n-2}$} -- (u5) node[above] {$u_{n-1}$};
	\draw[->] (u5) -- (s);
	\draw[->] (s) node[above] {$s\vphantom{u_{n-1}}$} -- (t) node[below] {$t\strut$};
	
	\draw[->] (u1) -- (v1);
	\draw[->] (v1) -- (u2);
	\draw[->] (u2) -- (v2);
	\draw[->] (v2) -- (u3);
	\draw[->] (u4) -- (v4);
	\draw[->] (v4) -- (u5);

	\draw (v1) node[below] {$v_{1}\strut$};
	\draw (v2) node[below] {$v_{2}\strut$};
	\draw (v4) node[below] {$v_{n-2}\strut$};

\newcommand{\inarc}[1]{	
	\draw[<-] (#1) -- +(-30:-0.8);
	\draw[<-] (#1) -- +(0:-0.8);
	\draw[<-] (#1) -- +(30:-0.8);
}
\newcommand{\outarc}[1]{	
	\draw[->] (#1) -- +(30:0.8);
	\draw[->] (#1) -- +(0:0.8);
	\draw[->] (#1) -- +(-30:0.8);
}
	\inarc{v1}
	\inarc{v2}
	\inarc{v4}
	\outarc{v1}
	\outarc{v2}
	\outarc{v4}
	\inarc{t}
%	\outarc{s}
	\draw[->] (s) -- (t);
	\draw[->] (s) -- +(-105:0.8);
	\draw[->] (s) -- +(-135:0.8);
	\draw[->] (s) -- +(-165:0.8);
\end{tikzpicture}
\caption{������ $D'$, ������������ ����� ������������� $\HDPst(2n-1)$, ������� ������������� ������������� $\Dipath(n)$.}
\label{fig:Dipath}
\end{figure}
��������, ��������� ������������������ �������� ������������� ������� �� $u_1$ � $t$ � ����� $D'$ �������� ����� ������������� $\HDPst(k)$ ��� $k=2n-1$.
�������, ��� ��� ����� ������� ������������ $\Dipath(n)$.

���������� ������������������� ������� ������������ ������ � $D'$, ��������������� ���� $(v, v')$, $v, v' \in V\cup U$, ���������� $x(v, v')$.
�� �������� ������� $D'$ �������, ��� ��� ������ ������������ ������ � $D'$ ������ ����������� ���������
\[
x(u_i, v_i) + x(u_i, u_{i+1}) = 1 = x(v_i, u_{i+1}) + x(u_i, u_{i+1}),\quad i \in [n-2],
\]
�, ����� ����, $x(u_{n-1}, s) = 1$.
�� ��������������� ������ ����� ������� �������������� �����������
\[
x(u_i, v_i) = 1 - \sum_{v \in V, \ v \ne v_i} x(v, v_i)  \quad \text{�} \quad
x(v_i, u_{i+1}) = 1 - \sum_{v \in V, \ v \ne v_i} x(v_i, v),
\]
��� $i \in [n-2]$.
����� �������, ��� ���������� ������������������� ������� ������������ ������ � $D'$ ������� ���������� ����� ���������� $x(v, v')$, $v,v' \in V$.
� ������ �������, ����� ��������� ��������� ����.
���� $H$ "--- ����������� ������ �� $u_1$ � $t$ � ������� $D'$, 
�� $\Set*{(v,v')\in H \given v,v'\in V}$ "--- $s$-$t$ ������ � $D$.
������ ����� $s$-$t$ ������ � $D$ ���������� ���������� ��������������� ����������� ������ � $D'$.
\end{proof}

����������� ����������� ����������� ��� �������������� $s$-$t$ ����� $\Path(n)$
� ������������� $s$-$t$ ����� $\HPst(n)$ � ����������������� �����.

\begin{lemma}
	$\HPst(n) \lea \Path(n) \lea \HPst(k)$ ��� $k=4n-6$.
\end{lemma}
\begin{proof}
�����, ��� � � ������ � ��������, $\HPst(n)$ �������� ������ ������������� $\Path(n)$, ������������ ������� ��������������� $H(\bm{1}, n-1)$.

���������� ������ ���� $G = (V,E)$ �� $n$ �������� $V = \{s, v_1, v_2, \dots, v_{n-2}, t\}$.
����� $\Path(n)$ "--- ��������� ������������������ �������� $s$-$t$ ����� � $G$.
�� �������� � ��������������� ��� �������, ���������� ������� ���� $G'$, ���������� $G$ � �������� ��������������� �������� � ������������ ��������������� ����� ������������� $\HPst(4n-6)$.
���� ���� ��������� �� ���.~\ref{fig:Path}.
\begin{figure}[tbh]
	\centering
	\tikzset{small circle/.style={inner sep = 1.5pt,draw,circle}}
	\begin{tikzpicture}[scale=1.3,
	>={Stealth[scale width=0.8]} % ���������� ��� �������
	]
	\node[small circle] (u1) at (0.8,2) {};	
	\node[small circle] (u2) at (2.2,2) {};	
	\node[small circle] (um2) at (3,2) {};	
	\node[small circle] (ue2) at (3.8,2) {};	
	\node[small circle] (u3) at (5.2,2) {};
	\node[small circle] (um3) at (6,2) {};
%	\node[small circle] (ue3) at (7,2) {};
%	\node[small circle] (u4) at (8,2) {};
	\node[small circle] (um4) at (8,2) {};
	\node[small circle] (ue4) at (8.8,2) {};
	\node[small circle] (u5) at (10.2,2) {};
	\node[small circle] (um5) at (11.3,2) {};
	\node[small circle] (s) at (12.4,2) {};
	\node[small circle] (t) at (12.4,0.5) {};
	\node[small circle] (v1) at (1.5,0.5) {};	
	\node[small circle] (v2) at (4.5,0.5) {};	
	\node[small circle] (v4) at (9.5,0.5) {};	
	\draw (u1) node[above] {$u_{1}\strut$} -- (u2) node[above] {$u'_{1}\strut$};
	\draw (u2) -- (um2) node[above] {$w_{1}\strut$};
	\draw (um2) -- (ue2) node[above] {$u_{2}\strut$};
	\draw (ue2) -- (u3) node[above] {$u'_{2}\strut$};
	\draw (u3) -- (um3) node[above] {$w_{2}\strut$};
%	\draw (um3) -- (ue3) node[above] {$u'_{3}\strut$};
	\path (um3) -- node {$\dotsc$} (um4);
%	\draw (u4) node[above] {$u_{n-2}\strut$} -- (um4) node[above] {$w_{n-2}\strut$};
	\draw (um4) node[above] {$w_{n-3}\strut$} -- (ue4) node[above] {$u_{n-2}\strut$};
	\draw (ue4) -- (u5) node[above] {$u'_{n-2}\strut$};
	\draw (u5) -- (um5) node[above] {$w_{n-2}\strut$};
	\draw (um5) -- (s);
	\draw (s) node[above] {$s\strut$} -- (t) node[below] {$t\strut$};
	
	\draw (u1) -- (v1);
	\draw (v1) -- (u2);
	\draw (ue2) -- (v2);
	\draw (v2) -- (u3);
	\draw (ue4) -- (v4);
	\draw (v4) -- (u5);
	
	\draw (v1) node[below] {$v_{1}\strut$};
	\draw (v2) node[below] {$v_{2}\strut$};
	\draw (v4) node[below] {$v_{n-2}\strut$};
	
	\newcommand{\inarc}[1]{	
		\draw (#1) -- +(-30:-0.8);
		\draw (#1) -- +(0:-0.8);
		\draw (#1) -- +(30:-0.8);
	}
	\newcommand{\outarc}[1]{	
		\draw (#1) -- +(30:0.8);
		\draw (#1) -- +(0:0.8);
		\draw (#1) -- +(-30:0.8);
	}
	\inarc{v1}
	\inarc{v2}
	\inarc{v4}
	\outarc{v1}
	\outarc{v2}
	\outarc{v4}
	\inarc{t}
	%	\outarc{s}
	\draw (s) -- (t);
	\draw (s) -- +(-105:0.8);
	\draw (s) -- +(-135:0.8);
	\draw (s) -- +(-165:0.8);
	\end{tikzpicture}
	\caption{���� $G'$, ������������ ����� ������������� $\HPst(4n-6)$, ������� ������������� ������������� $\Path(n)$.}
	\label{fig:Path}
\end{figure}
\end{proof}

� ���������� � ��� ������������� ����������� �������, ��� $\HPst(n)$ �������� ������ $\TSP(n)$, ������� � �������������� $x(s,t) = 1$,
� $\HDPst(n)$ �������� ������ $\HPst(2n-2)$ (�������������� ����� ��, ��� � � �����~\ref{lem:ATSP2TSP}).

% TSP  �� ������� ������ � HPst, ��� ��� ������� ����� ��� ���� ������������� ��� ����.
% HPst  �� ������� ������ � HDPst, ��� ��� ������� ����� ��� ���� ������������� ��� ����.

��� �������������� ��������������� (� ������ �������� ����������) ���� ������ �������� $\ATSP$, $\HDPst$, $\HDP$, $\Dipath$, $\TSP$, $\HPst$, $\HP$, $\Path$ �� ������� ���� ������ ����������� $\TSP \propto_A \ATSP$, ������� ���������� �� �������. 
�~������ �������, ��� �������������� $\TSP \npropto_A \ATSP$ ������������ ����� ������ �� ��������, ��� ��� ��� ��������� �������� � �������� ������ ��� 0/1-�������������.
� ����� �� ������������ �������, ������������ �� ���.~\ref{fig:TSPall}.

\begin{figure}[tbh]
	\centering
	\tikzset{small circle/.style={inner sep = 1.5pt,draw,circle}}
	\begin{tikzpicture}[scale=1.2,
	>={Stealth[scale width=0.8]} % ���������� ��� �������
	]
	\tikzstyle{every node}=[rounded corners,text centered,draw=black]
	\node (ATSP) at (0,3) {����������� ������};
	\node (HDP) at (0,2) {����������� ������};
	\node (HDPst) at (0,1) {����������� $s$-$t$ ������};
	\node (Dipath) at (0,0) {$s$-$t$ ������};
	\draw[<->] (ATSP) -- (HDP);
	\draw[<->] (HDPst) -- (HDP);
	\draw[<->] (HDPst) -- (Dipath);

	\node (SAT) at (5,3) {������������};
	\draw[->] (SAT) -- (ATSP);

	\node (HPst) at (5,1) {����������� $s$-$t$ ����};
	\node (Path) at (5,0) {$s$-$t$ ����};
	\draw[<->] (HPst) -- (Path);
	\draw[->] (HDPst) -- (HPst);
	
	\node (TSP) at (9.5,1) {����������� ����};
	\node (HP) at (9.5,0) {����������� ����};
	\draw[<->] (TSP) -- (HP);
	\draw[->] (HPst) -- (TSP);
	\end{tikzpicture}
\caption{�������� ���������� �������� ��������������, ��������� � ������� ������������.}
\label{fig:TSPall}
\end{figure}

� ���������� ������� ���������� ��� ���� ��������� ��������������.
\emph{�������������� ������������� ������\label{def:HamPolytope}} ����� �������� �������� �������� ��������� $\Ham(n)$ ������������������ �������� ���� ��������� ������� ����� �� $n$ ��������, ���������� ����������� ����.
��������, $\TSP(n)$ �������� ������ ������������� $\Ham(n)$, ������������ ������� ��������������� $H(\bm{1}, n)$.
� ������ �������, ����\'���� $\Ham \propto_A \TSP$ ��������, ������ ���� $P = \NP$, ��� ��� ������ ������������� ��������������� ����� (�, ��������������, ������� ������������� $\Ham(n)$) \NP-�����, ����� ��� ������ ������������� ������������ ����� �������� ������. 
��� �� �����, ��� ��������� �������� � �������� ������ ��� 0/1-�������������.
����� �������, ���� $P \neq \NP$, �� $\Ham(n)$ ������� ������������ ��������� ����� $\TSP(k)$, ������ ���� $k$ ������������������ ������������ $n$.


%%%%%%%%%%%%%%%%%%%%%%%%%%%%%%%%%%%%%%%%%%%%%%%%%%%%%%%
%
% ������ ������������� ������� $p$
%
%%%%%%%%%%%%%%%%%%%%%%%%%%%%%%%%%%%%%%%%%%%%%%%%%%%%%%%

\section[������ ������������� ������� p]{������ ������������� ������� $p$}
\label{sec:BQP-power}

��� ��������, ������ ������������ ������������� 3"~����������, �� �� 4"~����������~\cite{Deza:1992}.
����� ����, ������ 0/1-�������������, ��������� � ������������� �����������, �������� ��� ������� 2"~������������~\cite[p. 366]{Henk:2004}. %Page 366
��������, �. ��� ��������� ��������� ���� ��� �������������� ����.

\begin{theorem} [\cite{Onn:1993}]
	����� $n = \lambda_1 + \ldots + \lambda_k$, ��� $n, \lambda_1, \ldots, \lambda_k \in \N$, 
	$k \ge 2$, � $\lambda_1 \ge \ldots \ge \lambda_k \ge k^2$.
	����� ������������ ���� $P(\lambda_1, \ldots, \lambda_k)$ 
	�������� $\left\lfloor \frac{k^2}2 \right\rfloor$"~�����������.
\end{theorem}

��������~\cite{Barvinok:1988, Onn:1993}, ��� ������������� ��������� ����� ������������� ����������� 
(��������, �����������, ����������� ���������, ������������� ������������� ����, ���������� ������)
�������� �������� �������������� ���� ��� �������� �����������.

������, ��� ���������� ������� �������� �����������, 
$k$"~����������� ������������� ����������� ����, ��� ����� ���� ������.
��� �������������� ���������� �������.

����� ��������� ������� $x_1, x_2, \ldots, x_n \in \{0,1\}^d$ ������������ ���������� � ����������.
���� ���������� ���������, �� ������������ $P_{d,n}$ ������������ ���:
$P_{d,n} = \conv\{x_1, x_2, \ldots, x_n\}$ \cite{Gillmann:2006}.
��� ������, ����� ���������� ����������, ����� ������������ ������������
$Q_{d,n} = \conv\{x_1, x_2, \ldots, x_n\}$. 

\begin{theorem} [\cite{Bondarenko:2008}]
	���� $n = O(2^{d/6})$, �� ����������� $\Pr(Q_{d,n}\mbox{ 2"~����������})$ 
	��������� � 1 ��� $d \rightarrow \infty$.
\end{theorem}

\begin{theorem} [\cite{Gillmann:2006}]
	��� ������� $k \ge 2$ ���������� ��������� $c > 0$ � $\varepsilon > 0$ %, $\varepsilon < 0.001$,
	�����, ���
	\[
	\Pr(P_{d,n}\mbox{ $k$"~����������}) \ge 1 - 2^{-c d}, \quad \text{��� }n \le 2^{\varepsilon d}.
	\] 
\end{theorem}

���� ��� ������� $k\in\N$ ����� ������� ����������� ��������� $k$"~����������� 0/1-��������������.
����� $2^{\Theta\left(d^{(2\left\lceil k/3\right\rceil)^{-1}}\right)}$ �� ������
������������������ ������������ ����������� $d$ �������������. 
����� ����, �� �������, ��� ��� ��������� ������� �������� � ��������� ������� ������������ ��������������.


���������� ��������� 
\[
\Tensor(n) = \Set*{\bm{y}\otimes \bm{y} \given \bm{y} \in \{0,1\}^n},
\]
��� $\bm{y}\otimes \bm{y} \in \{0,1\}^{n\times n}$ "--- ��������� ������������.
��� ��������� ������� $\bm{x} \in \Tensor(n)$ ����� ������������ ����������� $x(i) = y_i y_i$, $i \in [n]$, � $x(j,k) = y_j y_k$, $j,k\in [n]$, $j \neq k$, �� ����
\begin{equation}
\label{eq:tensor1}
x(j,k) = x(k,j) = x(j) x(k).
\end{equation}
��������, ��������� $\Tensor(n)$ ������� ��������� ������ ������������� ������������� $\BQP(n)$, ������� �������� �� �������� ���������� $x(j,k)$ � ��������� $j > k$.
�� ���� ������� � ��������� ����������� ������� ������������ (��������������) �������������� ���������� �������� �������� $\conv(\Tensor(n))$ (��., ��������, \cite{FioriniPokutta:2015}).

���������� ����� ���������� � ��������� ������������, ������ � ������������ ���������
\[
\Tensor(n,p) = \Set*{\bm{y}\otimes \dots \otimes\bm{y} \in \{0,1\}^{n^p} \given \bm{y} \in \{0,1\}^n},
\qquad p \in \N, \quad n \ge p.
\]
��� ��������� ������� $\bm{x} \in \Tensor(n,p)$ ����� ������������ ����������� $x(i) = y_i \dots y_i$, $i \in [n]$, $x(j,k) = y_j \dots y_j y_k \dots y_k$, $j,k\in [n]$, $j \neq k$, \dots.
����, ��� ����� ��� ����� ����� ����������� �� �������� (��., ��������, ���������~\eqref{eq:tensor1}).
����������� �� ����������� ���������, ������ � ������������ ������� ���������� ������� \emph{����� ������������ ������� $p$} $\BPP(n,p)$.
���������� ������� $\bm{x} \in \BPP(n,p)$ ����� ������������� ��������� �������������� ��������� $[n]$, ���������� �� �� �����, ��� $p$ ���������.

���, ��������, $\BPP(n,3)$ ������� �� �������� $\bm{x} \in \{0,1\}^{d}$, $d = \binom{n}{1} + \binom{n}{2} + \binom{n}{3}$, ���������� ������� ������������� ����������
\begin{align*}
x(i,j) &= x(i) x(j), && 1\le i < j \le n,\\
x(i,j,k) &= x(i) x(j) x(k), && 1\le i < j < k \le n.
\end{align*}

���������� ��������� �������� ������ ������������� ������� $p$.
������ �����, $\BPP(n,1)$ �������� $n$"~������ 0/1-�����, $\BPP(n,2)$ "--- ����� ������������ ������������ $\BQP(n)$,
$\BPP(p,p)$ "--- �������� �� $2^p$ ��������.
����������� ������������� $\BPP(n,p)$, ��� $n \ge p$, ��������� � ������ ���������.
��� ������� �� ����, ��� ��� �������, ��������������� ������� $\sum_{i\in[n]} x(i) \le p$, ������� ����������.


\begin{lemma}
\label{MaksLemma1} 
������������ $\BPP(n,p)$ �������� $s$"~����������� ���
\begin{equation*}
s \le p + \left\lfloor p / 2 \right\rfloor.
%s \le 3 \left\lfloor p / 2 \right\rfloor + (p\mod 2).
%\label{MaksPower} 
\end{equation*}
\end{lemma}

\begin{proof}
�������, ��� $\BPP(n,1)$ 1"~����������, � $\BPP(n,2)$ 3"~���������� \cite{Deza:1992}.
�������������, ��� ������ ���� ��������� ������ 
$x^1, x^2, x^3 \in \BPP(n,2)$ ���������� ���������� ������� 
\begin{equation*}
f_{x^1 x^2 x^3}(x) = b + \sum_{1\le i \le j \le n} a_{ij} x(i) x(j), \quad b, a_{ij} \in \R,
%\label{MaksPower} 
\end{equation*}
�����, ��� $f_{x^1 x^2 x^3}(x^1) = f_{x^1 x^2 x^3}(x^2) = f_{x^1 x^2 x^3}(x^3) = 0$ 
� $f_{x^1 x^2 x^3}(x) > 0$ ��� ����� ������ ������� $x$ ����� �������������.
	
��������� �� �� ����� ��� $\BPP(n,4)$.
��� ����� �������� ��������� ������ $x^1$, $x^2$, $x^3$, $x^4$, $x^5$, $x^6$ ����� ��������� ��� �������
\begin{equation*}
\begin{aligned}
f_{x^1 x^2 x^3}(x) &= b + \sum_{1\le i \le j \le n} a_{ij} x(i)x(j),\\
f_{x^4 x^5 x^6}(x) &= d + \sum_{1\le i \le j \le n} c_{ij} x(i)x(j).
\end{aligned}
%\label{MaksPower} 
\end{equation*}
����, ��� �������
\[
F(x) = f_{x^1 x^2 x^3}(x) \cdot f_{x^4 x^5 x^6}(x)
\]
����� 0 ��� $x^m$, $m = 1, \ldots, 6$, � $F(x) > 0$ ��� ����� ������ ������� $x$ ������������� $\BPP(n,4)$.
� ������ �������, $F(x)$ "--- ������� 4-� ������� �� ���������� $x(i)$, $i \in [n]$,
%\[
%F(x) =  g + \sum_{1\le i \le j \le k \le l \le n} e_{ijkl} x(i)x(j)x(k)x(l),
%\]
�� ��� ���� ������ ������������ ��������� ������� $\bm{x} \in \BPP(n,4)$, ����������� �� ���������� ��������������� ������������ ���� $x(i)x(j)$ ��� $x(i)x(j)x(k)x(l)$.
����� �������, $F(x)$ ���������� ������� �������������� ��� $\BPP(n,4)$, � $\BPP(n,4)$ �������� 6"~�����������.
	
�������� ��� �� ��������, �������� ��������, ��� $\BPP(n,p)$ �������� $(3p/2)$"=����������� ��� ������ $p$.
	
��� �������� $p$ ���������� ��������, ��� ��� ������ ������� $x^0$ ���� $\BPP(n,1)$	�������� ������� �������� ������� 
\begin{equation*}
f_{x^0}(x) = b + \sum_{1\le i \le j \le n} a_i x(i), \quad b, a_i \in \R,
%\label{MaksPower} 
\end{equation*}
�����, ��� $f_{x^0}(x^0) = 0$ � $f_{x^0}(x) > 0$ ��� ����� ������ ������� $x$ ���� $\BPP(n,1)$.
\end{proof}


% ������ � $\BQP(n,p)$ ���� $2^p$ ������, �� ���������� ������.
% ��� ����� ���������� ����������� $\BQP(p+1,p)$ 
% � ��������� ��� ������ �������� �� ��� ����������� ������������: 
% � ������ � �������� ������ ������ �� ������� ���������.
% � ������ �� ��� ����� ����� �� $2^p$ ������ 
% � ����� ���� ���� ������ ����� ���������.

\begin{remark}
\label{MaksRemark}
��������, ��� $\BPP(n,p)$ �������� ������ ������������� $\BPP(n+1,p)$ � ������� ��������������� $x(n+1) = 0$. 
�������������, $\BPP(n,p)$ --- ����� ������������� $\BPP(k,p)$ ��� ���� $k > n$.
\end{remark}

\begin{lemma}
\label{MaksLemma2}
$\BPP(n,p)$ �� �������� $2^p$-����������� ��� $n \ge p+1$.
\end{lemma}
\begin{proof}
�� ��������� \ref{MaksRemark} �������, ��� ���������� �������� ����������� ����� ��� $n=p+1$.
	
%For $p=1,2$ the statement is true.
�������, ��� $\BPP(p+1,p)$ �� $2^p$-����������.
	
�����
$$
S(x) = \sum^{p+1}_{i = 1} x(i)
$$ 
���� ����� <<��������>> ��������� ������� $x\in\BPP(p+1,p)$.
��������� $X$ ���� ������ ������������� $\BPP(p+1,p)$ �������� �� $p+2$ �����������:
$$
X(k) = \Set*{x\in X \given S(x) = k}, \quad k = 0,1,\ldots, p+1.
$$
�����
$$
Y = \bigcup_{\text{$k$ �����}} X(k), \quad \text{�} \quad
Z = X \setminus Y.
$$
��������, $|X(k)| = \binom{p+1}{k}$, $|Y| = |Z| = |X|/2 = 2^p$.

�������, ���
\begin{equation}
\label{MaksParity}
\sum_{x\in Y} x = \sum_{x\in Z} x.
\end{equation}
���� ��� �����, �� $Y$ � $Z$ �� �������� ����� ������������� $\BPP(p+1,p)$.
	
�������� ��������� \eqref{MaksParity} �������������.
����� $\bar{r}$ "--- ���������� ������, �� ������������� $p+1$, $\bar{s} = 2p+1 - \bar{r}$ "--- ���������� ��������, �� ������������� $p+1$.
��� $i \in [p+1]$ �����
	\begin{equation}
	\label{MaksY1}
	\begin{aligned}
	\sum_{x\in Y} x(i) &= \sum_{x\in X(0)} x(i) + \sum_{x\in X(2)} x(i) + \sum_{x\in X(4)} x(i) + \ldots + \sum_{x\in X(\bar{r})} x(i) \\
	%                  &= 0 + \binom{p+1}{2} \frac{2}{p+1} + \binom{p+1}{4} \frac{4}{p+1} + \ldots + \binom{p+1}{\bar{r}} \frac{\bar{r}}{p+1}\\
	&= 0 + \binom{p}{1} + \binom{p}{3} + \ldots + \binom{p}{\bar{r}-1}
	\end{aligned}
	\end{equation}
	�
	\begin{equation}
	\label{MaksZ1}
	\begin{aligned}
	\sum_{x\in Z} x(i) &= \sum_{x\in X(1)} x(i) + \sum_{x\in X(3)} x(i) + \ldots + \sum_{x\in X(\bar{s})} x(i) \phantom{{} + \sum_{x\in X(0)} x(i)} \\
	%                  &= \binom{p+1}{1} \frac{1}{p+1} + \binom{p+1}{3} \frac{3}{p+1} + \ldots + \binom{p+1}{\bar{s}} \frac{\bar{s}}{p+1}\\
	&= \binom{p}{0} + \binom{p}{2} + \ldots + \binom{p}{\bar{s}-1}.
	\end{aligned}
	\end{equation}
	�� ������ �������
	$$
	\sum^{p}_{k=0} (-1)^k \binom{p}{k} = 0
	$$
	�������, ��� \eqref{MaksY1} � \eqref{MaksZ1} �����.
	
	��������� ��� �� ��������, ���������� ���������� $x(i_1, i_2, \ldots, i_m)$
	� $m$ ���������� ��������� $1\le i_1 < i_2 < \ldots < i_m \le p+1$, $1\le m \le p$.
	��� � ����,
	\begin{equation*}
	\sum_{x\in Y} x(i_1, i_2, \ldots, i_m) - \sum_{x\in Z} x(i_1, i_2, \ldots, i_m) = (-1)^m \sum^{p+1-m}_{k=0} (-1)^k \binom{p+1-m}{k} = 0 \enspace .
	\end{equation*}
	%\qed
\end{proof}


��������� ����� \ref{MaksLemma1} � \ref{MaksLemma2}, �������� ��������� �����������.


\begin{theorem}
\label{MaksT1} 
����� $s$ ���������� �����, ��� �������� ������������ $\BPP(n,p)$, $n > p$, �������� $s$-�����������.
�����
\[
%\label{MaksPower} 
p + \left\lfloor p / 2 \right\rfloor \le s < 2^p.
\]
\end{theorem}


���������� ��������� ��������� ���������:

\textbf{1.} ��� $\BPP(k,1)$, $k \ge 2$, �� �������� 2-�����������.
�������������, ��� $k \ge 2$, $\BPP(k,1)$ �� ����� ���� ������ 
������������� $\BPP(n,p)$ ��� $n \ge p \ge 2$.

\textbf{2.} $\BPP(k,1)$ �� ����� 2-����������� ������ �� ����������� 1-������ (�����).
�������������, ��� $n \ge p \ge 2$, $\BPP(n,p)$ �� ����� ���� ������ ���� $\BPP(k,1)$.

\textbf{3.} ����� ������������ ������������ $\BPP(k,2)$ �� �������� 4-�����������.
��� ��������, ��� $\BPP(k,2)$ �� ����� ���� ������ ������������� $\BPP(n,p)$ ��� $n \ge p \ge 3$.

������� ������, ��� ��� ������ $p \ge 4$, $\BPP(k,p)$ �������� ������ ������������� $\BQP(n)$ ��� ��������� $n = \Theta(k^{\lceil p/2\rceil})$.

��� ����� $k,m \in \N$, $k \ge 2m$, ��������� $H(k,m)$ ��������� �������:
\begin{equation}
H(k,m) = \binom{k}{m} + \binom{k}{\left\lceil m / 2 \right\rceil} 
+ \binom{k}{\left\lceil \left\lceil m / 2 \right\rceil / 2 \right\rceil} 
+ \ldots + \binom{k}{1}.
\notag%\label{} 
\end{equation}
�������, ��� $H(k,m) \ge \binom{k}{m}$, �, ��� ���������� $m$, 
$H(k,m) \sim \binom{k}{m}$ ��� $k \rightarrow \infty$.
������, ��� ����� �������� $k$, $H(k,m)$ ����� ����������� ���������� �� $\binom{k}{m}$. 
���������� ����������, ��� �������� ��� ����������� $H(k,m) / \binom{k}{m}$ ����� $41 / 20$ � ����������� ��� $k = 2m = 6$.


\begin{theorem}
\label{MaksT2} 
$\BPP(k, 2m) \lea \BQP(n)$ ��� ����� $m \in \N$, $k \ge 2m$ � $n \ge H(k,m)$.
\end{theorem}

\begin{proof}
�������, ��� $\BPP(k,4)$ ������� ��������� ����� ������ ������������� ������������� $\BQP(n)$	��� $n = \binom{k}{2} + \binom{k}{1}$.
������ ��������� $[n]$ ��� �������������� ��������� ������� $x \in \BQP(n)$ ����� ������������ ��������� 
\[
S = [k] \cup \Set{ij \given 1 \le i < j \le k},
\]
��� $ij$ ������� ������������ ��� ��������� $\{i,j\}$.
%�����������, ����� ����, ��� ��� �������� ����� ��������� ���-������ �����������, ����� ����� ���� ��������� ��������� ���� $s < t$ ��� $s,t\in S$.
������ �� �����, ��� ��������� ������� $x \in \BQP(n)$ ����� ������������ ����������� $x(s)$, $s\in S$, �
\[
x(s,t) \stackrel{\mathrm{def}}{\equiv} x(t,s), \qquad s,t\in S, \quad s \neq t.
\]
�������� ����������� ������ ������������� �������������,
\[
x(s,t) = x(s) x(t).
\]

���������� ��������� $F$ �������� $x \in \BQP(n)$, ��������������� ������������
\begin{equation}
\label{eq:MaksCond}
x(i,j) = x(ij) = x(i,ij) = x(j,ij), \qquad 1\le i < j \le k.
\end{equation}
� ���������, ��� $x\in F$ ��������� 
\[
x(ij,lm) = x(ij) x(lm) = x(i) x(j) x(l) x(m), \qquad  i,j,l,m \in [k], \quad i < j, \quad l < m.
\]
�������, ��� $F$ �������� ������ $\BQP(n)$.
	
������������� ���, ��� ��� ������� $\BQP(n)$ ������������� ������������
\[
x(s,t) \le x(s) \quad \text{�} \quad x(s,t) \le x(t), \qquad s,t\in S.
\]
������ ����� ����������� ���������� ����� ������������� $\BQP(n)$.
��������,
\[
F'_{ij} = \Set*{ x \in \BQP(n) \given x(i,ij) = x(ij), \ x(j,ij) = x(ij) }
\]
�������� ������ ������������� $\BQP(n)$, ��� $1\le i < j \le k$.
��� ��� $x(i,ij) \le x(i)$ � $x(j,ij) \le x(j)$, ��
\[
x(ij) \le x(i) \quad \text{�} \quad x(ij) \le x(j), \quad \forall x \in F'_{ij}.
\]
��������, ��� $x(i,j) = x(i) x(j)$, ��������
\[
x(ij) \le x(i,j), \quad \forall x \in F'_{ij}.
\]
�������������, 
\[
F_{ij} = \Set*{x \in F'_{ij} \given x(ij) = x(i,j)}
\]
�������� ������ $F'_{ij}$, � ����� ������ $\BQP(n)$.
����� ����, \eqref{eq:MaksCond} ��������� ��� ���� $x \in F_{ij}$.
	
����� �������,
\[
F = \bigcap_{1\le i < j \le k} F_{ij}
\]
�������� ������ $\BQP(n)$, � $F$ ������� ��������� ������������� $\BPP(k,4)$.
� ���������, ��� $y \in \BPP(k,4)$ ����� ��������
\[
\begin{aligned}
	y(i) &= x(i),    &\quad &1 \le i \le k,\\
	y(i,j) &= x(ij),   &\quad &1 \le i < j \le k,\\
	y(i,j,l) &= x(i,jl), &\quad &1 \le i < j < l \le k,\\
	y(i,j,l,m) &= x(ij,lm), &\quad &1 \le i < j < l < m \le k.
\end{aligned}
\]
	
�������� ��� �� �������� �� ������ ���������, ��� $\BPP(k, 2m)$ ������� ��������� ��������� ����� $\BQP(n)$ ��� $n = H(k,m)$, $k \ge 2m$, $m \in \N$.
\end{proof}

��������� ��� ����������� � �������� \ref{MaksT1} ��������

\begin{corollary} 
��� ������ $k \in \N$ � $n \ge 2^{2\cdot \lceil k/3\rceil}$ ����� ������������ ������������ $\BQP(n)$ ����� $k$-����������� ����� �� ������������������� ������ $2^{{\Theta}\left( n^{1 / {\left\lceil k/3\right\rceil}}\right)}$ ������.
\end{corollary} 

% The bound $k \le 3 \left\lfloor \frac{\log_2 (1.5n)}{2}\right\rfloor$ can be increased
% to $k \le 3 \left\lfloor \log_2 n\right\rfloor$


\section{\texorpdfstring{�������� ��������� ��� ������ � ���������� ������\\ � ������������ ������ � �����������}{�������� ��������� ��� ������ � ���������� ������ � ������������ ������ � �����������}}
\label{sec:ShortPath2Assignment}

� ���� ������� �� ���������� ������ ������ ����������� ������ �� ������� $s$ � ������� $t$ � ������ �������"=���������� ��������������� ����� $D = (V,A)$, $s,t \in V = [n]$.
� ����� ������ ��� ������ �������� NP-�������~\cite{Garey:1982}, 
� ������ � ������������� $\Dipath(n)$ ������� ������������ ������������� �������� $\ATSP(n-1)$ (��. ������~\ref{sec:TSPvarious}).
���� �� �~$D$ ����������� ������� ������������� �����, �� ������ ���������� ������������� ����������. � ��������� ������, ��� ������� \(\ShortP(n)\) ����� ���������� �������� (��. ������~\ref{subsec:polyhedra}).
����� �������� �������������� ������ �����������, ���������� ��������� ������.
�����������, ��� �� ��������� �������� ����� ������������ �������� �������� ��� ����. ����� ��� ����� �� ������ �� ������� ������� (�������), �� � ����������� ����������� ������� ������������ �������. �����, � ����� ������ �������������� ������, ��� ������� ����� ����� ������������� ���. �� ������� �������, ����������� ��� ����������� ���������� ���� (�������), ��� ������ ����� ���������� �������, ������ ����� ������������.

� ����� ���������� �����������, �������� � ������������ ������ �� ��������.
���� ����� ������� �������� ������ � ���������� ���� � �������"=���������� �������, ��� ������� �������� ����� ������������� ���.

\begin{theorem}
�������� ��������� ������ � ���������� ���� � ������� � ��������� �������������� ���� ������� �������� � ��������� ��������� ������ � �����������.
\end{theorem}
\begin{proof}
\newcommand{\us}{u\lefteqn{'}}
��� ������ ������ � ���������� ���� � ������� $D = (V, A)$, $V = \{v_1, \dots, v_{n+1}\}$, $n\in \N$, ���������� ������ � ����������� � ���������� ����� $G = (W,E)$ � ������ $U = \{u_1, \dots, u_n\}$ � $U' = \{\us_1, \dots, \us_n\}$.
����� ������� �� ��������� �������������� ���� ������ ������ � ��� ������ ������ (��. �����������~\ref{def:AffReductionRestriction}).

�� �������� ��������, ����� ������������, ��� ������� ������ ������ ���������� � ������� $s = v_1$ � ������������� � $t = v_{n+1}$. 
��������������, ���, �������� � $s$ ��� ��������� �� $t$, � ������� $D$ ���.
����� ����, ������ ��������� �����������. ���������� $c_{i,j}$ �������� ������� $\bm{c} \in \Q^{A}$ ��� ������ � ���������� ���� ����� ����� ���� ����, ������� �� $v_i$ � $v_j$, $i \ne n+1$, $j \ne 1$.
���������� $b_{i,j}$ �������� ������� $\bm{b} \in \Q^{E}$ ��� ������ � ����������� ����� ����� ���� ����� $\{u_i, \us_j\}$.

�������� ����������� $\alpha\from \Q^{A} \to \Q^{E}$ ��������� ��������� �������:
\[
b_{i,j} = \begin{cases}
0, & \text{���� } i = j \ne 1,\\
c_{i,n+1}, & \text{���� } j = 1,\\
c_{i,j}, & \text{� ��������� �������}.
\end{cases}
\]
����� ����� �� ������� ��������� �������"=����������� ������������ ����� ���������� ��� $A$ � ���������� ����� $E \setminus \{\{u_2, \us_2\},\dots,\{u_n, \us_n\}\}$.
����� ������� �������� ������ � ������� $D$ ����� ���������� � ������������ ��������� ����������� ������������� � ����� $G$, ������� ����� ����� �� ���.
(���� ������� $v_i$ �� ����������� ������, �� ��������������� ������������� �������� ����� $\{u_i, \us_i\}$ �������� ����.)

�������� ������� ������������ �����������, \emph{��������} � ���������� ����� $G$ ������� ��������� ����� ����
\[
\bigl\{\{u_{i_1}, \us_{i_2}\},\{u_{i_2}, \us_{i_3}\},\dots,\{u_{i_{n-1}}, \us_{i_n}\},\{u_{i_n}, \us_{i_1}\}\bigr\},
\]
��� $k \ge 2$, � $\{i_1, \dots, i_k\}$ "--- ��������� ������������� ������������ ��������� $[n]$.
��������, ����� ������������� � $G$ ������������ ����� ����� ����� �������� �, ��������, ����� ���� $\{u_i, \us_i\}$.
������ ������ ������, ���������� ����� ������� $u_1$ � $\us_1$, ������������� ���������� ������ � $D$.
������� � $G$, �� ���������� ����� $u_1$ � $\us_1$, ������������� �������� � $D$.

��� ��� ���� $D$ �������� ������ ������������� �������, �� ����������� ������������� � ����� $G$ � ������ $\bm{b}$ ����� ��������� ������ ���� ������, ���������� ����� $u_1$ � $\us_1$, � ��������������� ������ � $D$ � ��� �� ��������� ����� ���.
\end{proof}

\begin{corollary}
\label{cor:Short2Assign}
���� �������� $\ShortP(n+1)$ �������� ��������� ����� ������������� $\Birk(n)$, $n \in \N$.
\end{corollary}

%%%%%%%%%%%%%%%%%%%%%%%%%%%%%%%%%%%%%%%%%%%%%%%%%%%%%%%
%
% End of section
%
%%%%%%%%%%%%%%%%%%%%%%%%%%%%%%%%%%%%%%%%%%%%%%%%%%%%%%%

%% Глава 5
%%%%%%%%%%%%%%%%%%%%%%%%%%%%%%%%%%%%%%%%%%%%%%%%%%%%%%%%%%
%
%     ����������� �������� ����������
%
%%%%%%%%%%%%%%%%%%%%%%%%%%%%%%%%%%%%%%%%%%%%%%%%%%%%%%%%%%

\chapter{����������� �������� ����������}
\label{chap:ExtAff}
%\begin{flushright}
%	� ���� ������� �������� �������?
%\end{flushright}
%\medskip

� ���� ����� ��������������� ��������� ��������� ������� ��������� ��������������, ������������� � ���������� �����.
� ������, ����� <<������� ������������>> � �����������~\ref{def:ineA} ����� �������� �� <<�������� �������� �������>>. 
�� ������ ������ ������� ��������� �������������� � ������ ������� ����� �������� ������� ����������� �������� ���������� �������� ��������������, ��������� ��������� ��� �������� � ��������������� ����� ����� ���� �������� � ������� ������������� �������� ���������� �� �����~\ref{chap:AffTheory}.
�����, � �������~\ref{sec:ExtAffExamples} ���������� ������� ����������� �������� ����������. �~�����, ��������� ���������� �������, �������������� ��������������� ����������� ����������� ����������� ����� ��������, ��� ��� (�������) �������� ����������.
� ��������� ������� ����� ��������, ��� ����� ��������� ��������������, �������� ������������ �������� ����������� ������ NP, ���������� ������� �������� � ������� ������������ ��������������.
��� �����, ��� ����������� ���� � ��������� ������ ��������� �������������� ����������� ������������ ���� ����� � ����� ������ ����������� �������� ����������.

\section{������}

\begin{definition}\label{def:ineE}
	���� ��������� ����� ������������� $Q$ ��� �� ���� ���� ������������ �������� ����������� ������������� $P$, 
	%� ������, ����� ������������ $P$ �������� �������� ������� �������������~$Q$ ��� �� ��� �����, 
	����� ������������ ����������� $P \lee Q$.
\end{definition}

%�������, ������ �����, ��� �� $P \lea Q$ ������� $P \lee Q$.
��������� ���������, ���������� �� ����������� $P \lee Q$, ���� ������� ����, � �������� \ref{sec:Extension} � \ref{sec:RectCover} (��������, ��������~\ref{prop:xc-compare} � ��� ������ ��� ����� �������������� �������� ������� ���������� ������"=�����������). 

������ $\lea$ �� $\lee$ ����������� ��������� ����������� ��������� ��������������. ����� ����, �� ������ ������� �������������� ����������� ���� $P \lee Q$ ������������� �����, ��� ����������� $P \lea Q$.
������� ������ ���������� ����������� �������� ������ ��������� �������� �������.
� ���������, ��������� ������������ $P \lee Q$, ������ ������, ������ ���������� ����� ����� ��������������� �������������� � �������� ����� �� ������.

���������� ��������� ������� ��������.

����� (��. �.~\pageref{ProjOfSimplex}) ��� ���������� ��� ����, ��� 
����� ������������ $P$, ������� $n$ ������, �������� �������� ������� ��������� $\Delta_{n-1}$. � ���������, ���� $P$~--- $d$"~������ 0/1"~������������, ��
\[
P \lee \Delta_{n}, \quad \text{��� }  n \ge 2^d-1.
\]

%�� �������� � $d$-������ 0/1-�������������� ������ ������� $d$-�������
%\emph{$[k]$-�������������}, ��������� ������ �������� ����������� $[k]^d$.

����� $a,b \in Z$, $a < b$, � $[a,b]$~--- ��������� ����� ����� ���������������� �������. ����� $k = \lceil\log_2 (b-a+1)\rceil$, � ������� $\zeta \from [a,b] \to \{0,1\}^k$ ������� ������ $x \in [a,b]$ ������ � ������������ �������� ������ ����� $x-a$. ��������, ��� ������� ������������� ���������. � ������ �������, �������� �������������� 
\begin{equation}
\label{eq:01toZ}
\zeta^{-1} \from \bm{y} \mapsto a + y_1 + 2 y_2 + \dots + 2^{k-1} y_k
\end{equation}
������� � ���� ������������� ���������.
��� ���������, �������� ���������

\begin{prop}\label{prop:01}
����� $X \subseteq S^n$, ��� $S$~--- ��������� ����� ����� ���������� ������� $[a,b]$.
����� ���������� $Y \subseteq \{0,1\}^d$, $d = n \cdot \lceil\log_2 (b-a+1)\rceil$, ���
\[
X \lee Y.
\]
\end{prop}

����� �������, ��� ���������, ���������� ������ ������������, ������������ � �������������� ��������� �������� ��������� 0/1-�������������.

�� �������� � �������� ����������� (����������� \ref{def:Aff}) ������ ������� ����������� �������� ���������� �� ������ ��������� $\lee$.
�������������� ��������, ��� ��� �������� ������������� ��������������� ����� ����� ��� ���� � ����������� ������������.


\begin{definition}
	\label{def:Aff2}
	����� ��������, ��� ��������� �������������� $P$ \emph{���������� ������� ��������} � ��������� �������������� $Q$, ���� �������� ������������� ���������� (������������ ������� ������������� $p\in P$):
	\begin{enumerate}
		\item 
		�������������� $\tau$ ���� $I$ ������� ������������� $p = p(I)\in P$ � ��� $I'$ ������������� $q = q(I') \in Q$.
		\item 
		������� �������� ��������� $D\bm{y}=\bm{c}$, �������� �����
		\[
		F = \Set{\bm{y}\in q \given D\bm{y}=\bm{c}}
		\]
		������������� $q$.
		\item 
		�������� ���������� ��� ������� ���� $I$ ��������� ����������� 
		\[
		\beta\from \R^{d'} \to \R^{d}, \qquad d = d(I), \quad d' = d'(\tau(I)),
		\]
		������, ��� $p = \beta(F)$.
	\end{enumerate}
	�����������: $P \propto_E Q$.  
\end{definition}

\begin{lemma}
	��� ����� ���� �������� �������������� $P$ � $Q$ �� $P \propto_A Q$ ������� $P \propto_E Q$.
\end{lemma}
	
\begin{proof}
	� �����������~\ref{def:Aff} ��������� ������� ����������� ��������� ����������� $\alpha \from p \to F$,
	� � �����������~\ref{def:Aff2}~--- ������� ��������� ����������� $\beta \from F \to p$ � ������� �������� ���������, ������������ ����� $F$.
	���� ����������� ���������, �� ������� �� ������ ������� �������� � ������� ����� ����������� �� �������������� ����� (��., ��������, \cite{Winkler:1996}).
\end{proof}

�� �������� � ��������~\ref{thm:Aff2Cones} ����� ���� �������� ��������� ����� ����� ���� ���������� � �������� ����������� �������� ����� ������������� �����������.

\begin{theorem}
\label{thm:ExtAff2AffProblems}
����� �������� ������ ������������� ����������� $(d,S,g)$ ���������� ��������� �������������� $P = \{p(I)\}$, � ������ $(d',S',g')$ "--- ��������� �������������� $Q = \{q(I')\}$.
�����������, ��� ��������� $P$ ���������� ������� �������� � $Q$ �, ����� ����,
������� ��������� $D\bm{y}=\bm{c}$, �������� ����� ���������������� ������������� $q \in Q$, ������� �� ��������� ���������������, ������� � $q$.
����� ������ $(d,S,g)$ � ������������ $\bm{-1} \le \bm{c} \le \bm{1}$ ������� �������� � ������ $(d',S',g')$.
\end{theorem}

���������� ��������� ��������� �������� ����� ���� ����������,
���������� �� ������� ����������, ��������� � �������� \ref{sec:Extension} � \ref{sec:RectCover}. 

%\begin{prop}
%	����� $P \propto_E Q$. ����� ������ �������� ����������� �� �������������� ��������� $P$ ������������� �������� � ������ �������� ����������� �� �������������� ��������� $Q$.
%\end{prop}
	
\begin{prop}
	\label{thm:PropE}
	����� $P \propto_E Q$.
	�����������, ��� � ��������� $P$ ���� �������������, ������� ���� ��� ��������� �� ��������� �������:
	\begin{enumerate}
		\item C�������������������� ����� ������ (������������ ������� �������������).
		\item C������������������ ����� �������������� ��������.
		\item C������������������ ��������� ����������.
	\end{enumerate}
	\noindent
	����� � $Q$ ������� ������������� � ���� �� ����������.
\end{prop}

�� �������� � ������������ \ref{prop:01} ����� �������� ��������� �������.

\begin{theorem}[�� ��������������� 0/1-��������������]
	����� ��������� ������������� �������������� $P$ ������������ ��������� $(L,d,k,g)$ (��. �����������~\ref{def:family}).
	����� $P$ ���������� ������� �������� � ��������� ������������� 0/1"=��������������,
	������������ ��������� $(L,d',1,f)$, ��� $d' = d\cdot\lceil\log_2(k+1)\rceil$,
	$f = f(\bm{y}, I) = g(\gamma(\bm{y}), I)$, $\bm{y} \in \{0,1\}^{d'}$,
	� $\gamma$~--- �������� ��������, ������������ �� �������� � ��������~\eqref{eq:01toZ}.
\end{theorem}

���� ���� ������ ��������, � ������ ������������� ��������� �������������� ��������� �������� 0/1-��������������~\cite{Junger:1995,Papadimitriou:1984}.

%�����������. ����� ������� $d = d(n)$ �������������, � ������������������ �������� �������� $X_n \in \Z^d$ ������, ��� ��� ����� $n \in \N$ � $x \in \Z^d$ ������ �������� �������������� $x \in X_n$ ����������� ������ NP. ����� ��������� (������������������) �������������� $P(n) = \conv(X_n)$ ���������� \emph{�������������}.

%����������. �~\cite{Naddef:1981, MatsuiTamura:1995} �������������� ���������� �������������, � ������� ��� ������ ���� ��������� ������ �������� ������������ �� ������� �������� ����� ��������� �������, ������������ (���������) ������ ���� ������ ����� �������������.

�����������, ��������������� ��������� ������������� 0/1"=�������������� ��������� ����� ������� ���������.

\begin{theorem}
	\label{thm:NP2comb}
	����� �������� $(L,d,k,g)$ ���������� ��������� ��������� �������������� $P$,
	� �������� ������������ $g$ ����������� ������ NP.
	����� ��� ��������� ���������� ������� �������� � ���������� ��������� ������������� �������������� $Q$.
\end{theorem}
\begin{proof}
����� $g = g(\bm{x},I)$ ����������� ������ NP.
�����, �������� �����������~\ref{def:NP}, ���������� ������� $p\from\N\to\N$ � ������������� ���������� �������� $f\from \{0,1\}^* \to \{\text{����},\text{������}\}$ �����, ��� 
\[
g(\bm{x},I) \iff \text{�������� } \bm{u}\in \{0,1\}^{p(d+\size(I))} 
\text{ �����, ��� } f(\bm{x}, \bm{u}, I).
\]

� ������� ��������� $f$ ��������� �������������
\[
q(I) = \Set*{(\bm{x}, \bm{u}) \in \{0,1,\dots,k\}^d \times \{0,1\}^{p(d+\size(I))} \given f(\bm{x}, \bm{u}, I)}
\]
���������� �������������� ��������� �������������� $Q = \{q(I)\}$. 
�������� ��������, ��� ������������ 
\[
p(I) = \Set*{\bm{x}\in\{0,1,\dots,k\}^d \given g(\bm{x},I)}
\]
�������� ������������� ��������� ������������� $q(I)$.
\end{proof}

�������� ��������� �������������� � NP-������ ���������� ������������ ����� ������� ������������� ������������� ������ $\Ham(n)$ (��. ����������� �� �.~\pageref{def:HamPolytope}).

� ���������� ����� ������� �������, ��� ����������� $P \lee Q$ � ��������� ������� ����� ������������ ��� ��������� �������� ����� ������ �������������� $P$ � $Q$.

\begin{theorem}
����� ������������ $P \subseteq \R^d$ �������� ������� ������������� $Q \subseteq \R^n$ ��� ������������� ������������� $\pi \from \R^n \to \R^d$, $n > d$.
�����, ����� ����, $\pi(\ext Q) = \ext P$.
����� ���� ������������� $P$ �������� ��������� ����� ������������� $Q$.
\end{theorem}

\begin{proof}
��� ������ ������� $\bm{v} \in \ext P$ ��������� ���������
\[
W(\bm{v}) = \Set*{\bm{x} \in \ext Q \given \pi(\bm{x}) = \bm{v}}.
\]
��������, $\conv (W(\bm{v}))$ �������� ������ ������������� $Q$.
����� ����, ���� ��������� $V \subseteq \ext P$ �������� ���������� ������ ��������� ����� ������������� $P$, �� $\conv \Set{W(\bm{v})\given v\in V}$~--- ����� ������������� $Q$.

��� �������� ��������, ��� $\R^d$ ������� � $\R^n$,
� ������������� $\pi$ ����������� $(x_1, \dots, x_d, \dots, x_n)$ � $(x_1, \dots, x_d, 0, \dots, 0)$.
��������������� ������ $\bm{c} \in \R^n$ ��������� ��������� �������.
���� $n = d+1$, �� $\bm{c} = \bm{e_n}$,
����� $\bm{c} = \lambda_{d+1} \bm{e_{d+1}} + \dots + \lambda_{n} \bm{e_n}$
� ������������ $\lambda_{d+1}$, \dots, $\lambda_{n}$ ��������� ���, 
��� ��� ������ $\bm{v} \in \ext P$ � ����� ���� ������ $\bm{w}, \bm{w'} \in W(\bm{v})$ �� ����������� $\bm{w} \ne \bm{w'}$ ������� $\bm{c}^T \bm{w} \ne \bm{c}^T \bm{w'}$.
����, ��� � ���� ���������� ��������� ������ ������������� $Q$ ����� ������ $\bm{c}$ ����������. 
� ������� ����� ������� � ������ ��������� $W(\bm{v})$ ������� ���������� �������
\[
\bm{w_v} = \argmax_{\bm{w} \in W(\bm{v})} \Set{\bm{c}^T \bm{w}}.
\]

�������� ��������, ��� ���� ������� $\bm{v_1}$ � $\bm{v_2}$ ������������� $P$ ������, �� ��������������� ������� 
\[
\bm{w_1} = \bm{w_{v_1}} \quad \text{�} \quad \bm{w_2} = \bm{w_{v_2}}
\] ���� ������.

�����������, ��� ������� $\bm{v_1}$ � $\bm{v_2}$ ������������� $P$ ������.
����� ������������� $F_1 = \conv (W(\bm{v_1}))$ � $F_2 = \conv (W(\bm{v_2}))$, � ����� �� �������� �������� $F = \conv(F_1 \cup F_2)$ �������� ������� ������������� $Q$.

������� $\bm{b} = \bm{v_2} - \bm{v_1}$.
����� $\bm{b}^T \bm{w} = \bm{b}^T \bm{w'}$ ��� ����� $\bm{w}, \bm{w'} \in F_1$.
����������, $\bm{b}^T \bm{w} = \bm{b}^T \bm{w'}$ ��� ����� $\bm{w}, \bm{w'} \in F_2$.
����� ����, $\bm{b}^T \bm{w_1} \neq \bm{b}^T \bm{w_2}$.
%������ $\aff(F_1)$ � $\aff(F_2)$ ������������ ������� 
%(��������������, �������� ������� $\bm{b}^T \bm{x}$ ��������� ������ �������� ��� $\bm{w_1}$ � $\bm{w_2}$.)

�������� ����� $\beta, \gamma \in \R$, $\gamma > 0$, ���,
����� �������� ������� $f(\bm{x}) = \beta \bm{b}^T \bm{x} + \gamma \bm{c}^T \bm{x}$ ��������� ���������� �������� ��� $\bm{w_1}$ � $\bm{w_2}$.
����� �� ����������� ������� $\bm{c}$ � ������ $\bm{w_1}$ � $\bm{w_2}$ �������
\[
f(\bm{w_1}) = f(\bm{w_2}) > f(\bm{w}) \quad
\text{��� ���� }
\bm{w} \in W(\bm{v_1}) \cup W(\bm{v_2}) \setminus \{\bm{w_1}, \bm{w_2}\}.
\]
�������������, $\bm{w_1}$ � $\bm{w_2}$~--- ������� ������� ������������� $F = \conv(W(\bm{v_1}) \cup W(\bm{v_2}))$, ����������� ������ ������������� $Q$.
\end{proof}

\begin{corollary}
\label{cor:CliqueOfExtension}
����� ������������ $P \subseteq \R^d$ �������� ������� ������������� $Q \subseteq \R^n$ ��� �������� ����������� $\pi \from \R^n \to \R^d$.
�����, ����� ����, $\pi(\ext Q) = \ext P$.
����� �������� ����� ������ ���� �������������� ������� ������������ $\omega(P) \le \omega(Q)$.
\end{corollary}

����� �������, ��� ��� ��������� ������ ��������� ������ ������������ ������� ����������� �������� ���������� (� ��� ����� � ������������� � ���� �����) ������������� �������� ����� ���������.
%� ������ ������� ���� ������� ����������� ���, ��� � �������� ���� ���� � ���������� 0/1-��������������.

%%%%%%%%%%%%%%%%%%%%%%%%%%%%%%%%%%%%%%%%%%%%%%%%%%%%%%%%%%
%
%     �������
%
%%%%%%%%%%%%%%%%%%%%%%%%%%%%%%%%%%%%%%%%%%%%%%%%%%%%%%%%%%

\section{�������}
\label{sec:ExtAffExamples}

�� ������� ������������� ���������� �������, ����� ����������� �������� ���������� ����������� � ���������� ������� ����, ��� ����� �������� ����������.
���������� ��������� ��������� ��������:
\begin{enumerate}
	\item ��������������� ������������ $\Perm(n)$ �������� �������� ��������� ������������� ��������: $\Perm(n) \lee \Birk(n)$ (��. �.~\pageref{Perm2Birk}).
	%	\item ������������� ����������� ������������� �������� ��������������� �������� ���������: $\Match \propto_A \Pack$.
	\item ����� ����� ��������������� ������������� ������ � ��������������� ������������� �������� ����������� � �������~\ref{sec:Travelling}:
	$\TSP(n) \lee \ATSP(n)$.
	\item ������������ ����������� ������������� $\Match(2n)$ �������� ��������� ������������� ������������� ������~\cite{Yannakakis:1991}: $\Match(2n) \lee \TSP(6n)$.
	\item ��� ������ ����� $G=(V,E)$, $\Stable(G) \lee \ATSP(k)$, ��� $k = 4 |E| + |V|$~\cite{Yannakakis:1991}.
	\item ��� ������� $n\in \N$ ���������� ���� $G = (V,E)$, $|V| = 2n^2$, �����, ��� $\BQP(n) \lee \Stable(G)$~\cite{FioriniPokutta:2015} (��. ����� ������� \ref{thm:BQPStable}).
	\item ������������ ������ � 3-������������ $\SAT(U,C)$ �������� ��������� ����� ������������� ������������� ���������� ������ � �����������~\cite{AvisTiwary:2015}: $\SAT(U,C) \lee \TAP(m)$, ��� $m = O(kn)$, $k=|U|$, $n=|C|$. ������ �������� ���������� $\SAT \propto_A \TAP$ ���������� (��. ������~\ref{sec:3Ass}).
	\item ��� ������ ����� $G=(V,E)$ ���������� ��������� ���������� ���� $G'=(V',E')$, $|V'| = O(|E|^2)$, $|E'| = O(|E|^2)$, ��� $\Stable(G) \lee \Stable(G')$~\cite{AvisTiwary:2015}.
	\item �������� ��������, ��� ��� ������ ����� $G = ([n],E)$ ��������� $\Stable(G) \lee \BQP(n)$. (��������������� ����� ������������� $\BQP(n)$ ����� � ����������� ��������������� $x_{ij} = 0$, $\{i,j\} \in E$.) � �� �� �����,	���� ���� $G$ ��������, �� ����������� $\Stable(G) \lea \BQP(n)$ ���������� �� ��� ����� $n$ (��. �����������~\ref{prop:StableBQP}).
\end{enumerate}

��� ��������� ����������� ������� ������������������ �������� ������� �������������� ������ ����������� �������� ����������. 

%��� ����, ��� ��������� �������������� ������ � ������� � ���������� ������� �������� � ��������� �������������� ������ � �������, �������� (��. ����������� � ����������� �� �.~\pageref{eq:KnapEq})

������ � ����, ��� ��������� �������������� ������ � ������� ���������� ������� �������� � ��������� �������������� ������ � ������� � ����������
(��. ����������� �� �.~\pageref{eq:KnapEq}).

\begin{lemma}
	����� $n \in \N$, $\bm{a}=(a_1,\dots,a_n) \in \Z^n$ � $b \in \Z$.
	������� 
	\[
	S_{\min} = \begin{cases}
	0,& \text{���� }  \bm{a} \ge \bm{0},\\
	\sum\limits_{i:\ a_i < 0} a_i,& \text{�����,} 
	\end{cases}
	\]
	$M = \max\{1, b - S_{\min}\}$ � 
	\[
	\bm{c} = (a_1, \dots, a_n, 1, 2, \dots, 2^k), \quad \text{��� }
	k = \lceil\log_2 (M)\rceil.
	\]
	�����
	$\Knap(\bm{a},b) \lee \KnapEq(\bm{c},b)$.
\end{lemma}

\begin{lemma}
	��� ����� $n \in \N$, $\bm{a} \in \Z^n$ � $b \in \Z$ ���������
	$\KnapEq(\bm{a},b) \lee \BQP(n)$.
\end{lemma}
\begin{proof}
��������� $\bm{a}^T \bm{x} = b$ �� ����������� ������������� ������ � ������� � ���������� (��. �������~\eqref{eq:KnapEq}) ������� ��
\[
(\bm{a}^T \bm{x} - b)^2 = 0,
\]
���, ��� �� �� �����,
\[
\sum_i 2 b a_i x_i - \sum_{i,j} a_i a_j x_i x_j = b^2.
\]
��������, ��������� 
\[
\sum_i 2 b a_i y_{ii} - \sum_{i,j} a_i a_j y_{ij} = b^2, \quad \text{��� }
\bm{y} =(y_{ij}) \in \R^{n(n+1)/2},
\]
���������� ������� �������������� � ������������� $\BQP(n)$.
� ������ �������, ����������� $y_{ii} \mapsto x_i$, $i\in[n]$, ���������� ��������������� ����� ������������� $\BQP(n)$ �� ������������ $\KnapEq(\bm{a},b)$.
\end{proof}

\begin{lemma}[\cite{Maksimenko:2013TSP}]
$\ATSP(n) \lee \BQP\left((n-1)^2\right)$. 
\end{lemma}
\begin{proof}
���������� ������� $\bm{y} \in \BQP\left((n-1)^2\right)$ ����� ���������� $y(ij,kl)$, ��� $i,j,k,l \in [n-1]$, $i \le k$ � ���� $i = k$, �� $j \le l$.
�������� ����������� ������ ������������� ������������� (��. ������� \eqref{eq:BQP}), $y(ij,kl) = y(ij,ij) y(kl,kl)$.

��������, ��������� 
\[
y(ij,il) = 0, \qquad i,j,l \in [n-1], \quad j < l,
\]
������ ��������� ����� $F_1$ ����� �������������.
�� ����� �������
\[
\sum_{j\in[n-1]} y(ij,ij) \le 1, \qquad i \in [n-1], \text{ ��� ���� }\bm{y}\in F_1.
\]
������� 
\begin{equation}
\label{eq:ATSP2BQP1}
\sum_{j\in[n-1]} y(ij,ij) = 1, \qquad i \in [n-1],
\end{equation}
�������� � ������������ ����� $F_2 \subseteq F_1$.

����������, ���������
\[
y(ij,kj) = 0, \qquad i,j,k \in [n-1], \quad i < k,
\]
�
\begin{equation}
\label{eq:ATSP2BQP2}
\sum_{i\in[n-1]} y(ij,ij) = 1, \qquad j \in [n-1],
\end{equation}
������ ��������� ����� $F_3 \subseteq F_2$ ������������� $\BQP\left((n-1)^2\right)$.

����� �������, ��� ������� $\bm{y} \in F_3$,
�������� ���������� \eqref{eq:ATSP2BQP1} � \eqref{eq:ATSP2BQP2},  
���������� $y(ij,ij)$, $i,j\in[n-1]$, �������� ���������� 0/1"~�������,
� ������� � ������ ������ � ������ ������� ������� ����� ���� �������.
� ���������, �������� ����� $F_3$ �� ���������� $y(ij,ij)$, $i,j\in[n-1]$, ��������� � �������������� �������� $\Birk(n-1)$.

��� $\bm{x} \in \ATSP(n)$ �������
\begin{align*}
x_{n,j} &= y(1\,j, 1\,j),     & j&\in[n-1],\\
x_{j,n} &= y(n-1\,j, n-1\,j), & j&\in[n-1],\\
x_{j,l} &= \sum_{i\in[n-2]} y(i\,j, i+1\,l), & j,l&\in[n-1], \ j\neq l.
\end{align*}
�������� ���������, ��� ��� �������� ����������� ���������� $F_3$ �� $\ATSP(n)$.
\end{proof}


\begin{lemma}[\cite{Maksimenko:2017LOP}]
	$\Stable(G) \lee \LOP(2n)$ ��� ������ ����� $G = (V,E)$, $|V|=n$.
\end{lemma}
\begin{proof}
	���������� ����� $F$ ������������� $\LOP(2n)$, ������� � ����������� ������� ��������������� $x_{i, n+j} = x_{j, n+i} = 0$, $\{i,j\} \in E$, $i < j$.
	����� �� 3-��������� ���������� (��.~\eqref{3cycle})
	\[
	0 \le x_{i, j} + x_{j, n+i} - x_{i, n+i} \quad \text{�} \quad
	x_{i, j} + x_{j, n+j} - x_{i, n+j} \le 1
	\]
	������� $x_{i, n+i} + x_{j, n+j} \le 1$, $\{i,j\} \in E$.
	�� ���� ����������� $\alpha\from x_{i, n+i} \mapsto y_i$ ���������� ����� $F$ � ������������ $\Stable(G)$.
	
	�������� ��������, ��� ��� ������� $\bm{y} \in \Stable(G)$ �������� $\bm{x} \in F$ �����, ��� $\bm{y} = \alpha(\bm{x})$.
	
	������� ����������� $\bm{y} \in \Stable(G)$ � �������
	\[
	I_0 = \Set*{i \in [n] \given y_i = 0}, \quad I_1 = \Set*{i \in [n] \given y_i = 1}.
	\]
	����� ������������, ��� �������� �������� $I_0 = \{i_1, \dots, i_k\}$ � $I_1 = \{i'_1, \dots, i'_{n-k}\}$ ������������� (�� ����� ���).
	�������� ������� ��� ��������������� ������� $\bm{x} \in F$ ���������� ������������� $\pi \from [n] \to [n]$ (��. ������� \eqref{eq:piLinear}).
	�������
	\begin{align*}
	\pi(n+i_s) &= s, &
	\pi(i_s) &= 2n-k + s, & s &\in [k],\\
	\pi(n+i'_t) &= n + t, &
	\pi(i'_t) &= k + t, & t &\in [n-k].
	\end{align*}
	�� �������� ������������ $\pi$ �������, ��� 
	$x_{i_s, n+i_s} = 0$,  ��� $s \in [k]$, �
	$x_{i'_t, n+i'_t} = 1$, ��� $t \in [n-k]$.
	������, ���� $x_{i, n+j} = 1$ ��� ��������� $i,j \in [n]$, �� $x_{i, n+i} = x_{j, n+j} = 1$.
	�� ���� �� ������� $x_{i, n+i} + x_{j, n+j} \le 1$ ������� $x_{i, n+j} = x_{j, n+i} = 0$.
\end{proof}

�������������� ���������� ����������� ��������.

\begin{lemma}
����� � ��������� �������������� $P$ ��������� ������ ������� ������������� ����������� (������������) �� �������������� ������������ ������� ������������� �����.
%, � ����� ������ ���������� ������ ��������� �� ����������� (����� ���������).
����� $P$ ���������� ������� �������� � ��������� ���������� $\Delta$. (����� ��������� �������� ��������� ��� ������.)
\end{lemma}
%\begin{proof}
%\end{proof}

%%%%%%%%%%%%%%%%%%%%%%%%%%%%%%%%%%%%%%%%%%%%%%%%%%%%%%%%%%
%
%     ������� ����
%
%%%%%%%%%%%%%%%%%%%%%%%%%%%%%%%%%%%%%%%%%%%%%%%%%%%%%%%%%%

\section{������� ���� ��� ��������������}
\label{sec:Cook4Polytopes}

����� $C$~--- ������ ������� � ������������� ���������� ����� (��. ����������� �� �.~\pageref{def:CNF}).
��������� ����� $\var(C)$ ����� ����������, ����������� � ���� �������.
����� $C(\bm{x})$ ��������� ��������, ������������ ���� �������� ��� ������ �������� ���������� $\bm{x} \in \{0,1\}^{\var(C)}$.
��� ��������� ���� ����� ������� ������ ������ ����������� CNF.

�������� ������� ����~\cite{Cook:1971}, ��� ������ ����� $L \subseteq \{0,1\}^*$ �� ������ NP ���������� ������������� ���������� ������������ $n\in\N$ �������������� $T = T(n)$, $T(n) \in \text{CNF}$, �����, ��� ��� ���� $n\in\N$ ��������� $\var(T(n)) \ge n$, � ��� ������� $\bm{x} \in \{0,1\}^n$ ����������� %$\var(T_n) \ge n$ � 
\[
\bm{x}\in L  \iff \text{�������� }\bm{y}\in\{0,1\}^{\var(T)-n} \text{ �����, ��� } T(\bm{x},\bm{y}),
\]
��� $T(\bm{x},\bm{y})$~--- ��������, ������������ ������� �������� $T$ ��� ����������� ������ �������� ���������� $(\bm{x},\bm{y})$.

�������������� ��������� ������� ��������� ���������� �������������� �������~\ref{thm:NP2comb}.

\begin{theorem}
����� �������� $(L,d,k,g)$ ���������� ��������� ��������� �������������� $P$,
� �������� ������������ $g$ ����������� ������ NP.
����� ��� ��������� ���������� ������� �������� � ��������� �������������� ������ � ������������ $\SAT$.
\end{theorem}

� ���������, ��� ����������� ���� � ��������� ������ ��������� �������������� ������������� �������� ���� �������.

� ���������� ����� ���� ��������, ��� ������ ������������ ������������� ������� �������� � ������ ��������� ���������� ��������������, ��������������� � NP-�������� ��������.
������� ������, ��� ��� ��� ��������� ���������� ������� �������� � $\BQP$.

\begin{theorem}[\cite{Maksimenko:2012Cook}]
\label{thm:SAT2BQP}
����� $U$ "--- ����� ������� ����������, $\cC$ "--- ����� ���������� ��� $U$, $\len(\cC)$ "--- ��������� ����� ���� ���������� �� ������ $\cC$, ���������� � ���������.
�����
\[
\SAT(U,\cC) \lee \BQP(n), \quad \text{��� } n = |U| + \len(\cC).
\]
\end{theorem}

%\emph{��������������.}
\begin{proof}
������������� ���������� �������� ������ ������������ � ������ ����� �� ��������������� ������ ����� \cite{Karp:1972}.

������ ��������� $[n]$ ��� ���������� ��������� $x_{ij}$ ������� $\bm{x} \in \BQP(n)$ ����� ������������ ����������
\[
R = U \cup \Set*{(a, C_i) \given C_i \in \cC, \ \text{ $a$ "--- �������, �������� � $C_i$}}.
\]
��������������, ���������� ������� $\bm{x} \in \BQP(n)$ ���������� $x(r, r')$, $r, r' \in R$.
�~���������, $x(r, r') = x(r, r) x(r', r')$ ��� ���� $\bm{x} \in \BQP(n)$.

���������� ������������ ������ $F$ ������������� $\BQP(n)$, ��������������� ��������� ������������:
\begin{align}
x\bigl((u,C_i), (u,C_i)\bigr) &\le x\bigl(u,u\bigr), && C_i \in \cC, \quad u \in C_i, \label{eq:SAT2BQP1}\\
x\bigl(u,u\bigr) \cdot x\bigl((\bar{u},C_i), (\bar{u},C_i)\bigr) &= 0, && C_i \in \cC, \quad \bar{u} \in C_i, \label{eq:SAT2BQP2}\\
x\bigl((u,C_i), (u,C_i)\bigr) \cdot x\bigl((\bar{u},C_j), (\bar{u},C_j)\bigr) &= 0, && C_i, C_j \in \cC, \quad u \in C_i, \quad \bar{u} \in C_j, \label{eq:SAT2BQP3}\\
\sum_{a\in C_i} x\bigl((a,C_i), (a,C_i)\bigr) &= 1, && C_i \in \cC. \label{eq:SAT2BQP4}
\end{align}
�������, ��� $F$ �������� ������ ������������� $\BQP(n)$ (������, $\conv(F)$ �������� ������ $\conv(\BQP(n))$), ��� ��� ������ �� ���� ����������� ������������ �������������� � ��������� ������� ��������������. � ������, �����������~\eqref{eq:SAT2BQP1} ������������� ������� �������������� $x\bigl((u,C_i), (u,C_i)\bigr) = x\bigl(u,(u,C_i)\bigr)$, �����������~\eqref{eq:SAT2BQP2} ������������� ������� �������������� $x\bigl(u,(\bar{u},C_i)\bigr) = 0$, �~\eqref{eq:SAT2BQP3} ������������� $x\bigl((u,C_i), (\bar{u},C_j)\bigr) = 0$.
� ���� �������, ���������~\eqref{eq:SAT2BQP4} ������������ ���������������
\[
\sum_{a\in C_i} x\bigl((a,C_i), (a,C_i)\bigr) - 2 \sum_{\substack{a,b\in C_i,\\ a \ne b}} x\bigl((a,C_i), (b,C_i)\bigr) = 1.
\]

��������� ���������� ������� $\bm{y} \in \SAT(U,\cC)$ ����� $y(u)$, $u\in U$,
���������� �������� ����������� $\beta\from x(u,u) \mapsto y(u)$, ������������ ����� $F$ �� $\SAT(U,\cC)$.
��������� $\beta(F) \subseteq \SAT(U,\cC)$ ������� �� ����, ��� ����� $\bm{x} \in \BQP(n)$, ��������������� ������������ \eqref{eq:SAT2BQP1}--\eqref{eq:SAT2BQP4}, ������������� ���������� ������ �������� ���������� $U$, ������������ ������ ���������� �� $\cC$.
�������� ��������� ������� �� ����, ��� ��� ������ ������ �������� ���������� $U$, ������������ $\cC$, �������� ��������� ������ ������� $\bm{x} \in F$, ���������� $x(u,u)$ �������� ��������� �������� ��������������� ����������.
\end{proof}

%% Глава 6
% !TeX encoding = windows-1251
% !TEX root = MaksimenkoThesis.tex

%%%%%%%%%%%%%%%%%%%%%%%%%%%%%%%%%%%%%%%%%%%%%%%%%%%%%%%%%%
%
%     ����������� �������������
%
%%%%%%%%%%%%%%%%%%%%%%%%%%%%%%%%%%%%%%%%%%%%%%%%%%%%%%%%%%
\chapter{����������� �������������}
\label{chap:Cyclic}

%\hfill
%\begin{minipage}{0.5\textwidth}
%����������� ������������� �������� ������������ ������ ������ ����� ���� �������� �������������� ��� �� ����������� � � ����� �� ������ ������.
%\begin{flushright}
%Peter McMullen
%\emph{�.~���������}
%\end{flushright}
%\end{minipage}

��� ��������~\cite{McMullen:1970}, ����������� ������������� �������� ������������ ������ ������ (����� �����������) ����� ���� �������� �������������� ��� �� ����������� � � ����� �� ������ ������.
��������� ����� �������������� ��� �������� ������� ����������������� ����� ��� �������� ������� ���� ������������� �����������.
���� � ��������� ����� ����� �������� ��� �������� ���� ��������������.
� �������~\ref{sec:EF4Cyclic} ��� ������������� $\CP_d([n])$ ���������� �������� ����������� ������������ ������� $2\bigl(2\lfloor \log_2(n-1)\rfloor+2\bigr)^{\lfloor d/2 \rfloor}$ ��� $2 \le d < n$.
� �������~\ref{sec:RidgeGraph} ����������� ������ �������� ��� �������� ����� �������������, ������������� � ������������.


\section{����������� � ��������}

� ���� ������� ���������� ����������� ����������� �������������� � ������������� �� ��������� ��������.
����� ��������� ���������� � ����������� �������������� ����� ����� �~\cite{Grunbaum:2003} �~\cite{ZieglerBook}.

����� $T = \{t_1, \dots, t_n\} \subset \R$, $t_1 < t_2 < \dots < t_n$.
����������� $d \in \N$, $2 \le d < n$, � ������ �����������
\begin{equation}
\label{eq:x(t)}
\bm{x}(i) \coloneqq (t^{\phantom{1}}_i, t^2_i, \dots, t^d_i) \in \R^d, \qquad i\in[n]. 
\end{equation}
������������ ����������� ������������ ������������� (��. �.~\pageref{page:cyclic}) � �������������� ����� �����������.
\emph{����������� ��������������} ���������� �������� �������� ���������
\[
\CP_d(T) \coloneqq \{\bm{x}(1), \dots, \bm{x}(n)\}.
\]
������ $\CP_d(T)$ "--- ��������� ������ ����� �������������.
����� $i\in[n]$ ����� �������� \emph{�������} ������� $\bm{x}(i)$.

� ������������ ��������������� ������� ������ ������ ��������� ���������� �� ��������� ����� ����� ������� (��. ����������� �� �.~\pageref{def:PolyMax}).
� ������, ���� ������ ������������� ��������� �������������� ���� $\CP_d([a,b] \cap \Z)$.

������������� �������� ������������ ������������� ������������ ��������� ������������.

\begin{theorem}[������� �������� �����~\cite{Gale:1963}]
������������ $\CP_d(T)$ ������������, �� ���� ������ ��� ���������� �������� ����� $d$ ������. ������ ������������ ������ � �������� �� $S \subset [n]$, $|S|=d$, �������� ���������� ����� � ������ �����, ����� ��������� <<������� ��������>>:
\[
\text{�������� ��������� } [k_1, k_2]\cap S \text{ ����� ��� ���� } k_1, k_2 \in [n] \setminus S, \ k_1 < k_2.
\]
\end{theorem}

��� $n=7$ ��������� ��������, ��������������� ������� ��������, ����� �������
$\{1,5,6\}$, $\{1,7\}$, $\{2,3,4,5\}$ � $\{1,2,4,5,7\}$.
����� �������, ��������� $S \subset [n]$, ��������������� ������� ��������, ���������� ����������� �� ���� ���� $\{i,i+1\}$ �, ���� �����, �������� $1$ � $n$.

�������������� �������������� ������� �������� ����� ��������� �� ��������� �����:
\begin{enumerate}
\item ��� ������ $\bm{a} \in \R^d$ �������� �������� ������� $g(\bm{x}) = \bm{a}^T \bm{x}$ � ������ $\bm{x}(i)$, $i\in[n]$, ��������� $\CP_d(T)$ ��������� �� ���������� ���������� $f(t) = a_1 t + a_2 t^2 + \dots + a_d t^d$ � ������ $t_i$. �� ���� ��������� ���� $f(t) = b$ ������ �������������� $\bm{a}^T \bm{x} = b$.
\item ��������� ���� 
\begin{equation}
\label{eq:CyclicPol}
f(t) = (t - t_1)(t_n - t) \prod_{i \in I} \bigl((t - t_i)(t - t_{i+1})\bigr), \quad I\subset [n],
\end{equation}
��������� ������� �������� � ������ $t_1$, $t_n$, $t_i$, $t_{i+1}$, $i\in I$, � ������������� �������� � ��������� ������ �� $T$. �������������, �������� ����������� ������, ������������ ���������� $f(t)$ ���������� �������������� $\bm{a}^T \bm{x} = b$, ���������� ������� � $\CP_d(T)$ � ���������� ����� ����� $\bm{x}(1)$, $\bm{x}(n)$, $\bm{x}(i)$, $\bm{x}(i+1)$, $i\in I$.
\end{enumerate}
%����� �������, ��� ������� ���������, ���������������� ������� ��������, ����� ��������� ��������� ����~\eqref{eq:CyclicPol}, ������������ �������������� $f(t) = 0$, ������� ��� $\CP_d(T)$.


\section{���������� ����������� ������������}
\label{sec:EF4Cyclic}

�����, ��� � ������, $T = \{t_1, \dots, t_n\} \subset \R$, $t_1 < t_2 < \dots < t_n$, $d \in \N$, $2 \le d < n$.
� ���� ������� �� �������������� ������������� ������ ��� �������, ����� $t_{i+1} = t_i + 1$ ��� ���� $i\in[n-1]$, � ����� ������������ ����� ���������� ������������
\[
\CP_{d,n}(t_1) \coloneqq \CP_d(T).
\]

�������, ��� ������������� $\CP_{d,n}(t)$ � $\CP_{d,n}(s)$ ������� ��������� ��� ����� $t,s \in \R$. 
� ������, �� ������ $(t + (s-t))^i = \sum_{j=0}^i \binom{i}{j} (s-t)^{i-j} t^j$ �������� ��������� �������� �����������, ������������ ���������� ����� $\CP_{d,n}(t)$ � $\CP_{d,n}(s)$:
\begin{equation}\label{eq:affine_isomorphism} 
y_i:=(s-t)^i+\sum_{j=1}^i \binom{i}{j} (s-t)^{i-j} x_j, 
\end{equation}
��� $(x_1,\ldots,x_d) \in \CP_{d,n}(t)$, $(y_1,\ldots,y_d) \in \CP_{d,n}(s)$.

�������� ������ ��� ���������� ���������, �� ����� ������������ ������� ������������ $\CP_{d,n}$ � ��� �������, ����� ����� ���������� �������� �� ������ �� ��������� �����������.


\subsection{������ $d=2$}


\begin{lemma}[\cite{BogomolovFMP:2015}]
\label{lem:two_dim}
$\xc(\CP_{2,n}(t)) \le 2\lfloor \log_2(n-1)\rfloor+2$, ��� $n \ge 3$ � $t \in \R$.
\end{lemma}

\begin{proof}
��������~\eqref{eq:affine_isomorphism}, ������������~$\CP_{2,n}(t)$ ����� ������� ������������� � ������������
\[
\CP_{2,n}(-(n-1)/2).
\] 
	
� ���� �������, ��� ������� $k\in\N$ ������������~$\CP_{2,2k+1}(-k)$ ����� ���� ����������� ��� �������� �������� ���� �������������� $\CP_{2,k+1}(-k)$ �~$\CP_{2,k+1}(0)$:
%
\[
\CP_{2,2k+1}(-k) = \conv(\CP_{2,k+1}(-k) \cup \CP_{2,k+1}(0)).
\]
����������� ������������� ���������� � ��� ������������� $\CP_{2,2k}(-k + 1/2)$:
\[
\CP_{2,2k}(-k + 1/2) = \conv(\CP_{2,k}(-k + 1/2) \cup \CP_{2,k}(1/2)), \quad k\in\N.
\]
	
������������� ���, ��� ������������ $\CP_{2,k+1}(-k)$ �������� ������� ������������� $\CP_{2,k+1}(0)$ ��� ���������� ��������� ������������ �������������� $x_1 = 0$ (����������� ������������ ������ ����� ������ ����������). �� �� ����� � � ��������� $\CP_{2,k}(-k + 1/2)$ � $\CP_{2,k}(1/2)$.

���� ���� ��������� �� ������ �������� ���������� ��������� (��.~\cite[Theorem~2]{KaibelPashkovich:2013}) �~\eqref{eq:affine_isomorphism} ������� ����� � ���, ��� ������ ����������� ������������ ������� $h$ ��� ������������� $\CP_{2, \lceil n/2 \rceil}$ ���������� ����������� ������������ ������� $(h+2)$ ��� $\CP_{2,n}$. � ���������, ����������� ������������ ��� $\CP_{2,2k+1}(-k)$ ����� ���� �����:
\[
\CP_{2,2k+1}(-k)=\Set*{(x_1,x_2)\in\R^2 \given \text{�������� } z_1, \text{ ��� }(z_1,x_2)\in \CP_{2,k+1}(0) \text{ � } -z_1 \le x_1\le z_1}\,.
\]
����� �������, $\xc(\CP_{2k+1}) \le \xc(\CP_{k+1}) + 2$ � $\xc(\CP_{2k}) \le \xc(\CP_{k}) + 2$.
	
����������� $\xc(\CP_{2,n}) \le 2 \lfloor \log_2(n-1) \rfloor + 2$ ����� ����������� ��� $n = 3,4,5$. 
�������������, ��� ���������� �������������� ���������� ��������������� �������������
\[
\xc(\CP_{2,2k}) \le \xc(\CP_{2,k}) + 2 \le  2 \lfloor \log_2(k-1) \rfloor + 4 =2 \lfloor \log_2 (2k-2) \rfloor + 2
\]
�
\[\xc(\CP_{2,2k-1}) \le \xc(\CP_{2,k}) + 2 \le 2 \lfloor \log_2(k-1) \rfloor + 4 = 2 \lfloor \log_2 \big((2k-1) - 1\big) \rfloor + 2\,.
\]
����� �������,
\[
\xc(\CP_{2,n}) \le  2\lfloor \log_2(n-1)\rfloor+2
\]
��� ���� $n\ge 3$.
\end{proof}

\begin{remark}
����� ���������� ������ ��������� ����� �������������� �����~\ref{lem:two_dim}
��������� �������� ������� ������ ��������� ���������� �� 
\[
\xc(\CP_{2,n}) \le 2\lfloor \log_2(n-1)\rfloor + 1 + \delta_n,
\]
��� $\delta_n = 0$, ���� �������� $k\in \N$ �����, ��� $2^k < n \le 3 \cdot 2^{k-1}$, � $\delta_n = 1$ � ��������� �������.
\end{remark}


\subsection{������ $d \ge 3$}

� ���������, ��� $d \ge 3$ ����������� ����� ��������������� $\CP_{d,k+1}(-k)$ �~$\CP_{d,k+1}(0)$ �� ��� ������: ������������ $\CP_{d,k+1}(-k)$ �������� �������~$\CP_{d,k+1}(0)$ ��� ����� ����� ��������� � �������� ��������. ��� �� ������������� ��������� ������������ �������������� �, �������������, �� �� ����� ��������������� ���������� ���������� ���������. 

������, �� ����� ��������������� �������� ���������� (�������~\ref{thm:Yannakakis} �� �.~\pageref{thm:Yannakakis})
� ������������� ��������������� ������������� ������� ������� (��. �����������~\ref{def:slack} �~\ref{def:nonneg}) �������������~$\CP_{2,n}$, ������������� ������~\ref{lem:two_dim}. 

������ ������� ������� $M_{d,n}$ ������������� $\CP_{d,n}(t)$ ����� ������������� ���������� �� $[n]$, � ������� "--- ����������� $S \subset [n]$, $|S|=d$, ���������������� ������� �������� �����.
�������� ����������� ���� ������������ (��. �������~\eqref{eq:CyclicPol} � ����������� � ���), ��� ������� ����� ���� ������������ ���
%
\begin{equation}
\label{eq:slack}
M_{d,n}(i,S) := \prod_{j \in S} |t_j - t_i|\,,
\end{equation}
%
��� $i\in[n]$ � $S\subseteq [n]$, $|S|=d$, ������������� ������� �������� �����.
� ���������, ������� ������� �� ������� �� ���������� ��������� $t$ � ����������� ������������� $\CP_{d,n}(t)$, ��� ��� �� �� ������ �������� $t_j - t_i$. 

���������� ��������� ��������� �������� ���������������� ����� �������.

\begin{property}\label{prop:nonneg}
��������� �������� ��� �������� (���������) ������� �� ����������� �� ��������������� ����:
\begin{enumerate}
	\item ������������ ����� (��������).
	\item ������������ ������ (�������).
	\item �������� ������ (�������).
	\item ��������� ������ (�������) �� ��������������� �����.
	\item ���������� � ������� ����� ������ (�������), ������(���) ���������� ���������� ����� (��������).
\end{enumerate}
\end{property}

\begin{property}\label{prop:nonnegsum}
����� $M \in \R_+^{m\times n}$ � $S\subseteq[n]$. ����� $M_1$ "--- ���������� ������� $M$, ������������ �� �������� � �������� �� $S$, � $M_2$ "--- ����������, ������������ �� �������� � �������� �� $[n]\setminus S$.
�����
\[
\rank_+(M) \le \rank_+(M_1) + \rank_+(M_2).
\]
\end{property}

��� ���� ������ $A$ � $B$ ����������� �������, ��������� ������������ ������������ $A \circ B$ � ������� ��������� $(A \circ B)(i,j):=A(i,j) B(i,j)$. ����� ������������� ���� ������� ��� ����������� ��������� ����.

%\cite[Lemma~3.5]{FioriniKPT:13}
\begin{lemma}[\cite{FioriniKPT:13, BogomolovFMP:2015}] \label{lem:Kronecker}
��� ����� ���� ������ $A$ � $B$ � ���������� ������ ����� � ��������,
\[
\rank_+(A \circ B) \le \rank_+(A) \rank_+(B).
\]
\end{lemma}
%
%\begin{proof}
%����� $A,B \in \R_+^{n\times m}$ �
%\begin{align*}
%A &= TU, && \text{��� }T\in \R_+^{n\times r}, \quad U\in \R_+^{r\times m},\\
%B &= HW, && \text{��� }H\in \R_+^{n\times s}, \quad W\in \R_+^{s\times m}.
%\end{align*}
%����� $M_i$ ����� ���������� $i$-� ������ ������� $M$, � ����� $M^j$ "--- $j$-�� �������.
%������ $C_i$, $i\in[n]$, ������� $C \in \R_+^{n \times rs}$ ��������� ��������� �������������
%\[
% C_i = T_i\otimes H_i,
%\]
%� ������� $D^j$, $j\in[m]$, ������� $D \in \R_+^{rs \times m}$ "--- �������������
%\[
%D^j = U^j\otimes W^j.
%\]
%�����
%\[
%A\circ B = CD.
%\]
%\end{proof}

�������� ��������������� � �������������� ��������� ����������� ����� �������.
������ ����� ���������� ������ �����������, �������~$d = 2q$, $q\in \N$.

\begin{lemma}[\cite{BogomolovFMP:2015}]
\label{lem:even_case}
$\xc(\CP_{2q, n}) \le \big(\xc(\CP_{2, n})\big)^q$ ��� $q,n\in\N$, $2q < n$.
\end{lemma}

\begin{proof}
�� �������� $q$ ������ $B_1$, \dots, $B_q$ �����, ��� ������������ ������������ $B_1 \circ \cdots \circ B_q$ ����� ������� ������� $M_{2q, n}$. 
� ���� ����� �������, ��� ������ ��������� $S\subseteq[n]$, $|S|=2q$, ��������������� ������� �������� �����, ����� ���� ������� �� $q$ ��� $S_1$, \dots, $S_q$, ��� ������ ���� ���� ����� $\{1,n\}$, ���� ������� �� ���� ���������������� �����. 
������ ��������� $S_r$, $1\le r\le q$, ����� ������������� ������� �������� ����� � ������� �� ���� ���������. 
�������������, ��� ������� $S_r$ �������� ��������������� ������� � $M_{2, n}$ �����, ���
\[
	M_{2,n}(i,S_r)=\prod_{j\in S_r} |j-i|\,.
\]
������ ��������� ������ ������ $B_1$, \dots, $B_q$ � �������, ��������������� ���������� $S$, ��� $B_r(i,S):=M_{2,n}(i,S_r)$. 
�������, ���
\[
(B_1 \circ \cdots \circ B_q)(i,S) = \prod_{r = 1}^q \prod_{j \in S_r} |j-i| = \prod_{j \in S} |j-i| = M_{2q,n}(i,S)\,.
\]
	
�������� ���������, ��� ������� $B_1$, \dots, $B_q$ �������� �� $M_{2,n}$ �� ���� ������������, �������� � ������������ ��������. 
�������������, �������� c�������~\ref{prop:nonneg}, ��������������� ���� ������ �� ���� ������ ��������� ������ ��������������� ������ ������� $M_{2,n}$. 
����� �������, �������� �����~\ref{lem:Kronecker}, ������� ������� $M_{2q, n}$ ��������� ��������������� ������������ ������� $\big(\xc(\CP_{2,n})\big)^q$.
\end{proof}

%\subsection[������ d=2q+1]{������ $\di=2q+1$}


\begin{lemma}[\cite{BogomolovFMP:2015}]
\label{lem:odd_case}
$\xc(\CP_{2q+1,n}) \le 2 \xc(\CP_{2q,n-1})$ ��� $q,n\in\N$, $2q+1 < n$.
\end{lemma}
%
\begin{proof}
�������, ��� $\xc(\CP_{2q+1, n}(1)) \le \xc(\CP_{2q, n-1} (2)) + \xc(\CP_{2q, n-1}(1)) = 2 \xc(\CP_{2q, n-1})$.
	
��� ������� ��������� $S\subseteq[n]$, $|S|=2q+1$, ���������������� ������� �������� �����, ����������� ����� ���� �� ���� �������:
%
\begin{enumerate}
	\item \label{case:last_element} $n\in S$ � ��������� $S\setminus\{n\}$ ���������� ���������� ������������� $\CP_{2q, n-1}(1)$.
	\item \label{case:first_element} $1\in S$ � ��������� $S\setminus\{1\}$ ���������� ���������� ������������� $\CP_{2q, n-1}(2)$.
\end{enumerate}
�������� ������� ������� $M_{2q+1, n}$ �� ��� ����������. ���������� $M_1$ ����� �������� �� ��������, ��������������� �������~\ref{case:last_element}, � ���������� $M_2$ "--- �� ��������, ��������������� �������~\ref{case:first_element}.
�������� ��������~\ref{prop:nonnegsum}, 
\[
\rank_+(M_{2q+1, n}) \le \rank_+(M_1) + \rank_+(M_2).
\]

������ ����� �������, ��� ��������� ������ ������� $M_1$ � ������ ������ ������� $M_2$ ������� �� �����.
��� $i \in [n-1]$, $i$-� ������ ������� $M_1$ ����� $i$-�� ������ ������� $M_{2q, n-1}$, ���������� �� ������������� ����� $n-i$.
����� �������, $\rank_+(M_1) = \rank_+(M_{2q, n-1})$.
����������, ��� $i \in [2,n]$, $i$-� ������ ������� $M_2$ ����� $(i-1)$-�� ������ ������� $M_{2q, n-1}$, ���������� �� ������������� ����� $i-1$.
�������������, $\rank_+(M_2) = \rank_+(M_{2q, n-1})$.
\end{proof}

�� ���� \ref{lem:two_dim}, \ref{lem:even_case} � \ref{lem:odd_case} �������

\begin{theorem}[\cite{BogomolovFMP:2015}]
\label{thm:main}
$\xc(\CP_{d,n}) \le 2\bigl(2\lfloor \log_2(n-1)\rfloor+2\bigr)^{\lfloor d/2 \rfloor}$ ��� $2 \le d < n$.
\end{theorem}

�������, ��� �����~\ref{lem:two_dim} ����������� ���������� ��� ����, ��� $\CP_{2,n}$ �������� �������� ��������� ����� $(i,i^2)$ ��� $n$ \emph{����������������} ����� $i \in [n]$. � ����������������, �~\cite{Fiorini:2012polygons} ��������, ��� ����� ��������� ����������� �������������� ���� $P = \conv \Set{(i,i^2) \given i \in X}$, ��� $X \subset [2n]$ � $|X| = n$, ���� �����, ��������� ����������� ������������ ������� ���������� ����� ��������� $\Omega(\sqrt{n} / \sqrt{\log n})$. 



%%%%%%%%%%%%%%%%%%%%%%%%%%%%%%%%%%%%%%%%%%%%%%%%%%%%%%%
%
%  �������� ������������ ������� ������� ��� d=2
%
%%%%%%%%%%%%%%%%%%%%%%%%%%%%%%%%%%%%%%%%%%%%%%%%%%%%%%%


\subsection{�������� ������������ ������� ������� ��� $d=2$}

� ���� ������� ��� ������� ������� $M_{2,n}$ ������� ����� ��������������� ������������ ������� $2\lceil\log(n-1)\rceil + 1$. 
��� ������������ ������ �� ��, ��� ������� �~\cite{Fiorini:2012polygons} ��� ������� ������� ����������� ��������������.

�������� ����������� ������� ������� ������������ ������������� $\CP_{2,n}$,
\[
M_{2,n}(i,j) = 
\begin{cases}
(i-j)(i-j-1), & \text{��� } j < n,\\
(i-1)(n-i), & \text{��� } j = n.
\end{cases}
\]
��� $j \in [n-1]$, $j$-�� ������� ���� ������� ������������� ��������� $\{j,j+1\}$, � $n$-�� ������� "--- ��������� $\{1,n\}$.

� ���������,
\begin{equation*}
\begin{split}
M_{2, 8} & =
\begin{pmatrix}
0	&  2	&  6	& 12	& 20	& 30	& 42	&  0 \\
0	&  0	&  2	&  6	& 12	& 20	& 30	&  6 \\
2	&  0	&  0	&  2	&  6	& 12	& 20	& 10 \\
6	&  2	&  0	&  0	&  2	&  6	& 12	& 12 \\
12	&  6	&  2	&  0	&  0	&  2	&  6	& 12 \\
20	& 12	&  6	&  2	&  0	&  0	&  2	& 10 \\
30	& 20	& 12	&  6	&  2	&  0	&  0	&  6 \\
42	& 30	& 20	& 12	&  6	&  2	&  0	&  0
\end{pmatrix}               
\end{split}
\end{equation*}

��� ��������� ���������� ������������ ������� �� ���� ��������������� ����������� ����������.
������ �����, � ������� ���������~\ref{alg:leftfactor},
���������������� ������ ��������� ���� ��������� $\text{A}$, ����� $\text{q} = \lceil\log(n-1)\rceil$, � ����������� ����� ��������� ������������ $\text{T} \in \R_+^{n \times (2q+1)}$.
�����, ��������� ������ A, ��������~\ref{alg:rightfactor} ��������� ������ ��������� ������������ $\text{U} \in \R_+^{(2q+1)\times n}$.

��������� ���������� ����� ��������� ��� $n=8$ �������� ���:
\begin{equation*}
	M_{2, 8} = 
	\begin{pmatrix}
	0	& 7	& 0	& 4	& 0	& 2	& 0  \\
	0	& 5	& 0	& 2	& 0	& 0	& 6  \\
	0	& 3	& 0	& 0	& 2	& 0	& 10 \\
	0	& 1	& 2	& 0	& 0	& 0	& 12 \\
	1	& 0	& 2	& 0	& 0	& 0	& 12 \\
	3	& 0	& 0	& 0	& 2	& 0	& 10 \\
	5	& 0	& 0	& 2	& 0	& 0	& 6  \\
	7	& 0	& 0	& 4	& 0	& 2	& 0
	\end{pmatrix}               
	\cdot
	\begin{pmatrix}
	6	& 4	& 2	& 0	& 0	& 0	& 0	& 0 \\
	0	& 0	& 0	& 0	& 2	& 4	& 6	& 0 \\
	3	& 1	& 0	& 0	& 0	& 1	& 3	& 0 \\
	0	& 0	& 1	& 3	& 1	& 0	& 0	& 0 \\
	1	& 0	& 0	& 1	& 0	& 0	& 1	& 0 \\
	0	& 1	& 1	& 0	& 1	& 1	& 0	& 0 \\
	0	& 0	& 0	& 0	& 0	& 0 & 0	& 1
	\end{pmatrix}    \,.
\end{equation*}         

\SetAlgorithmName{��������}{������ ����������}{} % �������� �� �������
\SetAlgoCaptionSeparator{.} % �������� �������� ��������� �� ��� ������ ������
%\SetAlgoLined % ������������ �����, ����������� ������ � ����� �����
\DontPrintSemicolon % �� �������� ����� � �������
\SetKwProg{Proc}{���������}{}{�����} % ������� ��� ���������
\SetKwProg{Fn}{�������}{}{�����} % ������� ��� �������
% �������� ������� � �������� ������
\SetKwInOut{Input}{����}
\SetKwInOut{Output}{�����}
\SetKwRepeat{DoWhile}{�����}{����} % ������� ���� do-while
\SetKwFor{For}{���}{\string:}{�����~�����}
\SetKwIF{If}{ElseIf}{Else}{����}{��}{�����~����}{�����}{�����~����}
\SetKw{KwTo}{��}
%\SetKwBlock{Loop}{loop}{endloop} % ����������� ����

\begin{algorithm}
	\caption{����� ��������� ������������ ������� $M_{2,n}$} % ���������
	\label{alg:leftfactor}
	% ����������� �������� (�������-��������) ������ ���������
	\SetKwArray{A}{A} % ������
	\SetKwArray{T}{T} % ������
	\SetKwData{n}{n}
	\SetKwData{q}{q}
	% �������� ����� �������� � �������
	\SetKwFunction{LeftFactor}{LeftFactor}
	% �������� �����-������
	\Input{ ����� ������ \n}
	\Output{ ����� ��������� ������������ \T, ������ ��������� ���� ��������� \A � ��� ����� \q}
%	\BlankLine
%	\Fn{\LeftFactor{\n}}{
		\tcp{��������� ������ \A}
		$k \coloneqq \n-1$\;
		$\q \coloneqq 0$\;
		\DoWhile{$k > 1$}{
			$\A{\q} \coloneqq k$\;
			$k \coloneqq \lfloor(k+1)/2\rfloor$\;
			$\q \coloneqq \q + 1$\;
		}
		\tcp{��������� ����� ��������� \T}
		\For{$i \coloneqq 0$ \KwTo $\n - 1$}{
			$x \coloneqq i$\;
			\For{$j \coloneqq 0$ \KwTo $\q - 1$}{
				$r \coloneqq 2x - \A{j}$\;
				\eIf {$r > 0$} {
					$x \coloneqq x - r$\;
					$\T{i, 2j} \coloneqq r$\;
					$\T{i, 2j + 1} \coloneqq 0$\;
				}{
					$\T{i, 2j} \coloneqq 0$\;
					$\T{i, 2j + 1} \coloneqq -r$\;
				}
			}
			$\T{i, 2\q} \coloneqq i \cdot (\n - 1 - i)$ \tcp*[f]{��������� 	��������� �������}
		}
%	}
\end{algorithm}


\begin{algorithm}
	\caption{������ ��������� ������������ ������� $M_{2,n}$} % ���������
	\label{alg:rightfactor}
	% ����������� �������� (�������-��������) ������ ���������
	\SetKwArray{A}{A} % ������
	\SetKwArray{U}{U} % ������
	\SetKwData{n}{n}
	\SetKwData{q}{q}
	% �������� ����� �������� � �������
	\SetKwFunction{RightFactor}{RightFactor}
	% �������� �����-������
	\Input{ ����� ������ \n, ������ ���� ��������� \A � ��� ����� \q}
	\Output{ ������ ��������� ������������ \U}
	\BlankLine
%	\Fn{\RightFactor{\n, \A, \q}}{
		\For{$i \coloneqq 0$ \KwTo $\n - 2$}{
			$x \coloneqq i+1$\;
			\For{$j \coloneqq 0$ \KwTo $\q - 1$}{
				$r \coloneqq 2x - \A{j} - 1$\;
				\eIf {$r > 0$} {
					$x \coloneqq x - r$\;
					$\U{2j, i} \coloneqq 0$\;
					$\U{2j+1, i} \coloneqq r$\;
				}{
					$\U{2j, i} \coloneqq -r$\;
					$\U{2j+1, i} \coloneqq 0$\;
				}
			}
			$\U{2\q,i} \coloneqq 0$ \tcp*[f]{��������� ��������� ������}
		}
		\tcp{��������� ��������� �������}
		\For{$j \coloneqq 0$ \KwTo $2\q - 1$}{
			$\U{j, \n-1} \coloneqq 0$\;
		}    
		$\U{2\q, \n-1} \coloneqq 1$\;
%	}
\end{algorithm}

\FloatBarrier

%%%%%%%%%%%%%%%%%%%%%%%%%%%%%%%%%%%%%%%%%%%%%%%%%%%%%%%
%
% ����-���� ������������ �������������
%
%%%%%%%%%%%%%%%%%%%%%%%%%%%%%%%%%%%%%%%%%%%%%%%%%%%%%%%

\section{������� ����-����� ������������ �������������}
\label{sec:RidgeGraph}

����-���� $d$-������� ������������� ������������ ��������� �������. %(��. �.~\pageref{ridge-graph}). 
��� ������� ������������� ����������� �������������, � ��� ������� ������, ���� ��������������� ���������� ($(d-1)$-�����) ����� ����� ���� ($(d-2)$-�����).
����� �������, ����-���� ������������� �������� ������ ������������� �������������.

��� ��� ������������� ��������� ������������ ������������� ������� ������ �� ��� ����������� $d$ � ����� ������ $n$, � ���� ������� �� ����� ���������� ��� $\CP(d,n)$.
�� ������� ������������� $\CP(d,n)$ �������, ��� ������������ � ���� ������������ $\CP^*(d,n)$ �������� ������� � �������� ������������ ������ ������ ����� ���� $d$-������ ��������������, ������� $n$ �����������.
��� �������������� ������� ������� �� ��, ��� %, ����� ����� �������, 
$\CP^*(d,n)$ �������� �������� ������������� �~���������� ��������� �����. 
� ���������, ��� ��� $d=4$ � $n=9$ �� ������� ����� $4$,
� �� ����� ��� � ������~\cite{Klee:1967} ���������� 
������ $4$-������������� � ��� �� ������ ����������� � ��������� ����� ������~$5$.
%(������� ���� ��������~\cite{Altshuler:1980}, ��� ��� ������������, � ��������� �� ������������� ���������������, ��� ������ $d=4$ � $n=9$ ������ ������������� � ���������~$5$). 
������ �����������
\[
\dc=\diam \CP^*(d,n).
\]
� 1964 ���� �.~��� ������� \cite{Klee:1964}, ��� �������� ����� ����������� ��� $\CP^*(d,n)$: 
\begin{equation}
\label{Hirsh4Cycl}
\dc\le n-d,
\end{equation}
� ��� $d < n\le 2d$ � \eqref{Hirsh4Cycl} ����������� ���������.
��� �� ���� ��������� ������������� � ���, ��� ��� $n>2d$ ��������� 
��������� $\dc=\left\lfloor n/2\right\rfloor$ 
(���� ����� �~\cite{Klee:1967} ���� ��������, ��� ��� �������).
%�� ���� ��� ���������� � ���� ������������� (�� ������ ���������� ��������� ���������� ����� ������� "--- �������~\ref{thm:RidgeGraphDiamCyclic}) ������ \cite{Klee:1964} ��������� ����� ��������, ��� �� ���������� ����� 30~��� �������� ������� ���������� �������� $\dc$ ��� $n=2d$ � ������� �������������� ������� \cite{Ferrez:1998}, � � ������ \cite{Lagarias:1998} ��� ���������� �����~\eqref{Hirsh4Cycl} ���������� ������ �� ������ \cite{Klee:1966}, �� ���������� ����� ����������.

%������, ��� �������� ������ ����������� �� ������ V. Klee, P. Kleinschmidt, The d-step conjecture and its relatives, 1987, � ������� ���������� ������ �� \cite{Klee:1966}, ����, ���� �� ���������, ������ ���� �� \cite{Klee:1964}.

%��� ����, ����� ��������� ����� � ���� �������, 
%���� ���������� ������ �������� ��� $\dc$.


\begin{theorem}[\cite{Maksimenko:2009}]
\label{thm:RidgeGraphDiamCyclic}
������� $\dc$ ����-����� ������������ ������������� $\CP(d,n)$ ����������� �� �������:
\[
	\dc=
	\begin{cases}
	n-d 				& \text{ ��� } d < n \le 2d,\\
	n-d  - 
	\left\lceil 
	\frac{n-2d}{ \left\lfloor \frac d 2 \right\rfloor+1}
	\right\rceil 
	& \text{ ��� }  n > 2d.
	\end{cases}
\]
\end{theorem}

\subsection{\texorpdfstring{�������������� �������~\ref{thm:RidgeGraphDiamCyclic}}{�������������� �������}}

�������������� ������� ��� $d < n \le 2d$ �������� �.~���~\cite{Klee:1964}.
������� ����� ����� ������������, ���
\[
n>2d.
\]

��������� ����� $X=\{x^1, x^2, \ldots, x^n\}$ ��������� ������ ������������� $\CP(d,n)$, �����������, ��� ������� ������������� � ������� ����������� ��������� $t$: $t_1 < t_2 < \ldots < t_n$.

������ �� ������ ������ �����������.

\subsubsection{������ �����������}

����� \(d=2k\).
�������������� ����� �������� �� ���� ������. � ������ ����� �������� ������ ���� ����������� ������������ ������������� � �������, ��� ���������� ����� ���� ����� ����� ����� �� \(n-d - \frac{n-2d}{k+1}\).
�� ������ ����� �������, ���
\begin{equation}
\label{eq:RidgeIneq}
\dc \le n-d - \frac{n-2d}{k+1}.
\end{equation}

�� ������� �������� ����� �������, ��� ��������� ������ ������ ����������
������ ����� ������������ ������� ������� �� $k$ ���������������� ��� ����
\[
\{x^i, x^{i+1}\},
\]
��� $i\in[n]$ � �������� $i+1$ ����������� �� ������ $n$ (���� $i=n$, �� $i+1=1$).
�, ��������, ����� ������������ $Y\subset X$ ����
\[
Y=\{x^{i_1}, x^{i_1+1}, x^{i_2}, x^{i_2+1}, \ldots, x^{i_k}, x^{i_k+1}\}, \qquad \text{��� } i_j + 1 < i_{j+1}, \quad j \in [k-1],
\]
���� ��������� ������ ��������� ����������.

��� ����������� ����������� ��������� ������ ������� $x^i$ 
�������� � ������������ ����� $v_i$ �� ���������� ���������� �������:
\[
v_i=\bigl(\cos(2\pi i / n), \sin(2\pi i / n)\bigr).
\]
��� ����� $v_i$ � $v_{i+1}$ ����� �������� \emph{�����}, � ���������� $p_i$.
� ��������� ���� ����� ��� ��� $i\in[n]$ ��������� ${\mathcal P}$.

% ������������� ������� �����, ������ ����� � ����� ����� �� ���
\newcommand\BeginPic[2]{
	\def\rdot{1.5pt}
	\def\Rdot{3pt}
	\def\linew{1.5pt}
	\def\lineB{5pt}
	\def\lineW{3.3pt}
	\def\Radius{#1}
	\def\Larc{360/#2} % ����� ���. ���� � ��������
	%\def\Larc{20} % ����� ���. ���� � ��������
}	
\newcommand*{\Angle}[1]{(#1)*\Larc}
% ���������� ����� �� � ������
\newcommand*{\cvertex}[1]{({\Radius*cos(\Angle{#1})}, {\Radius*sin(\Angle{#1})})} % ���������� ��� ������ �����
\newcommand*{\nvertex}[1]{({\Radius*0.8*cos(\Angle{#1})}, {\Radius*0.8*sin(\Angle{#1})})}
% ���� ���������
\newcommand{\ArcO}[2]{
	\draw[dashed] \cvertex{#1} arc (\Angle{#1}:\Angle{#2}:\Radius);
}
% ���� ��������� � ���������������� ������� �� ���
\newcommand{\Arc}[2]{
	\ArcO{#1}{#2}
	\foreach \x in {#1, ..., #2} \draw[fill = black] \cvertex{\x} circle (\rdot*0.6) \nvertex{\x} node {\scriptsize\x};
}
% ������ �����
\newcommand\Fishka[1]{
	\draw[line width = \linew] 
	\cvertex{#1} arc (\Angle{#1}:\Angle{#1+1}:\Radius);
	\draw[fill = black] \cvertex{#1} circle (\rdot) 
	\cvertex{#1+1} circle (\rdot);
}
% ������ ������
\newcommand\Cell[1]{
	\draw[line width = \lineB] 
	\cvertex{#1} arc (\Angle{#1}:\Angle{#1+1}:\Radius);
	\draw[fill=white, line width = 0.5*\linew] 
	\cvertex{#1} circle (\Rdot) 
	\cvertex{#1+1} circle (\Rdot);
	\draw[line width = \lineW, draw=white] 
	\cvertex{#1} arc (\Angle{#1}:\Angle{#1+1}:\Radius);
}


% ������� 1
\begin{figure}[tbh]
	\begin{center}
		%  \PictA
		\begin{tikzpicture}[scale=1.2]
		\BeginPic{2}{18}
		\begin{scope}[xshift=-3cm]
		\ArcO{0}{1}
		\Arc{1}{18}
		\Fishka{18}
		\Fishka{2}
		\Fishka{5}
		\Fishka{8}
		\Fishka{16}
		\end{scope}
		
		\begin{scope}[xshift=3cm]
		\ArcO{0}{1}
		\Arc{1}{18}
		\Fishka{1}
		\Fishka{3}
		\Fishka{5}
		\Fishka{8}
		\Fishka{17}
		\end{scope}
		\end{tikzpicture}
	\end{center}
	\caption{��� ������� ���������� ������������� $\CP(10,18)$.}
	\label{fig:PictNeighborlyFaset}
\end{figure}


�����������, ��� ������� ��������� ���������� ������ ������� �� $k$ ���.
����� ������� � ������� ���������� �������� ����� ����� ��� ���������� �����������
���� � ����� ��� �� ���� ����� ����� ���������� (��. ���.~\ref{fig:PictNeighborlyFaset}). 
(�� ���������������� ������������ ������������� �������, 
��� ��� ��� ���������� ������, ���� ��� ����� ����� $d-1$ ����� ������.)
����� �������, ����� ������� � ����� ������� ����������, ���������� ������� ����
�� $k$ ��� � ����������� �������� (�� ������� ��� ������ ������� �������),
��� ����, ���� �� ���� �������� ���� ��������� ������ ����,
�� ��� ��������� � ��� �� �����������, ��� � ������, �~�.~�.
����� ����� ������������ ��������� ��������.

%%%%%%%%%%%%%%%%%%%%%%%%%%%%%%%%%%%%%%%%%%%%%%%%%%%%%%%%%%%%%%%%%%%%
%
%                      ������� ��������
%
%%%%%%%%%%%%%%%%%%%%%%%%%%%%%%%%%%%%%%%%%%%%%%%%%%%%%%%%%%%%%%%%%%%

\medskip
\textbf{������� ��������.} 
\emph{
	��� �������� ���� ����� ������������ ���������� � ���������� ����������� �������,
	������� � ��� �� ����, �� �� ����� ������������ � ������ �����.
}
\medskip

����, ������ ���������� ���������� ����� ��������� � ����-����� 
������������ ������������� �������� � ���������.
�� ���������� � $n$ ������� ������� $k$ ��� �����, ������������ ������� 
������ ����������, ����� �������� �� \emph{�������}, 
� �������� $k$ ���, ��������������� �������� ������ ����������, 
����� �������� �� \emph{��������} (��. ���.~\ref{fig:PictFishki}). 
��������� ����� ���������� ����� <<����������>> �������� �����, 
����������� ��� ����, ����� ��� ������ ���� ������ �������.
��������� ��� ����� $l(F_1,F_2)$, ����� $F_1$ --- ��� ��������� �����, 
� $F_2$ --- ��������� �����, $F_1,F_2\subset {\mathcal P}$. 


% ������� 2
\begin{figure}[tbh]
	\begin{center}
		\begin{tikzpicture}[scale=1.2, >={Stealth[scale width=0.8]} % ���������� ��� �������
		]
		\BeginPic{2}{18}
		
		\begin{scope}
		\ArcO{0}{1}
		\Arc{1}{18}
		\Cell{3}
		\Cell{5}
		\Cell{8}
		\Cell{10}
		\Cell{12}
		
		\Fishka{2}
		\Fishka{5}
		\Fishka{11}
		\Fishka{16}
		\Fishka{18}
		
		\draw[<-] \cvertex{8.5}  +(-0.15,0) -- +(-1,0) node[left] {������};
		\draw[<-] \cvertex{16.5} + (0.1,0) -- +(1,0) node[right] {�����};
		\draw[<-] \cvertex{1.5} + (0.1,0) -- +(1,0) node[right] {����������};
		\end{scope}
		\end{tikzpicture}
	\end{center}
	\caption{������ ����������� $F_1=\{1,2,3,5,6,11,12,16,17,18\}$ (�����)
		�~$F_2=\{3,4,5,6,8,9,10,11,12,13\}$ (������) ������������� $\CP(10,18)$.}
	\label{fig:PictFishki}
\end{figure}



������ ���� �� ��������� ${\mathcal P}\setminus(F_1\cup F_2)$ �������� � ������������ 
���� ���������� ����� ������� $v_i$ � $v_{i+1}$, �� ���������� ����� ���� �����.
������� ����� ���� \emph{�����������} (��. ���.~\ref{fig:PictFishki}) � ����� ���������� $a_i$. 
� ��������� ���� ����������� ��������� ����� $A$. 
��� ��� �� ���������� ����������� $k$ ����� � $k$ �����, �� $|A| \ge n - 2k > 0$.
������������ $W\subseteq A$ ������� \emph{��������� �����}, ���� �������� $i,j \in [n]$
�����, ��� $W=\{a_i, a_{i+1}, \ldots, a_j\}$,
������ $a_{i-1}$ � $a_{j+1}$ �� ������ � $A$.
�����, ��� � �����, ��� ���������� �������� �������� ����������� �� ������~$n$.
����, ��� $A$ ����������� ������������ ������� � �����
\begin{equation}
\label{GapSet}
A=W_1 \cup W_2 \cup \ldots \cup W_s, \mbox{ ��� } 0 < s \le 2k = d,
\end{equation}
$W_i$ --- ��������� ����. 
��������� ����� $l(W)$ ��������� ���� $W$ ��� ����� ����� $v_i$ ($1\le i\le n$), 
������������ ������ ���:
$$
l(W)=|W|-1.
$$
����� ��������� ����� ��������� ��� ����� ����� ������, 
�� ������������� �� ����� �� ���� ��������� �����������, �
$$
\sum^{s}_{i=1} l(W_i)\ge n-2d.
$$



%%%%%%%%%%%%%%%%%%%%%%%%%%%%%%%%%%%%%%%%%%%%%%%%%%%%%%%%%%%%%%%%%%%%
%
%                      ����� 1
%
%%%%%%%%%%%%%%%%%%%%%%%%%%%%%%%%%%%%%%%%%%%%%%%%%%%%%%%%%%%%%%%%%%%

\begin{lemma}
	\label{lemma:RidgeGe}
	\(\dc \ge \left\lfloor n-d - \frac{n-2d}{k+1} \right\rfloor\) ��� $n > 2d$ � $d = 2k$.
\end{lemma}

\begin{proof}
	����������� ��������� ����� �� $k$ �����: \[F_2 = \{\{1,2\},\dots,\{2k-1,2k\}\}.\]
	(����� $v_i$, $i\in[n]$, ��� ��������, �������� �������� $i$.)
	����� ����������� �� ���������� ��������� ���� <<����������>>, ���, ����� ������������ ��� ���� $k+1$ ��������� ��� ���������� ���� �� ����� �� ����� �� �����, ��� �� 1 (��. ���.~\ref{fig:PictDiameter}).
	�� ���� ����� ��������� ��� ����� �������, � ����� ����� ������� � �������� ����� ����� ���� $l_1 \coloneqq \left\lfloor\frac{n-2d}{k+1}\right\rfloor$,
	���� $l_2 \coloneqq \left\lceil\frac{n-2d}{k+1}\right\rceil$.
	%	��� ���� ����� ��������� ��� ����������� ���� ���� ����� $l_1$ � ���� ����� $l_2$.
	���� $n-2d$ �� ������� ������ �� $k+1$, 
	�� ��������� ��� � ������ $l_2$ ����� ����� $m \coloneqq (n-2d) \mod (k+1)$.
	
	%� ������~\cite{Klee:1967} 
	%���� ����� ������� 
	%���� ������ ������ ���� ��� $d=6$ � $n=23$. 
	%�������� ���������� ����� �������� ��� ������������ $d$ � $n$ ���� �� �������.
	%����� �������, ��� ������ ������ ����������� ������� ��������.
	
	
	% ������� 4
	\begin{figure}[tbh]
		\begin{center}
			\begin{tikzpicture}[scale=1.2]
			\BeginPic{2}{18}
			\begin{scope}[xshift=-4.6cm]
			\draw node {$\CP(4,18)$};
			\ArcO{0}{1}
			\Arc{1}{18}
			\Cell{1}
			\Cell{3}
			\Fishka{8}
			\Fishka{13}
			\end{scope}
			
			\BeginPic{2}{19}
			\begin{scope}[xshift=0cm]
			\draw node {$\CP(6,19)$};
			\ArcO{0}{1}
			\Arc{1}{19}
			\Cell{1}
			\Cell{3}
			\Cell{5}
			\Fishka{8}
			\Fishka{12}
			\Fishka{16}
			\end{scope}
			
			\BeginPic{2}{21}
			\begin{scope}[xshift=4.6cm]
			\draw node {$\CP(6,21)$};
			\ArcO{0}{1}
			\Arc{1}{21}
			\Cell{1}
			\Cell{3}
			\Cell{5}
			\Fishka{9}
			\Fishka{13}
			\Fishka{17}
			\end{scope}
			\end{tikzpicture}
		\end{center}
		\caption{������� ������������ ��������������� �����������.}
		\label{fig:PictDiameter}
	\end{figure}
	
	��������, ����� ����� ����������� ��������� ������� (��. ���.~3).
	������ ����� �������� ����� $d + l_1 + 1$ � $d + l_1 + 2$.
	$(k-m)$-� ����� "--- ����� $d + (k-m)(l_1 + 2) - 1$ � $d + (k-m)(l_1 + 2)$.
	$(k+1-m)$-� ����� "--- ����� $n - m (l_2+2) +2$ � $n - m (l_2+2) +1$ (��� $m > 0$).
	$k$-� ����� "--- ����� $n-l_2$ � $n-l_2-1$.
	
	� ���������� ���������� �������� ����� ������ ������ ������.
	� ����������� �� ����, ����� ����� ������ ������ $\{1,2\}$, ���������� ��� ������:
	\begin{enumerate}
		\item �����������, ��� ������ $\{1,2\}$ ������ ������ ����� $\{d + l_1 + 1, d + l_1 + 2\}$. ����� $k$-� ����� $\{n-l_2-1, n-l_2\}$ ������ ������ ������ $\{d-1, d\}$. �������� ��� ��������. ���� ����� �������� ������ ����� ������ ������� ������� ������ ����� �������� �� ������� ������� �� �����, ��� �� $n - d - l_1$. ���� ����� �������� $k$-� ����� �� ������� ������� ������ ����� �������� ������ ������� ������� �� �����, ��� �� $n - d - l_2$. �� ���� � ����� ������ ����� �������� �� ����� ���� ������ $n - d - l_2$.
		\item �����������, ��� ������ $\{1,2\}$ ������ $i$-� �����, $i > 1$. ����� ����� � ������� $i-1$ ������ ������ ������ $\{d-1,d\}$. ���, ��������, ����� ������� ��������� ��������. �������� $i$-� ����� ������ ������� ������� �� ��� ���, ���� ��� �� ������� � ������ $\{1,2\}$, �������� $(i-1)$-� ����� �� ������� ������� �� ���������� � ������� $\{d-1,d\}$. ����� ������ ��������� $n-d-l$ �����, ��� $l$ "--- ����� ��������� ���� ����� $(i-1)$-� � $i$-� �������. 
		
		�������, ��� ������ ������� �������� ��������� �������� ����� �����. �����������, ��� $i$-� ����� �� ������������ ������ ������� ������� �� ����� ����� ��������, ������� �������� � �� ������ $\{1,2\}$. ����� ��� ������ ������� � ������ ������, �������� � ��������������� �����������. ��� ���� $(i-1)$-� �����, ����� ������� � ���� ������, �������� ������� �� ����� $n$ �������� �� ������� �������. ����������, ���� ������������, ��� $(i-1)$-� ����� �� ������� ������ ����� �������� �� ������� �������, ����� ������� � ������ $\{d-1,d\}$, �� ��� ������ ������� � ��� ������, �������� � ��������������� �����������. �� ����� $i$-� ����� ������ ������� �� ����� $n$ �������� ������ ������� �������, ����� ������� � ������ $\{1,2\}$.
	\end{enumerate}
	
	����� �������, � ������ ������ ����������� �� ����� $n-d-l_2$ �����, ����� �������� ��� ����� � ������ (��������, ������ $k$-� ����� �� ������� ������� � ������ $\{d-1,d\}$).
\end{proof}

��� ������ <<����� ���������� ������ $W_i$ � $W_j$>> ����� �������� ��� ������� ����������,
������� ����� ������� ��� �������� ������ ������� ������� �� $W_i$ � $W_j$.
������ ��������������� ������� $f(i,j)$ ������ ����� �����, 
������������� ����� $W_i$ � $W_j$, ����� ����� ����� �� ���� �� �������.
��� ������� �������� ���������� ����������.

\emph{�������� 1.} 
$f(i,j)=-f(j,i)$.

\emph{�������� 2.} 
$f(i,j)=f(i,m)+f(m,j)$.

\emph{�������� 3 (�������������). } 
���� $f(i,j)>1$, �� ����� $W_i$ � $W_j$ �������� $W_m$, ��� $f(i,m)=1$.
� ���������, ���� $f(i,j_1) < 0$ � $f(i, j_2) > 0$, �� ����� $W_{j_1}$ � $W_{j_2}$ �������� $W_m$, ��� $f(i, m) = 0$.

���� ��� ��������� $i$ � $j$ ��������� $f(i,j)=0$, 
�� $W_i$ � $W_j$ ������� \emph{��������}.

\emph{�������� 4 (��������������). }
���� $W_i$ � $W_j$ ������, � $W_j$ � $W_m$ ������,
�� $W_i$ � $W_m$ ���� ������. 


% ������� 3
\begin{figure}[tbh]
	\begin{center}
		\begin{tikzpicture}[scale=1.2]
		\BeginPic{2}{18}
		
		\def\PicThree{
			\Cell{3}
			\Cell{5}
			\Cell{8}
			\Cell{10}
			\Cell{12}
			\Fishka{2}
			\Fishka{5}
			\Fishka{11}
			\Fishka{16}
			\Fishka{18}
		}
		
		\begin{scope}[xshift=-4.6cm]
		\draw node {$S_1$};
		\Arc{1}{1}
		\Arc{2}{4}
		\Arc{5}{6}
		\Arc{8}{18}
		\PicThree	
		\end{scope}
		
		\begin{scope}[xshift=0cm]
		\draw node {$S_2$};
		\Arc{1}{9}
		\Arc{10}{17}
		\Arc{18}{18}
		\PicThree	
		\end{scope}
		
		\begin{scope}[xshift=4.6cm]
		\draw node {$S_3$};
		\Arc{1}{13}
		\Arc{16}{18}
		\PicThree	
		\end{scope}
		\end{tikzpicture}
	\end{center}
	\caption{��� ������� ��� ���.~\ref{fig:PictFishki}: $l(S_1)=1$, $l(S_2)=0$, $l(S_3)=2$.}
	\label{fig:PictCut}
\end{figure}


�������� ��������� ���� ��������� ��� �� ������������ $S_i$, $i \in [t]$, 
%=W_{i_1} \cup W_{i_2} \cup \ldots \cup W_{i_t}$, 
���, ����� ��� ��������� ����, ������������� ������ ������������, ���� ������� ������
�, � �� �� �����, ����� ��� ��������� ���� �� ������ ����������� ���� �� ��������.
������������ $S_i$ ����� �������� \emph{���������} (��. ���.~\ref{fig:PictCut}). 
%����� $t$ ���� ��������, �����������, ������ ����� $s$ ���� ��������� ���.
��� ������ $l(S)$ ������� $S$ ����� �������� ��������� ����� �������� � ���� ��������� ���. 


%%%%%%%%%%%%%%%%%%%%%%%%%%%%%%%%%%%%%%%%%%%%%%%%%%%%%%%%%%%%%%%%%%%%
%
%                      ����� 2
%
%%%%%%%%%%%%%%%%%%%%%%%%%%%%%%%%%%%%%%%%%%%%%%%%%%%%%%%%%%%%%%%%%%%

\begin{lemma}
	%\label{lemma:Ridge1}
	\(l(F_1,F_2)\le n - d - l(S)\) ��� ������ ������� $S$.
\end{lemma}

\begin{proof}
������ �� ���������� ������ ��������� ���� $W \in S$ ������ � ������������� ������ ��� �������. ����� ���������� ���������� �� $m = |S|$ ���. ����� $L$ "--- ���� �� ����� ���. ��� ��� $S$ "--- ������, �� ����� �����, ������������� �� $L$, ��������� � ������ �����. ����� ����, �� $L$ ��� ��������� ���, ������� � ������ �� $S$. ��� ��������, ��� ���� �� ���� �� ���� ������� ����� ���� $L$ ������ ������������ ��������� �����.
(� ��������� ������ �� $L$ ���� �� ��������� ���� �, ����� ����, ����� ��� ������� �� ���� �� ����, ������� � ������ �� $S$.)
������� � ���� ����� � �������� ����� $L$, ����������� ��� ����� ���� ���� ������� �� $1$ �� $p$, ��� $p$ "--- ����� ����� ����� ����. ��������� ����������� ��������� ����� �������� \emph{����������}.
��� �� �������� (�������� � ��� �� ���������� �����������) ����������� ������� �� $L$ ����� �, ��������, ������ ������� �� $1$ �� $r$, ��� $r$ "--- ����� ����� ����� �� $L$.

�������, ��� ��� ������ $t \in [p]$ �� ��������� ����� $[t]$ ���� $L$ ����� �����, ������� ��������, �� ��������� ����� �����, ������� �������. �������������, �����������, ��� ��� �� ���, � ������� ����������� $t$, ��� �������� ����� �����, ������� ��������, ������ ����� �����, ������� �������. ��������, ��� $t > 1$ (��� ��� ����� 1 ������ ������). ����� ����, ����� $t$, � ����� ������, ������ �������, �� �� ������ ������. �������������, �� $[t-1]$ ����������� ����� ����� ����� � ����� ����� �� ����� �����. �� ����� ���� ����� ������� $t-1$ � $t$ �������� ��������� � ������ ������� � ������ $S$.

����� �������, ��� �������� ����� � ������ � ������ ���� $L$ ����� ��������������� ��������� ����������.
��� ������� $i$ �� $1$ �� $r$ ������� $i$-� ����� � ���������� ����������� �� ��� ���, ���� ��� �� ������ $i$-� ������.

������������ ������ ��������� ����������� ���, ��� � �������� ��� ������ ������� <<$\forall t \in [p]$ �� ��������� $[t]$ ����� �����, ������� ��������, �� ��������� ����� �����, ������� �������>> �� ����������.

��� ������ ������ ����� ��������, ����������� ��������� ����������, ������ � ������������ ���������� $u$, ������ ����� ����� �� $L$, ������� � ������ (�� �������) �����, ���������� $i$-� ������ �� $i$-� ����� ������ ���������, � ���������� ��������� ������ ������� $L$. � ������ ������ ��������� $u = p$. � ����� $r$-�� ����� $u=2$ (��������� ����� ������� � ��������� ������, ���������� ����� $p-1$ � $p$). ������ �������� � ���������� ����������� ��������� $u$ �� 1. ��� ���������� $i$-�� ����� � �������� � $(i+1)$-�� $u$ ����������� ��� ������� �� ��� �����, ���������� $i$-� ������.
�������������, ����� ����� �������� �� ��������� �������� $p - 2r$. 

����� �������, ��� ���� ��� ����� �������� �� ��������� $n - l(S) - d$.
\end{proof}


��������������� �� ������ ��� ���������� ����� �������
$$
l(F_1,F_2)\le n-d - \max_{1\le i\le t} l(S_i),
$$
��� $t$ --- ����� ���� ��������.
�, ��� ��� 
$$
\sum^{t}_{j=1} l(S_j)=\sum^{s}_{i=1} l(W_i)\ge n-2d,
$$
��
$$
l(F_1,F_2)\le n-d - \frac{n-2d}{t}. 
$$

�������� ��������, ��� 
\begin{equation}
\label{eq:CutRidge}
t\le k+1,
\end{equation}
��� $k=\frac d2$.
�� ������� \eqref{GapSet} ��������, ��� ����� ���� ��������� ��� $s\le d$.
�������, ����� ���������� �������������� ����������� \eqref{eq:CutRidge}, ���������� ��������, 
��� ����� ���� �������� �������� �� ����� ����, ���������� ����� �� ����� ��������� ����.

����� ������ $S'$ �������� ���� ������������ ��������� ���� $W_{i'}$. 
��� ��������, ��� ��� ������ $j\ne i'$, ���������
\begin{equation}
\label{NotEq}
f(i',j)\ne 0.
\end{equation}
� �����, � ���� �������� 3 (�������������), ���� �������� $f(i',j)$ �������� ��� ���� $j\ne i'$.

���������� ��� ���� ������ $S''$, ���������� ���� ���� ��������� ���� $W_{i''}$, $i''\ne i'$.
�������, ��� ��� ����� $j\ne i'$ � $m\ne i''$ ���������
$$
f(i',j) f(i'',m) < 0.
$$
�������� �� ����������, �����������, ��� ��� ��������� $j$ � $m$
$$
f(i',j) f(i'',m) > 0.
$$
(��������� ����� ����������� � ���� ������� \eqref{NotEq} 
� ������������ ����������� ���~$i''$). 
�� ����� ��� ����������� ����������� � ��� $j=i''$ � $m=i'$,
��� ���������� � ���� �������� 1. 
����, ����� ���� �������� �������� �� ����� ����, ���������� ����� �� ����� ��������� ����,
� �����������~\eqref{eq:CutRidge}, � ������ � ��� � �����������~\eqref{eq:RidgeIneq}, ��������. 



%%%%%%%%%%%%%%%%%%%%%%%%%%%%%%%%%%%%%%%%%%%%%%%%%%%%%%%%%%%%%%%%%%%%
%
%                      �������� �����������
%
%%%%%%%%%%%%%%%%%%%%%%%%%%%%%%%%%%%%%%%%%%%%%%%%%%%%%%%%%%%%%%%%%%%


\subsubsection{�������� �����������}

����� \(d=2k+1\).

�� ������� �������� ����� �������, ��� � ������ �������� ����������� 
������ ���������� ������ ��������� ���� �� ���� �� ���� �����: $1$ ��� $n$. ��������� �����, ������������� ����������, ���������� ����������� �� ���� ���� $\{i,i+1\}$.  ����� �������, �� �������� � ������ ������������, ����� �� ����������, ��������������� ����� $F_1$, ����������� �� $k$ ������� ����� � ���� �����, ��������� �� ����� ����� ($1$ ��� $n$). �����, ���������� ����� $F_2$, ����������� �� $k$ ������� ����� � ���� ������, ���������� ���� ����� ($1$ ��� $n$).
������ �����, ��������� �� ����� �����, ����� ������������ ������ �� ������� $1$ � $n$ � �������.
������� ����������� ����� � ����� �������� ���� ��, ��� � � ������ ������ �����������, �� ����������� ��������, ����� �� ���� ������� ����� ����������� ������, ��������� �� ����� �����. � ������, ����� ������ ����� �������� ������� $1$. �����:
\begin{enumerate}
	\item ���� ����� $\{2,3\}$ ��������� �� ������� �������, �� ��� ������������ � ������� $\{1,2\}$, � ������ ����� ��������� � ����� $n$.
	\item ���� ����� $\{n-1,n\}$ ��������� ������ ������� �������, �� ��� �������� ������� $\{1,2\}$, � ������ ����� ��������� � ����� $n$.
\end{enumerate}
���������� ������������ ����������� ������� �����, ���������� ������ ����� � ������� $n$. ����� �������, ����� �������� (�� ���� ���) ������� �����, ���������� ������ �����, ���������� �� �� ������� $1$ � $n$ ���, ��������, �� $n$ � $1$.

�������������� ����������� 
\[\dc \ge \left\lfloor n-d - \frac{n-2d}{k+1} \right\rfloor \]
��� $d=2k+1$ � ����� ���������� �������������� �����~\ref{lemma:RidgeGe}.
����������� ��������� ����� �����: \[F_2 = \{\{1\},\{2,3\},\dots,\{2k,2k+1\}\}.\]
������ ����� �������� � ������� $n$. ��������� $k$ ����� ����������� �� ���������� ��������� ���� <<����������>>. ����� �������� ���� ��������� ����������� �������������� �����~\ref{lemma:RidgeGe}.

��� �������������� ����������� 
\[\dc \le n-d - \frac{n-2d}{k+1}\]
������������� ������������ ������������� ��� ������ �����������. ����� ����� ������ � ������������ ������� ��������� ��� � ��������. 

�������� ��� ������:
\begin{enumerate}
	\item ������ ����� � ������ ������ ����������� � ������ ������. ����� ��������� ����� ���� �������� ���������� ����� ��������� $n-2d+1$, ����� ����� ��������� ��� �� ����������� $2k+1$, � ����� ����� �������� �� ����������� $k+1$. �������������,
	\[
	l(F_1,F_2)\le n-d - \frac{n-2d+1}{k+1}.
	\]
	\item ������ ����� � ������ ������ ����������� � ������ ��������. ����� ��������� ����� ���� �������� ���������� ����� $n-2d$. ���� ������ ������� �� ������������ ���� ����� ������� $n$ � $1$. ����� ����� ��������� �������� �� ����������� $k+1$.
	�������������,
	\[
	l(F_1,F_2)\le n-d - \frac{n-2d}{k+1}.
	\]
\end{enumerate}

%%%%%%%%%%%%%%%%%%%%%%%%%%%%%%%%%%%%%%%%%%%%%%%%%%%%%%%
%
% End of section
%
%%%%%%%%%%%%%%%%%%%%%%%%%%%%%%%%%%%%%%%%%%%%%%%%%%%%%%%

%% Глава 7
%%%%%%%%%%%%%%%%%%%%%%%%%%%%%%%%%%%%%%%%%%%%%%%%%%%%%%%%%%
%
%     ��������� ������� ����
%
%%%%%%%%%%%%%%%%%%%%%%%%%%%%%%%%%%%%%%%%%%%%%%%%%%%%%%%%%%

%\texorpdfstring{�������� ����� �������������� �����\\ � ��������� ������� ����}{�������� ����� �������������� ����� � ��������� ������� ����}
\chapter{��������� ������� ����}
\label{chap:Direct}

%\hfill
%\begin{minipage}{0.55\textwidth}
%��������� ����� ������������� ������ ������ ������ ������� �������������� ��������� � ������� ������ ����������, ���������� �� �������� ����������.
%~\hfill
%\begin{flushright}
%\emph{�.\,�.~����������}
%\end{flushright}
%\end{minipage}

\section{������ ���������� ������� ����}

� ���������� \cite{BondBook:1995} �������� ����� (���������� �~��������� ����������) ����� ������������� ��������������� ��� ������ ������ ��������� ��������������� ������
� <<��������� �������>> ������ ��� ���������� ���������� ������� ����. 
��������� ������� ���� ��������� � ������ �������� ����������� ����������,
������� ������ ������������ � ���� �������� ����������� ��������.

� ���� �����, ������ \cite{BondBook:1995}, ������� ����� ���������� ��������� ���������� ������� $X = X(I) \in \R^m$, �� �������� ��������� ������� ������� $\bm{x}$, ����������� ������������ ����������� �� ���� ������ �������� ������� $\bm{c} \in \R^m$.
������� �������� �������� �� ��, ���, � ������� �� ������������ � ���������� ������ �����������~\ref{def:LCOP} �~\ref{def:family}, ������� � ������ ����������� ������ ���������� �� ��������� �������� ���������� ������� ������, � ���� ���� �� ���� ��������.

��� ��, ��� ��� ���� ������� � �������~\ref{sec:Cones}, ��� ������� $\bm{x} \in X$ ��������� ����� �������� ������
\begin{equation}
\label{eq:cone}
K(\bm{x}) = \Set*{\bm{c}\in \R^m \given  \bm{c}^T \bm{x} \ge \bm{c}^T \bm{y}, \ \forall \bm{y} \in X}.
\end{equation}
� ������, ���� ��������� �������� ������ ������ ���������� ��������� $Q \subseteq \R^m$, ���������� ����������� $K(\bm{x}, Q) = K(\bm{x}) \cap Q$.

\begin{definition}[{{\cite[�. 33]{BondBook:1995}}}] %{\cite[�. 33]{BondBook:1995}}
	��������  ����������� ������� ������ $X \subset \R^m$ ���������� ��������������� ������, ���������� ���������� ����������:
	\begin{itemize}
		\item[�)] 
		� ������ ����, �� ����������� ������, �����������  ������,
		������ ����� ���� ����; ���, �������� � ������, ���;
		\item[�)] 
		��� ������� ���� ���� ������� ��� ��������� �� ����  ����,
		���� ����� ���  ���  ������;   �  ������  ������  ����  ����������
		����������, �� ������ "--- �������, ��� ������;
		\item[�)] 
		������� ����������� ���� ������������� ��������� ������ $\bm{f} \in \R^m$;
		\item[�)] 
		������� ����� ������������� ��������� ������� �� $X$, ������ ���������� ������� ����� ��������������� ���� � ��� �� ������� ��������� $X$;
		\item[�)] 
		������ ���� $d$ ������������� �����  $\sgn d$, ������ $1$ ���� $-1$;
		��� ����, ��������� �� ������ ����, ����� ��������� ��������;
		\item[�)] 
		��� ������ ���� $W = \bm{f_1} d_1 \bm{f_2} d_2 \ldots \bm{f_l} d_l \bm{x}$, ����������� ������ � ���� (� �����������  ����  �����������  ���������������  ��  ����� �������; ���� $d_i$ ������� �� ���� $\bm{f_i}$, $i\in[l]$), � ��� ������ $\bm{c}\in \R^m$ �� ���������� $\bm{c}^T \bm{f_i} \sgn d_i\ge 0$, $i\in[l]$, ������� ��������� $\bm{c}\in K(\bm{x})$.
	\end{itemize}
\end{definition}

\emph{���������� $C_{LSA}(X)$ ������ $X$ � ������ �������� ����������� ����������}
���������� ������ (�������) ������������ ��������� ������������ ������ ���� ������.
��������, $C_{LSA}(X) \ge \log_2 |X|$ (����������, ��� �������, 
��� $X$ �� �������� �������, �� ���������� ������������ �� ��� ����� ����� $\bm{c}$).
�~1982~�. �.\,�.~������ �������~\cite{Moshkov:1982}, ��� $C_{LSA}(X) = O (m^3 \log_2 |X|)$ ��� ����� ������ $X\subset\R^m$. �� ���� NP-������� ������ � ������ ���� ��������� �������������� � ������������� ����������.
����� ��������� ��� ���������� ������������, 
��� ������������ ��������� ������������� ��������� �������������� ������������,
����������� ��������� �� �������������.
���� �� ����� ����������� ���� ���������� �.\,�.~���������� �~1980"~�~��.

� \cite{BondBook:1995} ���������� ��� ��������� ������ �������� �����������
��������� ������� ����, � ������ ������� ����� ���� � �� �� ����,
���������� �� ������� ����� ������� ������.

��� � ������, �� �������������� ������������� ��� �������, ����� ��������� �������� ������ (���������� ������� ��������) ������ $X \subset \R^m$ ��������� � $\R^m$ ��� �� ������������ ����� ������� $Q \subseteq \R^m$.

\begin{definition}[\cite{MaksimenkoDiss:2004}]
\label{def:AdjSolutions}
��������� \emph{����� �������} ������ $X \subset \R^m$ � ������������ $Q \subseteq \R^m$ �������� �� ������� $\bm{x} \in X$, ������� ������������� ������� 
\begin{equation*}
	\exists \bm{c} \in Q \quad \forall \bm{z}\in X \setminus \{\bm{x}\} \quad \bm{c}^T \bm{x} > \bm{c}^T \bm{z}.
\end{equation*}
��� ������� $\bm{x}$ � $\bm{y}$ ����� ������� \emph{������}, ���� �������� $\bm{c} \in Q$ �����, ���
\begin{equation*}
%\label{eq:adjacent}
	%\exists \bm{c} \in Q \quad 
	\forall \bm{z}\in X \setminus \{\bm{x}, \bm{y}\} \quad \bm{c}^T \bm{x} = \bm{c}^T \bm{y} > \bm{c}^T \bm{z}.
\end{equation*}
\end{definition}

�������, ��� ����������� ������� ������� ������������ ����������� ������� ������� (�� �.~\pageref{AdjCones}) �, ���� ��������� �������� ������ ��������� � $\R^m$, ����������� ������� ������ ������������� $\conv(X)$.

������������ ������� $Y \subseteq X$ �������� \emph{������}, ���� ��������������� ������� ����� ������� ������� ������.
�������� ����� ����� ������� ���������� $\omega(X,Q)$, ���� �� ����������� $Q$ ����������� (�� ���� $Q = \R^m$), ���������� �������� $\omega(X)$.

������, ������ \cite{BondBook:1995}, ��������� ��������������� �����������.
����� $T$ "--- ��������  �����������  ������  ������ $X$  �
����� $\bm{f}$ "---  ���  ����������  ����.
��������� ����� $X_f$, $X_f \subseteq X$,  ��������� ������� ���� ������� ������ $T$,
�������  ������������  ���� $\bm{f}$, � ����� $X_f^+$ � $X_f^-$  ���������
������������ $X_f$, ���������������  ���� ��������� �� $\bm{f}$ �����.
��������� ����� $R_f^- = X_f^+ \setminus X_f^-$ ��������� �������,
������������� ��� �������� �� <<�������������>> ����.
�� �������� ��������� ��������� ������� $R_f^+ = X_f^- \setminus X_f^+$,
������������� ��� �������� �� <<�������������>> ����.

\begin{definition}[{{\cite[�.~39]{BondBook:1995}}}] %[{\cite[�. 39]{BondBook:1995}}]
	\label{def:direct-type}
	�������� �����������  ������ $T$ ������ $X$
	���������� ������� ������� ����, ���� ��� ������ ����������� ���� $\bm{f}$ �
	��� ����� ����� $Y \subseteq X$ ����������� �����������
	\begin{equation}
	\min \{ |R_f^+ \cap Y|, |R_f^- \cap Y| \} \le 1.
	\label{eq:direct-type}
	\end{equation}
\end{definition}


\begin{definition}[{{\cite[�. 40]{BondBook:1995}}}] %[{\cite[�. 40]{BondBook:1995}}]
	\label{def:direct-type2}
	������� <<�������  ����>> ������ $X$, $X\in R^m$,
	���������� �������� �����������  ������  ����  ������,  ���  ��������
	������ ���� $w = \bm{f_1} d_1 \bm{f_2} d_2 \ldots \bm{f_l} d_l \bm{x}$,  �����������  ������  �  ����,
	������������� ��������:
	\begin{itemize}
		\item[(*)] 
		��� ������ $\bm{y}\in X$ ��������  � $\bm{x}$,  ��������  �����  ����� $i\in [l]$, ��� ������� $\bm{c}^T \bm{f_i} \sgn d_i > 0$ � $\bm{c}\in K(\bm{y})$ �����������;
		\item[(**)] 
		��� ������ $i\in[l]$ �� �������������� �������
		\[
		\bm{c}^T \bm{f_i} \sgn d_i > 0 \qquad \mbox{�} \qquad \bm{c}\in K(\bm{y})
		\]
		��� $\bm{y}$, �������� � $\bm{x}$, � �� ���������� ������
		\[
		K(\bm{x}) \cap \Set*{ \bm{c}\in R^m \given \bm{c}^T \bm{f_i} \sgn d_i \le 0}
		\]
		�������, ��� �����, ������������ � ���� $\bm{f_i}$ �  �����  $-d_i$,  ����� ���� �� ���� ����, ���������� $\bm{x}$.
	\end{itemize}
\end{definition}

��� ���� ����������� ���������� ��������� ����.

\begin{theorem}[{\cite{BondBook:1995}}] %[{\cite[������ 2.4]{BondBook:1995}}]
\label{the:direct-type}
������ ������ ������� ���� (<<������� ����>>) ������ $X$ �� ������ $\omega(X)-1$.
\end{theorem}

����� �������, �������� ������������ ����� �������� ������ ������� ������������
��� ������ ���� ����������.

��������~(��. ����� � �������~\ref{sec:CliqueNumber}), ��� ��� ������������ ������������� ���������� ����� (����������, ����������� �������� ������, ����������� ������) ��� �������������� �� ����������� ����������� �������������.
����, � �������~\ref{sec:ShortPathClique}, ����� ��������, ��� ������ � ���������� ���� (� ������������ ����������������� ���� ��������) ���� ������ � ���� ������.

� ������ �������, � ������ \ref{chap:AffTheory} � \ref{chap:AffExamples} ��������, ��� ������ ������������ ������������� $\BQP$ ������� �������� � �������������� ����� �������������� �����, ��� �����������, ������, 3-������������, 3-���������, �������� � �������� ���������, ��������� �����, ���������� ������� � ������ ������. ��������, ��� �������� ����� ����� ������������� $\BQP(n)$ ����� $2^n$, �������� ����� ������ �������������� ��������� ����� ����� ������������������ �� ����������� ��������������.
����� ����, �~\cite{BondBook:1995} �����������, ��� ��������� ��������� ����������, ������ �������� ��� ������������ ��������� ������,
�������� �������� ��� ����������� ����, �������� �����--����� 
� ���������� ��������� ������ � ������ ��� ������ ������������
�������� ������� ��� <<�������>>.
(����, � ��������� ���������� ���� ���������, 
��������� �� ���� ����������� ����� ���� ����������� ��������.)

����, �~�������~\ref{sec:ShortPathClique}, ����� ��������, ��� �������� ����� ��� ������ � ���������� ���� � ������� �� $n$ �������� � ������������ ����������������� ���� �������� (� ����� ��� ������ � ������������ ������������ ����������������� ���� ���) ����� $\lfloor n^2 / 4\rfloor$.
� ������ ���������~\ref{cor:Short2Assign}, ��� ���� ������ ������ $\lfloor (n+1)^2 / 4\rfloor$ ��� ��������� ����� ����� ������������� ������ � ����������� $\Birk(n)$.
� ���� ����� ������� ��������� �����.
�~1977~�. �������� � ������ �������� (��.~\cite[Theorem~6.1, Corollary~6.5]{Brualdi:1977II} �~\cite[Corollary~3.7]{Brualdi:1977I}), ��� ����� 2-����������� ����� ������������� $\Birk(n)$, ����� ������ ������� �� ����� �����, �������� ����������, � ������������ ����� ������ ����� ����� ��������� � ���������� ���� ������� $\lfloor (n+1)^2 / 4\rfloor$.
�~\cite{BondBook:1995} ���������� ���������� ������� ������ ��� $\omega(\Birk(n))$, �� �������������� ��������� �� �������� ����������� (��. ���������~\ref{cor:Edmonds} ����).
�������������� ����������� ���� ��������� ������:
$\omega(\Birk(4)) = 6$, $\omega(\Birk(5)) = 13$, $\omega(\Birk(6)) = 18$, $\omega(\Birk(7)) = 23$, $\omega(\Birk(8)) = 29$, $\omega(\Birk(9)) \ge 39$.

� �������~\ref{sec:NondirectAlg} ������������� ��� ������, ��������������� �������������� ������������ ����� ������� � ������ ��������� �����.
���������� �������������� ����, ��� �������� ����--�������� ��� ������ � ����������� �� �������� ���������� ������� ����.
����� ����, ����������� ���������� ������������� ������ ����������� ����������,
����������� �� �������� �� ������������, �� �������������� ��������� �� �� ������ ���������� ������� ����.


%%%%%%%%%%%%%%%%%%%%%%%%%%%%%%%%%%%%%%%%%%%%%%%%%%%%%%%
%
%  ������ � ���������� ����
%
%%%%%%%%%%%%%%%%%%%%%%%%%%%%%%%%%%%%%%%%%%%%%%%%%%%%%%%

\section{�������� ����� ��� ������ � ���������� ������}
\label{sec:ShortPathClique}

����� $D = (V,A)$ "--- ������ ������, � ������� �������� ��� ������� $s,t\in V=\{v_1,v_2,\ldots,v_n\}$. ��� �������������� ����� �������, ��� $s=v_1$ � $t=v_n$. ������ ���� $a_{ij}\in A$ ($i\ne j$, $i\in[n-1]$, $j\in\{2,3,\ldots,n\}$) � ���� ����� ��������� ����� $c_{ij}$. 
������ � ���������� ������ ������� � ��������� � ������� $D$ ����, �������� ������ � ������� $s$ � ���������������� � $t$, ��������� ��� ��� �������� ��� �� �����������.

� �������~\ref{sec:ShortPath2Assignment} ��� ���� ������� � ���, ��� ������ ������������� $s$-$t$ ������� $\Dipath(n)$ ������� ������������ ������������� �������� $\ATSP(n-1)$. ������ �������, � ���������, ��� ��� ������ � ����� ������ NP-������, � �������� ����� ����� ������������� $\Dipath(n)$ ������������������.
����~�� �~�������~$D$ ����������� ������� ������������� �����, �� ��������������� ������� $\ShortP(n)$ ����� ���������� ��������, � ������ ���������� ������������� ����������. 

���� ����� ��������, ��� �������� ����� ����� ��������� ��������� ������ � ���������� ������ � ������������ ����������������� �������� �, ������ � ���, �������� ����� ������ ��������� $\ShortP(n)$, $\Dipath^{\uparrow}(n)$ � $\Path^{\uparrow}(n)$, ����� $\lfloor n^2/4\rfloor$.
��� ����� ��� ����������� �������� ��������� �������, ���������������� ����� � �����~\ref{lem:path}. 

%\begin{lemma}[\cite{MaksimenkoDiss:2004}]
%\label{thm:ShortPathAdj}
%���������� ������� $\bm{x}$ � $\bm{y}$ ������ $\Dipath(n)$ � ������������ ����������������� ���� �������� $S$ ������ ����� � ������ �����, ����� ��� ��������������� ����� $\tx$ � $\ty$ �������������� �������� $\tx \symdiff \ty$ �������� ������� ����.
%\end{lemma}
%\renewcommand{\proofname}{�������������� �����~\ref{lem:path}}
\begin{proof}[�����~\ref{lem:path}]
� ����� ���������� �����������, ����� ������������� ������ �� ��������, �� ���� ������ � ���������� ������ � ������������ ����������������� ���� ��������.
��������� ������� �������� (�� ���� �������� ���� ���), ��������������� ����� �����������, ����� ���������� $S$.

��� ������� ������� $\bm{x} \in \Dipath(n)$ ��������������� ��� ������ (������, ��������� ��� ���) � ������� $D$ ����� ���������� $\tx$.

����� ��������, ��� �������������� �������� $\tx \symdiff \ty$ �������� \emph{������� �����}, ���� ��� ������������ ����� ����������������� ����, ��������� �� ���� �������, ������� ����� ������ � ����� �����, � �� ������� ������ ����� ������.
%���� �������, ������� ����� ������, ��������������� � ������ ����� ������� � �� ������� ����� ������������� ������, ����� �������� \emph{������� �����}.

������� ���, ���� $\tx \symdiff \ty$ �� �������� ������� �����, �� ��������������� ������� $\bm{x}$ � $\bm{y}$ ��������.
�������� �����������~\ref{def:AdjSolutions} ������� �������, ��� �����
���������� ��������� � ���, ��� ��� ������� ������� ���� ��� $\bm{c} \in S$, 
���������������� ��������� $\bm{c}^T \bm{x} = \bm{c}^T \bm{y}$, �������� ����� 
������ $\bm{z_c}\in \Dipath(n)$, �������� �� $\bm{x}$ � �� $\bm{y}$, 
��� ��������� $\bm{c}^T \bm{z_c} \ge \bm{c}^T \bm{x}$.

�������� �� ����� $\tx$ � $\ty$ �� ��������� ������� $s$, ������ �� ������ ������� $b$, �������� � $\tx \symdiff \ty$ (������, ����������� ����� �� ��� ����� ���������). ��������, ��� ������� ���� ����� �� $s$ �� $b$ ���������. 
������ �� ������� $b$ ����� ��������� �� ���� $\tx$ �� ����������� � ����� $\ty$; ��������� ��� ������� $p$.
����� �� �������� ��������� �� ���� $\ty$. ����� ��������� �� ����, ������� � $b$, ���� �� ����������� � $\tx$; ������� ����������� ��������� $q$. 
���� $\tx \symdiff \ty$ "--- �� ������� ����, �����, ��������, ����� ���� 
$\tx$ �� $p$ �� �������� ������� $t$ �� ��������� � ������ ���� 
$\ty$ �� $q$ �� $t$ (� ���������, ����� �������� ���������,
����� $p\ne q$).

�������� ��� ����� ���� $\tilde{x_1}$ � $\tilde{y_1}$ ���������
�������. 
����� ���� $\tx_1$ ����� �������� �� ����� ���� $\ty$ �� ������� $s$ �� $q$
� ����� ���� $\tx$ �� $q$ �� $t$.
���� $\ty_1$ ����� �������� �� ����� ���� $\tx$ �� ������� $s$ �� $p$
� ����� ���� $\ty$ �� $p$ �� $t$. ��������, ��� ��� ����� ������� ���� ��� ���������������, �������� �� $\tx$ � $\ty$. ����� ���������� ��� ������.

������ � ��������. �����������, ��� ������� $p$ � $q$ 
���������. ����� ����� ���, �� ������� ������� ���� $\tx$ � $\ty$,
��������� � ������� ���, ���������� ���� $\tx_1$ � $\ty_1$.
�������������,
\[
\bm{c}^T \bm{x} + \bm{c}^T \bm{y} = \bm{c}^T \bm{x_1} + \bm{c}^T \bm{y_1},
\]
��� $\bm{x_1}$ � $\bm{y_1}$ "--- ������������������ ������� ����� $\tx_1$ � $\ty_1$.
��������, ��� � ����� ������ ��� ������� $\bm{c}\in S$, ����������������
��������� $\bm{c}^T \bm{x} = \bm{c}^T \bm{y}$, ������ ����� ����� ��������� ����� 
$\bm{z_c} \in \{\bm{x_1}, \bm{y_1}\}$, ��� �������� $\bm{c}^T \bm{z_c} \ge \bm{c}^T \bm{x} + \bm{c}^T \bm{y}$. 
� ������� $\bm{x}$ � $\bm{y}$ ��������.

������ ���������� ������, ����� ������� $p$ � $q$ �� ���������. ����� ����� ���� $\tx$ �� ������� $p$ �� ������� $q$ � ����� ���� $\ty$ �� $q$ �� $p$ 
�������� ��������� ��������� ��� $\tilde g$, �� �������� � ���� $\tx_1$ � $\ty_1$. �� ���� ��� ������ $\bm{c} \in \R^m$ ���������
\begin{equation*}
%\label{ShoEq}
\bm{c}^T \bm{x} + \bm{c}^T \bm{y} = \bm{c}^T \bm{g} + \bm{c}^T \bm{x_1} + \bm{c}^T \bm{y_1},
\end{equation*}
��� $\bm{g}$ "--- ������������������ ������ ��������� $\tilde g$.
��������, ��� $\tilde g$ ������������ ����� ������� ������ 
(�������� ������� ���������������), ��� ������ ������� �������� ����� �������� ��� ����� ����� ���������.

�������, ��� ������� ������ ������ ����� ����������� ��� �����������
������� ��������. ����� ��������� �� ���� $\tx$ �� ������� $p$ �� ��� ���,
���� �� ������� � ������� $r$, ����������� ����� �� ��� ��������� $\tilde g \cap \ty$ (����� �������� ����� ��������� ������� $q$).
�������� ����� ������ $\tilde g_1$ �� ������� ���� $\tx$ 
�� ������� $p$ �� $r$ � �� ������� ���� $\ty$ �� $r$ �� $p$; 
��������, ��� ��� ����� ������� ������. 
������ �� $\tilde g$ ��� ����, �������� � $\tilde g_1$, 
������� ����� ������ $\tilde g'$, ��������� �� ����� �������� ���� 
$\tx$, ������� �� ������� $r$ � $q$, � ����� �������� ���� $\ty$,
������� �� $q$ � $r$. ���� ������ $\tilde g'$ �������� �������, 
�� ����� ��������� ��������� ��������� �������� �������, ���� ��� ���� �� ����� ���������. 

����, ������� ������ $\tilde g$ ������ ����� ����������� ��� ����������� ������� �������� $\tilde g_k$, $k \in[l]$:
\[
\tilde g=\bigcup_{k=1}^l \tilde g_k.
\]
��������, ��� � ������� $D$ ����������� ������� ������������� �����, ��������
\[
\bm{c}^T \bm{g} \le 0.
\]
�������������,
\[
\bm{c}^T \bm{x} + \bm{c}^T \bm{y} \le \bm{c}^T \bm{x_1} + \bm{c}^T \bm{y_1}.
\]
����� �������, ��� � � ������ $p = q$, ������� $\bm{x}$ � $\bm{y}$ �������� ����������.

��� �������������� ��������� ����������� �����������, ��� ������� $\bm{x}$ � $\bm{y}$ ��������. �������, ��� � ����� ������ ��������� 
$\tx \symdiff \ty$ "--- �� �������� ������� �����.
��������� ���������� ������� $\bm{c} \in \R^m$ ��������
\[
c_i=\left\{
\begin{array}{ll}
0,      &\mbox{����} \quad x_i=1, \quad \mbox{���} \quad y_i=1\\
-1,      &\mbox{����} \quad x_i=y_i=0.\\
\end{array}
\right.
\]
����� $\bm{c}\in S$ � $\bm{c}^T \bm{x} = \bm{c}^T \bm{y} = 0$.
� �� ������� ����������� �������, ��� ���������� $\bm{z} \in \Dipath(n) \setminus \{\bm{x},\bm{y}\}$, ��� �������� $\bm{c}^T \bm{z} \ge 0$.
�� ��� ��������, ������ ���� $\bm{z} \le \bm{x} + \bm{y}$, �� ���� �� ��� ����� $\tx$ � $\ty$ ����� ��������� ����� ���� $\tilde z$. �������������, $\tx \symdiff \ty$ "--- �� ������� ����.

�������� ��������������, �������, ��� ������ ����������� $S$ ������������� �������� $\R^d_- = \Set{\bm{c} \in \R^d \given \bm{c} \le \bm{0}}$ ���� �������� ��������������, �������� �� ������������� ���������� �������� ������� �� ����������� �������.
\end{proof}

�������� ������������ �������� ��������� ������ ��� �������� $\Path^{\uparrow}(n)$. ����� ��������, ��� ������� $\bm{x}$ �������� $\Path^{\uparrow}(n)$ ������������� ������� $\bm{x^*}$ �������� $\Dipath^{\uparrow}(n)$,
���� ��������������� $s$-$t$ ���� $\tilde{x}$ � ������ ����� $G=(V,E)$ ���������� �� $s$-$t$ ������ $\tilde{x}^*$ � ������ ������� $D=(V,A)$ �� ���� ������ ��� �� �����.
�������, ��� ��� ������������ ������ �������� ����������� $\Dipath^{\uparrow}(n)$ �� $\Path^{\uparrow}(n)$ (������ �� ���� �������������� ������������ ��� $(u,v)$ � $(v,u)$ ������� $D$ �������� � ������������ ����� $\{u,v\}$ ����� $G$).

\begin{lemma} 
\label{lem:path2}
������� $\bm{x}$ � $\bm{y}$ �������� $\Path^{\uparrow}(n)$ ������ ����� � ������ �����,
����� ������ ��������������� �� ������� $\bm{x^*}$ � $\bm{y^*}$ �������� $\Dipath^{\uparrow}(n)$.
\end{lemma}
\begin{proof}
����� ������� $\bm{x^*}$ � $\bm{y^*}$ �������� $\Dipath^{\uparrow}(n)$ ������. �����, �������� ������ ��� ���������� �����~\ref{lem:path}, �������������� �������� $\tilde{x}^* \symdiff \tilde{y}^*$ �������� ������� ����. �� ����� �������, ��� �� ����� ��������������� (�����������������) ����� $\tx$ � $\ty$ ������ ���������� ��� ���� ���� $\tz$. �������� ����������� �������������� ������ �������������� �����~\ref{lem:path}, �������� � ������, ��� ��������������� ������� $\bm{x}$ � $\bm{y}$ �������� $\Path^{\uparrow}(n)$ ������.

�������� ��������������� ���, ��� $\Path^{\uparrow}(n)$ �������� �������� ������� �������� $\Dipath^{\uparrow}(n)$. �� ����� �������, ��� ��������� ������ $\bm{x}$ � $\bm{y}$ �������� $\Path^{\uparrow}(n)$ ������ ��������� ��������������� ������ $\bm{x^*}$ � $\bm{y^*}$ �������� $\Dipath^{\uparrow}(n)$.
\end{proof}

� ������~\cite{Bondarenko:1993SW3A} ���������� ��������� �����������, ������� � ���������� ���� ������~\ref{lem:path}, �� ��� ������������� $s$-$t$ ����� $\Path(n)$. 
�������� ��� ��������, �� ��������� ����������� ��������� ������:
\begin{quote} %������
	\textit{������� 1.}
	����� $x$ � $y$ �������� �������� ��������� ������������� $\Path(n)$ � ��� � ������ ��� ������, ����� �������������� �������� 
	$$
	\tx \symdiff \ty = 
	(\tx \setminus \ty) \cup (\ty \setminus \tx)
	$$
	�������� ���� ������� ����.
\end{quote}
%�������� ��������, ��� ��� ����������� � �����~\ref{lem:path} ����������� ���� � ��� �� ���������, �� ��� ������ ��������.
�����, � ��� �� ������ �� ������ ������������ �������� ��������� ������ ������������� $\Path(n)$ ������������ �������� ����� $p_n$ ��� �����:
\begin{quote} %������
	\textit{������� 2.} 
	����������� ���������
		$$
		p_n=\left\{
		\begin{array}{ll}
		k^2     &\mbox{���} \quad n=2k, \\
		k^2+k   &\mbox{���} \quad n=2k+1.
		\end{array}
		\right.
		%		\text{>>}
		$$
\end{quote}
�� ���� ������������, ��� �������� ����� ����� ������������� $s$-$t$ ����� $\Path(n)$ �������������. � ������ �������, �� ����������� �������~\ref{sec:TSPvarious} �������, ��� $\ATSP(n) =_A \HDPst(n+1) \lea \HPst(2n) \lea \Path(2n)$. �� ���� �������� ����� ����� ������������� $\Path(n)$ ������������������, ��� ������� � ������� ������ � �������~2 ��~\cite{Bondarenko:1993SW3A}. ����� ����, ������ �������� ����������� ������ ��� �������������� $\ATSP$ NP-�����. �������������, ��� NP-����� � ��� $\Path$. ��� �������� ��������� �������������� ����������� ������� 1 (��� ������� $\text{P} \ne \text{NP}$). ����� ������� ������ ��������������� ����.

%, ��� ������������ ������� 1 ��~\cite{BondBook:1993SW3A}.

��� ����, ����� �������� ������� �������� ���������, �������������� ������~\ref{lem:path2}, �� ����������������� � �������~1 �~\cite{Bondarenko:1993SW3A},
�� ���.~\ref{fig:contrpath2} ���������� ������ ���� ����� $\tx$ � $\ty$, ��� ������� �������������� �������� $\tx \symdiff \ty$ �������� ������� ������, �� ��������������� ������� $x$ �~$y$ �������� $\Path^{\uparrow}(4)$ (� ������������� $\Path(4)$) ��������.
����� �������, �������� ���������, ����������� � �������~1 %�~\cite{Bondarenko:1993SW3A} 
�� �������� �� ��� ������������� $\Path(n)$, �� ��� �������� $\Path^{\uparrow}(n)$.

\begin{figure}[h]%
	\centering
	\begin{tikzpicture}[scale=0.9,>=stealth']
	\newcommand{\ptsize}{1pt}
	\begin{scope}[yshift=0ex, xshift=-4cm]
	\foreach \i in {0,...,3} {
		\node[circle, draw, inner sep = \ptsize] (b\i) at (\i,0) {};
	}	
	\draw (b0) node[left] {$s$} -- (b1);
	\draw (b1) -- (b2);
	\draw (b2) -- (b3) node[right] {$t$};
	\node[left] at (-1,0) {$\tx:$};
	\end{scope}
	\begin{scope}[yshift=0ex, xshift=4cm]
	\foreach \i in {0,...,3} {
		\node[circle, draw, inner sep = \ptsize] (b\i) at (\i,0) {};
	}	
	\draw (b0) node[left] {$s$} to[bend right] (b2);
	\draw (b2) to (b1); %[bend right]
	\draw (b1) to[bend right] (b3) (b3) node[right] {$t$};
	\node[left] at (-1,0) {$\ty:$};
	\end{scope}
	\end{tikzpicture}
	\caption{������ ���� ����� $\tx$ � $\ty$, ��������������� ���� ��������� ������ �������� $\Path^{\uparrow}(4)$.}%
	\label{fig:contrpath2}%
\end{figure}

�������� �� ����������� ������ 1 � 2, %�~\cite{Bondarenko:1993SW3A}, 
�������� ���������, ��� ���� � �������������� ������� 2 �~\cite{Bondarenko:1993SW3A} 
�������� ������� 1 ������~\ref{lem:path}, �� ��� ���������� ������ ��� �������� $\Dipath^{\uparrow}(n)$. ����� �������, ��������� �����~\ref{lem:path} �~\ref{lem:path2}, � ��������� �������������� ������� 2 ��~\cite{Bondarenko:1993SW3A}, �������� ��������� ���������.

\begin{theorem}[\cite{Maksimenko:2004, MaksimenkoDiss:2004}]
\label{thm:ShoTh2}
�������� ����� ����� ������� ������ � ���������� ���� $\Dipath(n)$ � ������������ ����������������� ���� �������� ����� $\lfloor n^2/4\rfloor$.
�� �� ����� � ��� �������� ����� ������ ��������� $\Path^{\uparrow}(n)$, $\Dipath^{\uparrow}(n)$ � $\ShortP(n)$.
\end{theorem}

��� ��� ������� �������� �� ��, ��� ������� 2 �~\cite{Bondarenko:1993SW3A} �������������� ��� \emph{�������������} $s$-$t$ ����� $\Path(n)$, � �������~\ref{thm:ShoTh2} "--- ��� \emph{���������}. 


%%%%%%%%%%%%%%%%%%%%%%%%%%%%%%%%%%%%%%%%%%%%%%%%%%%%%%%
%
%  ������� ���������� ��������� ����
%
%%%%%%%%%%%%%%%%%%%%%%%%%%%%%%%%%%%%%%%%%%%%%%%%%%%%%%%

\section{������� ���������� ��������� ����}
\label{sec:NondirectAlg}

� ����������~\cite{BondBook:1995} ���������� ��� ������� ������������� ���������� �����, �������� ����� ������ ������� ���������������.

������ �������� �������� ������ �������� ����������� �� ��������� ������ ������������ �������������
\[
\CP_{d,N} = \Set*{(t, t^2, \dots, t^d) \in \R^d \given t \in [N]},
\]
��� $d = 4$, � $N$ ���������� ������ $2^n$. ��� ��� ��� $d \ge 4$ ��� ������� ����� ������������� ������� ������, �� �������� ����� ��� ����� ����� $N = 2^n$. � ������ �������, ������ ����������� ���������� $c_1 t + c_2 t^2 + c_3 t^3 + c_4 t^4$, ��� $t \in [N]$, ����� ���� ������ �� $O(n+b)$ ������� ��������~\cite{Pan:1996,Sagraloff:2016}, ��� $b$ "--- ������� ����� �������� �������. ����� ����, ��� ����� ������� � ������� ��������� ������������ ������ ������ $O(n)$~\cite{BondBook:1995}.
� ���� ����� ��������, ���, �������� ��������� ���������� �������~\ref{sec:EF4Cyclic}, ��� $\CP_{4,N}$ ����� ��������� ����������� ������������ ������� $O(n^2)$, ��� $n = \log N$. �� ���� � ���� ������� ��������� ���������� ����������� ������� ����� � �������� ������ ��������� ������, ��� �������� ����� �����. 

������ ������ ������� �� ��������� ������. ������ � �������������� ��������� \begin{equation*}
\Part(A) = \Set*{\bm{x}\in\{0,1\}^m \given A \bm{x} = \bm{1}},
\end{equation*}
��� $A \in \{0,1\}^{k \times m}$, ���������� ��� ����������
\begin{equation*}
\Part^r(A) = \Set*{\bm{x}\in\R^m_+ \given A \bm{x} = \bm{1}}.
\end{equation*}
��������, $\conv(\Part(A)) \subseteq \Part^r(A)$. 
����� �������� ��������, ��� ��������� $\Part(A)$ �������� ������������� ��������� ������ $\ext(\Part^r(A))$.
����� ����, ��� ������� �.\,�.~������~\cite{Trubin:1969}, ���� ���� ������������� $\conv(\Part(A))$ �������� ��������� ����� ������������� $\Part^r(A)$.
�� �� ����� � � ��������� ������ ������������� ������������� $\BQP(n)$ � ��� ����������~\cite{Bondarenko:1987,Padberg:1989}
\begin{equation}
\label{eq:BQPrelax}
\BQP^r(n) = \Set*{\bm{x} = (x_{ij})\in\R^{n(n+1)}_+ \given 
x_{ii} \ge x_{ij}, \ x_{jj} \ge x_{ij}, \ x_{ii} + x_{jj} - x_{ij} \le 1},
\end{equation}
��� ��� ��� ������������� ������� ������������ $\Part(A)$ � $\Part^r(A)$, ��� ��������������� ������ ������� $A \in \{0,1\}^{k \times m}$. (��� �����, ��������, ����� ��������������� ��������������� �������~\ref{thm:BQPStable}, � ����� $k = n(3n-1)/2$, $m = 2n^2$.)
����� �������, �������� ����� $\omega(\BQP^r(n)) \ge \omega(\BQP(n)) = 2^n$, ��� ��� ��� ������� ������������� $\BQP(n)$ ������� ������.
� ������ �������, ��������~\eqref{eq:BQPrelax} ������������� $\BQP^r(n)$ ���������, �������������, ������ �������� ����������� �� ��� ������������� ���������.
����� ����, ��������� ��� ������ ���� ����� ���������� ������� ��������~\cite{BondBook:1995}.

� ����� � ���� ������������ ����������� ��������� ������.
��������� ������� �������� ����� ���������� ������� (��� <<�������>>) ����?

%�������, ��� ����� �~\cite{Kolesov:2009} ��� ���������� ������ ���������� ��������
%��������� ��� ������� ������ � �����������, ������� �� �������� ������.
�����������, �������� ����--��������~\cite{Kuhn:1955,Munkres:1957} (��������� ����� ��� ��������� ����������� ���������) ��� ������� ������ � ����������� �� �������� ���������� ������� (� ����� <<�������>>) ����.
����� ����, ���� ����� ������ ���������� ������������� ������ ����������� ����������,
����������� �� �������� �� ������������, �� �������������� ��������� �� �� ���������� ������.

\begin{theorem}[\cite{Maksimenko:2014MAIS}]
	\label{the:assignment}
	�������� ����--�������� ��� ������ � ����������� �� ��������
	���������� ������� (� ����� <<�������>>) ����.
\end{theorem}

\begin{proof}
\newcommand{\cs}{c\lefteqn{'}}
	����� ������������, ��� � ��������������� ������ � ����������� ��������� 
	� �������� �������"=���������� ������ ���������� ����� ����� �����������
	������������� ������������ ����.
	����� ����� ����� �������� �������� ������� ������ ������.
	��� ������ ������������ � ���� �������
	\[
	\bm{c} = 
	\begin{pmatrix}
	c_{11} & c_{12} & \cdots & c_{1n} \\
	c_{21} & c_{22} & \cdots & c_{2n} \\
	\vdots & \vdots & \ddots & \vdots \\
	c_{n1} & c_{n2} & \cdots & c_{nn} 
	\end{pmatrix},
	\]
	��� $c_{ij}$ "--- ��� �����, ������������ $i$-� ������� <<������>> ���� �����
	� $j$-� ������� <<�������>> ����.
	���������� ���������� ������� ������ �������� ��������� $\Birk(n)$, ��������� �� ������� $n\times n$"~������, ������ ������ � ������ ������� �������
	�������� ����� �� ����� �������.
%	�������� �������� ���� ����� ������ ���������� \emph{�������������� ��������}.
	
	��� �������������� ��� �� ����������� �������� ����� ��������� ����--���\-���\-��.
	���������� ����������� ���� ������������� (������������� ����� � �������� ������� �����).
	�� ���� ����� ������������ ������� ������� ����� ���� ����� (��������, <<������>>),
	����� ������ (<<�������>>).
	��� ������ ������� ����� ����������� �� ����� ���������� ����� � ���������� �����.
	% (�.�. � ������ ������ ������� $c$ ���������� ���������� �������).
	��� ��� ���������� �� ����� ���� ����������� ���� ������� �����.
	����� � ���������� ����� ��� ���� ����������� ������� ���,
	��� ��������� "--- ���������������.
	�������, ��� ����� ����������� ����� ����� �������� ��������, 
	����� ����� � ������� ����� �������� �������������.
	�����, ����������, ������ ��� �������� �����������, � ������ ������.
	
	����� ������� �������� �������� �� ��,
	��� �� ���� ����� ������ �� ������������� ������ ������ ����� � ����������� �����.
	��� ������ �������� ����� ������� ����������� � �������������� 
	� ������ ���������� ������� ���� ���������� ��������������.
	������ �������������� ��������� ������������ ������.
	(� ���� �� ���������� ����������� �� ������������.)
	�� ������ ���� ������������ ���� ���� ����������� ��������� �����.
	�� �������� ��������, ����� �������, ��� ��� ������ � ������ ����� (� ������ �������).
	�����, ���������� �� ��� ������������ � ����� �������� ����� �~�.\,�.
	%� ���� �������, ���������� �� ��������� ���� ������������ � ��������� �� ������, � �.�.
	� ���������� ���������� ($(n-1)$-��, ���� ��������� $n$) ��������� ���������� ����� � ���������� �����.
	
	\textbf{1. �������� ����--�������� �� �������� ���������� ������� ����.}
	
	�������� ����������� \ref{def:direct-type}, ���������� �������
	���������� ���� $f$ (������, ��������� �������� ���������) 
	� ������ ��������� � ����� $Y$, ��������� �� ������� ���������� �������,
	���, ����� ��� ����� ���������� ��������� ��� �� ���� ������� �������
	����������� �� ������������:
	\[
	|R_f^+ \cap Y| = |R_f^- \cap Y| = 2.
	\]
	���� ����� �������� ������ ��� $n = 4$ ������ � ������ ����, 
	������� ����� �������������� �� ������������ �������� $n > 4$.
	
	���������� ������ ���������� ���� � �������� ��������� � ������ ����.
	����� ����� ����� ������������ � ���� �������
	\[
	\begin{pmatrix}
	c_{11} & c_{12} & c_{13} & c_{14} \\
	c_{21} & c_{22} & c_{23} & c_{24} \\
	c_{31} & c_{32} & c_{33} & c_{34} \\
	c_{41} & c_{42} & c_{43} & c_{44} 
	\end{pmatrix}
	\]
	�����������, ��� ���� ������� �����, ��� ��� ������ ������, ������� � ������,
	����������� ��������� ���������.
	��� $c_{i1}$ ������������ � ����� $c_{i2}$.
	�����, ���������� �� ��� ������������ � $c_{i3}$.
	�, �������, ���������� �� ��������� ���� ������������ � $c_{i4}$.
	
	������� �������� �� ��, 
	��� ������ ��������� ������������ ����� ��� ������ ���������� ������� ���� �� ������������. 
	� ��������� ������ �� � ����� ���� ���������������� ��������� ������������ 
	������ �� ������������� �� �� ���� �������. 
	(��� ����������� ����� ����� �������� � ���������������� �� ������ �� ������ ��������.) 
	
	����������� ������, ��� �� ������ �� ����� ��������� ������������ ������ �� ���������� ������
	(���� ������������ ����� ��������� ��������� ���� �����):
	1) $c_{11} < c_{12}$; \ 
	2) $c_{11} < c_{13}$; \ 
	3) $c_{11} < c_{14}$; \ 
	4) $c_{21} > c_{22}$; \ 
	5) $c_{22} < c_{23}$; \ 
	6) $c_{22} < c_{24}$.
	
	� ��������� ������ �� ��������� � ���� $c_{31} \compare c_{32}$,
	� ������� ��������� ��������� ���:
	\[
	\begin{pmatrix}
	0  & \cs_{12} & \cs_{13} & \cs_{14} \\
	\cs_{21} &     0  & \cs_{23} & \cs_{24} \\
	c_{31} & c_{32} & c_{33} & c_{34} \\
	c_{41} & c_{42} & c_{43} & c_{44} 
	\end{pmatrix},
	\]
	��� $\cs_{1i} = c_{1i} - c_{11}$, $\cs_{2i} = c_{2i} - c_{22}$, $i = 1,2,3,4$.
	�������, ��� ��� <<���������>> ���� � ���� ������� ����� ��������� �����
	������������� ��������, ����� �� ��������� ���� �� �����.
	(�� �������� ��������, ����� �������, ��� ��� �������� ���� ������������.) 
	
	�� ������ ���������, ��� �� ���� 24 ������������� ������� �����
	������������ �� ������ ����� ��������� �������� ���� ��������� ���
	(��� �������� ���������� ����������� ����� ��������������� �����):
	$\{\cs_{12}, \cs_{21}, c_{33}, c_{44}\}$
	�~$\{\cs_{12}, \cs_{21}, c_{34}, c_{43}\}$.
	�� ���� ��������� $X_f$ ���� $c_{31} \compare c_{32}$ ������� �� 22 �������������.
	
	���������� ������ ���������� ��������� $c_{31} \compare c_{32}$.
	
	1) ���� ����� $c_{31} < c_{32}$, �� ��������� ����� 
	������������� $\{\cs_{21}, c_{32}, \cs_{13}, c_{44}\}$ 
	� $\{\cs_{21}, c_{32}, \cs_{14}, c_{43}\}$.
	
	2) ���� ����� $c_{31} > c_{32}$, �� ��������� 
	$\{\cs_{12}, c_{31}, \cs_{23}, c_{44}\}$ 
	� $\{\cs_{12}, c_{31}, \cs_{24}, c_{43}\}$.
	
	������ ��� ������ ��������� ������������� ������� ������.
	��� ����� ����������� � ������� �������� ���������, ���������� � �����~\ref{lem:BirkAdj}.
%	������ ������������� �������� (��., ��������, \cite{Emelichev:1981}).
	
%	\begin{lemma}[�������� ���������]
%		��� ������������� ������ ����� � ������ �����, 
%		����� ����������� �� ����� �������� ����� ���� ����.
%	\end{lemma}
	
	�������������, ��� ����� ������ ��������� $c_{31} \compare c_{32}$
	�� ������ ����� �� ������� ������� ������������� ����� ���.
	
	\textbf{2. �������� ����--�������� �� �������� ���������� <<�������>> ����.}
	
	��� ���� ������� �����, ��� ����� ����������� ����� <<�������������>> �������� ��������, 
	����� ����� � ������� ����� �������� �������������.
	� ����� �������� �� �������� � ���� ��������� ������������ ������.
	��� ���� ����, ����������� ������ ������ � ���� ����,
	������� ������ �� ����� ���� $c_{ij} \compare c_{il}$.
	
	�������, ��� ��� ������ ������������� ����� ��������� ���� ����� ���,
	��� ������� �������� ����������� $c_{ij} < c_{il}$ ����� ���������, 
	� ��������� ��� ����� ����� �������������
	�������� ������ ���� ������ ������� �������������.
	����� �� �� ����� ����������� � � ��������� ����������� $c_{ij} > c_{il}$.
	������� �������, ������ ����� $K(\bm{x})$ ����� ����� ���������� ����� � ������ �� ��������������� $c_{ij} < c_{il}$ � $c_{ij} > c_{il}$.
	�������������, ������� (*) �� �����������~\ref{def:direct-type2} ����������
	��� <<��������>> �����, ��������������� �� ����� �������������.
\end{proof}

��������, ��� ������� ��������� ��������~\cite{Edmonds:1965} ��� ������ � �������������� � ������������ �������"=���������� ����� �������� �������� ����--��������, ��������.

\begin{corollary}
\label{cor:Edmonds}
	�������� �������� ��� ������ � �������������� �� ��������
	���������� ������� (� ����� <<�������>>) ����.
\end{corollary}

\begin{remark}
%� \cite[�. 77]{BondBook:1995} ���������� <<��������������>> ����, ��� �������� �������� �������� ���������� ������� ���� (������� ���� ���� ����������� � \cite[�. 87]{BondBook:2008}).
%����� �������������� ����� ��������� ����� � �������, ��� ��� ������ �������������� ��������� �������� ������ ���������� ������� ���� �~\cite{BondBook:1995} �������� �����������, ��� �� �������� ���������� ���� ���� ��������������. ��� �� �� ������ ����� <<�����>> ��������� �������������� ������� ������ ��� �������� ����� ������ ������������� ����������� ������������� $\Match(n)$ � ������������� ������ � ����������� $\Birk(n)$.
� \cite[������� 2.5.2]{BondarenkoDiss:1993} ���������� <<��������������>> ����, ��� �������� �������� �������� ���������� ������� ����. ����� �������������� ����� ��������� ����� � �������, ��� ��� ������ �������������� ��������� �������� ������ ���������� ������� ���� �~\cite{BondarenkoDiss:1993} �������� �����������, ��� �� �������� ���������� ���� ���� ��������������.
��� �� �� ������ ����� ����� ��������� �������������� ������� ������ ��� �������� ����� ������ ������������� ����������� ������������� $\Match(n)$ � ������������� ������ � ����������� $\Birk(n)$.
�������� ���������~\ref{cor:Edmonds}, ��� ���� �� ����� ���� ����������� ��� ������������ ������������� ����������� ��������� ������� ����, ���� ����������� ������ ��������� ��������. ��� ���������, ������ � ������������� �������������� ������� ������ �������� ����� ������ �������������� $\Match(n)$ � $\Birk(n)$ �������� ��������.
%, ����������� �������� ������� ����������� ��� ��������������� ������������.
\end{remark}

��� ������������ ������������ ���������� ��������� ������� � ���,
��� ������ ���������� ������� ���� � ����� �����������~\ref{def:direct-type}
�~\ref{def:direct-type2} ������������� � ����� �������������,
�� ������� ����������� ������� ������.
�������� ��� ����, �������� ��� ���� ������ ��������� ���������.

���������� ������ � ���������� ����:
����� ������ �������"=���������� ���� �� $n$ ��������,
� ������� �������� ��� ������� $s$ � $t$;
��������� ����� � ���� ����� ������� ���� � ����������� ��������� ����� �����, ����������� ������� $s$ � $t$.

� ���������� ������� ��� ���������� ��� ����, ��� �������� ����� ����� �������������
������ � ���������� ���� ������������������.
� ������ �������, ��� ����������� ����������������� ���� ����� ������ ���������� 
������������� ���������� (�� ������������ ����������� � ������ ����� � �������� �����).
������������� ����, ����� ��������� ������ ��������� ���������.

\begin{example}[\cite{Maksimenko:2014MAIS}]
	���������� ����� ������ �������� (������� ����) 
	��� ������� ������ � ���������� ���� � ����� ����.
	������� � ���� ���� �������������, ������������� �� ������� ������,
	����� ����������� ������� ����������������� ���� �����.
	(�������, ��� ��� ���� �������� ����� ����������� �� ��������, 
	��� ��� �������� ����������� �������� ������ ��������� ��� ������� �������.)
	��������������, ����� ������� ������ ��������� ���������� �� ����� ��������,
	����������� ��� �������� ������� �����������������.
	����� ������� �������� �� ��, ��� � ���� ���� ��������
	\emph{�� ���� ������� �� �������������}.
	(���� �� ������, ��� ����� ����� ������������ � �����, � �� ���� � ������.)
	� ��� ����� (����), ��� ��� ��� �������� ������� �������������, 
	� ������ ���������� ��� ���� �����, 
	����������� ������� ��� ������ ��������������� ����.
	������ ��������� ������� ������� ���� ����� ��������� � ���������� ���� �������.
	(��� ��� �� ���� ������� � ���� �������� ������� ����������������� �� ���� ���������.)
	����� ��, ��� ��� ����� ��������� ��������� ��������� ������� �������� ����� ������,
	�� ��� ���� ����� �������������� �������, ��� ��� ���������������
	������� ������ ������������� ��������.
	� ������ �������, ��� ���� �������� �����, 
	�������� ����� ��� ���� ������ ������������������.
	����������� ��� ����� � ��������~\ref{the:direct-type},
	�������� � ������, ��� ��������������� �����, � ������ � ��� � ��� ������, �� �������� �� ������, �� <<������>>.
\end{example}

%��� ��� ������� � ���, ��� �����������, ������������� �������������~\ref{def:direct-type} �~\ref{def:direct-type2}, �������� ������� �������� � �� �����������, � ��� �����, ��� ��������� ������������ ����������.

%����� ����, �������� ���� ����������� (� ��� ������, ���� ��� ������������� �����������), ��� �������, ������� ���������������� ������ ������������ ���������, ��� ����������� ������� �������� �������~\ref{the:direct-type}.

%� ���������� �������, ��� �������� �� ��������� ���������� ������ ���������� ������� ���� ��������� ��������� ������ ��� ���� ����� ������������, �������������� ������ ��� ���������� ��������� ���������������� ����� ������������� �����������.
%� ���������, ����� ��������� ������ ���������� �.�. ���������� �� ���������� � ���� ���������� �������.


%%%%%%%%%%%%%%%%%%%%%%%%%%%%%%%%%%%%%%%%%%%%%%%%%%%%%%%
%
% End of section
%
%%%%%%%%%%%%%%%%%%%%%%%%%%%%%%%%%%%%%%%%%%%%%%%%%%%%%%%

%% Глава 8
%%%%%%%%%%%%%%%%%%%%%%%%%%%%%%%%%%%%%%%%%%%%%%%%%%%%%%%%%%
%
%     ������������
%
%%%%%%%%%%%%%%%%%%%%%%%%%%%%%%%%%%%%%%%%%%%%%%%%%%%%%%%%%%

\chapter{������������}
\label{chap:Counterexamples}

\hfill
\begin{minipage}{0.4\textwidth}
������ �������� � ������,\\ �~������� ��������.
\begin{flushright}
	\emph{�.\,�.~��������}
\end{flushright}
\end{minipage}

%\hfill
%\begin{minipage}{0.55\textwidth}
%������ ��������� ������������� �� �������� ����, ��� ��������� �������� �������� ��� ��������� ����������, �� ��������� ������� ���������.
%\begin{flushright}
%�.\,�.~��������
%\end{flushright}
%\end{minipage}

����� ��������� ����� �������� ����� �� ��������� ������.
�����~��, ���� ������������� �������� 
%(���������� ������������ �������� ���������� ���\-���-��\-���\-���\-���) 
�������������, ������� ��������� ��������������� ��������������� ������?
�~������ ����� �~�������� ����� �������� ������������� ��������� ���������������: ����� ������ �������������, ����� ����������� �������������, ������� �~�������� ����� ��� �����, ����� �������������� �������� ������� ���������� ������"=����������� �~��������� ������.

� ������ ������� ������ ����� ���������� ������� �������� ��������������,
��� ������� �������� ���������� ���� ������������� ����������� ����������
�� �������� �������������� ��������� ��������������� ��������������� �����.
�� ������ ������� ���������� ������� ���� ����� ���������� �����������, 
������������� ������� ������������ ������������ 
�~����� �������� ������ ��������� ������ ���� �������������� ���������. 
��� ���� ������ ������ ��������� ��~�������������� �����, 
�~������ ������ ����� ���������������� ���������.

\section{������� �������}

������ � ����� ������. �����������, ��� ������� ������������� $\conv(X) \subset \R^d$ ����� ���� ���������� �����������. ����� ������ �������� ����������� �� $X$ ����� ���� ������ � ������� �������� ��������. ������ �������, ��� ����� ������ ������������� ����� ������� ������� ������� ��������� ��������������� ��������������� ������. ��, ��������� ������� �� ���������� ����� ���� ��� ������, ������������ ���������, ������ ��������� ����. ������ �������� ����������� �� ��������� ������ ���� $\{0,1\}^d$ ����� ���� ������ �� $d$ �����, ����� ��� ����� ������ ����� $2^d$.

��� ��� ������ ��������� ���������������� ������������� ���������~\cite{Khachiyan:1979,Karmarkar:1984}, �� ����� ����������� (������, ������� �� ���� � �����������) ������������� ����� ������� ������� ������� ���������.
��������, ����� ��� ������ ����������� ������� ������, ����� ������� ��������������� ������������ $\Perm(n)$, ����� ������ �������� ����� $n!$, ����� ����������� ����� $2^n-2$, � ������ �������� ����������� �� $\Perm(n)$ ����� ��������� ������� $O(n \log n)$.
�������, ��� ��������� ���������� ����� ������������� ���� ����� $\Theta(n \log n)$~\cite{Goemans:2015}.

�� ��� ��� ������������� ���������� ������ ��������� ������ ��������� �������������� (��. ����� � �������~\ref{sec:diameter}). ������������, ��� ������� �������������� �������� ����� ������ ������� �� ������ ��������� ����� ��������� ����������������~\cite{Santos:2013}. ����������������� ����� ����������� �������������� ��������� ������. �����������, ��� ��� �������� H-�������� \(\Set{\bm{x} \in \R^d \given A\bm{x} \le \bm{b}}\) ���������� (��� ������ ��������) ������������� $P$. (��� ����� ������������� ������ �������� ������ ������������ �����, ��� ��� ������������ ������������� ���������� ���� ��������"=������.) �����, �������� ����������� ������������ �� �������, ����� ����� ��������� �������� $Q$ ��� $P$. ��������,
\[
Q = \Set*{(\bm{x}, y) \in \R^{d+1} \given A\bm{x} + \bm{b} y \le \bm{b}, \ y \ge 0}.
\]
����� ������ �������� ����������� �� $P$ ����������� �������� � ����������� �� $Q$
(�� �������� ��������, ����� ������������, ��� $\bm{0} \in P$).
� ������ �������, ������� ����� ����� �������� �� ��������� ����, ��� ��� ������� (�����) �������� ��������������� ��������� ������� � ������ �������� ���������.
������� �����, ��� �������� $Q$ �������� ����������� ������������� $P$.

�����������, � ������� ���������� ������������� ����� ���� ������������������ � ����������������� ������������� ��������� ����� ����� ��� ������ ��������� �����.
��������, ��� ����������� ������ ������������� �� $n+1$ �������� �������� �������� $\Delta_n$. 

\begin{theorem}
	\label{th:simplex}
	��� ������ ��������� $\Delta_d\subset\R^d$ ���������� 
	��� ����������� ������������ $Q_{d+1}\subset\R^{d+1}$, ��� ������� $\omega(Q_{d+1}) = 2$.
\end{theorem}

\begin{proof}
	������������� ��� ������, ��� ����� ��� $d$'~������ ��������� ������� ������������ ���� �����.
	������� ����� ������������� ������ �������� ������� ��������:
	\[
	\Delta_{d-1} = \{x\in \R_+^d \mid x \in H\},
	\]
	���������� ������������ ���������������� ������� $\R_+^d$ �~��������������
	\[
	H = \left\{x=(x_1, \dots, x_d) \in \R^d \mid \mathbf{1}^T x = 1\right\},
	\quad \text{��� } \mathbf{1} = (1, \dots, 1) \in \R^d.
	\]
	
	�� �������� ����������� ������������ $Q_d \subset \R^d$ 
	���, ��� � ������������� �������� �� $H$ ��������� � $\Delta_{d-1}$.
	��� ������ $d=3$ ����� ���������� ���������� �� ���.~\ref{fig:1}.
	
	\begin{figure}
		\centering
		\begin{minipage}{0.2\textwidth}
	\centering
	\begin{tikzpicture}[scale=1.4, line join = round]
	\coordinate (1) at (1,0,0);
	\coordinate (2) at (0,0,-1);
	\coordinate (3) at (1,0,-1);
	\coordinate (4) at (0,1,0);
	\coordinate (5) at (1,1,0);
	\coordinate (6) at (0,1,-1);
	\coordinate (7) at (1,1,-1);
	\draw[densely dotted] (6) -- (2) -- (3);
	\draw (4) -- (5) -- (7) -- (6) -- cycle; % ����
	\draw (1) -- (3) -- (7) -- (5) -- cycle; % �����
	\draw (1) -- (4) -- (5) -- cycle; % �����
	\draw (1) -- (2) -- (4) -- cycle; % ����
	\draw (2) node[below left] {$P_3$};
	\end{tikzpicture}
\end{minipage}
\hfill
$\Longrightarrow$
\hfill
\begin{minipage}{0.25\textwidth}
	\centering
	\begin{tikzpicture}[scale=1.2, line join = round]
	%cnode/.style={inner sep = 0mm, minimum size = 2pt, circle, text centered,
	%		draw=black, thick, fill=black}]
	\coordinate (x) at (1, 0, 0);
	\coordinate (y) at (0, 0, 1);
	\coordinate (z) at (0,-1, 0); 
	
	\coordinate (a) at ($-1*(x)$);
	\coordinate (b) at (x);
	\coordinate (c) at (${sqrt(3)}*(y)$); %1.7321016);
	%\coordinate (c) at (0, 0, 1.7321016);
	\coordinate (d) at ($0.5*(a) + 0.5*(b) + sin(60)*(z)$);
	\coordinate (e) at ($0.5*(b) + 0.5*(c) + sin(60)*(z)$);
	\coordinate (f) at ($0.5*(a) + 0.5*(c) + sin(60)*(z)$);
	\coordinate (g) at ($1/3*(a) + 1/3*(b) + 1/3*(c) + 1/sin(60)*(z)$);
	
	\draw[densely dotted] (a) -- (d) -- (b) -- cycle (d) -- (g);
	\draw (b) -- (e) -- (c) -- cycle;
	\draw (c) -- (f) -- (a) -- cycle;
	\draw (c) -- (e) -- (g) -- (f) -- cycle; % �����
	\draw (a) -- (b) -- (c) -- cycle; % Bottom
	\draw (e) +(1ex, 0) node[right] {$Q^-$};
	
	\begin{scope}[yshift = 3mm]
	\coordinate (x) at (1, 0, 0);
	\coordinate (y) at (0, 0, 1);
	\coordinate (z) at ( 0, 1, 0); 
	
	\coordinate (a) at ($-1*(x)$);
	\coordinate (b) at (x);
	\coordinate (c) at (${sqrt(3)}*(y)$); %1.7321016);
	%\coordinate (c) at (0, 0, 1.7321016);
	\coordinate (d) at ($0.5*(a) + 0.5*(b) + sin(60)*(z)$);
	\coordinate (e) at ($0.5*(b) + 0.5*(c) + sin(60)*(z)$);
	\coordinate (f) at ($0.5*(a) + 0.5*(c) + sin(60)*(z)$);
	\coordinate (g) at ($1/3*(a) + 1/3*(b) + 1/3*(c) + 1/sin(60)*(z)$);
	
	\draw[densely dotted] (d) -- (a) -- (b);
	\draw (b) -- (e) -- (c) -- cycle;
	\draw (c) -- (f) -- (a) -- cycle;
	\draw (b) -- (d) -- (g) -- (e) -- cycle;
	\draw (c) -- (e) -- (g) -- (f) -- cycle;
	\draw (e) +(1ex, 0) node[above right] {$Q^+$};
	\end{scope}	
	\end{tikzpicture}
\end{minipage}
\hfill
$\Longrightarrow$
\hfill
\begin{minipage}{0.3\textwidth}
	\centering
	\begin{tikzpicture}[scale=1.2, line join = round]
	%cnode/.style={inner sep = 0mm, minimum size = 2pt, circle, text centered,
	%		draw=black, thick, fill=black}]
	
	\coordinate (a) at (-1, 0, 0);
	\coordinate (b) at ( 1, 0, 0);
	\coordinate (c) at ( 0, 1.7321016, 0);
	\coordinate (d) at ( 0, 0, 0.4330254);
	\coordinate (e) at ( 0.5, 0.8660508, 0.4330254);
	\coordinate (f) at (-0.5, 0.8660508, 0.4330254);
	\coordinate (g) at (   0, 0.5773672, 0.5773672);
	\coordinate (h) at (   0, 0, -0.4330254);
	\coordinate (i) at ( 0.5, 0.8660508, -0.4330254);
	\coordinate (j) at (-0.5, 0.8660508, -0.4330254);
	\coordinate (k) at (   0, 0.5773672, -0.5773672);
	
	\coordinate (ap) at (-1, 0, -2.0);
	\coordinate (bp) at ( 1, 0, -2.0);
	\coordinate (cp) at ( 0, 1.7321016, -2.0);
	\coordinate (dp) at ( 0, 0, -2.0);
	\coordinate (ep) at ( 0.5, 0.8660508, -2.0);
	\coordinate (fp) at (-0.5, 0.8660508, -2.0);
	\draw (f) +(0, 2ex) node[above] {$Q_3$};
	\draw ($(ep)!0.5!(fp)$) node {$\Delta_2$};
	
	\draw %[densely dotted] 
	(ap) -- (bp) -- (cp) -- cycle;
	\draw[thin, dashed] (a) -- (ap) (b) -- (bp) (c) -- (cp) (h) -- (dp) (i) -- (ep) (j) -- (fp);
	
	% Bottom
	\draw[densely dotted] (h) -- (k) -- (i);
	\draw[densely dotted] (j) -- (k);
	
	\draw (a) -- (d) -- (b);
	\draw[densely dotted] (b) -- (h) -- (a);
	\draw (b) -- (e) -- (c) -- (i) -- cycle;
	\draw (c) -- (f) -- (a);
	\draw[densely dotted] (a) -- (j) -- (c);
	
	\draw (f) -- (g) -- (d);
	\draw (g) -- (e);
	\end{tikzpicture}
\end{minipage}
		\caption{����� ���������� ���������� $Q_3$ ��� ��������� $\Delta_2$}
		\label{fig:1}
	\end{figure}
	
	������������ $Q_d$ ����� ������������ ������������ $H$.
	������� ����� ����� ������� ���� ���� ��� ��������,
	������������� �~$H^+ = \{x \in \R^d \mid \mathbf{1}^T x \ge 1\}$.
	��������� ��� �������� ����� $Q^+$.
	
	��������� ����� $P_d$ ����������� ���������� ����
	$C_d = \{x\in\R_+^d \mid x_i \le 1, \; i\in[d]\}$
	� ���������������� $H^+$.
	�� ����������, $P_d$ "--- ��� <<��� ��� ����� �������>>.
	������ ��������� $Q^+$ ��� ��������� ������������ ��������������
	(�� ��������� ������������� ���) ������������� $P_d$:
	\[
	%  x \mapsto \frac{x + \mathbf{1} \left(\sum_{i=1}^d x_i - 1 \right)}{\sum_{i=1}^d x_i}.
	Q^+ = \left\{\left.\frac{x + \mathbf{1} \left(\mathbf{1}^T x - 1 \right)}{\mathbf{1}^T x} \right| x\in P_d\right\}.
	\]
	�������, ��� �������������� $H$ ����������� ������������ ����� ��������������.
	�������������� ���� $S_i = \{x\in\R^d \mid x_i = 0\}$ 
	(� ������ � ���� ��������������� ���������� ������������� $P_d$)
	������ �������������� ��������� �~�������������� 
	\begin{equation}
	\label{eq:Cube1}
	S'_i = \{x\in\R^d \mid \mathbf{1}^T x - d x_i = 1\}, \quad i\in[d].
	\end{equation}
	������� �����, ��� ������ ������� �������������� $S'_i$ ����������� ������� ������� �������������� $H$.
	�������������, ������������� �������� $Q^+$ �� ��������� $H$ ��������� � $\Delta_{d-1}$.
	�~������ �������, $Q^+$ ������������ ������������ <<���� ��� ����� �������>>,
	�� ���� � ������� ������������ �~����� ������������� $Q^+$ ��� ������� ���� ����� ����� �~$H$,
	������, �~����������� $H$ �~����� �� ��������������� $S'_i$.
	
	�~�������� �� �� ��������� ����������� �~�~��������� ������������� $Q^-$,
	����������� ���������� ������ $Q^+$ ������������ �������������� $H$.
	����� �������, ��� ������� $Q^+$ �~$Q^-$ ��� �����, ������� �~$H$, 
	����������� ������ ����������� ������������� $Q_d = Q^+ \cup Q^-$,
	������������ ���������������� $S'_i$.
	�~������, ���� ������������� $Q_d$ �� �������� �������������.
\end{proof}

����� �������, �������� ����� ����� ������������� ����� ���������� ��������� �� ���� �������� � ���������� ������������� (�� ����, �� ����, � ����� ������� ������).

����, � �������~\ref{sec:NondirectAlg}, ���� ������� ��� ��� �������, ����� ��������� ���������� ������������� ���� ������� ����� ���������� ������ ��������� �����, ��� �������� ����� �����.
� ����� � ���� ��������� ��������� ������.
���������� �� ������� �������� ����� ������������� �����������, ��������� ������� ����������� ���������� �� ��������� ���������� ������������� ������?


%%%%%%%%%%%%%%%%%%%%%%%%%%%%%%%%%%%%%%%%%%%%%%%%%%%%%%%
%
%  ��������� ���������� � ����� �������������� ��������
%
%%%%%%%%%%%%%%%%%%%%%%%%%%%%%%%%%%%%%%%%%%%%%%%%%%%%%%%

\section{��������� ���������� � ����� �������������� ��������}

�~2014~���� ������� �������~\cite{Rothvoss:2014}, ��� ��������� ���������� ������������� $\Match(n)$ ������ � �������������� � ������ ����� �� $n$ �������� ����� $2^{\Omega(n)}$. � ������ �������, ��� � ������� � ��������� ����� �������� ��������� ��������� ���������� � ������������� $O(n^3)$~\cite{SchrijverCO:2003}.

������� �����, ��� ��� �������������� ������������� �� �������� �������������, 
��� ��� ����������� ������� �� ��� �������������� �������.
�~���������, ��� ��������������� �� ���������, ������� ���� �~�� �� ����� ������ $n$ 
(�, ��������������, ���� � ��� �� ������������� ���), 
��������� ����������� ������������ ����� ��������� ����������� ������ �������� 
��~$O(\log n)$ ��~$\Omega(\sqrt{n})$~\cite{Fiorini:2012polygons}.

������������� ���������������, ���������� ��������� ���������� �������������, ������ ����� �������������� �������� ������� ���������� ������"=�����������.
����������� �� ���� ��������� �������� (��.~����� � �������~\ref{sec:RectCover}) ��� ����� ����������� ����� � �������������� ��������� ������, ��� ��������� ����������.
��������, �~\cite{FioriniKPT:13} �����������, ��� $\rc(\Match(n)) \in [n^2, n^4]$.
��� �� ��������, ��� ��� ��������������� ������������� $P$
\begin{equation}
\label{eq:simpl}
\rc(P) = O(d^2 \log n),
\end{equation}
��� $d = \dim P$ "--- �����������, � $n = |\ext P|$ "--- ����� ������ ������������� $P$.
�������, ��� ��� ������ ����������� � �������������� ���������� ������ �������� ����������� �� �������� ������������ ������������� $\CP_{d,N}$. 
���� ��� ���������� ������ ���������� ������� $d$ � ��������� $2^{-n}$ (������������ ��������� ������) ��������� ������� $d^2 (d+n)$ ������� ��������~\cite{Pan:1996}.

���� ����� ������ ������ NP"~������� ������ �����,
��� ����� �������������� �������� ������� ���������� ��� �� ������������� ������������� � ��������� ������ ������������� ������� ����� �������������.
�������� ���� ����������� �~��������� ��������� ������ 0/1"~�������������
(���������������� � NP"~������� �������), �������� ��� ��������������.

%%%%%%%%%%%%%%%%%%%%%%%%%%%%%%%%%%%%%%%%%%%%%%%%%%%%%%%%
% 
%   Cyclic perturbation
%
%%%%%%%%%%%%%%%%%%%%%%%%%%%%%%%%%%%%%%%%%%%%%%%%%%%%%%%%

%\subsection{����������� �����������}

��� ������� $x\in \{0, 1\}^d$ ��������� ��� ����� $n(x)$, $0 \le n(x) \le 2^d - 1$:
\[
n(x) = \sum_{i=1}^d 2^{i-1} x_i. %, \quad 0 \le n(x) \le 2^d - 1.
\]
���������� ����������� $\M\colon \{0, 1\}^d \to \N^d$, ������������� $x\in \{0, 1\}^d$ �~$\eps \in \N^d$:
\[
\eps_i = (n(x))^i, \quad i\in[d].
\]
%\begin{align*}
%\eps_1 &= n(x), \\
%\eps_2 &= (n(x))^2, \\
%       & \dots\\%\dots\dots \\
%\eps_d &= (n(x))^d.
%\end{align*}
������ �~������������ ��������� <<���������� �������>> ���������
\[
K = 2^{d^3}.
\]
�������, ��� ��� ������ $x\in \{0, 1\}^d$ �������� $\|\M(x)\|$ <<����� ����>> �� ��������� � $K$:
\begin{equation}
\frac{\|\M(x)\|}{K} \le \frac{\|\M(x)\|_1}{K} \le \frac{1}{K} \sum_{i=1}^d (2^d-1)^i \le \frac{2^{d^2}}{2^{d^3}} 
= 2^{-d^2(d-1)}.
\label{eq:bigK}
\end{equation}

����� $X\subseteq \{0, 1\}^d$.
��������� 
\[
Y = \Cpert(X) = \bigl\{y\in\Z^d \mid y = K x + \M(x), \ x\in X\bigr\} 
\]
������� \emph{����������� ������������} $X$.
����, ��� ����� ����� ����������� ������ ����� �~�������� $X$ ������������� �~$\log_2 K = d^3$ ���.
����� ����, $Y = \ext \conv Y$, ��� ��� �������� $\|\M(x)\|$ <<����� ����>>.
��� ���� ������ �������� ����������� ����������� ������� � ���,
��� ���� ������ ������������� ������� ��� $X$ �������������,
�� � ��� $\Cpert(X)$ ��� ����� ��������������.

\begin{lemma}
	�������� �������� ����������� ����������� $X\subseteq \{0,1\}^d$ �������� �������������� ��������������.
\end{lemma}

\begin{proof}
	���������� ��������, ��� ����� $d+1$ �����\footnote{�� ������������� ������ <<����������>> ������ $|X| \ge d+1$.} 
	� ����������� ����������� 
	\[
	Y = \Cpert(X)
	\]
	������� ����������.
	%Now every point in $Y$ has integer coordinates.
	
	����, ��� ������� ������������ $\{y^1, y^2, \dots, y^{d+1}\} \subseteq Y$ ��� ����� ��������� �~��������������
	\begin{equation}
	\label{eq:matrix}
	\det\begin{vmatrix}
	1 & y^1_1 & y^1_2 & \!\dots & y^1_d \\[2pt]
	1 & y^2_1 & y^2_2 & \!\dots & y^2_d \\[2pt]
	\hdotsfor[1]{5} \\[2pt]
	1 & y^{d+1}_1 & y^{d+1}_2 & \!\dots & y^{d+1}_d
	\end{vmatrix} \ne 0.
	\end{equation}
	��������� $y^i = K x^i + \M(x^i)$ ��� ���������� $x^i \in X$, $i\in[d+1]$,
	�� ����� ������� ������� \eqref{eq:matrix} �� ����� ���� ������ $A,B \in \{0, K\}^{(d+1)\times(d+1)}$:
	\[
	A = \begin{pmatrix}
	0 & K x^1_1 & K x^1_2 & \!\dots & K x^1_d \\[2pt]
	0 & K x^2_1 & K x^2_2 & \!\dots & K x^2_d \\[2pt]
	\hdotsfor[1]{5} \\[2pt]
	0 & K x^{d+1}_1 & K x^{d+1}_2 & \!\dots & K x^{d+1}_d
	\end{pmatrix},
	\]
	%�
	\[
	B = \begin{pmatrix}
	1 & n(x^1) & (n(x^1))^2 & \!\dots & (n(x^1))^d \\[2pt]
	1 & n(x^2) & (n(x^2))^2 & \!\dots & (n(x^2))^d \\[2pt]
	\hdotsfor[1]{5} \\[2pt]
	1 & n(x^{d+1}) & (n(x^{d+1}))^2 & \!\dots & (n(x^{d+1}))^d \\[2pt]
	\end{pmatrix}.
	\]
	��� ������� ������������ $S\subseteq[d+1]$ ��������� $(d+1)\times (d+1)$-������� $D^S$:
	\[
	D^S_{ij} = 
	\begin{cases}
	A_{ij},& \text{if } i\in S,\\
	B_{ij},& \text{if } i\not\in S.
	\end{cases}
	\]
	�~���������, $D^{[d+1]} = A$, $D^{\emptyset} = B$.
	��� ��������, ������������ ����� ���� $n\times n$-������ ����� ���� �������
	� ���� ����� ������������� $2^n$ ������:
	\begin{equation}
	\label{eq:twomatrixsum}
	\det(A+B) = \sum_{S\subseteq[d+1]} \det D^S.
	\end{equation}
	��� ������� ��������� $S\subseteq[d+1]$ ������� $D^S$ ����� ��� ������� ���� ������ �� $A$.
	�~������, $\det D^S$ ������� �� $K$.
	�~������ �������, $\det D^{\emptyset} = \det B$ ���� ������������ �����������:
	\[
	\det B = \prod_{1\le j < k \le d+1} \bigl(n(x^k) - n(x^j)\bigr) > 0.
	\]
	�������, ���
	\[
	\det B = \prod_{1\le j < k \le d+1} \bigl(n(x^k) - n(x^j)\bigr) \le (2^d-1)^{d(d+1)/2} < K.
	%\prod_{1\le j < k \le d+1} (t_{i_k} - t_{i_j}) \le (t_N/2)^{d(d+1)/2} < 2^{d^3/2} \quad \text{for } d > 2.
	\]
	�������������, ����� \eqref{eq:twomatrixsum} �� ����� ���� ����� 0.
\end{proof}

�������� �� ������� \eqref{eq:simpl}, ��������

\begin{corollary}
	\label{cor:simplicial}
	����� �������������� �������� 
	\[
	\rc(\conv \Cpert(X)) = O(d^2 \log |X|) = O(d^3)
	\] 
	��� ������ $X \subseteq\{0,1\}^d$.
\end{corollary}

������ �� ������ � ���������� ��������������� ������� NP-������� ������, 
����� �������������� �������� ������� ���������� ������-����������� ������������� �������
�������������.

���������� ����������� ����������� ������ ������ ������������� �������������
%\footnote{����� �~����� �� ������ ���������� ���� �~�� �� ����������� ��� ������������� �~��������� ��� ������.}
\begin{equation*}
%\label{eq:BQP}
\BQP(n) = \left\{x=(x_{ij})\in\{0,1\}^{n(n+1)/2} \mid 
x_{ij} = x_{ii} x_{jj}, \; 1\le i < j\le n\right\}.
\end{equation*}
%�������, ��� ������� ��������� ������ ����������� ����������� $\CBQP(n) = \Cpert(\BQP(n))$ ������������� �� $n$.
�������� ���������~\ref{cor:simplicial}, 
$\rc(\conv \CBQP(n)) = O(n^5)$, �.\,�. �������������.
�~������ �������, �����������

\begin{prop}
������ ����������� �� $\CBQP(n)$ � ������� �������� $c \in \{-1,0,1\}^{n(n+1)/2}$ NP-������.
\end{prop}

\begin{proof}
	���������� NP-������� ������ ���������� ��������� ����� ����� $G=(V, E)$, 
	��� $V = [n]$ "--- ��������� ������.
	������� ������ $c = c(G) \in \Z^{n(n+1)/2}$ ��������� ��������� �������:
	\[
	c_{ij} = \begin{cases}
	1, & \text{���� } i=j,\\
	0, & \text{���� } \{i,j\} \in E,\\
	-1, & \text{���� } \{i,j\} \not\in E.
	\end{cases}
	\]
	����� ������, ���
	$\max\limits_{x\in \BQP(n)} c^T x$
	����� ��������� ����� ����� $G$.
	�� ��� ������ $x \in \BQP(n)$ �~$y = \Cpert(x)$, 
	�������� ����������� \eqref{eq:bigK},
	%the~difference
	\begin{equation*}
	%\begin{split}
	\left|c^T x - c^T y / K\right| = |c^T \M(x) / K|
	\le \frac{1}{2^{d^2(d-1)}} \le \frac{1}{2^{18}},
	%\end{split}
	\end{equation*}
	��� $d = n(n+1)/2$, $n \ge 2$.
	��~���� ������� ������ $\CBQP(n)$ ��� ���������� �������� ������� $c$
	����� ��������� ����� ����� $G$ � ��������� �� $2^{-18}$.
	�������������, ������ $\CBQP(n)$ �� ����� ������ ���������� ��������� ����� �����.
\end{proof}




%%%%%%%%%%%%%%%%%%%%%%%%%%%%%%%%%%%%%%%%%%%%%%%%%%%%%%%
%
% End of section
%
%%%%%%%%%%%%%%%%%%%%%%%%%%%%%%%%%%%%%%%%%%%%%%%%%%%%%%%

%� ���������� �������, ��� �������� �� ��������� ���������� ������ ���������� ������� ���� ��������� ��������� ������ ��� ���� ����� ������������, �������������� ������ ��� ���������� ��������� ���������������� ����� ������������� �����������.
%� ���������, ����� ��������� ������ ���������� �.�. ���������� �� ���������� � ���� ���������� �������.s


{\color{red}
	��� ��� ���� ������� ����, ��� �������������� ������������� ���������� ����� 	�������� ����� ����� ��� ������� �� ����������� ����������� �������������.
	�.�. ����� ����������� ������, ��� ����������� ������������� ���������� ��� ��������������.
	����� ����, ���� ��������� �� ��������� ������ ����� ����� <<�����������>> �������� �����.
	��������, ������ ������ ����������� �������� � ������� �� $n$ ��������� �������� ����� ������ ����������.
	������������ ������ ������ ������������ ����� �������� \cite{BondBook:1995}.
	��� ���� ������. �������������, �������� ����� ����� $n$.
	������������ ������ ������ ���������� \emph{���������������}, � �������� ����� ��� ����� ����� 2 \cite{Gaiha:1977}.
	�.�. � ������ ������ �������� ����� ����� �� ������� � �������� ���������� ������.
	
	����������� ������� ������� � ���, ��� ��� ������������� ���������� ����� ����������� ������������� �� ������ ������� ����������� ������� ����� ������ ��������������� ���������, ��� �������� ����� �����.
	� ������ �� �������������� NP-������� ����� ������� �������� ��������� �����
	����������� ���, ��� ������������ $\BQP(n)$ ����������� ������ ���� �������������� � ���� ������������ ������, ������������� �������� �����������.
}



% Заключение
%%%%%%%%%%%%%%%%%%%%%%%%%%%%%%%%%%%%%%%%%%%%%%%%%%%%%%%%%%
%
%     ������������
%
%%%%%%%%%%%%%%%%%%%%%%%%%%%%%%%%%%%%%%%%%%%%%%%%%%%%%%%%%%

\addtocounter{secnumdepth}{-5}
\chapter{����������}
\addtocounter{secnumdepth}{5}

%\hfill
%\begin{minipage}{0.55\textwidth}
%������ ��������� ������������� �� �������� ����, ��� ��������� �������� �������� ��� ��������� ����������, �� ��������� ������� ���������.
%\begin{flushright}
%�.\,�.~��������
%\end{flushright}
%\end{minipage}

� ��������������� ������ ��������������� �������� ������ ��������� �������� ����� ������������� ����������� � ������� ��������� ������������"=�������������� ������������� ��������������� �������������� �������� (�������������� � ��������� �����, �������� ��������� ������������ �������� ������). � �������� ����� ������������� ���������������: ����������� �������������, ����� ��� ������, ����� �����������, ������� ����� �������������, �������� ����� �����, ��������� ������ ������������� �����, ��������� ���������� ������������� � ����� �������������� �������� ������� ���������� ������"=�����������.

�\'������ ����� ����������� ������� � ����� �������� �������� ����������. 
���� ��� ���������� ��������� ���������� ��������� ���� �������������� �������������� (�� ����������� �������� �����), ��������������� � ��������� �������� ������������� �����������. � ������ ������������ ��������� �������������� ��������� NP-������� �����: �����������, ���������� ����, 3-������������, ������, ������ � �����, ����������� ��������� ������ �����, �������� � �������� ���������, ������ ������������ ����������������, ��������� �����, ���������� �������, �������� ��������������, ������ �������� � �����, ��������� �������� ������ � ����������� (������������ ����������, ���������� � ������������, 3-���������) � ��������� ������.
�� ������ ���� ������ ��������� �� ������� ����� �� ��������������� ������ �����~\cite{Karp:1972}. �� ����� �������������, �������������� � �����~\ref{chap:AffExamples}, �������, ��� ��������� ������� ������������ �������������� $\BQP$ ������� �������� � ���������� �������������� ������������� ���� �����.
����� �������, ��� ��� ��������� �������������� ��������� �� $\BQP$ ������� (�������������������) �������� ��������� ���� ������������"=�������������� ���������������. � ������ �������: ��������������������� ��������� ����������, ����� �������������� �������� ������� ���������� ������"=�����������, ��������� ����� �����. �������������, � �������~\ref{sec:BQP-power} ��������, ��� ��� ������ $k \in \N$ � $n \ge 2^{2\cdot \lceil k/3\rceil}$ ����� ������������ ������������ $\BQP(n)$ ����� $k$-����������� ����� �� ������������������� ������ $2^{{\Theta}\left( n^{1 / {\left\lceil k/3\right\rceil}}\right)}$ ������. ��������� ���������� �������� ����������, ������ �� ������������� �������� �������������� ���� �������� �������������, ������� $k$"~����������� ����� �� ������������������� (������������ ����������� �������������) ������ ������.

�������� ��������������� ��������� ��������������, ��� ������� ������ ������������� ����������� ������ �������� NP-������. ���� ���� � �������������� ��������� �����: �������� ���������, 3-������������, �����������, ���������� ����, ���������� �������, ������, ��������� ����� �� ��� ������ �����, ���������� � ������������, ��������� ��������������. ��������, ��� NP-������� �������� ����������� ������ ����������� ����� ��������������� �� ������������ ��������� �������������� $\NPadj$, ������� ���������� �.~�����~\cite{Matsui:1995} � 1995~�. �~������ �������, ��������, ��� ��������� ������ ����������� �� �������������� ����� ��� �������������� ��������� NP-������� �����: ������ � �����, ������������ ������ � ����������� � � �������� ��������������, �������� � ��������� ���������, 3-���������, ��������� �����, ����������� ��������� ������ �����. �~�������~\ref{sec:DoubleCovering} ��������, ��� �� ���� �� �������������� ��������� $\NPadj$ (�� ����������� ��������) �� ����� ���� ������ �� ��� ������ ������������� �� ������������� �������� � ������� ��������� ���������.

������ �������� ���������� ����� ���� ��������� � ��� ��������� ������������"=�������������� ������������� �������� ����� ������������� �����������, ������� �������������� ����������� �� ��������� �������� ������ (������� ��������).
�������� ����� �������� ����� ������ ������ � ���������� ������, ������������� ���������� ��� ���������� � ������� �������� ������������� �����. � �������~\ref{sec:ShortPath2Assignment} ��������, ��� �������� ��������� ��������� �������� ������ ���� ������ ������� �������� � ��������� ��������� ��������� �������� ������ ������ � �����������.

� �����~\ref{chap:ExtAff} �������� ������� ����������� �������� ����������, ������������ �� �������� ���������� ����������� ���������� ������������ ���������������� ��������� �����������. ���������� ������� ����������� �������� �������������� ������ ����������� �������� ���������� (� �������~\ref{sec:ExtAffExamples} ������������ ��������������� �������). ������ � ���, ����� ��� ����������, ������ ������, �� ��������� ���������� ��������� �������������� �������������� (��������, ����� �����������, �������� ����� ������ ��������������, NP-������� �������� ����������� ������). �~�������~\ref{sec:Cook4Polytopes} ��������, ��� ����� ��������� ��������������, ������ ������������ ������� ������� ����������� ������ NP, ���������� ������� �������� � ��������� ������� ������������ ��������������. ����� �������, ��� ������������� � ������~\ref{chap:AffTheory}--\ref{chap:ExtAff} ��������� �������������� ����������� ������������ ���� ����� ������������ ����������� �������� ����������.

� �����~\ref{chap:Cyclic} ����������� �������� ����������� ��������������, �������� ������ ���� � ������������� ������ ��������������. � �������~\ref{sec:EF4Cyclic} ��� $d$"~������� ������������ ������������� $\CP_{d,n}(t)$ �� $n$ �������� ������� ����������� ������������ ������� $O(\log n)^{\lfloor d/2 \rfloor}$. � �������~\ref{sec:RidgeGraph} ������ ������, ���������������� �.~��� � 1964 ����~\cite{Klee:1964}, "--- ������� ������ �������� �������� ����"~����� ������������ ������������� $\CP(d,n)$ ��� $n>2d$.

� �����~\ref{chap:Direct} ��������������� ������ ���������� ������� ����, � ������ ������� ������������, ��� �������� ����� ����� (�������������, ��������� ��������� ��������� �������� ������) ������ �������� ������ ������� ��������� ��������������� ��������������� ������ � ��������� ������� ������ ����������. �~�������~\ref{sec:ShortPathClique} ��������, ��� �������� ����� ����� ������ � ���������� ������ � ������� �� $n$ �������� � ������������ ����������������� ���� �������� �����~$\lfloor n^2/4\rfloor$. 
� �������~\ref{sec:NondirectAlg} ���������� �������������� ����, ��� �������� ����--�������� ��� ������ � ����������� (� ����� �������� �������� ��� ������ � ��������������) �� �������� ���������� ������� ����.
����� ����, ����������� ���������� ������������� ������ ����������� ����������,
����������� �� �������� �� ������������, �� �������������� ��������� �� �� ������ ���������� ������� ����.

� �����~\ref{chap:Counterexamples} ������������ ��������� �������� �������� ��������������, ��� ������� ���������� ���� ������������-�������������� �������������� ����������� ���������� �� �������������� ��������� ��������������� ��������������� �����. �~�������~\ref{sec:ExtensionCounterex} ������ ������ ��������� ��������������, ����� �������������� �������� ������� ���������� ������"=����������� ��� ������� ������������� ������������ �����������, � ������ �������� ����������� NP-������.
�~�������~\ref{sec:CounterexamplesOther} ���������� ������� ���� �����, ������������� ������� ������������ ������������ ���� �����, �� ���� �� ���� ����� ������������� ���������, � ������ ����� ���������������� ���������.
������ ������� ������� ���������� �� ���� ���������������� ��������� ������������� ������� ������������� ��������� ������.
���� ��������� ������� � ���, ��� �� ���� ����� ������������� �������������� ������������� �� ���� ����������� �������� ������������� ���������� ������ �� ����� � ���������������� ����������.
� ������ �������, ��� ������������� � ����������� ������� ��������� ������� ������������� � ���, ��� �������� �������������� ��������� ������ ����������� ������ ���������� ���������� �������������, � ����� "--- ������ �������������� �������� ������� ���������� ������"=�����������.

%�������� �������

%1. �� ��� ��� �������� �������� ��������� ������. ���������� �� ������� �����, ��������� ������� ���� �� ����������� ������ ����� �������������� �������� ������� ���������� ������"=����������� ���������������� �������������? �� ���� ��������� �������� ��������� ������ ����������� ������ ���� ��������. � ���������, ����� ������������, ��� ��������� ������ ������ ���������� ����� ������ �������������� ��������, � ������ "--- ���������� ����������� ������������.

%2. � ��������� ������� ��������� ����� ���� ��������� ��� ������� �����, ������������� ������� ������������ ������������, �� ���� �� ����� ������������� ���������, � ������ ����� ���������������� ���������. ���� ��������� ��������� �� ������ �� ���� ����, ��� ������������� ������� ������������� ������ ������ ����� ���������������� ���������, ��� �� ������������� �����������~\ref{def:family} ��������� \emph{�������������} �������������� � ���������~\ref{rem:PolyPred} � ����������� ������ ������������� �����������. ����� �������, �������� �������� ������ � ������������� ���� �������� �������� ������������� ��������������, ������ �������� ����������� �� ����� �� ������� ���� �� ������������� ���������, � �� ������ "--- NP-������ (��� ����� ���������������� ���������).

%3. � ��������� ����� ���������� ������ ������ ��������� ���������� $\CP_{d,n}$, ������� ��������������� �� ������� ������ $O(\log n)^{\lfloor d/2 \rfloor}$ �� �������~\ref{thm:main}. ��������, ��~\cite{FioriniKPT:13} �������, ��� $\xc(\CP_{d,n}) \ge \rc(\CP_{d,n}) = O(d^2 \log n)$.

%4. �� �������� ������ ������ �� ������������� ��������� ������ ������������� �������� ��������. 

%5. �������� �� ������������ ������������� ������ �������������� ������ ������������? ���, ����� �������, ����� �� ����������� <<������� �������� ������������� ������>> (��� ���� <<������� �������� ��\'�����>>) �������� ������� �������� �������������� ����� � ��� � �������������� ������ ����������, ���������������� ������� �����.

%�����������

%1. ������ ������ � ������������, ��������� ����������� ������������ �������� ������� ��� ������ ������ ��������� ����������� ������������ ������������� (�����) ������. ���������� � $\R^3$ ���������� $n$-�������� � ��������� �� ���������� $x^2+y^2=1$, $z=1$. ��������� ��� ������ ��������� $Y_n$. ��������� ����������� ������������ ������ $\cone(Y_n)$ ����� $\Theta(\log n)$. ����� �������, �� ���� ������ $n$ �� ����� ������� ��� ��������� ��� ������ �������. ���������� ����������� ������ ����������� �� ������� $z \in [0,1]$, $x=y=0$. ����� ����� ������� � ������ $\cone(Y_n)$ ��������� � �������.


% Список сокращений и условных обозначений
%\printnomenclature

% Словарь терминов
%\input{dict}

% Не добавлять длинное тире в качестве разделителя
%\newcommand\BibDash{}
% Выделять курсивом
\let\BibEmph=\emph
\bibliographystyle{gost2008ns} % 'n' for natbib compatibility
%\bibliographystyle{gost705s}

% Список литературы
\inputencoding{cp1251}
\bibliography{biblio}
% Когда список литературы будет сформирован окончательно, 
% нужно переименовать MaksimenkoThesis.bbl -> bibl-finished.tex и включить в текст последний файл непосредственно
%\input{bibl-finished}
%\bibliographystyle{plain}

% Список иллюстративного материала
%\listoffigures

% Приложения
%\appendix
%\input{a}

\end{document}
