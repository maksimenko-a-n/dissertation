% !TeX encoding = UTF-8 Unicode
%
% Внимание, при компиляции возникает ошибка, 
% связанная с выбором языка в .toc
% Проблема решается удалением строки \select@language {russian} из thesis.toc
% и включением декларации \nofiles.
% Или же закомментариванием \tableofcontents, как временная мера
% ПРАВИЛЬНЫЙ способ решения проблемы: \renewcommand\tocsectionfont{}
%
% Еще одна проблема -- опция times каким-то чудесным образом отключает пакет cmap
% Проблема находится в строках 
% \renewcommand\rmdefault{ftm} и \renewcommand\ttdefault{cmtt}
% после строки \usepackage{pscyr} в файле disser.cls
% Решение:
% \input glyphtounicode.tex 
% \pdfgentounicode=1
%
\documentclass[%
%draft,
doctor,         % тип документа
natbib,         % использовать пакет natbib для "сжатия" цитирований
%subf,           % использовать пакет subcaption для вложенной нумерации рисунков
pagebackref,    % обратные ссылки в списке литературы -- номера страниц 
href,           % использовать пакет hyperref для создания гиперссылок
%colorlinks=true, % цветные гиперссылки
%,fixint=false  % отключить прямые знаки интегралов
%,classified    % гриф секретности
%,libcat        % номер УДК
%,facsimile     % отображать факсимиле диссертанта
]{disser}

% Замена cmap
%\input glyphtounicode.tex 
%\pdfgentounicode=1
%\usepackage{cmap} %!!! cmap не работает ни здесь ни в стилевом файле

%\usepackage{mmap} % Рекомендация ВАК
\usepackage[T2A]{fontenc}
\usepackage[utf8]{inputenc}
\usepackage[english,russian]{babel}
%\usepackage[a-1b,usecharset]{pdfx} % Рекомендация ВАК
\usepackage[a-1b]{pdfx} % Рекомендация ВАК

%\usepackage{xcolor} В disser.cls нужно заменить пакет color на xcolor
\usepackage{tikz}
\usetikzlibrary{calc, graphs,babel,arrows.spaced, arrows.meta, bending}
\usetikzlibrary{shapes.misc} % for cross
%\usetikzlibrary{}

\renewcommand\tocsectionfont{}

\usepackage[
  a4paper, mag=1000,
  left=2.5cm, right=1cm, top=2cm, bottom=2cm, headsep=0.7cm, footskip=1cm
]{geometry}

\usepackage{verbatim} % Для окружения comment
\usepackage[intlimits]{amsmath}
\usepackage{amssymb,amsfonts}
%\usepackage{amsthm} % Конфликт с пакетом disser. Уже определена команда \openbox
\usepackage[amsmath, thref, hyperref, thmmarks]{ntheorem} % Для окружений типа Теорема
\usepackage{bm} % Для выделения жирным шрифтом математических символов
\usepackage{array}
\usepackage[linesnumbered,lined,ruled]{algorithm2e} % Для оформления псевдокода
\usepackage{placeins} % Для \FloatBarrier

\usepackage{paralist} % Для компактных по высоте списков (перечислений)

%\hypersetup{pagebackref = true, backref = page} % Обратные ссылки из списка литературы, нужно включать эту опцию сразу с подключением hyperref. Иначе (через \hypersetup) она не работает.

% Что-то новое ?????????????
%\ifpdf\usepackage{epstopdf}\fi
%\usepackage[autostyle]{csquotes}

% Список сокращений и условных обозначений
\usepackage[intoc,nocfg,russian]{nomencl}
\newcommand{\nomencl}[2]{#1 --- #2\nomenclature{#1}{#2}}
\setlength{\nomlabelwidth}{3em}
\setlength{\nomitemsep}{-\parsep}
\renewcommand{\nomlabel}[1]{#1 ---}
\makenomenclature

% Шрифт Times в тексте как основной
\usepackage{tempora}
% Команда \textsc{} НЕ РАБОТАЕТ
\newcommand{\problem}[1]{\emph{#1}}
\renewcommand{\textsc}[1]{\textbf{#1}}
% альтернативный пакет из дистрибутива TeX Live
%\usepackage{cyrtimes}

% Шрифт Times в формулах как основной
\usepackage[varg,cmbraces,cmintegrals]{newtxmath}
% альтернативный пакет
%\usepackage[subscriptcorrection,nofontinfo]{mtpro2}

% Номера страниц снизу и по центру
%\pagestyle{footcenter}
%\chapterpagestyle{footcenter}

% Точка с запятой в качестве разделителя между номерами цитирований
%\setcitestyle{semicolon}

% Ссылки на работы соискателя включаются в общий список литературы
\let\citemy=\cite
%\let\citeown=\cite

% Использовать полужирное начертание для векторов
\let\vec=\mathbf

% Путь к файлам с иллюстрациями
%\graphicspath{{fig/}}

%\setcounter{tocdepth}{2} % Глубина содержания. Значение 2 вызывает ошибку

% LaTeX �������

% New theorems
\theoremseparator{.} % ����� ����� ��������� �������
\theoremstyle{plain}
\newtheorem{theorem}{�������}[chapter]
\newtheorem{lemma}[theorem]{�����}
\newtheorem{prop}[theorem]{�����������}
\newtheorem{corollary}[theorem]{���������}
\newtheorem{conjecture}[theorem]{��������}
\newtheorem{property}[theorem]{��������}
\newtheorem{question}[theorem]{������}

%\theoremstyle{definition}
\theorembodyfont{\upshape}
\newtheorem{definition}{�����������}[chapter]
\newtheorem{remark}{���������}[chapter]

\theoremstyle{nonumberplain}
%\theoremseparator{}
\theoremsymbol{\rule{1ex}{1ex}}
\newtheorem{proof}{��������������}

% New commands
\newcommand{\Sum}{\sum\limits}
\newcommand{\eps}{\varepsilon}  %epsilon
\newcommand{\R}{\mathbb{R}}  %Set of real numbers
\newcommand{\N}{\mathbb{N}}  %Set of natural numbers
\newcommand{\Z}{\mathbb{Z}}  %Set of integer
\newcommand{\Q}{\mathbb{Q}}  %Set of rational
\renewcommand{\emptyset}{\varnothing} %Russian empty set
\newcommand{\NP}{\textup{NP}} 
\newcommand{\coNP}{\textup{co-NP}} 
\newcommand\op[1]{\mathop{\rm #1}\nolimits}
\renewcommand\vec[1]{\ensuremath{\mathbf{#1}}}

%\renewcommand\dim{\op{dim}}
\DeclareMathOperator*{\argmax}{argmax}
\DeclareMathOperator{\const}{const}
\DeclareMathOperator{\conv}{conv}
\DeclareMathOperator{\cone}{cone}
\DeclareMathOperator{\aff}{aff}
\DeclareMathOperator{\len}{len}
\DeclareMathOperator{\var}{var}
\DeclareMathOperator{\size}{size}
\DeclareMathOperator{\poly}{poly}
\DeclareMathOperator{\ext}{ext}
\DeclareMathOperator{\pyr}{pyr}
\DeclareMathOperator{\Size}{Size}  
\DeclareMathOperator{\Lat}{\mathcal L} % face lattice
\DeclareMathOperator{\Pert}{Perturb} % perturbation
\DeclareMathOperator{\xc}{xc} %Extension complexity
\DeclareMathOperator{\rc}{rc} %Rectangle covering number
\DeclareMathOperator{\rank}{rank} %Matrix rank
\DeclareMathOperator{\sgn}{sgn} %Sign
\DeclareMathOperator{\supp}{supp} %Support
\DeclareMathOperator{\diam}{diam} %Graph diameter
\newcommand{\dc}{{\Delta_{c}(d,n)}} % Diameter of the ridge graph of a cyclic polytope
\DeclareMathOperator{\vertices}{vert} %The number of vertices
\DeclareMathOperator{\facet}{facet} %The number of facets
\DeclareMathOperator{\face}{face} %The number of faces
\renewcommand{\le}{\leqslant}         
\renewcommand{\ge}{\geqslant}         
\renewcommand{\leq}{\le}
\renewcommand{\geq}{\ge}
\newcommand{\lea}{\le_A} 
\newcommand{\nelea}{\not\le_A} 
\newcommand{\lee}{\le_E} 
\newcommand{\npropto}{\lefteqn{\;\not}\propto}
\newcommand{\scalar}[1]{\langle #1\rangle}
\newcommand{\from}{\colon}
\newcommand{\symdiff}{\bigtriangleup}


\newcommand{\cC}{{\mathcal C}}  %���

\newcommand{\Cube}{\textup{Cube}} 
\newcommand{\Cross}{\textup{Cross}} 
\newcommand{\EP}{\textup{�}} % Difficult problem
\newcommand{\CP}{{\mathcal C}} % Cyclic polytope
\newcommand{\CPO}{\textup{�}_{\textup{���}}} % Cyclic polytope
\newcommand{\BQP}{P_{\textup{BQP}}} % Boolean quadric polytope (Correlation polytope)
\newcommand{\BPP}{P_{\textup{BPP}}} % Boolean p-power polytope
\newcommand{\Tensor}{P_{\textup{tensor}}} % Tensor product polytope
\newcommand{\RBQP}{P_{\textup{RBQP}}} % Relaxation of Boolean quadric polytope
%\newcommand{\SSP}{\textup{STAB}} % Stable set polytope
\newcommand{\Stable}{P_{\textup{stab}}} % Stable set polytope
\newcommand{\TSP}{P_{\textup{TSP}}} % Travelling salesman polytope
\newcommand{\ATSP}{P_{\textup{ATSP}}} % Travelling salesman polytope
\newcommand{\HDP}{P_{\textup{HDP}}} % Hamiltonian dipath polytope
\newcommand{\Ham}{P_{\textup{Hgraph}}} % Hamiltonian graph polytope
\newcommand{\HDPst}{P_{\textup{s-t-HDP}}} % Hamiltonian dipath polytope
\newcommand{\HP}{P_{\textup{HP}}} % Hamiltonian path polytope
\newcommand{\HPst}{P_{\textup{s-t-HP}}} % Hamiltonian stpath polytope
\newcommand{\Path}{P_{\textup{path}}} % Path polytope
\newcommand{\Dipath}{P_{\textup{dipath}}} % DiPath polytope
\newcommand{\ShortP}{P_{\textup{shortpath}}} % Short Path polyhedron
\newcommand{\MinCut}{P_{\textup{mincut}}} % Min Cut polyhedron
\newcommand{\Cycle}{\textup{Cycle}} % Cycles in complete digraph
\newcommand{\Knap}{P_{\textup{knap}}} % Knapsack polytope
\newcommand{\KnapEq}{P_{\textup{eq}}} % Equality knapsack polytope
\newcommand{\PRT}{P_{\textup{numpart}}} % Numbers partitioning polytope
\newcommand{\SAT}{P_{\textup{sat}}} % SAT polytope
\newcommand{\KSAT}[1]{P_{\textup{#1-sat}}} % SAT polytope
\newcommand{\Tree}{P_{\textup{tree}}} % Spanning tree polytope
\newcommand{\Match}{P_{\textup{match}}} % Matching polytope
\newcommand{\Cut}{P_{\textup{cut}}} % Cut polytope
\newcommand{\Cubic}{P_{\textup{3-factor}}} % Cubic subgraph polytope
\newcommand{\QAP}{P_{\textup{QA}}} % Quadratic assignment polytope
\newcommand{\QSAP}{P_{\textup{QSA}}} % Quadratic semi-assignment polytope
\newcommand{\PAP}[1]{P_{\textup{$#1$-A}}} % 3-assignment polytope
\newcommand{\TAP}{P_{\textup{3-A}}} % 3-assignment polytope
\newcommand{\CAP}{P_{\textup{CA}}} % Constrained assignment polytope
\newcommand{\Pack}{P_{\textup{pack}}} % Set packing polytope
\newcommand{\Cover}{P_{\textup{cover}}} % Set covering polytope
\newcommand{\Part}{P_{\textup{part}}} % Set partition polytope
\newcommand{\NPadj}{P_{\textup{matsui}}} % Matsui polytope
\newcommand{\DCP}{P_{\textup{2cover}}} % Double covering polytope
\newcommand{\POP}{P_{\textup{PO}}} % Partial ordering polytope
\newcommand{\LOP}{P_{\textup{LO}}} % Linear ordering polytope
\newcommand{\QLOP}{P_{\textup{QLO}}} % Quadratic linear ordering polytope
\newcommand{\ColorA}{P_{\textup{color1}}} % Graph coloring polytope
\newcommand{\ColorB}{P_{\textup{color2}}} % Graph coloring polytope
\newcommand{\ColorC}{P_{\textup{color3}}} % Graph coloring polytope
\newcommand{\ColorD}{P_{\textup{color4}}} % Graph coloring polytope
\newcommand{\Clique}{P_{\textup{clique}}} % Clique polytope
\newcommand{\Perm}{P_{\textup{perm}}} % Permutahedron
\newcommand{\Birk}{P_{\textup{birk}}} % Birkhoff polytope
\newcommand{\Steiner}{P_{\textup{steiner}}} % Steiner tree polytope

%\newcommand{\bigO}{\mathcal{O}}  % Big O 


%%%%%%%%%%%%%%%%%%%%%%%%%%%%%%%%%%%%%%%%
%% �������� ��� ���������� ������ ������������ ������ "~ � ��������� ���� "--
%% russianb.ldf        begin
%%%%%%%%%%%%%%%%%%%%%%%%%%%%%%%%%%%%%%%%
\makeatletter
\newcommand*{\glue}{\nobreak\hskip\z@skip}%  NEW!!!
%\declare@shorthand{russian}{"~}{\textormath{\leavevmode\hbox{-}}{-}}%  OLD!!!
\declare@shorthand{russian}{"~}{\glue\hbox{-}\glue}%  NEW!!!
\def\cdash#1#2#3{\def\tempx@{#3}%
	\def\tempa@{-}\def\tempb@{~}\def\tempc@{*}%
	\ifx\tempx@\tempa@\@Acdash\else
	\ifx\tempx@\tempb@\@Bcdash\else
	\ifx\tempx@\tempc@\@Ccdash\else
	%\errmessage{Wrong usage of cdash}%  OLD!!!
	\@Dcdash#3\fi\fi\fi}%  NEW!!!
%\def\@Acdash{\ifdim\lastskip>\z@\unskip\nobreak\hskip.2em\fi
%  \cyrdash\hskip.2em\ignorespaces}%
%\def\@Bcdash{\leavevmode\ifdim\lastskip>\z@\unskip\fi%  OLD!!!
% \nobreak\cyrdash\penalty\exhyphenpenalty\hskip\z@skip\ignorespaces}%  OLD!!!
%\def\@Ccdash{\leavevmode
% \nobreak\cyrdash\nobreak\hskip.35em\ignorespaces}%
\def\@Bcdash{\,\textendash\,\hskip\z@skip\ignorespaces}%  NEW!!!
\def\@Dcdash#1{\,\textendash\,\hskip\z@skip\ignorespaces#1}%  NEW!!!
\makeatother
%%%%%%%%%%%%%%%%%%%%%%%%%%%%%%%%%%%%%%%%
%% russianb.ldf        end
%%%%%%%%%%%%%%%%%%%%%%%%%%%%%%%%%%%%%%%%


\usepackage{mathtools} % ������� ���� ������ ��� ��� ������
%\mathtoolsset{showonlyrefs} % ���������� ������ ���������� ���������
% ,showmanualtags} % ��� ��������� � ������ ������
%\RequirePackage{mathtools} % �������������� ����������� ��� ������ ������
\providecommand\given{} % ��� ������������ �����, ������������ ������ �������
% ������� ������� ��� ���������� ��������
\newcommand\SetSymbol[1][]{%
	\nonscript\:#1\vert
	\allowbreak \nonscript\:	\mathopen{}}
\DeclarePairedDelimiterX\Set[1]\{\}{%
	\renewcommand\given{\SetSymbol[\delimsize]}	#1} 


\begin{document}

% Переопределение стандартных заголовков
%\def\contentsname{Содержание}
%\def\conclusionname{Выводы}
%\def\bibname{Литература}

% Включение файла с общим текстом диссертации и автореферата
% (текст титульного листа и характеристика работы).
% ����� ���� ���������� ����� ����������� � ������������
\institution{%����� ��� <<
����������� ��������������� ����������� ��.~�.\,�.~��������}%>>}

\topic{������������-�������������� ��������\\ ��������� ����� ������������� �����������}
%\topic{������������-�������������� ��������������\\ ��������� ����� ������������� �����������}

\author{���������� ��������� ����������}

\hypersetup{%
	pdftitle = {������������-�������������� �������� ��������� ����� ������������� �����������}, 
%	pdftitle = {������������-�������������� �������� ����� ������������� �����������}, 
	%pdftitle = {������������-�������������� �������������� ��������� �����~������������� �����������}, 
	%pdfsubject = {����������� �� ��������� ������ ������� ������� ������-�������������� ����},
	pdfauthor = {���������� ��������� ����������}, 
	pdfkeywords = {NP-������ ������, ������������ ������, �������� ����������, ����������� ������������, ��������� ����������, ���� �������������, ������� �����, �������� ����� �����, ����������� ������������, ����� �������������� �������� ������� ���������� ������-�����������}
} 

\specnum{01.01.09}
\spec{���������� ���������� � �������������� �����������}
%\specsndnum{01.01.04}
%\specsnd{��������� � ���������} 

%\scon{���������� �.\,�.}
%\sconstatus{�.~�.-�.~�., ����.}
%\sconsnd{��� ������� ������������}
%\sconsndstatus{�.~�.-�.~�., ����.}

\city{���������}
\date{\number\year}


% номер копии для грифа секретности
%\copynum{1}
% класс доступа
%\classlabel{Для служебного пользования}

% номер УДК
%\libcatnum{12345}

\title{ДИССЕРТАЦИЯ\\
на соискание ученой степени\\
доктора физико-математических наук}

\def\specskip{0pt} % Верт. отступ между специальностями
\maketitle

%%
%% Titlepage in English
%%
%
%\institution{Name of Organization}
%
%\title{Doctoral Dissertation}
%
%% Topic
%\topic{Dummy Title}
%
%% Author
%\author{Author's Name}
%
%\specnum{01.04.05}
%\spec{Optics}
%
%%\specsndnum{01.04.07}
%%\specsnd{Condensed matter physics}
%
%% Scientific consultants
%\scon{B.\,B.~Baranov}
%\sconstatus{Professor}
%%\sconsnd{P.\,P.~Petrov}
%%\sconsndstatus{Professor}
%
%% City & Year
%\city{Saint Petersburg}
%\date{\number\year}
%
%\maketitle[en]

% Содержание
\tableofcontents

% Введение
\intro
% ����� ������� ������������ �~�����������

\textbf{������ ������������.}
������ ���������� ������ ������������� ����������� ��������� ��������� ������������.
���� �������� ��������� $E$, ������� �������� $e$ �������� �������� ��������� ��� $c_e \in \R$, �~������������� ���������� \emph{�������� ������������} $f \colon 2^E \to \{\text{����}, \text{������}\}$.
������������ $s \subseteq E$ ���������� \emph{���������� ��������} ���� ������, ���� $f(s)$ �������.
��������� ���� ���������� ������� ��������� $S$, $S = \{s \subseteq E \mid f(s)\}$.
%, ������������ ��������� $S = \{s \subseteq E \mid f(s) = 1\}$ ���� ���������� ������� ������.
���� ������ ������� �~��������� \emph{������������} ������� $s \in S$ �~������������ (�����������) ����� $c(s) = \sum_{e \in s} c_e$.

� ������������� ����� ������, ������� ������� ������������ \emph{�������� ������}, ����� ��������� $E$, �������� $f$ � ���� $c_e$, $e\in E$, �� �������������, �� $E$ � $f$ ���������� ������������ ������� ������� ���������� $I$ �� ���������� �������, ���������������� ������ �������� ������. \emph{�������������� ������} ������������ ����� ������� ������ �������� ������, ����� ������� ������ $I$ � $c_e$, $e \in E$, �������������.
������ � �������� � �������������� �������� ���������� ������������� �������, ����� ��������� $I$ �������������, � ���� $c_e$ "--- ���.
��� ��� ��������� ���������� ������� $S$ � ����� ������ ������������ ����������, ��� ����� ������ ����� ������������ �����������~$S$.

��������, �~������ �~���������� ���� ������� ������ $I$ ���������� ���������� ��������� �������, ����� ������� �������� ������ $A$ � $B$, � ��������� (��������) ����� $E$, ����������� ���� �������, ���� $c_e$ �������� ������� �����, �~�������� ������������ $f$ ��������� �������� <<������>> ��� ������� ������������ �����, ��������������� ����� ������� ��~$A$ �~$B$.
������� ������������� ��������� �������� ������ ������ ������������ ��������� ������, ������ �~�����������, ������ ������������, ������ �~������� �~������ ������.

%������� ����������, ��� ��������� ��������� ���� ������, ����� ������.
���� ����� ����� ������ ����������� �~��������� ��� ����������� ������������� ��������,
������������ ������������ �������������� �~������������, ����������� �������� ������, ����������� ���������� �~�.\,�.~\cite{Paschos:2014}. 
����� ����, �~��������� ����� ������ ������������� ����������� ����������� ����������� �����: 
�������������� ������, ������������� ������������� �����, ������������� �������, 
������������ ���������������� �~������������� �����, ���������������� ���������,
������������ ������������ ������������ ���� (VLSI), %�~�������� ����, 
������������, �������� �������� �~�.\,�.~\cite{GrotschelCO:1995}.

��� ��������~\cite{SchrijverCO:2003}, �� ������ ������� ����� ������ ������� ���������� �� ���� ���������.
� ������, ��� ������������� $I$, ��� ������� ����������� ������� $s \in S$ ��������������� ��� \emph{������������������ ������}
$\bm{x} \in \{0,1\}^E$, ���������� $x_e$, $e \in E$, �������� ���������� ������� ������� ��� $e \in s$
� ����� ���� ��� $e \not\in s$. 
����� ��������� ���� ������������������ �������� ���������� ������� ���������� $X$, $X = X(S) \subseteq \{0,1\}^E$.
����� ����� �������������� �~���� ������� $\bm{c} = (c_e) \in \R^E$.
%, ����������� \emph{�������}.
���� ������ ��� ����� ������������� ����������� �~������ ������� $\bm{x} \in X$, �� ������� ������� ������� $\langle\bm{c}, \bm{x}\rangle$ ��������� ������������ (�����������) ��������.
����� ��, ��� ������������� �������� ������� ������� ��~�������� ��� ������ ������� ����������� $X$
� �������� ��������� $\conv(X)$.
����� �������, �~������ ������� $S$ ������������� ��������� �������� ������������ $\conv(X)$, ��������� �������� ������ ���������~$X = X(S)$.
%�������������, ������������ ����� ��������, ������� �������� ��������������.

��-������, ����� �������������� ������������� ���� ����������� ��� ������� ������ ������������ ���������� ��������������� �������������, �~���������, �������� ��������� ���������������� (��., ��������, \cite{Shevchenko:2004}).
��� ���������� ���� �������������� ������������ �~���� �������, �� ������ ������� ��� ��������� �� ������� ����������� �������� �~�������� �������~\cite{SchrijverCO:2003}.

��-������, ������������� ��������� ������������� �������� 
��������� ��������� ���������� ������� ��������������� ������.
%��� ��������� ����������� ��� ������������ ������� ��������� ������.
��������������� ��� ����� �~������� ������� ������� �������.
���������� ������� $s_1$ �~$s_2$ ������ $S$ ���������� \emph{��������},
���� ��� ���������� ������ ����� $\bm{c}$ ��� ��� ������� �������� ������������
��� ��������������� �������������� ������ �~������ ����������� ������� � �� ���.
��������� ������� $s_1$ �~$s_2$ ��������, ��� ��� �������������� ��������� ������ ����� $\bm{c}$ ����������� ������� �������������� ������ ����� �������� �~$s_1$ �� $s_2$ �~�������.
�� ���� �~���������, �������� ������ $S$, ������ ���� ������������� ��������,
�������������� �~����� ����������.
���, �~���� �������, ����������� ��� ����������� �� ��������� ��������� %, ��������� ������ ����������� �� ��������� $S$,
�, �~�����, ����� �������������� ��� ������������� ������ ��������� ��������������� ������.
��� �������������� ������� ��������� ������� ���������������� ��� ������� �����
����� ���������������� ��������� ������������� $\conv(X)$.
����� �������, ���� ������������� ������ �������� �~���� ��������� ���������� �~� ����������� ���������. 
��������� ��� �����, ������ ����������� ���������� �~�������� ������� ���� ������� ������ (��������� ���� ������, ������������� �� ���������) ������������� ������.
%��� �~�������� �������� �������� ������������ ��������� ������.


\textbf{������������ ����� �~������������ ���� ������������.}
������� ������� �~������� ������������� ����������� �������� �~1950-� ��.
��� ������������� ���� ����������� ����� ���������:
�������� ������ ��������������� ���, �������� ��������� ��������� ���������������� ������������ �~��������� �~���������� ��������-������ ��������,
� ����� ����������� ���� ������� ��������������� ����� �� ������.
������ �~1950-� ��. ��� ���������� ���������� ����� ������� ������ �~�����������,
�������� �����-���������� ��� ������ �~������������ ������, 
��������� ���������� ��� ���������� ���������� �����,
������ ������� ������ �~���������� ����� ��������� ��� ���������� ������������ ��������� ������,
� ����� �� ��������� ��� ���������� �~�������� ���� ������������ ����~\cite{SchrijverHistory:2005}.
�~��� �� �����, ����� ���������� �������� ��������-������, ���� ����������� ������ �������� �������
��� ���������� �~��������� ������� ������������� �����������.
�~���������, �~������� ������� ��������� ���������������� ��� ��������� ������������ �� ��� �������� �������� �~������� ������ ������������~\cite{DantzigFJ:1954}.

�������� ���������� ��������-������ ����� ����������
�������������� ������������ �~��� ������������� �������������.
���, ��������, ���� ��������, ��� ������ ������� ����� ����� ��������-������
����� ������� ������� ����� �������������.
�~����� �~���� ���� �~1957 ���� ������������� �������� �~���,
��� ������� ����� ������������� �� ����� ���� ������ �������� ����� ������ ��� ����������� �~������������.
�~��� ��� ���� �������� ��������� ������������ ��������, �� ���� �~2010 ���� �������
������� ��������� ������ 43-������� ������������� �~86 ������������, ������� �������� ������, ��� 43~\cite{Santos:2012}.
��� �� �����, �~����� ����\footnote{����� ��, ��� ������� ����� ��������� ������ ��������� �� ����� ����������� �~����������� �������������?} ��� �������� �� ��� ��� �������� �������� �~���������� �������� ������ ������~\cite{ZieglerHirsch:2012}.

����� ��~�������� ����������� ������ ��������� ������ �������� ������ �������� �������� �����
�������������, ������������� �~������������. 
�~1964~�. ���~\cite{Klee:1964} �����������, ��� ���� ������� ����� $\lfloor n/2\rfloor$,
��� $n$~--- ����� �����������. �� ����� ������ ������� �� �� ��� ������ �����������~\cite{Klee:1967}.
�~��� ��� ������ ���������� ��������.

������ �~�������� �~��������� �~������� ��������� ����� �~1950--60-�~��.
������������� ������� ������������ ���������, ������������ ���������������� �~������� 
��������~\cite{Edmonds:1965} �~�������~\cite{Cobham:1964}.
�~��� �� ����� �������~\cite{Edmonds:1965b} ���� ������� ������, ������� <<������� ��������������>>,
���, ��-����, �������� ������������ ����, ��� ������� ���� ������� ������� NP.
�� ��� ��������� ������������� �~�������� �~������ 1970-� ��. �����~\cite{Cook:1971}, ������~\cite{Karp:1972} �~�������~\cite{Levin:1973} NP"~������ �����. %~\cite{Garey:1982}.
������������� ��, ��� ������ ��~��� ��������� ���� ������ ����\'���� �����.
�~���������, ������� �������� ����������, ���������� ����� ��~�������� ����������� ��������� ������, �� ����� ���� ����� ����� �~���������� ������.

�������� NP"~������ ����� ��������� ������ ������� ��� ���������� ������������,
� ��� ����� ������� ��������������, ��������������� �~NP"~�������� ��������.
�~1978 ���� ������������~\cite{Papadimitriou:1978} �������, 
��� ������ �������� ����������� ���� ����������� ������ ������ ������������� ������ ������������ NP"~�����,
�� ���� ��� ����� ������, ��� �~���� ������ ������������.
������� ����������� ���������� ��� �������������� ��������� ������ NP"~������� ����� ��������� � ������� ��������� �������: ����; ����� �~�����; �����; ���������� �~����; ������� �~�����; ������� (��. ������ �~\citemy{Maksimenko:2013NP}).
�~������ �������, �~1975~�. ������ �������~\cite{Chvatal:1975}, ��� ��� ������������� ����������� �������� ��� ������ ������������� ���������.
����� ����, �~1984~�. ������� ���������~\cite{Greshnev:1984}, ��� ���� ������������� ������ �� $m$"~��������� ����� �����, �.\,�. ������ �������� ��������� ������ ��� ���� ����������.
���� ������� ����������� ���������� ���� �������� ��� ������������� ������ �~������������ �������~\cite{Beloshevskii:1986,Barahona:1986} � ��� ������������� ������ �~������������ ����� � �������"=���������� ������ �����~\cite{Bondarenko:1985}. ��������� ������������ �� ������ �������� ����� ������ � �������� ������������� �������������, ���� ����������� ������ ������~\cite{Bondarenko:1987, Padberg:1989}.

%��� ��������� ������������, ��� ��� ����� ����� ���� �~�� �� �������������� �������,
%�������������� �~������� �������� ����������~\citemy{Maksimenko:2013NP}.
%�����������, ��� ��� ����� ����� ���� �~�� �� �������������� �������,
%�������������� �~������� �������� ����������~\citemy{Maksimenko:2013NP}.
%����� ��������� ������ �������, ��� ��� ����������� �~���� ������� �������������
%�������� �~�������� ����� ������������ ������� ��������~\citemy{Maksimenko:2013NP},
%������ �������� ��������� ������ �������� NP"~�����~\cite{Matsui:1995}.

�~1979 ���� ������~\cite{Khachiyan:1979} ������ �������������� �������� ��� ������� ������ ��������� ����������������.
���� ���� ���� ������������� �������������� ������������� ��������������� ������� �~������� ����� ������������� �����������, ��� ��������� ������������ ������������ ������� ��������������� ��������������.
�~���������, ������� �������� ������������ ������� ������ �������������� ������� �~���������� ���������, �������� �~��������~\cite{Emelichev:1981}.

�~1980-� ��. ����������~\cite{BondBook:1995} ���������, ��� �������� �����\footnote{�~��������� ��� �������������� ���������� ���������� �����.} ����� ������������� ������ ������������� ��������������� ��������� ��������������� ��������������� ������. 
� ������, �� ���� �������� ��������������������� �������� ����� ������ �������������� ��������� NP"~������� �����: �����������, ������������ �����, 3-��������� �~��������� ������.
�~������ �������, �������� ����� ������ �������������� ��������� ������������� ��� ��������� ������������� ���������� �����: ����������, ����������� �������� ������, ������ �~���������� ����.
�� ��������� ���� ������ ���� ����������� ������ ���������� ������� ����, ������������, ��� �������� ����� �������� ������ ������� ��������� �~��������� <<������� ������ ����������>>~\cite{BondBook:1995}. 
������� ������ ������ �������� ����� ������ �������������� ����� ���������� ����������� ������ ������������~\cite{Shovgenov:2015, Nikolaev:2016, Nikolaev:2017, Shovgenov:2017}.
� ������ �������, ������� �������� ����������, ��������� ������� ��������� ��������������� ������, ��������� ��������������������� �������� ����� ������ ���, ��� ��� ������������� ������������� NP-������� ����� �������� �~�������� ����� ����� ������������ ������������, �������� ����� ����� �������� ���������������.

� �� �� �����, �������� �������� ��������������� ��������� ��� ������ ��������� ���������������� �������� ����� ������� ������ ����������� ��������� �������� ��� ������������� ������ ������������.
��� ��� ������� ���� ���������� �� ������������� ���� ����������� ������������ ������������� 
(��� ������ �������� �������). 
\emph{����������� �������������} ������������� $P$ ���������� ����� �������� �����������,
����������� ������������ $Q$ �����, ��� $P$ �������� ������������� ��������� $Q$.
��� ������������ $Q$ ���������� \emph{�����������} ������������� $P$.
� ���� ������� ��� ���� �������� �������, ����� ����� �������� ����������, ����������� ��� �������� �������������, ���������������, �~��� ��� ����������~--- ������������� ������������ ����� ������� ������ ������.
�� ���� ��~������� ������ ���������� ����������� ������������ ��� ������������� ������ ������������ �� ������� �~������ �~� 1988 ���� ���������~\cite{Yannakakis:1988}
�������, ��� ����� ������� �~�������� �� ����� �����������, 
���� ������������ ����������� ������������ ������������� ��������� ������������ �������� ���������.
��� �� ���� ��������� ��������, ��� ����������� �������� ������������ �~��� ������� ���������.
����� ����, ��������� �������, ��� ����� �������� ���������� �~����������� ������������ ������������� �� ����� ���� ������ ����� �������������� �������� ������� ���������� ������"=����������� �������������.
������������ ����������� ����� �������� ����������, ����������� ��� �������� ���������� �������������
���� ������� \emph{���������� ���������� �������������.}

�~����� 2000-� ��. ����������� ������������ ����� ��������� �������� ��������������~\cite{Conforti:2010,Kaibel:2011}, 
��� ������� �~��������� ������ ���� ����� ���������� ����������� �~������ �����������~\cite{FioriniPokutta:2012, Fiorini:2012polygons, KaibelPT:12, FioriniKPT:13, Rothvoss:2013, Rothvoss:2014, KaibelW:15}.
�~���������, �~2012~�. �������, ������, �������, ������ �~�� ���� �������� �������������� �������� ����������, �������,
��� ����� �������������� �������� ������� ���������� ������"=�����������
��� ������ ������������� ������������� ���������������~\cite{FioriniPokutta:2012}.

%\textbf{development}{������� ��������������� ���� ������������.}{
%����� �~������� ��������������� ����.
%}

\textbf{����� ������}
�������� ������ ������������� ������� ��������������, ��������������� �������������� ��������� ��������������� ����� ������������� �����������.
��� �������������: 
1)~����� ��������� ������ ��������� ������������� ������������� ��� �������� �������������� �������� ��������������, 
2)~���������� �����������, ���������� ������ ���� ������,
3)~������ ��������������� ������������� ��� ��� ���� ������������� �~�������� ������ ��������� ��������������� ��������������� �����.


\textbf{������ ������������.}
��� ������������ �������� ������������� �������������� ���������� ������������ ����� �~������ ������� ������� �������� ����������.
����� ������������ ������ ������ �������� ��������������, ��������� ����������������, ������ ������ � �������������� �������, ������ ��������� ���������� �~����������.


\textbf{������� �������.}
�������� ���������� ����������� �������� ������ �~����� ���� ������ �������������� ��������� �������:
%���������� �~����������� ���������� �������� ������. ��������� ���������� �������� ���������:
\begin{enumerate}
\item ������� ������� �������� ���������� �������� ������������� ��������������, ����������� ����� �������� ��������������, ��� NP"~������� �������� ����������� ������ �~��������������������� ��������� �������� �������������: ����� ������, ����� �����������, �������� ����� �����, ��������� ����������, ����� �������������� �������� ������� ���������� ������"=�����������. � ������� ����� ������� �������� ��������� ����������:
\begin{itemize}
	\item ��������, ��� ����������� ��������� ��������������, ������������� ����� �~1995 �., ������� �������� �~���������� �������������� ��������� �����: ������, �����������, ���������� �������, 3-������������, ���������� �~�������� ������������. %, �������� ��������� 
	����� ��~��������� ����� �������� ��, ��� ��� ��� ��������� ��������� �� �������������� ����� NP"~������� �������� ����������� ������.
	\item ��������, ��� ��������� �������������� ����������� ��������, �������������� �������� �~��������� ���������, �������������� ������ �� $n$"~����������� ��� $n \ge 3$ �~�������������� ��������� ����� ������������ �~������ �������� ���������� �~������� �������� �~��������� �������������� �����. ����� ����, �����������, ��� �� ���� ��~�������������� ����� (�� ����������� ��������) �� �������� ������ (� ��� ����� �������������) �� ��� ������ ��~�������������� ����������� ��������.
\end{itemize}
\item ���������� ������ �������� ������� ������������ ��������������, ����������� �� �� ������ ����� ����� ������ ��������� �������� �������������� NP-������� �����:
\begin{itemize}
	\item ��������, ��� ������ ������������ ������������� ������� �������� �~���������� ��������������, ������������� �~���������� ������ (������, ����������� �~�.\,�.), �~����� �~�������������� �������� ��������������, �������������� ������������ ���������� �~������������ �������� ��������������. 
	�� ����� �������, �~���������, ��� ����� �������� ������� ������������ ��������������, ��� ��������������������� ��������� ����� ����� �~����� �������������� �������� ������� ���������� ������"=����������� ������������� ����������� ����� ����� �����������.
	\item ������� �~������������ ��������� ������� �������������� �������~$p$.
	��������, ��� ��� ������������� $\lfloor 3p/2\rfloor$"~���������� �~������� �������� �~������� ������������ ��������������. �� ����� �������, ��� ��� ������ ������������ $k$ ������ ������������ ������������� (� ������ � ���� � ��� ��������� ������������� ���� ��������� �������������� NP"~������� �����) �������� $k$"~����������� ����� �� ������������������� (������������ ����������� �������������) ������ ������.
\end{itemize}
\item ������� ������� ����������� �������� ����������, ������������ �� �������� ���������� ����������� ���������� ������������ ���������������� ��������� �����������. ��������, ��� ��������� �������������� ����� �������� ������ ������������� ����������� �~���������� ������������ �� ������ NP ���������� ������� �������� �~��������� ������� ������������ ��������������. 
����� �������, ������������ ����������� �������� ���������� ��� ������������� ���� ��������� �������������� �������� ���� ����� ���������������.
������ ������ ��������� �������������� NP-������� ������, � �������� �� ����� ���� ���������� ������� ������� �� ���� ��~���������� ���� �������� ��������������.
\item ������� ������ ���� ������������� ��������� ��� ����������� ��������������\footnote{����������� ������������� �������� ������������ ������ ������ ����� ���� �������� ��������������, ������� �� �� ����������� �~�����~�� ����� ������.}:
\begin{itemize}
	\item ������� ���������� ����������� ������������ ������� $O(\log n)^{\lfloor d/2 \rfloor}$ ��� $d$"~������ ����������� �������������� �� $n$ ��������.
	\item ������� ������ �������� �������� ����� �������������, ������������� �~������������.
\end{itemize}
%\item ������� �������� ����� �������� ������ �~���������� ������ �~������������ ����������������� ���� ��������.
\item �������� ����������� ������ ��������������� ������������� ��������� ��������� ������������� ��������� �������������� � �������� ������ ��������� ��������������� ��������������� �����:
\begin{itemize}
	\item �� ������� ������ � ���������� ���� ��������, ��� ������ ���������� ������� ���� (������������� �.\,�.~����������) ��������� � � ��� �������, ����� �� ��������� ������� �������� ������������� �������� ����������� (��� ���� ������������ ������ ������������ � �������).
	\item ��������, ��� �������� ����--�������� (���������� �����) ��� ������ �~����������� �� ����������� ������ ���������� ������� ����.
	����� ����, ����������� ���������� ������������� ������ ����������� ����������,
	����������� �� �������� �� ������������, �� �������������� ��������� �� ��~����� ������.
	\item ��������, ��� � ������ ��������� ������������� ���� ����������, ���� �������� �� �������� �������������. 
	\item ���������� ������ ��������� �������������� NP"~������� ������, ����� �������������� �������� ������� ���������� ������"=����������� ������� �������������. 
	\item ���������� ������� ���� �����, ������������� ������� ������������ ������������ ���� �����, �� ���� ��~���� ����� ������������� ���������, �~������ ����� ���������������� ���������.
\end{itemize}
\end{enumerate}


\textbf{������������� �~������������ ����������.}
������ ����� ������������� ��������.
���������� ���������� ����� ���� ������������ ��� ������������
��������� ����� ������������� ����������� �~������ ����� ����������� ���������� �� �������. ������������ ����� ����������� ����� ���� ����� ������������ �~������������� ������������� ������� �������� ��������������.

�������� ���������� ����������� �������������� �� ������������ ��� ��������������,
��� �~����������� ������������� (������ �� �������� ��������� ����������):
�.�.~��������, �.�.~�����������, �.�.~��������, 
%V.~Pilaud, 
%H.~Fawzi, J.~Saunderson, P.A.~Parrilo,
%S.~Massar, M.K.~Patra, H.R.~Tiwary,
%L.B.~Beasley, H.~Klauck, T.~Lee, D.O.~Theis,
%K.~Qi, Q.~Feng, K.~Zhao,
%A.~Huq.
%A.~Makkeh, M.~Pourmoradnasseri, D.O.~Theis. The Graph of the Pedigree Polytope is Asymptotically Almost Complete (Extended Abstract)
%Huchette, J., Vielma, J. P. (2017) \url{https://arxiv.org/abs/1709.10132}
%Davis-Stober, C. P., Doignon, J. P., Fiorini, S., Glineur, F., Regenwetter, M. (2017) \url{https://arxiv.org/abs/1710.02679}
\foreignlanguage{english}{
L.B.~Beas\-ley, C.P.~Davis-Stober, J.P.~Doignon, H.~Fawzi, Q.~Feng, F.~Glineur, J.~Huchette, A.~Huq, H.~Klauck, T.~Lee, A.~Makkeh, S.~Massar, P.A.~Parrilo, M.K.~Patra, V.~Pilaud, M.~Pourmoradnasseri, K.~Qi, M.~Regenwetter, J.~Saunderson, D.O.~Theis, H.R.~Tiwary, J.P.~Vi\-el\-ma, K.~Zhao.}


%\textbf{results}{���������, ��������� �� ������:}{%
%����� �~���������� �~�����������.
%}

\textbf{��������� �����������.}
���������� ����������� ������������� �~����������� �� ��������� ������������, ��������� � �����������:
���������� ����������� <<���������� ����������� �~������������ ��������>> (�����, 2010), 
XVI ������������� ����������� <<�������� ������������� �����������>> (������ ��������, 2011),
������������� ����������� <<�������������� ���������������� �~����������>> (������������, 2011),
������������� ����������� <<���������� ���������>> ����������� 100-����� �.�.~������������ (���������, 2012),
������������� �������������� ����������� <<��������������� ������>> (������, 2012),
XXI ������������� ��������� �� ��������������� ���������������� (������, 2012),
������������� ��������� �� ������������� ����������� (�������, 2012),
������������� ����������� <<���������� ����������� �~������������ ��������>> (�����������, 2013), 
26-� ����������� ������������ ��������� �� ������������� ����������� (2013, �����),
������� ��������� �������������� ����������� ������������ ���� ��� ������ (���������, 2013),
������� �� ���������� ���������� �~��������� ������������ ������� (2013),
5-� ������� �� ������������� ����������� (������, �������, 2014),
9-� ������������� ����������� �� ������ ������ �~������������� (��������, �������, 2014),
XIII ������������� ����������� <<�������, ������ ����� �~���������� ���������: 
����������� �������� �~����������>>, ����������� 85-����� �.�.~������� (����, 2015),
5-� ������������� ����������� �� �������� ������� (������ ��������, 2015),
������� ����������� <<���������� �~�������������� ���������>> ���� ��.~�.�.~��������,
������������� ������� �� ���������� ����������,
������� �� ������ ��������� ���������� �� ��� �� ���,
������� ������� ���������������� ��������� �~���������� ��� ��.~�.�.~����������,
������� <<���������� � �������������� ���������>> ���� ���.


\textbf{��������� �~����� �����������.}
����������� ������� ��~��������, ������ ����, ���������� �~������ ���������� ��~195 ������������. �~����� ������ ����������������� ������ �~������ ���������� ���������� �������� �����������, ��� ��� ������ ������������.
����� ����� ����������� "--- 256 �������, ������� 22 �������. % 2+4+0+10+0+4+1+1
%������ ���������� �������� 187 ������������ �� 17 ���������.


\textbf{����������.}
����������� ���������� ������������, ������� ������� �~��� ������������ �����������~\citemy{MaksimenkoDiss:2004}.

��������� ����������� ������������ �~16 �������� �������, ��~��� 
12~������~\citemy{Maksimenko:2004,
	Maksimenko:2009,Maksimenko:2012DAN,Maksimenko:2012Cook,Maksimenko:2013k,Maksimenko:2013NP,Maksimenko:2013TSP,Maksimenko:2015DAN,BogomolovFMP:2015,Maksimenko:2016complexity,Maksimenko:2017,Maksimenko:2017LOP} "--- �~��������, ������������� � Scopus,
3~������~\citemy{Maksimenko:2011,Maksimenko:2014MAIS,Maksimenko:2016bool} "--- �~��������, �������� � RSCI, �~���� ����� �~����������~\citemy{BondBook:2008}.
%, ���� ����� �~������� �������~\citemy{BelovBM:2006}.
%{\color{red}2 ������ �~��������� ������ ����������� �~1 ������� ��������}.

���� ���������� ������������ �~����������� �~�.�.~�����������, �.~���������� �~�.~�������~\citemy{BogomolovFMP:2015}. ���������� ������ ��� �������� ����������� ���� ����������, ���������� � ����� �����������, ����������� ��������� ������� (������ ��������� ���� ������ �������� �� ����������). ������������ ������ � ����� � �������������� ������������� ���������� ������� $2 \lfloor\log_2 (n-1)\rfloor + 2$ ��� ������������ $n$"~��������� ����������� �����������; �������������� �������������� ���������� ����� ������� ��������� �~�.~�������; ����������� ��������� ��� ����������� ������������ ������������� ������� �.�.~�����������; ���������� ����������� ������������ ��� ������������ ����������� ������� ������������ � ���������� ������������ �~�.~����������. 


\textbf{���������� ���������.}
������������, ���������� �~�����������, ���� ���������� �������� ����~00-01-00662-a, 03-01-00822-a; 
��� �������� �~������-�������������� ����� ������������� ������ �� 2009--2013 ���� (���. �������� �~02.740.11.0207),
������������ <<���������� �~�������������� ���������>> ���� ��.~�.\,�.~��������
(����� ������������� �� �~11.G34.31.0057),
��������� �~477 �~�~984 �~������ ������� ����� ���. ������� �� ��� ���� (2014--2016~��.) �~���. �������� �~1.5768.2017/�220 �� ��� ����.


%\textbf{contrib}{������ ����� ������.}{%
%���������� ����������� �~�������� ���������, ��������� �� ������, �������� ������������ ����� ������ �~�������������� ������.
%���������� �~���������� ���������� ����������� ����������� ��������� �~����������, ������ ����� ����������� ��� ������������. 
%��� �������������� �~����������� ���������� �������� ����� �������.
%}

%%%%%%%%%%%%%%%%%%%%%%%%%%%%%%%%%%%%%%%%%%%%%%%%%%%%%%%%%%
%
%     ������� ���������� ������
%
%%%%%%%%%%%%%%%%%%%%%%%%%%%%%%%%%%%%%%%%%%%%%%%%%%%%%%%%%%

%\newpage % ��� ������������
%\section*{������� ���������� ������}
\nsection{������� ���������� ������}

%\textbf{�� ��������} ���������� ������������ ��������������� ������, �������������� ���� �~��������������� ������� ������� ������������, �������� ������������ ���������� ���������� �����������, ������������ ��������� �� ������ ������� ���������.

�~\textbf{������ �����} ������������� �~���������� ������� �������������� ������� �~�����, ������������ ����� �~�������� ����� �����������.
�~��������~1.1 �~1.2 �������� ����������� ������� 
������ ������ �~������ �������� ��������������, ��������������.
�~�������~1.3 ���������� ������������ ����� ������� ������ ��������� ����� �~���������� �, �~���������, ������ NP-������ �����. 

������������ ������������ ������ ������������� ����������� ���������� �~�������~1.4. ��� �� ���������� ����������� \emph{�������� ������ ������������� �����������}, �������������� ����� (� ������ ��������� �~������ ����������� ���������) ������ �������:
\begin{enumerate}
\item \emph{����������� ������} $d = d(I) \in \N$, ��� $I$ "--- \emph{��� ������}, ���������� ������ �� ������� ������.
\item ������� $k = k(I) \in \N$, �������� \emph{������������ �������} $U = \{0,\dots,k\}^d$. ��� ����������� ��������������� �~������ ����� $k\equiv 1$.
\item \emph{�������� ������������} $g = g(\bm{x}, I) \in \{\text{����}, \text{������}\}$, ��� $\bm{x} \in U$.
\end{enumerate}
��� �������������� ���� ������ $I$ ��� ������ ���������� ���������� \emph{��������� ���������� �������} $X = X(I) = \{u \in U \mid g(u, I)\}$.
�� ������� ������ ������ ����� ���������� \emph{������� ������} $\bm{c} \in \Z^d$.
���� ������ "--- ��� �������� ���� $I$ �~�������� ������� $\bm{c}$ ����� ����� ���� ���������� ������� $X$ �����, �� ������� ������� ������� $\langle\bm{c},\bm{x}\rangle$ ��������� ������������ ��������.
��������� ������� $\bm{x}$ ���������� \emph{�����������.}

%%%%%%%%%%%%%%%%%%%%%%%%%%%%%%%%%%%%%%%%%%%%%%%%%%%%%%%%%%
%     ����� 2
%%%%%%%%%%%%%%%%%%%%%%%%%%%%%%%%%%%%%%%%%%%%%%%%%%%%%%%%%%

�� \textbf{������ �����} �������� ����������� ��� ���� ��������������� ������ ������������ �~���������� ����� ��������� �� ���� ���� ������. 

�~�������~2.1
�������� ����������� ��������� �������������� �������� ������ ������������� �����������. �~������ ����� $I$ ��������������� ������ ����������� ������������ $\conv(X(I))$, �������������� ����� �������� �������� ��������� ���������� �������. ����� �������, �� ���� �������������� ������ ���� $I$, ���������� ��������� �������������� ������. ��������� �������������� ���������� \emph{�������������}, ���� ��� ��� �������, ������������ ��������������� ������, ������������� ���������. ����� ����� ��������������, ��� �������, ���������� ��������� ��� ������, �� ���� ��� V"~��������. 
��� ������, ����� �� ������� ������ $\bm{c}$ ������ ������������� �������������� �������� ����������� ���� $\langle\bm{c}, \bm{a_i}\rangle \le 0$, $i \in [k]$, ��������������� \emph{������� ������} ������������ ��� ����� ����������� ������������� $\conv(X(I))$ �~������ %<<������������>> ������� �������� 
$\cone\{\bm{a_1}, \dots, \bm{a_k}\}$.
�~���� �� ������� ���������� ����������� ���������, ����� ����������� �~���������� ������������� �������������� �~���������: ������ ������������� ������������� $\BQP(n)$, $n \in \N$; 
������������� ������ �~������� $\Knap(\bm{a},b)$, $\bm{a} \in \Z^n$, $b \in \Z$;
�������������� ����� $\Path(n)$ �~������� $\Dipath(n)$, $n\in \N$;
�������������� ������������� ������ $\TSP(n)$ �~������������� �������� $\ATSP(n)$;
���������������� ������������� $\Perm(n)$; ������������� ������ �~����������� $\Birk(n)$; ������������� $\Stable(G)$ ����������� �������� �~����� $G=(V,E)$;
�������� ������ �~���������� ������ �~������������ ����������������� ���� �������� $\ShortP(n)$; ��������� ������ �������������� �~���������.

�~�������~2.2 ����������� ������ ������������� ������ �������������� �����. �������� �������� ������� ������ ������������� ��������� ������. %, ��� ��� ��� ����� ����������� �~����������. 
����������� ��������� ���������� �� ���� ����.

������~2.3 �������� �������� ������ ��������� ������ ��� ����� ������������� ������ �������������� �����, ��� ����� ������, ������� �~�������� �����, �~�����, ��������� �~���������� ����� ����������������, �������� ����� �~������ ���������� ������� ����.

�~�������~2.4 �������� ������� ���������� ������������� (��������) �~���������� ������� ����� ��������� �� ���� ���� ������. \emph{�����������} ������������� (��������) $P \subseteq \R^d$ ���������� ������������ (�������) $Q \subseteq \R^n$ ������ �~�������� ������������ $\alpha \from \R^n \to \R^d$, ��������������� ������� $P = \alpha(Q)$.
����������, �~������ �������, ��������� ���, ��� ������ ����������� �� ������������� $P$ �������� �~������ ����������� �� ��� ���������� $Q$.
����� �������� ����������, ����������� ��� �������� ���������� $Q$, ���������� \emph{�������� ����������}. �������� �������, ����� ������ ���������� ����������� ����������� ������ ����� ����������, ����������� ��� �������� ��������� ������������� $P$. \emph{���������� ����������} $\xc(P)$ ������������� $P$ ���������� ����������� ������ ����� ���� ��� ����������. ������ ��������, ��� ��������� ���������� ���������� ����� ������������ �������������, �~������ "--- ������ ��� ������ �~������ �����������. ����� �������, ��������� ���������� ����� ������������� �~�������� ������� ������ ��������� ��������������� ��������������� ������. ����� ����, ��� �������������� �������� �~������� ����������� ����������. �~���������, �� �������� ���������� ����� ����� ������������� ��������������� ������������� "--- ������ �������������� �������� ������� ���������� ������"=����������� �������������~\cite{Yannakakis:1988}.

����� $M \in \{0,1\}^{n\times k}$~--- ������� ����������.
��������� $I\times J$, ��� $I\subseteq [n]$, $J\subseteq [k]$, ���������� \emph{0"~���������������} �~�������~$M$, ���� $M(i,j) = 0$ ��� ���� $i\in I$ �~$j\in J$.
\emph{������������� ���������} ������� $M$ ���������� ��������� 0"~���������������, ����������� ������� 
��������� �~���������� ����� �~$M$.
\emph{������ �������������� ��������} ������� ���������� ���������� ����� 0"~���������������, ����������� ��� � �������������� ��������.
����� �������������� �������� ������� ���������� ������"=�����������  ������������� $P$ ���������� $\rc(P)$.
�������������� ��������� ����� ������� �~���, ��� ��� �������������� ���� ������ ������ ������ ������ ��������� ��������������� ��������������� ������.

�~�������~2.5, �� ������ ������, ���������� �� ������ �����,
������������� ����� �������, ������ ������� �� ������� ��������� ����������� ����� �����������.
� ������, ��~������������� ������ �������, ��� ������������� NP"~������� ����� �� ������ ������� �������� ������� ����������. ��������: NP"~������� ������ ������������� ����������� ������, ��������� ������� �����, ������������������� �������� ����� �����, ������������������� ��������� ���������� �~����� �������������� �������� ������� ���������� ������"=�����������.
����� ��� �������� ����������� ������� ������� ��������������� ���������, ��������������� �~������ ����� ������� ���������������, ����� ������������� ����� ������ ������� ������������ ��� �������� ���������� ��������� ������ �������������� ������ ������.
�~����� �~���� ������������� �������� ��������� ������� ������ ���������. ����� �� �������������� ������������ ����� ������ ��������� ��� ��������� �������� ��������������? 
����� ������ �� ������ ��������� ������ ���� ����� ������� �~��������� ��������� ������������"=�������������� ������������� ��������������?
������ �� ��� ������� ���������� �~������ 3--5.
��� ������ ��~������������� �� ������ ����� ������������� ����� ����������� ������ ��������� ������.
���� �� ����� ����� ������ ��������������� ������������� �~���������� ��������������� ��������������� ������?
�~����� �~���� ��������� �~������� ����� ������ ���������.
����� ��������� �~��������� ����� ������������"=�������������� �������������� ������������� �������� ��������� �������� ��������� ��������������� ������?
���� �� ����� ����� ������������� ����� ������������� �~���������� ������ ����������� �� ���?
������������ ���� ������� ��������� ����� 7 �~8.


%%%%%%%%%%%%%%%%%%%%%%%%%%%%%%%%%%%%%%%%%%%%%%%%%%%%%%%%%%
%     ����� 3
%%%%%%%%%%%%%%%%%%%%%%%%%%%%%%%%%%%%%%%%%%%%%%%%%%%%%%%%%%

\textbf{�~������� �����} ������� ������ �������� ����������, ������� ������������ �~����������� ���� ������.

�~������ ������� �������� ����������� �������� ���������� �����, �������� ������� ��� ��������� ��� ����������� �~����������� ��������.
�������� ������ ������������� ����������� $(d,k,g)$ \emph{������� ��������} �~������ $(d',k',g')$, ���� ���������� ���������� �� �������������� (������������ ����� ������� ������ ������ ������) �����:
\begin{enumerate}
	\item 
	�������������� $\tau$ ������� ���� $I$ ������ ������ �~��� $I'$ ��� ������ ������: \[\tau \from I \mapsto I'.\]
	\item 
	�������� ���������� ��� ������� ���� $I$ ������ ������ ��������� ����������� 
	$\alpha\from \R^d \to \R^{d'}$, ��� $d = d(I)$, $d' = d'(\tau(I))$.
	\item 
	%���������
	������� $\beta\from Y \to X$, ��� $X = X(I)$ "--- ��������� ���������� ������� ������ ������, �~$Y$ "--- ��� ��������� ���� ����� ���������� ������� $\bm{y} \in X' = X'(\tau(I))$ ������ ������, ��� ������� ��~������� �������� ������� ������ $\bm{c} \in \R^d$ �����, ��� $\bm{y}$ �������� ����������� �������� ������ ������ �~������ $(\tau(I), \alpha(\bm{c}))$.
	%$\langle\alpha(\bm{c}), \bm{y}\rangle \ge \langle\alpha(\bm{c}), \bm{x'}\rangle$ ��� ���� $\bm{x'} \in X'$.
	%\[Y = \Set*{\bm{y} \in X'(I') \given \exists \bm{c} \in \R^d \ \bm{c'} = \alpha(\bm{c}), \ \forall \bm{x'} \in X'(I')\ \bm{c'}^T \bm{y} \ge \bm{c'}^T \bm{x'}}.\]
	������ ��� ������ $\bm{y} \in Y$ �~������ $\bm{c} \in \R^d$, $\bm{y}$ �������� ����������� �������� ������ ������ �~������ $(\tau(I), \alpha(\bm{c}))$ ����� �~������ �����, ����� $\beta(\bm{y})$ �������� ����������� �������� ������ ������ �~������ $(I,\bm{c})$.
%	\[
%	\Bigl(\forall \bm{x'} \in X' \quad \langle\alpha(\bm{c}), \bm{y}\rangle \ge \langle\alpha(\bm{c}), \bm{x'}\rangle\Bigr) \iff \Bigl(\forall \bm{x} \in X \quad \langle\bm{c}, \beta(\bm{y})\rangle  \ge \langle\bm{c}, \bm{x}\rangle\Bigr).
%	\] 
\end{enumerate}
��� ����������� �������� ������������������� ������� ����������� �������� ���������� ��~������������ ����������� ������~\citemy{MaksimenkoDiss:2004}.
�������� �������: ����������� ���������� ������������ ��������� ����������� $\alpha$ �~������� $\beta$; ������ ����������� ��������� ���������� ������� �� ������ ��������� ���� ������.


�~�������~3.2 ���������� 
%����������� ����������� ����� ������������� �~
����������� ��������� ��������� ������������ �������� ������ ������.
\emph{�������� ���������� ������������ �������� ������} ������ �������� ����������� �� ��������� $X \subset \R^d$ ���������� ������������ ������� ����:
\[
K(\bm{x}) = \Set*{\bm{c}\in \R^d \given  \langle\bm{c}, \bm{x}\rangle \ge \langle\bm{c}, \bm{y}\rangle, \ \forall \bm{y} \in X}, 
\]
��� $\bm{x} \in X$, ������ �~��������� ���������� ������ �� ������, ����������� ������� ����� ����������� ������������ $\R^d$.
������ $K(\bm{x})$ �~$K(\bm{y})$ ���������� \emph{��������}, ���� $\dim (K(\bm{x}) \cap K(\bm{y})) = d - 1$.
�������� ��������� �������� ������������ �~������������� $P=\conv(X)$ ������������~\cite{BondBook:1995}. �~���������, ������ ����� ������������� ������� 
������������� $P$ �~��� ������� ����� ������������� ������ ����� �~������ �����, ����� ��������������� ������ ������.
�� �������� �~�������� ���������� ����� ������������ ������������ ��������� ��������� �������� ������ $Q \subseteq \R^d$ (��������������, ��� $Q$ "--- �������) ������ �������� ����������� �� $X \subset \R^d$. ��� ������� ��~��������� $K(\bm{x},Q) = K(\bm{x}) \cap Q$, ����������� ������� ��������� �~������������ $Q$. ��� ����������� ������� �~��� �������, ����� �� ������� ������ ������������� �������� �����������. ��������, �~������������ ������ �~���������� ���� "--- ����������� ����������������� ���� ����� (���) �����.
�~����� ������� �������� ����������� �������� ���������� ��������� �������� ������ �����, ������������ �� ����������� �������� ���������� ����� ������������ ������������ ��������� ����������� $\alpha$ �~������� $\beta$.

�~�������~3.3 �������� ������������ ������ ��������� ��������������: ���� ������������ $P$ ������� ������������ �������������~$Q$ ��� �� ��� �����, ���������� ����������� $P \lea Q$. ���� �� ������������� $P$ �~$Q$ ������� ������������, ����� $P =_A Q$. ����������� $P \lea Q$ ��������� ���������� ��������� ������������"=�������������� �������������� ��������� �������������� $P$ �~$Q$: ����� ������ �~�����������, ��������� �������� ������ (��������, �������� �����), ����� ������������� �������� ������ ���������� ������"=����������� �~��������� ������.
����� �~���� ������� ���������� ��������� ������� �������� ������������� ����������� $\lea$. �~����� ������� ���� ����������� \emph{������������� ��������}, ��������������� ����� �������� �������� ��������� 
\begin{equation*}
\Pack(A) = \Set*{\bm{x}\in\{0,1\}^n \given A \bm{x} \le \bm{1}},
\quad \text{��� } A\in\{0,1\}^{m\times n},
\end{equation*}
� ������������� ���������
\begin{equation*}
\Part(A) = \Set*{\bm{x}\in\{0,1\}^n \given A \bm{x} = \bm{1}}.
\end{equation*}
��������������� ��~����������� ������� \(\Part(A) \lea \Pack(A)\).

�~��������� ������� ������� ����� �������� ����������� �������� ���������� �������� ��������������, �������� ����� ������������ �~��������� �����.
����� ����� ���� ������������� �~����������� ������������, �~������� �� ���������, ����� �������� \emph{��������} ������������� (�� ������ �~�������� ����������).
C�������� �������������� $P$ \emph{������� ��������} �~��������� �������������� $Q$, ���� �������� ������������� ���������� (������������ ������� ������������� $p\in P$):
\begin{enumerate}
	\item 
	�������������� $\tau$ ���� $I$ ������� ������������� $p = p(I)\in P$ �~��� $I'$ ������������� $q = q(I') \in Q$.
	\item 
	�������� ���������� ��� ������� ���� $I$ ��������� ����������� 
	\[
	\alpha\from \R^d \to \R^{d'}, \qquad d = d(I), \quad d' = d'(\tau(I)),
	\]
	������, ��� ������������ $\alpha(p)$ �������� ������ (�������� �������������) ������������� $q$ �~������� ������������ $p$.
\end{enumerate}
���� �������� ���������� $P$ �~$Q$ ���������� ���: $P \propto_A Q$.  

��������������� ��~����������� �~������������� ����� ������ ��������� ��������� �����������.
����� $P \propto_A Q$ �~� ��������� $P$ ���� �������������, ������� ���� ��� ��������� ��~��������� �������:
��������������������� ����� ������ ��� ����������� (������������ ������� �������������); ������������������� �������� ����� ����� �������������; NP-������� �������� ����������� ������; ������������������� ����� �������������� ��������; ������������������� ��������� ����������.
����� �~$Q$ ������� ������������� �~���� �� ����������.

�������� ���������� �������~3.4:
\begin{enumerate}
\item 
��������� �������������� ����������� ��������, �������������� �������� �~�������������� ��������� ������������ ������������ �������� ����������~\citemy{Maksimenko:2015DAN}.
\item
��� ������� $n\in \N$ ���������� ���� $G = (V,E)$, $|V| = n(n+1)$, $|E| = n(2n-1)$, �����, ��� $\BQP(n) \lea \Stable(G)$~\citemy{Maksimenko:2015DAN,Maksimenko:2016bool}.
���� �� ���� $G=(V,E)$ ��������, �� ����������� $\Stable(G) \lea \BQP(n)$ ���������� �� ��� ����� $n$.
\end{enumerate}
�~����� ������� ��������������� ����� ����� �������� ����������� �������� �������������� �~�������� ����������� �������� ��������� ����������� �������� ������ �����. 

%����������� �������� ���������� �������� ��������� ������������ �~������������ ����������� ������~\citemy{MaksimenkoDiss:2004}. ����������� �������� ���������� �������������� ������������ �~\citemy{Maksimenko:2017}.
%���������� ������� ����� ������������ �~\citemy{Maksimenko:2015DAN,Maksimenko:2016bool}.

%%%%%%%%%%%%%%%%%%%%%%%%%%%%%%%%%%%%%%%%%%%%%%%%%%%%%%%%%%
%     ����� 4
%%%%%%%%%%%%%%%%%%%%%%%%%%%%%%%%%%%%%%%%%%%%%%%%%%%%%%%%%%

�~\textbf{�����~4} ����������� ��� �����������, ��������� �~�������� �������� ����������. 

�� �������� �~��������������� �������� �~���������, �~�������~4.1 �������� ����������� �������������� �������� �~������� �������� ���������. 
\emph{�������������� ��������} ���������� �������� �������� ���������
\begin{equation*}
%\label{def:Cover}
\Cover(M) = \Set*{\bm{x}\in\{0,1\}^n \given M \bm{x} \ge \bm{1}},
\quad \text{��� } M\in\{0,1\}^{m\times n}.
\end{equation*}
\emph{�������������� ������� ��������} ���������� �������� �������� ���������
\begin{equation*}
%\label{def:DCP}
\DCP(B) =  \Set*{\bm{x}\in\{0,1\}^n \given B \bm{x} = \bm{2}},
\end{equation*}
��� $B \in \{0,1\}^{m\times n}$, ������ ������ ������ ������� $B$ �������� ����� ������ ������� �~�� ����� ������� ��������.
������� ��� ��������� �������������� ���� ����������� �����~\cite{Matsui:1995},
�� �� ���� ��������, ��� $\DCP(B) \lea \Cover(M)$, ��� ������� $M \in \{0,1\}^{4m\times n}$ �������� ����� ��� ������� �~������ ������.
������� �������� �~���� ������� ��������� ����������� �������������� $\NPadj(A)$, ��� ������� $A \in \{0,1\}^{m\times n}$ �������� ����� ��� ������� �~������ ������. ��� ������������� �������� ��������������� ������� ��������.
��������~\cite{Matsui:1995}, ��� ������ ������������� ����������� ������ ��� $\NPadj$ NP"~�����. ��������� ������������ ������� �������� ��� �����������, �������������� �~\citemy{Maksimenko:2017}:
\begin{enumerate}
	\item ������������� ����������� �������� $\Stable$ ������� �������� �~��������� �������������� $\NPadj$.
	\item ���� ������������ $\NPadj(A)$ �� �������� ��������, �� $\NPadj(A) \lea \Stable(G)$ ���������� �� ��� ������ ����� $G$.
\end{enumerate}
��������� �������� ������� �~����������� ����������� ������� �������������� ������� �������� �� �������������� ����������� �������� �~������� ���������� �~��� ��������.

{\sloppy
�����, �~�������~4.2 ��������������� ��������� �������������� �~NP-������ ��������� ����������� ������: ������������� ������ �~������� $\KnapEq(\bm{a},b)$, ������������� ������ �~��������� ����� $\PRT(\bm{a})$, ������������� ������ �~����������� �~������������ $\CAP(\bm{a},b)$, ������������� ������ �~������������ $\SAT(U,C)$, ������������� ������ �~��������� �������������� $\POP(n)$, ������������� ���������� ��������� $\Cubic(n)$.
���������� ������ �� ��������� �������~\cite{Fiorini:2003} �~���, ��� ��������� �������������� ������ �~3-������������ ������������ ��������� �������������� ������ �~��������� �������������� �~����� ������ �������� ���������� (������� �������� ���������� �~��� ������ �� ������������). �~��� �� ������ ��������, ��� ������������� ������ �~$k$"~������������ �� ����� ���� ������� ������� �~��������� �������������� ������ �� $m$"~������������, ���� $k > m$.
�������� ����������� ����� ������� �������� ����� ������������� ����, ��� ������������� ������� �������� $\DCP$ ������� �������� �~������������� ����������~\citemy{Maksimenko:2012DAN,Maksimenko:2013NP}.

}

�~�������~4.3 ��������������� ������������� �������� �������� �~������������� �������� �������� �~�����. ��������, ��� ������ ������������ ������������� $\BQP(n)$ ������� �������� �~������� ���������~\citemy{Maksimenko:2017LOP}, �~������������� ����������� �������� "--- �� �������.

�~�������~4.4 ��������������� ��������� ��������������, ������� ������� �������� ��������� ������. �~����� ������ �������� ���������� ��� ����������� �� ��� ������ ���������������.
������������� ������������� ������ �~����������� $\TAP(n)$ �~��������� �������� �������������� ��������� ����� ($\ColorA(G,k)$, $\ColorB(G)$ �~$\ColorC(G)$) ����� �~����� ������ ��������������� �~��������������� ����������� ��������. �~��������� �������������� ������������ ������ �������� �������������� $\QLOP(n)$ �~������������ ������ �~����������� $\QAP(n)$ ����������� ������������ ��������� ������� ������������ �������������� $\BQP(n)$.
���������� ����� ������� ������������ �~\citemy{Maksimenko:2016bool}.

�~�������~4.5 ��������������� ��������� �������������� �����, ����� ��������� �~������� ������������.
��������, ��� ������������� ������ �~������������ $\SAT(U,C)$ ������� �������� �~�������������� ������������� �������� $\ATSP(n)$~\citemy{Maksimenko:2011}.
����� ��~��������� ����� ����������� �������� ��, ��� ����� $d$-������ 0/1"~������������ �� $2^d - k$ �������� ($0 \le k \le 2^d - 1$) ������� ������������ ��������� ����� ������������� $\ATSP(n)$ ��� $n = (2k+1)d$.
����� ������� �~�������������~\cite{Billera:1996} �������� ��� ����������� ��� $n = (4k+1)d$ ����� ����������.
������ ��������� �������, �������������� �~\citemy{Maksimenko:2013TSP}, ������������� ��������� ����� ����� ����������� $\BQP$ �~$\ATSP$:
$\BQP(m) \lea \ATSP(n)$, ��� $n = 2 m^2 - m$.
�~����������~4.5.2 ��������������� ��������� �������������� ��������� �����: ����������� ����, ����������� (��)����, $s$"~$t$ (��)����, ����������� $s$"~$t$ (��)����.
��������, ��� ������������� ������������� �������� ������� �������� �� ���� ���� ����������. �� ����� �������, �~���������, ��� ����� �������������� ���� �������� �������� ������������������� �������� ������ �~������ ������������� ����������� ������ ��� ��� NP"~�����.

�� �������� �~�������� ������������� ��������������� �~�������~4.6 �������� �~������������ ������ ������������� $\BPP(n,p)$ ������� $p$. 
��� $p=2$, $\BPP(n,p)$ ��������� �~$\BQP(n)$, �~��� $p=1$, $\BPP(n,p)$ "--- $n$"~������ 0/1-���.
��������, ��� $\BPP(n,p)$ $s$"~���������� ���
$s \le p + \left\lfloor p / 2 \right\rfloor$.
��� $m \in \N$ �~$k \ge 2m$ ��������, ��� $\BPP(k,2m) \lea \BQP(n)$ ��� $n > 2 \binom{k}{m}$.
�������������, ��� ������ $k \in \N$ �~$n \ge 2^{2\cdot \lceil k/3\rceil}$, 
$\BQP(n)$ ����� $k$"~����������� ����� �� ������������������� ������
$2^{{\Theta}\left( n^{1 / {\left\lceil k/3\right\rceil}}\right)}$ ������.
�� ����� �~�� ������������� ����� �������� ����\'���� �������, ��� �� ���� ����������� ���� ���������� �������������� NP-������� ����� ������� �������������, ���������� $k$"~����������� ����� �� ������������������� (������������ ����������� �������������) ������ ������. ���������� ������� ������������ �~\citemy{Maksimenko:2013k}.

�~��������� ������� ����� 4 ��������������� ������ �~����������� �~������ �~���������� ������ �~������������ ����������������� ���� ��������. ��������, ��� �������� ��������� ��������� �������� ������ ��������� ������� �������� �~��������� ��������� ������������ �������� ������ ������~\citemy{MaksimenkoDiss:2004}.
��� ���������, ���� �������� ���������� ������� $\ShortP(n+1)$ �������� ��������� ����� ������������� �������� $\Birk(n)$, $n \in \N$.


%%%%%%%%%%%%%%%%%%%%%%%%%%%%%%%%%%%%%%%%%%%%%%%%%%%%%%%%%%
%     ����� 5
%%%%%%%%%%%%%%%%%%%%%%%%%%%%%%%%%%%%%%%%%%%%%%%%%%%%%%%%%%

�~������ ������� \textbf{�����~5} �������� ������� ����������� �������� ����������, ������������ �� �������� ���������� ����������� ����������� ������������ ��������� �����������. �������� ������� �~�������� ���������� �������������. ���� ��������� ����� ������������� $Q$ ��� �� ���� ���� ������������ �������� ����������� ������������� $P$, ����� ������������ ����������� $P \lee Q$.
��������� �������������� $P$ \emph{���������� ������� ��������} �~��������� �������������� $Q$, ���� �������� ������������� ���������� (������������ ������� ������������� $p\in P$):
\begin{enumerate}
\item 
�������������� $\tau$ ���� $I$ ������� ������������� $p = p(I)\in P$ �~��� $I'$ ������������� $q = q(I') \in Q$.
\item 
������� �������� ��������� $D\bm{y}=\bm{c}$, �������� �����
\(F = \Set{\bm{y}\in q \given D\bm{y}=\bm{c}}\)
������������� $q$.
\item 
�������� ���������� ��� ������� ���� $I$ ��������� ����������� 
\[
\beta\from \R^{d'} \to \R^{d}, \qquad d = d(I), \quad d' = d'(\tau(I)),
\]
������, ��� $p = \beta(F)$.
\end{enumerate}
�����������: $P \propto_E Q$.  
�� ������ ������� �������������� ����������� ���� $P \propto_E Q$ ������������� �����, ��� ����������� $P \propto_A Q$.
������� ������ ���������� ����������� �������� ������ ��������� �������� �������.
�~���������, ����� ��������, ��� NP"~������� �������� ����������� ������, ��������������������� ����� ����� �~��������������������� ��������� ����� �����, ������ ������, �� ����������� ��� ����������� �������� ����\'����.
����� �~���� �� ������� ���������� ��������� �������� ������ ��������� ��� ����� ���� ����������. ��������, ��������, ��� ���� ������������ $P \subseteq \R^d$ �������� ������� ������������� $Q \subseteq \R^n$ ��� �������� ����������� $\pi \from \R^n \to \R^d$ �, ����� ����, $\pi(\ext Q) = \ext P$, �� ���� ������������� $P$ �������� ��������� ����� ������������� $Q$.

�~�������~5.2 ���������� ��������� �������� ����������� �������� ����������. �~�����, ��������� ���������� �������, �������������� ��������������� ����������� ����������� ����������� ����� ��������, ��� ��� (�������) �������� ����������.

�~�������~5.3 ��������, ��� ����� ��������� ��������������, �������� ������������ �������� ����������� ������ NP, ���������� ������� �������� �~������� ������������ ��������������~\citemy{Maksimenko:2012Cook}.
��� �����, ��� ����������� ���� �~��������� ������ ��������� �������������� ����������� ������������ ���� ����� ������������ ����������� �������� ����������.


%%%%%%%%%%%%%%%%%%%%%%%%%%%%%%%%%%%%%%%%%%%%%%%%%%%%%%%%%%
%     ����� 6
%%%%%%%%%%%%%%%%%%%%%%%%%%%%%%%%%%%%%%%%%%%%%%%%%%%%%%%%%%

�~\textbf{�����~6} ��������������� ����������� �������������.
��� ��������~\cite{McMullen:1970}, ��� �������� ������������ ������ ������ (����� �����������) ����� ���� �������� �������������� ��� �� ����������� �~� ����� �� ������ ������.
��������� ����� �������������� ����������� ������������� �������� ������� ����������������� ����� ��� �������� ������� ���� ������������� �����������.
�~������ ������� ����� �������� ����������� ������������ ������������� �~������� �������� �����~\cite{Gale:1963}, ���������������� ������������ ������, ���������� ���������� ����� �������������.
�~�������~6.2 ��� ������������� $\CP_d([n])$ ���������� �������� ����������� ������������ ������� $2\bigl(2\lfloor \log_2(n-1)\rfloor+2\bigr)^{\lfloor d/2 \rfloor}$ ��� $2 \le d < n$ (��������� ����������� �~���������� ������~\citemy{BogomolovFMP:2015}).
�~�������~6.3 ����������� ������ �������� ��� �������� $\dc$ ����"=����� ������������ ������������� $\CP(d,n)$:
\(\dc= n-d  - 
\left\lceil 
\frac{n-2d}{ \left\lfloor \frac d 2 \right\rfloor+1}
\right\rceil\)
��� $n > 2d$~\citemy{Maksimenko:2009}. 
��������� $\dc= n-d$ ��� $d < n \le 2d$  ���� �������� ��� �~1964 ����~\cite{Klee:1964}.


%%%%%%%%%%%%%%%%%%%%%%%%%%%%%%%%%%%%%%%%%%%%%%%%%%%%%%%%%%
%     ����� 7
%%%%%%%%%%%%%%%%%%%%%%%%%%%%%%%%%%%%%%%%%%%%%%%%%%%%%%%%%%

\textbf{�����~7} ��������� ������ ���������� ������� ����
 ��� ������� �������� ����� ������������� �����������.
�~������ ������� ���������� �������� ���� ������, �������������� ��~\cite{BondBook:1995} (��. �����~\citemy{BondBook:2008}).
�������� ������������ ��������� ������� ���� �������� ��, ��� ��� ��������� ���������� ����� �������� ������ ����� ������������� (��������� ��������� ��������� �������� ������) �������� ������.
��������~(��. ����� �~�������~2.3.3), ��� ��� ������������ ������������� ���������� ����� (����������, ����������� �������� ������, ����������� ������) ��� �������������� �� ����������� ����������� �������������.
(�~�������~7.2 ��������, ��� ������ �~���������� ���� �~������������ ����������������� ���� �������� ���� ������ �~���� ������.)
�~������ �������, �~������~3 �~4 ��������, ��� ������ ������������ ������������� $\BQP$ ������� �������� �~�������������� ����� NP-������� �����, ��� �����������, ������, 3-������������, 3-���������, �������� �~�������� ���������, ��������� �����, ���������� ������� �~������ ������. ��������, ��� �������� ����� ����� ������������� $\BQP(n)$ ����� $2^n$, �������� ����� ������ �������������� ��������� ����� ����� ������������������ �� ����������� ��������������.
����� ����, �~\cite{BondBook:1995} �����������, ��� ��������� ��������� ����������, ������ �������� ��� ������������ ��������� ������,
�������� �������� ��� ����������� ����, �������� �����--����� 
� ���������� ��������� ������ �~������ ��� ������ ������������
�������� ������� ��� <<�������>>.

�~�������~7.2 ���������� �������������� �������� ��������� ��� ������ ������� ���� ������������� ���������� ��������� ������ � ���������� ����.
%�� ������ ����� �������� � �������������� ������� 2 ��~\cite{Bondarenko:1993SW3A} �������� ����� � ���, ��� �������� ����� ��� ������ �~���������� ���� �~������� �� $n$ �������� �~������������ ����������������� ���� �������� (� ����� ��� ������ �~������������ ������������ ����������������� ���� ���) ����� $\lfloor n^2 / 4\rfloor$.
�� ������ ����� �������� � �������������� ������� 2 ��~\cite{Bondarenko:1993SW3A} �������� ����� � ���, ��� �������� ����� ��� ���� ���� ��������� �����~$\lfloor n^2 / 4\rfloor$, ��� $n$ "--- ����� ������ �����, � ������� ������ ���������� ����.
�~������ ���������� �������~4.7, ��� ���� ������ ������ $\lfloor (n+1)^2 / 4\rfloor$ ��� ��������� ����� ����� ������������� ������ �~����������� $\Birk(n)$.
�~���� ����� ������� ��������� ����.
�~1977~�. �������� �~������ ��������~\cite[Theorem~6.1, Corollary~6.5]{Brualdi:1977II}, ��� ����� 2"~����������� ����� ������������� $\Birk(n)$, ����� ������ ������� �� ����� �����, �������� ����������, �~������������ ����� ������ ����� ����� ��������� �~���������� ���� ������� $\lfloor (n+1)^2 / 4\rfloor$.
���������� ����� ������� ������������ �~\citemy{MaksimenkoDiss:2004} �~\citemy{Maksimenko:2004}.

�~�������~7.3 ������������� ��� ������, ��������������� �������������� ������������ ����� ������� �~������ ��������� �����.
���������� �������������� ����, ��� �������� ����--�������� (���������� �����) ��� ������ �~����������� �� �������� ���������� ������� ����.
����� ����, ����������� ���������� ������������� ������ ����������� ����������,
����������� �� �������� �� ������������, �� �������������� ��������� �� ��~������ ���������� ������� ����. ���������� ������� ������������ �~\citemy{Maksimenko:2014MAIS}.

%%%%%%%%%%%%%%%%%%%%%%%%%%%%%%%%%%%%%%%%%%%%%%%%%%%%%%%%%%
%     ����� 8
%%%%%%%%%%%%%%%%%%%%%%%%%%%%%%%%%%%%%%%%%%%%%%%%%%%%%%%%%%

�~\textbf{�����~8} ��������� ��������� ������.
�����~��, ���� ������ ������������� �������� 
%(���������� ������������ �������� ���������� ���\-���-��\-���\-���\-���) 
�������������, �������� �������������� ������ �� ������������� ���������� (� ������ ����������� ������������� �~��������� �����)?
�~������ ����� �~�������� ����� �������� ������������� ��������� ���������������: ����� ������ �������������, ����� ��� �����������, ������� �~�������� ����� �����, ����� �������������� �������� ������� ���������� ������"=�����������.

�~�������~8.1 ���������� ������� �������� ��������������, ��� ������� �������� ���������� ���� ������������� (�� ����������� ����� �������������� ��������) ����������� ���������� �� ����������� ����������� � ���������� ������� ������������� � �������������� ��������� ��������������� ��������������� �����.

�~�������~8.2 ���������� �������� NP-������� ������ �����������, ������������� $\CBQP(n)$ ������� �������� �~���������� ���������� ��������� (�����������) ������ ������ ������������� ������������� $\BQP(n)$. ��������, ��� ������������ $\CBQP(n)$, $n \in \N$, ������������. �������������, ��������~\cite{FioriniKPT:13}, ����� �������������� �������� ��� ������� ���������� ������"=����������� ����� $O(n^5)$. 
����� �������, ��� ������ ��������� ������ ��������� �������������� NP"~������� ������, ����� �������������� �������� ��� ������� �������������.
��� ��������, ��� �� ���� ��~�������� �������������� NP"~������� �����, ������������� ����� �~������~3--5
�� ����� ���� ���������� ������� ������� �~��������� $\CBQP$.
�~������ �������, �~\cite[Theorem~4]{Braun:2015} �����������, ��� ����� ������������, ���������������� $\BQP(n)$ �~��������� $O(1/n)$, ����� ��������� ���������� ������� $2^{\Omega(n)}$.
�������������, ��������� ���������� ��� $\CBQP(n)$ ���������������: $\xc(\CBQP(n)) = 2^{\Omega(n)}$.
%��� ��������� ��������, ��� ������������� ������ �~�������������� �~������ ����� �������� ������������ ����������: �� ��������� ���������� ���������������, �~����� �������������� �������� �������������. 

�~�������~8.3 ���������� ������� ���� ����� ���������� �����������, ������������� ������� ������������ ������������ 
�~����� �������� ������ ��������� ������ ���� �������������� ���������. 
��� ���� ������ ������ ��������� ��~�������������� �����, 
�~������ ������ ����� ���������������� ���������.
(����� ������� ���������� �� ���� ���������������� ��������� ������������� ������� ������������� ��������� ������.)
���� ��������� ������� �~���, ��� �� ���� ����� ������������� �������������� ������������� �� ���� ����������� �������� ������������� ���������� ������ �� ����� �~���������������� ����������.

���������� ��������� ����� ������������ �~\citemy{Maksimenko:2016complexity}.

%�~\textbf{����������} ���������� ����� ���������������� ������������.

% Обзор литературы
%\input{review}

% Основная часть
%% Глава 1
\chapter{�������� �������}

� ���� ����� ���������� ����������� � ������������� ��������� �����, �������� ������� ��� ��������� ����������� ����������� � ����������� ������.
�~��������~\ref{sec:graphs} �~\ref{sec:polytopes} �������� ����������� ������� 
������ ������ � ������ �������� ��������������, ��������������.
�~�������~\ref{sec:complexity} ���������� ��������� ������� ������ ��������� ����� � ���������� �, � ���������, ������ NP-������ �����. ������������ ������������ ������ ������������� ����������� ���������� � �������~\ref{sec:CO}.
%������� ������������� (��������) ������ �������� � �������~\ref{sec:ProblemPolytopes}, ��� �� ���������� �������������� �������. ������~\ref{sec:Survey} �������� ������ ��������� ����������� �� ������ ����.
%� ���������, � ��� ������ ������������ ���������� � ��������� ������ �������������� �����, � ����� � ����������� ������������� ��������������.
%�~�������~\ref{sec:questions} ������������� ����� �������, ������ �� ������� ����� ������������ � ����������� ������.


\section{��������� � �����}
\label{sec:graphs}

\subsection{���������}

%��������� ����������� ����� ���������� ����� $\N$.
��� ��������� $\{1,2,\dots,n\}$, ��� $n\in \N$, ����� ������������ ������������~$[n]$. 
����� ����� %���������� �����, �� ������������� 
��������������� ����� $x$ ���������� $\lfloor x \rfloor$.
���������� �����, ������� ��� ������ ����� $x$, ���������� $\lceil x\rceil$.

����� $E$ "--- ��������� �������� ���������. 
��������� ���� ����������� ��������� $E$ ������������ $2^E$.
\emph{������������������ ��������} ������������ $T \in 2^E$ ���������� 0/1"~������ $v = \chi(T) \in \{0,1\}^E$ � ������������
\[
v_{e} = \begin{cases}
1,& \text{���� $e\in T$,}\\
0,& \text{���� $e\in E\setminus T$.}
\end{cases}
\]
���� �� �������� ��������� $E$ �������������, �� ������������������ ������ ����� ���� ��������� ��� $v = \chi(T) \in \{0,1\}^{|E|}$ � ������������
\[
v_{i} = \begin{cases}
1,& \text{���� $e_i\in T$,}\\
0,& \text{���� $e_i\in E\setminus T$.}
\end{cases}
\]
%����� �������, ������� ������ 0/1-������������� � $\R^d$ ����� ���������������� ��� ������������������ ������� ��������� ����������� $d$-����������� ��������� $S$.


\subsection{�����}
\emph{������} ��� \emph{����������������� ������} ���������� ������������� ���� $G = (V, E)$,
��� $V$ "--- �������� ���������, � $E$ "--- ��������� ��������� �������������� ����������� ��������� $V$.
�������� ��������� $V$ ���������� \emph{���������} ����� $G$, � �������� ��������� $E$ "--- ��� \emph{�������}.
������� $v$ � $u$ ���������� \emph{�������} ����� $\{v,u\}$.
��� ��������������� � ����������� ����� �� �������� ������� ����� (��������� $E$ �� �������� ���������� ���������) � ������ (����� ������ ����� $e \in E$ �������� ������� ���������).
���� ���������� \emph{������,} ���� ������ ���� ��� ������ �������� ����� ����� �����.
%������ ���� �� $n$ �������� ������������ $K_n$.

������� $v$ � $u$ ����� $G = (V,E)$ ���������� \emph{��������} � $G$, ���� $\{v,u\} \in E$.
���� �� $\{v,u\} \notin E$, �� ������� $v$ � $u$ ���������� \emph{����������.}
\emph{��������} ������� $v \in G$ ���������� ����� ������� � ��� ������.
����, ������� ������ ������� �������� ����� ����, ���������� \emph{����������.}
������������ ������ $V' \subseteq V$ ���������� \emph{������} � ����� $G$, ���� ����� ��� ������� �� $V'$ ������.
������������ ������ (��������) ����� � $G$ ���������� \emph{�������� ������} ����� $G$ � ������������ $\omega(G)$.
������������ ������ $V' \subseteq V$ ���������� \emph{�����������} � ����� $G$, ���� ����� ��� ������� �� $V'$ ��������.

����� $G' = (V', E')$ "--- ��� ���� ����.
����� $G$ � $G'$ ���������� �����������, ���� ���������� �������"=����������� ����������� $f \from V \to V'$ �����,
��� $\{v, u\} \in E$ ����� � ������ �����, ����� $\{f(v), f(u)\} \in E'$.
���� $G'$ ���������� \emph{���������} ����� $G$, ���� $V' \subseteq V$ � $E' \subseteq E$.
�����, ��� ���������, ����� ���� $G'$ ���������� �������� ����� $G$ ����� �������� ��������� ����� $G$.
������� $G'$ ����� $G$ ���������� \emph{�����������} ��� \emph{��������������}, ���� $\forall v, u \in V'$ �� $\{v, u\} \in E$ ������� $\{v, u\} \in E'$.

\emph{�����} � ����� $G$ ���������� ��������� ����� ���� 
\[P = \{\{v_1, v_2\}, \{v_2, v_3\}, \dots, \{v_{k-1}, v_k\}\},
\] 
��� $v_1$, $v_2$, \dots, $v_k$ "--- ������� ��������� �������, $k \ge 2$.
� ����� ������ ����� ��������, ��� ���� $P$ \emph{���������} ������� $v_1$ � $v_k$, � �������� ��� \emph{$v_1$-$v_k$ �����}.
���� ���������� \emph{��\'�����}, ���� ����� ��� ��� ������� ��������� ��������� ���� � ���� �����.
���� � ����� $G$ ���������� \emph{�������������}, ���� ������ ������� ����� ����������� ���� �� ������ ����� ����� ����.
%\emph{������} ���� ���������� ����� ������������ ��� �����.
\emph{�����������} ����� ��������� $v$ � $u$ � ����� $G$ ���������� ���������� ����� ����� � ����������� �� ����;
���� �� ����� ���� �� ����������, �� ���������� ���������� ������~$+\infty$.
\emph{���������} $\diam(G)$ ����� $G$ ���������� ���������� ���������� ����� ��� ��������� (����� �������������� ����� ���� ��� ������).

\emph{������} � ����� $G$ ���������� ��������� ����� ���� 
\[
C = \{\{v_1, v_2\}, \{v_2, v_3\}, \dots, \{v_{k-1}, v_k\}, \{v_k, v_1\}\},
\] 
��� $v_1$, $v_2$, \dots, $v_k$ "--- ������� ��������� �������, $k \ge 3$.
���� � ����� ���������� \emph{�������������}, ���� ������ ������� ����� ����������� ����� ���� ������ ����� �����.
���� � ����� ���� ����������� ����, �� � ��� ���� ���������� \emph{�������������.}
���� ��� ������ ���������� \emph{�����}, � ������� ��� "--- \emph{�������}.

������� $v$ � ����� $e$ ���������� \emph{������������}, ���� $v\in e$.
\emph{�������� ����������} ������"=����� ����� $G = (V,E)$ ���������� ������� $M \in \{0,1\}^{n\times k}$, $n = |V|$, $k = |E|$, �������� ������� ������������ ��������� �������:
\[
M_{ij} = 
\begin{cases}
1, &\text{���� $v_i\in e_j$,}\\
0, &\text{�����.}
\end{cases}
\]

����� ��������, ��� ������������ ����� $E' \subseteq E$ \emph{���������} ������� $V$,
���� ������ ������� $v \in V$ ���������� ���� �� ������ ����� �� $E'$.
����������, ������������ ������ $V' \subseteq V$ \emph{���������} ����� $E$,
���� ������ ����� $e \in E$ ���������� ���� �� ����� ������� �� $V'$.

��� ����� � ����� ���������� \emph{��������}, ���� ��� �������� ����� �������, � ��������� ������ ��� ���������� \emph{����������.}
��������� ������� ��������� ����� ����� ���������� \emph{��������������}. 
�������������, ����������� ��� ������� �����, ���������� \emph{�����������.} 
����� �������, ����������� ������������� ����� ���� ������ � ������ � ������ ������ ������.

\emph{��������} � ����� $G = (V, E)$ ���������� ��������� ����� ���� 
\[
\delta(U) \coloneqq \Set*{\{u,v\} \in E \given u\in U,\ v\in V \setminus U}, \quad \text{��� } U \subseteq V. 
\]
�� ����������� �������, ��� $\delta(U) = \delta(V \setminus U)$.
������ $\delta(U)$ ���������� \emph{$s$-$t$ ��������}, ���� $s \in U$ � $t \in V\setminus U$\label{def:stcut}.
%\emph{���������� ���������} $X$ ���������� ����� ��� ������� ���������������� �����������, ����������� ������� ��������� � $X$.
���� $G = (V, E)$ ���������� \emph{����������}, ���� ��������� ��� ������ $V$ ����� ������� �� ��� \emph{����} $U$ � $V \setminus U$ ���, ��� $\delta(U) = E$. 
���������� ���� ���������� \emph{������ ����������}, ���� $\{u, v\} \in E$ ��� ����� $u \in U$ � $v \in V \setminus U$.


���� $G = (V,E)$ ���������� \emph{�������"=����������}, ���� �� ��������� ��� ����� $E$ ������ ������� ����� $f \from E \to \R$.
����� $f(e)$ ���������� \emph{�����} ����� $e \in E$.
\emph{����� ������������} $E' \subseteq E$ ��� \emph{��������} $G' = (V', E')$ �������"=����������� ����� $G$ ���������� ����� ����� �������� � ���� �����.
���� $G$ ���������� \emph{��������"=����������}, ���� ������ ������� $g \from V \to \R$.
� ����� ������ ����� $g(v)$ ���������� \emph{�����} ������� $v \in V$, � \emph{����� ������������} $V' \subseteq V$ ���������� ����� ����� �������� � ���� ������.

\subsection{�������}

\emph{��������������� ������} ��� \emph{��������} ���������� ������������� ���� $D = (V, A)$, ��� $V$ "--- �������� ���������, ���������� \emph{���������� ������}, $A$ "--- ��������� ��������� ������������� ��� ������, ���������� \emph{������}. 
�����, ��� � ��� ������, ����� ������������, ��� � $A$ ��� ������� ��� � ������.
����������� ������������� ���� ������� ��� ������ ����������� (������ � ���������� �����������) �� �������.

����� $(v, u) \in A$.
������� $v$ ���������� \emph{�������} ���� $(v, u)$, � ������� $u$ "--- �� \emph{������}.
������ �� ���� ������ � ���� $(v, u)$ ���������� \emph{������������} ���� �����.
������ ����� �������� \emph{������,} ���� ������ ������������� ���� ��� ������ �������� ���� ����� �����, �� ���� $|A| = |V| (|V| - 1)$.
%������ ���������� \emph{��������,} ���� ��� ����� (���������������) ���� ������ $u,v \in V$ ����� ���� �� ��� $(u, v)$ � $(v, u)$ ����������� $A$.

\emph{�������} ��� ������ \emph{�����} � ������� $D$ ����� �������� ��������� ��� ���� 
\[P = \{(v_1, v_2), (v_2, v_3), \dots, (v_{k-1}, v_k)\},
\] 
��� $v_1$, $v_2$, \dots, $v_k$ "--- ������� ��������� �������, $k \ge 2$.
������� $v_1$ � $v_k$ ����������, ��������������, \emph{�������} � \emph{������} ���� $P$, 
� ������� $v_2$, \dots, $v_k$ "--- ��� \emph{����������� ���������}.

\emph{��������} � ������� $D$ ���������� ��������� ��� ���� 
\[
C = \{(v_1, v_2), (v_2, v_3), \dots, (v_{k-1}, v_k), (v_k, v_1)\},
\] 
��� $v_1$, $v_2$, \dots, $v_k$ "--- ������� ��������� �������, $k \ge 2$.
������, �� ���������� ��������, ���������� \emph{������������.}

\emph{��������} � ������� $D = (V, A)$ ���������� ��������� ��� ���� 
\[
\delta^+(U) \coloneqq \Set*{(u,v) \in A \given u\in U,\ v\in V \setminus U}, \quad \text{��� } U \subseteq V. 
\]
������ $\delta^+(U)$ ���������� \emph{$s$-$t$ ��������}, ���� $s \in U$ � $t \in V\setminus U$\label{def:stdicut}.


������ $D = (V, A)$ ���������� \emph{������������}, ���� �� ������� $(v, u) \in A$ � $(u, w) \in A$ ������� $(v, w) \in A$.
��� ��� �� �� ������������� ����� � �������, �� �� �������������� ������� ������������.
����� �������, ��������� ��� ������������� ����� ������ ��������� ������� �� ��������� ������ ����� �, ��������, ������ ��������� ������� ����� ���� ����������� ���������� ��� ���������� ������������� �����.
���� ��� ������ ���� ������ $u,v \in V$ � ��������� $A$ ������ ����� ���� �� ���� ��� $(v, u)$ � $(u, v)$, �� ��������������� ������ ���������� \emph{��������}.\label{def:linearOrdering}
������������ ������ ������ �������� ������� �� ��������� ������.
������� �����, ������ � ��������� (��������) ������� �� ����� ����� ������������� ��������������� ������������ ������ (������).


\section{�������������}
\label{sec:polytopes}

%� ���� ������� ������������� ��������� .
��� ��������� ���������������� ������� � ������ ������ �������� �������������� ����� �������������� ������������ ����������~\cite{Emelichev:1981} �~\cite{ZieglerBook}.


��� $\R^d$ ����� �������� ������������ ���� ������"=�������� ����� $d$ � ������������� ������������. 
���� ������-������� �� $\R^d$ ����� �������� ���������� �������: $\bm{x}, \bm{x_1}, \bm{y}, \bm{z} \in \R^d$.
������"=�������, ������������ �� ����� ����� ��� �� ����� ������, ����� ���������� $\bm{0}$ � $\bm{1}$ ��������������
(�� ����������� ���� �� ��������� �����, ��� ��� ����� ��������������).
��������� �������, ���������� ������������ ����� � $\R^d$, ���������� $\bm{e_1}$, \dots, $\bm{e_d}$
(����� �������, $\sum_{i\in[d]} \bm{e_i} = \bm{1}$).
������ ������������ ��������~\cite{ZieglerBook},  ������"=������� ����� ����� ���������� �������.


\emph{���������������} �~$\R^d$ ���������� ���������
\[
H(\bm{a},b) \coloneqq \Set*{\bm{x}\in\R^d \given \bm{a}^T \bm{x} = b}, 
%\qquad \bm{a}\in\R^d, \ \bm{a} \ne \bm{0}, \ b\in\R,
\]
��� $\bm{a} \in \R^d$ "--- ������ ������� ��������������, $\bm{a} \ne \bm{0}$, � ����� $b \in \R$ ���������� �������� �������� �������������� ������������ ������ ���������,
$\bm{a}^T \bm{x}$ "--- ��������� ������������ ������"=������ $\bm{a}^T$ �� ������"=������� $\bm{x}$, ���, ������� �������, ��������� ������������ $\langle\bm{a},\bm{x}\rangle$.


�������� ���������� $\sum_{i\in[n]} \lambda_i \bm{x_i}$ ����� $\bm{x_1}$, \dots, $\bm{x_n}$ �� $\R^d$,
��� $\lambda_i \in \R$, $i\in[n]$,
���������� \emph{�������� �����������},
���� $\sum_{i\in[n]} \lambda_i = 1$.
\emph{�������� ���������} $\aff(X)$ ��������� ��������� $X \subseteq \R^d$ ���������� ��������� ���� �������� ���������� ������� ������ ����� �� $X$.
��������� ����� ���������� \emph{������� �����������}, ���� �� ���� ����� ����� ��������� �� ����������� �������� �������� ��������� ��� �����.
\emph{�������� ������������} ��������� $X$ ���������� �������� ������� ������������ ������������ $S \subseteq X$ ����� ����, ��� ������� $\aff(S) = \aff(X)$. � ���������, �������� ����������� ������� ��������� ����� $-1$.
\emph{�����������} $\dim(X)$ ��������� $X$ �������� ������ ��� �������� �����������. (��� ��� ����� ��������������� ������ �������� ��������� �, � ���������, �������� �������� �������� ��������, �� ������������ � ������� ������������� ������������� ����������� �� ���������.)
��������� $X \in \R^d$ ���������� \emph{�������� ����������������}, ���� ������ � ������ ������ ����� ���������� ������� ��� �������� ��� �� �������� ����������.
�������������� �������� �������� ��������� ���������������.
����� ����, ������ �������� ��������������� ����������� $d-k$ � $\R^d$ ����� ���� ������������ ��� ����������� $k$ ���������������~\cite{Emelichev:1981}.

���������� $\sum_{i\in[n]} \lambda_i \bm{x_i}$ ����� $\bm{x_1}$, \dots, $\bm{x_n}$ �� $\R^d$,
��� $\lambda_i \ge 0$, $i\in[n]$,
���������� \emph{���������� �����������}.
\emph{���������� ���������} ��������� ��������� $X = \{\bm{x_1}, \dots, \bm{x_n}\} \subset \R^d$ ���������� ��������� ���� ���������� ���������� ��� �����:
\[
\cone(X) \coloneqq \Set*{\sum_{i=1}^n \lambda_i \bm{x_i} \given \lambda_i \ge 0}.
\]
�������� ����� ������� ������������� ������
\[
\R^d_+ \coloneqq \Set*{\bm{x} \in \R^d \given \bm{x} \ge \bm{0}} = \cone\{\bm{e_1},\dots,\bm{e_d}\}.
\]

�������� ���������� $\sum_{i\in[n]} \lambda_i \bm{x_i}$
����� $\bm{x_1}$, \dots, $\bm{x_n}$ �� $\R^d$ ���������� \emph{��������}, ���� $\lambda_i \ge 0$, $i\in[n]$.
��������� $X \subseteq \R^d$ ���������� \emph{��������}, ���� ��� ����� ���� ����� $\bm{x}, \bm{y} \in X$ 
��� �������� ��� �� �������� ����������.
%��� ��������� �������� ����������� �� ������� \([\bm{x}, \bm{y}] = \{\lambda \bm{x} + (1 - \lambda) \bm{y} \mid \lambda \in [0, 1]\}\).
������� �������� ��������� ��������� ����� ������� \emph{��������� ����������������} 
\[
H^+(\bm{a},b) \coloneqq \Set*{\bm{x}\in\R^d \given \bm{a}^T \bm{x} \ge b},
\]
������������ ��������������� $H(\bm{a},b)$.
����� ��������� ��������� ���������� \emph{�������}, ���� ��� �� �������� �������� ����������� ������� ���� ������ ����� ����� ���������.
��������� ���� ������� ����� ��������� $X$ ������������ $\ext(X)$.
\emph{�������� ���������} ��������� ��������� $X = \{\bm{x_1}, \dots, \bm{x_n}\}
\subset \R^d$ ���������� ��������� ���� �������� ���������� ��� �����:
\[
\conv(X) \coloneqq \Set*{ \sum_{i=1}^n \lambda_i \bm{x_i} \given \sum_{i=1}^n \lambda_i = 1, \ \lambda_i \ge 0, \ \lambda_i \in \R}.
\]
�������� �������� (�������������) ��������� $X \subseteq \R^d$ ������������ ����� ����������� �������� �������� ���� �������� ������� ����� �� $X$. 
%�������� ������� �����������~\cite{Caratheodory:1911}, ��� ������� $X \subseteq \R^d$,
%\[
%\conv(X) = \left\{\sum_{i=1}^n \lambda_i \bm{x_i} \;\bigg|\; 
%\{\bm{x_1}, \dots, \bm{x_n}\} \subseteq X, \ n \le d+1, \  \sum_{i=1}^n \lambda_i = 1, \ \lambda_i \ge 0\right\}.\]

� �������� �������� �������� ����� ������� ������� \emph{����� �����������} ���� �������� �������� $X\subseteq \R^d$ � $Y\subseteq \R^d$:
\[
X+Y \coloneqq \Set*{x+y \given x \in X, \ y\in Y}.
\]
� ���������, ��� ����� $X$ � $Y$
\[
\conv(X+Y) = \conv(X) + \conv(Y).
\]

\emph{�������� ��������������} ���������� �������� �������� ��������� ��������� �����. %� ��������� ������������ $\R^d$.
��� ��� ����� ���� ������ ������ � �������� ��������������, ����� �������� ����� ����������.
\emph{���������} ���������� ����������� ��������� ����� ��������� ���������������, ���, ������� �������, ��������� ������� ������� �������� ����������
\(A\bm{x} \ge \bm{b}\), ��� $A \in \R^{m\times d}$, $\bm{x}\in \R^d$, $\bm{b}\in \R^m$.

\begin{theorem}[�����--����������]
	\sloppy
	��������� $P$ �������� �������������� ����� � ������ �����, ����� $P$ "--- ������������ �������.
\end{theorem}

����� �������, ������ ������������ ����� ���� ������ ����� ��������������� ���������:
\begin{enumerate}
	\item ��� �������� �������� ��������� ����� ������. � ���� ������ ��������� ������ ���������� \emph{$V$-���������} �������������.
	\item ��� ����������� ��������� ����� ��������� ���������������. ����� ��������������� ������� �������� ���������� (�, ��������, ���������) ���������� ��� \emph{$H$-���������.}
\end{enumerate}
�� �� ����� ����� � � ��������� �������� ��� ��������� ���������~\cite{ZieglerBook}. \emph{$V$-��������� ��������} $P$ ���������� ������������ ��������� ��������� �����~$X$ �~��������� ��������� �������� $Y$ �����, ���
\[
P = \conv(X) + \cone(Y).
\]

$H$-�������� ����� ���������� \emph{��������}, \emph{��������} ��� \emph{�������} ���������~\cite{Schrijver:1998, Zolotykh:2012}.
� ���� �������, $V$-�������� �������� \emph{���������} ��� \emph{����������} ���������.
������ ���������� $H$-�������� ������������� (��������) �� ��� $V$-�������� ���������� \emph{������� ���������� �������� ��������}.
��� ������������ (�����������) ������ ���������� $V$-�������� �� $H$-�������� � ������� ��� ��� ������������ ����� ��������� \emph{������ ���������� ������������� �������� �������������}.
��� ������ �������� �������������� �������~\cite{Khachiyan:2008}.
������������� ����� ��������� � ��������� ����� �������� � ������� ����� ����� � �����������~\cite{BastrakovDiss:2016}.

%\emph{������������} $\dim(P)$ ������������� $P$ ���������� ����������� ������������ ����������� ��� ��������� ���������������.
������ ���������~\cite{Grunbaum:2003, Emelichev:1981, ZieglerBook}, 
�������� ������������ ����������� $d$ ����� �������� \emph{$d$"~��������������.}
���������� �������� $d$-������������� �������� \emph{$d$-��������}, �������������� ����� �������� �������� $d+1$ ������� ����������� ����� � $\R^n$, $n\ge d$.

����� ��������, ��� ����������� $\bm{a}^T \bm{x} \ge b$ \emph{���������} ��� ��������� $X \subseteq \R^d$, 
���� ��� ��������� ��� ���� $\bm{x}\in X$.
\emph{������} ������������� $P$ ���������� ����� ��������� ���� 
\[
F = \{\bm{x} \in P \mid \bm{a}^T \bm{x} = b\},
\]
��� ����������� $\bm{a}^T \bm{x} \ge b$ ��������� ��� $P$.
�� ����, ��� $\bm{a}$ ����� ���� ������~$\bm{0}$, �������, ��� ������ ��������� � ��� ������������ $P$ �������� �������~$P$, ��� ���������� \emph{��������������} ������� $P$.
��������� ����� ���������� \emph{������������}~\cite{Emelichev:1981}.

���� �������������� $H(\bm{a},b)$ ����� ���� �� ���� ����� ����� � �������������� $P \subseteq \R^d$ 
� ��� ���� $P$ ������� ����� � ���������������� $H^+(\bm{a},b)$, �� �������������� $H(\bm{a},b)$ � ���������������� $H^+(\bm{a},b)$ ���������� \emph{��������} �~$P$.
����� �������, ������ ����������� ����� ������������� ���� ����������� ������������� � ��������� ��� ������� ���������������.

\emph{������������} $\dim(F)$ ����� $F$ ���������� ����������� ������������ ����������� � ��������� ���������������.
����� ����������� $k$ ���������� \emph{$k$-�������}, 0-�����~--- \emph{���������} �������������, 1-�����~--- ��� \emph{�������}.
�������� ������������, ��� ��������� ������� ����� ������������� ��������� � ���������� ��� ������.
$(d-1)$-����� $d$-������������� ���������� \emph{������������}. 
��� $(d-2)$-������ � ������������� ����������� ��� ����������� �������, 
�� �������� ������������~\cite{ZieglerBook}, ��������~\cite{Bastrakov:2011}, �������~\cite{Deza:2001} (�� ����. ridge).
�� ����� �������������� �������� \emph{����}.

� ��������, ����� ������ � ����������� $d$-��������� ����� $d+1$, � ����� ��� $k$-������, $k \in [d-2]$, ����� $\binom{d+1}{k+1}$. 
����� �������, ����� ���� ������ $d$-��������� ����� $2^{d+1}$.
%(���� ������������ ���� ��� ����������� � ����������.)

\emph{������} ��� \emph{1-��������} ������������� ���������� ��������� ��� ������ � ����� (������, ��� ������, ���������� ����� �������������).
\label{ridge-graph}
\emph{����"=������} ������������� ����� �������� ��������� ��� ����������� � ������ (������, ��� �����������, ���������� �����).

��� ������ ������������� ����������� ��������� �����������.

\begin{prop}[\cite{ZieglerBook}]
	����� $P$ "--- ������������, � $V = \ext(P)$ "---  ��������� ��� ������. �����:
	\begin{enumerate}
		\item $P = \conv(V)$. %(������������ �������� �������� ��������� ����� ������).
		\item ���� $F$~--- ��������� ����� ������������� $P$, �� $F$~--- ���� ������������ � $\ext(F) = F \cap V$.
		\item ����� ����������� ������ ������������� $P$~--- ����� ����� $P$.
		\item ����� ����� ������������� ����� �������� ��� ������.
	\end{enumerate}	
\end{prop}

������������ ���������� \emph{��������������}, ���� ��� ��� ���������� �������� �����������.
������������ ���������� \emph{�������}, ���� ������ ��� ������� ����������� ����� $d$ �����������, ��� $d$ "--- ����������� �������������.
C�������� � �������� �������������� �������� ��������� ������� �, ������������, �������������� ��������������. 
�������� �������� ������������� �������� $d$-������ \emph{0/1-��������} (���, ������, \emph{$d$-���})~\cite{ZieglerBook}:
\[
\Cube_d \coloneqq \Set*{\bm{x} \in\R^d \given \bm{0} \le \bm{x} \le \bm{1}} 
= \conv\left\{\{0,1\}^d\right\}.
\]
�������� ��������������� ������������� ����� ������� $d$-������ \emph{������������}, � ������������� ���������� ���������� \emph{���������}~\cite{ZieglerBook}:
\[
\Cross_d \coloneqq \Bigl\{\bm{x} = (x_1,\dots,x_d)\in\R^d \Bigm| \sum_{i \in [d]} |x_i| \le 1\Bigr\} 
= \conv\left\{\bm{e_1},-\bm{e_1},\dots,\bm{e_d},-\bm{e_d}\right\}.
\]

������, ��� ���������� �������� ����������� ��������� $\{0, 1\}$, ���������� \emph{0/1"~��������}.
������������, ��� ������� �������� �������� 0/1"~���������, ���������� \emph{0/1"~��������������}.
������� �������, 0/1"~������������ ������������ �����
�������� �������� ���������� ������������ ������ ���� $\Cube_d$.
������� �����, ��� $\ext\conv(X) = X$ ��� ������ $X \subseteq \{0, 1\}^d$.

����� ��������, ��� ��������� �� $n \ge d+1$ ����� � $\R^d$ ��������� \emph{� ����� ���������}, ���� ������� $d+1$ �� ��� �� ����� � ����� ��������������~\cite{ZieglerBook}.
������� �������� ���������� 
\(A\bm{x} \ge \bm{b}\), ��� $A \in \R^{m\times d}$, $\bm{x}\in \R^d$, $\bm{b}\in \R^m$, $m \ge d+1$, ���������� \emph{�����}, ���� ��� ������ $\bm{x}\in \R^d$ ������������ ���������� � ��������� �� ����� ��� $d$ �� ���� ����������.

������ �������������� � ������� �������������� ����� � ��������� ������~\cite{ZieglerBook}.
�������� �������� ��������� �����, ����������� � ����� ���������, �������� �������������� ��������������.
����������, ������� �������� ���������� ������ ����, 
��������� ������� ������� ����������, ���������� ������� ������������.
��������������, ����� ������������ ����� ���� ������������ � �������������� �� ���� ���������� ��������� (�����������) ��� ������.
����������, ����� ������������ ������� ����� ���� ������������ � ������� ������������ �� ���� ���������� ��������� (�����������) ������������� ������� ����������� ��� �������� ����������.

������������ ���������� \emph{$k$-�����������} ($k\in \N$), ���� ����� $k$ ��� ������ �������� ���������� ������ ��������� ����������� ����� ����� �������������.
� ���������, ����� ������������ �������� 1-�����������, � $d$"~�������� �������� $k$-����������� ��� $k \in [d]$.

�������� ���������� ��������� $k$-����������� ��������������, ������������ �� ���������, �������� ����������� �������������~\cite{Grunbaum:2003,Emelichev:1981,ZieglerBook}.
��������� ������ \emph{������������ �������������} ������������ ��������� �������:
\label{page:cyclic}
\[
\CP_d(T) \coloneqq \Set*{(t, t^2, \dots, t^d) \in \R^d \given t \in T},
\]
��� ��������� $T \subset \R$ "--- �������.
(�������, ��� $\CP_d(T)$ �������� ���������� ��� $|T| \le d+1$.)
��������, ��� ��� ������������� �������������, $\lfloor d/2\rfloor$"=����������
(�� ���� ����� ������������ ������� ��������� ����� $d$"~��������������, �� ���������� �����������)
� �������� ���������� ������ ������ (������ �����������) ����� ���� $d$"~�������������� � ��� �� ������ ������ $n = |T|$.

���� �� ����� ������������ ��� $\CP_d(T)$ ������� ���� ����������� � ��� �������, ����� ��������� $T$ ����� ����������� ���.
��������, ���� $T$ "--- ��������� ����� ����� ������� $[a,b]$, 
�� $\CP_d(T)$ ����� ���������� ����� ������������� $\CP_d(a, b)$.

����� �� ����� ������� � ������ � ��� ������ �������� �������� ��� ��������������� �������� �������� ����������� ��������.
����� $P$ "--- ��������� $d$-������������, ��������� � ������������ $\R^n$, $n > d$,
�~����� ����� $\bm{x} \in \R^n$ ��~����������� �������� �������� ����� �������������.
\emph{���������} ��� $P$ ���������� �������� �������� $\conv\{P\cup \bm{x}\}$.
������������ $P$ ���������� \emph{����������} �������� $\conv\{P\cup \bm{x}\}$, � ����� $\bm{x}$ "--- �� \emph{�������} (��� \emph{�������� ��������}).
������� �������� �������� ��� ����� �� ���������, � ����� ��� �������� ��� ��� �������.
� ���������, ����� ����������� �������� ����� �� ������� ������ ����� ����������� ���������, �� �� ����� � ��� ����� ������.
�������� ��� �������������� �������������� ����� �������� �������������� ��������������.
�������� ��� $k$-����������� �������������� ���� $k$-����������.


\subsection{������� ������}

\emph{�������� ������} ������������� $P$ ���������� ��������� $L(P)$ ���� ��� ������, 
�������� ������������� �� ���������.

� �������� ������� ���������� 4-������ �������� ������������ $P$, ���������� ���������, ��������� ������� "--- 
���������� ��� ��� ����� ������� (��. ���.~\ref{fig:cube7}).
$P$ ����� 8 ������, 8 ����������� (���� �� ��� ���������� �� ���.~\ref{fig:cube7}), 19 ����� � 19 ������.
\begin{figure}[hb]%
	\centering
	%\includegraphics[width=\columnwidth]{filename}%
	\begin{tikzpicture}[scale=2.0, line join = round]
	\coordinate (4) at (0,0,1);
	\coordinate (6) at (1,0,0);
	\coordinate (7) at (0,0,0);
	\coordinate (2) at (1,0,1);
	\coordinate (1) at (0,1,1);
	\coordinate (3) at (1,1,0);
	\coordinate (5) at (0,1,0);
	\draw (6) -- (3) -- (2) -- (1) -- (4) (5) -- (3) -- (1) -- (5) (6) -- (2) -- (4);
	\draw[dashed, thin] (6) -- (7) -- (5) (7) -- (4);
	\foreach \i in {1,...,7} {\draw (\i) node[circle, draw, fill = white, inner sep = 1pt] {\i};}
	\end{tikzpicture}
	\caption{���������� ��� ��� ����� �������}%
	\label{fig:cube7}%
\end{figure}

������� ������ ������ ��������������� ����������� ��������� �����, 
�������������� ����� ������������ �� ��������� ����,
������� �������� ������������� ����� �������������.
������� �� ��������� ����� ��������� �� � ������ �� ���� ������ $f$ � $g$,
��� ������� ������������ ����������� ��������� �������:
\begin{compactenum}
	\item[1)] $f$ �������� ������ $g$ (� ���� ������ $g$ ����������� �� ��������� ���� $f$);
	\item[2)] �� ���������� ����� $h$, ������������ �� $f$ � $g$, � �����, ��� $f$~--- ����� $h$ � $h$~--- ����� $g$.
\end{compactenum}
������� ������ �������� ��� ���������� ����� ��� ������� ���������� �� ���.~\ref{fig:cube7Hasse}.

\begin{figure}%
	\centering
	%\includegraphics[width=\columnwidth]{filename}%
	\begin{tikzpicture}[x=4mm,y=13mm,new set=import nodes, >=stealth']
	\begin{scope}[nodes={set=import nodes}] % make all nodes part of this set
	%\node (p) at (0,4) {$p_1$};
	\node[circle, draw, inner sep = 0pt, minimum size = 12pt] (polytope) at (0,4) {};
	\node[circle, draw, inner sep = 0pt, minimum size = 12pt] (emptyset) at (0,-1) {};
	\foreach \i in {1,...,8} {
		\node[circle, draw, inner sep = 0pt, minimum size = 14pt] (f\i) at ({(\i-4.5)*2.0},3) {$f_{\i}$};
	}	
	\foreach \i in {1,...,19} {
		\node[circle, draw, inner sep = 2pt] (r\i) at ({\i - 10},2) {};
	}	
	\foreach \i in {1,...,19} {
		\node[circle, draw, inner sep = 2pt] (e\i) at ({\i - 10},1) {};
	}	
	\foreach \i in {1,...,8} {
		\node[circle, draw, inner sep = 0pt, minimum size = 14pt] (v\i) at ({(\i-4.5)*2.0},0) {$v_{\i}$};
	}	
	\end{scope}
	\draw (15, 4) node[left] {���� ������������};
	\draw (15, 3) node[left] {����������};
	\draw (15, 2) node[left] {�����};
	\draw (15, 1) node[left] {�����};
	\draw (15, 0) node[left] {�������};
	\draw (15, -1) node[left] {������ ���������};
	\graph {
		(import nodes); % "import" the nodes
		polytope -- {f1, f2, f3, f4, f5, f6, f7, f8};
		emptyset -- {v1, v2, v3, v4, v5, v6, v7, v8};
		e1 -- {v1, v2}; e2 -- {v1, v3}; e3 -- {v2, v3}; e4 -- {v2, v4}; e5 -- {v1, v4};
		e6 -- {v1, v5}; e7 -- {v3, v5}; e8 -- {v3, v6}; e9 -- {v2, v6}; e10 -- {v4, v7};
		e11 -- {v5, v7}; e12 -- {v6, v7};
		\foreach \i/\j in {1/13,2/14,3/15,4/16,5/17,6/18,7/19} {e\j -- {v\i, v8};};
		%\foreach \i in {1,...,14} {r\i -- e\i;}
		r1 -- {e1, e2, e3}; r2 -- {e1, e4, e5}; r3 -- {e2, e6, e7}; r4 -- {e3, e8, e9};
		r5 -- {e4, e9, e10, e12}; r6 -- {e5, e6, e10, e11}; r7 -- {e7, e8, e11, e12};
		r8 -- {e1, e13, e14}; r9 -- {e2, e14, e15}; r10 -- {e3, e13, e15}; r11 -- {e4, e13, e16}; r12 -- {e5, e14, e16};
		r13 -- {e6, e14, e17}; r14 -- {e7, e15, e17}; r15 -- {e8, e15, e18}; r16 -- {e9, e13, e18}; r17 -- {e10, e16, e19};
		r18 -- {e11, e17, e19}; r19 -- {e12, e18, e19};
		f1 -- {r1, r2, r3, r4, r5, r6, r7};
		f2 -- {r1, r8, r9, r10}; f3 -- {r2, r8, r11, r12}; f4 -- {r3, r9, r13, r14}; f5 -- {r4, r10, r15, r16};
		f7 -- {r5, r11, r16, r17, r19}; f6 -- {r6, r12, r13, r17, r18}; f8 -- {r7, r14, r15, r18, r19};
	};
	\end{tikzpicture}
	\caption{��������� ����� ������� ������ �������� ��� ���������� ����� ��� �������}%
	\label{fig:cube7Hasse}%
\end{figure}

��� ������������� ���������� \emph{������������ ��������������}, ���� �� ������� ������ ���������.
���� ��� ������������� ������������ ������������, �� �������, ��� ��� �������� ��������������� ������ \emph{�������������� ����}.
�������� � �������� �������������� �������������, ���������� ������������ ��� �������� ������, ���������� \emph{��������������}.
%, ����� ���� �������� �� �� ������ ������� � ������������� �������������� ��������� (��. ���������~\ref{rem:combinatorial} ����).
� ���������, ����������� �������������, ����� ��� ������ � ����� ����������� �������� �������������� ����������������, � ����������������, �������� � $k$"=������������� "--- �������������� ����������.

%\begin{remark}
%\label{rem:combinatorial}
%���������� ��������� ������ ����� � ������� ������������� �������������, � ������ ������� �������������� ���������� �� ������ ��������� ���� �������� � ��������������, �� � ��������� ������, ��� ����������� ������� ������ ������� ������ ������������ (��������, ��������� ���������� (��. ������~\ref{sec:Extension})). �����, � ������������ ������ ��������� �� ����� �������������� ���� ��������, �� � ��� �������, ����� ��� �����, ��������, ���������� ������������ �������� ������, ����� �������� ����� ��������������.
%, � ��������, ��������� �� ��������� �������������, "--- \emph{������������"=���������������}.
%\end{remark}

��� ������������� ���������� \emph{�������������} ���� � �����, ���� �� ������� ������ �������������.
� ���������, ���� $P$ � $Q$ �����������, �� ������� $P$ ������������� ����������� $Q$, ����� $P$ "--- ������ $Q$ �~�.\,�.
�������� ������������ �������������� ����� ������� $d$-��� � $d$-������ �������.
������������, ������������ �� ���.~\ref{fig:cube7}, ����������� ������ ����.
��� ����� �������� ������������� ������ ���� ������������� �������� $d$-��������.
������, ������������, ������������ ���������������, �������� �������, 
� ������������, ������������ ��������, "--- ��������������~\cite{Emelichev:1981}.


����� $d$-������������ $P$ ����� � ���� $P = \conv(V)$ � $\bm{0}$ �������� ���������� ������ ����� �������������. (���������� ���������� ������� ������ ����� �������� �� ���� �������� �������� $\bm{x} \mapsto \bm{x} + \bm{x_0}$.)
\emph{�������} � $P$ ���������� ������������
\[
P^* \coloneqq \Set*{\bm{x}\in \R^d \given \bm{y}^T \bm{x} \le 1, \ \forall \bm{y} \in V}.
\] 
������ $P^*$ �������� �������� ������������� � $P$ �������������~\cite{Emelichev:1981,ZieglerBook}.


����� $P$ "--- ��������� ������������, $V = \{v_1, \dots, v_n\}$~--- ��������� ��� ������,
� $F = \{F_1, \dots, F_k\}$~--- ��������� ��� �����������.
����� \emph{������� ���������� ������"=�����������} $M=(m_{ij})\in \{0,1\}^{n\times k}$ 
������������� $P$ ������������ ��������� �������:
\[
m_{ij} = \begin{cases}
1, & \text{���� } v_i \in F_j,\\
0, & \text{�����.}
\end{cases}
\]
������� $M^T$ ���������� \emph{�������� ���������� �����������"=������.}

������� ������ ������������� ���������� ����������������� �� ��� ������� ���������� ������"=�����������. 
(���� �� �������� ����������� ���������� ������� ���� ������ ������ �~\cite{KaibelP:02}.)
���, ��������, ��������� ����� �� ���.~\ref{fig:cube7Hasse} ����������������� �� ������� ����������
\begin{equation}
\begin{pmatrix}
1 & 1 & 1 & 1 & 0 & 1 & 0 & 0 \\
1 & 1 & 1 & 0 & 1 & 0 & 1 & 0 \\
1 & 1 & 0 & 1 & 1 & 0 & 0 & 1 \\
1 & 0 & 1 & 0 & 0 & 1 & 1 & 0 \\
1 & 0 & 0 & 1 & 0 & 1 & 0 & 1 \\
1 & 0 & 0 & 0 & 1 & 0 & 1 & 1 \\
1 & 0 & 0 & 0 & 0 & 1 & 1 & 1 \\
0 & 1 & 1 & 1 & 1 & 1 & 1 & 1 \\
\end{pmatrix}
\label{eq:Minc}
\end{equation}
����� �������, ��� 
%����� 
������������� �������� ������������� ���������� ������������ �� ��� ������� ����������, � ����� ������������ ����� �/��� �������� ���� ������� �� ������ ���� �������.
����� ��������, ��� ������� ���������� ������"=����������� ������������ �������������� ������������� ���� � ����� ��������� ���������������� �, ��������, ������������� ����� �/��� ��������.
��������, ��������� ������������ �������� (��������������) ��������� � ������� \eqref{eq:Minc} ������� � �������������� ���������������� ������������� ������ ����.


%\subsection{�������� � ����������� �������������� ��������������}
\subsection{�������� ���������������}

����������� ���� $\bm{x} \mapsto A \bm{x} + \bm{b}$, 
��� $\bm{x} \in \R^d$, $A \in \R^{m\times d}$, $\bm{b} \in \R^m$, ���������� \emph{��������}. 
������� ������� ��������� �������������� �������� \emph{������������� ��������} $(x_1, x_2, \dots, x_d) \mapsto (x_1, x_2, \dots, x_m, 0, \dots, 0)$, $d > m$.
��� ������������� $P\subseteq \R^d$ � $Q\subseteq \R^m$ ���������� \emph{������� ��������������},
���� ���������� �������"=����������� �������� ����������� $\alpha\from P \to Q$.
�� �������� ��������������� �������������� ������� �� ������������� ���������������.

����� ��� $d$-��������� ������� ������������.
������� ����� ������ ������������� $d$-��������� ����� ��������������� ��� ������������ �������
\[
\Delta_d = \Set*{\bm{x} \in \R^{d+1} \given \bm{1}^T \bm{x} = 1, \ \bm{x} \ge \bm{0}}
= \conv\{\bm{e_1}, \dots, \bm{e_{d+1}}\}.
\]\label{ProjOfSimplex}
�������� ��������, ��� ����� ������������, ������� $n$ ������, �������� �������� ������� ��������� $\Delta_{n-1}$.

%\emph{����������� ���������������} ���������� ������"=�������� ����������� ����
%\[
%\tau(\bm{x}) = \frac{\alpha(\bm{x})}{\bm{a}^T \bm{x} + b},
%\]
%��� $\alpha$ "--- �������� �����������, ����������� �������� $\bm{a}$ � $\bm{x}$ ���������, $b\in \R$.

%����������� �������������� �������� ���������� ����������~\cite{ZieglerBook}:
%\begin{enumerate}
%\item ����� $P$ � $Q$ "--- �������������. ���� ����������� �������������� $\tau \from P \to Q$ �������"=����������,	�� ������������� $P$ � $Q$ ������������ ������������.
%\item ����� ������������ $Q$ �������� �������� ������� ��������� ����� ������������� $P$. ����� ���������� ����������� �������������� $\tau \from P \to Q$.
%\end{enumerate}

 


\section{��������� ����� � ����������}
\label{sec:complexity}

��������������, ��� �������� ������ � �������� ������ ��������� ������~\cite{Arora:2009, Goldreich:2008} � ������ NP"=������ ����� �~���������~\cite{Garey:1982}.
��� �� �����, ����� �������� ��������������� � ����������, ���������� ��������� �������� ������� � ����������.

��� ��������������� ��������� ���� ������� $f\from \N \to \R_+$ � $g\from \N \to \R_+$ ������������ ����������� �����������:

$f = O(g)$, ���� �������� $c > 0$ � $n_0 \in \N$ �����, ��� $f(n) \le c \cdot g(n)$ $\forall n \ge n_0$.

$f = \Omega(g)$, ���� �������� $c > 0$ � $n_0 \in \N$, ��� $f(n) \ge c \cdot g(n)$ $\forall n \ge n_0$.

$f = \Theta(g)$, ���� $f = O(g)$ � $f = \Omega(g)$.

$f = o(g)$, ���� $\forall c > 0$ �������� $n_c \in \N$, ��� $f(n) < c \cdot g(n)$ $\forall n \ge n_c$.

�������~$f\from \N \to \R_+$ ����� �������� \emph{��������������} � ���������� $f(n) = \poly(n)$,
���� �������� $k \in \N$, ��� $f(n) = O(n^k)$.
%������� $p=p(n)$ �����, ��� $f(n) \le p(n)$ ��� ���� $n \in \N$.
������� $f$ ���������� \emph{�������������������},
���� $f(n) = \Omega(n^k)$ ��� ������ $k \in \N$. 
%��������~$p$.
���, ��������, ������� $f(n) = a^{\ln n}$, ��� $a > 0$, �������� ��������������, �~������� $g(n) = a^{\ln^{1+\varepsilon} n}$ ��� $a > 1$ � $\varepsilon > 0$ "--- �������������������.
������������ ����� ����������� ������ ��������� ��������� �� ��������������� �������� ����� ��������������� � �������������������� ���������.
������, ����, ��� �������, ���� �� ������� ����� ��������������� ��������� �~��������� ���� $f(n) = \Omega\left(a^{n^{\varepsilon}}\right)$,
 ��� $a > 1$, $\varepsilon > 0$.
������� $f$ ���������� \emph{����������������},
���� $f(n) = 2^{\Theta(n)}$.
%���� $f(n) = \Omega(a^n)$ ��� ��������� $a > 1$ �, ������ � ����, $f(n) = O(2^{\poly(n)})$.
%��������� ������ ���������� \emph{�����������������}~\cite{BondBook:2008}.

������� ������ �������������� ������ ������ ������������
����� ����������� ������� ������������� ����� ����� � ����� (��������� ������ ����� �������� ������������ �������).
%�� ����������� ������������, ��� ������� ������ �������������� ������ ���������� � �������� ������������������ ��������� ������������ �������. 
�� ����������� ������������, ��� ������� ������ ���������� � �������� ������������������ ��������� ������������ �������. 
�� ���� ���� ������ ����������� ��������� $\{0,1\}^* = \bigcup_{m \in \N} \{0,1\}^m$.

� ���������, ������ ������������ ����� $n$ �������� $\lceil\log_2 (n+1)\rceil$ ���.
����������: $\size(n) = \lceil\log_2 (n+1)\rceil$.
(�����������, ����� ����� $I$ ������������ ���: $|I|$.
� ������, ���� ������ �������� ������� (����� ��� ������������) �����, ��� ������� ��������. ������� � ��������� ������ ������������ ����� ������������� $\size(I)$.)
��������������, $\size(k) = \size(|k|) + 1$ ��� ������ �����~$k$.
������������ ����� $p$ �������������� ����� ������� ������� ����� $k$ (���������) � $n$ (�����������), ��� $k \in \Z$, $n \in \N$.
%, �� ���� $\size(p) = \size(k) + \size(n) + 1$.  
� ������������� ������������� ����� ������� ������ ������ ������ ���������� ������ ����� ���� ��������������� �����.
�� �������� ��, ���� �����, ����� ������� ������ ��������������� ������������ ���������� ����� �� ����� ������ ����������� �� ���.
��� �� �����, ����� ������ �� ���� ��������� ������������� (� ������ ������, �� �����, ��� �����������) ������������ ����� ������.

\emph{�������\'�� ���������� ���������} ���������� �������, 
������� ������� ������������ $n$ ������ � ������������ ������������ ����� (����� ��������), ������������� ���������� ��� ��������� ������� ������ ����� $n$~\cite[�.~18]{Garey:1982},~\cite[p.~32]{Goldreich:2008}.
����� ����� ��� \emph{���������� ���������} ����� �������� 
��� �������\'�� ���������.
��� (�������\'��) \emph{���������� ������} ����� �������� ��������� (��������������) �������� ������ ��������� ��� ������ ���������.
(� ��������� ������ ��������������� ������ ���������� ������.)
��������� ���������� �, ��������������, ����� ����������� ������� �� ������ ����������. ��������� �����, ����� ����, ����� �������� �� �������������� ����������� (�������������� ���� ��� ����� ����������������), ������������� �� ����� ��������������� ����������.
� ������ �������, ��� ������������ � ��������� ����� �������������� ���������� ������������ � ������ ������ �������--��������~\cite[c.~33]{Goldreich:2008},
������� ����� �������� ������� ׸���--�������� � ������� �����~\cite[c.~26]{Arora:2009}.

\textbf{����� �������--��������.}
\emph{����� ��������� ����������� �������������� ������ 
����� ���� ��������������� (�������������) ������� ��������
� (�� ����� ���) �������������� ����������� ������� ������.}

� ���������, ���� ����� ����������, ��� �������������� ������ � ������ $P$ (������������� ���������� �����) �� ������� �� ������ ��������� ����������� �������������� ������.

�������������� ���� �������, ���, ������������, � ������� ��������� ���������� ����� ������ ������, �������������� ������� � ������ $P$ � ��������� ����� �� ��������.
��� �� �����, ����������� ���������� ���������� ��������� ����������� (������, ����������� �� ���������������) ���� �������� ��� �������� (����������� ����� ����������� ������ ������� �~\cite{Aaronson:2008}).

������ ����� ������ ������������ � ���� �������, ����������� ������� �������� ������� ������, � ���������~--- ������� ������.
������� ����� �������� \emph{������������� ����������}, ���� ��������������� ������ ����� �������������� ��������� (�� ������ ��������).

\emph{����������} ���������� ������� � ���������� �������� $\{\text{����},\text{������}\}$ (��� $\{0,1\}$).
����� �������� $g$, �������� �� ��������� $\{0,1\}^*$, ���������� ��������� ����
$L = \Set{x \in \{0,1\}^* \given g(x)}$, �, ��������, ����� ���� $L \subseteq \{0,1\}^*$ ��������� ������� ���������� ��������� ��������, ��������������� ������ �������� ������� $x \in L$.

\begin{definition}[������ NP � co-NP~{\cite{Arora:2009}}]
	\label{def:NP}
	���� $L \subseteq\{0,1\}^*$ ����������� ������ NP, ���� �������� ������� $p\from\N\to\N$ � ������������� ���������� �������� $g\from \{0,1\}^* \times \{0,1\}^* \to \{\text{����},\text{������}\}$
	�����, ��� ��� ������� $x \in \{0,1\}^*$
	\[
	x\in L \iff \text{�������� } u\in \{0,1\}^{p(\size(x))} 
	\text{ �����, ��� } g(x,u).
	\]

���� $L$ ����������� ������ co-NP, ���� ���� $\{0,1\}^* \setminus L$ ����������� NP.

�������, ��� ������ �������� ������� $x \in L$ ����������� ������ NP (co"~NP),
���� ���� $L$ ����������� ����� ������.
\end{definition}

������������ ������� �������������� ���������� ����� ������������ �� ������� � ����������� �� ���� �������� �����~\cite{Garey:1982, Goldreich:2008}.
����� �������� �������� (� ����� ������ �������� ����������� �����������) ������������, ����� �����, ���� �� ��������� ����, ��� �������������� ����������� �� ����� �������� �������������� ���������� �� �������� (������ ��� ���������� ����������� �� ����).
�������� �������� �������� ����������� ����� ���� ����������, ��������������, ��� �������� ������ � �������� ���������� ������ ��������~\cite{Garey:1982, Goldreich:2008}:

\begin{definition}[�������������� ����������]
	������ $\Pi$ \emph{������������� ��������} � ������ $\Pi'$,
	���� ���������� ���������� ������ $M$ �����, ��� ��� ����� ������� $f$, �������� ������ $\Pi'$, ������ $M$ � �������� $f$ ������ ������ $\Pi$ �� �������������� �����. (��������� � ������� $f$ ����������� �� ������� �������.)
\end{definition}

������ �� ������ NP (co-NP) ���������� \emph{NP-������} (\emph{co-NP-������}),
���� ����� ������ ������ �� ����� ������ ������������� �������� � ���.

%�� �� �������� ����� ����������� ������� NP � co-NP, � ����� ������� NP-������ (� co-NP-������) ������~\cite{Garey:1982}, ��� ��� ��� ������� �������� ����������� � ���������� ����� ������������� ��������� (� �������, ��������, �� ������� NP-������� ������).

� �������� ������� �������� ������������ �������� ��������� NP-������ ������ � ������������ ������� �������.

����� $U = \{u_1, u_2, \ldots, u_k\}$ "---
��������� ������� ����������. 
������ ������ ���������� ����� ��������� ���� ���� �� ���� ��������: 
$1$ ��� $0$.
%$1 \overset{\text{def}}{=} \text{<<������>>}$ ��� $0 \overset{\text{def}}{=} \text{<<����>>}$.
���� $u\in U$, �� $u$ � $\bar{u}$, $\bar{u} = 1 - u$, ���������� \emph{����������}.
��������� ���������, �������� $\{u_2, \bar{u}_4, u_5\}$,
���������� \emph{�����������} ��� $U$ � ������ ������������ $u_2 \vee \bar{u}_4 \vee u_5$.
�������, ��� ���������� \emph{���������} (��������� �������� 1) ��� ��������� ������ �������� ����������, ���� ���� �� ���� �� �������� � ��� ���������
����� 1.
����� $C = \{D_1, D_2, \ldots, D_m\}$ "--- ��������� ����� ����������, ����� ���������� \emph{�����������}.
� ����� ������� �������, ��� ������ ������� $C$ ������ � \emph{������������� ���������� �����}\label{def:CNF} (���).
���������� $C$ ���������� \emph{����������}, ���� ���������� ����� �������� ���������� �����, ��� ������������ ����������� ��� ���������� �� $C$.
\medskip

\problem{������ � ������������.}
���� ��������� ���������� $U$ � ����������~$C$ ��� $U$.
����� ��, ��� $C$ ���������?
\medskip

%� ������ ������ NP-����� ������� ���������� ��������� ���� ������� ������ (� ������ ������ ����������) ��� ������� ����� �� ������ ������ �����������.
%��� ��������� ������ ���������� ������, � ������� ������ �������� ������ �������������� ������ "--- ������~\cite{Garey:1982}.
����������� ������ � ������������ ��������� ������ NP-������ ������� (���� ��� ������ ������ �������)~\cite{Cook:1971}.
\emph{������ � $k$-������������} ������������ ����� ������� ������ ������ � ������������, ����� ������ ���������� �������� ����� $k$ ���������.
��� ��� ��� ������ ������������ $k \ge 3$ ������ � ������������ ������������� ��������~\cite{Karp:1972} � ������ � $k$-������������, �� ��������� ����� �������� NP-������.

\emph{NP-��������} �������� ����� �������� ��, � ������� ������������� �������� NP-������ ������.
�������, ��� ��������� ������������� ��� ����������� NP-������� ������ ������������ ������������ ���������� �� �����. 
� ���� ������, � ���������, ������ NP-������� � co-NP-������� ����� ���������� ���� �� ����� ��� ������� $\NP \ne \coNP$.
�� �� ����� ����� ������������ ������������� �������������� ���������� �� ��������. 
��� ����� ����������� co-NP-������ ������ �������� NP-��������.

%����� ��� ����� ����������� ����������� ������ $D^p$~\cite{PapadimitriouY:1984}:
%\[
%D^p = \{L_1 \cap L_2 \mid L_1\in \NP, \ L_2\in \coNP\}.
%\]
%� ���������, NP � co-NP �������� �������������� $D^p$, ������ �������, ��� ��� $D^p$-������ ������ �������� NP-��������.
%� �������� ������� �������� ��� $D^p$-������ ������~\cite{PapadimitriouY:1984}:
%\begin{enumerate}
%	\item \problem{������������--��������������.} ���� ��� ������ �������. ����� ��, ��� ������ ���������, � ������ "--- �����������.
%	\item \problem{������ �����.} ��� ���� $G$ � ����� $k \in \N$. ����� ��, ��� $k$ �������� �������� ������ ����� �����.
%\end{enumerate}



\section{������ ���������� �����������}
\label{sec:CO}

\emph{������ ���������� �����������} ������������ ���������� ���������� 
\emph{���������� �������} $X$ � \emph{������� ��������} $f \colon X \to \R$.
���� ������ ����������� � ��������� \emph{������������} ������� $x\in X$,
�� ������� ������� ������� $f$ ��������� ������ ��������� (��������).
������ ����������� ������ ����� ���� ������������� � ������ ������������ (� ��������) �� ���� ��������� �������� ������� ������� �� $-1$.
������� �����, ���� �� ��������� ����, ��� ������� ����������� ����� ��������������� ������ ���������� ���������.

���������� ������� ������������ ������, � ������� ��������� ���������� ������� $X$ �������� ������~--- ��� ��������� ��������, ��������������� ���������� ������ �������.
������� ����� ���� ��������. 
��������, � $X$ ������ ��� ����������� �����, �� ������������� 1000. %����� ����� ������ �������. 
� ����� ���� � ��������. 
��������, $X$ ������� �� (������� �������������) �������� �� ����� C ������ �� ����� 1~�����.
%��� �� �����, ������ ��������������, ��� �������� �������������� $x\in X$ ������������� ���������. 
%� ���� ������ ��������� ������ �������� ������ �����������, ��� ��� ���������������� 
%�������� �������� ����� ���� �������� ���� ������������ �������~\cite{Agrawal:2004},
%� ������������ �������� �������� �����������.

� ������ ���������� ��������� ������ ���������� ����������� ������������ ����� ��������:
\begin{compactitem}
\item \emph{������������ �������} $U$, 
\item \emph{�������� ������������} $g \colon U \to \{\text{����}, \text{������}\}$,
������������ ��������� ���������� ������� $X = \{x\in U\mid g(x)\}$,
\item ������� ������� $f \colon U \to \R$,
\item \emph{����������� �����������:} $\min$ ��� $\max$. �� ���������, $\max$.
\end{compactitem}
��� ����� ���������� ������ ��������� ��������������, ��� �������� ������������ � ������� ������� ����������� �� �������� �����. � ��������� ������ ������ ���������� ������������.
�� ��������, ��� �������, ���� ���� � ������������ �� �������������� (������������ ����� ���������) �����.

��� ������������ ��������� ������������ ��� ����������� ��������� �������� �������� ����� �����.
� ����� ������� ������� ���������� �������� ��������������� ��������.
%�� ���������, ����������� �����������~--- ��������.

\problem{������ �������������� ��������� ����������������.}
����: ������������ ������� $A \in \Q^{m \times n}$ � ������������ ������� $\bm{b} \in \Q^m$ � $\bm{c} \in \Q^n$. 
����� ����� ������������� ������ $\bm{x} \in \Z^n$, 
��� �������� �������� ������� $\bm{c}^T \bm{x}$
��������� ������������ �������� ��� ����������� $A \bm{x} \le \bm{b}$.
��� ����� ������, ������������ ������� $U$ � ���� ������ ��������� � $\Z^n$,
���������� ������������ �������� ����������� $A \bm{x} \le \bm{b}$,
� ������� ������� $f(x) = \bm{c}^T \bm{x}$.

\problem{������ ��������� ��������� ���������� �� ��������� ����� ����� �������.}
����: ����� (������������) ������������� ���������� $p(x) = \sum_{k = 1}^d c_k x^k$
� ��� ����� $a, b \in \Z$, $a \le b$.
����� ����� $x \in [a, b]$ ��� ������� $p(x)$ ��������� ���������� ��������.
����� $U = \Z$, �������� ������������ ��������� ���������� ����������� $a \le x \le b$,
������� ������� $f(x) = p(x)$.

\problem{������ ������ ������������� ����������������.}
��� ����� (������������) ������������� ���������� 
\[
p(\bm{x}) = \sum\limits_{\mathclap{1\le i \le j \le n}} c_{ij} x_i x_j, 
 \quad \bm{x} = (x_1, \dots, x_n) \in \{0, 1\}^n. 
\]
��������� ����� $\bm{x}$, ��� ������� $p(\bm{x})$ ��������� ���������.
����� $U = \{0, 1\}^n$, �������� $g$ ������������ ����� ������, 
� ������� �������� �������� ���������~$p$.

\problem{������ � �������.} 
���� ��������� ��������� $E$.
��� ������� �������� $e \in E$ �������� ��� ������ $a_e \in \Q$ � �������� $c_e \in \Q$.
����� ����, �������� ������ ������� $A \in \Q$.
��������� ������� ������������ ��������� $x \subseteq E$ ���, ����� �� ��������� ������ ��� ������ ������� �������, � �� ��������� �������� ���� �� ������������.
� ���� ������ ������������ ������� $U$ ���� ��������� $2^E$ ���� ����������� ��������� $E$,
�������� ������������ $g = g(x)$ ������������ ������������ $\sum_{e\in x} a_e \le A$,
� ������� ������� $f(x) = \sum_{e\in x} c_e$.

\problem{������ � ���������� ����.} 
���� ��������� ������� $V$ � ��������� ����������� �� �������� ����� (������, ��� �������) $E$. ��� ������� ������� ������ $e \in E$ �������� ��� ����� $c_e \in \Q$. 
(����� �� ������������, ��� ��������� $E$ �������� ��� ���� �������. ���� �� �����-���� ��� ������ �� ��������� ������� ���������������, �� �������� ����� ���� ������ ������ ���������� ���������� �������� �����.)
� ��������� $V$ �������� ��� ������ $s$ � $t$.
��������� ����� ���������� (�� ��������� ����� �������� �����) ����, ����������� $s$ �~$t$.
������������ ������� � ���� ������ ������� �� ������������ ���������� �� $V \setminus \{s,t\}$.
������ ���������� "--- ������������������ ������� ���������� ����, ������������ $s$ � $t$.
�������� ������������ ������������ ����� ������.
%�������� ������������ ���������, �������� �� ������ ��������� $u \in U$ �������� ����� ������� ����, ����������� ��� ��������� ������. 
%������� �������, ��� � � ���������� �������, �������: $f(u) = \sum_{e\in u} c_e$. %, $u \in U$.
������� ������� ����� ����� ���� �������� �����, 
������������ ������ ����.
����������� �����������~--- �������.

%\problem{������ ������������.} 
%���� ��������� ������� $V$ � ��������� ����������� �� �������� ����� $E$. ��� ������� ������� ������ $e \in E$ �������� ��� ����� $c_e \in \Q$.
%��������� ����� ���������� �������, ���������� ����� ��� ������ �� ������ ���� � �������������� � �������� �����.
%������������� ������� ����� �������� ��������� ���� ������������ ������� $S_n$, $n = |V|$, �������� ������������ ������������ ����� ������, %, �������� �� ������ ��������� �������� ����� ��������� �������, ���������� �� ������ ���� ����� ��� ������, 
%� ������� ������� $f(u)$ ����� ����� �������� ����� ���������������� ��������.
%= \sum_{e\in u} c_e$, $u \in U$.
%����������� �����������~--- �������.

\hypertarget{pColor}{\problem{������ � ��������� ������ �����.}\label{problem:Color}}
����� ���� $G = (V, E)$.
��������� ������ ��� ������� $v \in V$ ��������� ��������� ����� $k_v \in \N$
(��������������� ����� �����) ���, ����� ����� ��� ������� ������� ����� ������ �����, � �������� ���������� ����� ���� �� �����������.
����� �������, $U = [n]^n$, ��� $n$~--- ����� ������ �����, 
�������� $g$ ������������ <<��������������>> ������� ������,
������� ������� $f(\bm{u}) = \max\{u_1, \dots, u_n\}$, $\bm{u} = (u_1, \dots, u_n) \in U$.
����������� �����������~--- �������.

����� ��������� ����� ���������� ����������� ����� �������� 
\emph{������ ������������� �����������}, 
������������ ������� ������� �������~\cite{SchrijverCO:2003} � ����� ������������� ��������.

���, ��������, ������ �������������� ��������� ���������������� �� �������� �������������, ��� ��� �� ������������ �������, ������ ������, �� ����������.
�� ��� �������������� ����������� $|x_i| \le B$, $i\in[n]$, $B \in \Z$, ������ ���������� �������������.
� ������ ��������� ��������� ���������� �� ��������� ����� ����� ������� ������������ ������� �������, �� ����� ��������������� ��������. ��� �� �����, ��� ������ ���������� �������������, ���� � �������� ������������ �������
����������� ��������� ������������������� �����~$n$, ��������� �� ����� � ������,
� ���������������� ��� ������������������ ��� �������� ������ ����� ����� �� ������� $[0, 2^n-1]$, $2^n > b$.

������, ��������������� �������������, ��� ������������ $U$ ������������ ����� ���� ��������� ����������� (������ � �������),
���� ��������� ���������� � ������������ (������ ������ ������������� ����������������, ������ � ��������� �����),
���� ��������� ���������� ��� ���������� (������ � ���������� ����). 
� ����� ��������� ���� ���� � �������� �� ������� ����� ������������, ��� ������������ ������� 
������������� ������ ������������ ����� ��������� ������� $S^d$
���������� ��������� ��������� $S$ (� ������������ ���������� ����� ����������� ground set),
��� ������� $d \in \N$ ���������� \emph{������������ ������}.
��� ����, ����������, � �������� ������������ ����� ���� ��������� �������������� ������� 
(��������, �������� �� ������� ������������� ��� �� ����������� ��� ����������).

����������� ��������� ������������� ����������� �������� ������� \emph{�������������� ������}. 
��� ��� ������������ ������� ��� ������ ������������� ����������� �������, 
�� �� ������ (� ������) ����� ������ � ������� ������� �������� ���� �������.
������� ����� ������ ��� �������� \emph{�����������}.
��� �� �����, ��� ����� ����������� ������ ����� ������������� ������� ���������� ���������������: $|U| = |S|^d$.
��� ���� ������ ������� ���������� ����������� ����������, ��� ��������� ������ ����� ������������ ������� ������ ������������ �������.
%� ����� �������� ������� ������� ������������� ���������� ��������� ��� ������� ������,
%��� �������, �������� ����������� ����� ��� ����� �� ����� ����� ������.

�������� ���� �������� ���������� ��� �������� ������� �������� ������.
� \emph{�������������� ������} ���������� (�������������) ����������� ��� ������� ������
(��������, ������������ ������) �������������.
\emph{�������� ������} ������������� ����������� ������������ ����� ������������������
��� ������� ��������� ���������� �������������� �����,
����� ������� ���� ������ � ��� ������ ������� ������������.
���, ��������, �������� ������ � ������� �������� � ���� ��� �������������� ������,
����������� ��������������� ������������.
������� �������, � �������� ������ � ������� �� ������������� ��������� ���������, �� ������� � ��������.

� �������� ������ ������������� ����������� ������� �������, ��� �������, 
������� �� ���������� ������ ����������, �������� ������� �������� �������������� ������ �� ��������.
��������, � ������ � ������� ������ ����������� �������� �������� ���������.
�� �� ����� ����� � � ��������� ��������� ������������.
� ������ � ������� �� ������� �� �������� ���������.
����� �������, �������������� ������ �� ���� ������������ ����� ������ ���������� ����� ������� ������ �������� ������.

%�����, ������, �������� ������ ����, ������� �� ������������ � �����������:
%���� ����������� $d$, ������ $r$ � ������� ������ $\bm{c} \in \R^d$, 
%����� ����� ����� $\bm{x} \in \Z^d$ ������ $d$-������� ���� ������� $r$ � ������� � ������ ���������,
%��� ������� ������� ������� $\bm{c}^T \bm{x}$ ��������� ���������.
%����� ������������ ������� ������� ��� �� $d$, ��� � �� $r$ (�.�. �� ���� ����������� ����������).

� ������ ���� ��������� ���� ��������� �������� ��������� ����������� ������ ������������� �����������.

%http://www.nada.kth.se/~viggo/problemlist/
\begin{definition} %[������ ������������� �����������]
\label{def:COP}
\emph{������ ������������� �����������}
������������ ����� �������:
\begin{enumerate}
	\item \emph{�������� ������� ������}, ������������ �� ��������� ���� ���������� �������������� ��������
	(��� �������, �������������� ����� ������ �����).
	������ ����� (����� �����) $I$, ��������������� ����� ���������, ���������� \emph{�������� �������} ��� \emph{�����} ������.
	\item \emph{����������� ������} $d = d(I) \in \N$.
	\item �������� ��������� $S = S(I)$,\\ ������������ \emph{������������ �������} $U = S^d$.
	\item \emph{�������� ������������} $g = g(u, I) \in \{\text{����}, \text{������}\}$, ��� $u \in U$.
	\item \emph{������� �������} $f = f(u, I) \in \R$, ��� $u \in U$.
	\item \emph{����������� �����������:} $\min$ ��� $\max$.  �� ���������, $\max$.
\end{enumerate}
���� ������� ������ $I$ �� �������������, �� ������ ���������� \emph{��������}, ����� "--- \emph{��������������.}

��� ������������� ������� ������ $I$ �������� $g$ ���������� ���������� \emph{��������� ���������� �������} $X = X(I) = \{u \in U \mid g(u, I)\}$.
���� ������ "--- ����� ���� ���������� ������� $X$ ����� �����, �� ������� ������� ������� $f$ ��������� ����������� ��������.
��������� ������� ���������� \emph{�����������.}
\end{definition}

\begin{remark}
\label{rem:PolyPred}
���� ��������� ���������� ������ �� ���������� ��� �� ������� ������� �������� � ������ ������������ �������, �� �������� �������������� ������ ��� ������� ����� ������ ������� �� ������ ����.
�������, ��� �������, ��������������, ��� \emph{��� ������� � �����������~\ref{def:COP} ������������� ���������.}
\end{remark}

\begin{remark}
\label{rem:CombOpt4}
%������� $d = d(I)$, $S = S(I)$ � $g = g(u,I)$ ���������� ��������� ���������� ������� ��������������� ������. �������� ������� ������ ����� ���� ������� � ��� ������� ����� �������, ����� ��� ������������� ����� ��������� ������� ���� �� ������. 
%�� ���� ������� 
� ����������, ��� �������� ����� �� �� ������������� �������� ������� ������, �����������, ��� �� ���� ������ �������� ������ ���������� ������.
��������, ��� ����������� ����������� �� ��������� "--- ��������, ������ ������������� ����������� ����� ������������ ����� �� ��������, � ��������.
\end{remark}

�� �������� ��������, ��������� $S$ ����� ������������ ������������� ����� �����.
����� ����� ������������, ��� $S = \{0,1,\dots,k-1\}$, ��� $k = |S(I)|$. 

\begin{definition}[\cite{Junger:1995,Onn:2004}]
\label{def:LCOP}
������ ������������� ����������� ���������� \emph{��������},
���� ������� ������� �������:
$f(\bm{u}) = f(\bm{u}, \bm{c}) = \bm{c}^T \bm{u}$, ��� $\bm{u} \in U = S^d$, �~\emph{������� ������} $\bm{c} \in \Q^d$ ���������� �� ������� ������ ������.
%��������������� ������ ���������� \emph{�������� ������� ������������� �����������}.
\end{definition}

����� �������, � ������ ���������~\ref{rem:CombOpt4}, �������� �������� ������ ������������� ����������� ������������ ����� ������ ������� $d = d(I)$, $S = S(I)$ �~$g = g(u,I)$, ������������ ��������� ���������� �������.

\begin{remark}
��������� �������� ������� $\bm{c}$ �� ������������� ������ �� ������ ������������� �������. 
����� ����, �� ���� ��������� $\bm{c}$ �� ���������� ����� ������� ������������ ��� ���������, ��� ����� ������� ������������� �� �������������� ������������ ������� $\size(\bm{c})$ �����.
�������, �� �������� ��������, ����� ������������, ��� ������ $\bm{c}$ "--- �������������.
\end{remark}

����� �� ����� ������������� ������������� �������� ������ ������������� �����������.
� ����� �������, ������ � ����� ������������ �������� ����� ����������� �� �������� (��������, ������ �������������� ��������� ����������������, ������ � �������).
� ������ �������, ������ ������� ����� ������ ���� � ��������� ����� ������� ��������� ������, ��� ���� ������ � ���������� ������� �������� �� ������ ������� �������� ����� ����������, ���� �������� �� � ��������� ����.
� �������� ������� �������, ��� ����������� ��� �������������� ��� ���������, �������� ���������� �����.

� ������ ������ ������������� ���������������� ������ $n$"~������� ������� $\bm{x} = (x_1, \dots, x_n) \in \{0, 1\}^n$ ��������������� ������ $\bm{y} \in \{0, 1\}^{n(n+1)/2}$, ���������� �������� $y_{ij} = x_i x_j$, $1 \le i \le j \le n$.
� ���� ������ ������������ ������� $U = \{0, 1\}^{n(n+1)/2}$, �������� ������������ ��������� ���������� ������� $y_{ij} = y_{ii} y_{jj}$,
� ������� ������� �������: $f(\bm{y}) = \sum_{1\le i \le j \le n} c_{ij} y_{ij}$.

�� �������� � ���������� �������, � ������ ��������� ��������� ���������� ����� ������ ���������� $x \in \Z$ ����������� ������ $\bm{y} \in \Z^d$, ���������� �������� ������ ������������� ������� $y_k = y_1^k$, $k \in [d]$. ����� ������� ������� ���������� ��������: $f(\bm{y}) = \sum_{k=1}^{d} c_k y_k$.

� ������ � ���������� ���� � �������� ������������ ������� ������ ������������� ��������� $\{0, 1\}^E$
������������������ �������� ��������� ����� �����.
����� �������� ������������ ��� ������� $\bm{u} = (u_e) \in \{0, 1\}^E$
������ ���������, �������� �� ��������������� ������������ �������� ����� �����, ����������� ������ $s$ � $t$, � ������� ������� $f(\bm{u}) = \sum_{e\in E} c_e u_e$.

� �������� ������������ ������������� ������ � ��������� ������ ����� ������ ��������� �������� � ������������ � ������������������ ������ $\bm{u} \in \{0, 1\}^{n^2}$, ���������� �������� ������������ ��������� �������: $u_{ji} = 1$, ���� $j$-� ������� �������� � $i$-� ����, � ��������� ������ $u_{ji} = 0$.
�������� ������������ � ����� ������ ��������� ������������ ������� $\sum_{i\in[n]} u_{ji} = 1$, $j \in [n]$, (������ ������� �������� ����� ������) � $u_{ji} + u_{ki} \le 1$, $i\in [n]$, ���� $j$-� � $k$-� ������� ������.
����� ������� ������� ����� ���������� ��������� �������:
$f(\bm{u}) = \sum_{i,j\in[n]} n^i u_{ji}$.
����, ��� ������� ����� ����������� ��� ����������� ����� �������������� ������.
(�~�������~\ref{sec:Color} ���������� ����� ��������� ������������� ���� ������, � ������� ������������ �������� ������� ����������� ��������� $\{0,1\}$.)

% ����� ������� �������� �������� ����� ������������� ����������� � �������������� ���������������� ���� � \cite{Junger:1995} "Practical problem solving with cutting plane algorithms in combinatorial optimization"
% ����� �� ��� ������ ���������� NP-��������

�������� ����� ��������� �������� ����� ������������� ����������� ������������� �������������� �������� $S = \{0,1\}$. ��� ������� � ���, ��� ������ ���������� ������ ��������� ��������� ������������~\cite{Junger:1995,Onn:2004}.
���� �������� ��������� $E$, �������� ������������ $g\from 2^E \to \{\text{����}, \text{������}\}$ � ������� ����� $c\from E \to \Q$.
��� ������� ������������ $T\subseteq E$ ����������
�������� ������� ������� $f(T) = \sum_{e\in T} c(e)$.
��������� �����
\[
T^* = \argmax_{T\subseteq E}\bigl\{f(T) \mid g(T)\bigr\}.
\]
������������ ������� $U = \{0,1\}^E$ ����� ������ ������� �� ������������������ �������� ����������� ��������� $E$.

������, ����� ������������� ����� �����������, �������� �����������, ������������� �� ����� �������� �������. � ������������ �����������~\ref{def:COP}, ���������� ���� ������� ����������� (������������� ����������) ���������� ������� ������.
�� ������ ������� ��� ����������� �������� ���������.
�������� ����� ������� ������������ ������ � ���������� ����, � ������� ����������� ����������������� ���� �������� �����, �� ���� $\bm{c} \ge \bm{0}$, ��������� ������ �� ������ NP-������� � ����� ������������� ����������~\cite[sec.~7.5b, 8.6b]{SchrijverCO:2003}.
����� ����� �������� ����� ������ ��������� �������� ������������� ����������� \emph{� ������������ �� ��������� �������� ������ (������� ��������)}.
� ����� ������, ���� ���� ������ � ����������� �� �������������� ����� ������������ ���� $\bm{c} \in P$, ��� $P=P(I)$ "--- ��������� �������.
(� ������ ������ � ���������� ����, $P = \R^d_+$.)


%% Глава 2
\chapter{������������� �����}
%\begin{flushright}
%��� �������� ���� ������� �������� ������\\ \emph{�.~������}
%\end{flushright}

� ���� ����� ���������� ����������� � ������������� ��������� �����, �������� ������� ��� ��������� ����������� ����������� � ����������� ������.
�~��������~\ref{sec:graphs} �~\ref{sec:polytopes} �������� ����������� ������� 
������ ������ � ������ �������� ��������������, ��������������.
�~�������~\ref{sec:complexity} ���������� ������� ������ ��������� ����� � ���������� �, � ���������, ������ NP-������ �����. ������������ ������������ ������ ������������� ����������� ���������� � �������~\ref{sec:CO}.
������� ������������� (��������) ������ �������� � �������~\ref{sec:ProblemPolytopes}, ��� �� ���������� �������������� �������. ������~\ref{sec:Survey} �������� ������ ��������� ����������� �� ������ ����.
� ���������, � ��� ������ ������������ ���������� � ��������� ������ �������������� �����, � ����� � ����������� ������������� ��������������.
�~�������~\ref{sec:questions} ������������� ����� �������, ������ �� ������� ����� ������������ � ����������� ������.


\section{��������� � �����}
\label{sec:graphs}

\subsection{���������}

%��������� ����������� ����� ���������� ����� $\N$.
��� ��������� $\{1,2,\dots,n\}$, ��� $n\in \N$, ����� ������������ ������������~$[n]$. 
����� ����� %���������� �����, �� ������������� 
��������������� ����� $x\in \R$ ���������� $\lfloor x \rfloor$.
���������� �����, ������� ��� ������ $x\in\R$, ���������� $\lceil x\rceil$.

\emph{�������������� ���������} ���� �������� $X$ � $Y$ ���������� ���������
\[
X \symdiff Y = (Y \setminus X) \cup (X \setminus Y).
\]
\begin{property}\label{prop:symdiff}
�������������� �������� �������� ���������� ����������:
\begin{enumerate}
	\item $X \symdiff Y = \emptyset \iff X = Y$.
	\item ��������� ��������� $X \symdiff Y \symdiff Z$ �� ������� �� ������������ �������� � ������� ���������� �������� (��������������� � ���������������).
	\item $X \symdiff Y = Z \iff X \symdiff Z = Y$.
\end{enumerate}
\end{property}

����� $E$ "--- ��������� �������� ���������. 
��������� ���� ����������� ��������� $E$ ������������ $2^E$.
\emph{������������������ ��������} ������������ $T \in 2^E$ ���������� 0/1"~������ $v = \chi(T) \in \{0,1\}^E$ � ������������
\[
v_{e} = \begin{cases}
1,& \text{���� $e\in T$,}\\
0,& \text{���� $e\in E\setminus T$.}
\end{cases}
\]
���� �� �������� ��������� $E$ �������������, �� ������������������ ������ ����� ���� ��������� ��� $v = \chi(T) \in \{0,1\}^{|E|}$ � ������������
\[
v_{i} = \begin{cases}
1,& \text{���� $e_i\in T$,}\\
0,& \text{���� $e_i\in E\setminus T$.}
\end{cases}
\]
%����� �������, ������� ������ 0/1-������������� � $\R^d$ ����� ���������������� ��� ������������������ ������� ��������� ����������� $d$-����������� ��������� $S$.

����� $E = \{e_1, \ldots, e_m\}$ "--- �������� ���������
� $S = \{S_1, \ldots, S_n\} \subseteq 2^E$.
���������� ������������ ������������ $T \subseteq S$.
���� ������ ������� $e_i \in E$ ����������� �� ����� (�� �����) ��� ������ �� ��������� $T$, �� $T$ ���������� \emph{��������� (���������) ��������� $E$}.
��������, ���������� ������������ ���������, ���������� \emph{���������� ��������� $E$}.
\emph{�������� ����������} ��������� ��������� $E$ � ��������� ��������� $S$ ���������� ������� $A\in\{0,1\}^{m\times n}$ � ����������
\[
A_{ij} = 
\begin{cases}
1, &\text{���� $e_i\in S_j$,}\\
0, &\text{�����.}
\end{cases}
\]


\subsection{�����}
\emph{������} ��� \emph{����������������� ������} ���������� ������������� ���� $G = (V, E)$,
��� $V$ "--- �������� ���������, � $E$ "--- ��������� ��������� �������������� ����������� ��������� $V$.
�������� ��������� $V$ ���������� \emph{���������} ����� $G$, � �������� ��������� $E$ "--- ��� \emph{�������}.
������� $v$ � $u$ ���������� \emph{�������} ����� $\{v,u\}$.
��� ��������������� � ����������� ����� �� �������� ������� ����� (��������� $E$ �� �������� ���������� ���������) � ������ (����� ������ ����� $e \in E$ �������� ������� ���������).
���� ���������� \emph{������,} ���� ������ ���� ��� ������ �������� ����� ����� �����.
������ ���� �� $n$ �������� ������������ $K_n$.

������� $v$ � $u$ ����� $G$ ���������� \emph{��������} � $G$, ���� $\{v,u\} \in E$.
���� �� $\{v,u\} \notin E$, �� ������� $v$ � $u$ ���������� \emph{����������.}
\emph{��������} ������� $v \in G$ ���������� ����� ������� � ��� ������.
������� � ������� �������� ���������� \emph{�������������.}
����, ������� ������ ������� �������� ����� ����, ���������� \emph{����������.}
������������ ������ $V' \subseteq V$ ���������� \emph{������} � ����� $G$, ���� ����� ��� ������� �� $V'$ ������.
������������ ������ (��������) ����� � $G$ ���������� \emph{�������� ������} ����� $G$ � ������������ $\omega(G)$.
������������ ������ $V' \subseteq V$ ���������� \emph{�����������} � ����� $G$, ���� ����� ��� ������� �� $V'$ ��������.

����� $G' = (V', E')$ "--- ��� ���� ����.
����� $G$ � $G'$ ���������� �����������, ���� ���������� �������"=����������� ����������� $f \from V \to V'$ �����,
��� $\{v, u\} \in E$ ����� � ������ �����, ����� $\{f(v), f(u)\} \in E'$.
���� $G'$ ���������� \emph{���������} ����� $G$, ���� $V' \subseteq V$ � $E' \subseteq E$.
�����, ��� ���������, ����� ���� $G'$ ���������� �������� ����� $G$ ����� �������� ��������� ����� $G$.
������� $G'$ ����� $G$ ���������� \emph{�����������}, ���� $\forall v, u \in V'$ �� $\{v, u\} \in E$ ������� $\{v, u\} \in E'$.

\emph{�����} � ����� $G$ ���������� ��������� ����� ���� 
\[P = \{\{v_1, v_2\}, \{v_2, v_3\}, \dots, \{v_{k-1}, v_k\}\},
\] 
��� $v_1$, $v_2$, \dots, $v_k$ "--- ������� ��������� �������, $k \ge 2$.
� ����� ������ ����� ��������, ��� ���� $P$ \emph{���������} ������� $v_1$ � $v_k$ � �������� ��� \emph{$v_1$-$v_k$ �����}.
���� ���������� \emph{��\'�����}, ���� ����� ��� ��� ������� ��������� ��������� ���� � ���� �����.
���� � ����� $G$ ���������� \emph{�������������}, ���� ������ ������� ����� ����������� ���� �� ������ ��� �����.
%\emph{������} ���� ���������� ����� ������������ ��� �����.
\emph{�����������} ����� ��������� $v$ � $u$ � ����� $G$ ���������� ���������� ����� ����� � ����������� �� ����,
���� �� ����� ���� �� ����������, �� ���������� ���������� ������ $+\infty$.
\emph{���������} $\diam(G)$ ����� $G$ ���������� ���������� ���������� ����� ��� ��������� (����� �������������� ����� ���� ��� ������).

\emph{������} � ����� $G$ ���������� ��������� ����� ���� 
\[
C = \{\{v_1, v_2\}, \{v_2, v_3\}, \dots, \{v_{k-1}, v_k\}, \{v_k, v_1\}\},
\] 
��� $v_1$, $v_2$, \dots, $v_k$ "--- ������� ��������� �������, $k \ge 3$.
���� � ����� ���������� \emph{�������������}, ���� ������ ������� ����� ����������� ����� ���� ������ ����� �����.
���� � ����� ���� ����������� ����, �� � ��� ���� ���������� \emph{�������������.}
���� ��� ������ ���������� \emph{�����}, � ������� ��� "--- \emph{�������}.

������� $v$ � ����� $e$ ���������� \emph{������������}, ���� $v\in e$.
\emph{�������� ����������} ������"=����� ����� $G = (V,E)$ ���������� ������� $M \in \{0,1\}^{n\times k}$, $n = |V|$, $k = |E|$, �������� ������� ������������ ��������� �������:
\[
M_{ij} = 
\begin{cases}
1, &\text{���� $v_i\in e_j$,}\\
0, &\text{�����.}
\end{cases}
\]

����� ��������, ��� ������������ ����� $E' \subseteq E$ \emph{���������} ������� $V$,
���� ������ ������� $v \in V$ ���������� ���� �� ������ ����� �� $E'$.
����������, ������������ ������ $V' \subseteq V$ \emph{���������} ����� $E$,
���� ������ ����� $e \in E$ ���������� ���� �� ����� ������� �� $V'$.

��� ����� � ����� ���������� \emph{��������}, ���� ��� �������� ����� �������, � ��������� ������ ��� ���������� \emph{����������.}
��������� ������� ��������� ����� ����� ���������� \emph{��������������}. 
�������������, ����������� ��� ������� �����, ���������� \emph{�����������.} 
����� �������, ����������� ������������� ����� ���� ������ � ������ � ������ ������ ������.

\emph{��������} � ����� $G = (V, E)$ ���������� ��������� ����� ���� $\delta(U) = \Set*{\{u,v\} \in E \given u\in U,\ v\in V \setminus U}$, ��� $U \subseteq V$. �� ����������� �������, ��� $\delta(U) = \delta(V \setminus U)$.
������ $\delta(U)$ ���������� \emph{$s$-$t$ ��������}, ���� $s \in U$ � $t \in V\setminus U$\label{def:stcut}.
\emph{���������� ���������} $X$ ���������� ����� ��� ������� ���������������� �����������, ����������� ������� ��������� � $X$.
���� $G = (V, E)$ ���������� \emph{����������}, ���� ��������� ��� ������ $V$ ����� ������� �� ��� \emph{����} $U$ � $V \setminus U$ ���, ��� $\delta(U) = E$. 
���������� ���� ���������� \emph{������ ����������}, ���� $\{u, v\} \in E$ ��� ����� $u \in U$ � $v \in V \setminus U$.


���� $G = (V,E)$ ���������� \emph{�������"=����������}, ���� �� ��������� ��� ����� $E$ ������ ������� ����� $f \from E \to \R$.
����� $f(e)$ ���������� \emph{�����} ����� $e \in E$.
\emph{����� ������������} $E' \subseteq E$ ��� \emph{��������} $G' = (V', E')$ �������"=����������� ����� $G$ ���������� ����� ����� �������� � ���� �����.
���� $G$ ���������� \emph{��������"=����������}, ���� ������ ������� $g \from V \to \R$.
� ����� ������ ����� $g(v)$ ���������� \emph{�����} ������� $v \in V$, � \emph{����� ������������} $V' \subseteq V$ ���������� ����� ����� �������� � ���� ������.

\subsection{�������}

\emph{��������������� ������} ��� \emph{��������} ���������� ������������� ���� $D = (V, A)$, ��� $V$ "--- �������� ���������, ���������� \emph{���������� ������}, $A$ "--- ��������� ��������� ������������� ��� ������, ���������� \emph{������}. 
�����, ��� � ��� ������, ����� ������������, ��� � $A$ ��� ������� ��� � ������.
����������� ������������� ���� ������� ��� ������ ����������� (������ � ���������� �����������) �� �������.

����� $(v, u) \in A$.
������� $v$ ���������� \emph{�������} ���� $(v, u)$, � ������� $u$ "--- �� \emph{������}.
������ �� ���� ������ � ���� $(v, u)$ ���������� \emph{������������} ���� �����.
������ ����� �������� \emph{������,} ���� ������ ������������� ���� ��� ������ �������� ���� ����� ����� (�� ���� $|A| = |V| (|V| - 1)$).
������ ���������� \emph{��������,} ���� ��� ����� (���������������) ���� ������ $u,v \in V$ ����� ���� �� ��� $(u, v)$ � $(v, u)$ ����������� $A$.

\emph{�������} ��� ������ \emph{�����} � ������� $D$ ����� �������� ��������� ��� ���� 
\[P = \{(v_1, v_2), (v_2, v_3), \dots, (v_{k-1}, v_k)\},
\] 
��� $v_1$, $v_2$, \dots, $v_k$ "--- ������� ��������� �������, $k \ge 2$.
������� $v_1$ � $v_k$ ����������, ��������������, \emph{�������} � \emph{������} ���� $P$, 
� ������� $v_2$, \dots, $v_k$ "--- ��� \emph{����������� ���������}.

\emph{��������} � ������� $D$ ���������� ��������� ��� ���� 
\[
C = \{(v_1, v_2), (v_2, v_3), \dots, (v_{k-1}, v_k), (v_k, v_1)\},
\] 
��� $v_1$, $v_2$, \dots, $v_k$ "--- ������� ��������� �������, $k \ge 2$.
������, �� ���������� ��������, ���������� \emph{������������.}

\emph{��������} � ������� $D = (V, A)$ ���������� ��������� ��� ���� $\delta^+(U) = \Set*{(u,v) \in A \given u\in U,\ v\in V \setminus U}$, ��� $U \subseteq V$. ������ $\delta^+(U)$ ���������� \emph{$s$-$t$ ��������}, ���� $s \in U$ � $t \in V\setminus U$\label{def:stdicut}.


������ $D = (V, A)$ ���������� \emph{������������}, ���� �� ������� $(v, u) \in A$ � $(u, w) \in A$ ������� $(v, w) \in A$.
��� ��� �� �� ������������� ����� � �������, �� �� �������������� ������� ������������.
����� �������, ��������� ��� ������������� ����� ������ ��������� ������� �� ��������� ������ ����� �, ��������, ������ ��������� ������� ����� ���� ����������� ���������� ��� ���������� ������������� �����.
�� ��������, ������������ ������ ������ �������� ������� �� ��������� ������ �, ����� ����, ������ �������� ������� ���������� ������������ ��������� ������������ ��������.
������� �����, ������ � ��������� (��������) ������� �� ����� ����� ������������� ��������������� ������������ ������ (������).


\section{�������������}
\label{sec:polytopes}

� ���� ������� ������������� ��������� ���������������� ������� � ����� ������ �������� ��������������.
��� �� ��������� ����� �������������� ������������ ����������~\cite{Emelichev:1981} �~\cite{ZieglerBook}.


��� $\R^d$ ����� �������� ������������ ���� ������"=�������� ����� $d$ � ������������� ������������. 
���� ������-������� �� $\R^d$ ����� �������� ���������� �������: $\bm{x}, \bm{x_1}, \bm{y}, \bm{z} \in \R^d$.
������"=�������, ������������ �� ����� ����� ��� �� ����� ������, ����� ���������� $\bm{0}$ � $\bm{1}$ ��������������
(�� ����������� ���� �� ��������� �����, ��� ��� ����� ��������������).
��������� �������, ���������� ������������ ����� � $\R^d$, ���������� $\bm{e_1}$, \dots, $\bm{e_d}$
(����� �������, $\sum_{i\in[d]} \bm{e_i} = \bm{1}$).
������ ������������ ��������~\cite{ZieglerBook},  ������"=������� ����� ����� ���������� �������.


\emph{���������������} �~$\R^d$ ���������� ���������
\[
H(\bm{a},b) = \left\{\bm{x}\in\R^d \mid \bm{a}^T \bm{x} = b\right\}, 
%\qquad \bm{a}\in\R^d, \ \bm{a} \ne \bm{0}, \ b\in\R,
\]
��� $\bm{a} \in \R^d$ "--- ������ ������� ��������������, $\bm{a} \ne \bm{0}$, � ����� $b \in \R$ ���������� �������� �������� �������������� ������������ ������ ���������,
$\bm{a}^T \bm{x}$ "--- ��������� ������������ ������"=������ $\bm{a}^T$ �� ������"=������� $\bm{x}$.


�������� ���������� $\sum_{i\in[n]} \lambda_i \bm{x_i}$ ����� $\bm{x_1}$, \dots, $\bm{x_n}$ �� $\R^d$,
��� $\lambda_i \in \R$, $i\in[n]$,
���������� \emph{�������� �����������},
���� $\sum_{i\in[n]} \lambda_i = 1$.
\emph{�������� ���������} $\aff(X)$ ��������� ��������� $X \subseteq \R^d$ ���������� ��������� ���� �������� ���������� ��������� ������ ����� �� $X$.
��������� ����� ���������� \emph{������� �����������}, ���� �� ���� ����� ����� ��������� �� ����������� �������� �������� ��������� ��� �����.
\emph{�������� ������������} ��������� $X$ ���������� �������� ������� ������������ ������������ $S \subseteq X$ ����� ����, ��� ������� $\aff(S) = \aff(X)$. � ���������, �������� ����������� ������� ��������� ����� $-1$.
\emph{�����������} $\dim(X)$ ��������� $X$ �������� ������ ��� �������� �����������. (��� ��� ����� ��������������� ������ �������� ��������� �, � ���������, �������� �������� �������� ��������, �� ������������ � ������� ������������� ������������� ����������� �� ���������.)
��������� $X \in \R^d$ ���������� \emph{�������� ����������������}, ���� ������ � ������ ������ ����� ���������� ������� ��� �������� ��� �� �������� ����������.
�������������� �������� �������� ��������� ���������������.
����� ����, ������ �������� ��������������� ����������� $d-k$ � $\R^d$ ����� ���� ������������ ��� ����������� $k$ ���������������~\cite{Emelichev:1981}.

���������� $\sum_{i\in[n]} \lambda_i \bm{x_i}$ ����� $\bm{x_1}$, \dots, $\bm{x_n}$ �� $\R^d$,
��� $\lambda_i \ge 0$, $i\in[n]$,
���������� \emph{���������� �����������}.
\emph{���������� ���������} ��������� ��������� $X = \{\bm{x_1}, \dots, \bm{x_n}\} \subset \R^d$ ���������� ��������� ���� ���������� ���������� ��� �����:
\[
\cone(X) = \left\{\sum_{i=1}^n \lambda_i \bm{x_i} \;\bigg|\; \lambda_i \ge 0\right\}.
\]
�������� ����� ������� ������������� ������
\[
\R^d_+ = \Set{\bm{x} \in \R^d \given \bm{x} \ge \bm{0}} = \cone\{\bm{e_1},\dots,\bm{e_d}\}.
\]

�������� ���������� $\sum_{i\in[n]} \lambda_i \bm{x_i}$
����� $\bm{x_1}$, \dots, $\bm{x_n}$ �� $\R^d$ ���������� \emph{��������}, ���� $\lambda_i \ge 0$, $i\in[n]$.
��������� $X \subseteq \R^d$ ���������� \emph{��������}, ���� ��� ����� ���� ����� $\bm{x}, \bm{y} \in X$ 
��� �������� ��� �� �������� ����������.
%��� ��������� �������� ����������� �� ������� \([\bm{x}, \bm{y}] = \{\lambda \bm{x} + (1 - \lambda) \bm{y} \mid \lambda \in [0, 1]\}\).
������� �������� ��������� ��������� ����� ������� \emph{��������� ����������������} 
\[
H^+(\bm{a},b) = \{\bm{x}\in\R^d \mid \bm{a}^T \bm{x} \ge b\},
\]
������������ ��������������� $H(\bm{a},b)$.
����� ��������� ��������� ���������� \emph{�������}, ���� ��� �� �������� �������� ����������� ������� ���� ������ ����� ����� ���������.
��������� ���� ������� ����� ��������� $X$ ������������ $\ext(X)$.
\emph{�������� ���������} ��������� ��������� $X = \{\bm{x_1}, \dots, \bm{x_n}\}
\subset \R^d$ ���������� ��������� ���� �������� ���������� ��� �����:
\[
\conv(X) = \left\{ \sum_{i=1}^n \lambda_i \bm{x_i} \;\bigg|\; \sum_{i=1}^n \lambda_i = 1, \ \lambda_i \ge 0, \ \lambda_i \in \R\right\}.
\]
�������� �������� (�������������) ��������� $X \subseteq \R^d$ ������������ ����� ����������� ���� �������� �������� �������� ������� ����� �� $X$. 
%�������� ������� �����������~\cite{Caratheodory:1911}, ��� ������� $X \subseteq \R^d$,
%\[
%\conv(X) = \left\{\sum_{i=1}^n \lambda_i \bm{x_i} \;\bigg|\; 
%\{\bm{x_1}, \dots, \bm{x_n}\} \subseteq X, \ n \le d+1, \  \sum_{i=1}^n \lambda_i = 1, \ \lambda_i \ge 0\right\}.\]

� �������� �������� �������� ����� ������� ������� \emph{����� �����������} ���� �������� �������� $X\subseteq \R^d$ � $Y\subseteq \R^d$:
\[
X+Y = \Set*{x+y \given x \in X, \ y\in Y}.
\]
� ���������, ��� ����� $X$ � $Y$
\[
\conv(X+Y) = \conv(X) + \conv(Y).
\]

\emph{�������� ��������������} ���������� �������� �������� ��������� ��������� �����. %� ��������� ������������ $\R^d$.
��� ��� ����� ���� ������ ������ � �������� ��������������, ����� �������� ����� ����������.
\emph{���������} ���������� ����������� ��������� ����� ��������� ���������������, ���, ������� �������, ��������� ������� ������� �������� ����������
\(A\bm{x} \ge \bm{b}\), ��� $A \in \R^{m\times d}$, $\bm{x}\in \R^d$, $\bm{b}\in \R^m$.

\begin{theorem}[�����--����������]
	��������� $P$ �������� �������������� ����� � ������ �����, ����� $P$ "--- ������������ �������.
\end{theorem}

����� �������, ������ ������������ ����� ���� ������ ����� ��������������� ���������:
\begin{enumerate}
	\item ��� �������� �������� ��������� ����� ������. � ���� ������ ��������� ������ ���������� \emph{$V$-���������} �������������.
	\item ��� ����������� ��������� ����� ��������� ���������������. ����� ��������������� ������� �������� ���������� (�, ��������, ���������) ���������� ��� \emph{$H$-���������.}
\end{enumerate}
�� �� ����� ����� � � ��������� �������� ��� ��������� ���������~\cite{ZieglerBook}. \emph{$V$-��������� ��������} $P$ ���������� ������������ ��������� ��������� �����~$X$ �~��������� ��������� �������� $Y$ �����, ���
\[
P = \conv(X) + \cone(Y).
\]

$H$-�������� ����� ���������� \emph{��������}, \emph{��������} ��� \emph{�������} ���������~\cite{Schrijver:1998, Zolotykh:2012}.
� ���� �������, $V$-�������� �������� \emph{���������} ��� \emph{����������} ���������.
������ ���������� $H$-�������� ������������� (��������) �� ��� $V$-�������� ���������� \emph{������� ���������� �������� ��������}.
��� ������������ (�����������) ������ ���������� $V$-�������� �� $H$-�������� (�� ���� ���������� ������) � ������� ��� ��� ������ ���������� \emph{������� ���������� ������������� �������� �������������}.
��� ������ �������� �������������� �������~\cite{Khachiyan:2008}.
������������� ����� ��������� � ��������� ����� �������� � ������� ����� ����� � �����������~\cite{BastrakovDiss:2016}.

%\emph{������������} $\dim(P)$ ������������� $P$ ���������� ����������� ������������ ����������� ��� ��������� ���������������.
������ ���������~\cite{Grunbaum:2003, Emelichev:1981, ZieglerBook}, 
�������� ������������ ����������� $d$ ����� �������� \emph{$d$"~��������������.}
���������� �������� $d$-������������� �������� \emph{$d$-��������}, �������������� ����� �������� �������� $d+1$ ������� ����������� ����� � $\R^n$, $n\ge d$.

����� ��������, ��� ����������� $\bm{a}^T \bm{x} \ge b$ \emph{���������} ��� ��������� $X \subseteq \R^d$, 
���� ��� ��������� ��� ���� $\bm{x}\in X$.
\emph{������} ������������� $P$ ���������� ����� ��������� ���� 
\[
F = \{\bm{x} \in P \mid \bm{a}^T \bm{x} = b\},
\]
��� ����������� $\bm{a}^T \bm{x} \ge b$ ��������� ��� $P$.
�� ����, ��� $\bm{a}$ ����� ���� ������~$\bm{0}$, �������, ��� ������ ��������� � ��� ������������ $P$ �������� �������~$P$, ��� ���������� \emph{��������������} ������� $P$.
��������� ����� ���������� \emph{������������}~\cite{Emelichev:1981}.

���� �������������� $H(\bm{a},b)$ ����� ���� �� ���� ����� ����� � �������������� $P \subseteq \R^d$ 
� ��� ���� $P$ ������� ����� � ���������������� $H^+(\bm{a},b)$, �� �������������� $H(\bm{a},b)$ � ���������������� $H^+(\bm{a},b)$ ���������� \emph{��������} �~$P$.
����� �������, ������ ����������� ����� ������������� ���� ����������� ������������� � ��������� ��� ������� ���������������.

\emph{������������} $\dim(F)$ ����� $F$ ���������� ����������� ������������ ����������� � ��������� ���������������.
����� ����������� $k$ ���������� \emph{$k$-�������}, 0-�����~--- \emph{���������} �������������, 1-�����~--- ��� \emph{�������}.
�������� ������������, ��� ��������� ������� ����� ������������� ��������� � ���������� ��� ������.
$(d-1)$-����� $d$-������������� ���������� \emph{������������}. 
��� $(d-2)$-������ � ������������� ����������� ��� ����������� �������, 
�� �������� ������������~\cite{ZieglerBook}, ��������~\cite{Bastrakov:2011}, �������~\cite{Deza:2001} (�� ����. ridge).
�� ����� �������������� �������� \emph{����}.

� ��������, ����� ������ � ����������� $d$-��������� ����� $d+1$, � ����� ��� $k$-������, $k \in [d-2]$, ����� $\binom{d+1}{k+1}$. 
����� �������, ����� ���� ������ $d$-��������� ����� $2^{d+1}$.
%(���� ������������ ���� ��� ����������� � ����������.)

\emph{������} ��� \emph{1-��������} ������������� ���������� ��������� ��� ������ � ����� (������, ��� ������, ���������� ����� �������������).
\emph{����"=������} ������������� ����� �������� ��������� ��� ����������� � ������ (������, ��� �����������, ���������� �����).

��� ������ ������������� ����������� ��������� �����������.

\begin{prop}[\cite{ZieglerBook}]
	����� $P$ "--- ������������, � $V = \ext(P)$ "---  ��������� ��� ������. �����:
	\begin{enumerate}
		\item $P = \conv(V)$. %(������������ �������� �������� ��������� ����� ������).
		\item ���� $F$~--- ��������� ����� ������������� $P$, �� $F$~--- ���� ������������ � $\ext(F) = F \cap V$.
		\item ����� ����������� ������ ������������� $P$~--- ����� ����� $P$.
		\item ����� ����� ������������� ����� �������� ��� ������.
	\end{enumerate}	
\end{prop}

������������ ���������� \emph{��������������}, ���� ��� ��� ���������� �������� �����������.
������������ ���������� \emph{�������}, ���� ������ ��� ������� ����������� ����� $d$ �����������, ��� $d$ "--- ����������� �������������.
C�������� � �������� �������������� �������� ��������� ������� �, ������������, �������������� ��������������. 
�������� �������� ������������� �������� $d$-������ \emph{0/1-��������} (���, ������, \emph{$d$-���})~\cite{ZieglerBook}:
\[
\Cube_d = \Set*{\bm{x} = (x_1,\dots,x_d)\in\R^d \given 0 \le x_i \le 1, \ i\in[d]} 
= \conv\left\{\{0,1\}^d\right\}.
\]
�������� ��������������� ������������� ����� ������� $d$-������ \emph{������������}, � ������������� ���������� ���������� \emph{���������}~\cite{ZieglerBook}:
\[
\Cross_d = \Set*{\bm{x} = (x_1,\dots,x_d)\in\R^d \given \ \sum_{i \in [d]} |x_i| \le 1} 
= \conv\left\{\bm{e_1},-\bm{e_1},\dots,\bm{e_d},-\bm{e_d}\right\}.
\]

������, ��� ���������� �������� ����������� ��������� $\{0, 1\}$, ���������� \emph{0/1"~��������}.
������������, ��� ������� �������� �������� 0/1"~���������, ���������� \emph{0/1"~��������������}.
������� �������, 0/1"~������������ ������������ �����
�������� �������� ���������� ������������ ������ ���� $\Cube_d$.
������� �����, ��� $\ext\conv(X) = X$ ��� ������ $X \subseteq \{0, 1\}^d$.

����� ��������, ��� ��������� �� $n \ge d+1$ ����� � $\R^d$ ��������� \emph{� ����� ���������}, ���� ������� $d+1$ �� ��� �� ����� � ����� ��������������~\cite{ZieglerBook}.
������� �������� ���������� 
\(A\bm{x} \ge \bm{b}\), ��� $A \in \R^{m\times d}$, $\bm{x}\in \R^d$, $\bm{b}\in \R^m$, $m \ge d+1$, ���������� \emph{�����}, ���� ��� ������ $\bm{x}\in \R^d$ ������������ ���������� � ��������� �� ����� ��� $d$ �� ���� ����������.

������ �������������� � ������� �������������� ����� � ��������� ������~\cite{ZieglerBook}.
�������� �������� ��������� �����, ����������� � ����� ���������, �������� �������������� ��������������.
����������, ������� �������� ���������� ������ ����, 
��������� ������� ������� ����������, ���������� ������� ������������.
��������������, ����� ������������ ����� ���� ������������ � �������������� �� ���� ���������� ��������� (�����������) ��� ������.
����������, ����� ������������ ������� ����� ���� ������������ � ������� ������������ �� ���� ���������� ��������� (�����������) ������������� ������� ����������� ��� �������� ����������.

������������ ���������� \emph{$k$-�����������} ($k\in \N$), ���� ����� $k$ ��� ������ �������� ���������� ������ ��������� ����������� ����� ����� �������������.
� ���������, ����� ������������ �������� 1-�����������, � $d$"~�������� �������� $k$-����������� ��� $k \in [d]$.

�������� ���������� ��������� $k$-����������� ��������������, ������������ �� ���������, �������� ����������� �������������~\cite{Grunbaum:2003,Emelichev:1981,ZieglerBook}.
��������� ������ ������������ ������������� ������������ ��������� �������:
\[
\CP_d(T) = \Set*{(t, t^2, \dots, t^d) \in \R^d \given t \in T},
\]
��� ��������� $T \subset \R$ "--- �������.
(�������, ��� $\CP_d(T)$ �������� ���������� ��� $|T| \le d+1$.)
��������, ��� ��� ������������� �������������, $\lfloor d/2\rfloor$"=����������
(�� ���� ����� ������������ ������� ��������� ����� $d$"~��������������, �� ���������� �����������)
� �������� ���������� ������ ������ (������ �����������) ����� ���� $d$"~�������������� � ��� �� ������ ������ $n = |T|$.

���� �� ����� ������������ ��� $\CP_d(T)$ ������� ���� ����������� � ��� �������, ����� ��������� $T$ ����� ����������� ���.
��������, ���� $T$ "--- ��������� ����� ����� ������� $[a,b]$, 
�� $\CP_d(T)$ ����� ���������� ����� ������������� $\CP_d(a, b)$.

����� �� ����� ������� � ������ � ��� ������ �������� �������� ��� ��������������� �������� �������� ����������� ��������.
����� $P$ "--- ��������� $d$-������������, ��������� � ������������ $\R^n$, $n > d$,
�~����� ����� $\bm{x} \in \R^n$ ��~����������� �������� �������� ����� �������������.
\emph{���������} ��� $P$ ���������� �������� �������� $\pyr(P) = \conv\{P\cup \bm{x}\}$.
������������ $P$ ���������� \emph{����������} �������� $\pyr(P)$, � ����� $\bm{x}$ "--- �� \emph{�������} (��� \emph{�������� ��������}).
������� �������� $\pyr(P)$ �������� ��� ����� �� ��������� $P$, � ����� ��� �������� ��� ��� �������.
� ���������, ����� ����������� �������� ����� �� ������� ������ ����� ����������� ���������, �� �� ����� � ��� ����� ������.
�������� ��� �������������� �������������� ����� �������� �������������� ��������������.
�������� ��� $k$-����������� �������������� ���� $k$-����������.


\subsection{������� ������ � ������� ����������}

\emph{�������� ������} ������������� $P$ ���������� ��������� $L(P)$ ���� ��� ������, 
�������� ������������� �� ���������.

� �������� ������� ���������� 4-������ �������� ������������ $P$, ���������� ���������, ��������� ������� "--- 
3-������ ��� ��� ����� ������� (��. ���.~\ref{fig:cube7}).
$P$ ����� 8 ������, 8 ����������� (���� �� ��� ���������� �� ���.~\ref{fig:cube7}), 19 ����� � 19 ������.
\begin{figure}[hb]%
	\centering
	%\includegraphics[width=\columnwidth]{filename}%
	\begin{tikzpicture}[scale=2.0, line join = round]
	\coordinate (4) at (0,0,1);
	\coordinate (6) at (1,0,0);
	\coordinate (7) at (0,0,0);
	\coordinate (2) at (1,0,1);
	\coordinate (1) at (0,1,1);
	\coordinate (3) at (1,1,0);
	\coordinate (5) at (0,1,0);
	\draw (6) -- (3) -- (2) -- (1) -- (4) (5) -- (3) -- (1) -- (5) (6) -- (2) -- (4);
	\draw[dashed, thin] (6) -- (7) -- (5) (7) -- (4);
	\foreach \i in {1,...,7} {\draw (\i) node[circle, draw, fill = white, inner sep = 1pt] {\i};}
	\end{tikzpicture}
	\caption{3-������ ��� ��� ����� �������}%
	\label{fig:cube7}%
\end{figure}

������� ������ ������ ��������������� ����������� ��������� �����, 
�������������� ����� ������������ �� ��������� ����,
������� �������� ������������� ����� �������������.
������� �� ��������� ����� ��������� ��, � ������ �� ���� ������ $f$ � $g$,
��� ������� ������������ ����������� ��������� �������:
\begin{compactenum}
	\item[1)] $f$ �������� ������ $g$ (� ���� ������ $g$ ����������� �� ��������� ���� $f$);
	\item[2)] �� ���������� ����� $h$, ������������ �� $f$ � $g$, � �����, ��� $f$~--- ����� $h$ � $h$~--- ����� $g$.
\end{compactenum}
������� ������ �������� ��� 3-������ ����� ��� ������� ���������� �� ���.~\ref{fig:cube7Hasse}.

\begin{figure}%
	\centering
	%\includegraphics[width=\columnwidth]{filename}%
	\begin{tikzpicture}[x=4mm,y=13mm,new set=import nodes, >=stealth']
	\begin{scope}[nodes={set=import nodes}] % make all nodes part of this set
	%\node (p) at (0,4) {$p_1$};
	\node[circle, draw, inner sep = 0pt, minimum size = 12pt] (polytope) at (0,4) {};
	\node[circle, draw, inner sep = 0pt, minimum size = 12pt] (emptyset) at (0,-1) {};
	\foreach \i in {1,...,8} {
		\node[circle, draw, inner sep = 0pt, minimum size = 14pt] (f\i) at ({(\i-4.5)*2.0},3) {$f_{\i}$};
	}	
	\foreach \i in {1,...,19} {
		\node[circle, draw, inner sep = 2pt] (r\i) at ({\i - 10},2) {};
	}	
	\foreach \i in {1,...,19} {
		\node[circle, draw, inner sep = 2pt] (e\i) at ({\i - 10},1) {};
	}	
	\foreach \i in {1,...,8} {
		\node[circle, draw, inner sep = 0pt, minimum size = 14pt] (v\i) at ({(\i-4.5)*2.0},0) {$v_{\i}$};
	}	
	\end{scope}
	\draw (15, 4) node[left] {���� ������������};
	\draw (15, 3) node[left] {����������};
	\draw (15, 2) node[left] {�����};
	\draw (15, 1) node[left] {�����};
	\draw (15, 0) node[left] {�������};
	\draw (15, -1) node[left] {������ ���������};
	\graph {
		(import nodes); % "import" the nodes
		polytope -- {f1, f2, f3, f4, f5, f6, f7, f8};
		emptyset -- {v1, v2, v3, v4, v5, v6, v7, v8};
		e1 -- {v1, v2}; e2 -- {v1, v3}; e3 -- {v2, v3}; e4 -- {v2, v4}; e5 -- {v1, v4};
		e6 -- {v1, v5}; e7 -- {v3, v5}; e8 -- {v3, v6}; e9 -- {v2, v6}; e10 -- {v4, v7};
		e11 -- {v5, v7}; e12 -- {v6, v7};
		\foreach \i/\j in {1/13,2/14,3/15,4/16,5/17,6/18,7/19} {e\j -- {v\i, v8};};
		%\foreach \i in {1,...,14} {r\i -- e\i;}
		r1 -- {e1, e2, e3}; r2 -- {e1, e4, e5}; r3 -- {e2, e6, e7}; r4 -- {e3, e8, e9};
		r5 -- {e4, e9, e10, e12}; r6 -- {e5, e6, e10, e11}; r7 -- {e7, e8, e11, e12};
		r8 -- {e1, e13, e14}; r9 -- {e2, e14, e15}; r10 -- {e3, e13, e15}; r11 -- {e4, e13, e16}; r12 -- {e5, e14, e16};
		r13 -- {e6, e14, e17}; r14 -- {e7, e15, e17}; r15 -- {e8, e15, e18}; r16 -- {e9, e13, e18}; r17 -- {e10, e16, e19};
		r18 -- {e11, e17, e19}; r19 -- {e12, e18, e19};
		f1 -- {r1, r2, r3, r4, r5, r6, r7};
		f2 -- {r1, r8, r9, r10}; f3 -- {r2, r8, r11, r12}; f4 -- {r3, r9, r13, r14}; f5 -- {r4, r10, r15, r16};
		f7 -- {r5, r11, r16, r17, r19}; f6 -- {r6, r12, r13, r17, r18}; f8 -- {r7, r14, r15, r18, r19};
	};
	\end{tikzpicture}
	\caption{��������� ����� ������� ������ �������� ��� 3-������ ����� ��� �������}%
	\label{fig:cube7Hasse}%
\end{figure}

��� ������������� ���������� \emph{������������ ��������������}, ���� �� ������� ������ ���������.
���� ��� ������������� ������������ ������������, �� �������, ��� ��� �������� ��������������� ������ \emph{�������������� ����}.
�������� � �������� �������������� �������������, ���������� ������������ ��� �������� ������, ����� �������� \emph{��������������}.
� ���������, ����������� �������������, ����� ��� ������ � ����� ����������� �������� �������������� ����������������, � ����������������, �������� � $k$-������������� "--- �������������� ����������.

��� ������������� ���������� \emph{�������������} ���� � �����, ���� �� ������� ������ �������������.
� ���������, ���� $P$ � $Q$ �����������, �� ������� $P$ ������������� ����������� $Q$, ����� $P$ "--- ������ $Q$ �~�.\,�.
�������� ������������ �������������� ����� ������� $d$-��� � $d$-������ �������.
������������, ������������ �� ���.~\ref{fig:cube7}, ����������� ������ ����.
��� ����� �������� ������������� ������ ���� ������������� �������� $d$-��������.
������, ������������, ������������ ���������������, �������� �������, 
� ������������, ������������ ��������, "--- ��������������~\cite{Emelichev:1981}.


����� $d$-������������ $P$ ����� � ���� $P = \conv(V)$ � $\bm{0}$ �������� ���������� ������ ����� �������������. (���������� ���������� ������� ������ ����� �������� �� ���� �������� ��������.)
\emph{�������} � $P$ ���������� ������������
\[
P^* = \{\bm{x}\in \R^d \mid \bm{x}^T V \le \bm{1}^T\}.
\] 
������ $P^*$ �������� �������� ������������� � $P$ �������������~\cite{Emelichev:1981,ZieglerBook}.


����� $P$ "--- ��������� ������������, $V = \{v_1, \dots, v_n\}$~--- ��������� ��� ������,
� $F = \{F_1, \dots, F_k\}$~--- ��������� ��� �����������.
����� \emph{������� ���������� ������"=�����������} $M=(m_{ij})\in \{0,1\}^{n\times k}$ 
������������� $P$ ������������ ��������� �������:
\[
m_{ij} = \begin{cases}
1, & \text{���� } v_i \in F_j,\\
0, & \text{�����.}
\end{cases}
\]
������� $M^T$ ���������� \emph{�������� ���������� �����������"=������.}

������� ������ ������������� ���������� ����������������� �� ��� ������� ���������� ������"=�����������. 
(���� �� �������� ����������� ���������� ������� ���� ������ ������ �~\cite{KaibelP:02}.)
���, ��������, ��������� ����� �� ���.~\ref{fig:cube7Hasse} ����������������� �� ������� ����������
\begin{equation}
\begin{pmatrix}
1 & 1 & 1 & 1 & 0 & 1 & 0 & 0 \\
1 & 1 & 1 & 0 & 1 & 0 & 1 & 0 \\
1 & 1 & 0 & 1 & 1 & 0 & 0 & 1 \\
1 & 0 & 1 & 0 & 0 & 1 & 1 & 0 \\
1 & 0 & 0 & 1 & 0 & 1 & 0 & 1 \\
1 & 0 & 0 & 0 & 1 & 0 & 1 & 1 \\
1 & 0 & 0 & 0 & 0 & 1 & 1 & 1 \\
0 & 1 & 1 & 1 & 1 & 1 & 1 & 1 \\
\end{pmatrix}
\label{eq:Minc}
\end{equation}
����� �������, ��� ������������� �������� ������������� ���������� ������������ �� ��� ������� ����������, � ����� ������������ ����� �/��� �������� ���� ������� �� ������ ���� �������.
����� ��������, ��� ������� ���������� ������"=����������� ������������ �������������� ������������� ���� � ����� ��������� ���������������� �, ��������, ������������� ����� �/��� ��������.
��������, ��������� ������������ �������� (��������������) ��������� � ������� \eqref{eq:Minc} ������� � �������������� ���������������� ������������� ������ ����.


\subsection{�������� � ����������� �������������� ��������������}

����������� ���� $\bm{x} \mapsto A \bm{x} + \bm{x_0}$, 
��� $\bm{x} \in \R^d$, $A \in \R^{m\times d}$, $\bm{x_0} \in \R^m$, ���������� \emph{��������}. 
������� ������� ��������� �������������� �������� \emph{������������� ��������} $(x_1, x_2, \dots, x_n) \mapsto (x_1, x_2, \dots, x_d, 0, \dots, 0)$, $n > d$.
��� ������������� $P\subseteq \R^d$ � $Q\subseteq \R^m$ ���������� \emph{������� ��������������},
���� ���������� �������"=����������� �������� ����������� $\alpha\from P \to Q$.
�� �������� ��������������� �������������� ������� �� ������������� ���������������.
����� ��� $d$-��������� ������� ������������.
������� ����� ������ ������������� $d$-��������� ����� ��������������� ��� ������������ �������
\[
\Delta_d = \Set*{\bm{x} \in \R^{d+1} \given \bm{1}^T \bm{x} = 1, \ \bm{x} \ge \bm{0}}
= \conv\{\bm{e_1}, \dots, \bm{e_{d+1}}\}.
\]
�������� ��������, ��� ����� ������������, ������� $n$ ������, �������� �������� ������� ��������� $\Delta_{n-1}$.

\emph{����������� ���������������} ���������� ������"=�������� ����������� ����
\[
\tau(\bm{x}) = \frac{\alpha(\bm{x})}{\bm{a}^T \bm{x} + b},
\]
��� $\alpha$ "--- �������� �����������, ����������� �������� $\bm{a}$ � $\bm{x}$ ���������, $b\in \R$.

����������� �������������� �������� ���������� ��������� ����������~\cite{ZieglerBook}:
\begin{enumerate}
	\item ����� $P$ � $Q$ "--- �������������.
	���� ����������� �������������� $\tau \from P \to Q$ �������"=����������,
	�� ������������� $P$ � $Q$ ������������ ������������.
	
	\item ����� ������������ $Q$ �������� �������� ������� ��������� ����� ������������� $P$.
	����� ���������� ����������� �������������� $\tau \from P \to Q$.
	
\end{enumerate}

 


\section{��������� ����� � ����������}
\label{sec:complexity}

��������������, ��� �������� ������ � �������� ������ ��������� ������~\cite{Arora:2009, Goldreich:2008} � ������ NP"=������ ����� �~���������~\cite{Garey:1982}.
��� �� �����, ����� �������� ��������������� � ����������, ���������� ��������� �������� ������� � ����������.

��� ��������������� ��������� ���� ������� $f\from \N \to \R_+$ � $g\from \N \to \R_+$ ������������ ����������� �����������:

$f = O(g)$, ���� �������� $c > 0$ � $n^* \in \N$ �����, ��� $f(n) \le c \cdot g(n)$ $\forall n \ge n^*$.

$f = \Omega(g)$, ���� �������� $c > 0$ � $n^* \in \N$, ��� $f(n) \ge c \cdot g(n)$ $\forall n \ge n^*$.

$f = \Theta(g)$, ���� $f = O(g)$ � $f = \Omega(g)$.

$f = o(g)$, ���� $\forall c > 0$ �������� $N_c \in \N$, ��� $f(n) < c \cdot g(n)$ $\forall n \ge N_c$.

�������~$f$ ����� �������� \emph{��������������} � ���������� $f(n) = \poly(n)$,
���� �������� ������� $p=p(n)$ �����, ��� $f(n) \le p(n)$ ��� ���� $n \in \N$.
������� $f$ ����� �������� \emph{�������������������},
���� $f = \Omega(p)$ ��� ������ ��������~$p$.
���, ��������, ������� $f(n) = a^{\ln n}$, ��� $a > 0$, �������� ��������������, �~������� $g(n) = a^{\ln^{1+\varepsilon} n}$ ��� $a > 1$ � $\varepsilon > 0$ "--- �������������������.
������������ ����� ����������� ������ ��������� ��������� �� ��������������� �������� ����� ��������������� � �������������������� ���������.
������, ����, ��� �������, ���� �� ������� ����� ��������������� ��������� �~��������� ���� $f(n) = \Omega\left(a^{n^{\varepsilon}}\right)$,
 ��� $a > 1$, $\varepsilon > 0$.
��������� ������ ���������� \emph{�����������������}~\cite{BondBook:2008}.



������� ������ �������������� ������ ������ ������������
����� ����������� ������� ������������� ����� ����� � ����� (��������� ������ ����� �������� ������������ �������).
%�� ����������� ������������, ��� ������� ������ �������������� ������ ���������� � �������� ������������������ ��������� ������������ �������. 
�� ����������� ������������, ��� ������� ������ ���������� � �������� ������������������ ��������� ������������ �������. 
� ���������, ������ ������������ ����� $n$ �������� $\lceil\log_2 (n+1)\rceil$ ���.
����������: $\size(n) = \lceil\log_2 (n+1)\rceil$.
��������������, $\size(k) = \size(|k|) + 1$ ��� ������ �����~$k$.
������������ ����� $p$ �������������� ����� ������� ������� ����� $k$ (���������) � $n$ (�����������), ��� $k \in \Z$, $n \in \N$.
%, �� ���� $\size(p) = \size(k) + \size(n) + 1$.  
� ������������� ������������� ����� ������� ������ ������ ������ ���������� ������ ����� ���� ��������������� �����.
�� �������� ��, ���� �����, ����� ������� ������ ��������������� ������������ ���������� ����� �� ����� ������ ����������� �� ���.
��� �� �����, ����� ������ �� ���� ��������� ������������� (� ������ ������, �� �����, ��� �����������) ������������ ����� ������.

\emph{�������\'�� ���������� ���������} ���������� �������, 
������� ������� ������������ $n$ ������ � ������������ ������������ ����� (����� ��������), ������������� ���������� ��� ��������� ������� ������ ����� $n$~\cite{Garey:1982, Goldreich:2008}.
����� ����� ��� \emph{���������� ���������} ����� �������� 
��� �������\'�� ���������.

��� (�������\'��) \emph{���������� ������} ����� �������� ��������� (��������������) �������� ������ ��������� ��� ������ ���������.
(� ��������� ������ ��������������� ������ ���������� ������.)
������ ������, ��������� ������ ����������� ������� �� ������ ����������.
� ������ �������, ��� �������� �������������� ���������� ������������ � ������ ������ �������--��������~\cite[c.~33]{Goldreich:2008},
������� ����� �������� ������� ׸���--�������� � ������� �����~\cite[c.~26]{Arora:2009}.

\textbf{����� �������--��������.}
\emph{����� ��������� ����������� �������������� ������ 
����� ���� ��������������� (�������������) ������� ��������
� (�� ����� ���) �������������� ����������� ������� ������.}

� ���������, ���� ����� ����������, ��� �������������� ������ � ������ $P$ (������������� ���������� �����) �� ������� �� ������ �������������� ������.

�������������� ���� �������, ���, ������������, � ������� ��������� ���������� ����� ������ ������, �������������� ������� � ������ $P$ � ��������� ����� �� ��������.
��� �� �����, ����������� ���������� ���������� ��������� ����������� (������, ����������� �� ���������������) ���� �������� ��� �������� (����������� ����� ����������� ������ ������� �~\cite{Aaronson:2008}).

������������ ������� �������������� ���������� ����� ������������ �� ������� � ����������� �� ���� �������� �����~\cite{Garey:1982, Goldreich:2008}.
����� �������� �������� (� ����� ������ �������� ����������� �����������) ������������, ����� �����, ���� �� ��������� ����, ��� �������������� ����������� �� ����� �������� �������������� ���������� �� �������� (������ ��� ���������� ����������� �� ����).
�������� �������� �������� ����������� ����� ���� ����������, ��������������, ��� �������� ������ � �������� ���������� ������ ��������~\cite{Garey:1982, Goldreich:2008}:

\begin{definition}[�������������� ����������]
	������ $\Pi$ \emph{������������� ��������} � ������ $\Pi'$,
	���� ���������� ���������� ������ $M$ �����, ��� ��� ����� ������� $f$, �������� ������ $\Pi'$, ������ $M$ � �������� $f$ ������ ������ $\Pi$ �� �������������� �����. (��������� � ������� $f$ ����������� �� ������� �������.)
\end{definition}

�� �� �������� ����� ����������� ������� NP � co-NP,
� ����� ������� NP-������ (� co-NP-������) ������~\cite{Garey:1982}, ��� ��� ��� ������� �������� ����������� � ���������� ����� ������������� ��������� (� �������, ��������, �� ������� NP-������� ������).
� �������� ������� �������� ������������ �������� ��������� NP-������ ������ � ������������ ������� �������.

����� $U = \{u_1, u_2, \ldots, u_k\}$ "---
��������� ������� ����������. 
������ ������ ���������� ����� ��������� ���� ���� �� ���� ��������: $1 \overset{\text{def}}{=} \text{<<������>>}$ ��� $0 \overset{\text{def}}{=} \text{<<����>>}$.
���� $u\in U$, �� $u$ � $\bar{u} = 1 - u$ ���������� \emph{����������}.
��������� ���������, �������� $\{u_2, \bar{u}_4, u_5\}$,
���������� \emph{�����������} ��� $U$ � ������ ������������ $u_2 \vee \bar{u}_4 \vee u_5$.
�������, ��� ���������� \emph{���������}
��� ��������� ������ �������� ����������, 
���� ���� �� ���� �� �������� � ��� ���������
����� 1.
����� $C = \{D_1, D_2, \ldots, D_m\}$ "--- ��������� ����� ����������, ����� ���������� \emph{�����������}.
���������� $C$ ���������� \emph{����������}, ���� ���������� ����� �������� ���������� �����, ��� ������������ ����������� ��� ���������� �� $C$.
\medskip

\textsc{������ � ������������.}
���� ��������� ���������� $U$ � ����������~$C$.
����� ��, ��� $C$ ���������?
\medskip

� ������ ������ NP-����� ������� ���������� ��������� ���� ������� ������ (� ������ ������ ����������) ��� ������� ����� �� ������ ������ �����������.
��� ��������� ������ ���������� ������, � ������� ������ �������� ������ �������������� ������ "--- ������~\cite{Garey:1982}.
����������� ������ � ������������ ��������� ������ NP-������ �������~\cite{Cook:1971}.
\emph{������ � $k$-������������} ������������ ����� ������� ������ ������ � ������������, ����� ������ ���������� �������� ����� $k$ ���������.
��� ��� ������ � ������������ ������������� ��������~\cite{Karp:1972} � ������ � $k$-������������ ��� $k \ge 3$, �� ��������� ����� �������� NP-������.

\emph{NP-��������} �������� ����� �������� ��, � ������� ������������� �������� NP-������ ������.
�������, ��� ��������� ������������� ��� ����������� NP-������� ������ ������������ ������������ ���������� �� �����. 
� ���� ������, � ���������, ������ NP-������� � co-NP-������� ����� ���������� ���� �� ����� ��� ������� $\NP \ne \coNP$.
�� �� ����� ����� ������������ ������������� �������������� ���������� �� ��������. 
��� ����� ����������� co-NP-������ ������ �������� NP-��������.

����� ��� ����� ����������� ����������� ������ $D^p$~\cite{PapadimitriouY:1984}:
\[
D^p = \{L_1 \cap L_2 \mid L_1\in \NP, \ L_2\in \coNP\}.
\]
� ���������, NP � co-NP �������� �������������� $D^p$, ������ �������, ��� ��� $D^p$-������ ������ �������� NP-��������.
� �������� ������� �������� ��� $D^p$-������ ������~\cite{PapadimitriouY:1984}:
\begin{enumerate}
	\item \textsc{������������--��������������.} ���� ��� ������ �������. ����� ��, ��� ������ ���������, � ������ "--- �� ���������.
	\item \textsc{������ �����.} ��� ���� $G$ � ����� $k \in \N$. ����� ��, ��� $k$ �������� �������� ������ ����� �����.
\end{enumerate}



\section{������ ���������� �����������}
\label{sec:CO}

\emph{������ ���������� �����������} ������������ ���������� ���������� 
\emph{���������� �������} $X$ � \emph{������� ��������} $f \colon X \to \R$.
���� ������ ����������� � ��������� \emph{������������} ������� $x\in X$,
�� ������� ������� ������� $f$ ��������� ������ ��������� (��������).
������ ����������� ������ ����� ���� ������������� � ������ ������������ (� ��������)
�� ���� ��������� �������� ������� ������� �� $-1$.
������� �����, ���� �� ��������� ����, ��� ������� ����������� ����� ��������������� ������ ���������� ���������.

���������� ������� ������������ ������, � ������� ��������� ���������� ������� $X$ �������� ������~--- ��� ��������� ��������, ��������������� ���������� ������ �������.
������� ����� ���� ��������. 
��������, � $X$ ������ ��� ����������� �����, �� ������������� 1000. %����� ����� ������ �������. 
� ����� ���� � ��������. 
��������, $X$ ������� �� (������� �������������) �������� �� ����� C ������ �� ����� 1~�����.
%��� �� �����, ������ ��������������, ��� �������� �������������� $x\in X$ ������������� ���������. 
%� ���� ������ ��������� ������ �������� ������ �����������, ��� ��� ���������������� 
%�������� �������� ����� ���� �������� ���� ������������ �������~\cite{Agrawal:2004},
%� ������������ �������� �������� �����������.

� ������ ���������� ��������� ������ ���������� ����������� ������������ ����� ��������:
\begin{compactitem}
\item \emph{������������ �������} $U$, 
\item \emph{�������� ������������} $g \colon U \to \{\text{����}, \text{������}\}$,
������������ ��������� ���������� ������� $X = \{x\in U\mid g(x)\}$,
\item ������� ������� $f \colon U \to \R$,
\item \emph{����������� �����������:} $\min$ ��� $\max$.
\end{compactitem}
��� ����� ���������� ������ ��������� ��������������, ��� �������� ������������ � ������� ������� ����������� �� �������� �����. � ��������� ������ ������ ���������� ������������.
�� ��������, ��� �������, ���� ���� � ������������ �� �������������� (������������ ����� ���������) �����.

��� ������������ ��������� ������������ ��� ����������� ��������� �������� �������� ����� �����.
� ����� ������� ������� ���������� �������� ��������������� ��������.

\textsc{������ �������������� ��������� ����������������.}
����: ������������ ������� $A \in \Q^{m \times n}$ � ������������ ������� $\bm{b} \in \Q^m$ � $\bm{c} \in \Q^n$. 
����� ����� ������������� ������ $\bm{x} \in \Z^n$, 
��� �������� �������� ������� $\bm{c}^T \bm{x}$
��������� ������������ �������� ��� ����������� $A \bm{x} \le \bm{b}$.
��� ����� ������, ������������ ������� $U$ � ���� ������ ��������� � $\Z^n$,
���������� ������������ �������� ����������� $A \bm{x} \le \bm{b}$,
� ������� ������� $f(x) = \bm{c}^T \bm{x}$.

\textsc{������ ��������� ��������� ���������� �� ��������� ����� ����� �������.}
����: ����� (������������) ������������ ���������� $p(x) = \sum_{k = 1}^d c_k x^k$
� ��� ����� $a, b \in \Z$, $a \le b$.
����� ����� $x \in [a, b]$ ��� ������� $p(x)$ ��������� ���������� ��������.
����� $U = \Z$, �������� ������������ ��������� ���������� ����������� $a \le x \le b$,
������� ������� $f(x) = p(x)$.

\textsc{������ ������ ������������� ����������������.}
��� ����� (������������) ������������� ���������� 
\[
p(\bm{x}) = \sum\limits_{\mathclap{1\le i \le j \le n}} c_{ij} x_i x_j, 
 \quad \bm{x} = (x_1, \dots, x_n) \in \{0, 1\}^n. 
\]
��������� ����� $\bm{x}$, ��� ������� $p(\bm{x})$ ��������� ���������.
����� $U = \{0, 1\}^n$, �������� $g$ ������������ ����� ������, 
� ������� �������� �������� ���������~$p$.

\textsc{������ � �������.} 
���� ��������� ��������� $E$.
��� ������� �������� $e \in E$ �������� ��� ������ $a_e \in \Q$ � �������� $c_e \in \Q$.
����� ����, �������� ������ ������� $A \in \Q$.
��������� ������� ������������ ��������� $x \subseteq E$ ���, ����� �� ��������� ������ ��� ������ ������� �������, � �� ��������� �������� ���� �� ������������.
� ���� ������ ������������ ������� $U$ ���� ��������� $2^E$ ���� ����������� ��������� $E$,
�������� ������������ $g = g(x)$ ������������ ������������ $\sum_{e\in x} a_e \le A$,
� ������� ������� $f(x) = \sum_{e\in x} c_e$.

\textsc{������ � ���������� ����.} 
���� ��������� ������� $V$ � ��������� ����������� �� �������� ����� (������, ��� �������) $E$. ��� ������� ������� ������ $e \in E$ �������� ��� ����� $c_e \in \Q$. 
(����� �� ������������, ��� ��������� $E$ �������� ��� ���� �������. ���� �� �����-���� ��� ������ �� ��������� ������� ���������������, �� �������� ����� ���� ������ ������ ���������� ���������� �������� �����.)
� ��������� $V$ �������� ��� ������ $s$ � $t$.
��������� ����� ���������� (�� ��������� ����� �������� �����) ����, ����������� $s$ �~$t$.
������������ ������� � ���� ������ ������� �� ������������ ���������� �� $V \setminus \{s,t\}$.
������ ���������� "--- ������������������ ������� ���������� ����, ������������ $s$ � $t$.
�������� ������������ ������������ ����� ������.
%�������� ������������ ���������, �������� �� ������ ��������� $u \in U$ �������� ����� ������� ����, ����������� ��� ��������� ������. 
%������� �������, ��� � � ���������� �������, �������: $f(u) = \sum_{e\in u} c_e$. %, $u \in U$.
������� ������� ����� ����� ���� �������� �����, 
������������ ������ ����.
����������� �����������~--- �������.

%\textsc{������ ������������.} 
%���� ��������� ������� $V$ � ��������� ����������� �� �������� ����� $E$. ��� ������� ������� ������ $e \in E$ �������� ��� ����� $c_e \in \Q$.
%��������� ����� ���������� �������, ���������� ����� ��� ������ �� ������ ���� � �������������� � �������� �����.
%������������� ������� ����� �������� ��������� ���� ������������ ������� $S_n$, $n = |V|$, �������� ������������ ������������ ����� ������, %, �������� �� ������ ��������� �������� ����� ��������� �������, ���������� �� ������ ���� ����� ��� ������, 
%� ������� ������� $f(u)$ ����� ����� �������� ����� ���������������� ��������.
%= \sum_{e\in u} c_e$, $u \in U$.
%����������� �����������~--- �������.

\textsc{������ � ��������� ������ �����.}\label{problem:Color}
����� ���� $G = (V, E)$.
��������� ������ ��� ������� $v \in V$ ��������� ��������� ����� $k_v \in \N$
(��������������� ����� �����) ���, ����� ����� ��� ������� ������� ����� ������ �����, � �������� ���������� ����� ���� �� �����������.
����� �������, $U = [n]^n$, ��� $n$~--- ����� ������ �����, 
�������� $g$ ������������ <<��������������>> ������� ������,
������� ������� $f(\bm{u}) = \max\{u_1, \dots, u_n\}$, $\bm{u} = (u_1, \dots, u_n) \in U$.
����������� �����������~--- �������.

����� ��������� ����� ���������� ����������� ����� �������� 
\emph{������ ������������� �����������}, 
������������ ������� ������� �������~\cite{SchrijverCO:2003} � ����� ������������� ��������.

���, ��������, ������ �������������� ��������� ���������������� �� �������� �������������, ��� ��� �� ������������ �������, ������ ������, �� ����������.
�� ��� �������������� ����������� $\|\bm{x}\|_{\infty} \le B$, $B \in \Z$, ������ ���������� �������������.
� ������ ��������� ��������� ���������� �� ��������� ����� ����� ������� ������������ ������� �������, �� ����� ��������������� ��������. ��� �� �����, ��� ������ ���������� �������������, ���� � �������� ������������ �������
����������� ��������� ������������������� �����~$n$, ��������� �� ����� � ������,
� ���������������� ��� ������������������ ��� �������� ������ ����� ����� �� ������� $[0, 2^n-1]$, $2^n > b$.

������, ��������������� �������������, ��� ������������ $U$ ������������ ����� ���� ��������� ����������� (������ � �������),
���� ��������� ���������� � ������������ (������ ������ ������������� ����������������, ������ � ��������� �����),
���� ��������� ���������� ��� ���������� (������ � ���������� ����). 
� ����� ��������� ���� ���� � �������� �� ������� ����� ������������, ��� ������������ ������� 
������������� ������ ������������ ����� ��������� ������� $S^d$
���������� ��������� ��������� $S$ (� ������������ ���������� ����� ����������� ground set),
��� ������� $d \in \N$ ���������� \emph{������������ ������}.
��� ����, ����������, � �������� ������������ ����� ���� ��������� �������������� ������� 
(��������, �������� �� ������� ������������� ��� �� ����������� ��� ����������).

����������� ��������� ������������� ����������� �������� ������� \emph{�������������� ������}. 
��� ��� ������������ ������� ��� ������ ������������� ����������� �������, 
�� �� ������ (� ������) ����� ������ � ������� ������� �������� ���� �������.
������� ����� ������ ��� �������� \emph{�����������}.
��� �� �����, ��� ����� ����������� ������ ����� ������������� ������� ���������� ���������������: $|U| = |S|^d$.
��� ���� ������ ������� ���������� ����������� ����������, ��� ��������� ������ ����� ������������ ������� ������ ������������ �������.
%� ����� �������� ������� ������� ������������� ���������� ��������� ��� ������� ������,
%��� �������, �������� ����������� ����� ��� ����� �� ����� ����� ������.

�������� ���� �������� ���������� ��� �������� ������� �������� ������.
� \emph{�������������� ������} ���������� (�������������) ����������� ��� ������� ������
(��������, ������������ ������) �������������.
\emph{�������� ������} ������������� ����������� ������������ ����� ������������������
��� ������� ��������� ���������� �������������� �����,
����� ������� ���� ������ � ��� ������ ������� ������������.
���, ��������, �������� ������ � ������� �������� � ���� ��� �������������� ������,
����������� ��������������� ������������.
������� �������, � �������� ������ � ������� �� ������������� ��������� ���������, �� ������� � ��������.

� �������� ������ ������������� ����������� ������� �������, ��� �������, 
������� �� ���������� ������ ����������, �������� ������� �������� �������������� ������ �� ��������.
��������, � ������ � ������� ������ ����������� �������� �������� ���������.
�� �� ����� ����� � � ��������� ��������� ������������.
� ������ � ������� �� ������� �� �������� ���������.
����� �������, �������������� ������ �� ���� ������������ ����� ������ ���������� ����� ������� ������ �������� ������.

%�����, ������, �������� ������ ����, ������� �� ������������ � �����������:
%���� ����������� $d$, ������ $r$ � ������� ������ $\bm{c} \in \R^d$, 
%����� ����� ����� $\bm{x} \in \Z^d$ ������ $d$-������� ���� ������� $r$ � ������� � ������ ���������,
%��� ������� ������� ������� $\bm{c}^T \bm{x}$ ��������� ���������.
%����� ������������ ������� ������� ��� �� $d$, ��� � �� $r$ (�.�. �� ���� ����������� ����������).

� ������ ���� ��������� ���� ��������� �������� ��������� ����������� ������ ������������� �����������.

%http://www.nada.kth.se/~viggo/problemlist/
\begin{definition} %[������ ������������� �����������]
\label{def:COP}
\emph{������ ������������� �����������}
������������ ����� �������:
\begin{enumerate}
	\item \emph{������� ������} $I$, �������������� ����� ����� ���������� (�����).
	\item \emph{����������� ������} $d = d(I) \in \N$.
	\item �������� ��������� $S = S(I)$,\\ ������������ \emph{������������ �������} $U = S^d$.
	\item \emph{�������� ������������} $g = g(u, I) \in \{0, 1\}$, ��� $u \in U$.
	\item \emph{������� �������} $f = f(u, I) \in \R$, ��� $u \in U$.
	\item \emph{����������� �����������:} $\min$ ��� $\max$.
\end{enumerate}
�������� $g$ ���������� \emph{��������� ���������� �������} $X = X(I) = \{u \in U \mid g(u, I) = 1\}$.
���� ������ "--- ����� ���� ���������� ������� $X$ ����� �����, �� ������� ������� ������� $f$ ��������� ����������� ��������.
��������� ������� ���������� \emph{�����������.}

���� ������� ������ $I$ �� �������������, �� ������ ���������� \emph{��������}, ����� "--- \emph{��������������.}
\end{definition}

\begin{remark}
\label{rem:PolyPred}
���� ��������� ���������� ��������� ������������ ��� �� ������� ������� �������� � ������ ������������ �������, �� �������� �������������� ������ ��� ������� ����� ������ ������� �� ������ ����.
�������, ��� �������, ��������������, ��� \emph{��� ������� � �����������~\ref{def:COP} ������������� ���������.}
\end{remark}

�� �������� ��������, ��������� $S$ ����� ������������ ������������� ����� �����.
����� ����� ������������, ��� $S = \{0,1,\dots,k-1\}$, ��� $k = |S(I)|$. 

���������� ��������, ����� ������� ������� �������:
$f(\bm{u}) = f(\bm{u}, \bm{c}) = \bm{c}^T \bm{u}$, ��� $\bm{u} \in U = S^d$, � \emph{������� ������} $\bm{c} \in \Q^d$ ���������� �� ������� ������ ������.
��������������� ������ ���������� \emph{�������� ������� ������������� �����������}~\cite{Junger:1995}.
� ����� �������, ������ � ����� ������������ �������� ����� ����������� �� �������� (��������, ������ �������������� ��������� ����������������, ������ � �������).
� ������ �������, ������ ������� ����� ������ ���� � ��������� ����� ������� ��������� ������, ��� ���� ������ � ���������� ������� �������� �� ������ ������� �������� ����� ����������, ���� �������� �� � ��������� ����.
� �������� ������� �������, ��� ��� ������ �������� ��� ���������, �������� ���������� �����.

� ������ ������ ������������� ���������������� ������ $n$"~������� ������� $\bm{x} = (x_1, \dots, x_n) \in \{0, 1\}^n$ ��������������� ������ $\bm{y} \in \{0, 1\}^{n(n+1)/2}$, ���������� �������� $y_{ij} = x_i x_j$, $1 \le i \le j \le n$.
� ���� ������ ������������ ������� $U = \{0, 1\}^{n(n+1)/2}$, �������� ������������ ��������� ���������� ������� $y_{ij} = y_{ii} y_{jj}$,
� ������� ������� �������: $f(\bm{y}) = \sum_{1\le i \le j \le n} c_{ij} y_{ij}$.

�� �������� � ���������� �������, � ������ ��������� ��������� ���������� ����� ������ ���������� $x \in \Z$ ����������� ������ $\bm{y} \in \Z^d$, ���������� �������� ������ ������������� ������� $y_k = y_1^k$, $k \in [d]$. ����� ������� ������� ���������� ��������: $f(\bm{y}) = \sum_{k=1}^{d} c_k y_k$.

� ������ � ���������� ���� � �������� ������������ ������� ������ ������������� ��������� $\{0, 1\}^E$
������������������ �������� ��������� ����� �����.
����� �������� ������������ ��� ������� $\bm{u} = (u_e) \in \{0, 1\}^E$
������ ���������, �������� �� ��������������� ������������ �������� ����� �����, ����������� ������ $s$ � $t$, � ������� ������� $f(\bm{u}) = \sum_{e\in E} c_e u_e$.

� �������� ������������ ������������� ������ � ��������� ������ ����� ������ ��������� �������� � ������������ � ������������������ ������ $\bm{u} \in \{0, 1\}^{n^2}$, ���������� �������� ������������ ��������� �������: $u_{ji} = 1$, ���� $j$-� ������� �������� � $i$-� ����, � ��������� ������ $u_{ji} = 0$.
�������� ������������ � ����� ������ ��������� ������������ ������� $\sum_{i\in[n]} u_{ji} = 1$, $j \in [n]$, (������ ������� �������� ����� ������) � $u_{ji} + u_{ki} \le 1$, $i\in [n]$, ���� $j$-� � $k$-� ������� ������.
����� ������� ������� ����� ���������� ��������� �������:
$f(\bm{u}) = \sum_{i,j\in[n]} n^i u_{ji}$.
����, ��� ������� ����� ����������� ��� ����������� ����� �������������� ������.
(� ��������� ����� ���������� ����� ��������� ������������� ���� ������, � ������� ������������ �������� ������� ����������� ��������� $\{0,1\}$.)

% ����� ������� �������� �������� ����� ������������� ����������� � �������������� ���������������� ���� � \cite{Junger:1995} "Practical problem solving with cutting plane algorithms in combinatorial optimization"
% ����� �� ��� ������ ���������� NP-��������

�������� ����� ��������� �������� ����� ������������� ����������� ������������� �������������� �������� $S = \{0,1\}$. ��� ������� � ���, ��� ������ ���������� ������ ��������� ��������� ������������~\cite{Junger:1995}.
���� �������� ��������� $E$, �������� ������������ $g\from 2^E \to \{0, 1\}$ � ������� ����� $c\from E \to \R$.
��� ������� ������������ $T\subseteq E$ ����������
�������� ������� ������� $f(T) = \sum_{e\in T} c(e)$.
��������� �����
\[
T^* = \argmax_{T\subseteq E}\Bigl\{f(T) \bigm| g(T)=1\Bigr\}.
\]



\section{������������� � �������� �����}
\label{sec:ProblemPolytopes}

������� ������ �������� �� ��, ��� �������������� �������� ������ ������������� ����������� �� ���� �������� � ����������� �������� ������� ������� �� ��������� �������� ��������� ���������� ������� $X \in \Z^d$.
(� ���������, $X$ �������� ������������� ������ ���� $\Cube_d$ � ������ ���������������� ������, ����� $S = \{0,1\}$.)
�������� �������� $\conv(X)$ ���������� \emph{��������������} ��������������� �������������� ������.
� �������� ������ ������������� � ���������� �������������� (�������������� �����).
� ���������, � ������~\cite{Papadimitriou:1984}
������������ ������������� ������ ������������� ����������� ��� ������������������ 0/1"~�������������� $\{P_n \mid n\in\N\}$ �����,
��� ��� ������ ������� $\bm{v}$ � ������� $n$
�� ����� �� �������������� ����� ���������,
�������� �� $\bm{v}$ �������� ������������� $P_n$.

��������� � ��������.

� ������� ������ ������������� ���������������� ������������� \emph{����� ������������ ������������}
\begin{equation}
\label{eq:BQP}
\BQP(n) = \Set*{\bm{x}=(x_{ij}) \in \{0, 1\}^{n(n+1)/2} \given x_{ij} = x_{ii} x_{jj}, \ 1 \le i < j \le n},
\end{equation}
���������� ������� ����������� ����������~\cite{Deza:2001}.

\begin{remark}
	����� � ����� ��� ����������� ����������� �������������� �� ������ �������� �������������� ��������� ��� ������, ������������ V-�������� �������������.
\end{remark}

����� ������������ ������������ ����� ������ � \emph{�������������� ��������} $\Cut(n) \subseteq \R^{n(n-1)/2}$, ��������� �������� �������� ������������������ ������� �������� ������� ������������������ ����� �� $n$ ��������~\cite{Deza:2001}.

� ������� ��������� ��������� ���������� �� ��������� ����� ����� ������� ������ ����������� ������������
\begin{equation}
\label{eq:Cyclic}
\CP_d(a,b) = \Set*{(x, x^2, \dots, x^d) \given \ a \le x \le b, \ x\in \Z}.
\end{equation}

\emph{������������ ������ � �������} ������������ ����� �������� �������� ������������ 0/1-��������,
%������ ��������� $\Cube_d$, 
������������� ���������������� $H^-(\bm{a}, b)$, $\bm{a} \in \Z^n$, $b \in \Z$:
\begin{equation}
\label{eq:KNAP}
\Knap(\bm{a},b) = \Set*{\bm{x} \in \{0,1\}^{n} \given \bm{a}^T \bm{x} \le b}.
\end{equation}

����� $G(V,E)$ "--- ������ ����������������� ���� �� $n$ ($n=|V|$) ��������, ����� ������� �������� ���: $s$ � $t$.
����� $W \subseteq 2^E$ "--- ��������� ���� $s$-$t$ ����� � ���� �����.
\emph{�������������� �����} ���������� �������� �������� ��������� ���� ������������������ �������� $\Path(n) \subseteq \{0,1\}^E$ ��� ����� �� $W$.
���������� ������������ \emph{������������ �������} $\Dipath(n) \subseteq \{0,1\}^A$ ��� ������� ���������������� ����� $D(V,A)$ �� $n$ ��������.

� �������������� ����� ����� ������ \emph{������������ ������ ������������} ��� \emph{������������ ������������� ������}~\cite{Emelichev:1981} $\TSP(n)$, �������������� ����� �������� �������� ��������� ������������������ �������� ���� ������������� ������ ������� ������������������ ����� �� $n$ ��������.
� ���� �������, �������� �������� ��������� ������������������ �������� ������������� �������� � ������ ��������������� ����� �� $n$ �������� ���������� \emph{�������������� ������������� ������ ������������} ��� \emph{�������������� ������������� ��������} � ������������ $\ATSP(n)$.

������ � ���������� ������� (��-��������, ����� �� ����� �������������� � ��������� �����) ������������� ��������� \emph{��������������� ��������������}~\cite{Emelichev:1981} ��� \emph{������������}~\cite{ZieglerBook}.\label{def:perm-birk}
��������� ����������� $\Perm(n)$ �������� �������,
���������� ������������� �������������� ��������� �������"=������� $(1, 2, \dots, n)^T$.
���� �� ��� ������������ $\pi \from [n] \to [n]$
������ ������� $(\pi(1), \pi(2), \dots, \pi(n))^T$
�� ���������� ��������������� ������� $\bm{x} \in \{0,1\}^{n\times n}$ � ������������
\[
x_{ij} = \begin{cases}
1,& \text{���� $\pi(i) = j$,}\\
0& \text{�����,}
\end{cases}
\]
�� ������� \emph{������������ ��������} $\Birk(n)$, ������� ��� ���������� \emph{�������������� ���������������� ������} � \emph{�������������� ������ � �����������}~\cite{Emelichev:1981}.
��������������� ������� $\bm{x}$ ����� ����� ���������������� ��� ������������������ ������ ������������ ������������� � ������ ���������� �����,
������ ���� �������� �������� �� $n$ ������. 
� ���� ����� ������ ������������ $\Birk(n)$ ����� ���� ����� ������ �������������� ����������� ������������� � ���������� �����.

\emph{������������ ����������� �������������} $\Match(n)$ ������������ ��� �������� �������� ���� ������������������ �������� ����������� ������������� � ������ ����� �� $n$ ��������.

������ � ������ � ������ �������"=���������� ����� $G(V,E)$ ������������ (�� ���������� ���� �������� �����) ��������� (�� ���� ������������ ��� ������� �����) ������ ����� ���� ����������������� ��� ������ ����������� �� \emph{������������� �������� ��������} $\Tree(n) \subset \{0,1\}^{E}$, $n = |V|$, ��������� �������� �������� ������������������ ������� �������� �������� � �����~$G$.

���������� �������������� �������� �������� �������� ������������� ��������� (������, ��� ���������). 
�~\cite{Feichtner:2005} ���������� ������ ������������, �� ������ � ��� ������� � ����� ����������� ���� ��������������.
0/1"~������������ � $\R^n$ ���������� \emph{�������������� ��������}, ���� �� ����� � �������������� $H(\bm{1}, r)$ ��� ��������� ����� $r \in [n]$ � ��� ������� ������������� ���������� �������� ���������\label{matroid}: ������� $\bm{x}$ � $\bm{y}$ ������ ����� � ������ �����, ����� �������� $i,j \in [n]$, $i \ne j$, �����, ��� $\bm{x} - \bm{y} = \bm{e_i} - \bm{e_j}$. ��� ���� ����� $r$ ���������� \emph{������} ��������. � ���������, ���� �������� �������� �������� � ������ ����� �� $n$ �������� ����� $n-1$.

��� ���� ��������� ��������������, ����� ����������� � ����������, "--- \emph{������������� ����������� ��������} � ����� $G=(V,E)$~\cite{Chvatal:1975}, ����� ���������� \emph{��������������� �������� ������}~\cite{Nemhauser:1975}:
\[
\Stable(G) = \Set*{\bm{x}\in\{0,1\}^V \given x_v + x_u \le 1 \text{ ��� ������� ����� } \{v,u\} \in E}.
\]
%��������� ������ \emph{������������� ����������� ��������} $\Stable(G) \subseteq \{0,1\}^V$ ������� �� ������������������ �������� ����������� �������� ����� $G(V,E)$, �������� $n$ ������.

%� ����� ������������� �������� ����� ������������� ����������� ����� ���������� \emph{�������������� ���������������}. 
%��� ����� �� ����������� ��������, ���� ����� �������� ���������� ������������� ������������� �������� 0/1-���������������.

���������������� �������� ������� �������������� ����� ����������� ������� ���������.
�������� ������� ������� � ���, ��� ����������� �������� ������� ������� �� ��������� $X$ � �� ��� �������� �������� $\conv(X)$ ���������.
�� ���� �������� ������ ������������� ����������� ����� ���� �������������� ��� ������ ����������� �������� ������� �� �������� �������������.
��� � 1954 ���� ������, ��������� � �������~\cite{DantzigFJ:1954}, ���������������� ���� ������������� �������� � ������������� �������� ��������"=�������, �������� ������������� �� ��� �������� ��������� � ������� ������ ������������.
������������ ���� ������ ����� ���������� ����������, ������� ������� ����� ��������� ������� � �� ����������� ��� ������� ����� ������ ����.
�����������, ������������ ���� ��� ����� ������ (���������) ������� �� ��������� ������� ��� �������� ������������� ��������������� ��������������.
� ���������, �� ������������� �������������.
��������� ����� ������������� �������� ����������� �������������, ����� ��� ������, ����� �����������, �������� �������������� ����� �������������, �������� �������������� ������� ���������� ������"=�����������.
��������� ��������������� ������������� ����� ������� ��������� ������������� ����� ������������� (����������, �������, ����� � �.�.), ��������� ������ ����������� ��� ������� �������������, ����������� ����� ����������� ���������� ������������� � ������ ������.
���� ������ �� ���� ������������� � ������ ��������� �������� ����� ������������� ����������� ����� ����������� ���� ����,
� �������~\ref{sec:Survey}.
�� ������ ��� ������� ����� ����������� �� ����� ������������� �� �������� ������ �����, ����� �� ������� ������ ������������� ��������� ������������ �����������. ��������, ����������������� ��������� �������� �������.

\subsection{�������� �����}
\label{subsec:polyhedra}

������ ��������, ��� ������ � ���������� (��)���� ������������� ��������� ��� �������, ��� ����� ����� ��������������~\cite{Dijkstra:1959}.
������ � ���, ���� ����� ��� �����������, �� ������ ���������� NP-�������~\cite{Garey:1982}.
�� ���� ������ ����������� �������� ������� ������� �� ������������� (��)����� NP-������.
��� �� ������� ������������ ������ � �������� ����������������� ��������� �������� �������?
������������� ��������� � ���� ������ �������� ������� ��������� �������������~\cite{SchrijverCO:2003}.

\emph{����������} ������������� $P \subseteq \R^d$ ���������� �������
\[
P^{\uparrow} = \Set{\bm{y} \in \R^d \given \bm{y} \ge \bm{x} \text{ ��� ���������� $\bm{x} \in P$}} = P + \R^d_+,
\]
��� $\R^d_+ = \Set{\bm{x} \in \R^d \given \bm{x} \ge \bm{0}} = \cone\{\bm{e_1}, \dots, \bm{e_d}\}$.

%�������� ������, ��� $P^{\uparrow}$ �������� ������������� ��������� ������������� $Q = \Set{(\bm{y},\bm{x}) \in \R^d\times\R^d \given \bm{y} \ge \bm{x} \text{ ��� ���������� $\bm{x} \in P$}}$.

����� �������, ������ � ���������� (��)���� � ������������ ����������������� ���� ����� (���) ������������� ������ ����������� �� �������� $\Path^{\uparrow}(n)$ ($\Dipath^{\uparrow}(n)$).

����������� ������� ����� ��������� ������� ��� ������������� ���������� ������ � ����������� ������� � ������ ����������������� ������� ���������� ����� �� $n$ ��������.
� ���� ����� ���������� ��������� ������ $\ext(\Cut(n))$ ������������� �������� � ������ �� ���� ������� � �������� ������������, ��������������� ������� �������.
��������� �������� �������� ����� ��������� ��������� $\MinCut(n)$
� ����� �������� \emph{��������� ��������}~\cite{Conforti:2004}. 
��� ������� ������� �������, ��� �~\cite{Skutella:2010} ����������� �������� �������� �������������� $s$-$t$ �������� � ������������ (��������) ������.
� ������ ������� ����� ��� �������� ����������� � ��������� $\Path^{\uparrow}(n)$~\cite{SchrijverCO:2003}.

������� ������� ������������ ������� ������ � ���������� ���� � ������ ������� $G(V,A)$ ��� �������, ��� �~$G$ ����������� ����� ������������� �����:
\[
\ShortP(n) = \Dipath(n) + \cone(\Cycle(n)),
\]
��� $n = |V|$, � $\Cycle(n)$ "--- ��������� ������������������ �������� �������� � ����� $G$.
������ ��� (� �� ������� $\Dipath^{\uparrow}(n)$) � ����� � ���������� �������� \emph{��������� ���������� �������}.
��������~\cite{Saigal:1969, Vohra:2011}, ��� ��� H-�������� ����������� ����������, ��� � �������� $\Dipath^{\uparrow}(n)$.
� ���� ����� $|A|$ �����������, ������������ ������������� ���� $x_a \ge 0$, $a \in A$.
� ��� ������� ����� � ����������� ���������������, ������������ ���������� ����������� (��� �������� ����������������� � �������� ������� $G$).
�������� ����� ������ ��� ��������� �� (���������) �������~$s$ � ������ �������� � �� ��� ����� �������.
�������� ����� ������ ��� �������� � (��������) �������~$t$ �~������ ��������� �� �� ����� �������.
�� �� �������� ��� ����� ������ ������� ����� ���� (������� ���������� ������).

����� �������, ��� ����� ������ � ��������� ������������� �� ������� ������ ��������������� ������� ������������ ����� ����� ����������� ������������� (��������) ������ � ������ <<������������>> ������� ��������.

\section{������������� ��������������}
\label{sec:Survey}

������, ��� ������� � ���������� ������������� ���������,
������� �������� �� ������ ������������� ����� (�������, �����, ���������� �~�.\,�.) ��������.

\subsection{������ ������������� ����� ��������}
\label{subsec:Ident}

������� ������ \emph{������ ������������� �����} ������� �� �������� ������ �������� � �������� ����������� �����.

���� ��������� �������� ������������ ������, �� ��� ��� ��������� ��������������� �������� ������������ $g$ �� �����������~\ref{def:COP}, �������������� ��� �� ����������� �������������� ������������� ������� $\bm{x}$ ��������� ���������� ������� $X \subseteq \R^d$ (� ���� ������ �������������� ������ �������� �������� �������� $\conv(X)$). 
���� �� ���� ���� � ��������, �������������� ����� ����� ����������� ������������� ������ � ������ <<������������>> ������� ��������, �� � ���������� � ���������, ����������������� ������� �������������, ������ ����������� ��������, ���������������� ������������� ���� ������.
��� �������, ��� ���������� ����� ��� ��������� ��������� �������� ������������� ����������� (��. ���������~\ref{rem:PolyPred}).

������ �������� ����������� ����� ������� �� � �����������.
������������� ������� ������������ �������� ���������,
����� "--- ����� ���������, 
���������� "--- �������������� ���������������� ��������� �����������.

������� ������ ������������� ����� �������� ����� �� ��� ���.

\subsubsection{������������� �������}

����������� ��� ���� �������������� ��������� � ��������� ����� ���������� ����� ������������� ������� �������� �������.
� �������� ��� ������� � ���, ��� ��� � �������������, ��� � � ���������� ������������� ���������� �������� ��������� ��� �������, ����� ��������� ���������� ������� $X$ ������� ������ �� 0/1-��������.
� ���� ������ $X$ ��������� � ���������� ������ ������������� $\conv(X)$, � �������� ������������ $g$ �������������� ������� ������������� ������.
���� �� ����� ������ ������������� ���� �� ������ 0/1-�������, ��, ������ ������, ������������� ������� �� �������� � ���������� ������ ��������� ������������ � ����� ����������� ��������� ����� ��������� ���������� ������� $X$. � ����� ������, ��� ������ �������� co-NP-������~\cite[Theorem 18.5]{Schrijver:1998}.

������������� ������� ����� ������������ �������������,
��� ������� ��� ������ ���������� ��������� ������������ �������� NP-������� ��� �� ����� ���������������� ���������.
��������, �~\cite{Yannakakis:1991} ��������������� ������������, ��������� �������� �������� ������������������ ������� ���� ������������� ��������� ������� ����� (�� $n$ ��������).
��� ��� ������ �������� ��������������� ����� �������� NP-������~\cite{Karp:1972}, �� � ������ ������������� ������� (� ����� ������ ���������� ��������� ������������) ������ ������������� NP-�����.

\subsubsection{������������� ����������}

��������~\cite{SchrijverCO:2003}, ��� ��� ������� $\NP \ne \coNP$ ������� ������ ������������� ���������� ��� ��������� �������������� ����� NP-������� �������� ������ ������������� ����������� �� ����� ���� ����������� �� �������������� �����.
����� ����, ������ ������������� ����������
�������� $D^p$-������ (�, �������������, NP-�������)
��� �������������� ������ ������������~\cite{PapadimitriouW:1988}, ������ � �����~\cite{PapadimitriouY:1984} � ������ � �������� ������������~\cite{Fiorini:2006}.

� ������� ������������� ���������� ����� ������� ������ \emph{������������� ������� ��������������}:
��� ������ $\bm{a}\in\Z^d$ � $b \in \Z$ ���������,
�������� �� �������������� $H(\bm{a}, b)$ ������� ��� ������� ������������� $P$.
��������, ��� ��� ������ $D^p$-����� ��� �������������� ������ ������������ � ������ � ����� ~\cite{PapadimitriouY:1984}.
� ��� ������������� ������ � ������� �������� co-NP-������� ������������� ����������� �����������~\cite{Hartvigsen:1992}.

\subsubsection{������������� �����}

������ ������������� ����� ������������� ����� ���������� \emph{������� � ��������� ������}.
������� � ���� ������ ������ �������������� ���, ��� �������� ��������� ������ ������������� ����� ��������� ������� ��� ���������� ������������ ���������, ������������� ������� ���������� ������~\cite{MatsuiTamura:1995}.
����� ����, ����� �� ��������� � ������������ ����� ������������� ������ ��������������, ��� ������� � �������� �����, ��� ������������ ����� ��������� ��������� ��������� � ���� ������� ����������� � ��������� ������.

������ �����, ������������ �������� ��������� ������ ��� ��������� (� ��� ������� �������������) ������������� ���������� ����� ������������� �����������:
������ � ���������� �������, ������ � ����������� �������� ������, ������ � ����������� � � �������������� (� ������ �������"=���������� �����), ������ � ���������� ����, ������ � ����������� ������� � ������ ��������� ��������� ���������� �� ��������� ����� ����� �������.

%���������� ��������� ��������� ���������� � ������ ��������� ������ ��� �������������� ����� ������������� �����������. ������ � ������������� ���������� �����.

���� (������~\ref{matroid}, �.~\pageref{matroid}), ��� ����������� ������������� �������� ��� ������������� �������� ��������� ��� ������.
������� $\bm{x}$ � $\bm{y}$ ����������� $\Perm(n)$ ������ ����� � ������ �����, ����� �������� $i\in[n-1]$ �����, ��� ������ $\bm{y}$
���������� �� ������� $\bm{x}$ ������������� $i$-� � $(i+1)$-� ���������~\cite{Emelichev:1981}.

\begin{lemma}[\cite{Balinski:1974}]
��� ������� ������������� �������� $\Birk(n)$ ������ ����� � ������ �����, ����� �������������� �������� $p_1 \bigtriangleup p_2 = (p_1 \setminus p_2) \cup (p_2 \setminus p_1)$ ��������������� ������������� $p_1$ � $p_2$
�������� (����) ����.
\end{lemma}

� �������� ��� �� ������������� �������� ��������� � ��� ������������� ����������� ������������� $\Match(n)$~\cite{PadbergRao:1974, Chvatal:1975}.

\begin{lemma}[\cite{SchrijverCO:2003}]
\label{lem:path}
����� $p_1$ � $p_2$ "--- ��� ��������� $s$-$t$ ���� � ������ ������� $D(V,A)$ �� $n$ ��������.
����� �� ������������������ ������� $\chi(p_1)$ � $\chi(p_2)$ �������� �������� ��������� �������� ������� $\Dipath^{\uparrow}(n)$ (�������� ���������� ������� $\ShortP(n)$), ����, � ������ ����, �������������� �������� $p_1 \bigtriangleup p_2$ �������� ����������������� ����, ��������� �� ���� �������, ������� ����� ������ � ����� �����, � �� ������� ������ ����� ������.
\end{lemma}

\begin{remark}
�������� ���������, ��������� � �����~\ref{lem:path}, ���������� ��� ��� �������� $\Dipath^{\uparrow}(n)$, ��� � ��� $\ShortP(n)$.
�������������� ���� ����� �~\cite[theorem~13.4, p.~202]{SchrijverCO:2003} �������� ����������. (�~������~\cite{Rispoli:1992}, ����������� �������� ����� �������� $\ShortP(n)$, ����� ���������� ���������� ��� � ����������� ������ ��������, ��� � � �������� �������� ��������� ������.)
��� �������, ��� ���� ����������� ���� ������� $p_1$ � $p_2$ �������� ������ ������ $p_3$, �� ������ $\bm{x} = \chi(p_1) + \chi(p_2) - \chi(p_3)$ ���� �������� ������������������ �������� ���������� ������.
� ���������, ��� ����������� ������� ��� �������, ������������ �� ���.~\ref{fig:contrpath}.
��� �� �����, ��� ���������� ����� �����������, ���� ��������, ��� ������ $\bm{x}$ ������������ ����� ����� ������������������� ������� ���������� ������ �, ���� �����, ������������������ �������� ���������� ��������. 
����� �������, $\bm{x}$ ����������� �������� $\ShortP(n)$ (�, ��� ���������, �������� $\Dipath^{\uparrow}(n)$), � �� ��������� $\chi(p_1) + \chi(p_2) = \chi(p_3) + \bm{x}$ ������� ����������� $\chi(p_1)$ � $\chi(p_2)$.
\end{remark}
\begin{figure}%
	\centering
	\begin{tikzpicture}[>=stealth']
	\begin{scope}[yshift=2ex, xshift=-5cm]
		\foreach \i in {0,...,3} {
			\node[circle, draw, inner sep = 2pt] (b\i) at (\i,0) {};
		}	
		\draw[->] (b0) node[left] {$s$} -- (b1);
		\draw[->] (b1) -- (b2);
		\draw[->] (b2) -- (b3) node[right] {$t$};
		\node[left] at (-1,0) {$p_1:$};
	\end{scope}
	\begin{scope}[yshift=-2ex, xshift=-5cm]
		\foreach \i in {0,...,3} {
			\node[circle, draw, inner sep = 2pt] (b\i) at (\i,0) {};
		}	
		\draw[->] (b0) node[left] {$s$} to[bend right] (b2);
		\draw[->] (b2) to[bend right] (b1);
		\draw[->] (b1) to[bend right] (b3) (b3) node[right] {$t$};
		\node[left] at (-1,0) {$p_2:$};
	\end{scope}
	\begin{scope}[yshift=2ex, xshift=5cm]
		\foreach \i in {0,...,3} {
			\node[circle, draw, inner sep = 2pt] (b\i) at (\i,0) {};
		}	
		\draw[->] (b0) node[left] {$s$} to[bend right] (b2);
		\draw[->] (b2) -- (b3) node[right] {$t$};
		\node[left] at (-1,0) {$p_3:$};
	\end{scope}
	\begin{scope}[yshift=-2ex, xshift=5cm]
		\foreach \i in {0,...,3} {
			\node[circle, draw, inner sep = 2pt] (b\i) at (\i,0) {};
		}	
		\draw[->] (b0) node[left] {$s$} -- (b1);
		\draw[->] (b2) to[bend right] (b1);
		\draw[->] (b1) -- (b2);
		\draw[->] (b1) to[bend right] (b3) (b3) node[right] {$t$};
		\node[left] at (-1,0) {$p_1\cup p_2 \setminus p_3:$};
	\end{scope}
	\end{tikzpicture}
	\caption{������ ���� ������� $p_1$ � $p_2$, ��������������� ���� ��������� ������ ������������� $\Dipath^{\uparrow}(4)$}%
	\label{fig:contrpath}%
\end{figure}

\begin{lemma}[\cite{Nikolaev:2016}]
\label{lem:adjmincut}
������� $\bm{x}$ � $\bm{y}$ �������� �������� $\MinCut(n)$ ������ ����� � ������ �����, ����� ��� ��������������� �������� $\delta(A)$ � $\delta(B)$
� ������ ����� $G(V,E)$ ����������� ���� �� �������:
\[
A \cap B = \emptyset \quad \text{���} \quad 
A \subset B \quad \text{���} \quad 
B \subset A \quad \text{���} \quad 
A \cup B = V.
\]
\end{lemma}
\begin{remark}
� ������������ �������� ��������� �~\cite{Nikolaev:2016} �������� ��������� �������. ��� �� �����, ��� ����������� ����������� ����� ��������������.
\end{remark}

������������ �������� ��������� ������ ��� ��������� $s$-$t$ �������� � ������ (��)�����~\cite{Skutella:2010} �������� ������� ������� ����������� �����~\ref{lem:adjmincut} � ��� ������,
��� ��� $s$-$t$ �������� ������ ��������� ������� $A \cap B \ne \emptyset$ (��� ���  $s \in A \cap B$) � $A \cup B \ne V$ (��� ��� $t \notin A \cup B$).
%\ref{def:stdicut}

�������� ��������� ������ ������������ ������������� $\CP_d(a,b)$ �������� �����~\cite{Gale:1963}. ������ ��� ��� ������� ������ ��� $d \ge 4$.

��������� ������ � ����������� ������������� ��������� ������ �� �������������� NP-������� �����.
�� ����� ��������� �� ��� ������.
������ �����, ���������� �������������, ��� ������� ����������� �������������� ������������ ������ � ��������� ������.

�������� ��������� ������ ������������� �������� $\Cut(n)$ � ������ ������������� ������������� $\BQP(n)$ ���� ���������� ����������� ����������� ��������~\cite{Greshnev:1984,Beloshevskii:1986,Barahona:1986}.

%{\color{red}������ �������. ��������� \cite{MatsuiTamura:1995}.}

��������~\cite{Chvatal:1975}, ���� ������ ������������� ����������� �������� $\Stable(G)$ ������, ����� � ������ �����, ����� �������������� �������� ��������������� ����������� �������� ���������� ������� ������� �~$G$.
��-����, ���� �� �������� ��������� ����� � ��� �������������� �������� �������� � �������������� ��������� ��������~\cite{Ikura:1985}.
����� ����, ������� ������� �������������� ��������� �������� �������� ������������� ������������� (� ����� ��������������) ������ � ����������� {\color{red}(��., �������, \cite{Balas:1989}), ����� ������ �� ��������������� ������ � ��������� ����� <<�������� ����������>>}.

�������������� �������� ��� �������� ��������� ������ ������������� �������� �������� ������ �~\cite{Young:1978}.

������ � ���, ������ � ��������� ������ co-NP-����� ��� �������������� ��������� �����: ������ ������������~\cite{Papadimitriou:1978}, ������ � �������~\cite{Chung:1980, Geist:1992, Matsui:1995}, 
������ � �������� ���������~\cite{Matsui:1995}, 
������ � ���������� ��������~\cite{Bondarenko:1996},
������ � ����������� � ������������~\cite{Alfakih:1998}, 
������ 3-������������
� ������ � ��������� ��������������~\cite{Fiorini:2003}.


\subsection{����������� � ������� �������������}

����������� ������������� ������ ������ ������� ��������� ��������������� ������, ��� ��� ����� ����� �������� ������� ��� ������ ������������ �������.

����� ������ ������������� �������� ������� ������� ����� ��������, ����������� ��� ������ �������� ���� ���������� �������.
����������, ��� ������ ����������� ������ � ��� �������, ����� ������� ����������� ��������� �������� ���� ���������� �������. 
� ������ ������ ������ ��������� ���������� ������� ���������� ���������� ��� ������������ ������� $U = S^d$ � ������� ������� ���������� ���������� $|S|^d$.

�������, ��� �������������� ��������� ������������� ������� ������������� ��� �� ����������� ���������������� ���������� ������ ����� ������.
� ����� ����������� ����� ����������� ���������� ������ �������� ����� ������ ������������� ������ � ������� $\Knap(\bm{a}, b)$.
������ ������������� ������� ����� ������������� �������� �������. ���������� ���������, ��� ������ ������ $\bm{x}$ �������� 0/1-�������� � ������������� ����������� $\bm{a^T} \bm{x} \le b$.

\begin{prop}
	������ ���������� ����� ������ ������������� $\Knap(\bm{a}, b)$ �������� NP"~�������.
\end{prop}
\begin{proof}
���������� ������� ������ ������, ������� $2b = \bm{a}^T \bm{1}$.
�������, ��� ��� ����� ������� ����� 0/1"~�������� $\bm{x}$,
��������������� ����������� $\bm{a}^T \bm{x} \le b$, 
��������� � ������ 0/1"~�������� $\bm{y}$, ��������������� ����������� $\bm{a}^T \bm{y} \ge b$.
(����������� $\bm{y} = \bm{1} - \bm{x}$ ���������� �������"=����������� ������������ ����� ����� �����������.)

����� $N$ "--- ����� ������ ������������� $\Knap(\bm{a}, b)$,
� $K$ "--- ����� ��� ������, ��������������� ��������� $\bm{a}^T \bm{x} = b$.
� ���� ���������� ���� ���������, ��� ����� ������� ������������ $2 N = 2^n + K$, ��� $n$ "--- ����������� ������� $\bm{a}$, � $2^n$ "--- ����� ���� 0/1-�������� ���� ����������� �����������.
����� �������, ������ ���������� $N$ ������������ ���������� $K$.
�� ��� ������ �������� ����������� $K > 0$ �������� NP"~������ (������ � ����� ��������~\cite{Garey:1982}).
\end{proof}

%\subsection{���������� � ������� �����}

%�������� ������� ���� �����������, ������������ ������ �������� �������� ��������� ��������� ���������� ������� $X$.
%������� ����� ����� � ���� ������� �������������� <<(��)������ ������������>> ��������, ��� ������� ������������� ���������� ��������, ���������������� ��������� $X$.

\subsection{���������� � ������ �����������}

������ ������, ������ ���������� H-�������� �������� �� ��� V"~�������� �������� �������������� �������~\cite{Khachiyan:2008}.
���� �� H"~�������� �������� ������ ��� ���� ������� �����, �� �������� ������ ������������� ����������� ������������� � ������ ��������� ����������������. ���������, ��� ��������~\cite{Khachiyan:1979, Karmarkar:1984}, ������������� ��������� ������������ ����������� (����� ����������), ����� ����������� (����������), � ������� ������������� (����� �����).

�� ��������, ��� �������, ���������� H-�������� �������� �������� ����������� ������� �������.
�� ������ ������� ������� �������� ��� ������ ������������� ���������� (��. ������~\ref{subsec:Ident} ����).

������� �����, ��� ��� ������� �������� ������ ������������� ����������� �������� ��������� ���������������� �� ����������� ������� H-�������� ������������� ������.
� 1982 ���� ���� � ������������ ��������~\cite{KarpP:1982}, ��� ��� ������������ ������� �������� ������ ������������� ����������� ���������� ����� ����������� �������� ������� ������ �����������: ��� ��������� ������������� $P$ � ������� $\bm{v}$ � ������������� ������������
����������, ����������� �� ������ �������������
�, ���� �� �����������, �� ������������� �������������� (�������� �����������), ���������� $v$ �� $P$.

\subsection{������� �����}

������� ����� ������������� $P$ ����� ���������� $\delta(P)$.

�������� ���������� ��� ������ ��������� ������ �������������� �������� ��� ����, ��� ��������"=�����~\cite{Dantzig:1951} ������ ������ ��������� ����������������, �������� �� ������ ����� ������������� �� �������� ������� � �����������.
����� �������, ���� ������� ������ ����� (����. pivot rule) � ��������"=������ �������� (�������� ���������� ���� �� ����������� �������), �� ������� ����� ������������� ����� ����� ����� ��������"=������ ��� ��������� ������ �������� �������.

� 1957 ���� ������ ���� �������� �������� � ���, ��� ������� ����� �������� �� ����������� �������� ����� ������ ��� ����������� � ������������~\cite{Dantzig:1963}.
������ ����� ���������� ���� �������� ��� ������ ����������� "--- 4-������ ������� � 8 ������������ � 15 ���������, ������� ����� �������� ����� 5~\cite{Klee:1967}.
������� �������� ���� ��������������� �� ������ ������������ ��������� (��������������) � ������ � 2010 ���� ��� ������ ����������� ��� ��������������~\cite{Santos:2012}.
��� �� �����, ���������� ��������������, ������� ����� ������� ��� �� ���� �� � ��� ���� ������ ����� �����������, �������� ����� ����������� �������~\cite{Santos:2013}.
� ������ �������, �� ��� ��� �� �������� �������������� ������� ������ �������� ����� �������������� �������������.
������� ������� ����� ����� ����������� ������������ �������� �������������� �������� �����:

\begin{conjecture}[�������������� �������� �����]
C��������� ����� �������������� ������� $f(n, d)$,
��� ��� ������ $d$"~������� ������������� (��������)
� $n$ ������������ ��� ������� ����� �� ��������� $f(n, d)$.
\end{conjecture}

��� ������� �� ����������� ���� (� ������ �������) �����������,
��� �������� ������� � ������������ ���������� ������ ��������������� (�� ���� �� ���������� �� ������� ������� �����) ��������� ��������� ���������������� �� ������ ��������"=������.
������ ���������� ������ ��������� �������� ������� � ������ �������������� ����� XXI ����~\cite{Smale:1998}.

\begin{remark}
��� ������ ��������� ������ � ������� �������� ����� �� ������������� ������� �������� �������� �� ��, ��� ��� ������ ����������, ���� H-�������� ������������� ����������� ��� �� �� ����� ������������ ��������.
\end{remark}

� ���� ���������� ���� ������ ������ ��������� ������ ��� ��������� �������� �������������� ��������� ������� ����� �����.
����� �� ���, � ����� ������ �� ����� ������ ������ �� ���� ���� ����� ����� �~\cite{Santos:2013}. 
����� �� �� ���������� � ������ ������� ��������� ��������� ������ ��������� ������ �������������� ��������������� � �������� ������������� �����������:
\begin{enumerate}
\item ������� ����������� $\Perm(n)$ ����� $n(n-1)/2$~\cite{Emelichev:1981}.
\item ��������������� �� �������� ��������� ������ ������������� �������� �������, ��� ������� ��� ����� ������ ���� ����� ����� $r$. � ���������, ������� ����� ������������� �������� �������� $\Tree(n)$ ����� $n-1$ ��� $n \ge 4$.
\item ������� ����� �������� ���������� ����� $\ShortP(n)$ (� �������� $\Dipath^{\uparrow}(n)$) ����� ���� ��� $n\ge 4$. ���������� ��������, ��� �������, ��������������� ����, ������������ �� ����� ������������ ���� $(s,t)$, ������ �� ����� ���������� ��������� �������� (��. �������� ��������� � �����~\ref{lem:path}).
%~\cite{Rispoli:1992} ({\color{red}������ � ������������ ������������� � � �������� ���������}).
\item ������� ����� �������� $\MinCut(n)$ (� ��������� $s$-$t$ �������� � ������ (��)�����) ����� ���� ��� $n\ge 4$. ���������� ��������, ��� �������, ��������������� ������� $\delta(S)$, ��� $S$ ������� �� ����� �������, ������ �� ����� ���������� ��������� �������� (��. �������� ��������� � �����~\ref{lem:adjmincut}).
\item ������� ����� ������������� �������� $\Birk(n)$ ����� ���� ��� $n \ge 4$~\cite{Balinski:1974}. �� �� ����� � ��� ������������� ������������� $\Match(n)$~\cite{PadbergRao:1974}.
\item ������� ����� ������������ ������������� ����� �������~\cite{Gale:1963}.
\item �������� ������ ������������� �������� $\Cut(n)$ � ������ ������������� ������������� $\BQP(n)$ ����� �������~\cite{Greshnev:1984,Beloshevskii:1986,Barahona:1986}.
\item ������� ����� ������������� ������������ $\ATSP(n)$ ����� ���� ��� $n \ge 6$~\cite{PadbergRao:1974}.
� ��� �� ������ ������� ����������� ���� ��� ������������� $k$"~���������� � ��������� ������.
\end{enumerate}
�������� ������ ���������� �����������~\cite{Ceballos:2015,Ceballos:2016}.
��� �������������� ��������� ���������~\cite{Borgwardt:2013}.
��� ������������ ��������������~\cite{Borgwardt:2015}.
������� ����� $\TSP(n)$ �� ����������� �������~\cite{RispoliCosares:1998}
������� ������������� $k$-������ �� ����������� 5~\cite{Girlich:2006}.
������� ������������� �������� �������� ����� ����~\cite{Young:1978}.
%���������� (��\'��� ��???) �������� ��� ������������� � ������ ����� � ����������� ����~\cite{Rispoli:1992}.

������, ������� ����� 0/1"~������������� �� ��������� ��� �����������~\cite{Naddef:1989}.
����� ����, ���� $X \subseteq \{0,1,\dots,k\}^d$, �� ������� ����� ������������� $\conv(X)$ �� ��������� $k\cdot \dim(\conv(X)) \le k d$~\cite{KleinschmidtOnn:1992}.
� �������� ������� �������, ��� ��� ����������� $k=n-1$, $\dim(\Perm(n)) = n-1$, � ������� ����� $n(n-1)/2$.

\subsection{���������� ��������������}

����� ��� �������������, ��������������� � �������� ������������� �����������, ����� ���������������� ����� �����������, ��� ������ ����������� ����������� ���������������� ������������� ������� ��������� ����������������. 
���� �� �������� � ������� ���� �������� ������� � �������� � ����������� ������������ �������������~\cite{Kaibel:2011,Conforti:2013}. 
 
\emph{�����������} �������� $P \subseteq \R^d$ ���������� ������� $Q \subseteq \R^n$ ������ � �������� ������������ $\alpha \from \R^n \to \R^d$, ��������������� ������� $P = \alpha(Q)$.
��� ��� ��� ��� ����� � ������ ������� �������� �������� $Q$, �� ����� �� ����� ����� �������� ����������� ��������������� ���� �������, ������������, ��� �������� ����������� $\alpha$ ������������� ���� �������� �����������.
%� ���������, ����� ������� �������� ����� �����������.
����� H-�������� ���������� ������ � ������������ $\alpha$ ���������� \emph{����������� ���������} ��� \emph{����������� �������������}.

�� ����������� ���������� �������, ��� ������ ����������� �������� ������� �� �������� �������� � ������ ����������� �������� ������� �� ��� ����������.
� ������, ���� ������� $P\subseteq \R^d$ �������� ������� �������� $Q\subseteq \R^n$ ��� ����������� $\alpha(\bm{y}) = A \bm{y} + \bm{b}$, ��� $A \in \Q^{d\times n}$, $\bm{y} \in \Q^n$, $\bm{b} \in \Q^d$,
�� ������ ����������� �������� ������� $\bm{c}^T \bm{x}$ ��� ����������� $\bm{x} \in P$
������������ ����������� �������� ������� $\bm{c}^T A \bm{y} + \bm{c}^T \bm{b}$ ��� ����������� $\bm{y} \in Q$.

������������ �������� ���������� ������������� ����������� ������������ �������� ���������� $\Perm(n)$.
��� H-�������� ��������� ��������~\cite{Rado:1952,ZieglerBook},
� ����� ����������� ����� $2^n-2$.
� ������ �������, �������� �������� (��. ����������� �� �.~\pageref{def:perm-birk}), ��� $\Perm(n)$ �������� �������� ��������� ������������� �������� $\Birk(n)$ ��� �������������� $p(x) = x \cdot (1,2,\dots,n)^T$, $x \in \{0,1\}^{n\times n}$.
(��~���� ������������ �������� �������� ����������� �����������.)
�������������, 
%��� ������ �������� ������� $\bm{c}\in\Q^n$
%�������� ��� $\bm{c}^T \bm{y}$ ��� $\bm{y}\in \Perm(n)$
%��������� � ���������� ��������� $\sum_{i,j} c_{i} x_{ij} j$ ��� $x \in \Birk(n)$. ��~���� 
������ �������� ����������� �� ����������� �������� �~������ �������� ����������� �� ������������� ��������.
��� ���� ��� H-�������� ���������� ��������� ����� $2n-1$ ��������� � $n^2$ ����������~\cite{Birkhoff:1946}.
����� ����, �� ��� �����~\cite{Goemans:2015} ��� ����������� $\Perm(n)$ ���� ������� ����������� ������������ � ������ ���������� $\Theta(n \log n)$ (���������� ����������� ��������� � ��������� �� ����������� ���������).

\emph{�������� ���������� (����������� ������������)}
���������� ����� ����������� (����� ���������� � H-��������) ����������.
����������� ���� ������ ����������, ��� ������ ���������� ����� ���� ����������� ������ ������� ��������� �������������. 
� ��������� ����� �������� ����� ������ �������� ����� �� ������������� ���������� ������� �� ���� �������� � ����������� ������������~\cite{Kaibel:2011, Conforti:2013}.

\emph{���������� ���������� (����������� ������������)} $\xc(P)$ ������������� $P$ ���������� ����������� ������ ����� ���� ��� ����������.
��������~\cite{FioriniKPT:13}, ��� �������� $\xc(P)$ �� ���������, ���� � �������� ���������� ������������� ������ ������������ �������� (�������������).

��������� ���������� �������������� ����� ������ ����������� � ������ �������������� ���� ����������������, ��� ����������� �������������, ����� ��� ������ � �����������.
\begin{property}\label{prop:xc-base}
	����� $P$~--- �������� ������������, $\vertices(P)$~--- ����� ��� ������, $\facet(P)$~--- ����� ��� �����������, $\face(P)$~--- ����� ���� ������ (� ��� ����� �������������). �����
\begin{enumerate}
	\item $\xc(P) > \dim(P)$, ��� ��� ����� ����������� ������������� ������ ������ ��� �����������.
	\item $\xc(P) \le \facet(P)$.
	\item $\xc(P) \le \vertices(P)$, ��� ��� ����� ������������ �� $n$ �������� �������� �������� ������� ��������� $\Delta_{n-1}$.
	\item $\xc(P) \ge \log_2 \face(P)$~\cite{Goemans:2015}.
% ������� ������ ������������� ������������ � ������� ������ ��� ����������.
\end{enumerate}
\end{property}

����� ����, ��� �������������� �������� ��������� ������������ ���������,
����������� ������ ������ �� ������ ���������.

\begin{property}\label{prop:xc-compare}
	���� ������������ $Q$ ��� ���� �� ��� ������ �������� ����������� ������������� $P$, �� $\xc(P) \le \xc(Q)$.
\end{property}

�� ��������� ��������� ��� � ���� ����������� ���� �������� �������� ����� ���������� �����������.
������, ���������, ��� ��������� ���������� �������������� ������ ����������� � �������� ��������������� � ��������� ��������������� �������� ����� ������������� �����������.
���������� �������� ���������� �� ��������� �����������.
\begin{enumerate}
	\item ��� ��� ���� ������� ����, ��������� ���������� ��� ���������������� ������������� $\Perm(n)$ ����� $\Theta(n \log n)$~\cite{Goemans:2015}.
	\item ��������� ���������� ������������� �������� �������� ������� ����� ����� $O(n^3)$, ��� $n$~--- ����� ������ �����~\cite{Martin:1991}.
	\item ��������� ���������� �������� �������� $\MinCut(n)$ ����� $O(n^3)$~\cite{Carr:2009,Conforti:2013}.
	\item ��������� ���������� �������� $s$-$t$ �������� ����� $\Theta(n^2)$, ��� $n$~--- ����� ������ �����~\cite{Garg:1995, Conforti:2013}.
	\item ����� ���������� � �������� �������� ���������� ������� $\ShortP(n)$ ��������� � ������ ��� � ��������������� ������� (��. ������~\ref{subsec:polyhedra}), ��~����~$\xc(\ShortP(n)) \le n(n-1)$.
	\item ����� ��� �����������, ��� ��� �������� ������������� �������� $\Birk(n)$ ���������� $n^2$ ����������. �~\cite{FioriniKPT:13} ��������, ��� ��� �������� ��� ���������� �������� ����� ���������� ����� ������������: $\xc(\Birk(n)) = n^2$.
	\item ��������� ���������� ��� ������������� ����������� ������������� � ��������� ������ �������������~\cite{Barahona:1993}.
	\item ���������� ��������� ��� ���������� ����������� ������������ �� ������ ����������� �������������� � ������������� ������������ ���������� ������������� ����������������~\cite{Kaibel:2010}.
	\item ���������������� ��������� ���������� ��� ������ ������������� �������������: $\xc(\BQP(n)) = 2^{\Theta(n)}$~\cite{FioriniPokutta:2015, KaibelW:15}.
	\item ���������������� ��������� ���������� ��� ������������� ������ ������������: $\xc(\TSP(n)) = 2^{\Omega(n)}$~\cite{Rothvoss:2014}.
\end{enumerate}
��� ������, �� �������� � ���� ������ �����������, ����� ����� � �������~\cite{Vanderbeck:2010, Kaibel:2011, Conforti:2013}.

��� �� �����, ������� ��� ������ ������ ������������� ����������� �������� �������������� ��������� ������ � ��������� ���������� � �������������. 
��� ������������ ������ � �������������� � ������ ����� �� $n$ ��������.
��� � ������� � ��������� ����� �������� ��������� ��������� ���������� � ������������� $O(n^3)$, ��� $n$~--- ����� ������ �����~\cite{SchrijverCO:2003}.
�~������ �������, �~2014~���� ������� �������~\cite{Rothvoss:2014}, ��� ��������� ���������� ������������� $\Match(n)$ ���� ������ ����� $2^{\Omega(n)}$.

������� �����, ��� ��������� ����������� ������������ �� �������� ������������� ��������������� �������������.
���� ���� �������������� ��������� ������� ��������.
��������� ���������� ����������� $n$"~��������� ����� $O(\log n)$~\cite{BenTal:2001}.
��� ���� ���������� ������� (������������) $n$"~����������, ��������� ���������� ������� ����� $\Omega(\sqrt{n})$~\cite{Fiorini:2012polygons}.
� ������ �������, ������������� �������� $n$"~��������� ���������� ������������ ������ ��� ������.

������ ��������� ������ ��� ��������� ���������� ������ ����� ��������� �� ������� �������� \ref{prop:xc-base} � \ref{prop:xc-compare}.
���������������� ������� ������ ������������ ���������� ��������������� �������� ����������.
������ ������, ��� �������, ��������� �� ���������������� ������� ����������~\cite{Yannakakis:1991}, ��� ������������ ������� ��� ����������� ���� �����������. 

\begin{definition}\label{def:slack}
����� $X = \{\bm{x_1}, \dots, \bm{x_n}\}$~--- ��������� ������ (V-��������) ������������� $P \subseteq \R^d$, � ����������� $\bm{a_i}^T \bm{x} \le b_i$, $\bm{a_i}\in \R^d$, $b_i \in \R$, $i \in [m]$, ���������� ��� ���������� (H-�������� ��� ����� ���������).
������ $M_{ij}$ \emph{������� �������} $M \in \R^{m\times n}$ ��� �������� V-�������� � H-�������� ������������� $P$ ������������ ��������� �������:
\[
M_{ij} = b_i - \bm{a_i}^T \bm{x_j}.
\]
� ���������, $M_{ij} \ge 0$, �� ���� ������� $M$ \emph{��������������.}
\end{definition}

�� ����������� ����� �������, ��� ������� ������� $M$ ������� � �������� ���������� �����������"=������ $K \in \{0,1\}^{m\times n}$ ������������� $P$:
\[
K_{ij} = \begin{cases}
1, & \text{���� $M_{ij} = 0$},\\
0, & \text{�����}.\\
\end{cases}
\]

\begin{definition}\label{def:nonneg}
\emph{��������������� ������} $\rank_+(M)$ ������� $M \in \R_+^{m\times n}$ ���������� ���������� ����������� $r$ �����, ��� $M$ ����������� � ���� ������������ ��������������� ������ $T$ � $U$ ������� $m\times r$ � $r\times n$, ��������������.
��������������� ���� ����� ���� ����� ��������� ��� 
���������� $r\in\N$ �����, ��� $M$ ����������� � ����
����� $r$ ��������������� ������ ����� ����.
\end{definition}

% ������ $\Q$ �� $\R$ ��� ������ $T$ � $U$ ������������ ����� ��������� ��������������� ����, �� �������������� ������������ ����������� ������������ ������ ����� �������� ������������� (��� �����, ��� �� �������� ��� � ����� �������) � ��� ���� ������������� �������� � ��������. �� ���� � ����� ������ �������� ������� ����� ��������������� � ������������� ������� �������������.

\begin{theorem}[���������~\cite{Yannakakis:1988, FioriniKPT:13}]
��������� ���������� ������������� ����� ���������������� ����� ��� ������� �������.
\end{theorem}

��������~\cite{Conforti:2010}, ��� ����������� ����������� ����� ����� � ��� ���������� ��������� ��� ��������������� ������������� ������� ������� �������.

� ��������� ������� ��������������� ������������� ������ ���� ������� (�������~\ref{thm:rcxc}), ����������� ���������� ��������� ������ ������ ��������� ����������.

\subsection{����� �������������� ��������}

����� ��� ���� ��������, ��� ������� ������� ����� ������� � �������� ���������� �����������"=������.
� � ��������������� ������ ������� ������� ������� ����� �������������� �������� ������� ����������, ���������� � ��������� ������ ��� ������������� ��������.

\begin{definition}[\cite{FioriniKPT:13}]\label{def:rect}
����� $M \in \{0,1\}^{n\times k}$~--- ������� ����������.
��������� $I\times J$, ��� $I\subseteq [n]$, $J\subseteq [k]$, ���������� \emph{0-���������������} �~�������~$M$, ���� $M_{ij} = 0$ ��� ���� $i\in I$ �~$j\in J$.
\emph{������������� ���������} ������� $M$ ���������� ��������� 0-���������������, ����������� ������� 
��������� � ���������� ������� ����� �~$M$.
\emph{������ �������������� ��������} ������� ���������� ���������� ����� 0-���������������, ����������� ��� � �������������� ��������.
����� �������������� �������� ������� ���������� �����������"=������ (������"=�����������) ������������� $P$ ���������� $\rc(P)$.
\end{definition}

� �������� ������� �� ���.~\ref{fig:8gon} ���������� ������� ���������� ���������������, � �� ���.~\ref{fig:8gonRect}~--- ����� ����������� � 0-���������������. ������ ���������� ����������� ���������������, �������� 6 �����������, ����������� �� ���.~\ref{fig:8gonEF}.

\newcommand{\paintentry}[4]{\node[fill opacity = #4] at ({#2 - 0.5}, {8.5 - #1}) {#3};}
% ������ ����
\newcommand{\paintzero}[2]{\paintentry{#1}{#2}{0}{1.0}}
% ������ 0-�������������
\newcommand{\paintr}[2]{%
	\foreach \i in {#1}{
		\foreach \j in {#2}
		{\paintzero{\i}{\j}}
	}	
}
% ������ �������
\newcommand{\paintone}[3]{\paintentry{#1}{#2}{1}{#3}}
% ������ ��� ������� ������� ���������� 8-���������
\newcommand{\paintones}[1][1.0]{\paintone11{#1} \paintone18{#1} \foreach \i in {2,...,8} {\paintone{\i}{\i}{#1} \paintone{\i}{\i-1}{#1}}}
% ����� ����������� ������� � ������������� �� ���������
\tikzset{eightgon/.style={scale=0.6, thin, line join = round, %baseline=-1mm,
baseline={([yshift=-\the\dimexpr\fontdimen22\textfont2\relax] current bounding box.center)}}}

\begin{figure}[hb]
\centering
$\left(\tikz[eightgon]{
%	\node at (-0.1,4) {$\left(\rule{0pt}{2.6cm}\right.$}; 
	\paintones % ������ ��������
	% ������ ����
	\paintr{1}{2,...,7}
	\paintr{2}{3,...,8}
	\paintr{3}{1,4,5,...,8}
	\paintr{4}{1,2,5,6,7,8}
	\paintr{5}{1,2,3,6,7,8}
	\paintr{6}{1,...,4,7,8}
	\paintr{7}{1,...,5,8}
	\paintr{8}{1,...,6}
}\right)$
\caption{������� ���������� �����������"=������ ���������������.}
\label{fig:8gon}	
\end{figure}


\begin{figure}
	\newcommand{\paintrect}[2]{%
		$\left(\tikz[eightgon]{
			\paintones[0.5] % ������ ��������
			\paintr{#1}{#2} % ������ 0-�������������
		}\right)$
	}
	\centering
	\paintrect{1,...,4}{5,6,7}
	\quad
	\paintrect{5,...,8}{1,2,3}
	\quad
	\paintrect{4,5}{1,2,6,7,8}
	\\
	\paintrect{3,6}{1,4,7,8}
	\quad
	\paintrect{2,7}{3,4,5,8}
	\quad
	\paintrect{1,8}{2,...,6}
	\caption{����� 0-���������������, ����������� ������� ���������� ���������������. (������� ��������� ��� �������� ����������.)}
	\label{fig:8gonRect}	
\end{figure}

\begin{figure}
	\centering
	\begin{tikzpicture}[scale=2]
	\foreach \i in {0,1,4,5} {\coordinate (\i) at
		({cos(\i*45-45)},{sin(\i*45-45)});}
	\foreach \i in {2,3,6,7} {\coordinate (\i) at
		({cos(\i*45-45)},{sin(\i*45-45)});}
	\draw (7) \foreach \i in {0,...,7} {-- (\i)};
	\draw (0) -- (6) (1) -- (5) (2) -- (4);
	\draw[dashed] (3) -- (7);
	\end{tikzpicture}
	\caption{���������� ��������������� � $\R^3$ (��� ������), ������� 6 �����������.}
	\label{fig:8gonEF}	
\end{figure}

��������������� �� ����������� \ref{def:nonneg} � \ref{def:rect} �������, ��� ����� �������������� �������� ������� ���������� �� ����������� ���������������� ����� ������� �������.
����� �������, ����������� ���������

\begin{theorem}[���������~\cite{Yannakakis:1988,FioriniKPT:13}]
	\label{thm:rcxc}
	$\xc(P) \ge \rc(P)$ ��� ������ ��������� ������������� $P$.
\end{theorem}

�������� �����, ��� ������ ������ � ��������~\ref{prop:xc-base} ����� � ��� ����� �������������� ��������.

\begin{property}[\cite{FioriniKPT:13}]\label{prop:rc-base}
	����� $P$ "--- �������� ������������, $\face(P)$ "--- ����� ���� ��� ������. �����
	\[
	\dim(P) + 1 \le \log_2 \face(P) \le \rc(P).
	\]
\end{property}

����� ����, ��������~\ref{prop:xc-compare} ��� ������ $\xc(\cdot)$ �� $\rc(\cdot)$ ���� �������� ������.
�~����� ������ ������� ����� �������������� �������� ������� ���������� �����������"=������ ������������� ����� ������������ �~\cite{FioriniKPT:13}.

����������� ��� ��������� � ��������� ����� ������ ������ ��������� ���������� �������������� �������� � �������������� �������~\ref{thm:rcxc} � (���������� �������������) ������� \ref{prop:rc-base}, \ref{prop:xc-compare}.
����������� �������� ��������� ��� �������:
\begin{enumerate}
	\item ��� ������������� ������������� $\Match(n)$ ����������� ���������������� ������ ������ ��������� ����������~\cite{Rothvoss:2014}.
	��� ���� $\rc(\Match(n)) \in [n^2, n^4]$, ��������~\cite{FioriniKPT:13}.
	\item ���������� ���� ������������� $n$-����������, ��������� ���������� ������� �� ������ $\sqrt{2n}$~\cite{Fiorini:2012polygons}, ����� ���
	����� �������������� �������� ������� ���������� $n$"~��������� ��������� � ��������� $\left[\log_2 (2n), 2\log_2 (2n)\right]$ (��� ������� ��~\cite{BenTal:2001, Fiorini:2012polygons}).
	\item �������� ������������� �������� �������������� ��������� � ���������������� ���������� ����������~\cite{Rothvoss:2013} (� �������������� ������������ ��� ����, ��� ����� ��������� ��������� ������ ��������������� ������������ ����� ��������� ���������"=��������~\cite{Dukes:2003}). � �� �� ����� ����� �������������� �������� ��� ��� �� ��������� �������� �� ����� ��������� ���������"=��������~\cite{Kaibel:2016}.
\end{enumerate}
�������, ��� �� ���� ���� �������� ���� ���� � ������������� ���������� ������� ������������� �����������.
����� �������, ��� ��������� ����� ������� � ���, ��� ����� �������������� �������� ������� ���������� ������"=����������� ������������� ���� ������ ������ ������ ������ ��������� ��������������� ��������������� ������.
(������������� ������������� ����������� ����� �������� ���� ����������.)


\subsection{�������� ����� ����� �������������}

�~1980-� ����� �.\,�.~����������~\cite{BondBook:1995} ���� ������� ������ ���������� ������� ���� ��� ����� ������������� �����������. 
�������� ������������ ���������� ����� ������ �������� ��, 
��� �� ������������ ����������� ����� �������� ������ ����� ������������� ��������������� �������� ������ ������������� ����������� (� ���������~\cite{BondBook:1995} �������� ����� ���������� ����������).
����� ��������� �������� ���� ������ ���������� ����, � �������~\ref{sec:Direct}.
����� �� �� �������� ���� ����� ��������� �����������.
����� ����� �������� ����� ����� ������������� $P$ ���������� $\omega(P)$.

�~\cite{BondBook:1995} ��������, ��� ��������� ����������, 
������ �������� ��� �������� (� ���������, ��� ������ � ����������� �������� ������), 
�������� �������� ��� ����������� ���� �~�����,
%�������� �������� ��� ������ � ���������� ������������� � ������������ �����,
�������� �����--�����--�������� � �������� ������ �~������ ��� ������ ������������
%, �~����� ��������� ������ ������������� ���������
�������� ����������� ������� ����.
��� ���� ����������� ��������������������� �������� ����� ������ ��������������, ���������������
� ������ NP-����\-���� ��������, ��� ������
� ������������ �������~\cite{Beloshevskii:1986,Barahona:1986},
������ � �����~\cite{Greshnev:1984}, 
������ �� �������� �������� � ��������������� �������������~\cite{Shovgenov:2015}, 
������ � ���������� �����~\cite{Bondarenko:1996},
������ ������������~\cite{Bondarenko:1983}, 
%������ 3-������������, %������ �������� �~�����,
 ������ � 3-�����������,
 ������ ��������� �� �����
% ������ � �������� �~�������� ��������� 
�~��������� ������~\cite{BondBook:1995}.
�~�� �� �����, �������� ����� ������������� ��� ��������� �������������� (��������������� � ������������� ����������� ��������): 
��� ����������� ��� ����� ����, ��� ������������� �������� �������� � ������ ����� �� $n$ �������� ����� $\lfloor n^2/4\rfloor$ ��� $n > 3$~\cite{Belov:1985}, 
������� ��� ���������� ��������������� ��������������~\cite{BondBook:1995}.

��� �������������� ����� ���� ����� ���������� ��� ��� �������,
����� �� ������� ������ (������������ �������� ������� �������) ������������� ��������� �����������.
��������, ����������� ����������������� ��� ������������ ����� � ���������� ���� � ����������� ������� �~�����.
������ ��������, ��� ��� ������ ����������� ��� ������ NP-������, � � ���~--- ������������� ���������.
� ����� ������� ������ ����� ������������� ��������������� ���� ��������� ��������� ������������ �������� ������ ������~\cite{Maksimenko:2004}.
� ���������, ��� ����� � ���������� ���� � ����������� ������� � ������������ ����������������� ����� �������� ��������� �������� ������� ��������������� ��������� $\ShortP(n)$ � $\MinCut(n)$.
��������, ��� ��� ����� ���������� ����� �������� ����� ������ �������������� ���������������, � ��� ����������� �����������������~--- �������������~\cite{Maksimenko:2004, Nikolaev:2013}.

������ � ���, �������� ������� ������������� ���������� ����� � ���������������� $\omega(P)$.

� ������ �������, ��� �������������� ������������ ������ ������ ������������� ����������������, ������� �������������� H-�������� � ���������������� �������� ����� �����~\cite{Bondarenko:1987, Padberg:1989}.

������ ������ ������ � ������� ��������� ��������� ���������� �� ��������� ����� ����� �������.
��� ��������, ��� ������ ��������� ������������� ������������ ������� ���������� � ������� (����� ������) ��� �������������~\cite{Pan:2002, Sagraloff:2016}. (� ��������� �� ������������� ������������ ���������� $O(d^2(d + \tau))$, ��� $d$ "--- ������� ����������, � $\tau$ "--- ����� ������ �������������.)
�~������ �������, ��������� � ���� ������� ����������� ������������ $\CP_d(a,b)$ 2-���������� (������� �������, ��� ���� �����) ��� $d\ge 4$, � ����� ��� ������ ��������� � ������ ����� ����� ������� $[a,b]$.
�� ���� �������� ����� $\omega(\CP_d(a,b)) = b-a+1$ ��������� ���������������� ��������, ���� ����� ������� $[a, b]$ ���������������.
%������� �� ��������� ����������� �������������� � ������ ����, � �������~\ref{sec:cyclic}.



\section{�������}
\label{sec:questions}

{\color{red}������ �����. ������� ����� ������ ���������!}

����� $S = \{P(n)\}$ "--- ��������� ��������� ��������������, ������ �� ������� ����������� ��� �������� �������� ���������� ��������� $X_n \subset \R^d$, $n\in\N$, $d = d(n)$. 
� ���� ���������� ������� ��������������� ������ OPT(S): 
��� ����� $n\in\N$ � ������� ������ $\bm{c} \in \R^d$; 
��������� ����� $\max \Set*{\bm{c}^T \bm{x} \given \bm{x} \in P(n)}$.

\begin{question}
	\label{que:1}
	���� �� ����� ����� ���������� ������������� ������� ������������� � ���������� ��������������� ��������������� ������?
\end{question}

\begin{question}
	\label{que:2}
	���� �� ����� ����� ���������� ������������� ����� ������������� � ���������� ��������������� ������?
\end{question}

\begin{question}
	\label{que:3}
	���� �� ����� ����� ���������� ������������� ���������� ������������� � ���������� ��������������� ������?
\end{question}
������ ������.

\begin{question}
	\label{que:4}
	���� �� ����� ����� ��������� ����� ������������� � ���������� ��������������� ������?
\end{question}

\begin{question}
	\label{que:5}
	���� �� ����� ����� �������� ������ ����� ������������� � ���������� ��������������� ������?
	� ���������, ���������� �� ������� (NP-����\-���) ������ � ���������~$\omega(X)$?
\end{question}

\begin{question}
	\label{que:6}
	���� �� ����� ����� ���������� ���������� ������������� � ���������� ��������������� ������?
\end{question}

\begin{question}
	\label{que:7}
	���� �� ����� ����� ������ �������������� �������� ������� ���������� ������"=����������� ������������� � ���������� ��������������� ������?
\end{question}

\begin{question}
	\label{que:8}
	����� ��������� � ��������� ����� ������������"=�������������� �������������� ������������� �������� ��������� �������� ��������� ��������������� ������.
\end{question}


\begin{question}
	\label{que:9}
	���� �� ����� ����� ������������� ����� ������������� � ���������� ��������������� ������?
\end{question}





%������ � ���������� �����.

%������ � ����� � ���� ���������: ��� ���. ����������, � ���. �����������~--- ������� ����������� �����.
%��� ������������ ����� ������������� �����������, ��� �������, ����������� ����������� $X \subset \{0, 1\}^d$. 

%�������� ������� ������ ��������� �����.
%������, ������ 2-��� ������������� ���������, �� � ��������������� ������� --- ���.

%% Глава 3
%%%%%%%%%%%%%%%%%%%%%%%%%%%%%%%%%%%%%%%%%%%%%%%%%%%%%%%%%%
%
%     �������� ����������
%
%%%%%%%%%%%%%%%%%%%%%%%%%%%%%%%%%%%%%%%%%%%%%%%%%%%%%%%%%%
\chapter{�������� ����������}
\begin{flushright}
�� ��������� � ���������
\end{flushright}


\section{�������� ���������� �����}

� ���� ������� ���������� ����������� �������� ���������� �����, �������� ������� ��� ��������� ����������� �������� ���������� �������������� �����.

�������� ������������ \ref{def:COP}, \ref{def:LCOP} � \ref{def:family}, ������� ������ �������� ������ ������������� ����������� ������� �� ���� $I$, ����������� ����� ������������� (������������� ��������� ���������� ������� $X(I)$), � �������� ������� $\bm{c}$.

���������� ��� �������� �������� ������ ������������� �����������.
��� ���� ������������� � �������� ������� ������ ������ ���������� ����������� $I$ � $\bm{c}$, ��������������. ��� ������ ������ "--- $I'$ � $\bm{c'}$.
������ �������, ������������ ��������� ���������� ������� $X = X(I)$ ������ ������, ���������� $d$, $S$ � $g$. ��� ������ ������ "---  $d'$, $S'$ � $g'$. ��������� ���������� ������� ������ ������ ���������� $X' = X'(I')$.

\begin{definition}
	����� ��������, ��� �������� ������ ������������� ����������� $(d,S,g)$ \emph{������� ��������} � ������ $(d',S',g')$, ���� ����������:
	\begin{enumerate}
		\item 
		�������������� �������� $A$ �������������� ������� ���� $I$ ������ ������ � ��� $I'$ ��� ������ ������: $A \from I \mapsto I'$.
		\item 
		�������������� �������� ���������� ��� ������� ���� $I$ ��������� ����������� $\alpha\from \R^d \to \R^{d'}$, ��� $d = d(I)$, $d' = d'(A(I))$.
		\item 
		%���������
		������� $\beta\from Y \to X$, ��� $X = X(I)$, � $Y$ "--- ��� ��������� ���� ����� $\bm{y} \in X' = X'(I')$, ��� ������� �� ������� �������� ������� ������ $\bm{c} \in \R^d$ �, ��������������, $\bm{c'} = \alpha(\bm{c})$ �����, ��� $\bm{c'}^T \bm{y} \ge \bm{c'}^T \bm{x'}$ ��� ���� $\bm{x'} \in X'$.
		%\[Y = \Set*{\bm{y} \in X'(I') \given \exists \bm{c} \in \R^d \ \bm{c'} = \alpha(\bm{c}), \ \forall \bm{x'} \in X'(I')\ \bm{c'}^T \bm{y} \ge \bm{c'}^T \bm{x'}}.\]
		
		������ ��� ������ $\bm{y} \in Y$ � ������ $\bm{c} \in \R^d$
		\[
		\Bigl(\forall \bm{x'} \in X' \quad \alpha(\bm{c})^T \bm{y} \ge \alpha(\bm{c})^T \bm{x'}\Bigr) \Longrightarrow \Bigl(\forall \bm{x} \in X \quad \bm{c}^T \beta(\bm{y})  \ge \bm{c}^T \bm{x}\Bigr).
		\] 
	\end{enumerate}
\end{definition}
%������: 1) ��������� -- ����������� ��, ��� �� �������, ����������� ��� ��������� ��������; 2) �������� -- ������ ����� ������������� (�� ����� ���) ������������� ���������, ���, ���� ������ ���������, �� � ��������� ���������.

��������������� �� ����������� �������, ��� �������� ���������� ����� ������ �� �������������� ����������.

����� ��� � ������ ������� ����� ������������ �������� �������������� ����� � ��������� � ���� �������������� �����������. ������� ������ ����������� ����� �������������� � ������������ � ������ �������� ��������������� �������������� ��������.

%%%%%%%%%%%%%%%%%%%%%%%%%%%%%%%%%%%%%%%%%%%%%%%%%%%%%%%%%%
%
%     �������� ���������
%
%%%%%%%%%%%%%%%%%%%%%%%%%%%%%%%%%%%%%%%%%%%%%%%%%%%%%%%%%%
\section{�������� ��������� ������������ �������� ������}
\label{sec:Cones}

����� $P$ "--- �������� ������������ � $\R^d$. 
\emph{����� �����} $F$ ������������� $P$ ��������� ��������� �������:
\[
K(F) \coloneq \Set*{\bm{c}\in\R^d \given  \bm{c}^T \bm{x} \ge \bm{c}^T \bm{y}, \ \forall \bm{x} \in F, \ \forall \bm{y} \in P}.
\]
� ���������, $\bm{0} \in K(F)$ ��� ����� ����� $F$.
������� �����, ��� ����������� ������ ����� �� ���������, ���� ������������ $P$ � ��� ����� $F$ �������� ���������������� ����������� ������.

��������� ���� ������� $K(F)$, ��� $F$ ��������� ��������� �������� ������ ������������� $P$, ���������� \emph{���������� ������}~\cite[�.~257]{ZieglerBook}.

���������� ���� �������� ������������ � ������������� �������������� ����������.
� ������, $\dim K(F) = d - \dim(F)$ �, ���� $F$ �������� ����������� ������ ����� $G$ ������������� $P$, �� $K(G)$ �������� ������ ������ $K(F)$.
�������, ����� ����� ������ $K(F)$ ���� ����� $K(G)$, ��������������� ��������� ����� $G$ ������������� $P$ �����, ��� $F \subset G$.

��� ��� ����� �������� ������������ $P$ ���������� ������������ ���������� ����� ������ $X = \ext(P)$, �� � ��� ���������� ���� ���������� ������������ ������� ������� ��� ������
\[
\K(X) \coloneq \Set*{K(\bm{x}) \given \bm{x} \in X},
\]
������� ����� ����� �������� \emph{�������� ����������} ������������ �������� ������ ������ �������� ����������� �� $X$.
��� �������� ����������� ���, ��� ������ ����������� ����� �������� ������� $\bm{c}\in\R^d$ �� ��������� $X \in \R^d$ ����� ����������������� ��� ������ ������ ������ $\bm{x} \in X$, ��� $\bm{c} \in K(\bm{x})$.

����� $\bm{x}$ � $\bm{y}$ "--- ������� ������������� $P \subset \R^d$.
����� �������� ������ $K(\bm{x})$ � $K(\bm{y})$ \emph{��������}, ���� ��� ����� ����� ����������:
\[
\dim (K(\bm{x}) \cap K(\bm{y})) = d-1.
\]
��������, ������ $K(\bm{x})$ � $K(\bm{y})$ ������ ����� � ������ �����, ����� ������ ������� $\bm{x}$ � $\bm{y}$.
��������������, ���� ������������� $\conv(X)$ ��������� � ������ ��������� ��������� $\K(X)$.

������������ ������������� ������ $\conv(X)$ ������������ ��������� ��������� $\K(X)$ ������������ �������� ������ ����������� � ���, ��� ���������� ����������� ������� ������������� ��������� �������������� �������� ����������� �� ������� ������ $\bm{c}$.
���, ��������, � ������������ ������ � ���������� ���� ������������� ����������� $\bm{c} \le \bm{0}$ (��� ������ ������������).
����� �������, ����� ������� �������������� (��������������) ��������� ��������� ����� ������������ �������� ������.
� ������ ������ � ���������� ���� ���� ���� � �������� ��������� �������������� �������.

�� �������� � ������������ ������ $K(\bm{x})$ ��� ������� $\bm{x}$ ������������� $P\subset \R^d$ ������ �����������
\[
K(\bm{x}, Q) \coloneq \Set*{\bm{c}\in Q \given  \bm{c}^T \bm{x} \ge \bm{c}^T \bm{y}, \ \forall \bm{y} \in P}, 
\]
��� $Q$ "--- ������� � $\R^d$.
��������� ���� ����� ������� ��� ������������� $Q$ ��������� $\K(X,Q)$ � ����� �������� \emph{���������� $Q$ �� ��������� $X$}.
���� ������ ��������� ������������ �� �������� � ������ ��������� ���������.
����� �������, ���� ��������� $\K(X,Q)$ �������� ��������� ����� ��������� ��������� $\K(X)$.

��������, ��� ��� �������� �� ����� ������������ �������� ������ � �������� $Q$ ������ ����������� ����� ����� ������ ���� �������������� ��������� (����������� � ��������) � ����������� �� �����������, ������ �� ������� ������������. � ����� ������ ������������� ���������, ���������� ���� ��� ���� ����� ������ ����� ����� ���������, ����� �������� ��������� �� �����.

�������, ��� ������� ������ ����������� ����������� ������������ ��������� �������� ������� �� ������������� ������. ����� �������, �� �������� ��������, ������ ��������� ��������� $\K(X)$ ����� ������������ ������������� ��������� $\K(X,\Cube(-1,1))$, ��� 
\[
\Cube(-1,1) \coloneq \Set{\bm{x} \given \bm{-1} \le \bm{x} \le \bm{1}}.
\]
%� ���������, ����� ���� ��������� ���������.




%\section{�������� ����������}

{\color{red}
	����� ����������� ������� �������� ���������� (��� �������� ��������������)
	�������� ����������� � ���, ��� ��������������� ������ ����������� ������������� ����������������! ������, ����� ����� ��������, ��� ���������� �� ������� ��������������, � ������� ������������� ������ �����, �� ��������� ������ ������������� �������� � ������������� ���������. �� ����� ����� � �� ������, ���� ���� ������ �� ���������, � ��������� � ������ ������� �������� ��������������.
	
	������� ��������� ��������� � �������������� ������� ���������.
	�������� ����������� �������� ���������� ���� (��������� ��������������, ��������� ���������) � ������ ����� ����. ���� ���������� ��������������� �����.
}

\begin{definition} 
	����� ��������, ��� ������ $(X,S)$
	��� ������������� ��������� �������� ������ $S\subseteq R^m$
	{\it ������� ��������} � ������ $Y=(Y,T)$, ��� $Y,T \in \R^n$, $m\le n$,
	���� �������� 
	\begin{enumerate}
		\item ����������� �������� ����������� $A: S\to T$ � ������������� ����������� ��������������, 
		\item � �������-����������� ������������ $B: Y'\to X$ ����� ��������� ������������� $Y'\subseteq Y$ � ���������� $X$ 
	\end{enumerate}
	�����, ��� ��� ������� ������� $s\in S$ ��������� ��������� �������:
	\begin{equation}
	\label{TheoryAff}
	\begin{array}{c}
	y_0 \mbox{ �������� �������� ������ } [Y,A(s)], 
	\mbox{ ����� � ������ �����,}\\
	\mbox{����� }
	y_0\in Y' \mbox{ � } x_0=B(y_0) \mbox{ --- ������� ������ } [X,s].
	\end{array}
	\end{equation}
	\[
	A(s) \in K_Y(y_0,T) \iff y_0 \in Y' \text{ � } s \in K_X(B(y_0), S).
	\]
	��� ����� ����� ������ �����������: $(X,S)\propto_A (Y,T)$. 
	���������� �������� �������� ����������.
\end{definition}

{\color{red}

%\prop{��������} 
{��������� $T$ �� ����������� �������� ���������� 
	������� ������� ������������� ��������� $S$.} 
��� ������� �� ��������������� ����������� $A$.

%\prop{\label{Theory2} ��������} 
{��� ������ $x\in X$ � $y=B(x)\in Y'$ �������� $K(x,S)$ � 
	$K(y,T)$ ���� ������� �������.} ���������� $K(x,S)\sim K(B(x),T)$. 
��� �������� �������� �� ��������������� ��������� ����������� $A$ 
� ������� (\ref{TheoryAff}).

%\prop{\label{Theory3} ��������} 
{����� ���� $K(x_1,S)\cap K(x_2,S)$, ��� $x_1,x_2\in X$,
	�������� ����������� $A$ ��������� � �������� ����� $K(y_1,T)\cap K(y_2,T)$,
	��� $y_1=B(x_1)$, $y_2=B(x_2)$.}  

%\prop{\label{Theory4} ��������} 
{$\bigcap\limits_{i=1}^l K(x_i,S) \sim \bigcap\limits_{i=1}^l K(y_i,T)$, 
	��� $x_i\in X$, $y_i=B(x_i)$, ��� $1\le i\le l$.}  
��� �������� �������� ������������ ���������� ���������� ���� �������.

������, �������� �� ��������� ��������, ������������ ��� �����������.

%\prop{�������}
{����� $(X,S)\propto_A Y$ � $A: S\to T$, ����� 
	$G_S(X)=G_T(Y)\prec G(Y)$.}

1. �������� ��������� � ��� ��������.
������������: ������ � ��������������� ������������� �� ��������� �������� ������.
������ -- ���������� ����.

2. �������� ���������� ��� �������� ���������. �������: ��� -> ����������, ����������� -> ����������� � ������������ �����������������.
}

%%%%%%%%%%%%%%%%%%%%%%%%%%%%%%%%%%%%%%%%%%%%%%%%%%%%%%%%%%
%
%     �������� ���������� ��������������
%
%%%%%%%%%%%%%%%%%%%%%%%%%%%%%%%%%%%%%%%%%%%%%%%%%%%%%%%%%%

\section{�������� ���������� ��������������}

\subsection{��������� ��������������}

��� ��������, ��������� �������� ����� �� �������� ���������������� � ������������� ������� ������������.
����� ��������������� ��, ������ ��������� �������� ���������,
�������� ��������� ������� �� ��������� ���� �������� ��������������.

\begin{definition}\label{def:ineA}
� ������, ����� ������������ $P$ ������� ������������ �������������~$Q$ ��� �� ��� �����, ����� ������������ ����������� $P \lea Q$.
���� �������� ��������������� �������������� $P$ � $Q$ ���������� $P =_A Q$.
\end{definition}

� ��������� ����� ����� ���������� ����� ������ ������� ����� �����������,
� ������� ����� <<������� ������������>> �������� �� <<�������� �������� �������>>, � ��� ���������������� ����������� ������������ ����������� $\lee$. ��� �� ����� �������� ������ �������� � ������������� ���� ���� ����������� ��� ������������ ������� ��������������.

�������� ��������, ��� ����������� $\lea$ ����������� �������� ��������
��� ��������� ������������� ������� (�.\,�. ������� ������� ������) ��������������.

\begin{property}
���� $P \lea Q$, �� ������� ������ ������������� $P$ 
��������� ���� ���� ������� ������ ������������� $Q$ (���� $P$ � $Q$ ������������), ���� ��������� ���������� (�������������� ������ ������������� $Q$), � ������� ���������� �����������"=������ ������������� $P$ �������� ����������� ������� ���������� ������������� $Q$. � ���������:
\begin{enumerate}
	\item ����� ������ ������������� $P$ �� ����������� ����� ������ $Q$.
	\item ����� $i$-������ ������������� $P$ �� ����������� ����� $i$-������ $Q$ ��� $i \le \dim(P)$.
	\item ���� ������������� $P$ ��������� ���������� �������� ����� ������������� $Q$.
	\item ����� ����������� $P$ �� ����������� ����� ����������� $Q$.
	\item ����� ������������� �������� ������������� ����������� $\rc(P) \le \rc(Q)$.
\end{enumerate}
\end{property}

� �������� �������� ���������� ��������� ��������� �����������
��� ���������� � ����������� ��������������.
������ �����, �������� $\Delta_n$ �������� ������ ��������� $\Delta_{n+1}$, 
� � ���� �������������� ����������� $\lea$ 
�������� 
\begin{equation}
\label{eq:compareDelta}
\Delta_n \lea \Delta_{n+k} \quad \forall n,k \in \N.
\end{equation}
����� ����, 
\begin{equation}
\label{eq:compareDeltaCP}
\Delta_m \lea \CP_n(S) \quad \text{��� } m < n \le |S|, 
\end{equation}
��� ��� ����������� ������������ $\CP_n(S)$ ������������.
�� ��� �� ������� 
\begin{equation}
\label{eq:compareCP}
\CP_n(S) \nelea \CP_{n+k}(S')
\end{equation}
��� ���� $n, k \in \N$
� ����� �������� $S$ � $S'$, ��� ������� $|S| > n+1 > 2$.

�������, ��� ��������� ���������� $\Delta = \{\Delta_n\}$ � ��������� ����������� �������������� $\CP = \Set*{\CP_n(S) \given n\in\N, \ S \subset \Q, \ |S| < \infty}$ ������������ ����� ������� �������� �������������� ����� ������������� �����������.
� ����� � ���� ������� ������������ ��������� ������.
����� �� ���� �������� \eqref{eq:compareDelta} ���~\eqref{eq:compareCP} �������� �������� �������� ��� ��������� �������� �������������� �����?
���� ����������, ��� � ����������� ������� ����� ����������� �������������� ����������� ����~\eqref{eq:compareDelta}.
��� �������������� ������ ���� ����������� ������ ������������ ��������� ��������� ������������.

\begin{lemma}
	\label{lem:01face}
	����� $P \in \R^d$ "--- 0/1"~������������. ����� $F_i = \Set*{\bm{x}\in P \given x_i = 0}$ � $G_i = \Set*{\bm{x}\in P \given x_i = 1}$, $i\in[d]$,
	�������� ������� (���� ����� ��������������) ������������� $P$.
\end{lemma}

���������� ��� �������� ����� ������������� � ���������� ��������� ��������������: ������ ������������ ������������� $\BQP(n)$, ������������� ������������� ������ ������������ $\ATSP(n)$ � ������������� ������ � ������� $\Knap(n, \bm{a}, b)$.


\begin{prop}
\begin{align*}
\BQP(n) &\lea \BQP(n+1),\\
\ATSP(n) &\lea \ATSP(n+1),\\
\Knap(\bm{a}, b) &\lea \Knap((\bm{a},0), b), \quad \bm{a} \in \R^{n}, \ b\in \R.
\end{align*}
\end{prop}
\begin{proof}
������������� ������~\ref{lem:01face}.
���������� ����� $F$ ������������� $\BQP(n+1)$, ������������ ��������������� $x_{n+1, n+1} = 0$. 
����� ��� ���� $\bm{x} \in F$ ����������� $x_{i, n+1} = 0$ ��� $i\in[n+1]$.
�������� �������, ��� ��������� ������ ����� $F$ ������������� � ��������� ������ ������������� $\BQP(n)$ (�, ��������, $\BQP(n)$ � $F$) �������� ������������ (������������� ���������) $x_{ij} \mapsto y_{ij}$, $1 \le i \le j \le n$. 

��� �������������� ����������� $\ATSP(n) \lea \ATSP(n+1)$ 
���������� ���������� ������� ����������� ������������ ����� ���������� ������������� �������� ������� ������� $D=(V,A)$ �� $n$ �������� � ������������� 
������������� �������� ������� $D'=(V',A')$ �� $n+1$ ��������, � ������� ������� ��� ����, �������� � ������� $v'_1$, � ����, ��������� �� $v'_{n+1}$, �� ����������� ���� $(v'_{n+1}, v'_1)$.
�������, ��� ������������������ ������� ���������� ������������ �������� ������ $D'=(V',A')$ �������� ��������� ����� 
\[F = \Set*{\bm{x} \in \ATSP(n+1) \given x_{(v'_{n+1}, v'_1)} = 1}
\]
������������� $\ATSP(n+1)$.
��������, ����� $F$ � ������������ $\ATSP(n)$ ������� ���������� �������� ������������
\[
	y_{(v_i, v_j)} = 
	\begin{cases}
	x_{(v'_i, v'_j)}, & \text{��� }1 \le i \le n, \ 2 \le j \le n, \ i \ne j,\\
	x_{(v'_i, v'_{n+1})}, & \text{��� }2 \le i \le n, \ j=1,\\
	\end{cases}
\]
��� $\bm{x} \in F$, $\bm{y} \in \ATSP(n)$.

��� �������������� ����������� $\Knap(\bm{a}, b) \lea \Knap((\bm{a},0), b)$ ���������� ��������, ��� 
\[
\Knap((\bm{a},0), b) = \Set*{(\bm{x},x_{n+1}) \in \{0,1\}^{n+1} \given \bm{x} \in \Knap(\bm{a}, b)}.
\]
�������������, ������������ $\Knap(\bm{a}, b)$ ������� ������������ ����� ������������� $\Knap((\bm{a},0), b)$, ������������ ��������������� $x_{n+1} = 0$.
% (��� $x_{n+1} = 1$).
\end{proof}

����������� \eqref{eq:compareDeltaCP} �������� ������� �������� ��������� �������������� �� ������ ��������.
��� ����� ����� ������, �������� ���������� ����������� ��������~\cite[�.~84]{Deza:2001}, �������� \emph{������������ �����������} $\xi\from \BQP(n) \to \Cut(n+1)$, ���������� �����������
\[
y_{ij} = 
\begin{cases}
x_{ii}, & \text{��� } 1 \le i \le n, \ j=n+1,\\
x_{ii} + x_{jj} - 2 x_{ij}, & \text{��� } 1 \le i < j \le n.\\
\end{cases}
\]
%��� $\bm{x} \in \BQP(n)$, $\bm{y} \in \Cut(n+1)$.
�� ��������������� ����� ����������� �������
\[
\BQP(n) =_A \Cut(n+1).
\]

���������� ��� ��� ��������� ��������������, ��������������� � �������� �� �������� � ��������� ���������.

����� $A\in\{0,1\}^{m\times n}$ "--- ������� ���������� ��������� ��������� $G = \{g_1, \ldots, g_m\}$ � ��������� ���������� ��������� $S = \{S_1, \ldots, S_n\} \subseteq 2^G$.
�������� �������� ���������
\[
\Pack(A) = \Set*{\bm{x}\in\{0,1\}^n \given A \bm{x} \le \bm{1}}
%\quad \text{��� } A\in\{0,1\}^{m\times n},
\]
���������� \emph{�������������� �������� ���������}~\cite{Balas:1976}.
(������ ������� $\bm{x} \in \Pack(A)$ ����� ������������� �������� ������������������ �������� ��������� �������� $T \subseteq S$.)

��������� ������ \emph{������������� ��������� ���������} ������������ �� ��������:
\begin{equation}
\label{eq:Part}
\Part(A) = \Set*{\bm{x}\in\{0,1\}^n \given A \bm{x} = \bm{1}}.
\end{equation}
��������������� �� ����������� �������
\begin{equation}
\label{eq:PartPack}
\Part(A) \lea \Pack(A).
\end{equation}

������� �����, ��� ������������ ����������� �������� $\Stable(G)$ (��. ����������� �� �.~\pageref{def:Stable}) �������� ������� ������� ������������� �������� ��������:
\begin{equation}
\label{eq:StablePack}
\Stable(G) =_A \Pack(A),
\end{equation}
���� $A$ �������� �������� ���������� �����"=������ ����� $G$.



\subsection{��������� �������� ��������������}

������, ��� ������� � ��������� �������� ��������������, ��������� ��� ��������� ������� ����������� ����� $\Stable(G)$, $\Part(A)$, $\Pack(A)$ � $\BQP(n)$.

\begin{lemma}
	\label{lem:PackStable}
	��� ����� ������� $A\in\{0,1\}^{m\times n}$ ���������� ���� $G$ �� $n$ �������� �����, ��� $\Pack(A) =_A \Stable(G)$.
\end{lemma}

\begin{proof}
	���������� ��������, ��� ������ ����������� ����
	$$
	x_1 + x_2 + \ldots + x_k \le 1
	$$
	�� ������� $A\bm{x} \le \bm{1}$ ��� ������� $\bm{x}\in \{0,1\}^n$
	������������ ������ ����������
	$$
	x_i + x_j \le 1, \quad 1\le i < j \le k,
	$$
	������������ ��������� ������������ ����������� ��������.
\end{proof}

�������� �����������~\eqref{eq:StablePack}, ����� ������� ����� � ���, ��� ��������� $\{\Pack(A)\}$ � $\{\Stable(G)\}$ ��������� (������� �� ����� � ��� �� ��������������).

\begin{lemma}
	\label{lem:StablePart}
	��� ������ ����� $G=(V,E)$ ���������� ������� $A\in\{0,1\}^{m\times n}$, $m = |E|$, $n = |V|+|E|$, �����, ��� $\Stable(G) =_A \Part(A)$.
\end{lemma}
	
\begin{proof}
	��� ������� ����������� 
	\begin{equation}
		\label{eq:Stable2}
		x_v + x_u \le 1,  \quad \{v,u\} \in E,
	\end{equation}
	�� �������� ������������� $\Stable(G)$ ������ ��������������� ���������� $y_{vu} = 1 - x_v - x_u$.
	�������� ��������, ��� ��������� 0/1-��������, ��������������� ������������~\eqref{eq:Stable2}, ������� ������������ ��������� 0/1-��������, ��������������� ����������
	\begin{equation*}
		x_v + x_u + y_{vu} = 1,  \quad \{v,u\} \in E.
	\end{equation*}
\end{proof}

����� �������, �������� �����������~\eqref{eq:PartPack}, ��������� $\{\Part(A)\}$ �������� �� ������ ��� ������������� �������� $\{\Pack(A)\}$ � $\{\Stable(G)\}$, �� � ��������� �� �����.

������ ������ �������� ������� ���� ����� "--- ������ �������������� ���������� ����--�����--������~\cite{Garey:1982} ��� �������� ��������������.
����� ����� ���� ������������� � ����������� ������������, � ������� �� ���������, ����� �������� \emph{��������} �������������.
%��������, ������ ������������� ���� �� ��� ����, ��� ����� ����� ��������������� ������ ������������� �����������.
����� �������, ������ ������������� �������������� ����� ����� ��������������� ������ ������������� �����������, ���� ������ ��������� �������� ������� ������ ��������� ������ ��������� ����������.

%\begin{definition}
%	\label{def:Aff}
%	��������� �������������� $P$ \emph{������� ��������} � ��������� �������������� $Q$, ���� ��� ������� ������������� $p\in P$ �������� $q\in Q$ �����, ��� $p \lea q$, ������ ����������� ������������, � ������� ��������� ������������ $q$, ���������� ������ ��������� �� ����������� ������������, � ������� ����� $p$.
%	�����������: $P \propto_A Q$.  
%\end{definition}

%�������, ��� ����� �����\footnote{�������������� ������������� �������� ����� ����������� ������� ������~\cite{Garey:1982}.} ��� ������ �������� ����������� ������ ��������������� ����� ����� ���� $s$ � ����������� $d(s)$.


\begin{definition}
	\label{def:Aff}
	����� ��������, ��� ��������� �������������� $P$ \emph{������� ��������} � ��������� �������������� $Q$, ���� ��� ������� ������������� $p\in P$ �������� $q\in Q$ � �������� ����������� $\alpha$, ���������� �������� 
	\[
	\bm{y} = A\bm{x} + \bm{b}, \qquad \bm{x} \in p, \quad \bm{y} \in q, 
	\]
	�����, ���:
	\begin{enumerate}
		\item\label{item:1} ����� $\alpha(p)$ �������� ������ (�������� �������������) ������������� $q$ � ������� ������������ $p$.
		%����� ���� ���� ���������� ���: $p \lea q$.
%		\item\label{item:2} ����������� ������������, � ������� ��������� ������������ $q$, ���������� ������ ��������� �� ����������� ������������ ��� $p$.
		\item\label{item:3} ������ ������������� $q$ ��������� ������ ��������� �� ������� ������������� $p$.
		\item\label{item:4} ������������ %��������� ����������� $\alpha$
		������� $A$ � ������� $\bm{b}$ 
		����������� �� �������������� ����� ������������ ������� ������������� $p$.
	\end{enumerate}
	���� �������� ���������� $P$ � $Q$ ���������� ���: $P \propto_A Q$.  
\end{definition}

%�������, ��� ��� ����������� ���������� �� ����������� �������� ����������,
% ���������� ����� � ���������� \cite{Bondarenko:2008}, 
% � ������� �������� �������.

\begin{remark}
	� ����������� \ref{def:Aff} ������ �������� �������� �������������� ����������� �������� ��������������, �~��~�� ������������. 
	���� � ���, ��� ������� �����������, ������������ ������������, ����� ��������� ������ �� ����� ��������� �������� ��� ��������. ��� �� �����, ����� ����� �������� ������ ��������� ������� ������ �����, ��������������� � ���������������.
	
	� �������� ������� ������������� � ����������� ��������� ����� ����������� $\Part(A)$, ���������� �����������
	\[
	x_1 + x_i + x_j = 1, \quad 2 \le i < j \le n.
	\]
	��������, �� ������� �� ����� ������������ ����� $(1,0,\dots,0)$.
	������ � ���, �������� \eqref{eq:PartPack}, �� �������� ������ ������������� $\Pack(A)$, ������������� �������������
	\[
	x_1 + x_i + x_j \le 1, \quad 2 \le i < j \le n,
	\]
	� �������� ����������� $n$.
	����� <<�����������>> ������ ��������� �������������� ������ ����������� ������������~\ref{def:Aff}.
\end{remark}

\begin{remark}
	����������� \ref{def:Aff} ���������� �� ����������� �������� ���������� �~\cite{Maksimenko:2013NP,Maksimenko:2016bool} �������� �������~\ref{item:3} �~\ref{item:4} (���������������� ������� ������������� $q$ � ������������� ��������� �����������).
	��� �� �����, ��� ���� ������ �������� ����������, ����������� � �������~\cite{Maksimenko:2013NP,Maksimenko:2016bool}, �������������� ���� ������� ����� �����������, ��� ��� ��������������� �������� ����������� ������� ����� �������.
\end{remark}

\begin{remark}
��� �������, �������������� �������� ���������� ��������� $P$ � ��������� $Q$ ����������� �� ��������� �����.
��� ������� ���� $I$, ��������� ������������ $p(I)\in P$ ���������� �������� ���� $I'$ ������������� $q(I')\in Q$, ��� ����� $F$ � ����������� ��������� ����������� $\alpha\colon p(I) \to F$.
�������������� ������� \ref{item:3} � \ref{item:4} ����������� \ref{def:Aff}, ��� �������, ��������.
������� � ���������� �� �� ������� �������� �������� ���� �������, � ��� ���� �������� ���������� ����������� ��� ����������� ����
<<��� ������� ���� $I$, ��������� ������������ $p(I)$ �� ��������� $P$, ���������� ��� $I'$, ������������ ������������ $q(I')\in Q$, �����, ��� $p(I) \lea q(I')$>>.
(��������� %������ ���� ����������� 
����� ������� ����� \ref{lem:PackStable} � \ref{lem:StablePart}.)
��� ����� ����������� ���, ��� ������ �����������, � ������� �� $P \propto_A Q$, �������� ���������� � �������� ����\'����. 
\end{remark}


�� ������ ���������� ���� ����������� �������� ��������� �������� ������������� ����������� �� �����������~\ref{def:Aff}.
���, ����������� \eqref{eq:compareDeltaCP} ����� ���������� � ���� $\Delta \propto_A \CP$. 
� �� ���� \ref{lem:StablePart}, \ref{lem:PackStable} � ����������� 	\eqref{eq:PartPack} �������
 
\begin{theorem}
	\label{thm:Class1}
	$\Stable \propto_A \Part \propto_A \Pack \propto_A \Stable$,
	��� $\Stable = \{\Stable(G)\}$, $\Part = \{\Part(A)\}$, $\Pack = \{\Pack(A)\}$.
\end{theorem}

���������� ��������� ��������� �������� ����� ���� ����������. 

\begin{theorem}
	\label{thm:Prop}
	����� $P \propto_A Q$. 
	�����������, ��� � ��������� $P$ ���� �������������, ������� ���� ��� ��������� �� ��������� �������:
	\begin{enumerate}
	\item C�������������������� ����� ������ ��� ����������� (������������ ������� �������������).
	\item C������������������ �������� ����� ����� �������������.
	\item NP-������� �������� ����������� ������.
	\item C������������������ ����� �������������� ��������.
	\item C������������������ ��������� ����������.
	\end{enumerate}
	\noindent
	����� � $Q$ ������� ������������� � ���� �� ����������.
\end{theorem}

������� ������ ��������� $\BQP = \{\BQP(n)\}$ � $\Stable$.


\begin{theorem}
	\label{thm:BQPStable}
	��� ������� $n\in \N$ ���������� ���� $G = (V,E)$, $|V| = n(n+1)$, $|E| = n(2n-1)$, �����, ���
	$\BQP(n) \lea \Stable(G)$.
\end{theorem}
(������� ��������� ������� �~\cite{FioriniPokutta:2015}, �� � ����� ������ ������������ $\lee$ (��. �����������~\ref{def:ineE} �� �.~\pageref{def:ineE}) � ��� $|V| = 2 n^2$.)
% ����� ����, �������������� ���� �������������� ����������� ����� ����������� �~\cite{FioriniPokutta:2015}.)

\begin{proof}
	������ ��������� $x_{ij} = x_{ii} x_{jj}$ �� ���������~\eqref{eq:BQP}, ������������� ����� ������������ ������������, ������������ ������������
	\begin{equation}
	\label{eq:Clique}
	\begin{aligned}
	x_{ii} - x_{ij} &\ge 0, \\
	x_{jj} - x_{ij} &\ge 0, \\
	x_{ii} + x_{jj} - x_{ij} &\le 1,
	\end{aligned}
	\end{equation}
	��� ������� $x_{ij}\in\{0, 1\}$, $1\le i \le j \le n$.
	�������� ������������� �� � ������� ���������� ���� $y_l + y_m \le 1$. %��~\eqref{SSP}.
	��� ����� ������ $n(n+1)$ ����� 0/1-����������:
	\begin{equation}
	\label{eq:BQP2SSP}
	\begin{aligned}
	s_{ij} &= x_{ij},           & 1 &\le i  <  j \le n,\\
	t_{ij} &= x_{ii} - x_{ij},  & 1 &\le i  <  j \le n,\\
	u_i    &= x_{ii},           & 1 &\le i \le n,\\
	\bar{u}_i    &= 1 - x_{ii}, & 1 &\le i \le n.\\
	\end{aligned}
	\end{equation}
	����� �����������~\eqref{eq:Clique} ������������
	%\begin{equation}\label{eq:SSP2}
	$$
	\begin{aligned}
	s_{ij} + \bar{u}_j & \le 1, \\
	t_{ij} + u_j       & \le 1, \\
	u_i    + \bar{u}_i & =   1, \\
	s_{ij} + t_{ij} + \bar{u}_i & = 1,
	\end{aligned}
	$$
	%\end{equation}
	��� ������� ��������������� ���� ����������.
	��������, ��������� ��� ��������� (������, $n(n+1)/2$ �������� ��������)
	���������� ��������� ����� ������������� $\Stable(G)$, ��� ����� ������ ����� $G$ ����� $n(n+1)$, � $n (2n - 1)$ ��� ����� ���������� ������� ����������
	%\begin{equation}\label{eq:SSP2}
	$$
	\begin{aligned}
	s_{ij} + \bar{u}_j & \le 1, \\
	t_{ij} + u_j       & \le 1, \\
	u_i    + \bar{u}_i & \le 1, \\
	%   s_{ij} + t_{ij}    & \le 1, \\
	s_{ij} + \bar{u}_i & \le 1, \\
	t_{ij} + \bar{u}_i & \le 1.
	\end{aligned}
	$$
	%\end{equation}
	����� ����, �����������~\eqref{eq:BQP2SSP} ��������� ��� ����� � �������������� $\BQP(n)$ ������������� �������� ������������.
\end{proof}

\begin{prop}
	\label{prop:StableBQP}
	���� ���� $G=(V,E)$ ��������, �� ����������� $\Stable(G) \lea \BQP(n)$ ���������� �� ��� ����� $n$.
\end{prop}

\begin{proof}
������ ����� �������, ��� � ��������� ������ $\Stable(G)$ ������ ������ ������� $\bm{0}$, $\bm{e_1}$, \dots, $\bm{e_d}$, ��� $d=|V|$.
���� ���� $G=(V,E)$ ��������, �� $\Stable(G)$ ����� <<������������>> �������� �������� ��� ��� ������� ���� 0/1"~������. ����� $\bm{x}$ "--- ���� �� ����� (��������������) ��������. ����� ��������, ��� ������� $\bm{0}$ � $\bm{x}$ ������������� $\Stable(G)$ ��������, ��� ��� ����������� �� ������� ������������ � �������� ��������� ������ $\bm{e_1}$, \dots, $\bm{e_d}$.
�������������, ������������ $\Stable(G)$ �� �������� 2"~�����������.
�������� ��������, ��� ������������ $\BQP(n)$ (� ������ � ��� � ��� �����) �������� 2"~�����������~\cite{Padberg:1989}.
\end{proof}


%%%%%%%%%%%%%%%%%%%%%%%%%%%%%%%%%%%%%%%%%%%%%%%%%%%%%%%
%
% End of section
%
%%%%%%%%%%%%%%%%%%%%%%%%%%%%%%%%%%%%%%%%%%%%%%%%%%%%%%%

%% Глава 4
%%%%%%%%%%%%%%%%%%%%%%%%%%%%%%%%%%%%%%%%%%%%%%%%%%%%%%%%%%
%
%     �������� ����������
%
%%%%%%%%%%%%%%%%%%%%%%%%%%%%%%%%%%%%%%%%%%%%%%%%%%%%%%%%%%
\chapter{�������� ����������}
\begin{flushright}
�� ��������� � ���������
\end{flushright}


\section{����������� � �������}

\subsection{��������� ��������������}

��� ��������, ��������� �������� ����� �� �������� ���������������� � ������������� ������� ������������.
����� ��������������� ��, ������ ��������� �������� ���������,
�������� ��������� ������� �� ��������� ���� �������� ��������������.

\begin{definition}
� ������, ����� ������������ $P$ ������� ������������ �������������~$Q$ ��� �� ��� �����, ����� ������������ ����������� $P \lea Q$.
���� �������� ��������������� �������������� $P$ � $Q$ ���������� $P =_A Q$.
\end{definition}

� ��������� ����� ����� ���������� ����� ������ ������� ����� �����������,
� ������� ����� <<������� ������������>> �������� �� <<�������� �������� �������>>, � ��� ���������������� ����������� ������������ ����������� $\lee$. ��� �� ����� �������� ������ �������� � ������������� ���� ���� ����������� ��� ������������ ������� ��������������.

�������� ��������, ��� ����������� $\lea$ ����������� �������� ��������
��� ��������� ������������� ������� (�.\,�. ������� ������� ������) ��������������.

\begin{property}
���� $P \lea Q$, �� ������� ������ ������������� $P$ 
��������� ���� ���� ������� ������ ������������� $Q$ (���� $P$ � $Q$ ������������), ���� ��������� ���������� (�������������� ������ ������������� $Q$), � ������� ���������� �����������"=������ ������������� $P$ �������� ����������� ������� ���������� ������������� $Q$. � ���������:
\begin{enumerate}
	\item ����� ������ ������������� $P$ �� ����������� ����� ������ $Q$.
	\item ����� $i$-������ ������������� $P$ �� ����������� ����� $i$-������ $Q$ ��� $i \le \dim(P)$.
	\item ���� ������������� $P$ ����������� ��������� ���������� �������� ����� ������������� $Q$.
	\item ����� ����������� $P$ �� ����������� ����� ����������� $Q$.
	\item $\rc(P) \le \rc(Q)$.
\end{enumerate}
\end{property}

� �������� �������� ���������� ��������� ��������� �����������
��� ���������� � ����������� ��������������.
������ �����, �������� $\Delta_n$ �������� ������ ��������� $\Delta_{n+1}$, 
� � ���� �������������� ����������� $\lea$ 
�������� 
\begin{equation}
\label{eq:compareDelta}
\Delta_n \lea \Delta_{n+k} \quad \forall n,k \in \N.
\end{equation}
����� ����, $\Delta_m \lea \CP_n(S)$ ��� $m \le n < |S|$, 
��� ��� ����������� ������������ $\CP_n(S)$ ������������.
�� ��� �� ������� 
\begin{equation}
\label{eq:compareCP}
\CP_n(S) \nelea \CP_{n+k}(S')
\end{equation}
��� ���� $n, k \in \N$
� ����� �������� $S$ � $S'$, ��� ������� $|S| > n+1 > 2$.

�������, ��� ��������� ���������� $\{\Delta_n\}$ � ��������� ����������� �������������� $\{\CP_n(S)\} = \Set*{\CP_n(S) \given n\in\N, \ S \subset \Q, \ \text{$S$ �������}}$ ������������ ����� ������� �������� �������������� ����� ������������� �����������.
� ����� � ���� ������� ������������ ��������� ������.
����� �� ���� �������� \ref{eq:compareDelta} ���~\ref{eq:compareCP} �������� �������� �������� ��� ��������� �������� �������������� �����?
���� ����������, ��� � ����������� ������� ����� ����������� �������������� ����������� ����~\ref{eq:compareDelta}.
��� �������������� ������ ���� ����������� ������ ������������ ��������� ��������� ������������.

\begin{lemma}
	\label{lem:01face}
	����� $P \in \R^d$ "--- 0/1"~������������. ����� $F_i = \Set*{\bm{x}\in P \given x_i = 0}$ � $G_i = \Set*{\bm{x}\in P \given x_i = 1}$, $i\in[d]$,
	�������� ������� (���� ����� ��������������) ������������� $P$.
\end{lemma}

���������� ��� �������� ����� ������������� � ���������� ��������� ��������������: ������ ������������ ������������� $\BQP(n)$, ������������� ������������� ������ ������������ $\ATSP(n)$ � ������������� ������ � ������� $\Knap(n, \bm{a}, b)$.


\begin{prop}
\begin{align*}
\BQP(n) &\lea \BQP(n+1),\\
\ATSP(n) &\lea \ATSP(n+1),\\
\Knap(n, \bm{a}, b) &\lea \Knap(n+1, (\bm{a},0), b), \quad (\bm{a},0) \in \R^{n+1}, \ b\in \R.
\end{align*}
\end{prop}
\begin{proof}
������������� ������~\ref{lem:01face}.
���������� ����� $F$ ������������� $\BQP(n+1)$, ������������ ��������������� $x_{n+1, n+1} = 0$. 
����� ��� ���� $\bm{x} \in F$ ����������� $x_{i, n+1} = 0$ ��� $i\in[n+1]$.
�������� �������, ��� ��������� ������ ����� $F$ ������������� � ��������� ������ ������������� $\BQP(n)$ (�, ��������, $\BQP(n)$ � $F$) �������� ������������ (������������� ���������) $x_{ij} \mapsto y_{ij}$, $1 \le i \le j \le n$. 

��� �������������� ����������� $\ATSP(n) \lea \ATSP(n+1)$ 
���������� ���������� ������� ����������� ������������ ����� ���������� ������������� �������� ������� ������� $D=(V,A)$ �� $n$ �������� � ������������� 
������������� �������� ������� $D'=(V',A')$ �� $n+1$ ��������, � ������� ������� ��� ����, �������� � ������� $v'_1$, � ����, ��������� �� $v'_{n+1}$, �� ����������� ���� $(v'_{n+1}, v'_1)$.
�������, ��� ������������������ ������� ���������� ������������ �������� ������ $D'=(V',A')$ �������� ��������� ����� 
\[F = \Set*{\bm{x} \in \ATSP(n+1) \given x_{(v'_{n+1}, v'_1)} = 1}
\]
������������� $\ATSP(n+1)$.
��������, ����� $F$ � ������������ $\ATSP(n)$ ������� ������������� �������� ������������
\[
	y_{(v_i, v_j)} = 
	\begin{cases}
	x_{(v'_i, v'_j)}, & 1 \le i \le n, \ 2 \le j \le n, \ i \ne j,\\
	x_{(v'_i, v'_{n+1})}, & 2 \le i \le n, \ j=1,\\
	\end{cases}
\]
��� $\bm{x} \in F$, $\bm{y} \in \ATSP(n)$.

��� �������������� ����������� $\Knap(n, \bm{a}, b) \lea \Knap(n+1, (\bm{a},0), b)$ ���������� ��������, ��� 
\[
\Knap(n+1, (\bm{a},0), b) = \Set*{(\bm{x},x_{n+1}) \in \{0,1\}^{n+1} \given \bm{x} \in \Knap(n, \bm{a}, b)}.
\]
�������������, ������������ $\Knap(n, \bm{a}, b)$ ������� ������������ ����� ������������� $\Knap(n+1, (\bm{a},0), b)$, ������������ ��������������� $x_{n+1} = 0$ (��� $x_{n+1} = 1$).
\end{proof}


��������� �������������� �� ������ ��������.

\BQP � \Cut.

\Stable, \Pack, \Part.

\BQP � \Stable.

%\subsection{�������� ����������}

Billera Sarangarajan.

�������� ����������.


\section{������ �������� ����������}

�����������.
����� ������������������ �������� �������� $X_n \in \Z^d$, 
��� $d = d(n)$ �������������, 
������, ��� ��� ����� $n \in \N$ � $x \in \Z^d$
������ �������� �������������� $x \in X_n$ ����������� ������ NP.
����� ��������� (������������������) �������������� $P(n) = \conv(X_n)$
���������� \emph{�������������}.

����������.
�~\cite{Naddef:1981, MatsuiTamura:1995} �������������� ���������� �������������, � ������� ��� ������ ���� ��������� ������ �������� ������������ �� ������� �������� ����� ��������� �������, ������������ ������ ���� ������ ����� �������������.

�����.
����� ��������� �������������� $P(n) \in \Z^d$, $d = d(n)$,  ������������.
����� ���������� ������������� ��������� �������������� $Q(n) \in \{0,1\}^k$, $k = k(n)$, ���
$Q(n)$ �������� ����������� $P(n)$.

�������.
����� ������������ $P \subseteq \R^d$ �������� ������� ������������� $Q \subseteq \R^n$ ��� ������������� ������������� $\pi \from \R^n \to \R^d$, $n > d$.
�����, ����� ����, $\pi(\ext Q) = \ext P$.
����� ���� ������������� $P$ �������� ��������� ����� ������������� $Q$.

��������������.
��� ������ ������� $\bm{v} \in \ext P$ ��������� ���������
\[
W(\bm{v}) = \Set*{\bm{x} \in \ext Q \given \pi(\bm{x}) = \bm{v}}.
\]
��������, $\conv (W(\bm{v}))$ �������� ������ ������������� $Q$.

��� �������� ��������, ��� $\R^d$ ������� � $\R^n$,
� ������������� $\pi$ ����������� $(x_1, \dots, x_n)$ � $(x_1, \dots, x_d, 0, \dots, 0)$.
��������������� ������ $\bm{c} \in \R^n$ ��������� ��������� �������.
���� $n = d+1$, �� $\bm{c} = \bm{e_n}$,
����� $\bm{c} = \lambda_{d+1} \bm{e_{d+1}} + \dots + \lambda_{n} \bm{e_n}$
� ������������ $\lambda_{d+1}$, \dots, $\lambda_{n}$ ��������� ���, 
��� ��� ������ $\bm{v} \in \ext P$ � ����� ���� ������ $\bm{w_1}, \bm{w_2} \in W(\bm{v})$ �� ����������� $\bm{w_1} \ne \bm{w_2}$ ������� $\bm{c}^T \bm{w_1} \ne \bm{c}^T \bm{w_2}$.
����, ��� � ���� ���������� ��������� ������ ������������� $Q$ ����� ������ $\bm{c}$ ����������. 
� ������� ����� ������� � ������ ��������� $W(\bm{v})$ ������� ���������� �������
\[
\bm{w_v} = \argmax_{\bm{w} \in W(\bm{v})} \Set{\bm{c}^T \bm{w}}.
\]

�������� ��������, ��� ���� ������� $\bm{v_1}$ � $\bm{v_2}$ ������������� $P$ ������, �� ��������������� ������� 
$\bm{w_1} = \bm{w_{v_1}}$ � $\bm{w_2} = \bm{w_{v_2}}$ ���� ������.

�����������, ��� ������� $\bm{v_1}$ � $\bm{v_2}$ ������������� $P$ ������.
����� ������������� $F_1 = \conv (W(\bm{v_1}))$ � $F_2 = \conv (W(\bm{v_2}))$, � ����� �� �������� �������� $F = \conv(F_1 \cup F_2)$ �������� ������� ������������� $Q$.
������ $\aff(F_1)$ � $\aff(F_2)$ ������������ ������� $\bm{b} = \bm{v_2} - \bm{v_1}$.
(��������������, �������� ������� $\bm{b}^T \bm{x}$ ��������� ������ �������� ��� $\bm{w_1}$ � $\bm{w_2}$.)
�������� ����� $\beta, \gamma \in \R$, $\gamma > 0$, ���,
����� �������� ������� $f(\bm{x}) = \beta \bm{b}^T \bm{x} + \gamma \bm{c}^T \bm{x}$ ��������� ���������� �������� ��� $\bm{w_1}$ � $\bm{w_2}$.
�������� ���������, ��� ����� 
\[
f(\bm{w_1}) = f(\bm{w_2}) > f(\bm{w}) \quad
\text{��� ���� }
\bm{w} \in W(\bm{v_1}) \cup W(\bm{v_2}) \setminus \{\bm{w_1}, \bm{w_2}\}.
\]
�������������, $\bm{w_1}$ � $\bm{w_2}$ ������.
��������.

���������.
����� ������������ $P \subseteq \R^d$ �������� ������� ������������� $Q \subseteq \R^n$ ��� �������� ����������� $\pi \from \R^n \to \R^d$.
�����, ����� ����, $\pi(\ext Q) = \ext P$.
����� $\omega(P) \le \omega(Q)$.

%% Глава 5
%%%%%%%%%%%%%%%%%%%%%%%%%%%%%%%%%%%%%%%%%%%%%%%%%%%%%%%%%%
%
%     ����������� �������� ����������
%
%%%%%%%%%%%%%%%%%%%%%%%%%%%%%%%%%%%%%%%%%%%%%%%%%%%%%%%%%%

\chapter{����������� �������� ����������}
\begin{flushright}
	� ���� ������� �������� �������?
\end{flushright}

\section{����������� � �������}

� ���� ����� �������� ����������� ��������� ��������� ������� ��������� �������������� (��. ����������� \ref{def:ineA}), ������������� � ���������� �����.
� ������, ����� <<������� ������������>> � �����������~\ref{def:ineA} ����� �������� �� <<�������� �������� �������>>. 

\begin{definition}\label{def:ineE}
	� ������, ����� ������������ $P$ �������� �������� ������� �������������~$Q$ ��� �� ��� �����, ����� ������������ ����������� $P \lee Q$.
\end{definition}

%�������, ������ �����, ��� �� $P \lea Q$ ������� $P \lee Q$.

������ $\lea$ �� $\lee$ ����������� ��������� ����������� ��������� ��������������. ����� ����, �� ������ ������� �������������� ����������� ���� $P \lee Q$ ������������� �����, ��� ����������� $P \lea Q$.

���������� ��������� ������� ��������.

������������ � ��������.

%�� �������� � $d$-������ 0/1-�������������� ������ ������� $d$-�������
%\emph{$[k]$-�������������}, ��������� ������ �������� ����������� $[k]^d$.



�����������.
����� ������� $d = d(n)$ �������������, � ������������������ �������� �������� $X_n \in \Z^d$ ������, ��� ��� ����� $n \in \N$ � $x \in \Z^d$
������ �������� �������������� $x \in X_n$ ����������� ������ NP.
����� ��������� (������������������) �������������� $P(n) = \conv(X_n)$
���������� \emph{�������������}.

����������.
�~\cite{Naddef:1981, MatsuiTamura:1995} �������������� ���������� �������������, � ������� ��� ������ ���� ��������� ������ �������� ������������ �� ������� �������� ����� ��������� �������, ������������ (���������) ������ ���� ������ ����� �������������.

�����.
����� ��������� �������������� $P(n) \in \Z^d$, $d = d(n)$,  ������������.
����� ���������� ������������� ��������� �������������� $Q(n) \in \{0,1\}^k$, $k = k(n)$, ���
$Q(n)$ �������� ����������� $P(n)$.

�������.
����� ������������ $P \subseteq \R^d$ �������� ������� ������������� $Q \subseteq \R^n$ ��� ������������� ������������� $\pi \from \R^n \to \R^d$, $n > d$.
�����, ����� ����, $\pi(\ext Q) = \ext P$.
����� ���� ������������� $P$ �������� ��������� ����� ������������� $Q$.

��������������.
��� ������ ������� $\bm{v} \in \ext P$ ��������� ���������
\[
W(\bm{v}) = \Set*{\bm{x} \in \ext Q \given \pi(\bm{x}) = \bm{v}}.
\]
��������, $\conv (W(\bm{v}))$ �������� ������ ������������� $Q$.

��� �������� ��������, ��� $\R^d$ ������� � $\R^n$,
� ������������� $\pi$ ����������� $(x_1, \dots, x_n)$ � $(x_1, \dots, x_d, 0, \dots, 0)$.
��������������� ������ $\bm{c} \in \R^n$ ��������� ��������� �������.
���� $n = d+1$, �� $\bm{c} = \bm{e_n}$,
����� $\bm{c} = \lambda_{d+1} \bm{e_{d+1}} + \dots + \lambda_{n} \bm{e_n}$
� ������������ $\lambda_{d+1}$, \dots, $\lambda_{n}$ ��������� ���, 
��� ��� ������ $\bm{v} \in \ext P$ � ����� ���� ������ $\bm{w_1}, \bm{w_2} \in W(\bm{v})$ �� ����������� $\bm{w_1} \ne \bm{w_2}$ ������� $\bm{c}^T \bm{w_1} \ne \bm{c}^T \bm{w_2}$.
����, ��� � ���� ���������� ��������� ������ ������������� $Q$ ����� ������ $\bm{c}$ ����������. 
� ������� ����� ������� � ������ ��������� $W(\bm{v})$ ������� ���������� �������
\[
\bm{w_v} = \argmax_{\bm{w} \in W(\bm{v})} \Set{\bm{c}^T \bm{w}}.
\]

�������� ��������, ��� ���� ������� $\bm{v_1}$ � $\bm{v_2}$ ������������� $P$ ������, �� ��������������� ������� 
$\bm{w_1} = \bm{w_{v_1}}$ � $\bm{w_2} = \bm{w_{v_2}}$ ���� ������.

�����������, ��� ������� $\bm{v_1}$ � $\bm{v_2}$ ������������� $P$ ������.
����� ������������� $F_1 = \conv (W(\bm{v_1}))$ � $F_2 = \conv (W(\bm{v_2}))$, � ����� �� �������� �������� $F = \conv(F_1 \cup F_2)$ �������� ������� ������������� $Q$.
������ $\aff(F_1)$ � $\aff(F_2)$ ������������ ������� $\bm{b} = \bm{v_2} - \bm{v_1}$.
(��������������, �������� ������� $\bm{b}^T \bm{x}$ ��������� ������ �������� ��� $\bm{w_1}$ � $\bm{w_2}$.)
�������� ����� $\beta, \gamma \in \R$, $\gamma > 0$, ���,
����� �������� ������� $f(\bm{x}) = \beta \bm{b}^T \bm{x} + \gamma \bm{c}^T \bm{x}$ ��������� ���������� �������� ��� $\bm{w_1}$ � $\bm{w_2}$.
�������� ���������, ��� ����� 
\[
f(\bm{w_1}) = f(\bm{w_2}) > f(\bm{w}) \quad
\text{��� ���� }
\bm{w} \in W(\bm{v_1}) \cup W(\bm{v_2}) \setminus \{\bm{w_1}, \bm{w_2}\}.
\]
�������������, $\bm{w_1}$ � $\bm{w_2}$ ������.
��������.

���������.
����� ������������ $P \subseteq \R^d$ �������� ������� ������������� $Q \subseteq \R^n$ ��� �������� ����������� $\pi \from \R^n \to \R^d$.
�����, ����� ����, $\pi(\ext Q) = \ext P$.
����� $\omega(P) \le \omega(Q)$.

%% Глава 6
%%%%%%%%%%%%%%%%%%%%%%%%%%%%%%%%%%%%%%%%%%%%%%%%%%%%%%%%%%
%
%     �������� ����������
%
%%%%%%%%%%%%%%%%%%%%%%%%%%%%%%%%%%%%%%%%%%%%%%%%%%%%%%%%%%
\chapter{����������� �������������}

\hfill
\begin{minipage}{0.5\textwidth}
����������� ������������� �������� ������������ ������ ������ ����� ���� �������� �������������� ��� �� ����������� � � ����� �� ������ ������.
\begin{flushright}
%Peter McMullen
����� ���������
\end{flushright}
\end{minipage}

\section{����������� � ��������}

� ���� ������� ���������� ����������� ����������� �������������� � ������������� �� ��������� ��������.
����� ��������� ���������� � ����������� �������������� ����� ����� �~\cite{Grunbaum:2003} �~\cite{ZieglerBook}.

����� $T = \{t_1, \dots, t_n\} \subset \R$, $t_1 < t_2 < \dots < t_n$.
����������� $d \in \N$, $2 \le d < n$, � ������ �����������
\[
\bm{x}(i) \coloneq (t^{\phantom{1}}_i, t^2_i, \dots, t^d_i) \in \R^d, \qquad i\in[n]. 
\]
������������ ����������� ������������ ������������� (��. �.~\pageref{page:cyclic}) � �������������� ����� �����������.
\emph{����������� ��������������} ���������� �������� �������� ���������
\[
\CP_d(T) \coloneq \{\bm{x}(1), \dots, \bm{x}(n)\}.
\]
������ $\CP_d(T)$ "--- ��������� ������ ����� �������������.
����� $i\in[n]$ ����� �������� \emph{�������} ������� $\bm{x}(i)$.

\begin{theorem}[������� �������� �����~\cite{Gale:1963}]
������������ $\CP_d(T)$ ������������, �� ���� ������ ��� ���������� �������� ����� $d$ ������. ������ ������������ ������ � �������� �� $S \subset [n]$, $|S|=d$, �������� ���������� ����� � ������ �����, ����� ��������� <<������� ��������>>:
\[
\text{�������� ��������� } [k_1, k_2]\cap S \text{ ����� ��� ���� } k_1, k_2 \in [n] \setminus S, \ k_1 < k_2.
\]
\end{theorem}

��� $n=7$ ��������� ��������, ��������������� ������� ��������, ����� �������
$\{1,5,6\}$, $\{1,7\}$, $\{2,3,4,5\}$ � $\{1,2,4,5,7\}$.
����� �������, ��������� $S \subset [n]$, ��������������� ������� ��������, ���������� ����������� �� ���� ���� $\{i,i+1\}$ �, ���� �����, �������� $1$ � $n$.

�������������� �������������� ������� �������� ����� ��������� �� ��������� �����:
\begin{enumerate}
\item ��� ������ $\bm{a} \in \R^d$ �������� �������� ������� $g(\bm{x}) = \bm{a}^T \bm{x}$ � ������ $\bm{x}(i)$, $i\in[n]$, ��������� $\CP_d(T)$ ��������� �� ���������� ���������� $f(t) = a_1 t + a_2 t^2 + \dots + a_d t^d$ � ������ $t_i$. �� ���� ��������� ���� $f(t) = b$ ������ �������������� $\bm{a}^T \bm{x} = b$.
\item ��������� ���� 
\begin{equation}
\label{eq:CyclicPol}
f(t) = (t - t_1)(t_n - t) \prod_{i \in I} \bigl((t - t_i)(t - t_{i+1})\bigr), \quad I\subset [n],
\end{equation}
��������� ������� �������� � ������ $t_1$, $t_n$, $t_i$, $t_{i+1}$, $i\in I$, � ������������� �������� � ��������� ������ �� $T$.
\end{enumerate}
����� �������, ��� ������� ���������, ���������������� ������� ��������, ����� ��������� ��������� ����~\eqref{eq:CyclicPol}, ������������ �������������� $f(t) = 0$, ������� ��� $\CP_d(T)$.


\section{���������� ����������� ������������}
\label{sec:EF4Cyclic}

�����, ��� � ������, $T = \{t_1, \dots, t_n\} \subset \R$, $t_1 < t_2 < \dots < t_n$, $d \in \N$, $2 \le d < n$.
� ���� ������� �� �������������� ������������� ������ ��� �������, ����� $t_{i+1} = t_i + 1$ ��� ���� $i\in[n-1]$, � ����� ������������ ����� ���������� ������������
\[
\CP_{d,n}(t_1) \coloneq \CP_d(T).
\]


\subsection{������ $d=2$}

�������, ��� ������������� $\CP_{d,n}(t)$ � $\CP_{d,n}(s)$ ������� ��������� ��� ����� $t,s \in \R$. 
� ������, �� ������ $(t + (s-t))^i = \sum_{j=0}^i \binom{i}{j} (s-t)^{i-j} t^j$ �������� ��������� �������� �����������, ������������ ���������� ����� $\CP_{d,n}(t)$ � $\CP_{d,n}(s)$:
\begin{equation}\label{eq:affine_isomorphism} 
y_i:=(s-t)^i+\sum_{j=1}^i \binom{i}{j} (s-t)^{i-j} x_j, 
\end{equation}
��� $(x_1,\ldots,x_d) \in \CP_{d,n}(t)$, $(y_1,\ldots,y_d) \in \CP_{d,n}(s)$.

�������� ������ ��� ���������� ���������, �� ����� ������������ ������� ������������ $\CP_{d,n}$ � ��� �������, ����� ����� ���������� �������� �� ������ �� ��������� �����������.


\begin{lemma}\label{lem:two_dim}
$\xc(\CP_{2,n}(t)) \le 2\lfloor \log_2(n-1)\rfloor+2$, ��� $n \ge 3$ � $t \in \R$.
\end{lemma}

\begin{proof}
��������~\eqref{eq:affine_isomorphism}, ������������~$\CP_{2,n}(t)$ ����� ������� ������������� � ������������
\[
\CP_{2,n}(-(n-1)/2).
\] 
	
� ���� �������, ��� ������� $k\in\N$ ������������~$\CP_{2,2k+1}(-k)$ ����� ���� ����������� ��� �������� �������� ���� �������������� $\CP_{2,k+1}(-k)$ �~$\CP_{2,k+1}(0)$:
%
\[
\CP_{2,2k+1}(-k) = \conv(\CP_{2,k+1}(-k) \cup \CP_{2,k+1}(0)).
\]
����������� ������������� ���������� � ��� ������������� $\CP_{2,2k}(-k + 1/2)$:
\[
\CP_{2,2k}(-k + 1/2) = \conv(\CP_{2,k}(-k + 1/2) \cup \CP_{2,k}(1/2)), \quad k\in\N.
\]
	
������������� ���, ��� ������������ $\CP_{2,k+1}(-k)$ �������� ������� ������������� $\CP_{2,k+1}(0)$ ��� ���������� ��������� ������������ �������������� $x_1 = 0$ (����������� ������������ ������ ����� ������ ����������). �� �� ����� � � ��������� $\CP_{2,k}(-k + 1/2)$ � $\CP_{2,k}(1/2)$.

���� ���� ��������� �� ������ �������� ���������� ��������� (��. Theorem~2 �~\cite{KaibelPashkovich:2013}) �~\eqref{eq:affine_isomorphism} ������� ����� � ���, ��� ������ ����������� ������������ ������� $h$ ��� ������������� $\CP_{2, \lceil n/2 \rceil}$ ���������� ����������� ������������ ������� $(h+2)$ ��� $\CP_{2,n}$. � ���������, ����������� ������������ ��� $\CP_{2,2k+1}(-k)$ ����� ���� �����:
\[
\CP_{2,2k+1}(-k)=\Set*{(x_1,x_2)\in\R^2 \given \text{�������� } z_1, \text{ ��� }(z_1,x_2)\in \CP_{2,k+1}(0) \text{ � } -z_1 \le x_1\le z_1}\,.
\]
����� �������, $\xc(\CP_{2k+1}) \le \xc(\CP_{k+1}) + 2$ � $\xc(\CP_{2k}) \le \xc(\CP_{k}) + 2$.
	
����������� $\xc(\CP_{2,n}) \le 2 \lfloor \log_2(n-1) \rfloor + 2$ ����� ����������� ��� $n = 3,4,5$. 
�������������, ��� ���������� �������������� ���������� ��������������� �������������
\[
\xc(\CP_{2,2k}) \le \xc(\CP_{2,k}) + 2 \le  2 \lfloor \log_2(k-1) \rfloor + 4 =2 \lfloor \log_2 (2k-2) \rfloor + 2
\]
�
\[\xc(\CP_{2,2k-1}) \le \xc(\CP_{2,k}) + 2 \le 2 \lfloor \log_2(k-1) \rfloor + 4 = 2 \lfloor \log_2 \big((2k-1) - 1\big) \rfloor + 2\,.
\]
����� �������,
\[
\xc(\CP_{2,n}) \le  2\lfloor \log_2(n-1)\rfloor+2
\]
��� ���� $n\ge 3$.
\end{proof}

\begin{remark}
����� ���������� ������ ��������� ����� �������������� �����~\ref{lem:two_dim}
��������� �������� ������� ������ ��������� ���������� �� 
\[
\xc(\CP_{2,n}) \le 2\lfloor \log_2(n-1)\rfloor + 1 + \delta_n,
\]
��� $\delta_n = 0$, ���� $2^k < n \le 3 \cdot 2^{k-1}$ ��� ���������� $k\in \N$, � $\delta_n = 1$ � ��������� �������.
\end{remark}


\subsection{������ $d \ge 3$}

� ���������, ��� $d \ge 3$ ����������� ����� ��������������� $\CP_{d,k+1}(-k)$ �~$\CP_{d,k+1}(0)$ �� ��� ������: ������������ $\CP_{d,k+1}(-k)$ �������� �������~$\CP_{d,k+1}(0)$ ��� ����� ����� ��������� � �������� ��������. ��� �� ������������� ��������� ������������ �������������� �, �������������, �� �� ����� ��������������� ���������� ���������� ���������. 

������, �� ����� ��������������� �������� ���������� (�������~\ref{thm:Yannakakis} �� �.~\pageref{thm:Yannakakis})
� ������������� ��������������� ������������� ������� ������� (��. �����������~\ref{def:slack} �~\ref{def:nonneg}) �������������~$\CP_{2,n}$, ������������� ������~\ref{lem:two_dim}. 

������ ������� ������� $M_{d,n}$ ������������� $\CP_{d,n}(t)$ ����� ������������� ���������� �� $[n]$, � ������� "--- ����������� $S \subset [n]$, $|S|=d$, ���������������� ������� �������� �����.
�������� ����������� ���� ������������ (��. �������~\eqref{eq:CyclicPol} � ����������� � ���), ��� ������� ����� ���� ������������ ���
%
\begin{equation}
\label{eq:slack}
M_{d,n}(i,S) := \prod_{j \in S} |t_j - t_i|\,,
\end{equation}
%
��� $i\in[n]$ � $S\subseteq [n]$, $|S|=d$, ������������� ������� �������� �����.
� ���������, ������� ������� �� ������� �� ���������� ��������� $t$ � ����������� ������������� $\CP_{d,n}(t)$, ��� ��� �� �� ������ �������� $t_j - t_i$. 

���������� ��������� ��������� �������� ���������������� ����� �������.

\begin{property}\label{prop:nonneg}
��������� �������� ��� �������� (���������) ������� �� ����������� �� ��������������� ����:
\begin{enumerate}
	\item ������������ ����� (��������).
	\item ������������ ������ (�������).
	\item �������� ������ (�������).
	\item ��������� ������ (�������) �� ��������������� �����.
	\item ���������� � ������� ����� ������ (�������), ������(���) ���������� ���������� ����� (��������).
\end{enumerate}
\end{property}

\begin{property}\label{prop:nonnegsum}
����� $M \in \R_+^{m\times n}$ � $S\subseteq[n]$. ����� $M_1$ "--- ���������� ������� $M$, ������������ �� �������� � �������� �� $S$, � $M_2$ "--- ����������, ������������ �� �������� � �������� �� $[n]\setminus S$.
�����
\[
\rank_+(M) \le \rank_+(M_1) + \rank_+(M_2).
\]
\end{property}

��� ���� ������ $A$ � $B$ ����������� �������, ��������� ������������ ������������ $A \circ B$ � ������� ��������� $(A \circ B)(i,j):=A(i,j) B(i,j)$. ����� ������������� ���� ������� ��� ����������� ��������� ����������� ����. 

\begin{lemma}
\label{lem:Kronecker}
��� ����� ���� ������ $A$ � $B$ � ���������� ������ ����� � ��������,
\[
\rank_+(A \circ B) \le \rank_+(A) \rank_+(B).
\]
\end{lemma}
%
\begin{proof}
����� $A,B \in \R_+^{n\times m}$ �
\begin{align*}
A &= TU, && \text{��� }T\in \R_+^{n\times r}, \quad U\in \R_+^{r\times m},\\
B &= HW, && \text{��� }H\in \R_+^{n\times s}, \quad W\in \R_+^{s\times m}.
\end{align*}
����� $M_i$ ����� ���������� $i$-� ������ ������� $M$, � ����� $M^j$ "--- $j$-�� �������.
������ $C_i$, $i\in[n]$, ������� $C \in \R_+^{n \times rs}$ ��������� ��������� �������������
\[
C_i = T_i\otimes H_i,
\]
� ������� $D^j$, $j\in[m]$, ������� $D \in \R_+^{rs \times m}$ "--- �������������
\[
D^j = U^j\otimes W^j.
\]
�����
\[
A\circ B = CD.
\]
\end{proof}

�������� ��������������� � �������������� ��������� ����������� ����� �������.
������ ����� ���������� ������ �����������, �������~$d = 2q$, $q\in \N$.

\begin{lemma}
\label{lem:even_case}
$\xc(\CP_{2q, n}) \le \big(\xc(\CP_{2, n})\big)^q$ ��� $q,n\in\N$, $2q < n$.
\end{lemma}

\begin{proof}
�� �������� $q$ ������ $C_1$, \dots, $C_q$ �����, ��� ������������ ������������ $C_1 \circ \cdots \circ C_q$ ����� ������� ������� $M_{2q, n}$. 
� ���� ����� �������, ��� ������ ��������� $S\subseteq[n]$, $|S|=2q$, ��������������� ������� �������� �����, ����� ���� ������� �� $q$ ��� $S_1$, \dots, $S_q$, ��� ������ ���� ���� ����� $\{1,n\}$, ���� ������� �� ���� ���������������� �����. 
������ ��������� $S_r$, $1\le r\le q$, ����� ������������� ������� �������� ����� � ������� �� ���� ���������. 
�������������, ��� ������� $S_r$ �������� ��������������� ������� � $M_{2, n}$ �����, ���
\[
	M_{2,n}(i,S_r)=\prod_{j\in S_r} |j-i|\,.
\]
������ ��������� ������ ������ $C_1$, \dots, $C_q$ � �������, ��������������� ���������� $S$, ��� $C_r(i,S):=M_{2,n}(i,S_r)$. 
�������, ���
\[
(C_1 \circ \cdots \circ C_q)(i,S) = \prod_{r = 1}^q \prod_{j \in S_r} |j-i| = \prod_{j \in S} |j-i| = M_{2q,n}(i,S)\,.
\]
	
�������� ���������, ��� ������� $C_1$, \dots, $C_q$ �������� �� $M_{2,n}$ �� ���� ������������, �������� � ������������ ��������. 
�������������, �������� c�������~\ref{prop:nonneg}, ��������������� ���� ������ �� ���� ������ ��������� ������ ��������������� ������ ������� $M_{2,n}$. 
����� �������, �������� �����~\ref{lem:Kronecker}, ������� ������� $M_{2q, n}$ ��������� ��������������� ������������ ������� $\big(\xc(P_{2,n})\big)^q$.
\end{proof}

%\subsection[������ d=2q+1]{������ $\di=2q+1$}


\begin{lemma}\label{lem:odd_case} 
$\xc(P_{2q+1,n}) \le 2 \xc(P_{2q,n-1})$ ��� $q,n\in\N$, $2q+1 < n$.
\end{lemma}
%
\begin{proof}
�������, ��� $\xc(P_{2q+1, n}(1)) \le \xc(P_{2q, n-1} (2)) + \xc(P_{2q, n-1}(1)) = 2 \xc(P_{2q, n-1})$.
	
��� ������� ��������� $S\subseteq[n]$, $|S|=2q+1$, ���������������� ������� �������� �����, ����������� ����� ���� �� ���� �������:
%
\begin{enumerate}
	\item \label{case:last_element} $n\in S$ � ��������� $S\setminus\{n\}$ ���������� ���������� ������������� $P_{2q, n-1}(1)$.
	\item \label{case:first_element} $1\in S$ � ��������� $S\setminus\{1\}$ ���������� ���������� ������������� $P_{2q, n-1}(2)$.
\end{enumerate}
�������� ������� ������� $M_{2q+1, n}$ �� ��� ����������. ���������� $M_1$ ����� �������� �� ��������, ��������������� �������~\ref{case:last_element}, � ���������� $M_2$ "--- �� ��������, ��������������� �������~\ref{case:first_element}.
�������� ��������~\ref{prop:nonnegsum}, 
\[
\rank_+(M_{2q+1, n}) \le \rank_+(M_1) + \rank_+(M_2).
\]

������ ����� �������, ��� ��������� ������ ������� $M_1$ � ������ ������ ������� $M_2$ ������� �� �����.
��� $i \in [n-1]$, $i$-� ������ ������� $M_1$ ����� $i$-�� ������ ������� $M_{2q, n-1}$, ���������� �� ������������� ����� $n-i$.
����� �������, $\rank_+(M_1) = \rank_+(M_{2q, n-1})$.
����������, ��� $i \in [2,n]$, $i$-� ������ ������� $M_2$ ����� $(i-1)$-�� ������ ������� $M_{2q, n-1}$, ���������� �� ������������� ����� $i-1$.
�������������, $\rank_+(M_2) = \rank_+(M_{2q, n-1})$.
\end{proof}

�� ���� \ref{lem:two_dim}, \ref{lem:even_case} � \ref{lem:odd_case} �������

\begin{theorem}
\label{thm:main}
$\xc(P_{d,n}) \le 2\bigl(2\lfloor \log_2(n-1)\rfloor+2\bigr)^{\lfloor d/2 \rfloor}$ ��� $2 \le d < n$.
\end{theorem}



%%%%%%%%%%%%%%%%%%%%%%%%%%%%%%%%%%%%%%%%%%%%%%%%%%%%%%%
%
%  �������� ������������ ������� ������� ��� d=2
%
%%%%%%%%%%%%%%%%%%%%%%%%%%%%%%%%%%%%%%%%%%%%%%%%%%%%%%%


\subsection{�������� ������������ ������� ������� ��� $d=2$}

� ���� ������� ��� ������� ������� $M_{2,n}$ ������� ����� ��������������� ������������ ������� $2\lceil\log(n-1)\rceil + 1$. 
��� ������������ ������ �� ��, ��� ������� �~\cite{Fiorini:2012polygons} ��� ������� ������� ����������� ��������������.

�������� ����������� ������� ������� ������������ ������������� $\CP_{2,n}$,
\[
M_{2,n}(i,j) = 
\begin{cases}
(i-j)(i-j-1), & \text{��� } j < n,\\
(i-1)(n-i), & \text{��� } j = n.
\end{cases}
\]
��� $j \in [n-1]$, $j$-�� ������� ���� ������� ������������� ��������� $\{j,j+1\}$, � $n$-�� ������� "--- ��������� $\{1,n\}$.

� ���������,
\begin{equation*}
\begin{split}
M_{2, 8} & =
\begin{pmatrix}
0	&  2	&  6	& 12	& 20	& 30	& 42	&  0 \\
0	&  0	&  2	&  6	& 12	& 20	& 30	&  6 \\
2	&  0	&  0	&  2	&  6	& 12	& 20	& 10 \\
6	&  2	&  0	&  0	&  2	&  6	& 12	& 12 \\
12	&  6	&  2	&  0	&  0	&  2	&  6	& 12 \\
20	& 12	&  6	&  2	&  0	&  0	&  2	& 10 \\
30	& 20	& 12	&  6	&  2	&  0	&  0	&  6 \\
42	& 30	& 20	& 12	&  6	&  2	&  0	&  0
\end{pmatrix}               
\end{split}
\end{equation*}

��� ��������� ���������� ������������ ������� �� ���� ��������������� ����������� ����������.
������ �����, � ������� ���������~\ref{alg:leftfactor},
���������������� ������ ��������� ���� ��������� $\text{A}$, ����� $\text{q} = \lceil\log(n-1)\rceil$, � ����������� ����� ��������� ������������ $\text{T} \in \R_+^{n \times (2q+1)}$.
�����, ��������� ������ A, ��������~\ref{alg:rightfactor} ��������� ������ ��������� ������������ $\text{U} \in \R_+^{(2q+1)\times n}$.

��������� ���������� ����� ��������� ��� $n=8$ �������� ���:
\begin{equation*}
	M_{2, 8} = 
	\begin{pmatrix}
	0	& 7	& 0	& 4	& 0	& 2	& 0  \\
	0	& 5	& 0	& 2	& 0	& 0	& 6  \\
	0	& 3	& 0	& 0	& 2	& 0	& 10 \\
	0	& 1	& 2	& 0	& 0	& 0	& 12 \\
	1	& 0	& 2	& 0	& 0	& 0	& 12 \\
	3	& 0	& 0	& 0	& 2	& 0	& 10 \\
	5	& 0	& 0	& 2	& 0	& 0	& 6  \\
	7	& 0	& 0	& 4	& 0	& 2	& 0
	\end{pmatrix}               
	\cdot
	\begin{pmatrix}
	6	& 4	& 2	& 0	& 0	& 0	& 0	& 0 \\
	0	& 0	& 0	& 0	& 2	& 4	& 6	& 0 \\
	3	& 1	& 0	& 0	& 0	& 1	& 3	& 0 \\
	0	& 0	& 1	& 3	& 1	& 0	& 0	& 0 \\
	1	& 0	& 0	& 1	& 0	& 0	& 1	& 0 \\
	0	& 1	& 1	& 0	& 1	& 1	& 0	& 0 \\
	0	& 0	& 0	& 0	& 0	& 0 & 0	& 1
	\end{pmatrix}    \,.
\end{equation*}         

\SetAlgorithmName{��������}{������ ����������}{} % �������� �� �������
\SetAlgoCaptionSeparator{.} % �������� �������� ��������� �� ��� ������ ������
%\SetAlgoLined % ������������ �����, ����������� ������ � ����� �����
\DontPrintSemicolon % �� �������� ����� � �������
\SetKwProg{Proc}{���������}{}{�����} % ������� ��� ���������
\SetKwProg{Fn}{�������}{}{�����} % ������� ��� �������
% �������� ������� � �������� ������
\SetKwInOut{Input}{����}
\SetKwInOut{Output}{�����}
\SetKwRepeat{DoWhile}{�����}{����} % ������� ���� do-while
\SetKwFor{For}{���}{\string:}{�����~�����}
\SetKwIF{If}{ElseIf}{Else}{����}{��}{�����~����}{�����}{�����~����}
\SetKw{KwTo}{��}
%\SetKwBlock{Loop}{loop}{endloop} % ����������� ����

\begin{algorithm}
	\caption{����� ��������� ������������ ������� $M_{2,n}$} % ���������
	\label{alg:leftfactor}
	% ����������� �������� (�������-��������) ������ ���������
	\SetKwArray{A}{A} % ������
	\SetKwArray{T}{T} % ������
	\SetKwData{n}{n}
	\SetKwData{q}{q}
	% �������� ����� �������� � �������
	\SetKwFunction{LeftFactor}{LeftFactor}
	% �������� �����-������
	\Input{ ����� ������ \n}
	\Output{ ����� ��������� ������������ \T, ������ ��������� ���� ��������� \A � ��� ����� \q}
%	\BlankLine
%	\Fn{\LeftFactor{\n}}{
		\tcp{��������� ������ \A}
		$k \coloneq \n-1$\;
		$\q \coloneq 0$\;
		\DoWhile{$k > 1$}{
			$\A{\q} \coloneq k$\;
			$k \coloneq \lfloor(k+1)/2\rfloor$\;
			$\q \coloneq \q + 1$\;
		}
		\tcp{��������� ����� ��������� \T}
		\For{$i \coloneq 0$ \KwTo $\n - 1$}{
			$x \coloneq i$\;
			\For{$j \coloneq 0$ \KwTo $\q - 1$}{
				$r \coloneq 2x - \A{j}$\;
				\eIf {$r > 0$} {
					$x \coloneq x - r$\;
					$\T{i, 2j} \coloneq r$\;
					$\T{i, 2j + 1} \coloneq 0$\;
				}{
					$\T{i, 2j} \coloneq 0$\;
					$\T{i, 2j + 1} \coloneq -r$\;
				}
			}
			$\T{i, 2\q} \coloneq i \cdot (\n - 1 - i)$ \tcp*[f]{��������� 	��������� �������}
		}
%	}
\end{algorithm}


\begin{algorithm}
	\caption{������ ��������� ������������ ������� $M_{2,n}$} % ���������
	\label{alg:rightfactor}
	% ����������� �������� (�������-��������) ������ ���������
	\SetKwArray{A}{A} % ������
	\SetKwArray{U}{U} % ������
	\SetKwData{n}{n}
	\SetKwData{q}{q}
	% �������� ����� �������� � �������
	\SetKwFunction{RightFactor}{RightFactor}
	% �������� �����-������
	\Input{ ����� ������ \n, ������ ���� ��������� \A � ��� ����� \q}
	\Output{ ������ ��������� ������������ \U}
	\BlankLine
%	\Fn{\RightFactor{\n, \A, \q}}{
		\For{$i \coloneq 0$ \KwTo $\n - 2$}{
			$x \coloneq i+1$\;
			\For{$j \coloneq 0$ \KwTo $\q - 1$}{
				$r \coloneq 2x - \A{j} - 1$\;
				\eIf {$r > 0$} {
					$x \coloneq x - r$\;
					$\U{2j, i} \coloneq 0$\;
					$\U{2j+1, i} \coloneq r$\;
				}{
					$\U{2j, i} \coloneq -r$\;
					$\U{2j+1, i} \coloneq 0$\;
				}
			}
			$\U{2\q,i} \coloneq 0$ \tcp*[f]{��������� ��������� ������}
		}
		\tcp{��������� ��������� �������}
		\For{$j \coloneq 0$ \KwTo $2\q - 1$}{
			$\U{j, \n-1} \coloneq 0$\;
		}    
		$\U{2\q, \n-1} \coloneq 1$\;
%	}
\end{algorithm}

\FloatBarrier

%%%%%%%%%%%%%%%%%%%%%%%%%%%%%%%%%%%%%%%%%%%%%%%%%%%%%%%
%
% ����-���� ������������ �������������
%
%%%%%%%%%%%%%%%%%%%%%%%%%%%%%%%%%%%%%%%%%%%%%%%%%%%%%%%

\section{������� ����-����� ������������ �������������}

����-���� $d$-������� ������������� ������������ ��������� ������� (��. �.~\pageref{ridge-graph}). 
��� ������� ������������� ����������� �������������, � ��� ������� ��������� ������ ����-�����, ���� ��������������� ���������� ($(d-1)$-�����) ����� ����� ���� ($(d-2)$-�����).
����� �������, ����-���� ������������� ��������, ��-��������, ������ ������������� � ������� �������������.

��� ��� ������������� ��������� ������������ ������������� ������� ������ �� ��� ����������� $d$ � ����� ������ $n$, � ���� ������� �� ����� ���������� ��� $\CP(d,n)$.
�� ������� ������������ ������������� �������, ��� ������������ � ���� ������������ $\CP^*(d,n)$ �������� ������� � �������� ������������ ������ ������ ����� ���� $d$-������ ��������������, ������� $n$ �����������.
��� �������������� ������� ������� �� ��, ���, ����� ����� �������, 
$\CP^*(d,n)$ �������� �������� ������������� � ���������� ��������� �����. 
� ���������, ��� ��� $d=4$ � $n=9$ �� ������� ����� $4$,
� �� ����� ��� � ������~\cite{Klee:1967} ���������� 
������ $4$-������������� � ��� �� ������ ����������� � ��������� ����� ������ $5$ 
(������� ���� ��������~\cite{Altshuler:1980}, ��� ��� ������������, 
� ��������� �� ������������� ���������������, ��� ������ 
$d=4$ � $n=9$ ������ ������������� � ��������� $5$). 
���������
\[
\dc=\diam \CP^*(d,n).
\]
� 1964 ���� �.~��� ������� \cite{Klee:1964}, ��� �������� ����� ����������� ��� $\CP^*(d,n)$: 
\begin{equation}
\label{Hirsh4Cycl}
\dc\le n-d,
\end{equation}
� ��� $d < n\le 2d$ � \eqref{Hirsh4Cycl} ����������� ���������.
��� �� ���� ��������� ������������� � ���, ��� ��� $n>2d$ ��������� 
��������� $\dc=\left\lfloor n/2\right\rfloor$ 
(���� ����� �~\cite{Klee:1967} ���� ��������, ��� ��� �������).
�� ���� ��� ���������� � ���� ������������� (�� ������ ���������� ��������� ���������� ����� ������� "--- �������~\ref{thm:RidgeGraphDiamCyclic}) ������ \cite{Klee:1964} ��������� ����� ��������, ��� �� ���������� ����� 30~��� �������� ������� ���������� �������� $\dc$ ��� $n=2d$ � ������� 
�������������� ������� \cite{Ferrez:1998}, � � ������ \cite{Lagarias:1998} 
��� ���������� �����~\eqref{Hirsh4Cycl} ���������� ������ �� ������ 
\cite{Klee:1966}, �� ���������� ����� ����������.

%��� ����, ����� ��������� ����� � ���� �������, 
%���� ���������� ������ �������� ��� $\dc$.


\begin{theorem}\label{thm:RidgeGraphDiamCyclic}
������� $\dc$ ����-����� ������������ ������������� $\CP(d,n)$ ����������� �� �������:
\[
	\dc=
	\begin{cases}
	n-d 				& \text{ ��� } d < n \le 2d,\\
	n-d  - 
	\left\lceil 
	\frac{n-2d}{ \left\lfloor \frac d 2 \right\rfloor+1}
	\right\rceil 
	& \text{ ��� }  n > 2d.
	\end{cases}
\]
\end{theorem}

\begin{proof}
�������������� ������� ��� $d < n \le 2d$ �������� �.~��� \cite{Klee:1964}.
������� ����� ����� ������������, ���
\[
n>2d.
\]

��������� ��������� ������ ������������� $\CP(d,n)$ ����� $X=\{x^1, x^2, \ldots, x^n\}$, �����������, ��� ������� ������������� � ������� ����������� ��������� $t$: $t_1 < t_2 < \ldots < t_n$.


�������� �������������� ������� ��� ������ ������ �����������:
\[
d=2k.
\]
�������, ��� 
\begin{equation}
\label{eq:RidgeIneq}
\dc \le n-d - \frac{n-2d}{k+1}
\end{equation}
� �������� ������ ���� ����������� ������������ �������������,
���������� ����� �������� ����� ����� ����� ��~\eqref{eq:RidgeIneq}.

�� �������� ����� �������, ��� ��������� ������ ������ ����������
������ ����� ������������ ������� ������� �� $k$ ���������������� ��� ����
\[
\{x^i, x^{i+1}\},
\]
��� $i\in[n]$ � �������� $i+1$ ����������� �� ������ $n$ (���� $i=n$, �� $i+1=1$).
�, ��������, ����� ������������ $Y\subset X$ ����
\[
Y=\{x^{i_1}, x^{i_1+1}, x^{i_2}, x^{i_2+1}, \ldots, x^{i_k}, x^{i_k+1}\}
\]
���� ��������� ������ ��������� ����������.

��� ����������� ����������� ��������� ������ ������� $x^i$ 
�������� � ������������ ����� $v_i$ �� ���������� ���������� �������:
\[
v_i=\bigl(\cos(2\pi i / n), \sin(2\pi i / n)\bigr).
\]
��� ����� $v_i$ � $v_{i+1}$ ����� �������� \emph{�����}, � ���������� $p_i$.
� ��������� ���� ����� ��� ��� $i\in[n]$ ��������� ${\mathcal P}$.

% ������������� ������� �����, ������ ����� � ����� ����� �� ���
\newcommand\BeginPic[2]{
	\def\rdot{1.5pt}
	\def\Rdot{3pt}
	\def\linew{1.5pt}
	\def\lineB{5pt}
	\def\lineW{3.3pt}
	\def\Radius{#1}
	\def\Larc{360/#2} % ����� ���. ���� � ��������
	%\def\Larc{20} % ����� ���. ���� � ��������
}	
\newcommand*{\Angle}[1]{(#1)*\Larc}
% ���������� ����� �� � ������
\newcommand*{\cvertex}[1]{({\Radius*cos(\Angle{#1})}, {\Radius*sin(\Angle{#1})})} % ���������� ��� ������ �����
\newcommand*{\nvertex}[1]{({\Radius*0.8*cos(\Angle{#1})}, {\Radius*0.8*sin(\Angle{#1})})}
% ���� ���������
\newcommand{\ArcO}[2]{
	\draw[dashed] \cvertex{#1} arc (\Angle{#1}:\Angle{#2}:\Radius);
}
% ���� ��������� � ���������������� ������� �� ���
\newcommand{\Arc}[2]{
	\ArcO{#1}{#2}
	\foreach \x in {#1, ..., #2} \draw[fill = black] \cvertex{\x} circle (\rdot*0.6) \nvertex{\x} node {\scriptsize\x};
}
% ������ �����
\newcommand\Fishka[1]{
	\draw[line width = \linew] 
	\cvertex{#1} arc (\Angle{#1}:\Angle{#1+1}:\Radius);
	\draw[fill = black] \cvertex{#1} circle (\rdot) 
	\cvertex{#1+1} circle (\rdot);
}
% ������ ������
\newcommand\Cell[1]{
	\draw[line width = \lineB] 
	\cvertex{#1} arc (\Angle{#1}:\Angle{#1+1}:\Radius);
	\draw[fill=white, line width = 0.5*\linew] 
	\cvertex{#1} circle (\Rdot) 
	\cvertex{#1+1} circle (\Rdot);
	\draw[line width = \lineW, draw=white] 
	\cvertex{#1} arc (\Angle{#1}:\Angle{#1+1}:\Radius);
}


% ������� 1
\begin{figure}[tbh]
	\begin{center}
		%  \PictA
		\begin{tikzpicture}
		\BeginPic{2}{18}
		\begin{scope}[xshift=-3cm]
		\ArcO{0}{1}
		\Arc{1}{18}
		\Fishka{18}
		\Fishka{2}
		\Fishka{5}
		\Fishka{8}
		\Fishka{16}
		\end{scope}
		
		\begin{scope}[xshift=3cm]
		\ArcO{0}{1}
		\Arc{1}{18}
		\Fishka{1}
		\Fishka{3}
		\Fishka{5}
		\Fishka{8}
		\Fishka{17}
		\end{scope}
		\end{tikzpicture}
	\end{center}
	\caption{��� ������� ���������� ������������� $\CP(10,18)$.}
	\label{fig:PictNeighborlyFaset}
\end{figure}


�����������, ��� ������� ��������� ���������� ������ ������� �� $k$ ���.
����� ������� � ������� ���������� �������� ����� ����� ��� ���������� �����������
���� � ����� ��� �� ���� ����� ����� ���������� (��. ���.~\ref{fig:PictNeighborlyFaset}). 
(�� ���������������� ������������ ������������� �������, 
��� ��� ��� ���������� ������, ���� ��� ����� ����� $d-1$ ����� ������.)
����� �������, ����� ������� � ����� ������� ����������, ���������� ������� ����
�� $k$ ��� � ����������� �������� (�� ������� ��� ������ ������� �������),
��� ����, ���� �� ���� �������� ���� ��������� ������ ����,
�� ��� ��������� � ��� �� �����������, ��� � ������, �~�.~�.
����� ����� ������������ ��������� ��������.

%%%%%%%%%%%%%%%%%%%%%%%%%%%%%%%%%%%%%%%%%%%%%%%%%%%%%%%%%%%%%%%%%%%%
%
%                      ������� ��������
%
%%%%%%%%%%%%%%%%%%%%%%%%%%%%%%%%%%%%%%%%%%%%%%%%%%%%%%%%%%%%%%%%%%%

\medskip
\textbf{������� ��������.} 
\emph{
	��� �������� ���� ����� ������������ ���������� � ���������� ����������� �������,
	������� � ��� �� ����, �� �� ����� ������������ � ������ �����.
}
\medskip

����, ������ ���������� ���������� ����� ��������� � ����-����� 
������������ ������������� �������� � ���������.
�� ���������� � $n$ ������� ������� $k$ ��� �����, ������������ ������� 
������ ����������, ����� �������� �� \emph{�������}, 
� �������� $k$ ���, ��������������� �������� ������ ����������, 
����� �������� �� \emph{��������} (��. ���.~\ref{fig:PictFishki}). 
��������� ����� ���������� ����� <<����������>> �������� �����, 
����������� ��� ����, ����� ��� ������ ���� ������ �������.
��������� ��� ����� $l(F_1,F_2)$, ����� $F_1$ --- ��� ��������� �����, 
� $F_2$ --- ��������� �����, $F_1,F_2\subset {\mathcal P}$. 


% ������� 2
\begin{figure}[tbh]
	\begin{center}
		\begin{tikzpicture}[>={Stealth[scale width=0.8]} % ���������� ��� �������
		]
		\BeginPic{2}{18}
		
		\begin{scope}
		\ArcO{0}{1}
		\Arc{1}{18}
		\Cell{3}
		\Cell{5}
		\Cell{8}
		\Cell{10}
		\Cell{12}
		
		\Fishka{2}
		\Fishka{5}
		\Fishka{11}
		\Fishka{16}
		\Fishka{18}
		
		\draw[<-] \cvertex{8.5}  +(-0.15,0) -- +(-1,0) node[left] {������};
		\draw[<-] \cvertex{16.5} + (0.1,0) -- +(1,0) node[right] {�����};
		\draw[<-] \cvertex{1.5} + (0.1,0) -- +(1,0) node[right] {����������};
		\end{scope}
		\end{tikzpicture}
	\end{center}
	\caption{������ ����������� $F_1=\{1,2,3,5,6,11,12,16,17,18\}$ (�����)
		� $F_2=\{3,4,5,6,8,9,10,11,12,13\}$ (������) ������������� $\CP(10,18)$.}
	\label{fig:PictFishki}
\end{figure}



������ ���� �� ��������� ${\mathcal P}\setminus(F_1\cup F_2)$ �������� � ������������ 
���� ���������� ����� ������� $v_i$ � $v_{i+1}$, �� ���������� ����� ���� �����.
������� ����� ���� \emph{�����������} (��. ���.~\ref{fig:PictFishki}) � ����� ���������� $a_i$. 
� ��������� ���� ����������� ��������� ����� $A$. 
��� ��� �� ���������� ����������� $k$ ����� � $k$ �����, �� $|A| \ge n - 2k > 0$.
������������ $W\subseteq A$ ������� \emph{��������� �����}, ���� �������� $i,j\in[n]$
�����, ��� $W=\{a_i, a_{i+1}, \ldots, a_j\}$,
������ $a_{i-1}$ � $a_{j+1}$ �� ������ � $A$.
�����, ��� � �����, ��� ���������� �������� �������� ����������� �� ������~$n$.
����, ��� $A$ ����������� ������������ ������� � �����
\begin{equation}
\label{GapSet}
A=W_1 \cup W_2 \cup \ldots \cup W_s, \mbox{ ��� } 0 < s \le 2k = d,
\end{equation}
$W_i$ --- ��������� ����. 
��������� ����� $l(W)$ ��������� ���� $W$ ��� ����� ����� $v_i$ ($1\le i\le n$), 
������������ ������ ���:
$$
l(W)=|W|-1.
$$
����� ��������� ����� ��������� ��� ����� ����� ������, 
�� ������������� �� ����� �� ���� ��������� �����������, �
$$
\sum^{s}_{i=1} l(W_i)\ge n-2d.
$$

%%%%%%%%%%%%%%%%%%%%%%%%%%%%%%%%%%%%%%%%%%%%%%%%%%%%%%%%%%%%%%%%%%%%
%
%                      �����
%
%%%%%%%%%%%%%%%%%%%%%%%%%%%%%%%%%%%%%%%%%%%%%%%%%%%%%%%%%%%%%%%%%%%

\begin{lemma}
	\label{lemma:Ridge1}
	�����������, ��� �� ���������� �������� ��������� ���� $W$ 
	������ � ������������� ������ ��� �������,
	� ������ �� $F_1$ ��������� ������������� ����� ��� �����.
	����� 
	$$
	l(F_1,F_2)\le n - d - l(W).
	$$
\end{lemma}

\begin{proof}
��� ������ ���������� �������� �� ����� ������������� ����� ���� �����, 
� ����� ���� ����� ���������� �������. 
�������, ��� ��� ����������� ������������ �����
������� �������������� ����� �� ����� ���� ������ �����.
����� ��������� � ����, ����������� ���������.
����� �������� ��� ��������:
\begin{enumerate}
	\item �����, ������������ �����, ������ �� �����.
	�� ����� ��������������� ����������� ��������.
	\item �����, �������� �����, ���������� � ��� �� �����������,
	��� � �����, ������������ ��� �����. �� ����� ��� ��� ��������
	(������������ � ������� ������ �����) ����� ���� �� ������� �� ���� ���.
\end{enumerate}
����� ������� ����������� ����� �������� �� ����� ��������� ����� ���� ����� �� ������� �������.
\end{proof}

��������������� �� �����~\ref{lemma:Ridge1} ������� �����������
$$
l(F_1,F_2)\le n-d - \max_{1\le i\le s} l(W_i).
$$
� ���������, � ��� ������� ����� �������� ������� ��� $d<n\le 2d$. 
���������� �������� ������ ���� �����������, ������� $2d-n$ ����� ������. 
��������, ��� ����� ���������� �������� ��� ��� ����� ����� $n-d$.

��� ������ <<����� ���������� ������ $W_i$ � $W_j$>> ����� �������� ��� ������� ����������,
������� ����� ������� ��� �������� ������ ������� ������� �� $W_i$ � $W_j$.
������ ��������������� ������� $f(i,j)$ ������ ����� �����, 
������������� ����� $W_i$ � $W_j$, ����� ����� ����� �� ���� �� �������.
��� ������� �������� ���������� ����������.

\emph{�������� 1.} 
$f(i,j)=-f(j,i)$.

\emph{�������� 2.} 
$f(i,j)=f(i,m)+f(m,j)$.

\emph{�������� 3 (�������������). } 
���� $f(i,j)>1$, �� ����� $W_i$ � $W_j$ �������� $W_m$, ��� $f(i,m)=1$.

���� ��� ��������� $i$ � $j$ ��������� $f(i,j)=0$, 
�� $W_i$ � $W_j$ ������� \emph{��������}.

\emph{�������� 4 (��������������). }
���� $W_i$ � $W_j$ ������, � $W_j$ � $W_m$ ������,
�� $W_i$ � $W_m$ ���� ������. 

����� $W_i$ � $W_j$ ������. 
�����, ������� �� �� ���������� � �������� �����~\ref{lemma:Ridge1}
��� ������ �� ���� �������������� ���, ��������
$$
l(F_1,F_2)\le n-d - l(W_i)-l(W_j).
$$
���������� �� ����� ��������� � � ����������� ������� �������� ���������� ������.


% ������� 3
\begin{figure}[tbh]
	\begin{center}
		\begin{tikzpicture}
		\BeginPic{2}{18}
		
		\def\PicThree{
			\Cell{3}
			\Cell{5}
			\Cell{8}
			\Cell{10}
			\Cell{12}
			\Fishka{2}
			\Fishka{5}
			\Fishka{11}
			\Fishka{16}
			\Fishka{18}
		}
		
		\begin{scope}[xshift=-4.6cm]
		\draw node {$S_1$};
		\Arc{1}{1}
		\Arc{2}{4}
		\Arc{5}{6}
		\Arc{8}{18}
		\PicThree	
		\end{scope}
		
		\begin{scope}[xshift=0cm]
		\draw node {$S_2$};
		\Arc{1}{9}
		\Arc{10}{17}
		\Arc{18}{18}
		\PicThree	
		\end{scope}
		
		\begin{scope}[xshift=4.6cm]
		\draw node {$S_3$};
		\Arc{1}{13}
		\Arc{16}{18}
		\PicThree	
		\end{scope}
		\end{tikzpicture}
	\end{center}
	\caption{��� ������� ��� ���.~\ref{fig:PictFishki}: $l(S_1)=1$, $l(S_2)=0$, $l(S_3)=2$.}
	\label{fig:PictCut}
\end{figure}


��������� ����, �������� ��������� ���� ��������� ��� �� ������������ $S_i$ 
%=W_{i_1} \cup W_{i_2} \cup \ldots \cup W_{i_t}$, 
���, ����� ��� ��������� ���� ������������� ������ ������������ ���� ������� ������
�, � �� �� �����, ����� ��� ��������� ���� �� ������ ����������� ���� �� ��������.
������������ $S_i$ ����� �������� \emph{���������} (��. ���.~\ref{fig:PictCut}). 
%����� $t$ ���� ��������, �����������, ������ ����� $s$ ���� ��������� ���.
��� ������ $l(S)$ ������� $S$ ����� �������� ��������� ����� �������� � ���� ��������� ���. 
�����, ��������,
$$
l(F_1,F_2)\le n-d - \max_{1\le i\le t} l(S_i),
$$
��� $t$ --- ����� ���� ��������.
�, ��� ��� 
$$
\sum^{t}_{j=1} l(S_j)=\sum^{s}_{i=1} l(W_i)\ge n-2d,
$$
��
$$
l(F_1,F_2)\le n-d - \frac{n-2d}{t}. 
$$

�������� ��������, ��� 
\begin{equation}
\label{eq:CutRidge}
t\le k+1,
\end{equation}
��� $k=\frac d2$.
�� ������� \eqref{GapSet} ��������, ��� ����� ���� ��������� ��� $s\le d$.
�������, ����� ���������� �������������� ����������� \eqref{eq:CutRidge}, ���������� ��������, 
��� ����� ���� �������� �������� �� ����� ����, ���������� ����� �� ����� ��������� ����.

����� ������ $S'$ �������� ���� ������������ ��������� ���� $W_{i'}$. 
��� ��������, ��� ��� ������ $j\ne i'$, ���������
\begin{equation}
\label{NotEq}
f(i',j)\ne 0.
\end{equation}
� �����, � ���� �������� 3 (�������������), ��� ������ $m\ne i'$ 
$$
f(i',j) f(i',m) > 0.
$$
�� ���� ������� $g(j)=f(i',j)$ ��� ���� $j$ ��������� �������� ������ �����.

���������� ��� ���� ������ $S''$, ���������� ���� ���� ��������� ���� $W_{i''}$, $i''\ne i'$.
�������, ��� ��� ����� $j\ne i'$ � $m\ne i''$ ���������
$$
f(i',j) f(i'',m) < 0.
$$
�������� �� ����������, �����������, ��� ��� ��������� $j$ � $m$
$$
f(i',j) f(i'',m) > 0.
$$
(��������� ����� ����������� � ���� ������� \eqref{NotEq} 
� ������������ ����������� ���~$i''$). 
�� ����� ��� ����������� ����������� � ��� $j=i''$ � $m=i'$,
��� ���������� � ���� �������� 1. 
����, ����� ���� �������� �������� �� ����� ����, ���������� ����� �� ����� ��������� ����,
� �����������~\eqref{eq:CutRidge}, � ������ � ��� � �����������~\eqref{eq:RidgeIneq}, ��������. 

�� �������~\ref{fig:PictDiameter} ���������� ������� �����������, 
���������� ����� �������� ����� ����������� ������, 
�� �������������� ������ �����~\eqref{eq:RidgeIneq}. 
� ������~\cite{Klee:1967} ���� ����� ������� 
���� ������ ������ ���� ��� $d=6$ � $n=23$. 
�������� ���������� ����� �������� ��� ������������ $d$ � $n$ ���� �� �������.
����� �������, ��� ������ ������ ����������� ������� ��������.


% ������� 4
\begin{figure}[tbh]
	\begin{center}
		\begin{tikzpicture}
		\BeginPic{2}{18}
		\begin{scope}[xshift=-4.6cm]
		\draw node {$\CP(4,18)$};
		\ArcO{0}{1}
		\Arc{1}{18}
		\Cell{1}
		\Cell{3}
		\Fishka{8}
		\Fishka{13}
		\end{scope}
		
		\BeginPic{2}{19}
		\begin{scope}[xshift=0cm]
		\draw node {$\CP(6,19)$};
		\ArcO{0}{1}
		\Arc{1}{19}
		\Cell{1}
		\Cell{3}
		\Cell{5}
		\Fishka{8}
		\Fishka{12}
		\Fishka{16}
		\end{scope}
		
		\BeginPic{2}{21}
		\begin{scope}[xshift=4.6cm]
		\draw node {$\CP(6,21)$};
		\ArcO{0}{1}
		\Arc{1}{21}
		\Cell{1}
		\Cell{3}
		\Cell{5}
		\Fishka{9}
		\Fishka{13}
		\Fishka{17}
		\end{scope}
		\end{tikzpicture}
	\end{center}
	\caption{������� ������������ ��������������� �����������.}
	\label{fig:PictDiameter}
\end{figure}


%%%%%%%%%%%%%%%%%%%%%%%%%%%%%%%%%%%%%%%%%%%%%%%%%%%%%%%%%%%%%%%%%%%%
%
%                      �������� �����������
%
%%%%%%%%%%%%%%%%%%%%%%%%%%%%%%%%%%%%%%%%%%%%%%%%%%%%%%%%%%%%%%%%%%%


�������� ������ ������, ����� $d=2k+1$.

�� �������� ����� �������, ��� � ������ �������� ����������� 
������ ���������� ������ ��������� ���� �� ���� �� ���� �����: $v_1$ ��� $v_n$. 


����� $F_1$ � $F_2$ --- ��������� ����������. 
�������� ��� ��������.

1-� �������. 
��� $F_1$ � $F_2$ ���� �� ���� �� ����� $v_1$ � $v_n$ �����.
����� ��� ����� ����� ��������� �� ������������ �, ��� �����, 
������� � ������ ������ �����������. 
�������������,
$$
l(F_1,F_2)\le (n-1)-(d-1) - \frac{(n-1)-2(d-1)}{k+1}<n-d - \frac{n-2d}{k+1}.
$$

2-� �������. 
$F_1$ � $F_2$ �� ����� ����� ����� � ��������� $\{v_1, v_n\}$.
�� �������� ��������, �����������, ��� $F_1$ �������� $v_1$, �� �� �������� $v_n$,
� $F_2$ �������� $v_n$, �� �� �������� $v_1$. 
�� �������� ����� �������, ��� ���� �� ��� ��������� �������� ��������� $v_1$, 
�� ����������� ������ ������ $v_n$. 
�. �. ��������� ����, ��������� �� ���������� $a_n$, 
�� ����� ������� � �����-�� �� �� ���� ������. 
����� ����, ���� ��������� ������ ������� �������, 
�� ��������� ����, ������������� ������ $v_1$ ����� ������ ��������� ����, 
������������� ����� $v_n$. 
����� �������, ���������� ����� ��������, ��� � � ������ ������ �����������, 
����� ����� $k+1$. 

������ �������� ��������� ���� �� ����� ����������� 
�������� ���������� ������ ������ �����������.
\end{proof}

%%%%%%%%%%%%%%%%%%%%%%%%%%%%%%%%%%%%%%%%%%%%%%%%%%%%%%%
%
% End of section
%
%%%%%%%%%%%%%%%%%%%%%%%%%%%%%%%%%%%%%%%%%%%%%%%%%%%%%%%

%% Глава 7
%%%%%%%%%%%%%%%%%%%%%%%%%%%%%%%%%%%%%%%%%%%%%%%%%%%%%%%%%%
%
%     ��������� ������� ����
%
%%%%%%%%%%%%%%%%%%%%%%%%%%%%%%%%%%%%%%%%%%%%%%%%%%%%%%%%%%
\chapter{��������� ������� ����}
\label{sec:Direct}

\hfill
\begin{minipage}{0.5\textwidth}
��������� ����� ������������� ������ ������ ������ ������� �������������� ��������� � ������� ������ ����������, ���������� �� �������� ����������.
\begin{flushright}
�.�. ����������
\end{flushright}
\end{minipage}

%\section{������� � �����������}

1. �������� ��������� � ��� ��������.
������������: ������ � ��������������� ������������� �� ��������� �������� ������.
������ -- ���������� ����.

2. �������� ���������� ��� �������� ���������. �������: ��� -> ����������, ����������� -> ����������� � ������������ �����������������.

3. �������� ����������� ���������. ��������� ������� ����.

4. ����������� -- �������� ����--��������.


%%%%%%%%%%%%%%%%%%%%%%%%%%%%%%%%%%%%%%%%%%%%%%%%%%%%%%%
%
% End of section
%
%%%%%%%%%%%%%%%%%%%%%%%%%%%%%%%%%%%%%%%%%%%%%%%%%%%%%%%

%% Глава 8
%%%%%%%%%%%%%%%%%%%%%%%%%%%%%%%%%%%%%%%%%%%%%%%%%%%%%%%%%%
%
%     ������������
%
%%%%%%%%%%%%%%%%%%%%%%%%%%%%%%%%%%%%%%%%%%%%%%%%%%%%%%%%%%

\chapter{������������}
\label{chap:Counterexamples}

\hfill
\begin{minipage}{0.55\textwidth}
������ ��������� ������������� �� �������� ����, ��� ��������� �������� �������� ��� ��������� ����������, �� ��������� ������� ���������.
\begin{flushright}
�.\,�.~��������
\end{flushright}
\end{minipage}

%� ���������� �������, ��� �������� �� ��������� ���������� ������ ���������� ������� ���� ��������� ��������� ������ ��� ���� ����� ������������, �������������� ������ ��� ���������� ��������� ���������������� ����� ������������� �����������.
%� ���������, ����� ��������� ������ ���������� �.�. ���������� �� ���������� � ���� ���������� �������.s


{\color{red}
	��� ��� ���� ������� ����, ��� �������������� ������������� ���������� ����� 	�������� ����� ����� ��� ������� �� ����������� ����������� �������������.
	�.�. ����� ����������� ������, ��� ����������� ������������� ���������� ��� ��������������.
	����� ����, ���� ��������� �� ��������� ������ ����� ����� <<�����������>> �������� �����.
	��������, ������ ������ ����������� �������� � ������� �� $n$ ��������� �������� ����� ������ ����������.
	������������ ������ ������ ������������ ����� �������� \cite{BondBook:1995}.
	��� ���� ������. �������������, �������� ����� ����� $n$.
	������������ ������ ������ ���������� \emph{���������������}, � �������� ����� ��� ����� ����� 2 \cite{Gaiha:1977}.
	�.�. � ������ ������ �������� ����� ����� �� ������� � �������� ���������� ������.
	
	����������� ������� ������� � ���, ��� ��� ������������� ���������� ����� ����������� ������������� �� ������ ������� ����������� ������� ����� ������ ��������������� ���������, ��� �������� ����� �����.
	� ������ �� �������������� NP-������� ����� ������� �������� ��������� �����
	����������� ���, ��� ������������ $\BQP(n)$ ����������� ������ ���� �������������� � ���� ������������ ������, ������������� �������� �����������.
}


%%%%%%%%%%%%%%%%%%%%%%%%%%%%%%%%%%%%%%%%%%%%%%%%%%%%%%%
%
% End of section
%
%%%%%%%%%%%%%%%%%%%%%%%%%%%%%%%%%%%%%%%%%%%%%%%%%%%%%%%


% Заключение
%%%%%%%%%%%%%%%%%%%%%%%%%%%%%%%%%%%%%%%%%%%%%%%%%%%%%%%%%%
%
%     ������������
%
%%%%%%%%%%%%%%%%%%%%%%%%%%%%%%%%%%%%%%%%%%%%%%%%%%%%%%%%%%

\chapter{����������}

%\hfill
%\begin{minipage}{0.55\textwidth}
%������ ��������� ������������� �� �������� ����, ��� ��������� �������� �������� ��� ��������� ����������, �� ��������� ������� ���������.
%\begin{flushright}
%�.\,�.~��������
%\end{flushright}
%\end{minipage}

� ��������������� ������ ��������������� �������� ������ ��������� �������� ����� ������������� ����������� � ������� ��������� ������������"=�������������� ������������� ��������������� �������������� ����������� (�������������� � ��������� �����, �������� ��������� ������������ �������� ������). � �������� ����� ������������� ���������������: ����������� �������������, ����� ��� ������, ����� �����������, ������� ����� �������������, �������� ����� �����, ��������� ������ ������������� �����, ��������� ���������� ������������� � ����� �������������� �������� ������� ���������� ������"=�����������.

�\'������ ����� ����������� ������� � ����� �������� �������� ����������. ���� ��� ���������� ��������� ���������� ��������� ���� �������������� �������������� (�� ����������� �������� �����), ��������������� � ��������� �������� ������������� �����������. � ������ ������������ ��������� �������������� ��������� NP-������� �����: �����������, ���������� ����, 3-������������, ������, ������ � �����, ����������� ��������� ������ �����, �������� � �������� ���������, ������ ������������ ����������������, ��������� �����, ���������� �������, �������� ��������������, ������ �������� � �����, ��������� �������� ������ � ����������� (������������ ����������, ���������� � ������������, 3-���������) � ��������� ������.
�� ������ ���� ������ ��������� �� ������� ����� �� ��������������� ������ �����~\cite{Karp:1972}. �� ����� �������������, �������������� � �����~\ref{chap:AffExamples}, �������, ��� ��������� ������� ������������ �������������� ������� �������� � ���������� �������������� ������������� ���� �����.
����� �������, ��� ��� ��������� �������������� ��������� �� $\BQP$ ������� (�������������������) �������� ��������� ���� ������������"=�������������� ���������������. � ������ �������: ��������������������� ��������� ����������, ����� �������������� �������� ������� ���������� ������"=�����������, ��������� ����� �����. �������������, � �������~\ref{sec:BQP-power} ��������, ��� ��� ������ $k \in \N$ � $n \ge 2^{2\cdot \lceil k/3\rceil}$ ����� ������������ ������������ $\BQP(n)$ ����� $k$-����������� ����� �� ������������������� ������ $2^{{\Theta}\left( n^{1 / {\left\lceil k/3\right\rceil}}\right)}$ ������. ��������� ���������� �������� ����������, ������ �� ������������� �������� �������������� ���� �������� �������������, ������� $k$"~����������� ����� �� ������������������� (������������ ����������� �������������) ������ ������.

�������� ��������������� ��������� ��������������, ��� ������� ������ ������������� ����������� ������ �������� NP-������. ���� ���� � �������������� ��������� �����: �������� ���������, 3-������������, �����������, ���������� ����, ���������� �������, ������, ��������� ����� �� ��� ������ �����, ���������� � ������������, ��������� ��������������. ��������, ��� NP-������� �������� ����������� ������ ����������� ����� ��������������� �� ������������ ��������� �������������� $\NPadj$, ������� ���������� � 1995~�. �.~�����~\cite{Matsui:1995}. �~������ �������, ��������, ��� ��������� ������ ����������� �� �������������� ����� ��� �������������� ��������� NP-������� �����: ������ � �����, ������������ ������ � ����������� � � �������� ��������������, �������� � ��������� ���������, 3-���������, ��������� �����, ����������� ��������� ������ �����. �~�������~\ref{sec:DoubleCovering} ��������, ��� �� ���� �� �������������� ��������� $\NPadj$ (�� ����������� ����������) �� ����� ���� ������ �� ��� ������ ������������� �� ������������� �������� � ������� ��������� ���������.

������ �������� ���������� ����� ���� ��������� � ��� ��������� ������������"=�������������� ������������� �������� ����� ������������� �����������, ������� �������������� ����������� �� ��������� �������� ������ (������� ��������).
�������� ����� �������� ����� ������ ������ � ���������� ������, ������������� ���������� ��� ���������� � ������� �������� ������������� �����. � �������~\ref{sec:ShortPath2Assignment} ��������, ��� �������� ��������� ��������� �������� ������ ���� ������ ������� �������� � ��������� ��������� ��������� �������� ������ ������ � �����������.

� �����~\ref{chap:ExtAff} �������� ������� ����������� �������� ����������, ������������ �� �������� ���������� ����������� ���������� �������� ���������������. ���������� ������� ����������� �������� �������������� ������ ����������� �������� ���������� (� �������~\ref{sec:ExtAffExamples} ������������ ��������������� �������). ������ � ���, ����� ��� ����������, ������ ������, �� ��������� ���������� ��������� �������������� �������������� (��������, ����� �����������, �������� ����� ������ ��������������, NP-������� �������� ����������� ������). �~�������~\ref{sec:Cook4Polytopes} ��������, ��� ����� ��������� ��������������, ������ ������������ ������� ������� ����������� ������ NP, ���������� ������� �������� � ��������� ������� ������������ ��������������. ����� �������, ��� ������������� ��������� �������������� ����������� ������������ ���� ����� ������������ ����������� �������� ����������.

� �����~\ref{chap:Cyclic} ����������� �������� ����������� ��������������, �������� ������ ���� � ������������� ������ ��������������. � �������~\ref{sec:EF4Cyclic} ��� ������������� $\CP_{d,n}(t)$ ������� ����������� ������������ ������� $O(\log n)^{\lfloor d/2 \rfloor}$. � �������~\ref{sec:RidgeGraph} ������ ������, ���������������� �.~��� � 1964 ����~\cite{Klee:1964}, "--- ������� ������ �������� �������� ����"~����� ������������ ������������� $\CP(d,n)$ ��� $n>2d$.

� �����~\ref{chap:Direct} ��������������� ������ ���������� ������� ����, � ������ ������� ������������, ��� �������� ����� ����� (�������������, ��������� ��������� ��������� �������� ������) ������ �������� ������ ������� ��������� ��������������� ��������������� ������ � ��������� ������� ������ ����������. �~�������~\ref{sec:ShortPathClique} ��������, ��� �������� ����� ����� ������ � ���������� ������ � ������� �� $n$ �������� � ������������ ����������������� ���� �������� �����~$\lfloor n^2/4\rfloor$. 
� �������~\ref{sec:NondirectAlg} ���������� �������������� ����, ��� �������� ����--�������� ��� ������ � ����������� (� ����� �������� �������� ��� ������ � ��������������) �� �������� ���������� ������� ����.
����� ����, ����������� ���������� ������������� ������ ����������� ����������,
����������� �� �������� �� ������������, �� �������������� ��������� �� �� ������ ���������� ������� ����.

� �����~\ref{chap:Counterexamples}



\section{�������� �������}

1. �� ��� ��� �������� �������� ��������� ������. ���������� �� ������� �����, ��������� ������� ���� �� ����������� ������ ����� �������������� �������� ������� ���������� ������"=����������� ���������������� �������������? �� ���� ��������� �������� ��������� ������ ����������� ������ ���� ��������. � ���������, ����� ������������, ��� ��������� ������ ������ ���������� ����� ������ �������������� ��������, � ������ "--- ���������� ����������� ������������.

2. � ��������� ������� ��������� ����� ���� ��������� ��� ������� �����, ������������� ������� ������������ ������������, �� ���� �� ����� ������������� ���������, � ������ ����� ���������������� ���������. ���� ��������� ��������� �� ������ �� ���� ����, ��� ������������� ������� ������������� ������ ������ ����� ���������������� ���������, ��� ������������ ����������� ��������� �������������� � ��������� � ����������� ������ ������������� �����������. 

3. � ��������� ����� ���������� ������ ������ ��������� ���������� $\CP_{d,n}$, ������� ��������������� �� ������� ������ �� �������~\ref{thm:main}. ��������, ��~\cite{FioriniKPT:13} �������, ��� $O(d^2 \log n)$ �������� ������ ������ �������, ����������� �� ������������� �������� �������������.



% Список сокращений и условных обозначений
%\printnomenclature

% Словарь терминов
%\input{dict}

% Не добавлять длинное тире в качестве разделителя
%\newcommand\BibDash{}
% Выделять курсивом
\let\BibEmph=\emph
\bibliographystyle{gost2008ns} % 'n' for natbib compatibility
%\bibliographystyle{gost705s}

% Список литературы
\inputencoding{cp1251}
\bibliography{biblio}
% Когда список литературы будет сформирован окончательно, 
% нужно переименовать MaksimenkoThesis.bbl -> bibl-finished.tex и включить в текст последний файл непосредственно
%\input{bibl-finished}
%\bibliographystyle{plain}

% Список иллюстративного материала
%\listoffigures

% Приложения
%\appendix
%\input{a}

\end{document}
